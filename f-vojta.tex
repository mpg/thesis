% !TEX root = main.tex

\begin{landscape} \centering

\vspace*{\stretch{1}}

\begin{tikzpicture}[
  auto,
  hypo/.style={
    text height=0.7\baselineskip, text depth=0.3\baselineskip,
    draw,
  },
  result/.style={
    text height=0.7\baselineskip, text depth=0.3\baselineskip,
  },
  result2/.style={
    text height=1.7\baselineskip, text depth=0.3\baselineskip,
    align=center,
  },
  stop/.style={
    text height=0.7\baselineskip, text depth=0.3\baselineskip,
    draw, thick, rounded rectangle,
  },
  section/.style={
    draw, rounded corners,
    inner sep=0.5cm,
  },
  iter/.style={
    draw, rounded corners, thick, dashed,
    inner xsep=0.3cm, inner ysep=1cm,
  },
  dimdep/.style={
    draw, dotted, ->,
  },
  dep/.style={
    draw, ->,
  },
  dimloop/.style={
    to path={-- ++(0, #1) -| (\tikztotarget)}, rounded corners,
  },
  puiss/.style={
    draw, decorate, decoration={
      snake,
      amplitude=.4mm, segment length=2mm, 
      pre length=.5mm, post length=.5mm,
    },
  },
  ]

  % matrice pour les étapes de base
  \matrix [row sep=1cm, column sep=2.4cm] {
      \node[result]  (var-deg)   { \( \deg \var \) };
    & \node[result2] (aux-ex)    { Existence \\ et degrés };
    & \node[result2] (ind-ok)    { Indice \\ extrapolé };
    & \node[result2] (fo-ex)     { Existence \\ et degré };
    & \node[result]  (next-deg)  { \( \deg \var' \) };
    \\
      \node[result] (var-ht)     { \( h(\var) \) };
    & \node[result] (aux-ht)     { Hauteur };
    & \node[result] (ind-ht)     { À \( h(\ex[1]) \) fixé };
    & \node[result] (fo-ht)      { Hauteur };
    & \node[result] (next-ht)    { \( h(\var') \) };
    \\
  };

  % nœuds supplémentaires
  \node[hypo] (in-deg)     [left=of var-deg] { \( \deg \va \) };
  \node[hypo] (in-ht)      [left=of var-ht]  { \( h(\va) \) };

  \node[hypo] (eps-zgti)   [above=of aux-ex] { \( \epsz \gg \epsi \) };

  \node[hypo] (Vcos)       [above=of ind-ok] { \( \Vcos \ll 1 \) };
  \begin{scope}[node distance=2mm]
    \node[hypo] (eps)      [right=of Vcos]   { \( \eps > 0 \) };
    \node[hypo] (eps-igtz) [left=of Vcos]    { \( \epsi \gg \epsz \) };
  \end{scope}

  \node[hypo] (Vfar)       [above=of fo-ex]  { \( \Vfar \gg 1 \) };

  \begin{scope}[node distance=3cm]
    \node[result] (out-ht) [below=of var-ht] { \( h(\ex*) \) };
    \node[stop]   (end)    [below=of aux-ht] { Contradiction };
    \node[hypo]   (Vbig)   [below=of ind-ht] { \( \Vbig \gg 1 \) };
  \end{scope}

  % dépendences
  \draw[dimdep] (in-deg)     to                  (var-deg);
  \draw[dimdep] (in-ht)      to                  (var-ht);

  \draw[dep]    (eps-zgti)   to                  (aux-ex);
  \draw[dep]    (aux-ex)     to                  (aux-ht);
  \draw[dep]    (var-deg)    to                  (aux-ht);
  \draw[dep]    (var-ht)     to                  (aux-ht);

  \draw[dep]    (aux-ex)     to                  (ind-ok);
  \draw[dep]    (var-deg)    to [bend left=18]   (ind-ok);
  \draw[dep]    (eps-igtz)   to                  (ind-ok);
  \draw[dep]    (Vcos)       to                  (ind-ok);
  \draw[dep]    (eps)        to                  (ind-ok);
  \draw[dep]    (ind-ok)     to                  (ind-ht);
  \draw[dep]    (aux-ht)     to                  (ind-ht);
  \draw[dep]    (Vbig.west)  to [bend left=30]   (ind-ht.190);

  \draw[dep]    (ind-ok)     to                  (fo-ex);
  \draw[dep]    (Vfar)       to                  (fo-ex);
  \draw[dep]    (fo-ex)      to                  (fo-ht);
  \draw[dep]    (aux-ht)     to [bend left=12]   (fo-ht);

  \draw[dep]    (fo-ex)      to                  (next-deg);
  \draw[dep]    (fo-ex)      to                  (next-ht);
  \draw[dep]    (fo-ht)      to                  (next-ht);
  \draw[dep]    (var-ht)     to [bend right=5]   (next-ht.190);

  \draw[dimdep] (next-deg)   to [dimloop=+2.6cm] (var-deg);
  \draw[dimdep] (next-ht)    to [dimloop=-1.5cm] (var-ht.280);

  \draw[dimdep] (var-ht.260) to                  (out-ht);

  \draw[dep]    (out-ht)     to                  (end);
  \draw[dep]    (Vbig)       to                  (end);

  \draw[puiss]  (eps-zgti)   to node [swap] {
    \( \scriptstyle \puiss > \genre \) } (eps-igtz);

  % cadres et leurs étiquettes
  \node[section] (aux) [fit=(aux-ex)(aux-ht)] {};
  \node[section] (ind) [fit=(ind-ok)(ind-ht)] {};
  \node[section] (fo)  [fit=(fo-ex)(fo-ht)]   {};

  \node[below] (laux) at (aux.south) {Forme auxiliaire};
  \node[below] (lind) at (ind.south) {Extrapolation};
  \node[below] (lfo)  at (fo.south)  {Forme obstructrice};

  \node[iter] (rec) [fit=(var-ht)(var-deg)(Vcos)(next-deg)(next-ht)(laux)]
  {\hfill\null};
  \node[above] at (rec.north)  {
    Partie itérée.
    Entrée : \( \var = \va^\puiss \) ;
    sortie : \( \var* = \set{ \ex* } \) pour un certain \( \fct \).
  };

\end{tikzpicture}

\vspace*{\stretch{1}}

\captionof{figure}{%
  Dépendances entre les différentes quantités en jeu.
  Toutes les quantités dans le cadre en traitillé qui dépendent indirectement
  de \( \deg Z \) en dépendent aussi directement ; les flèches correspondantes
  sont omises pour alléger.
}
\label{fig:vojta}

\vspace*{\stretch{1}}

\end{landscape}

\endinput
