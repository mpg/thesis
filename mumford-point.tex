\documentclass{mpg-preth}

\title{Une inégalité de Mumford explicite\\
  pour les approximations d'un point}

\begin{document}

\maketitle

\section*{Introduction et notations}

On travaille dans une variété abélienne $\A$ de dimension $g$, définie sur
$\Qbar$, qu'on suppose munie d'un fibré $\mathcal{M}$ ample et symétrique, de
degré $d$. On choisit sur $\mathcal{M}^{\otimes 16}$ une structure thêta, et
on plonge $\A$ dans $\Proj^n$, où $n=16^g d - 1$, par un plongement de
\bsc{Mumford} modifié, tel que décrit au §3.1 de \cite{daphimhva2}, que nous
noterons $\Theta$. On identifiera volontiers $\A$ et son image par $\Theta$.
On reprend sans les rappeler les définitions de \lat{loc. cit.}, ainsi que les
notations, sauf sur les points suivants : on omettra systématiquement les
indices se rapportant à la puissance de $\mathcal{M}$ utilisée et on notera
plus volontiers en indice les coordonnées. On écrira ainsi $\coa_{(a, l)} =
\Delta_{(a, l)}(\coa)  = \Delta_{a, l}^{(2)}(\coa) =
\coa_{\mathcal{L}^{\otimes 4}}(a, l)$. Par ailleurs, bien que les coordonnées
soient naturellement indexées (\lat{ibid.} p. 651) par $\mathcal{Z}_2 \times
\widehat{K_2(4)}$, on les indexera souvent par $\{0, \dots, n\}$ pour alléger
l'écriture.

On fixe un corps de nombres $\cdn$ contenant un corps de définition de $\A$
plongée ; on suppose de plus qu'il contient un système de coordonnées $\coa =
(\coa_0,\dots,\coa_n)$ de l'origine $\OA$, ainsi que les racines quatrièmes de
l'unité. Si $\p{x}$ est un point de $\A(\cdn)$, on notera $x = (x_0, \dots,
x_n)$ un de ses représentants dans $\cdn^{n+1}$. De même, on notera $\cdn[X] =
\cdn[X_0, \dots X_n]$ l'anneau des coordonnées homogènes. Si $P$ est une forme
sur $\Proj^n$, on notera $P= \sum_i a_i X^i$ où $i \in \N^{n+1}$ est un
multiindice, et par convention $X^i = X_0^{i_0}\cdots X_n^{i_n}$. On notera
également $\frac{\partial^{|i|} P}{\partial X^i} = \frac{\partial^{|i|}
  P}{\partial X_0^{i_0}\cdots \partial X_n^{i_n}}$, où $|i| = i_0 + \cdots +
i_n$ est la longueur du multiindice $i$.

Définissons maintenant les normes locales utilisées pour les vecteurs. Si $v$
est une place finie, on pose $\nv{x} = \sup_i(\av{x_i})$ ; pour une place $v$
infinie on utilise sauf précision contraire la norme euclidienne $\nv{x} = (
\sum_i \av{x_i}^2 )^{1/2}$. Notant $e_v = [\cdn_v : \Q_v]/[\cdn : \Q]$ le degré
local relatif de $\cdn$, on définit ainsi la hauteur projective d'un point
$\p{x} \in \Proj^n(\cdn)$ par $H(\p{x}) = \prod_v \nv{x}^{e_v}$ ; on utilisera
couramment la hauteur additive $h = \log H$.

On définit ensuite des normes locales pour les formes homogènes. Si $P =
\sum_{\lgr{i} =d} a_i X^i$ est une forme homogène de degré $d$, on pose aux
places finies $\nv{P} = \sup_i \av{a_i}$, et aux place infinies $\nv{P} =
\left( \sum_i \av{a_i}^2/\binom{d}{i} \right)^{1/2}$, où on a noté
$\binom{d}{i}$ le coefficient multinomial $d!/(i_0!\cdots i_n!)$. On a ainsi
l'inégalité $\av{P(x)} \le \nv{P} \nv{x}^d$. Concernant les dérivées, on a
$\nv{\frac{1}{i!}\frac{\partial^{\lgr{i}}P}{\partial X^i}} \le \binom{d}{i}^\dv
\nv{P}$, où l'on a noté $i! = i_0! \cdots i_n!$, et $\dv$ valant $0$ si $v$
est finie, et $1$ sinon. Nous utiliserons couramment cette notation $\dv$ pour
regrouper en une seule formule les cas $v$ finie et $v$ infinie.

Passons maintenant au cas multihomogène ; seul le cas d'une puissance de
projectifs nous sera utile. Pour chaque point $\p{x} = (\p{x}^{(1)}, \dots,
\p{x}^{(m)})$ de $(\Proj^n)^m$, on note $x = (x^{(1)}, \dots x^{(m)}) =
(x_0^{(1)}, \dots, x_n^{(m)})$ un système de coordonnées multiprojectives. On
note de plus $\nv{x} = (\nv{x^{(1)}},\dots, \nv{x^{(m)}})$ son vecteur norme.
Si $P$ est une forme multihomogène de multidegré $d= (d^{(1)}, \dots,
d^{(m)})$, on écrira encore $P = \sum_{\vlg{i}=d} a_i X^i$, avec cette fois $i
\in \N^{m(n+1)}$ de vecteur longueur $\vlg{i} = (|i^{(1)}|, \dots, |i^{(m)}|$,
et la notation évidente $X^i = \prod_{k} (X^{(k)})^{i^{(k)}} = \prod_{k, l}
(X_l^{(k)})^{i_l^{(k)}}$. On définit une norme similaire au cas homogène par
\[
  \nv{P} = \left( \sum \frac{ \av{a_i}^2}{\binom{d^{(1)}}{i^{(1)}} \cdots
  \binom{d^{(m)}}{i^{(m)}} } \right)^{1/2}
\]
aux places archimédiennes, $\nv{P} = \max(\av{a_i})$ aux places finies. Si
$F = (F_0, \dots, F_p)$ est une famille de formes multihomogènes de même
multidegré, on note de plus $\nv{F} = (\nv{F_0}, \dots, \nv{F_p})$ son vecteur
norme, et enfin $\nnv{F} = \nv{(\nv{F})}$. On a bien ainsi $\nv{F(x)} \le
\nnv{F}\nv{x}^d$.

Pour chaque place $v$ de $\cdn$, on définit une distance locale entre les
points de $\Proj^n(\cdn)$ par :
\[
  \Dv(\p{x}, \p{y}) = \frac{\nv{x\wedge y }}{\nv{x}\nv{y}}\pmm{.}
\]
On s'intéresse alors aux approximations exceptionnelles de l'origine,
c'est-à-dire aux points $\p{x} \in \A(\cdn)$ satisfaisant $\Dv(\p{x}, \OA) \le
H(\p{x})^{-\eps}$, où $0<\eps<1$ est fixé.  Plus généralement, on considère un
ensemble fini $S$ de places de $\cdn$, et une famille de réels positifs
$(\lambda_v)_{v\in S}$ telle que $\sum_{v\in S} e_v \lambda_v = 1$, où on
rappelle que $e_v = [\cdn_v : \Q_v]/[\cdn : \Q]$ est le degré local relatif.
On considère alors les points satisfaisant le système d'inégalités
\[
  \Dv(\p{x}, \OA) \le t_v H(\p{x})^{-\lambda_v \eps} \quad
  \forall v \in S \pmm{,} \tag{HP} \label{hyp}
\]
où $0<t_v\le 1$ est une constante présente pour des raisons techniques, que
l'on précisera.

D'après un résultat plus général de \bsc{Faltings}, ces points sont en nombre
fini avec $t_v = 1$ (et $S= \{v\}$). On cherche à préciser ce résultat en
majorant le nombre de tels points, en fonction du $\eps$ choisi, de la
dimension $g$ de $\A$, de son rang $r$ sur $\cdn$, et de paramètres
diophantiens liés à $\A$ plongée, que sont la hauteur projective de $\OA$, (le
degré $d_\OA$ de son corps de définition ?), et pour chaque $v$ de $S$ les
constantes locales $c_v(\A) = \nv{\OA}\nv{\BA}$, où $\BA$, qui sera défini
précisément plus bas, est obtenu en inversant certaines coordonnées de
l'origine. De tels résultats quantitatifs sont usuellement obtenus en
combinant une inégalité à la \bsc{Vojta} et une inégalité à la \bsc{Mumford},
toutes deux explicites.

On prouve ici une inégalité explicite à la \bsc{Mumford} sur ces
approximations. Celle-ci s'énonce dans l'espace de \bsc{Mordell-Weil} de
$\A(\cdn)$, qui est par définition l'espace $\A(\cdn)\otimes_\Z \R$ muni de la
forme quadratique définie positive donnée par la hauteur normalisée de
\bsc{Néron-Tate}, notée $\hn$. C'est un espace vectoriel euclidien dont la
dimension est le rang de \bsc{Mordell-Weil} $r$ de $\A$ sur $\cdn$. Si $\p{x}$
est un point de $\A(\cdn)$, nous noterons encore $\p{x}$ son image dans
l'espace de \bsc{Mordell-Weil}. On rapelle la notation $\dv = 1$ si $v$ est
archimédienne, $0$ sinon. On est alors en mesure d'énoncer :

\begin{thm} \label{mumpt}
  Soient $\p{x}$ et $\p{y}$ deux points de $\A(\cdn)$ satisfaisant (\ref{hyp}),
  avec
  \[
  r_v = \av{2}^g(2^g20n(n+1)^2)^{-\dv} c_v(\A)^{-1} \pmm.
  \]
  Posons $B =
  4^gh(\OA)+ 3g\log(2)$, et choisissons $\phi > 0$ et $\rho > 0$ tels que
  $(\rho^2/4) + 2\phi + \rho\phi< \eps$. Supposons :
  \begin{enumerate}
  \item $\cos(\p{x},\p{y}) \ge 1-\phi$, \label{cone}
  \item $\hn(\p{x}) \le \hn(\p{y}) \le (1+\rho)\hn(\p{x})$, \label{sect}
  \item $\hn(\p{x}) > (1+\eps)B/(\eps - (\rho^2/4) - 2\phi - \rho\phi)$,
    \label{tronk}
  \end{enumerate}
  alors $\p{x} = \p{y}$.
\end{thm}

Les conditions sur $\p{x}$ et $\p{y}$ ont une interprétation géométrique très
simple dans l'espace de \bsc{Mordell-Weil}. En effet (\ref{cone}) demande que
les points se trouvent dans un même demi-cône d'angle petit, (\ref{sect})
précise qu'ils doivent se trouver dans un même secteur de petite largeur
exponentielle, et (\ref{tronk}) que ce secteur est suffisamment loin de
l'origine. Le théorème affirme alors qu'il y a au plus une bonne approximation
de l'origine dans un secteur de cône d'angle $\arccos(1-\phi)$, de largeur
exponentielle $1+\rho$, et extérieur à une sphère de rayon $(1+\eps)B/(\eps -
\rho^2/4 - 2\phi - \rho\phi )$ centrée en l'origine.

\section{Au travail !}

On suppose désormais que $r_v$ est fixé comme dans les hypothèses du théorème.
Le point central de la démonstration est le fait suivant : si $\p{x}$ et
$\p{y}$ sont suffisament proches $v$-adiquement de $\OA$, alors leur
différence aussi. Le lemme principal consistera à rendre cette affirmation
quantitative. On aura besoin du lemme technique suivant.

\begin{lem} \label{pv}
  Soit $v$ une place de $k$, on peut choisir $i_v \in \{ 0, \dots, n\}$ de
  sorte que $\frac{\av{\coa_{i_v}}}{\nv{\coa}} \ge (\frac{1}{\sqrt{n+1}})^\dv
  $.  Soient aussi $\eps_0 < (\frac{1}{\sqrt{n+1}})^\dv$ et $\p{x} \in
  \A(\cdn)$ satisfaisant $\Dv(\p{x}, \OA) < \eps_0$, on a alors
  $\frac{\av{x_{i_v}}}{\nv{x}} \ge (\frac{\sqrt{n+1}}{n+2}-\eps_0)^\dv$. \\ En
  particulier, si $\eps_0 < (\frac{1}{4\sqrt{n+1}})^\dv$ on a
  $\frac{\av{x_{i_v}}}{\nv{x}} \ge (\frac{1}{2\sqrt{n+1}})^\dv$ et $x_{i_v}$
  est non nul.
\end{lem}

\begin{proof}
  Pour le premier point, il suffit de choisir $i_v \in \{ 0, \dots, n\}$
  maximisant $\av{\coa_i}$. Afin d'alléger les notations, on supposera par la
  suite $i_v = 0$. Pour chaque $z = (z_0, \dots, z_n)$ on notera $\hat{z} =
  (z_1, \dots, z_n)$ le vecteur obtenu en omettant la première coordonnée. On
  a alors, vu l'hypothèse sur $\p{x}$ et la définition de la distance :
  \begin{equation}
  \eps_0 \nv{x}\nv{\coa}  \ge \nv{x \wedge \coa} \ge \nvp{\coa_0 \hat{x} -
    x_0 \hat{\coa}} \pmm{,} \label{pva}
  \end{equation}
  où la deuxième inégalité vient en remarquant que toutes les coordonnées de
  $\coa_0 \hat{x} - x_0 \hat{\coa}$ apparaissent également comme coordonnées
  de $x \wedge \coa$.

  On traite d'abord le cas ultramétrique, par l'absurde. En effet, si on avait
  $\av{x_0}< \nv{\hat{x}} = \nv{x}$, il viendrait $\nvp{\coa_0 \hat{x}} >
  \nvp{x_0 \hat{\coa}}$ et la propriété ultramétrique appliquée au dernier
  membre de (\ref{pva}) donnerait $\eps_0 \nv{x}\nv{\coa}  \ge
  \av{\coa_0}\nvp{\hat{x}}$, puis $\eps_0 \ge 1$ contrairement aux hypothèses.

  Pour le cas archimédien, l'inégalité triangulaire dans (\ref{pva}) donne
  $\eps_0 \nv{x}\nv{\coa}  \ge \av{\coa_0}\nvp{\hat{x}} -
  \av{x_0}\nvp{\hat{\coa}} $. En regroupant les termes et en divisant par
  $\nv{x}\nv{\coa}$, puis en remarquant que $\nvp{\hat{\coa}}/\nvp{\coa} \le
  1$ il vient successivement :
  \begin{align}
  \frac{\nvp{\hat{\coa}}}{\nvp{\coa}} \cdot \frac{\av{x_0}}{\nv{x}}
  & \ge \frac{\nv{\hat{x}}}{\nv{x}}\cdot \frac{\av{\coa_0}}{\nv{\coa}} -
  \eps_0 \pmm{,} \notag \\
  \frac{\av{x_0}}{\nv{x}} & \ge
  \frac{\nv{\hat{x}}}{\nv{x}}\cdot\frac{1}{\sqrt{n+1}} - \eps_0 \pmm{.}
  \label{contrainte}
  \end{align}
  Notons $t = \av{x_0}/\nv{x} \in [0; 1]$ ; comme $\nv{x}^2 = \av{x_0}^2 +
  \nv{\hat{x}}^2$, on réécrit (\ref{contrainte}) sous la forme $t \ge
  \sqrt{\frac{1-t^2}{n+1}}-\eps_0$, ou encore :
  \[
  \left(\frac{n+2}{n+1}\right)t^2 + 2t\eps_0 + \eps_0^2 - \frac{1}{n+1} \ge
  0 \pmm{,}
  \]
  qui, compte tenu de l'hypothèse sur $\eps_0$ et de la positivité de $t$,
  implique :
  \begin{align*}
  t & \ge \frac{n+1}{n+2}\sqrt{\frac{-\eps_0^2}{n+1}+\frac{n+2}{(n+1)^2}} -
  \left(\frac{n+1}{n+2}\right)\eps_0\\
  & \ge \frac{\sqrt{n+1}}{n+2} - \eps_0 \pmm{,}
  \end{align*}
  comme annoncé. Le cas particulier s'en déduit immédiatement en substituant
  et en observant que $n \ge 2$.
\end{proof}

On va maintenant s'attacher à contrôler la variation de distance quand on
prend la différence de deux points proches de l'origine. D'après la
proposition 3.7 de \cite{daphimhva2}, il existe des familles de formes
bihomogènes de degrés $(2, 2)$ représentant le morphisme de différence
$\A\times\A \to \A$ dans le plongement choisi en tout point où elles
définissent un morphisme projectif, c'est-à-dire ne sont pas identiquement
nulles. De plus, pour chaque couple $(x, y)$, il existe une famille de telles
formes ne s'y annulant pas identiquement. On peut en particulier choisir une
famille de formes représentant la différence dans un voisinage de
\bsc{Zariski} de $(\OA, \OA)$. Un tel voisinage contient certainement un
produit de boules $v$-adiques de rayon non nul centrées en l'origine. Le lemme
suivant donne une valeur admissible pour le rayon.

\begin{lem} \label{rayonRel}
  Soient $v$ une place de $\cdn$ et $D$ une famille de formes représentant la
  différence dans un voisinage de \bsc{Zariski} de $(\OA, \OA)$. On pose
  $c_v(D) = \nv{D(\coa, \coa)}/(\nnv{D}\nv{\coa}^4)$ et $r_v(D) =
  c_v(D)/(20n\sqrt{n+1})^\dv$. La famille $D$ représente alors la différence
  sur tout $(B_v(\OA, r_v(D)))^2 \subset \A^2$.
\end{lem}

\begin{proof}
  Après avoir fixé $v$, on choisit $i_v$ donné par la première partie du lemme
  \ref{pv}, et comme on travaille à $v$ fixé on supposera encore $i_v = 0$
  pour alléger les notations. On choisit de plus $\p{x}$ et $\p{y}$ dans
  $(B_v(\OA, r_v(D)))^2$ ; on peut ainsi écrire $\max(\Dv(\p x, \OA), \Dv(\p
  y, \OA)) = r_v(D)\cdot\eps_{1, v}$ pour un certain $0 < \eps_{1, v} < 1$.
  Ces points vérifient \lat{a fortiori} les hypothèses du lemme \ref{pv} ; en
  particulier $\coa_0$, $x_0$, $y_0$ sont tous non nuls et on peut écrire
  comme suit le développement de \bsc{Taylor} bihomogène de $D$ au voisinage
  de $(\coa, \coa)$ :
  \[
  D(x, y) = \frac{x_0^2y_0^2}{\coa_0^4}D(\coa, \coa) + \sum_E R_{i, j}
  \pmm{, où :}
  \]
  \[
  R_{i, j} =
  \frac{x_0^{2-\lgr{i}}y_0^{2-\lgr{j}}}{\coa_0^{4-\lgr{i}-\lgr{j}}}\cdot
  \frac{1}{i!j!} \cdot
  \frac{\partial^{\lgr{i}+\lgr{j}}D}{\partial X^i \partial Y^j}(\coa, \coa)
  \cdot
  \prod_{k=1}^n \left( x_{k} - \frac{x_0}{\coa_0} \coa_{k} \right)^{i_k}
  \cdot \prod_{k=1}^n \left( y_{k} - \frac{y_0}{\coa_0} \coa_{k}
  \right)^{j_k}
  \]
  et la somme est prise sur :
  \[
  E = \left\{ (i, j) \in (\{0\} \times \N^n)^2 \text{ tels que }
    \lgr{i},\lgr{j} \le 2 \text{ et } \lgr{i}+\lgr{j} \neq 0 \right\}
  \pmm{.}
  \]

  On majore $\nv{x_k - \frac{x_0}{\coa_0} \coa_k}$ par $\nv{x \wedge
  \coa}/\av{\coa_0}$, qui, vu l'hypothèse sur $\p{x}$, est lui-même majoré
  par $r_v(D)\cdot\eps_{1, v} \nv{x} (\nv{\coa}/\av{\coa_0})$ ; on procède de
  même pour les facteurs en $y$.

  On note $a = \lgr{i}$, $b=\lgr{j}$ ; chaque $\nv{R_{i, j}}$ est alors majoré
  comme suit :
  \begin{align*}
  \nv{R_{i, j}} & \le
  \nv{\frac{x_0^2y_0^2}{\coa_0^4}D(\coa, \coa)}
  \frac{(c(a)c(b))^\dv \nnv{D}\nv{\coa}^4}{\nv{D(\coa, \coa)}}
  \frac{\nv{x}^a}{\av{x_0}^a} \frac{\nv{y}^b}{\av{y_0}^b} (r_v(D)\eps_{1,
    v})^{a+b}\\
  & \le \left( \frac{c(a)c(b)}{\left( 10n \right)^{a+b}} \right)^\dv
  \nv{\frac{x_0^2y_0^2}{\coa_0^4}D(\coa, \coa)} \eps_{1, v} \pmm{,}
  \end{align*}
  où $c(u)$ vaut $1$ si $u=0$, $2$ si $u=1$, $\sqrt{2}$ si $u=2$, de sorte que
  \[
  \frac{1}{i!j!}\nnv{\frac{\partial^{a+b}D}{\partial X^i \partial Y^j}} \le
  (c(a) c(b))^\dv \nnv{D}\pmm{,}
  \]
  comme on s'en convainc en le vérifiant sur chaque type de monôme intervenant
  dans $D$.

  Aux places finies, chaque $R_{i, j}$ est ainsi strictement plus petit en
  norme que $\frac{x_0^2y_0^2}{\coa_0^4}D(\coa, \coa)$, et $\nv{D(x, y)} =
  \nv{\frac{x_0^2y_0^2}{\coa_0^4}D(\coa, \coa)}$ par la propriété
  ultramétrique.  Aux places infinies, $\nv{D(x, y)} \ge (1-\sum_E R'_{i, j})
  \nv{\frac{x_0^2y_0^2}{\coa_0^4}D(\coa, \coa)}$, où l'on a posé $R'_{i, j} =
  \frac{c(a)c(b)}{ \left( 10n \right)^{a+b}}$. Il s'agit alors de majorer la
  somme des $R'_{i, j}$. Pour celà on remarque que, dans chaque sous-ensemble
  de $E$ où $a$ et $b$ sont fixés, $R'_{i, j}$ est constant. En comptant
  maintenant le nombre de termes intervenant dans chacune de ces sommes
  partielles, il vient $\sum_E R'_{i, j} \le 1/2$, et finalement $\nv{D(x, y)}
  \ge \frac{1}{2} \nv{\frac{x_0^2y_0^2}{\coa_0^4}D(\coa, \coa)}$. En
  particulier, $D(x, y)$ n'est pas identiquement nulle et $D$ représente bien
  la différence au voisinage de $(\p{x}, \p{y})$.
\end{proof}

On remarque que la valeur annoncée pour le rayon aux places archimédiennes
n'est pas optimale ; on pouvait prendre $r_v(D) = c_v(D)/(11n\sqrt{n+1})$ sans
plus de peine. L'intérêt de la valeur choisie est de pouvoir réutiliser tels
quels les calculs ci-dessus dans le lemme suivant, qui énonce quantitativement
ce que l'on perd en précision d'approximation de l'origine en prenant une
différence.

\begin{lem} \label{DistRel}
  On reprend les notations du lemme précédent ; on fixe de plus $0 < \eps_{1,
  v} < 1$. Pour tous points $\p{x}$ et $\p{y}$ de $\A(\cdn)$ satisfaisant
  \[
  \max ( \Dv(\p{x}, \OA), \Dv(\p{y}, \OA) ) < r_v(D) \cdot \eps_{1,v} \pmm{,}
  \]
  on a : $\Dv(\p{x}-\p{y}, \OA) \le \eps_{1,v}$.
\end{lem}


\begin{proof}
  Le lemme précédent s'applique, et nous assure que la distance de
  $\p{x}-\p{y}$ à l'origine  est donnée par :
  \begin{align*}
  \Dv(\p{x}-\p{y}, \OA) & = \frac{\nv{D(x, y) \wedge \coa}}{\nv{D(x,
    y)}\nv{\coa}} \\
  & \le \av{\frac{x_0^2y_0^2}{\coa_0^4}}\frac{\nv{D(\coa, \coa)\wedge
    \coa}}{\nv{D(x, y)}\nv{\coa}} + \frac{\nv{\sum_E (R_{i, j} \wedge
    \coa)}}{\nv{D(x, y)}\nv{\coa}} \pmm{.}
  \end{align*}
  Le premier terme est nul car, $D(\coa, \coa)$ et $\coa$ étant deux systèmes
  de coordonnées de l'origine, ils sont colinéaires et ont donc un produit
  extérieur nul. On majore ensuite brutalement chaque $\nv{R_{i, j} \wedge
    \coa}$ par $\nv{R_{i, j}} \cdot \nv{\coa}$. Vu la minoration précédente de
  $\nv{D(x, y)}$ et après simplification, la norme de chaque terme de la somme
  restante est majorée par $(2R'_{i, j})^\dv \eps_{1, v}$. La propriété
  ultramétrique aux places finies, resp. la majoration ci-dessus de $\sum_E
  R'_{i, j}$ aux places infines, permettent alors de conclure.
\end{proof}

Le défaut des deux énoncés précédents est que les constantes $c_v(D)$, donc
$r_v(D)$ dépendent \lat{a priori} du choix de $D$. On aimerait donc choisir une
famille $D$ pour laquelle on puisse contrôler $c_v(D)$ en fontion de paramètres
plus intrinsèques. Le candidat naturel est la constante locale $c_v(\A)$, que
nous allons maintenant définir précisément. D'après le lemme 3.5 de
\cite{daphimhva2}, il existe dans chaque classe de $K_2 \times
\widehat{K_2(4)} \mod 2K_2 \times \widehat{K_2(4)}^2$ au moins un indice $(a,
l)$ tel que $\coa_{(a,l)} \neq 0$. On note $\mathcal{Z}_\BA$ un système complet
de représentants vérifiant cette condition, et on pose $\BA = (\coa_{(a,
l)}^{-1})_{(a, l) \in \mathcal{Z}_\BA}$. La constante $c_v(\A) =
\nv{\OA}\nv{\BA} \ge 1$ est alors bien définie ; c'est l'équivalent local de
$h(\OA)$. Le lemme suivant affirme qu'un bon choix de $D$ est possible et
précise la constante de comparaison.

\begin{lem} \label{RayonAbs}
  Pour toute place $v$ de $\cdn$, il existe une famille $D$ de formes
  multihomogènes de degré $(2, 2)$ représentant la différence dans un
  voisinage de $(\OA, \OA)$, et telle que $c_v(D) \ge
  \nv{2}^g(2^g(n+1)^{3/2})^{-\dv} c_v(\A)^{-1}$.
\end{lem}

Avant de démontrer le lemme, il convient de rappeler quelques faits concernant
le plongement utilisé. On a dit que les coordonnées étaient naturellement
indexées par $\mathcal{Z}_2 \times \widehat{K_2(4)}$, où $\mathcal{Z}_2$ est
un système complet de représentants des classes de $K_2$ modulo $K_2(4)$. Il
est parfois plus commode d'utiliser un système étendu de coordonnées, indexées
par tout $K_2 \times \widehat{K_2(4)}$, comme définies dans
\cite{daphimhva2} p.~651 ; nous noterons $\tilde{\Theta}$ le plongement
correspondant, $\tilde{\coa}$ des coordonnées de l'origine dans ce plongement,
et ainsi de suite. Le passage de $\tilde{\Theta}$ à $\Theta$ consiste en une
simple projection linéaire ; le sens contraire est donné par le point (iv) du
fait 3.3 de \lat{loc. cit.} : il nous suffira de savoir que les coordonnées
manquantes s'obtiennent en multipliant les anciennes par des racines
quatrièmes de l'unité.

\begin{proof}
  Après avoir fixé la place $v$, on choisit comme au lemme \ref{pv} un indice
  $i_v$, correspondant à $(a, l) \in \mathcal{Z}_2 \times \widehat{K_2(4)}$ et
  maximisant $\av{\coa_i}$. On définit alors une famille de formes $\tilde{D}$
  en posant, pour tout $(b, k)$ de $K_2 \times \widehat{K_2(4)}$ :
  \begin{multline*}
  \tilde{D}_{(b, k)}(\tilde{X}, \tilde{Y}) = \tilde{\coa}_{(b, k)}^{-1}
  \sum_{\stackrel{d \in K(4)/K(2)}{u \in \widehat{K(2)}}} l(d)
  \tilde{X}_{(\frac{c_1+d}{2}, l_1\cdot2\star u)}
  \tilde{X}_{(\frac{c_2+d}{2}, l_2\cdot2\star u)} \\\times
  \tilde{Y}_{(\frac{c_3+d}{2}, l_3\cdot2\star u)}
  \tilde{Y}_{(\frac{c_4+d}{2}, l_4\cdot2\star u)} \pmm{,}
  \end{multline*}
  où les différents indices $(a', l')$, $(b', k')$, $c_i$, $l_i$, sont définis
  comme indiqué dans la proposition 3.7 de \cite{daphimhva2}. On peut de
  plus choisir ici $(a', l')= (a, l)$, vu que $\coa_{(a, l)}$ est déjà non nul
  par hypothèse, et exiger $(b', k') \in \mathcal{Z}_\BA$. La proposition
  citée affirme alors que la famille $\tilde{D}$ représente le morphisme de
  différence $(\tilde\Theta(\A))^2 \to \tilde\Theta(\A)$ et qu'en
  $(\tilde\coa, \tilde\coa)$ on a plus précisément l'égalité vectorielle
  $\tilde{D}(\tilde\coa, \tilde\coa) = \left( 2^g (\tilde\coa_{(a,
    l)})^2\right) \tilde\coa$. On savait bien sûr que ces deux vecteurs
  étaient colinéaires, le point est qu'on connait explicitement le coefficient
  qui les relie.

  On redescend de $\tilde\Theta$ au plongement $\Theta$ comme évoqué dans la
  discussion précédente. On déduit ainsi de $\tilde D$ une famille $D$ de
  formes bihomogènes de degré $(2, 2)$ représentant la différence au voisinage
  de $(\OA, \OA)$ dans le plongement $\Theta$, satisfaisant $D(\coa, \coa) =
  2^g \coa_{i_v}^2 \cdot \coa$, et de la forme $D_i(X, Y)= \coa_{j(i)} \sum
  \zeta_{k,l,m,n} X_kX_lY_mY_n$, où $\zeta_{k,l,m,n}$ est une racine de
  l'unité et la somme est prise sur une famille\footnote{Rien ne permet \lat{a
    priori} d'affirmer qu'il s'agit d'un sous-ensemble, certains indices
  pouvant être répétés.} à $4^g$ éléments de $\{0, \dots, n\}^4$. Sans avoir
  besoin d'expliciter davantage, un calcul direct donne $\nv D \le 2^{g\dv}
  \nv\BA$ puis $\nnv{D} \le (2^g\sqrt{n+1})^\dv \nv\BA$. Comme on a de plus
  $\nv{D(\coa, \coa)} \ge \av{2}^g(n+1)^{-\dv}\nv\coa^3$, il vient finalement
  $c_v(D) \ge \av2^g (2^g(n+1)^{3/2})^{-\dv} c_v(\A)^{-1}$ et le lemme.
\end{proof}

Mettant bout à bout les résultats des lemmes \ref{DistRel} et \ref{RayonAbs},
on a ainsi montré le lemme principal, qui s'énonce comme suit.

\begin{lem} \label{clef}
  Soient $0 < \eps_{1,v} < 1$ et $r_v = \av 2^g
  (2^g20n(n+1)^2)^{-\dv}c_v(\A)^{-1}$. Pour tous $\p x$ et $\p y$ de
  $\A(\cdn)$ satisfaisant
  \[
  \max ( \Dv(\p{x}, \OA), \Dv(\p{y}, \OA) ) < r_v \cdot \eps_{1,v} \pmm{,}
  \]
  on a : $\Dv(\p{x}-\p{y}, \OA) \le \eps_{1,v}$.
\end{lem}

On est maintenant en mesure de prouver le théorème \ref{mumpt}. On procède par
l'absurde en supposant que $\p{z} = \p{x} - \p{y}$ est différent de $\OA$, on
en déduit alors grâce au lemme précédent et à l'hypothèse principale
(\ref{hyp}) sur $\p{x}$ et $\p{y}$ que sa hauteur projective ne peut pas être
trop petite devant celle de $\p{x}$. Or la géométrie euclidienne élémentaire
et les hypothèses sur les positions de $\p{x}$ et $\p{y}$ dans l'espace de
\bsc{Mordell-Weill} montrent que sa hauteur normalisée $\hn(\p{z})$ est au
contraire négligeable devant $\hn(\p{x})$. Il n'y aura alors plus qu'à
comparer $h$ et $\hn$ pour conclure. Les lemmes suivants précisent ces deux
affirmations.

\begin{lem} \label{papti}
  Soient $\p{x}$ et $\p{y}$ sont deux points distincts satisfaisant
  (\ref{hyp}) ; si leur différence $\p{z}$ n'est pas nulle, elle est de hauteur
  projective minorée par $\eps \min(h(\p{x}), h(\p{y})) - h(\OA)$.
\end{lem}

\begin{proof}
  Comme $\p{x}$ et $\p{y}$ satisfont (\ref{hyp}), on peut leur appliquer le
  lemme \ref{clef} avec $\eps_{1, v} = \min(H(\p{x}),
  H(\p{y}))^{-\eps\lambda_v}$. Par ailleurs, dire que $\p{z} \neq \OA$, c'est
  dire qu'il exite $i$ et $j$ tels que le nombre $z_i \coa_j - \coa_i z_j$ est
  non nul. On peut donc lui appliquer la formule du produit, après avoir
  remarqué que $\av{z_i \coa_j - \coa_i z_j} \le \nv{z\wedge \coa}$. Il vient
  :
  \[
  \prod_v \Dv(\p{z}, \OA)^{e_v} \ge \prod_v \frac{\av{z_i \coa_j - \coa_i
    z_j}^{e_v}}{\nv{z}^{e_v}\cdot\nv{\coa}^{e_v}} \ge \frac{1}{H(\p{z})
    H(\OA)} \pmm{.}
  \]
  On majore ensuite $\Dv(\p{z}, \OA)$ par $\min(H(\p{x}),
  H(\p{y}))^{-\eps\lambda_v}$ aux places $v$ de $S$, et par $1$ ailleurs.
  Grâce à l'hypothèse sur les $\lambda_v$, le produit de gauche est alors
  majoré par $\min(H(\p{x}), H(\p{y}))^{-\eps}$. Prenant alors les opposés des
  logarithmes dans l'inégalité précédente, on a bien $\eps \cdot
  \min(h(\p{x}), h(\p{y})) \le h(\p{z}) + h(\OA)$.
\end{proof}

\begin{lem} \label{pti}
  Si $\p{x}$ et $\p{y}$ satisfont aux hypothèses \ref{cone} et \ref{sect} du
  théorème \ref{mumpt}, alors leur différence $\p{z}$ est de hauteur
  normalisée majorée par $(\rho^2/4 + 3\phi) \hn(\p{x})$.
\end{lem}

\begin{proof}
  On note $\left<\truc ; \truc \right>$ le produit scalaire et $\nv[N]{\truc} =
  \sqrt{\hn(\truc)}$ la norme dans l'espace de \bsc{Mordell-Weil}. On remarque
  de plus que l'hypothèse \ref{sect} implique $\nv[N]{\p{y}} \le
  (1+\rho)^{1/2}\nv[N]{\p{x}} \le (1+\rho/2) \nv[N]{\p{x}}$. Il vient alors :
  \begin{align*}
  \hn(\p{z}) & = \nv[N]{\p{x}-\p{y}}^2 \\ & = \nv[N]{\p{x}}^2 +
  \nv[N]{\p{y}}^2 - 2\left< \p{x} ; \p{y} \right> \\
  & = \left(\nv[N]{\p{y}} - \nv[N]{\p{x}}\right)^2 +
  2\nv[N]{\p{x}}\nv[N]{\p{y}}\left( 1- \cos(\p{x}, \p{y}) \right) \\
  & \le \left( \frac{\rho}{2}\nv[N]{\p{x}} \right)^2 + 2\left(
  1+\frac{\rho}{2} \right) \nv[N]{\p{x}}^2 \cdot \phi \\
  & \le ((\rho^2/4) + 2\phi + \rho\phi) \hn(\p{x})\pmm{.}\qedhere
  \end{align*}
\end{proof}

\begin{proof}[Démonstration du théorème \ref{mumpt}.]
  Elle consiste à confronter les conclusions des deux lemmes précédents en
  supposant $\p{z} \neq \OA$, et à en déduire une contradiction grâce à
  l'hypothèse \ref{tronk} du thèorème. On a besoin de pouvoir comparer les
  hauteurs projective et normalisée. On utilise à cet effet le lemme 3.9 de
  \cite{daphimhva2} et la remarque subséquente : $\av[]{\hn(\truc) -
    h(\truc)} \le (4^g-1)h(\OA) + 3g\log(2)$. On passe ainsi aux hauteurs
  normalisées dans le résultat du lemme \ref{papti} :
  \begin{align*}
  \hn(\p{z}) & \ge h(\p{z}) - (4^g-1)h(\OA) - 3g\log(2)\\
  & \ge \eps \min(h(\p{x}), h(\p{y})) - h(\OA) - (4^g-1)h(\OA) - 3g\log(2)
  \\
  & \ge \eps \left(\hn(\p{x}) - (4^g-1)h(\OA) - 3g\log(2)\right) - B \\
  & \ge \eps \hn(\p{x}) - (1+\eps)B \pmm{.}
  \end{align*}
  Mais le lemme \ref{pti} donne alors $((\rho^2/4) + 3\phi) \hn(\p{x}) \ge
  \eps \hn(\p{x}) - (1+\eps)B$, ce qui contredit très précisément l'hypothèse
  \ref{tronk}.
\end{proof}

\printbibliography

\end{document}
