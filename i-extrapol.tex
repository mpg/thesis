% !TEX root = main.tex

\section{Extrapolation} \label{sec:vojta-extrap}

Nous allons maintenant montrer que la forme auxiliaire construite
précédemment s'annule avec un indice élevé en \( \excep \). Plus précisément,
posons
\begin{equation}
  \wtsum( \dermp )
  =
  \frac {\lgr{\dermp[1]}} {\wts[1]} + \dots
  + \frac {\lgr{\dermp[\puiss*]}} {\wts[\puiss*]}
  + \frac {\lgr{\dermp[\puiss]}} {\puiss*}
\end{equation}
pour tout \( \dermp \in \N^{\lgr\vdim} \).  Si \( h \) est une forme
multihomogène et \( \point \) où elle est définie, on définit son indice en \(
  \point \), noté \( \inda[\point](h) \),  comme le minimum des \(
  \wtsum*(\dermp) \) pour \( \dermp \) tel que \( \der[\dermp] h(\point) \neq 0
\).

On reprend les notations introduites par la scholie~\ref{s:aux-co} et on pose
\( f_1 = F' / \vmp[][\vdim]^\Di \). Le but de cette section est alors de montrer
la proposition suivante.

\begin{prop}
  On a \( \inda**(f_2) \ge \epsiv\delta \), où \( \epsiv \) est un réel
  strictement positif tel que \dots
\end{prop}

Commençons par un lemme facile sur l'indice.

\begin{lem}
  Soient \( h_1 \), \( h_2 \) et \( \alpha \) des fonctions rationnelles
  telles que \( h_1 = \alpha h_2 \) et \( \point \) un point où elles sont
  toutes les trois définies.
  \begin{enumthm}
    \item Si \( \dermp \) est tel que \( \der[\gmp\nu] h_2(x) = 0 \) pour tout
      \( \gmp\nu < \dermp \) pour l'ordre produit sur \( \N^{\lgr\vdim} \),
      alors \( \der[\dermp] h_1(\point) = \alpha(\point) \, h_2(\point) \).
    \item Si \( \alpha(\point) \neq 0 \) on a \( \inda**(h_1) = \inda**(h_2)
      \).
  \end{enumthm}
\end{lem}

\begin{proof}
  Le premier point découle facilement de la formule de \bsc{Leibniz} :
  \begin{equation}
    \der[\dermp] h_1(\point)
    =
    \sum_{\gmp\nu \le \dermp}
    \der[\dermp - \gmp\nu] \alpha(\point) \,
    \der[\gmp\nu] h_2(\point)
    \pmm.
  \end{equation}
  Or, par hypothèse, tous les termes de cette somme sont nuls sauf peut-être
  celui où \( \gmp\nu = \dermp \).

  Si \( \alpha(\point) \neq 0 \) alors \( \alpha^{-1} \) est également définie
  en \( \point \) et les deux autres fonctions jouent donc un rôle symétrique.
  Ainsi, si un indice \( \dermp \) est minimal pour la condition \(
    \der[\dermp] h_1(\point) \neq 0 \), il l'est aussi pour la condition \(
    \der[\dermp] h_2(\point) \neq 0 \) grâce au point précédent, ce qui prouve
  que les deux fonctions ont le même indice en \( \point \).
\end{proof}


\clearpage

\begin{lem}
  Soit \( \point \) un point de \( \var \) satisfaisant~\ref{e:HA} et de
  grande hauteur ; on note \( \cmp \) ses coordonnées multihomogènes.  Pour
  chaque \( \fct \in \{ 1, \dots, \puiss \} \) et chaque place \( \place \) de
  \( \cdn \), il existe un indice \( \ind_{\place, \fct} \in \{ 1, \dots,
      \vdim* \} \) tel que
  \begin{equation}
    \av{\cmp*[\ind_{\place, \fct}]}
    \le
    \cst{ind-norm-dim-loc} \cdot \Onv{\cmp*}
    \pmm,
  \end{equation}
  avec
  \begin{equation}
    \newcst{ind-norm-dim-loc} = \dots
  \end{equation}
\end{lem}

\begin{ideas}
  C'est une variante plus précise du lemme~\ref{ChoixI}. Les points sont :
  \begin{enumthm}
    \item pouvoir choisir \( \ind \) plus petit que \( \vdim* \). Pour ça, on
      utilise que le point est sur \( \var \) pour écrire que les dernières
      coordonnées sont intégralement dépendantes sur les premières
      (fait~\ref{f:plong-adapt-dep} ou directement Rémond si besoin) avec des
      relations connues, donc majorer la norme en fonction du maximum des
      valeurs absolues des première coordonnées.
    \item pouvoir choisir \( \ind \) différent de \( 0 \). Là on utilise le
      fait que le point est proche de \( \divi \), donc par définition de la
      distance, sa coordonnée d'indice \( 0 \) est petite devant sa norme,
      donc ça ne peut pas être elle qu'on a choisi avant.
  \end{enumthm}
\end{ideas}

\begin{lem}
  Quitte à faire sur chaque facteur un changement de coordonnées linéaire
  laissant invariantes les coordonnées \( \cmp*[0], \cmp*[\vdim*+1], \dots,
    \cmp*[\dimp] \), on peut supposer que
  \begin{equation}
    \prod_{\place \in \placesapx}
    \av{\vmp*[\vdim*]}
    \le
    \prod_{\place \in \placesapx}
    \cst{ind-norm-dim} \cdot \Onv{\coord\mexp*}
    \pmm,
  \end{equation}
  avec
  \begin{equation}
    \newcst{ind-norm-dim} = \dots
  \end{equation}
  De plus, on peut choisir la transformation linéaire donnée par une matrice
  de hauteur \dots
\end{lem}

\begin{scho}
  On peut supposer que la partie en \( \pden \) des dénominateurs dans les
  développements en série ne s'annulle pas en \( \excep \).
\end{scho}

\begin{ideas}
  En effet, ces dénominateurs sont de degré bornés en fonction uniquement de
  \( \var \) indépendamment de \( \delta \). Donc si l'un d'eux s'annulle en
  \( \excep \), il fournit la forme par laquelle on veut couper \( \var \)
  pour montrer qu'elle n'est pas minimale, donc on n'a pas besoin de toute la
  construction qui suit.
\end{ideas}

\begin{lem}
  Pour tout \( \dermp \), on a
  \worknote{On a un problème d'homogénéité là, non ?}
  \begin{equation}
    \prod\placerange
    \Bigl( \prod\fctrange
      \prod_{\ind=\vdim*+1}^{\dimp} (\pden\mexp**)^{2\lgr{\dermp*}}
    \Bigr)^{-1}
    \le
    \cst{form-prod}
    \prod\fctrange \bigl(
      \hautm[1]{\varfc*} \hautm[1]{e_\fct}^{\vdeg* - 1}
    \bigr)^{2 (\dimp - \vdim*) \lgr{\dermp*}}
    \pmm.
  \end{equation}
\end{lem}

\begin{ideas}
  C'est la formule du produit.
\end{ideas}

\begin{lem} \label{l:par-var-spe}
  Le développement de $F''$ en $\excep$ s'écrit $\tau(F'')(\excep) = \sum_\dermp
  \xi_\dermp \psmp^\dermp$ avec : \( \av{ \xi_\dermp } \le c(\eps,
    \hautl\excep, \epsiv, \dots) \) dès que que \( \dermp \in
    \stairs[\delta\epsiv] \) et pour \( \place \in \placesapx \).
\end{lem}

\begin{ideas}
  On spécialise en \( \excep \) le résultat du lemme~\ref{l:par-var} appliqué
  à \( F'' \), en tenant compte des résultats des lemmes précédents :
  \begin{enumthm}
    \item l'un contrôle la contribution du dénominateur ;
    \item pour le numérateur, on utilise le fait qu'il a un indice élevé le
      long de \( \divi \) pour faire sortir \( \av{\cex*[0]} /
        \av{\cex*[\vdim*]} \), qui se compare à \( \distv\excep\divi \) ;
    \item cette dernière se compare alors à \( \hautl\excep \).
  \end{enumthm}
\end{ideas}

\begin{lem}
  La hauteur de \( \wemb(\excep) \) n'est pas trop grosse dès que \( \excep \)
  satisfait à la condition de secteur de cône. Plus précisément, il faut
  considérer la hauteur sur chaque composante, et c'est sur les dernières que
  ce n'est pas trop gros.
\end{lem}

\begin{ideas}
  Avec la hauteur normalisée, c'est de la géométrie euclidienne. Il suffit
  ensuite de comparer à une hauteur projective.
\end{ideas}

\begin{lem} \label{l:par-img-spe}
  Le développement de $RF'$ en $\excep$ s'écrit
  \( \tau(F'')(\excep) = \sum_\dermp \zeta_\dermp \psmp^\dermp \)
  avec
  \( \hautl{\zeta_\dermp} \le c(\hautl\excep, \lgr\dermp, \dots ) \).
\end{lem}

\begin{ideas}
  On spécialise en \( \excep \) le résultat du lemme~\ref{l:par-var} appliqué
  à \( RF' \), en tenant compte des points suivants :
  \begin{enumthm}
    \item la contribution du dénominateur est contrôlée précédemment ;
    \item on utilise le lemme~\ref{l:der-wemba} appliqué à \( R F \) pour dire
      que les dénominateurs proviennent de formes sur \( \wemb(\var) \), puis
      l'estimation de hauteur de \( \wemb(\excep) \) précédente pour gagner
      sur le dénominateur ;
    \item au cours de cette estimation, on risque de sortir de facteurs
      \( \nv1{\excep}^{\dots} / \nv1{Q(\excep)} \) où \( Q \) est une des
      formes de multiplication-soustraction. On contrôle alors ces facteurs
      par un lemme adéquat.
  \end{enumthm}
  Note : on a \( \wemba(R F) = R F' \), de sorte qu'on a bien le droit
  d'appliquer le lemme à \( R F \) pour contrôler le développement de \( R F'
  \).
\end{ideas}

\begin{lem}
  On a \( \inda**(R \cdot F') \ge \epsiv\delta \), avec
  \begin{equation}
    \epsiv = \dots
  \end{equation}
\end{lem}

\begin{ideas}
  On commence par constater que les deux développements en série autour de \(
    \excep \) étudiés précédemment sont égaux, car il s'agit du développement
  de la même fonction dans \( \korper\var \), relativement aux mêmes
  paramètres.

  Du coup, pour les coefficients de \( \stairs[\delta\epsiv] \), on contrôle
  finement la valeur absolue en les places de \( \placesapx \), et on contrôle
  aussi la hauteur. Si tout va bien, c'est assez précis pour contredire la
  formule du produit, donc ça prouve qu'ils sont nuls.
\end{ideas}

\endinput

% vim: spell spelllang=fr
