% !TEX root = main.tex

\section{Extrapolation} \label{sec:vojta-extrap}

Nous allons maintenant montrer que la forme auxiliaire construite
précédemment s'annule avec un indice élevé en \( \excep \). Plus précisément,
posons
\begin{equation}
  \wtsum( \dermp )
  =
  \frac {\lgr{\dermp[1]}} {\wts[1]} + \dots
  + \frac {\lgr{\dermp[\puiss*]}} {\wts[\puiss*]}
  + \frac {\lgr{\dermp[\puiss]}} {\puiss*}
\end{equation}
pour tout \( \dermp \in \N^{\lgr\vdim} \).  Si \( h \) est une forme
multihomogène et \( \point \) où elle est définie, on définit son indice en \(
  \point \), noté \( \inda[\point](h) \),  comme le minimum des \(
  \wtsum*(\dermp) \) pour \( \dermp \) tel que \( \der[\dermp] h(\point) \neq 0
\).

On reprend les notations introduites par la scholie~\ref{s:aux-co} et on pose
\( f_1 = F' / \vmp[][\vdim]^\Di \). Le but de cette section est alors de montrer
la proposition suivante.

\begin{prop}
  On a \( \inda**(f_2) \ge \epsi \delta / \sigma \), où \( \sigma \) est tel
  que \dots
\end{prop}

Commençons par un lemme facile sur l'indice.

\begin{lem} \label{l:indice-inversible}
  Soient \( g_1 \), \( g_2 \) et \( \alpha \) des fonctions rationnelles
  telles que \( g_1 = \alpha g_2 \) et \( \point \) un point où elles sont
  toutes les trois définies.
  \begin{enumthm}
    \item Si \( \dermp \) est tel que \( \der[\gmp\nu] g_2(x) = 0 \) dès que
      \( \gmp\nu < \dermp \) pour l'ordre produit sur \( \N^{\lgr\vdim} \),
      alors \( \der[\dermp] g_1(\point) = \alpha(\point) \, g_2(\point) \).
    \item Si \( \alpha(\point) \neq 0 \) on a \( \inda**(g_1) = \inda**(g_2)
      \).
  \end{enumthm}
\end{lem}

\begin{proof}
  Le premier point découle facilement de la formule de \bsc{Leibniz} :
  \begin{equation}
    \der[\dermp] g_1(\point)
    =
    \sum_{\gmp\nu \le \dermp}
    \der[\dermp - \gmp\nu] \alpha(\point) \,
    \der[\gmp\nu] g_2(\point)
    \pmm.
  \end{equation}
  Or, par hypothèse, tous les termes de cette somme sont nuls sauf peut-être
  celui où \( \gmp\nu = \dermp \).

  Si \( \alpha(\point) \neq 0 \) alors \( \alpha^{-1} \) est également définie
  en \( \point \) et les deux autres fonctions jouent donc un rôle symétrique.
  Ainsi, si un indice \( \dermp \) est minimal pour la condition \(
    \der[\dermp] g_1(\point) \neq 0 \), il l'est aussi pour la condition \(
    \der[\dermp] g_2(\point) \neq 0 \) grâce au point précédent, ce qui prouve
  que les deux fonctions ont le même indice en \( \point \).
\end{proof}

On note \( \cex \) un système de coordonnées multihomogènes du point \(
  \Excep \) introduit à la section~\ref{sub:wemba}. On peut supposer qu'aucune
des formes \( \pden_\fct \) introduites avant le lemme~\ref{l:par-var} ne
s'annulle ne \( \cex* \), car elles sont de degré borné. De même, on peut
supposer que la forme \( R \) donnée par le corollaire~\ref{c:hmat-relim} ne
s'annulle pas en \( \cex \), car c'est un produit de formes de degré bornés.
Pour la même raison, on peut en fait supposer qu'aucune des \( \cex** \) n'est
nul ; ci-dessous on utilisera le fait que \( \cex*[\vdim*] \neq 0 \).

On choisit, pour tout \( \fct \), un indice \( \indv* \in \set{0, \dots,
    \vdim*} \) de sorte que \( \av{ \cex*[\indv*] } \) soit maximal sous cette
condition. De même, on choisit \( \indiv* \in \set{0, \dots, \dimp} \)
maximisant \( \av{ \clab*[\indiv*](\cex) } \), ainsi que \( \indig* \in
  \set{0, \dots, \dimp} \) tel que \( \clab*[\indig*](\cex) \neq 0 \). Notons
que \( \indig \) et \( \indiv \) ne dépendent pas du choix des formules \(
  \clab \) utilisées, et que \( \indig \) ne dépend pas de \( \place \).

Il est alors clair que \( \av{ \clab*[\indiv*](\cex) } \ge \nv1{ \clab*(\cex)
  } \dimp**^{-\dv} \).  Le lemme suivant fournit un analogue pour les \(
  \indv* \).

\begin{lem} \label{l:coord-norm}
  Avec les notations précédentes, on a
  \begin{equation}
    \av{ \cex*[\indv*] }
    \ge
    \nv1{ \cex* } \cdot \nv1{ \varfc* }^{-1} \dimp**^{-\dv}
    \pmm.
  \end{equation}
\end{lem}

\begin{proof}
  Il suffit de montrer qu'on a \( \av{ \cex** } \le \nv1{ \varfc* } \av{
      \cex*[\indv] } \) pour tout \( \ind \in \set{\vdim* + 1, \dots, \dimp}
  \). Notons \( \poldep** \) la relation dépendance intégrale donnée par la
  scholie~\ref{s:plong-adapt} ; en décomposant suivant les puissance de \(
    \vmp** \) on a
  \begin{equation}
    (\vmp**) ^{ \vdeg* }
    =
    \sum_{ \alpha=1 }^{ \vdeg* }
    (\vmp**) ^{ \vdeg* - \alpha }
    \poldep*[\ind, \alpha]
    \quad\text{où }
    \deg \poldep*[\ind, \alpha] = \alpha
    \text{ et }
    \sum_{ \alpha=1 }^{ \vdeg* } \nv1{ \poldep*[\ind, \alpha] }
    \le \nv1{ \varfc* }
  \end{equation}
  En passant aux valeurs absolues et en divisant, il vient
  \begin{equation}
    \av{ \cex** }
    \le
    \sum_{ \alpha=1 }^{ \vdeg* }
    \poldep*[\ind, \alpha]
    \left(
      \frac{ \av{\cex*[\indv]} }{ \av{\cex**} }
    \right) ^{ \vdeg* - \alpha }
  \end{equation}
  Si le quotient apparaissant dans le membre de droite est inférieur à \( 1
  \), notre assertion initiale est vérifiée. Sinon, elle l'est aussi, car car
  \( \varfc* \) est normalisé (scholie citée) de façon à ce qu'un de ses
  coefficients soit \( 1 \) ce qui assure \( \nv1{ \varfc* } \ge 1 \) en toute
  place.
\end{proof}

Introduisons maintenant une nouvelle fonction rationnelle définie par
\begin{equation}
  f_2 =
  \frac{ R(\vmp) }{ \vmp[][\vdim]^r }
  \cdot
  \frac{
    F( \vmp, \clab(\vmp) )
  }{
    \vmp[][\vdim] ^{\epsz \wt \delta}
    \cdot
    \clab[][\indig](\vmp) ^{ \delta }
  }
  \pmm,
\end{equation}
où au dénominateur on a noté
\begin{equation}
  \clab[][\indig](\vmp) ^{ \delta }
  =
  \prod\fctirange \clab*[\indig](\vmp) ^{ \delta }
  \pmm.
\end{equation}
D'après le lemme~\ref{l:indice-facteur}, \( f_1 \) et \( f_2 \) ont le même
indice en \( \cex \). Nous allons maintenant décomposer \( f_2 \) en facteurs
qui seront plus faciles à contrôler.

On commence par choisir pour chaque \( \place \) des formes \( \clab[\place,
    \fcti] \) en appliquant le fait~\ref{f:hclab} avec \( (\pp, \ppi) =
  (\excep[\fcti], \excep[\puiss]) \) et \( (\alpha, \beta) = (\wti*, \wt**)
\). On écrit alors
\begin{align}
  f_2
  & =
  \frac{ R(\vmp) }{ \vmp[][\vdim]^r }
  \cdot
  \frac{
    F( \vmp, \clab[\place](\vmp) )
  }{
    \vmp[][\vdim] ^{\epsz \wt \delta}
    \cdot
    \clab[\place][\indig](\vmp) ^{ \delta }
  }
  \\ & =
  \frac{ R(\vmp) }{ \vmp[][\indv]^r }
  \cdot
  \frac{
    F( \vmp, \clab[\place](\vmp) )
  }{
    \vmp[][\indv] ^{\epsz \wt \delta}
    \cdot
    \clab[\place][\indiv](\vmp) ^{ \delta }
  }
  \left(
    \frac{ \clab[\place][\indiv](\vmp) }{ \clab[\place][\indig](\vmp) }
  \right) ^{ \delta }
  \left(
    \frac{ \vmp[][\indv] }{ \vmp[][\vdim] }
  \right) ^{ \epsz \wt \delta + r }
  \\ & =
  \frac{
    R(\vmp) \cdot F( \vmp, \clab[\place](\vmp) )
  }{
    \vmp[][\indv] ^{ \Diii }
  }
  \left(
    \prod\fctirange
    \frac{
      ( \vmp[\fcti ][{\indv[\fcti ]}] )^{ 2\wts[\fcti ] }
      ( \vmp[\puiss][{\indv[\puiss]}] )^{ 2\wts[\puiss] }
    }{
      \clab[\place, \fcti][\indiv*]( \vmp )
    }
  \right)^\delta
  \left(
    \frac{ \clab[][\indiv](\vmp) }{ \clab[][\indig](\vmp) }
  \right) ^{ \delta }
  \left(
    \frac{ \vmp[][\indv] }{ \vmp[][\vdim] }
  \right) ^{ \epsz \wt \delta + r }
\end{align}

Soit maintenant \( \dermp \in \N^{\lgr\vdim} \) un indice minimal tel que \(
  \der[\dermp] f_2 \neq 0 \). Comme les derniers facteurs de l'écriture
précédente sont tous inversibles en \( \cex \), le
lemme~\ref{l:indice-inversible} montre que \( \der[\dermp] f_2 \) est égal à

\begin{equation}
  \der[\dermp]
    \frac{
      R(\vmp) \cdot F( \vmp, \clab[\place](\vmp) )
    }{
      \vmp[][\indv] ^{ \Diii }
    }
  ( \cex )
  \cdot
  \left(
    \prod\fctirange
    \frac{
      ( \cex[\fcti ][{\indv[\fcti ]}] )^{ 2\wts[\fcti ] }
      ( \cex[\puiss][{\indv[\puiss]}] )^{ 2\wts[\puiss] }
    }{
      \clab[\place, \fcti][\indiv*]( \cex )
    }
  \right)^\delta
  \left(
    \frac{ \clab[][\indiv](\cex) }{ \clab[][\indig](\cex) }
  \right) ^{ \delta }
  \left(
    \frac{ \cex[][\indv] }{ \cex[][\vdim] }
  \right) ^{ \epsz \wt \delta + r }
\end{equation}
Nous allons maintenant estimer la valeur absolue de chacun des facteurs, en
distinguant selon que \( \place \in \placesapx \) ou pas pour le premier
facteur, et montrer que l'on contredit la formule du produit si \( \wtsum(
  \dermp ) < \epsi \delta / \sigma \). Rappelons qu'on note \( \degv \) le
degré local de \( \cdn \) en \( \place \).

Pour le facteur le plus à droite, on a facilement
\begin{equation}
  \prod_\place
  \prod\fctrange
  \left(
    \frac{ \av{\cex[][\indv]} }{ \av{\cex[][\vdim]} }
  \right) ^{ (\epsz \delta \wt* + r_\fct) \degv }
  \le
  \prod\fctrange
  \hautm[\infty]{ \excep* }^{\epsz \delta \wt* + r_\fct}
  \le
  \hautm[1]{ \excep[1] }^{2\puiss \epsz \delta \wt[1]} \cdot o(\expb^\delta)
\end{equation}
De plus, en notant \( \excepi \) l'image de \( \excep \) par la deuxième
partie du plongement éclatant, on a\worknote{Farhi p. 96 en haut}
\begin{equation}
  \prod_\place
  \prod\fctirange
  \left(
    \frac{ \av{\clab[][\indiv](\cex)} }{ \av{\clab[][\indig](\cex)} }
  \right) ^{ \delta \degv }
  \le
  \prod\fctirange
  \hautm[1]{ \excepi* }^\delta
  \le
  \hautm[1]{ \excep[1] }^{2\puiss \epsiv \wts[1] \delta}
  \cdot \cst{exci}^{\wts[1] \delta}
\end{equation}
Enfin, grâce au choix des formules \( \clab[\place] \), on a
\begin{equation}
  \prod_\place
  \left(
    \prod\fctirange
    \frac{
      \nv1{ \clab[\place, \fcti][\indiv*] }
      ( \cex[\fcti ][{\indv[\fcti ]}] )^{ 2\wts[\fcti ] }
      ( \cex[\puiss][{\indv[\puiss]}] )^{ 2\wts[\puiss] }
    }{
      \clab[\place, \fcti][\indiv*]( \cex )
    }
  \right)^{ \delta \degv }
  \le
  \hclab^{2\wts[1] \delta}
\end{equation}

Passons maintenant au premier facteur. Remarquons pour commencer que
\begin{equation}
  \nv1{ R(\vmp) \cdot F( \vmp, \clab[\place](\vmp) ) }
  \le
  \nv1{ F } \nv1{ \clab }^\delta \cdot o(\expb^\delta)
\end{equation}
et que le facteur en \( \nv1{ \clab } \) a déjà été intégré à l'estimation
précédente. Appliquons maintenant le lemme~\ref{l:par-var} à cette forme :
\worknote{À adapter pour remplacer \( \vdim \) par une autre coordonnée ?}
\begin{equation}
  \der[\dermp]
    \frac{
      R(\vmp) \cdot F( \vmp, \clab[\place](\vmp) )
    }{
      \vmp[][\indv] ^{ \Diii }
    }
  ( \cex )
  =
  \frac{ P^\dermp_\dv(\cex) }{ \cex[][\indv]^\Diii \pden(\cex)^{\vlg\dermp} }
  =
  \frac{ P^\dermp_\dv(\cex) }{ \cex[][\indv]^{\deg P^\dermp_\dv } }
  \cdot
  \frac{
    \cex[][\indv]^{\vlg\dermp \cdot \deg \pden}
  }{
    \pden(\cex)^{\vlg\dermp}
  }
  \pmm.
\end{equation}
On estime facilement le second facteur :
\begin{equation}
  \prod_\place
  \prod\fctrange
  \frac{
    \av{ \cex*[\indv] }^{ 2 (\vdeg*-1) (\dimp-\vdim*) \lgr{\dermp*} \degv }
  }{
    \av{ \pden_\fct(\cex*) }^{ \lgr{\dermp*} \degv }
  }
  =
  \prod\fctrange
  \hautm[\infty]{ \cex* }^{ 2 (\vdeg*-1) (\dimp-\vdim*) \lgr{\dermp*} }
  \le
  \hautm[1]{ \cex[1] }^{ \newcst[]{degdim} \wts[1] \epsi\delta/\sigma }
\end{equation}
Le premier facteur est majoré en valeur absolue par \( \nv1{ P^\dermp_\dv }
\), qui est à peu près \( \nv1{ F } \cdot \nv1{ \varfc }^{\wts[1] \epsi\delta
    / \sigma } \). Cependant, aux places de \( \placesapx \), on peut obtenir
une estimation plus fine.

En effet, chaque monôme divisé intervenant dans le déshomogénéisé de \(
  P^\dermp_\dv \) est composé d'un facteur des valeurs absolue inférieure à \(
  1 \) et d'un facteur de la forme
\begin{equation}
  \prod\fctrange
  \left(
    \frac{ \av{ \cex*[0] } }{ \av{ \cex*[\indv*] } }
  \right)^{\imp*[0]}
  \pmm,
\end{equation}
où \( \imp \) est tel que \( \wtsum*(\imp) \ge \epsi (1 - \frac1\sigma)
  \delta \). Or, par le lemme~\ref{l:coord-norm} et l'hypothèse principale, on
a
\begin{equation}
  \prod\placerange
  \left(
    \frac{ \av{ \cex*[0] } }{ \av{ \cex*[\indv*] } }
  \right)^{\degv}
  \le
  \left(
    \frac{ \av{ \cex*[0] } }{ \nv1{ \cex* } }
    \nv1{ \varfc* } \dimp**^\dv
  \right)^{\degv}
  \le
  \hautm[2]{ \excep* }^{-\eps}
  \hautm[1]{ \varfc* } \dimp**^\dv
\end{equation}
Ainsi, \todo[en négligeant pour l'instant les partie archimédiennes des
constantes], on a
\begin{align}
  \prod\placerange
  \prod\fctrange
  \left(
    \frac{ \av{ \cex*[0] } }{ \av{ \cex*[\indv*] } }
  \right)^{\imp*[0] \degv}
  & \le
  \prod\fctrange
  \left(
    \hautm[2]{ \excep* }^{-\eps}
    \hautm[1]{ \varfc* } \dimp**^\dv
  \right)^{ \imp*[0] }
  \\ & \le
  \hautm[1]{ \excep[1] }^{ -\eps \epsi(1 - \frac1\sigma) \delta \wts[1] }
  \hautm[1]{ \varfc }^{2 \wts[1] \delta }
\end{align}

En appliquant la formule du produit d'un côté et en mettant bout à bout les
estimations ci-dessus de l'autre, il vient, après avoir pris les logarithmes :
\begin{align}
  0
  & \le
  \delta \Bigl(
    2\puiss \epsz \wts[1] \hautl{ \excep[1] }
    +
    2\puiss \epsiv \wts[1] \hautl{ \excep[1] } + \wts[1] \cst{exci}
    +
    2\wts[1] \log\hclab
    +
    \cst{degdim} \wts[1] \epsi \sigma^{-1} \hautl{ \excep[1] }
    \\ & \qquad +
    \wts[1] \epsi \sigma^{-1} (\hautl{ F } + \hautl{ \varfc })
    +
    2 \wts[1] \hautl{ \varfc }
    -
    \eps \epsi(1 - \sigma^{-1}) \wts[1] \hautl{ \excep[1] }
  \Bigr)
  + o(\delta)
  \\ & \le
  \delta \Bigl(
    \wts[1] \hautl{ \excep[1] }
    \bigl(
      2\puiss \epsz + 2\puiss \epsiv + \cst{degdim} \epsi \sigma^{-1}
      -
      \eps \epsi(1 - \sigma^{-1})
    \bigr)
    + o( \wts[1] \hautl{ \excep[1] } )
  \Bigr)
  + o(\delta)
\end{align}
ce qui devient absurde dès que \( \delta \) et \( \wts[1] \hautl{ \excep[1] }
\) sont assez grands, pour peu que
\begin{equation}
  \frac{
    2\puiss \epsz + 2\puiss \epsiv + \cst{degdim} \epsi \sigma^{-1}
  }{
    \epsi(1 - \sigma^{-1})
  }
  \le
  \eps
\end{equation}

\subsection{Appendice}

\begin{fact} \todo
  On a
  \begin{equation}
  \prod\fctirange
  \hautm[1]{ \excepi* }^\delta
  \le
  \hautm[1]{ \excep[1] }^{2\puiss \epsiv \delta} \cdot \cst{exci}^{\delta}
  \end{equation}
  avec \( \newcst[]{exci} = \dots \)
\end{fact}



\begin{fact} \label{f:hclab} \todo
  Il existe une famille de constantes \( \hclab* \) ne dépendant que de \( \va
  \) telle que, pour tout point \( ( \pp, \ppi ) \in \va^2 \),  tout place
  \( \place \) de \( \cdn \), et tout couple \( ( \alpha, \beta ) \)i de
  nombres n'admettant que \( 2 \) et \( 3 \) comme facteurs premiers, il
  existe un choix de formules \( \clab \) représentant \( [ \alpha, \beta ] \)
  au voisinage de \( ( \pp, \ppi ) \) et satisfaisant
  \begin{equation}
    \frac{
      \nv1{ \clab }
      \nv1{ \cp   }^{ 2 \alpha^2 }
      \nv1{ \cpi  }^{ 2 \beta ^2 }
    }{
      \nv1{ \clab( \cp, \cpi ) }
    }
    \le \hclab*^{\alpha^2 \beta^2}
    \pmm.
  \end{equation}
  De plus, \( \hclab* = 1 \) sauf pour un nombre fini de places ; ainsi \(
    \hclab = \prod_\place \hclab*^{\degv} \) est bien défini.
\end{fact}

\endinput

% vim: spell spelllang=fr
