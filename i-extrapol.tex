% !TEX root = main.tex

\section{Extrapolation} \label{sec:vojta-extrap}

Nous allons maintenant montrer que la forme auxiliaire construite
précédemment s'annule avec un indice élevé en \( \excep \). Plus précisément,
posons
\begin{equation}
  \wtsum( \dermp )
  =
  \frac {\lgr{\dermp[1]}} {\wts[1]} + \dots
  + \frac {\lgr{\dermp[\puiss*]}} {\wts[\puiss*]}
  + \frac {\lgr{\dermp[\puiss]}} {\puiss*}
\end{equation}
pour tout \( \dermp \in \N^{\lgr\vdim} \).  Si \( h \) est une forme
multihomogène et \( \point \) où elle est définie, on définit son indice en \(
  \point \), noté \( \inda[\point](h) \),  comme le minimum des \(
  \wtsum*(\dermp) \) pour \( \dermp \) tel que \( \der[\dermp] h(\point) \neq 0
\).

On reprend les notations introduites par la scholie~\ref{s:aux-co} et on pose
\( f_1 = F' / \vmp[][\vdim]^\Di \). Le but de cette section est alors de montrer
la proposition suivante.

\begin{prop}
  On a \( \inda**(f_2) \ge \epsiv\delta \), où \( \epsiv \) est un réel
  strictement positif tel que \dots
\end{prop}

Commençons par un lemme facile sur l'indice.

\begin{lem}
  Soient \( g_1 \), \( g_2 \) et \( \alpha \) des fonctions rationnelles
  telles que \( g_1 = \alpha g_2 \) et \( \point \) un point où elles sont
  toutes les trois définies.
  \begin{enumthm}
    \item Si \( \dermp \) est tel que \( \der[\gmp\nu] g_2(x) = 0 \) dès que
      \( \gmp\nu < \dermp \) pour l'ordre produit sur \( \N^{\lgr\vdim} \),
      alors \( \der[\dermp] g_1(\point) = \alpha(\point) \, g_2(\point) \).
    \item Si \( \alpha(\point) \neq 0 \) on a \( \inda**(g_1) = \inda**(g_2)
      \).
  \end{enumthm}
\end{lem}

\begin{proof}
  Le premier point découle facilement de la formule de \bsc{Leibniz} :
  \begin{equation}
    \der[\dermp] g_1(\point)
    =
    \sum_{\gmp\nu \le \dermp}
    \der[\dermp - \gmp\nu] \alpha(\point) \,
    \der[\gmp\nu] g_2(\point)
    \pmm.
  \end{equation}
  Or, par hypothèse, tous les termes de cette somme sont nuls sauf peut-être
  celui où \( \gmp\nu = \dermp \).

  Si \( \alpha(\point) \neq 0 \) alors \( \alpha^{-1} \) est également définie
  en \( \point \) et les deux autres fonctions jouent donc un rôle symétrique.
  Ainsi, si un indice \( \dermp \) est minimal pour la condition \(
    \der[\dermp] g_1(\point) \neq 0 \), il l'est aussi pour la condition \(
    \der[\dermp] g_2(\point) \neq 0 \) grâce au point précédent, ce qui prouve
  que les deux fonctions ont le même indice en \( \point \).
\end{proof}

On note \( \cex \) un système de coordonnées multihomogènes du point \(
  \Excep \) introduit à la section~\ref{s:wemb}. On peut supposer qu'aucune
des formes \( \pden_\fct \) introduites avant le lemme~\ref{l:par-var} ne
s'annulle ne \( \cex* \), car elles sont de degré borné. De même, on peut
supposer que la forme \( R \) donnée par le corollaire~\ref{c:hmat-relim} ne
s'annulle pas en \( \cex \), car c'est un produit de formes de degré bornés. 

On choisit, pour tout \( \fct \), un indice \( \indv* \in \set{0, \dots,
    \vdim*} \) de sorte que \( \av{ \cex*[\indv*] } \) soit maximal sous cette
condition. De même, on choisit \( \indiv* \in \set{0, \dots, \dimp} \)
maximisant \( \av{ \clab*[\indiv*] } \), ainsi que \( \indig* \in \set{0,
    \dots, \dimp} \) tel que \( \clab*[\indig*] \neq 0 \). Notons que \(
  \indig \) et \( \indiv \) ne dépendent pas du choix des formules \( \clab \)
utilisées, et que \( \indig \) ne dépend pas de \( \place \).

\begin{lem}
  Avec les notations précédentes, on a
  \begin{equation}
    \av{ \cex*[\indv*] }
    \ge
    \nv1{ \cex* } \cdot \nv1{ \varfc* }^{-1} (\dots)^{-\dv}
    \pmm.
  \end{equation}
\end{lem}

\endinput

% vim: spell spelllang=fr
