% !TEX root = main.tex

\section{Extrapolation} \label{sec:vojta-extrap}

On rappelle qu'on a construit un forme auxiliaire $F$ sur $\wemb(\var)$, ou
encore $F'$ sur $\var$, de degré contrôlé par $\delta$ et pondéré par $\wt$,
telle que $R \cdot F'$ s'annule le long de $\divi$ modulo l'idéal de
$\var$ avec un indice pondéré par $\wt$ d'ordre $\delta$, où $R$ est une forme
ne dépendant que de $\var$ plongée, donc pas de $\delta$. On introduit une
forme $G$ congrue à $R \cdot F'$ modulo l'idéal de $\var$ et de degré
comparable (on peut prendre l'image de $F'$ par les différents morphismes de
réduction si on veut).

\worknote{Comment formuler efficacement la partie sur les indices ?}
On a également des morphismes de paramétrisation de $\var$
(lemme~\ref{l:par-var}) et de son image par $\wemb$ (lemme~\ref{l:par-img}).
De plus, quand on considère l'image d'un monôme par ces morphismes, si
l'exposant de $\cmh\fct[0]$ est $\alpha$, alors celui de la même variable dans
le numérateur de l'image est au moins $\alpha - \lambda\pexp\fct[0]$.

\begin{lem}
  Soit \( \point \) un point de \( \var \) satisfaisant~\ref{e:HA} et de
  grande hauteur ; on note \( \coord \) ses coordonnées multihomogènes.  Pour
  chaque \( \fct \in \{ 1, \dots, \puiss \} \), il existe un indice \( \ind
    \in \{ 1, \dots, \vdim* \} \) tel que
  \begin{equation}
    \av{\coord\mexp*[\ind]}
    \le
    \cst{ind-norm-dim} \cdot \Onv{\coord\mexp*}
    \pmm,
  \end{equation}
  avec
  \begin{equation}
    \newcst{ind-norm-dim} = \dots
  \end{equation}
  En particulier, \( \coord\mexp*[\ind] \) n'est pas nul.
\end{lem}

\begin{ideas}
  C'est une variante plus précise du lemme~\ref{ChoixI}. Les points sont :
  \begin{enumthm}
  \item pouvoir choisir \( \ind \) plus petit que \( \vdim* \). Pour ça, on
    utilise que le point est sur \( \var \) pour écrire que les dernières
    coordonnées sont intégralement dépendantes sur les premières
    (fait~\ref{f:dep-int} ou directement Rémond si besoin) avec des relations
    connues, donc majorer la norme en fonction du maximum des valeurs absolues
    des coordonnées.
  \item pouvoir choisir \( \ind \) différent de \( 0 \). Là on utilise le fait
    que le point est proche de \( \divi \), donc par définition de la
    distance, sa coordonnée d'indice \( 0 \) est petite devant sa norme, donc
    ça ne peut pas être elle qu'on a choisi avant.
  \end{enumthm}
\end{ideas}

\begin{lem}
  On peut supposer que les dénominateurs dans les développements en série ne
  s'annullent pas en \( \excep \).
\end{lem}

\begin{ideas}
  En effet, ces dénominateurs sont de degré bornés en fonction uniquement de
  \( \var \) indépendamment de \( \delta \). Donc si l'un d'eux s'annulle en
  \( \excep \), il fournit le forme par laquelle on veut couper \( \var \)
  pour montrer qu'elle n'est pas minimale, donc on n'a pas besoin de toute la
  construction qui suit.
\end{ideas}

\begin{lem}
  On peut dont majorer la valeur absolue de l'inverse de la valeur de ces
  dénominateurs en \( \excep \) par une constante fois sa hauteur.
\end{lem}

\begin{ideas}
  C'est la formule du produit.
\end{ideas}

\begin{lem} \label{l:par-var-spe}
  Le développement de $G$ en $\excep$ s'écrit
  $\tau(G)_\excep = \sum_\lambda \xi_\lambda t^\lambda$ avec :
  \begin{enumthm}
  \item \( \hautl{\xi_\lambda} \le c(\hautl\excep, \lgr\lambda, \dots ) \) ;
  \item \( \av{ \xi_\lambda } \le c(\distv(\excep, \divi), \hautl\excep,
      \sigma, \dots) \) dès que que \( \lambda \in \Stairs_\wt^{\delta/\sigma}
    \) et pour \( \place \in \placesapx \).
  \end{enumthm}
\end{lem}

\begin{ideas}
  On spécialise en \( \excep \) le résultat du lemme~\ref{l:par-var}, en tenant
  compte des résultats des lemmes précédents.

  Note : c'est pas vraiment sûr que l'estimation de hauteur soit intéressante
  ici, mais j'aimerais quand même la faire pour me convaincre que celle qu'on
  obtiendra après en passant par le plongement pondéré est meilleur.
\end{ideas}

\begin{lem}
  La hauteur de \( \wemb(\excep) \) n'est pas trop grosse dès que \( \excep \)
  satisfait à la condition de secteur de cône. Plus précisément, il faut
  considérer la hauteur sur chaque composante, et c'est sur les dernières que
  ce n'est pas trop gros.
\end{lem}

\begin{ideas}
  Avec la hauteur normalisée, c'est de la géométrie euclidienne. Il suffit
  ensuite de comparer à une hauteur projective.
\end{ideas}

\begin{lem} \label{l:par-img-spe}
  Le développement de $F$ en $\excep$ s'écrit
  $\Omega_\wt(G)_{\wemb(\excep)} = \sum_\lambda \zeta_\lambda t^\lambda$ avec
  \( \hautl{\zeta_\lambda} \le c(\hautl\excep, \lgr\lambda, \dots ) \).
\end{lem}

\begin{ideas}
  On spécialise les formes données par le lemme~\ref{l:par-img} en \(
    \wemb(\excep) \), en tenant compte de l'estimation des hauteurs précédentes,
  et en utilisant les lemmes du début de la partie pour gérer les
  dénominateurs, qui sont les mêmes que pour le lemme~\ref{l:par-var-spe}.
\end{ideas}

\begin{lem}
  La forme $R \cdot F'$ a un indice pondéré par \( \wt \) supérieur à \(
    \delta/\sigma \) en \( \excep \) le long des \( \divi \).
\end{lem}

\begin{ideas}
  On veut montrer que les coefficients de son développement appartenant à \(
    \Stairs_\wt^{\delta/\sigma} \) sont nuls. Or ce sont les mêmes que ceux du
  développement de \( G \) car ces deux formes sont égales sur \( \var \).
  \worknote{Pile les mêmes, où à constante près ?} On contrôle donc
  précisément leur valeur absolue aux places de \( \placesapx \).

  D'un autre côté, le développement de \( F' \) en \( \excep \) est le même
  que celui de \( F \) en \( \wemb(\excep) \).\worknote{Le plongement ne
    change rien pour les dérivées et tout ?} Donc on contrôle le hauteur des
  coefficients. Or, à \( o(\delta) \), ça sera aussi la hauteur des
  coefficients de \( R \cdot F' \), puis quel la hauteur de \( R \) est un \(
    O(1) \).

  Du coup, pour les coefficients de \( \Stairs_\wt^{\delta/\sigma} \), on
  contrôle finement la valeur absolue en les places de \( \placesapx \), et on
  contrôle aussi la hauteur. Si tout va bien, c'est assez précis pour
  contredire la formule du produit, donc ça prouve qu'ils sont nuls.
\end{ideas}

\begin{rem} \worknote{D'accord ou pas ?}
  On peut peut-être s'arrêter là, c'est-à-dire ne pas montrer que \( F' \)
  tout seul a un indice élevé. En effet, on a déjà \( R \cdot F' \) qui a
  cette propriété et qui les mêmes degrés et hauteurs que \( F' \) à \(
    o(\delta) \) près. Le fait que \( F' \) provient de \( F \) n'est sans
  doute pas intéressant pour la suite (c'est pour extrapoler qu'on en avait
  besoin).
\end{rem}

\endinput

% vim: spell spelllang=fr
