% !TEX root = main.tex

\section{Extrapolation} \label{sec:vojta-extrap}

Nous allons maintenant montrer que la forme auxiliaire construite
précédemment s'annule avec un indice élevé en \( \excep \). Plus précisément,
posons
\begin{equation}
  \wtsum( \dermp )
  =
  \frac {\lgr{\dermp[1]}} {\wts[1]} + \dots
  + \frac {\lgr{\dermp[\puiss*]}} {\wts[\puiss*]}
  + \frac {\lgr{\dermp[\puiss]}} {\puiss*}
\end{equation}
pour tout \( \dermp \in \N^{\lgr\vdim} \).  Si \( h \) est une forme
multihomogène et \( \point \) où elle est définie, on définit son indice en \(
  \point \), noté \( \inda[\point](h) \),  comme le minimum des \(
  \wtsum*(\dermp) \) pour \( \dermp \) tel que \( \der[\dermp] h(\point) \neq 0
\).

On reprend les notations introduites par la scholie~\ref{s:aux-co} et on pose
\( f_1 = F' / \vmp[][\vdim]^\Di \). Le but de cette section est alors de montrer
la proposition suivante.

\begin{prop}
  On a \( \inda**(f_2) \ge \epsiv\delta \), où \( \epsiv \) est un réel
  strictement positif tel que \dots
\end{prop}

Commençons par un lemme facile sur l'indice.

\begin{lem}
  Soient \( h_1 \), \( h_2 \) et \( \alpha \) des fonctions rationnelles
  telles que \( h_1 = \alpha h_2 \) et \( \point \) un point où elles sont
  toutes les trois définies.
  \begin{enumthm}
    \item Si \( \dermp \) est tel que \( \der[\gmp\nu] h_2(x) = 0 \) pour tout
      \( \gmp\nu < \dermp \) pour l'ordre produit sur \( \N^{\lgr\vdim} \),
      alors \( \der[\dermp] h_1(\point) = \alpha(\point) \, h_2(\point) \).
    \item Si \( \alpha(\point) \neq 0 \) on a \( \inda**(h_1) = \inda**(h_2)
      \).
  \end{enumthm}
\end{lem}

\begin{proof}
  Le premier point découle facilement de la formule de \bsc{Leibniz} :
  \begin{equation}
    \der[\dermp] h_1(\point)
    =
    \sum_{\gmp\nu \le \dermp}
    \der[\dermp - \gmp\nu] \alpha(\point) \,
    \der[\gmp\nu] h_2(\point)
    \pmm.
  \end{equation}
  Or, par hypothèse, tous les termes de cette somme sont nuls sauf peut-être
  celui où \( \gmp\nu = \dermp \).

  Si \( \alpha(\point) \neq 0 \) alors \( \alpha^{-1} \) est également définie
  en \( \point \) et les deux autres fonctions jouent donc un rôle symétrique.
  Ainsi, si un indice \( \dermp \) est minimal pour la condition \(
    \der[\dermp] h_1(\point) \neq 0 \), il l'est aussi pour la condition \(
    \der[\dermp] h_2(\point) \neq 0 \) grâce au point précédent, ce qui prouve
  que les deux fonctions ont le même indice en \( \point \).
\end{proof}


\clearpage

\begin{lem}
  Soit \( \point \) un point de \( \var \) satisfaisant~\ref{e:HA} et de
  grande hauteur ; on note \( \cmp \) ses coordonnées multihomogènes.  Pour
  chaque \( \fct \in \{ 1, \dots, \puiss \} \) et chaque place \( \place \) de
  \( \cdn \), il existe un indice \( \ind_{\place, \fct} \in \{ 1, \dots,
      \vdim* \} \) tel que
  \begin{equation}
    \av{\cmp*[\ind_{\place, \fct}]}
    \le
    \cst{ind-norm-dim-loc} \cdot \Onv{\cmp*}
    \pmm,
  \end{equation}
  avec
  \begin{equation}
    \newcst{ind-norm-dim-loc} = \dots
  \end{equation}
\end{lem}

\begin{ideas}
  C'est une variante plus précise du lemme~\ref{ChoixI}. Les points sont :
  \begin{enumthm}
    \item pouvoir choisir \( \ind \) plus petit que \( \vdim* \). Pour ça, on
      utilise que le point est sur \( \var \) pour écrire que les dernières
      coordonnées sont intégralement dépendantes sur les premières
      (fait~\ref{f:plong-adapt-dep} ou directement Rémond si besoin) avec des
      relations connues, donc majorer la norme en fonction du maximum des
      valeurs absolues des première coordonnées.
    \item pouvoir choisir \( \ind \) différent de \( 0 \). Là on utilise le
      fait que le point est proche de \( \divi \), donc par définition de la
      distance, sa coordonnée d'indice \( 0 \) est petite devant sa norme,
      donc ça ne peut pas être elle qu'on a choisi avant.
  \end{enumthm}
\end{ideas}

\endinput

% vim: spell spelllang=fr
