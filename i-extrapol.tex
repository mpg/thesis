% !TEX root = main.tex

\section{Extrapolation} \label{sec:vojta-extrap}

Le but de cette section est de montrer que la fonction auxiliaire que nous
venons de construire s'annulle avec un indice élevé en \( \ex \), pour une
définition de l'indice que nous préciserons.

Auparavant, nous aurons besoin de représenter les dérivées de certaines
fonctions rationnelles sur \( \var \) par des fractions rationnelles de
dénominateur explicite dont on contrôle le degré et la hauteur du numérateur.
Pour commencer, nous effectuons ce travail sur une variété projective
quelconque (plongée de façon adaptée) avant de passer à une variété produit.


\subsection{Estimation de dérivées} \label{sec:vojta-param}

Soit \( \anyvar \) une variété projective de dimension \( \anydim \), plongée
de façon adaptée dans un espace projectif \( \projd \), de degré \( \anydeg \)
dans ce plongement. Alors \( \cdn(\anyvar) \) est une extension finie de
\begin{equation}
  \cdn\Big(
    \frac{ \vp[0]           }{ \vp[\anydim] }, \dots,
    \frac{ \vp[\anydim-1]   }{ \vp[\anydim] }
  \Big)
\end{equation}
dont \( \frac{ \vp[\anydim+1] }{ \vp[\anydim] } \) est un élément primitif.
Sur ce dernier corps, on dispose des dérivations standard définies par
\(
  \diff_\ind \frac{ \vp[\indi] }{ \vp[\anydim] } = \delta_\ind^\indi
\)
qui forment une base de l'espace des dérivations, et s'étendent de façon
unique à \( \cdn(\anyvar) \) pour former une base de son espace de
dérivations. Les propriétés classiques des dérivations montrent alors que
l'application
\begin{equation}
  \begin{aligned}
    \pmor \colon \cdn(Z)
    & \to \cdn(Z)\series\psp
    \\
    f
    & \mapsto
    \sum_{\derp \in \N^\anydim} \der[\derp] \!f \, \psp^\derp
  \end{aligned}
  \qquad \text{où }
  \der[\derp]
  =
  \frac1{\derp!}
  \prod_{\ind = 0}^\anydim \diff_\ind^{\derp*}
  \pmm.
\end{equation}
est un morphisme de \( \cdn \)-algèbres.

Il est intéressant de pouvoir le représenter par un morphisme \( \pmor* \)
faisant commuter le diagramme
\begin{equation} \label{e:pmor} % FIXME: tag not displayed
  \xymatrix{
    \cdn[\vp]_{(\ideal\anyvar)}                 \ar[d]^\pi \ar@{.>}[r]^{\pmor*}
    & \cdn[\vp]_{(\ideal\anyvar)} \series\psp   \ar[d]^\pi
    \\ \cdn(\anyvar)                                           \ar[r]^{\pmor}
    & \cdn(\anyvar)\series\psp
  }
\end{equation}
où les flèches verticales sont les projections canoniques. Posons
\( \pmor* = \sum_\derp \psp^\derp \pdiff*^\derp \) où les \( \pdiff*^\derp \)
sont des applications linéaires sur \( \cdn[X]_{(\ideal\anyvar)} \) vérifiant
la règle de \bsc{Leibniz} pour les dérivées divisées. Pour chaque fraction
rationnelle \( f = F/G \), nous exhiberons (au moins) un polynôme \( R_\derp
\), dépendant éventuellement de la représentation \( F/G \) choisie, tel que
\( R_\derp \cdot \pdiff*^\derp(f) \) soit un polynôme de degré et normes
locales contrôlées.

En pratique, nous construirons en fait deux morphismes, \( \pmor*_0 \) et \(
  \pmor*_1 \), et deux familles d'applications \( \pdiff^\derp \) telles que
l'on contrôlera la norme \( \place \)-adique de \( R_\derp \cdot \pdiff^\derp
\), où l'on rappelle que \( \dv \) vaut \( 1 \) si \( \place \) est
archimédienne et \( 0 \) sinon.

Pour définir \( \pmor*_\dv \), il suffit de définir les images des différents
\( \vp* / \vp[\anydim] \). En effet, ceci donne un morphisme de \( \cdn[ \vp /
  \vp[\anydim] ] \) dans \( \cdn[X]_{(\ideal\anyvar)} \series\psp \). Si l'on
impose de plus que \( \pdiff^0 \) soit l'identité, on constate que l'image du
complémentaire de \( \ideal\anyvar \) ne contient que des séries inversibles
(car leur terme constant l'est), ce qui permet d'étendre le morphisme à \(
  \cdn[X]_{(\ideal\anyvar)} \).

Pour tout \( \ind \in \set{0, \dots, \anydim-1} \), on peut poser \(
  \pmor*_\dv(\vp*/\vp[\anydim]) = \vp*/\vp[\anydim] + \psp* \). Pour \( \ind >
  \anydim \), nous utiliserons le lemme suivant, que l'on énonce dans un cadre
affine.

\begin{lem} \label{l:param-aff}
  Soient \( \anyvp[1], \dots, \anyvp[\anydim], Y \) des variables et \( L \)
  une algébrique extension finie de \( \cdn(\anyvp[1], \dots, \anyvp[\anydim])
  \). On fixe \( y \) un élément de \( L \) ; on note \( \pi \) le
  morphisme de \( \cdn[\anyvp[1], \dots, \anyvp[\anydim], Y ] \) dans \( L \)
  qui laisse stable les \( \anyvp* \) et envoie \( Y \) sur \( y \). Soit \(
    \Pi \in \ker \pi \) tel que \( \pden = \diff_Y \Pi \notin \ker \pi \).

  On considère les dérivations standard \( \diff_\ind \) sur \( \cdn[
    \anyvp[1], \dots, \anyvp[\anydim] ] \) ainsi que leurs extensions à \( L
  \), et \( \der[\derp] = \frac1{\derp!} \prod_{\ind = 0}^\anydim
    \diff_\ind^{\derp*} \).  Il existe des polynômes \( P_\dv^\derp \), pour
  \( \derp \in \N^{\anydim} \minusset0 \), tels que :
  \begin{enumthm}
    \item \( \der[\derp] y
        = \pi\left(
          \frac{ P_\dv^\derp }{ \pden^{2\lgr\dermp - 1} }
        \right)
      \) ;
    \item \( \deg P_\dv^\derp \le (\deg \Pi - 1) (2\lgr\derp - 1) \) ;
    \item \( \nv1{ P_\dv^\derp }
        \le \nv1 \poldep ^{2\lgr\derp - 1}
        \cdot \left(
          (4\anydim)^{\lgr\derp -1} (\deg\Pi)^{3\lgr\derp -2}
        \right)^\dv \).
  \end{enumthm}
\end{lem}

\begin{proof}
  Il s'agit en fait de compléter\footnote{On contrôle en fait le développement
    autour d'un point générique, alors que \bsc{Rémond} l'étudie en un point
    fixé.} la preuve du lemme~6.1 de \cite{remivds}, en
  utilisant aux places archimédiennes une généralisation de
  \cite[relation~2.3.1, p.~63]{farhith}.

  On va construire $P_0^\derp$ et $P_1^\derp$ indépendamment par récurrence
  sur la longueur de $\derp$, en partant à chaque fois de $P_\dv^\derp = -
  \diff_{\ind_0} \Pi$ quand $\derp_{\ind_0} = 1$ et $\derp_\ind = 0$ sinon
  (cas $\lgr\derp = 1$), car ce choix convient. Pour la suite, on fixe un
  $\dv$, un $\derp$ de longueur au moins $2$, et on suppose qu'on a choisi un
  $P_\dv^{\derp'}$ convenable pour chaque $\derp'$ de longueur strictement
  inférieure à celle de $\derp$.

  On commence par le cas ultramétrique et on note donc provisoirement $P^\derp
  = P_0^\derp$ pour alléger. Les polynômes recherchés sont caractérisés par la
  relation
  \begin{equation}
    \Pi \left(
      \anyvp[1] + \psp[1], \dots, \anyvp[\anydim] + \psp[\anydim],
      Y + \sum_{ \derp \in \N^\anydim \minusset 0 }
      \frac {P^\derp} {\pden^{2\lgr\derp -1}} \psp^\derp
    \right)
    = 0 \mod (\Pi)
    \pmm.
  \end{equation}
  On remplace alors \( \Pi \) par son développement de \bsc{Taylor}, pour
  obtenir les égalités suivantes modulo \( \Pi \) :
  \begin{align}
    0
    & =
    \sum_{(\ip, \mu) \in \N^{\anydim+1}}
    \der[\ip, \mu] \Pi
    \cdot \psp^\ip
    \cdot \left(
      \sum_{ \derp \in \N^\anydim \minusset 0 }
      \frac {P^\derp} {\pden^{2\lgr\derp - 1}} \psp^\derp
    \right)^\mu
    \\
    & =
    \sum_{\substack{ (\ip, \mu) \in \N^{\anydim+1} \minusset{(0, 0)}
        \\ \gmp\nu* \in \N^\anydim \minusset 0 }}
    \left(
      \der[\ip, \mu] \Pi
      \cdot \prod_{\fct = 1}^\mu
      \frac {P^{\gmp\nu*}} {\pden^{2\lgr{\gmp\nu*} - 1}}
    \right)
    \psp^{\sum_\fct \gmp\nu* + \ip}
    \\
    & =
    \sum_{\derp \in \N^\anydim \minusset 0}
    \Biggl(
    \frac {P^\derp} {\pden^{2\lgr\derp - 2}}
    + \sum_{\substack{
        (\ip, \mu) \in \N^{\anydim+1} \minusset{(0, 0), (0, 1)}
        \\ \gmp\nu* \in \N^\anydim \minusset 0
        \\ \sum_\fct \gmp\nu* + \ip = \derp }}
    \der[\ip, \mu] \Pi
    \cdot \prod_{\fct = 1}^\mu
    \frac {P^{\gmp\nu*}} {\pden^{2\lgr{\gmp\nu*} - 1}}
    \Biggr)
    \psp^\derp
    \pmm,
  \end{align}
  où l'on a noté \( (\ip, \mu) = (\ip[1], \dots, \ip[\anydim], \mu) \).
  Il suffit donc de définir \( P^\derp \) par la relation de récurrence
  \begin{equation}
    - P^\derp
    =
    \sum_{\substack{
        (\ip, \mu) \in \N^{\anydim+1} \minusset{(0, 0), (0, 1)}
        \\ \gmp\nu* \in \N^\anydim \minusset 0
        \\ \sum_\fct \gmp\nu* + \ip = \derp }}
    \der[\ip, \mu] \Pi
    \cdot \pden^{2\lgr\derp - 2}
    \cdot \prod_{\fct = 1}^\mu
    \frac {P^{\gmp\nu*}} {\pden^{2\lgr{\gmp\nu*} - 1}}
  \end{equation}
  qui consiste à imposer que chaque terme de la série précédente soit nul, ce
  qui assure bien sa nullité modulo \( \Pi \).

  On majore alors le degré de \( P^\derp \) par récurrence :
  \begin{align}
    \deg P^\derp
    & \le
    \deg\Pi - \lgr\ip - \mu + (\deg\Pi - 1) (2\lgr\derp - 2)
    \\ & \le
    1 - \lgr\ip - \mu + (\deg\Pi - 1) (2\lgr\derp - 1)
    \\ & \le
    (\deg\Pi - 1) (2\lgr\derp - 1)
    \pmm,
  \end{align}
  car \( \ip \) et \( \mu \) ne sont pas simultanément nuls.  La majoration de
  norme locale est immédiate par analogie avec le degré vu les propriétés de
  la norme aux places ultramétriques.

  Traitons maintenant le cas archimédien (désormais \( P^\derp = P^\derp_1 \)
  pour alléger). On utilise la relation de récurrence suivante, établie dans
  la démonstration de \cite[lemme~6.1]{remivds}, avec \( Q_\derp =
    P^\derp \cdot \derp! \) et où, rappelons-le, \( \derp' \) est tel que
  \( \derp[\ind_0] = \derp[\ind_0]' + 1 \) et \( \derp* = \derp*' \) sinon :
  \begin{equation}
    Q_\derp
    =
    \pden^2 \, \diff_{\ind_0} Q_{\derp'}
    - \pden \, \diff_{\ind_0} P \, \diff_{\anydim+1} Q_{\derp'}
    + (2\lgr{\derp'} - 1)
    (\diff_{\ind_0} P \, \diff_Y \pden - \pden \diff_{\ind_0} \pden)
    Q_{\derp'}
    \pmm.
  \end{equation}
  On en déduit immédiatement l'estimation de degré suivante :
  \begin{equation}
    \deg P^\derp
    \le 2 (\deg\Pi - 1) + \deg P^{\derp'}
    \le (\deg\Pi - 1) (2\lgr\derp - 1)
    \pmm.
  \end{equation}
  Pour la norme, on prouve que \(
    \nv1{Q_\derp}
    \le
    \nv1\Pi^{2\lgr\derp-1} 4^{\lgr\derp-1} (\deg\Pi)^{3\lgr\derp-2}
    (\lgr\derp - 1) !
  \)
  (ce qui implique le résultat annoncé vu que \( \binom{\lgr\derp}{\derp}
    \le \anydim^{\lgr\derp-1} \)) en exploitant la majoration de degré
  sous la forme \( \deg P^\derp \le 2\lgr\derp \deg\Pi \).
  \begin{align}
    \nv1{ Q_\derp }
    & \le
    2(\deg\Pi)^2 \cdot \deg Q_{\derp'} \cdot \nv1\Pi^2 \nv1{ Q_{\derp'} }
    + 2 (2\lgr{\derp'} - 1) (\deg\Pi)^3 \nv1\Pi^2
    \\ & \le
    \nv1\Pi^2 \nv1{ Q_{\derp'} } \cdot 4 (\deg\Pi)^{3\lgr{\derp'}}
    \\ & \le
    \nv1\Pi^{2\lgr\derp-1} \cdot 4^{\lgr\derp-1} (\deg\Pi)^{3\lgr\derp-2}
    (\lgr\derp - 1) !
    \qedhere
  \end{align}
\end{proof}

Nous sommes maintenant en mesure de définir les morphismes \( \pmor*_\dv \).

\begin{lem} \label{l:def-pmor}
  Dans les notations précédentes, on peut définir \( \pmor*_\dv \) faisant
  commuter le diagramme~\eqref{e:pmor} par :
  \begin{enumthm}
    \item \( \pmor*(\frac{ \vp* }{ \vp[\anydim] })
        = \frac{ \vp* }{ \vp[\anydim] } + \psp* \)
      pour tout \( \ind \in \set{0, \dots, \anydim-1} \) ;
    \item \( \pmor*(\frac{ \vp* }{ \vp[\anydim] })
        = \frac{ \vp* }{ \vp[\anydim] }
        + \sum_{\derp \neq 0}
        \frac{ P^\derp_{\ind, \dv} }{ U_\ind^{\lgr\derp} }
        \psp^\derp
      \)
  \end{enumthm}
  où les polynômes \( P^\derp_{\ind, \dv} \) et \(  U_\ind \) satisfont
  \begin{enumthm}
    \item \( \deg  U_\ind   =   2 (\anydeg - 1) \) ;
    \item \( \nv1{ U_\ind } \le \nv1{ \chow\anyvar }^2 4^{(\anydeg-1)\dv} \) ;
    \item \( \deg  P^\derp_{\ind, \dv}   =   2 (\anydeg - 1) \lgr\derp \) ;
    \item \( \nv1{ P^\derp_{\ind, \dv} } \le \left(
          \nv1{ \chow\anyvar }^2
          \bigl( 4^\anydeg \anydim \anydeg^{3} \bigr)^\dv
        \right)^{\lgr\derp} \).
  \end{enumthm}
\end{lem}

\begin{proof}
  Fixons un \( \ind > \anydim \) et notons \( \tilde\Pi = P_\ind(\anyvp[1],
    \dots, \anyvp[\anydim], 1, Y ) \), où \( P_\ind \) est donné par le
  fait~\ref{f:plong-adapt-dep}, puis \( \Pi \) une des dérivées successives de
  \( \tilde\Pi \) par rapport à la dernière variable, telle que \( \diff_Y \Pi
    \notin \ideal\anyvar \). On pose alors
  \begin{equation}
    \pdiff^\derp \left( \frac{ \vp* }{ \vp[\anydim] } \right)
    =
    \frac {
      P_\dv^\derp ( \vp[0] / \vp[\anydim], \dots,
      \vp[\anydim-1] / \vp[\anydim], \vp* / \vp[\anydim] )
    }{
      \pden ( \vp[0] / \vp[\anydim], \dots,
      \vp[\anydim-1] / \vp[\anydim], \vp* / \vp[\anydim] )
      ^{2\lgr\derp - 1}
    }
    \pmm,
  \end{equation}
  où les \( P_\dv^\derp \) sont donnés par le lemme précédent. Quitte à
  multiplier par une puissance convenable de \( \vp[\anydim] \), on peut
  supposer que le degré de \( \pden \) est exactement \( 2\anydeg - 1 \) et
  que celui de \( P_\dv^\derp \) est exactement \( (2\anydeg - 1) (2\lgr\derp
    - 1) \), de sorte que par homogénéité :
  \begin{equation}
    \pdiff^\derp \left( \frac{ \vp* }{ \vp[\anydim] } \right)
    =
    \frac {
      P_\dv^\derp ( \vp[0], \dots, \vp[\anydim-1], \vp* )
    }{
      \pden ( \vp[0], \dots, \vp[\anydim-1], \vp* )
      ^{2\lgr\derp - 1}
    }
    =
    \frac {
      P_\dv^\derp
    }{
      \pden
      ^{2\lgr\derp - 1}
    }
  \end{equation}
  On pose finalement\footnote{Ceci n'a pour but que de simplifier les calculs
    ultérieurs en rendant le degré exactement linéaire en \( \lgr\derp \).} \(
    P_{\ind, \dv}^\derp = \pden \cdot P_\dv^\derp \) et \( \pden_\ind =
    \pden^2 \). Les estimations de norme annoncées sont alors immédiates.
\end{proof}

Les morphismes \( \pmor*_\dv \) s'étendent à \( \cdn[X]_{(\ideal\anyvar)} \)
comme expliqué au deuxième paragraphe précédant le lemme~\ref{l:param-aff}.
Nous allons maintenant étudier (dénominateur, degré et hauteur des
coefficients) les images de certaines fonctions, en commençant par celles des
monômes en \( \vp / \vp[\anydim] \).

\begin{lem} \label{l:par-anyvar-mono}
  Soit \( \ip \in \N^{\dimp*} \) et \( M_\ip = \prod\limits\indrange
    \bigl( \frac{\vp*}{\vp[\anydim]} \bigr)^{\ip*} \).  Posons \( \pden =
    \prod_{\ind = \anydim + 1}^{\dimp} \pden_\ind \) où les \( \pden_\ind \)
  sont donnés par le lemme précédent. Alors :
  \begin{enumthm}
    \item \( P^\derp_{\ip, \dv}
        = \pdiff^\derp(M_\ip)
        \cdot \vp[\anydim]^{\lgr\ip} \pden^{\lgr\derp} \) est un polynôme ;
    \item \( \deg P^\derp_{\ip, \dv}
        = \lgr\ip + 2 (\anydeg - 1) (\dimp - \anydim) \lgr\derp \) ;
    \item \( \nv1{ P^\derp_{\ip, \dv} }
        \le
        \nv1{ \chow\anyvar }^{2 (\dimp - \anydim) \lgr\derp}
        \left(
          \bigl( 4^{\anydeg+1} \anydim \anydeg^{3} \bigr) ^{
            (\dimp - \anydim) \lgr\derp }
          \cdot 2^{ \anydim \lgr\ip }
        \right)^\dv
      \).
  \end{enumthm}
  De plus, \( \ord_{\vp[0]} \bigl( P^\derp_{\ip, \dv} \bigr)
    \ge \ip[0]  - \derp[0] \).
\end{lem}

\begin{proof}
  \newcommand \indl {{ \gmp\nu[\ind, \indi_\ind] }}
  On utilise la règle de \bsc{Leibniz} :
  \begin{equation}
    \pdiff^\derp(M_\ip) =
    \sum_{\gmp\nu \in N}
    \prod\indrange
    \prod_{ \indi_\ind = 1 }^{ \ip* }
    \pdiff^\indl \biggl( \frac{ \vp* }{ \vp[\anydim] } \biggr)
    \pmm,
  \end{equation}
  où la somme est prise sur l'ensemble
  \begin{equation}
    N = \left\{
      \gmp\nu \in (\N^\anydim)^{\lgr\ip}
      \text{ tels que }
      \sum\indrange \sum_{\indi_\ind = 1}^{\ip*} \indl = \derp
    \right\}
    \pmm.
  \end{equation}

  On vérifie alors que
  \begin{equation}
    P^\derp_{\ip, \dv}
    =
    \sum_{\gmp\nu \in N}
    \prod\indrange
    \prod_{ \indi_\ind = 1 }^{ \ip* }
    \pdiff^\indl \biggl( \frac{ \vp* }{ \vp[\anydim] } \biggr)
    \pden_\ind^{\lgr\indl}
    \vp[\anydim]
  \end{equation}
  est bien un polynôme : pour chaque facteur de chaque terme, si \( \indl = 0
  \) alors \(
    \pdiff^\indl ( \frac{ \vp* }{ \vp[\anydim] } )
    \vp[\anydim]
  \) est un polynôme, sinon \(
    \pdiff^\indl ( \frac{ \vp* }{ \vp[\anydim] } )
    \pden_\ind^{\lgr\indl}
  \) en est un, d'après le lemme précédent.

  Le calcul du degré est direct et on ne détaille donc que l'estimation de
  norme : chaque terme de la somme est majoré en norme par
  \begin{align}
    \prod_{\ind = \anydim + 1}^{\dimp}
    \prod_{ \indi_\ind = 1 }^{ \ip* }
    \left(
      \nv1{ \chow\anyvar }^2
      \bigl( 4^\anydeg \anydim \anydeg^{3} \bigr)^\dv
    \right)^{\lgr\indl}
    \le
    \nv1{ \chow\anyvar }^{2 (\dimp - \anydim) \lgr\derp}
    \bigl( 4^\anydeg \anydim \anydeg^{3} \bigr)
    ^{(\dimp - \anydim) \lgr\derp \dv}
  \end{align}
  On remarque alors que l'ensemble de sommation \( N \) s'écrit aussi
  \begin{equation}
    \prod_{p = 0}^{\anydim - 1} \left\{
      \gmp\nu[][p] \in \N^{\lgr\ip}
      \text{ tels que }
      \sum\indrange \sum_{\indi_\ind = 1}^{\ip*}
      \gmp\nu[\ind, \indi_\ind][p]
      = \derp[p]
    \right\}
    \pmm.
  \end{equation}
  Chacun des facteurs de ce produit est de cardinal
  \(
    \binom{ \derp[p] + \lgr\ip - 1 }{ \lgr\ip - 1 }
    \le
    2^{ \derp[p] + \lgr\ip - 2 }
  \).
  L'estimation annoncée suit en prenant le produit en majorant assez largement
  certains facteurs.

  Enfin, on constate que chaque terme de la somme définissant
  \( P^\derp_{\ip, \dv} \) contient un facteur de la forme
  \(
    \prod_{\indi_0 = 1}^{\ip[0]}
    \pmnum_{1, \dv}^{\gmp\nu[0, \indi_0]}( \frac{ \vp[0] }{ \vp[\anydim] } )
    \vp[\anydim]
  \). On peut supposer que \( \gmp\nu[0, \indi_0] \in \N \times
    \set{0}^{\anydim-1} \) car sinon ce facteur, donc le terme correspondant,
  est nul. De plus, on a \( \sum_{\indi_0 = 1}^{\ip[0]} \gmp\nu[0, \indi_0][0]
    \le \derp[0] \). Ainsi, il y a au moins \( \ip[0] - \derp[0] \) termes
  nuls dans cette somme, donc \( \vp[0]^{\ip[0] - \derp[0]} \) est en
  facteur de chaque terme non nul de la somme définissant \( P^\derp_{\ip,
      \dv} \), ce qui prouve l'assertion sur l'ordre.
\end{proof}

On souhaite maintenant étudier le développement en série des fonctions
rationnelles sur notre variété produit \( \var \).  On étend les notations de
précédentes au cas multi-homogène de la façon évidente. En particulier, on a
un morphisme \( \pmor \) de développement en série, qu'on représente par des
morphismes \( \pmor*_\dv \) faisant commuter le diagramme
\begin{equation} \label{pmor*-prod}
  \xymatrix{
    \cdn[\vmp]_{(\varid)}                 \ar[d]^\pi  \ar@{.>}[r]^{\pmor*_\dv}
    & \cdn[\vmp]_{(\varid)} \series\psmp  \ar[d]^\pi
    \\ \cdn(\var)                                     \ar[r]^{\pmor}
    & \cdn(\var)\series\psmp
  }
\end{equation}
et qu'on écrira encore sous la forme \( \pmor*_\dv = \sum_\dermp \psmp^\dermp
  \cdot \pdiff^\dermp \).

Plus précisément, pour chaque \( \fct \in \set{1, \dots, \puiss} \), on a un
morphisme \( \pmor*\pexp\fct \) obtenu en appliquant le lemme~\ref{l:def-pmor}
avec \( V = \var* \) et \( \vp = \vmp* \), et on définit \( \pmor \) comme le
produit tensoriel de ces morphismes. On note par ailleurs \( \pden_\fct =
  \prod_{\ind=\vdim*+1}^{\dimp} \gmp\pden** \) où les \( \gmp\pden** \) sont
donnés par le lemme cité.

Nous allons maintenant contrôler l'image par \( \pmor* \) de fonctions
particulières, en commençant par celles qui admettent un monôme en les \(
  \vmp*[\vdim*] \) comme dénominateur.

\begin{lem} \label{l:par-var}
  Soit \( G \) une forme multi-homogène de multidegré \( \alpha = (\alpha_1,
    \dots, \alpha_\puiss ) \) et \( g = G / \vmp[][\vdim]^{\alpha} \) où
  l'on a noté \( \vmp[][\vdim]^{\alpha} = \prod\fctrange
    (\vmp*[\vdim*])^{\alpha_\fct} \). Alors :
  \begin{enumthm}
    \item \( P^\dermp_{G, \dv}
        = \pdiff^\dermp(g)
        \cdot \vmp[][\vdim]^{\alpha}
        \prod\fctrange \pden_\fct^{\lgr{\dermp*}}
      \) est un polynôme ;
    \item \( \deg_\fct P^\dermp_{G, \dv}
        = \alpha_\fct + 2 (\vdeg* - 1) (\dimp - \vdim*) \lgr{\dermp*} \) ;
    \item \( \nv1{ P^\dermp_{G, \dv} }
        \le
        \nv1{ G }
        \prod\fctrange
        \nv1{ \varfc* }^{2 (\dimp - \vdim*) \lgr{\dermp*}}
        \left(
          \bigl( 4^{\vdeg*+1} \vdim* \vdeg*^{3} \bigr) ^{
            (\dimp - \vdim*) \lgr{\dermp*} }
          \cdot 2^{ \vdim* \alpha_\fct }
        \right)^\dv
      \).
  \end{enumthm}
  De plus, \( \inda* \bigl( P^\derp_{G, \dv} \bigr) \ge \inda*(G) -
    \wtsum*(\dermp) \).
\end{lem}

\begin{proof}
  Les trois premiers points s'obtiennent en remarquant que \( g \) est une
  combinaison linéaire de monômes en \( \vmp / \vmp[][\vdim] \), que l'image
  d'un tel monôme par \( \pdiff^\dermp \) est donnée par
  \begin{equation}
    \pdiff^\dermp \Biggl(
      \prod\fctrange \prod\indrange
      \biggl( \frac{ \vmp** }{ \vmp*[\vdim*] } \biggr)^{\imp**}
    \Biggr)
    =
    \prod\fctrange
    \pdiff^{\fct, \dermp*} \Biggl(
      \prod\indrange
      \biggl( \frac{ \vmp** }{ \vmp*[\vdim*] } \biggr)^{\imp**}
    \Biggr)
  \end{equation}
  et en appliquant le lemme~\ref{l:par-anyvar-mono}.

  Seul le point sur l'indice reste à vérifier. Pour cela, considérons \(
    \vmp^\imp \) un monôme apparaissant dans l'écriture \( G \), puis \(
    \vmp^{\gmp\nu} \) un monôme apparaissant dans \( P^\dermp_{G, \dv} \).
  D'après le lemme cité, on alors \( \gmp\nu*[0] \ge \imp*[0] - \dermp*[0] \)
  d'où, en sommant, \( \wtsum*(\gmp\nu) \ge \wtsum*(\imp) - \wtsum*(\dermp) \)
  qui est équivalent à l'estimation annoncée vu la définition de l'indice.
\end{proof}

Notons que les estimations obtenues ne valent que pour des fonctions admettant
un monôme en \( \vmp[][\vdim] \) comme dénominateur.  On pourrait aisément en
déduire des estimations pour des fonctions rationelles de dénominateur
quelconque en les écrivant comme un quotient de deux telles fonctions, mais
ce n'est pas utile ici.


\subsection{Minoration de l'indice}

Nous allons maintenant montrer que la forme auxiliaire construite
précédemment s'annule avec un indice élevé en \( \ex \). Plus précisément,
posons
\begin{equation}
  \wtsum( \dermp )
  =
  \frac {\lgr{\dermp[1]}} {\wts[1]} + \dots
  + \frac {\lgr{\dermp[\puiss*]}} {\wts[\puiss*]}
  + \frac {\lgr{\dermp[\puiss]}} {\puiss*}
\end{equation}
pour tout \( \dermp \in \N^{\lgr\vdim} \).  Si \( h \) est une forme
multihomogène et \( \point \) où elle est définie, on définit son indice en \(
  \point \), noté \( \inda[\point](h) \),  comme le minimum des \(
  \wtsum*(\dermp) \) pour \( \dermp \) tel que \( \der[\dermp] h(\point) \neq 0
\).

On reprend les notations introduites par la scholie~\ref{s:aux-co} et on pose
\( f_1 = F' / \vmp[][\vdim]^\Di \). Le but de cette section est alors de montrer
la proposition suivante.

\begin{prop}
  On a \( \inda**(f_2) \ge \epsi \delta / \sigma \), où \( \sigma \) est tel
  que \dots
\end{prop}

Commençons par un lemme facile sur l'indice.

\begin{lem} \label{l:indice-inversible}
  Soient \( g_1 \), \( g_2 \) et \( \alpha \) des fonctions rationnelles
  telles que \( g_1 = \alpha g_2 \) et \( \point \) un point où elles sont
  toutes les trois définies.
  \begin{enumthm}
    \item Si \( \dermp \) est tel que \( \der[\gmp\nu] g_2(x) = 0 \) dès que
      \( \gmp\nu < \dermp \) pour l'ordre produit sur \( \N^{\lgr\vdim} \),
      alors \( \der[\dermp] g_1(\point) = \alpha(\point) \, g_2(\point) \).
    \item Si \( \alpha(\point) \neq 0 \) on a \( \inda**(g_1) = \inda**(g_2)
      \).
  \end{enumthm}
\end{lem}

\begin{proof}
  Le premier point découle facilement de la formule de \bsc{Leibniz} :
  \begin{equation}
    \der[\dermp] g_1(\point)
    =
    \sum_{\gmp\nu \le \dermp}
    \der[\dermp - \gmp\nu] \alpha(\point) \,
    \der[\gmp\nu] g_2(\point)
    \pmm.
  \end{equation}
  Or, par hypothèse, tous les termes de cette somme sont nuls sauf peut-être
  celui où \( \gmp\nu = \dermp \).

  Si \( \alpha(\point) \neq 0 \) alors \( \alpha^{-1} \) est également définie
  en \( \point \) et les deux autres fonctions jouent donc un rôle symétrique.
  Ainsi, si un indice \( \dermp \) est minimal pour la condition \(
    \der[\dermp] g_1(\point) \neq 0 \), il l'est aussi pour la condition \(
    \der[\dermp] g_2(\point) \neq 0 \) grâce au point précédent, ce qui prouve
  que les deux fonctions ont le même indice en \( \point \).
\end{proof}

On rappelle que \( \cex \) désigne un système de coordonnées multihomogènes du
point \( \ex \) supposé contredire le théorème~\ref{t:vojta-div}. On peut
supposer, d'après le scholie~\ref{s:part-cases}, qu'aucune des formes \(
  \pden_\fct \) s'annulle en \( \cex* \) et qu'aucun des \( \cex** \) n'est
nul ; ci-dessous on utilisera le fait que \( \cex*[\vdim*] \neq 0 \).  De
même, on peut supposer que la forme \( R \) donnée par le
corollaire~\ref{c:hmat-relim} ne s'annulle pas en \( \cex \).  \todo{Vérifier
  que c'est bien le cas.}

On choisit, pour tout \( \fct \), un indice \( \indv* \in \set{0, \dots,
    \vdim*} \) de sorte que \( \av{ \cex*[\indv*] } \) soit maximal sous cette
condition. De même, on choisit \( \indiv* \in \set{0, \dots, \dimp} \)
maximisant \( \av{ \wembcl*[\indiv*](\cex) } \), ainsi que \( \indig* \in
  \set{0, \dots, \dimp} \) tel que \( \wembcl*[\indig*](\cex) \neq 0 \). Notons
que \( \indig \) et \( \indiv \) ne dépendent pas du choix des formules \(
  \wembcl \) utilisées, et que \( \indig \) ne dépend pas de \( \place \).

Il est alors clair que \( \av{ \wembcl*[\indiv*](\cex) } \ge \nv1{
    \wembcl*(\cex) } \dimp**^{-\dv} \).  Le lemme suivant fournit un analogue
pour les \( \indv* \).

\begin{lem} \label{l:coord-norm}
  Avec les notations précédentes, on a
  \begin{equation}
    \av{ \cex*[\indv*] }
    \ge
    \nv1{ \cex* } \cdot \nv1{ \varfc* }^{-1} \dimp**^{-\dv}
    \pmm.
  \end{equation}
\end{lem}

\begin{proof}
  Il suffit de montrer qu'on a \( \av{ \cex** } \le \nv1{ \varfc* } \av{
      \cex*[\indv] } \) pour tout \( \ind \in \set{\vdim* + 1, \dots, \dimp}
  \). Notons \( \poldep** \) la relation dépendance intégrale donnée par la
  scholie~\ref{s:plong-adapt} ; en décomposant suivant les puissance de \(
    \vmp** \) on a
  \begin{equation}
    (\vmp**) ^{ \vdeg* }
    =
    \sum_{ \alpha=1 }^{ \vdeg* }
    (\vmp**) ^{ \vdeg* - \alpha }
    \poldep*[\ind, \alpha]
    \quad\text{où }
    \deg \poldep*[\ind, \alpha] = \alpha
    \text{ et }
    \sum_{ \alpha=1 }^{ \vdeg* } \nv1{ \poldep*[\ind, \alpha] }
    \le \nv1{ \varfc* }
  \end{equation}
  En passant aux valeurs absolues et en divisant, il vient
  \begin{equation}
    \av{ \cex** }
    \le
    \sum_{ \alpha=1 }^{ \vdeg* }
    \poldep*[\ind, \alpha]
    \left(
      \frac{ \av{\cex*[\indv]} }{ \av{\cex**} }
    \right) ^{ \vdeg* - \alpha }
  \end{equation}
  Si le quotient apparaissant dans le membre de droite est inférieur à \( 1
  \), notre assertion initiale est vérifiée. Sinon, elle l'est aussi, car car
  \( \varfc* \) est normalisé (scholie citée) de façon à ce qu'un de ses
  coefficients soit \( 1 \) ce qui assure \( \nv1{ \varfc* } \ge 1 \) en toute
  place.
\end{proof}

Introduisons maintenant une nouvelle fonction rationnelle définie par
\begin{equation}
  f_2 =
  \frac{ R(\vmp) }{ \vmp[][\vdim]^r }
  \cdot
  \frac{
    F( \vmp, \wembcl(\vmp) )
  }{
    \vmp[][\vdim] ^{\epsz \wt \delta}
    \cdot
    \wembcl[][\indig](\vmp) ^{ \delta }
  }
  \pmm,
\end{equation}
où au dénominateur on a noté
\begin{equation}
  \wembcl[][\indig](\vmp) ^{ \delta }
  =
  \prod\fctirange \wembcl*[\indig](\vmp) ^{ \delta }
  \pmm.
\end{equation}
D'après le lemme~\ref{l:indice-inversible}, \( f_1 \) et \( f_2 \) ont le même
indice en \( \cex \). Nous allons maintenant décomposer \( f_2 \) en facteurs
qui seront plus faciles à contrôler.

On commence par choisir pour chaque \( \place \) des formes \( \wembcl[\place,
    \fcti] \) en appliquant le lemme~\ref{l:hclab} avec \( (\pp, \ppi) =
  (\ex[\fcti], \ex[\puiss]) \) et \( (\alpha, \beta) = (\wti*, \wt**)
\). On écrit alors
\begin{align}
  f_2
  & =
  \frac{ R(\vmp) }{ \vmp[][\vdim]^r }
  \cdot
  \frac{
    F( \vmp, \wembcl[\place](\vmp) )
  }{
    \vmp[][\vdim] ^{\epsz \wt \delta}
    \cdot
    \wembcl[\place][\indig](\vmp) ^{ \delta }
  }
  \\ & =
  \frac{ R(\vmp) }{ \vmp[][\indv]^r }
  \cdot
  \frac{
    F( \vmp, \wembcl[\place](\vmp) )
  }{
    \vmp[][\indv] ^{\epsz \wt \delta}
    \cdot
    \wembcl[\place][\indiv](\vmp) ^{ \delta }
  }
  \left(
    \frac{ \wembcl[\place][\indiv](\vmp) }{ \wembcl[\place][\indig](\vmp) }
  \right) ^{ \delta }
  \left(
    \frac{ \vmp[][\indv] }{ \vmp[][\vdim] }
  \right) ^{ \epsz \wt \delta + r }
  \\ & =
  \frac{
    R(\vmp) \cdot F( \vmp, \wembcl[\place](\vmp) )
  }{
    \vmp[][\indv] ^{ \Diii }
  }
  \left(
    \prod\fctirange
    \frac{
      ( \vmp[\fcti ][{\indv[\fcti ]}] )^{ 2\wts[\fcti ] }
      ( \vmp[\puiss][{\indv[\puiss]}] )^{ 2\wts[\puiss] }
    }{
      \wembcl[\place, \fcti][\indiv*]( \vmp )
    }
  \right)^\delta
  \left(
    \frac{ \wembcl[][\indiv](\vmp) }{ \wembcl[][\indig](\vmp) }
  \right) ^{ \delta }
  \left(
    \frac{ \vmp[][\indv] }{ \vmp[][\vdim] }
  \right) ^{ \epsz \wt \delta + r }
\end{align}

Soit maintenant \( \dermp \in \N^{\lgr\vdim} \) un indice minimal tel que \(
  \der[\dermp] f_2 \neq 0 \). Comme les derniers facteurs de l'écriture
précédente sont tous inversibles en \( \cex \), le
lemme~\ref{l:indice-inversible} montre que \( \der[\dermp] f_2 \) est égal à

\begin{equation}
  \der[\dermp]
    \frac{
      R(\vmp) \cdot F( \vmp, \wembcl[\place](\vmp) )
    }{
      \vmp[][\indv] ^{ \Diii }
    }
  ( \cex )
  \cdot
  \left(
    \prod\fctirange
    \frac{
      ( \cex[\fcti ][{\indv[\fcti ]}] )^{ 2\wts[\fcti ] }
      ( \cex[\puiss][{\indv[\puiss]}] )^{ 2\wts[\puiss] }
    }{
      \wembcl[\place, \fcti][\indiv*]( \cex )
    }
  \right)^\delta
  \left(
    \frac{ \wembcl[][\indiv](\cex) }{ \wembcl[][\indig](\cex) }
  \right) ^{ \delta }
  \left(
    \frac{ \cex[][\indv] }{ \cex[][\vdim] }
  \right) ^{ \epsz \wt \delta + r }
\end{equation}
Nous allons maintenant estimer la valeur absolue de chacun des facteurs, en
distinguant selon que \( \place \in \placesapx \) ou pas pour le premier
facteur, et montrer que l'on contredit la formule du produit si \( \wtsum(
  \dermp ) < \epsi \delta / \sigma \). Rappelons qu'on note \( \degv \) le
degré local de \( \cdn \) en \( \place \).

Pour le facteur le plus à droite, on a facilement
\begin{equation}
  \prod_\place
  \prod\fctrange
  \left(
    \frac{ \av{\cex[][\indv]} }{ \av{\cex[][\vdim]} }
  \right) ^{ (\epsz \delta \wt* + r_\fct) \degv }
  \le
  \prod\fctrange
  \hautm[\infty]{ \ex* }^{\epsz \delta \wt* + r_\fct}
  \le
  \hautm[1]{ \ex[1] }^{2\puiss \epsz \delta \wt[1]} \cdot o(\expb^\delta)
\end{equation}
De plus, en notant \( \exi \) l'image de \( \ex \) par la deuxième
partie du plongement éclatant, on a\worknote{\bsc{Farhi} p. 96 en haut}
\begin{equation}
  \prod_\place
  \prod\fctirange
  \left(
    \frac{ \av{\wembcl[][\indiv](\cex)} }{ \av{\wembcl[][\indig](\cex)} }
  \right) ^{ \delta \degv }
  \le
  \prod\fctirange
  \hautm[1]{ \exi* }^\delta
  \le
  \hautm[1]{ \ex[1] }^{2\puiss \epsiv \wts[1] \delta}
  \cdot \cst{exci}^{\wts[1] \delta}
\end{equation}
Enfin, grâce au choix des formules \( \wembcl[\place] \), on a
\begin{equation}
  \prod_\place
  \left(
    \prod\fctirange
    \frac{
      \nv1{ \wembcl[\place, \fcti][\indiv*] }
      ( \cex[\fcti ][{\indv[\fcti ]}] )^{ 2\wts[\fcti ] }
      ( \cex[\puiss][{\indv[\puiss]}] )^{ 2\wts[\puiss] }
    }{
      \wembcl[\place, \fcti][\indiv*]( \cex )
    }
  \right)^{ \delta \degv }
  \le
  \hclab^{2\wts[1] \delta}
\end{equation}

Passons maintenant au premier facteur. Remarquons pour commencer que
\begin{equation}
  \nv1{ R(\vmp) \cdot F( \vmp, \wembcl[\place](\vmp) ) }
  \le
  \nv1{ F } \nv1{ \wembcl }^\delta \cdot o(\expb^\delta)
\end{equation}
et que le facteur en \( \nv1{ \wembcl } \) a déjà été intégré à l'estimation
précédente. Appliquons maintenant le lemme~\ref{l:par-var} à cette forme :
\worknote{À adapter pour remplacer \( \vdim \) par une autre coordonnée ?}
\begin{equation}
  \der[\dermp]
    \frac{
      R(\vmp) \cdot F( \vmp, \wembcl[\place](\vmp) )
    }{
      \vmp[][\indv] ^{ \Diii }
    }
  ( \cex )
  =
  \frac{ P^\dermp_\dv(\cex) }{ \cex[][\indv]^\Diii \pden(\cex)^{\vlg\dermp} }
  =
  \frac{ P^\dermp_\dv(\cex) }{ \cex[][\indv]^{\deg P^\dermp_\dv } }
  \cdot
  \frac{
    \cex[][\indv]^{\vlg\dermp \cdot \deg \pden}
  }{
    \pden(\cex)^{\vlg\dermp}
  }
  \pmm.
\end{equation}
On estime facilement le second facteur :
\begin{equation}
  \prod_\place
  \prod\fctrange
  \frac{
    \av{ \cex*[\indv] }^{ 2 (\vdeg*-1) (\dimp-\vdim*) \lgr{\dermp*} \degv }
  }{
    \av{ \pden_\fct(\cex*) }^{ \lgr{\dermp*} \degv }
  }
  =
  \prod\fctrange
  \hautm[\infty]{ \cex* }^{ 2 (\vdeg*-1) (\dimp-\vdim*) \lgr{\dermp*} }
  \le
  \hautm[1]{ \cex[1] }^{ \newcst[]{degdim} \wts[1] \epsi\delta/\sigma }
\end{equation}
Le premier facteur est majoré en valeur absolue par \( \nv1{ P^\dermp_\dv }
\), qui est à peu près \( \nv1{ F } \cdot \nv1{ \varfc }^{\wts[1] \epsi\delta
    / \sigma } \). Cependant, aux places de \( \placesapx \), on peut obtenir
une estimation plus fine.

En effet, chaque monôme divisé intervenant dans le déshomogénéisé de \(
  P^\dermp_\dv \) est composé d'un facteur des valeurs absolue inférieure à \(
  1 \) et d'un facteur de la forme
\begin{equation}
  \prod\fctrange
  \left(
    \frac{ \av{ \cex*[0] } }{ \av{ \cex*[\indv*] } }
  \right)^{\imp*[0]}
  \pmm,
\end{equation}
où \( \imp \) est tel que \( \wtsum*(\imp) \ge \epsi (1 - \frac1\sigma)
  \delta \). Or, par le lemme~\ref{l:coord-norm} et l'hypothèse principale, on
a
\begin{equation}
  \prod\placerange
  \left(
    \frac{ \av{ \cex*[0] } }{ \av{ \cex*[\indv*] } }
  \right)^{\degv}
  \le
  \left(
    \frac{ \av{ \cex*[0] } }{ \nv1{ \cex* } }
    \nv1{ \varfc* } \dimp**^\dv
  \right)^{\degv}
  \le
  \hautm[2]{ \ex* }^{-\eps}
  \hautm[1]{ \varfc* } \dimp**^\dv
\end{equation}
Ainsi, \todo[en négligeant pour l'instant les partie archimédiennes des
constantes], on a
\begin{align}
  \prod\placerange
  \prod\fctrange
  \left(
    \frac{ \av{ \cex*[0] } }{ \av{ \cex*[\indv*] } }
  \right)^{\imp*[0] \degv}
  & \le
  \prod\fctrange
  \left(
    \hautm[2]{ \ex* }^{-\eps}
    \hautm[1]{ \varfc* } \dimp**^\dv
  \right)^{ \imp*[0] }
  \\ & \le
  \hautm[1]{ \ex[1] }^{ -\eps \epsi(1 - \frac1\sigma) \delta \wts[1] }
  \hautm[1]{ \varfc }^{2 \wts[1] \delta }
\end{align}

En appliquant la formule du produit d'un côté et en mettant bout à bout les
estimations ci-dessus de l'autre, il vient, après avoir pris les logarithmes :
\begin{align}
  0
  & \le
  \delta \Bigl(
    2\puiss \epsz \wts[1] \hautl{ \ex[1] }
    +
    2\puiss \epsiv \wts[1] \hautl{ \ex[1] } + \wts[1] \cst{exci}
    +
    2\wts[1] \log\hclab
    +
    \cst{degdim} \wts[1] \epsi \sigma^{-1} \hautl{ \ex[1] }
    \\ & \qquad +
    \wts[1] \epsi \sigma^{-1} (\hautl{ F } + \hautl{ \varfc })
    +
    2 \wts[1] \hautl{ \varfc }
    -
    \eps \epsi(1 - \sigma^{-1}) \wts[1] \hautl{ \ex[1] }
  \Bigr)
  + o(\delta)
  \\ & \le
  \delta \Bigl(
    \wts[1] \hautl{ \ex[1] }
    \bigl(
      2\puiss \epsz + 2\puiss \epsiv + \cst{degdim} \epsi \sigma^{-1}
      -
      \eps \epsi(1 - \sigma^{-1})
    \bigr)
    + o( \wts[1] \hautl{ \ex[1] } )
  \Bigr)
  + o(\delta)
\end{align}
ce qui devient absurde dès que \( \delta \) et \( \wts[1] \hautl{ \ex[1] }
\) sont assez grands, pour peu que
\begin{equation}
  \frac{
    2\puiss \epsz + 2\puiss \epsiv + \cst{degdim} \epsi \sigma^{-1}
  }{
    \epsi(1 - \sigma^{-1})
  }
  \le
  \eps
\end{equation}

\subsection{Appendice provisoire}

\begin{fact} \todo
  On a
  \begin{equation}
  \prod\fctirange
  \hautm[1]{ \exi* }^\delta
  \le
  \hautm[1]{ \ex[1] }^{2\puiss \epsiv \delta} \cdot \cst{exci}^{\delta}
  \end{equation}
  avec \( \newcst[]{exci} = \dots \)
\end{fact}

\endinput

% vim: spell spelllang=fr
