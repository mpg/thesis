% !TEX root = main.tex

\section{Construction d'une forme auxiliaire} \label{sec:siegel}

L'objectif de cette section est de construire une forme non nulle sur \( \var
\), provenant d'une forme sur \( \wemb(\var) \), de degré prescrit, de
hauteur contrôlée, et d'indice élevé le long de \( \divi \) dans \( \var \).

Nous commençons par définir la notion d'indice utilisée, puis précisons la
stratégie de construction de la forme recherchée. Nous établissons ensuite les
estimations de dimension et de hauteur nécessaires avant de conclure en
appliquant un lemme de \bsc{Siegel}.


\subsection{Idéal d'annulation}

Pour tout \( \imp \in \N^{\puiss\dimp**} \), notons
\begin{equation}
  \wtsum*( \imp )
  =
  \frac {\imp[1][0]} {\wts[1]} + \dots
  + \frac {\imp[\puiss*][0]} {\wts[\puiss*]}
  + \frac {\imp[\puiss][0]} {\puiss*}
\end{equation}
On définit ainsi l'indice
\begin{equation}
  \inda*(H)
  =
  \max \set{
    \alpha \in \R
    \text{ tels que }
    \wtsum*(\imp) \le \alpha \implies h_\imp = 0
  }
\end{equation}
le long de \( \divi \) d'une forme \( H = \sum h_\imp \vmp^\imp \).

On introduit alors, pour \( \beta > 0 \), les idéaux
\begin{equation}
  \vanish*[\beta]
  = \left\{
    H \in \cdn [\cmmh]
    \text{ tels que }
    \inda*(H) \ge \beta
  \right\}
\end{equation}
et $\vanish[\beta] = \vanish*[\beta] + \varid$. Par abus, on notera de même
l'image de \( \vanish[\beta] \) dans $\ring\var = \cdn [\cmmh] / \varid$, ce
qui est légitime puisqu'on a fait en sorte que $\varid \subset
\vanish[\beta]$. On dira alors d'une forme qu'elle s'annule le long de \(
  \divi \) dans \( \var \) avec un indice au moins \( \beta \) si elle
appartient à \( \vanish[\beta] \).  Par ailleurs, on notera également
\begin{equation}
  \stairs[\beta] = \set{
    \imp \in \N^{\puiss\dimp**}
    \text{ tels que }
    \wtsum*(\imp) \le \beta
  }
\end{equation}
l'ensemble des indices correspondant aux coefficients qui doivent être nuls
pour assurer l'appartenance à \( \vanish*[\beta] \).


\subsection{Stratégie de construction de la forme auxiliaire}

Il s'agit de trouver une forme multihomogène non nulle sur $\var$, de degrés
prescrits et de hauteur contrôlée, provenant d'une forme sur $\wemb(\var)$ et
ayant indice élevé le long de \( \divi \) dans \( \var \).

Plus précisément, introduisons un rationnel $\epsz > 0$, un entier $\delta$
tel que $\epsz\delta \in \N$, et un réel $\epsi > 0$. Pour fixer les idées,
dans l'énoncé de la proposition~\ref{p:build-aux}, on demandera à $\delta$
d'être assez grand, et à $\epsi$ d'être majoré en fonction de \( \epsz \) et
d'autres paramètres, voir~\ref{e:def-epsii}. Posons alors
\begin{align}
  d & = \bigl(
  \epsz \wts[1],
  \dots,
  \epsz \wts[\puiss],
  1, \dots, 1
  \bigr) \in \Q^{2\puiss-1}
  \label{e:def-d}
  \\
  d' & = \bigl(
  \wts[1] (\frac94 + \epsz),
  \dots,
  \wts[\puiss*] (\frac94 + \epsz),
  2\puiss - 2 + \epsz
  \bigr) \in \Q^\puiss
  \pmm.
  \label{e:def-dp}
\end{align}
On a déjà supposé $\delta\epsz$ entier ; on supposera désormais de plus
$\delta/4$ entier, de sorte que si $F$ est une forme de multidegré $\delta d$
dans $\cdn[\cmmh, \cmmhi]$, son image par $\wemba$ est de multidegré
exactement $\delta d'$.

On cherche alors une forme
\begin{equation}
  F'
  \in \cdn[\cmmh]_{\delta d'}
  \cap \vanish
  \cap \wemba(\cdn[\cmmh, \cmmhi]_{\delta d})
\end{equation}
non nulle modulo l'idéal de $\wemb(\var)$, ou de façon équivalente (avec $F' =
\wemba(F)$), on cherche une forme
\begin{equation}
  F
  \in \cdn[\cmmh, \cmmhi]_{\delta d}
  \cap \wemba^{-1}(\vanish)
\end{equation}
non nulle modulo l'idéal de $\var$, de hauteur relativement petite.

Ce problème est un cas classique d'application du lemme de \bsc{Siegel}.
Cependant, il est difficile d'expliciter un système essentiellement minimal
d'équations linéaires exprimant l'appartenance à $\vanish$. Il est en
revanche immédiat d'expliciter un tel système pour $\vanish*$, mais
il comporte beaucoup trop d'équations (géométriquement, il correspond à une
condition d'annulation beaucoup trop forte : avoir un indice élevé le long de
$\divi$ dans $\projd$, alors qu'on travaille avec des formes sur $\var$).

Nous allons donc devoir procéder à quelques réductions pour pouvoir écrire le
système d'équations. L'idée fondamentale est la suivante : ramener $F'$ dans
un sous-espace de $\cdn[\cmh]_{\delta d'}$ dans lequel $\vanish$
est de codimension suffisamment petite pour pouvoir demander que l'image de
$F$ par ces réductions soit dedans, sans pour autant avoir trop d'équations.

Pour commencer, on choisit dans $\cdn[\cmmh, \cmmhi]_{\delta d}$ un
supplémentaire de $(\ideal{\wemb(\var)})_{\delta d}$ engendré par des
monômes, que l'on notera $\Ideal_{\delta d}$, et dans lequel on cherchera
$F$.

On considérera des sous-espaces vectoriels de partie homogènes de
$\cdn[\cmmh]$ définis par des restrictions de degré ; introduisons à cet effet
quelques notations. Pour chaque $C \in (\N \cup \{ +\infty
  \})^{\puiss(\dimp+1)}$, on notera
\begin{equation} \label{e:C-spaces}
  \cdn[\cmmh]^C
  = \{
    F \in \cdn[\cmmh]
    \text{ tels que }
    \deg_{\cmh{\fct}[k]} F \le C\pexp{\fct}[k]
    \quad \forall i, k
    \}
  \pmm.
\end{equation}
On utilisera trois tels vecteurs $C'$, $C''$, $C'''$, définis respectivement
par
\begin{gather} \label{e:C-i-iii}
  (C')\pexp\fct[k] =
  \begin{cases}
    +\infty & \text{si $k \le \vdim*$} \\
    \vdeg* - 1 & \text{sinon}
  \end{cases}
  \qquad
  (C'')\pexp\fct[k] =
  \begin{cases}
    +\infty & \text{si $k \le \vdim* + 1$} \\
    0 & \text{sinon}
  \end{cases}
  \\
  C''' = \min(C', C'')
  \pmm,
\end{gather}
pour $0 \le \fct \le \puiss$ et $0 \le k \le \dimp$, et avec l'ordre produit
évident pour la dernière définition. Par ailleurs, on notera $C'_\Delta$ et
$C'''_\Delta$ les vecteurs obtenus en remplaçant $D$ par un certain $\Delta$
dans la définition précédente.

Le plan général de l'écriture du système se résume alors par le diagramme
commutatif suivant, où l'on note
\( \ring{\wemb(\var)} = \cdn[\cmmh, \cmmhi] / \ideal{\wemb(\var)} \)
et
\( \ring{\opdef} = (\cdn[\cmmh])_\opdef \)
et où les flèches verticales sont les morphismes de réduction.

\begin{equation} \label{e:diag-f-aux}
  \xymatrix{
    \Ideal_\Dz            \ar [r] ^{\wemba}        \ar [d] _{\sim}
    & \cdn [\cmmh]_\Dii   \ar [r] ^{\rdiv^\Dii}    \ar [d]
    & \cdn [\cmmh]\Ci     \ar [r] ^{\relim}        \ar [d]
    & \cdn [\cmmh]\Cii    \ar [r] ^{\rdiv^\Diii}   \ar [d]
    & \cdn [\cmmh]\Ciii                            \ar [d]
    \\ \ring{\wemb(\var)} \ar [r] ^{\wemb^*}
    & \ring\opdef         \ar [r] ^{\I}
    & \ring\opdef         \ar [r] ^{R \cdot \null}
    & \ring\opdef         \ar [r] ^{\I}
    & \ring\opdef
  }
\end{equation}

Les morphismes $\rdiv$ et $\relim$ seront précisés ultérieurement ; le premier
représente une réduction par division euclidienne par certaines relations de
dépendance intégrale, les second permet d'éliminer les variables en utilisant
des relations linéaires dans $\korper\var$ et correspond modulo $\varid$ à
une multiplication par une certaine forme $R$ de degré $r$ ne dépendant que de
$\var$, de façon à éliminer les dénominateurs apparaissant dans ces relations
linéaires. Notons qu'il est essentiel d'effectuer $\rdiv$ avant $\relim$ pour
que le degré en les variables à éliminer ne dépende pas de $\delta$, ce qui
assure que $R$ ne dépend en effet que de $\var$.

Compte tenu de la présence de ce dénominateur $R$, révisons une dernière fois
notre objectif : il s'agira en fait de construire une forme $F \in
\Ideal_{\delta d} \setminus \{0\}$ telle que
\begin{equation}
  \rdiv^{\delta d' + r}(\relim(\rdiv^{\delta d'}(\wemba(F))))
  \in \vanish*
  \pmm,
\end{equation}
ce qui implique que $R \cdot \wemba(F) \in \vanish$ car
\begin{equation}
  R \cdot \wemba(F)
  - \rdiv^{\delta d' + r}(\relim(\rdiv^{\delta d'}(\wemba(F))))
  \in \varid
  \subset \vanish
  \pmm.
\end{equation}

Au final, le système d'équations auquel on appliquera le lemme de \bsc{Siegel}
sera composé des équations définissant $\vanish$ dans
\( \cdn[\cmmh]_{\delta d' + r'}^{C'''} \),
tirées en arrière sur $\Ideal_{\delta d}$ par
\( \rdiv^{\delta d'} \circ \relim \circ \rdiv^{\delta d' + r} \circ \wemba \).
Pour pouvoir procéder, nous devons donc :
\begin{enumerate}
  \item calculer la codimension de $\Ideal_{\delta d}$ ;
  \item calculer la codimension de $\vanish*$ dans
    $\cdn[\cmmh]_{\delta d' + r}^{C'''}$ et s'assurer qu'elle est plus
    petite que la dimension précédente ;
  \item expliciter les morphismes de la première ligne du diagramme et majorer
    la hauteur de leurs matrices dans les bases monomiales évidentes.
\end{enumerate}

Dans tous ces calculs, $\delta$ sera moralement arbitrairement grand devant
les autres termes, on n'explicitera donc à chaque fois que le terme de plus
haut degré en $\delta$ ; ainsi lorsqu'on utilisera la notation $o(\truc)$ ou
$\sim$, il s'agira d'équivalents quand $\delta$ tend vers l'infini.


\subsection{Deux calculs de dimension} \label{sec:comp-dim}

La dimension de $\Ideal_{\delta d}$ est donnée par le théorème de
\bsc{Hilbert} multihomogène dès qu'on connaît les différents multidegrés de
$\wemb(\var)$. Il est \lat{a priori} difficile de tous les calculer, mais il
comme suffit en fait de minorer la dimension, le lemme suivant nous donne tout
ce qu'on aura besoin de savoir sur le degré.

\begin{lem}
  Avec les notations précédentes, on a
  \begin{equation}
    \deg_{(0, \dots, 0, \vdim[\puiss]; \vdim[1], \dots, \vdim[\puiss-1])}
    \bigl( \wemb(\var) \bigr)
    =
    \prod\fctrange
    \vdeg* \wt ^{2\vdim*}
    \pmm.
  \end{equation}
\end{lem}

\begin{proof}
  Ce degré est donné par le cardinal de l'intersection de $\wemb(\var)$ avec
  des hyperplans génériques choisis de la façon suivante : $\vdim[\puiss]$
  provenant du $\puiss$-ième facteur $\projd$, et $\vdim*$ hyperplans
  provenant du facteur $\puiss + \fct$ pour $\fct \in \{1, \dots, \puiss-1\}$.

  On commence par choisir les hyperplans sur le facteur $\puiss$ : on remarque
  qu'ils se remontent par $\wemb$ en des hyperplans sur le dernier facteur de
  l'espace de départ $(\projd)^\puiss$. Ainsi, couper $\wemb(\var)$ par ces
  hyperplans revient à imposer à $\point_\puiss$ de parcourir un ensemble de
  cardinal $\vdeg[\puiss]$.

  Fixons maintenant un point $p$ dans cet ensemble. On constate que
  $\wemb(\var) \cap \zeros{\cmh\puiss = p}$ coïncide avec l'image de
  \begin{align}
    \wemb[\wt, p]'
    \colon
    \var[1] \times \dots \times \var[\puiss]
    & \to
    \va^{2\puiss-1}
    \\
    (\point_1, \dots, \point_{\puiss-1})
    & \mapsto
    (\point_1, \dots, \point_{\puiss-1}, p;
    \wt[1] \point_1 - p,
    \dots,
    \wt[\puiss-1] \point_{\puiss-1} - p)
  \end{align}
  qui est le produit d'un point par des variétés de la forme
  $\wemb[\wt*, p]''(\var*)$ pour $\fct$ variant de $1$ à $\puiss-1$ en
  notant
  \begin{align}
    \wemb[\wt*, p]''
    \colon
    \var*
    & \to
    \va^2
    \\
    \point
    & \mapsto
    (\point, \wt* \point - p)
  \end{align}
  Il suffit donc de calculer le degré de ces variétés. La translation par $p$
  n'ayant pas d'influence sur le degré, il suffit de regarder l'action de la
  multiplication par $\wt*$. Or, celle-ci pouvant être représentée
  globalement par des formes de degré $\wts*$, en tirant en arrière par
  $\wemb[\wt*, p]''(\var*)$ une famille de $\vdim*$ hyperplans
  génériques sur le second facteur, on obtient des hypersurfaces génériques de
  degré $\wts*$ qui coupent donc $\var*$ en $\vdeg*
  \wt*^{2\vdim*}$ points.

  Le résultat suit en prenant le produit et en se rappelant que
  $\wt[\puiss] = 1$.
\end{proof}

\begin{lem}
  Avec les notations précédentes, on a
  \begin{align}
    \dim \Ideal_{\delta d}
    \ge
    \frac{ \epsz^{\vdim[\puiss]}
      \prod\fctrange \vdeg* \, \wt* ^{2\vdim*}
      }{ \vdim ! }
    \delta^{\vlg\vdim}
    + o( \delta^{\vlg\vdim} )
    \pmm.
  \end{align}
\end{lem}

\begin{proof}
  Il suffit d'appliquer le théorème de \bsc{Hilbert} multihomogène en
  utilisant le lemme précédent, car une somme de nombre positifs est minorée
  par chacun de ses termes. Il vient
  \begin{align}
    \dim \Ideal_{\delta d}
    & =
    \Biggl(
    \sum_{\substack{ t \in \N^{2\puiss-1} \\ \vlg t = \vlg u }}
    \deg_t \wemb(\var) \frac{ d^t }{ t! }
    \Biggr)
    \delta^{\vlg\vdim}
    + o( \delta^{\vlg\vdim} )
    \\
    & \ge
    \deg_{(0, \dots, 0, \vdim[\puiss]; \vdim[1], \dots, \vdim[\puiss-1])}
    \bigl( \wemb(\var) \bigr)
    \cdot
    \frac { \epsz^{\vdim[\puiss]} }{ \vdim ! }
    \delta^{\vlg\vdim}
    + o( \delta^{\vlg\vdim} )
  \end{align}
  par définition de \( d \), voir~\eqref{e:def-d}.
\end{proof}

Nous allons maintenant calculer la codimension de $\vanish*$ dans
$\cdn[\cmmh]\Ciii$. On introduit à cet effet le sous-ensemble de $\stairs$
suivant :

\begin{equation} \label{e:stairs-c3}
  \begin{split}
    (\stairs)\Ciii
    & =
    \Biggl\{
      ( \imp[1], \dots \imp[\puiss] )
      \in
      \prod\fctrange \bigl(
        \N^{\vdim* + 1}
        \times \{ 0, \dots, \vdeg* - 1 \}
        \times \{ 0 \}^{\dimp - \vdim* - 1}
      \bigr)
      \\ & \quad
      \text{tel que }
      \wtsum*(\lambda) \le \delta \epsi
      \text{ et }
      \lgr{\imp*}
      = \delta d' + r \quad \forall \fct
    \Biggr\}
    \pmm,
  \end{split}
\end{equation}
dont il s'agit de calculer le cardinal. En effet,
\( \vanish* \cap \cdn [\cmmh]\Ciii \)
est le sous-espace engendré par les monômes dont les exposants ne sont pas
dans \( (\stairs)\Ciii \).

\begin{lem}
  Avec les notations précédentes,
  \begin{equation}
    \card (\stairs)\Ciii
    \le
    \frac {
      \epsi^\puiss (\puiss-1)
      (2\puiss + \epsz) ^{\vdim[\puiss]-1}
      (\frac94 + \epsz) ^{\lgr\vdim - \puiss - \vdim[\puiss] + 1}
      }{
      \puiss! \prod\fctrange (\vdim* - 1)!
      }
    \prod\fctrange \vdeg* \wt*^{2\vdim*} \delta^{\lgr\vdim}
    + o(\delta^{\lgr\vdim})
  \end{equation}
\end{lem}

\begin{proof}
  Pour choisir un point dans \( (\stairs)\Ciii \), on peut commencer par
  choisir \( \imp*[\vdim*] \) entre \( 0 \) et \( \vdeg* \)
  pour tout \( \fct \), ce qui représente \( \prod\fctrange \vdeg* \)
  possibilités.

  On peut ensuite choisir \( \imp[1][0], \dots \imp[\puiss][0] \)
  sujets à la seule condition
  \begin{equation}
    \wtsum*(\lambda) \le \delta \epsi \pmm.
  \end{equation}
  Le lemme~2.14.5 de \cite{farhith} donne le nombre de choix possibles, qui
  est
  \begin{equation}
    \frac {\prodwt} {\puiss !} (\delta\epsi)^\puiss
    + o(\delta^\puiss)
    \pmm.
  \end{equation}

  Il reste alors à choisir pour tout \( \fct \) un élément de l'ensemble
  \begin{equation}
    \left\{
      (\imp*[1],  \dots, \imp*[\vdim*])
      \in \N ^{\vdim*}
      \text{ tels que }
      \sum_{j=1}^{\vdim*} \imp*[j]
      =
      \delta d'_\fct + r_\fct - \imp*[0] - \imp*[\vdim*]
    \right\}
    \pmm,
  \end{equation}
  qui est de cardinal
  \begin{align}
    \binom {
      \delta d'_\fct + r_\fct - \imp*[0] - \imp*[\vdim*] - 1
      }{
      \vdim* - 1
      }
    & \le
    \binom {
      \delta d'_\fct + r_\fct - 1
      }{
      \vdim* - 1
      }
    \\
    & \le
    \frac {(d'_\fct)^{\vdim* - 1}} {(\vdim* - )!} \delta^{\vdim* - 1}
    + o( \delta^{\vdim* - 1} )
  \end{align}
  On prend alors le produit :
  \begin{equation}
    \card (\stairs)\Ciii
    \le
    \frac {\prodwt} {\puiss !} (\delta\epsi)^\puiss
    \cdot \prod\fctrange
    \frac {(d'_\fct)^{\vdim* - 1}} {(\vdim* - )!}
    \vdeg* \delta^{\vdim* - 1}
    + o( \delta^{\lgr\vdim} )
  \end{equation}
  et le résultat suit en remplaçant \( d'_\fct \) par sa valeur : \( 2\puiss +
  \epsz \) si \( \fct = \puiss \) et \( \wts* (\frac94 + \epsz) \) sinon.
\end{proof}

Au vu des lemmes précédents, on introduit la constante
\begin{equation} \label{e:def-epsii}
  \epsii
  =
  \frac {
    \epsi^\puiss (\puiss-1)
    (2\puiss + \epsz) ^{\vdim[\puiss]-1}
    (\frac94 + \epsz) ^{\lgr\vdim - \puiss - \vdim[\puiss] + 1}
    \prod\fctrange \vdim*
    }{
    \puiss! \epsz^{\vdim[\puiss]}
    }
  \pmm,
\end{equation}
de sorte que
\begin{equation} \label{e:good-codim}
  \frac {\card (\stairs)\Ciii} {\dim \Ideal_{\delta d}}
  \le
  \epsii + o(1)
  \pmm.
\end{equation}


\subsection{Trois lemmes de hauteurs}

Nous allons maintenant expliciter les différents morphismes intervenant dans
l'écriture du système et contrôler la hauteur de leurs matrices, en commençant
par le morphisme \( \wemba \) qui est déjà bien défini.

On étudie en fait le prolongement
\begin{equation}
  \wemba \colon
  \cdn [\cmmh, \cmmhi]_\Dz
  \to
  \cdn [\cmmh]_\Dii
\end{equation}
donné par les mêmes formules. La base évidente de ce nouvel espace de départ
est formée par les monômes \( \cmmh^p \cmmhi^q \) pour
\begin{equation}
  (p, q)
  \in \N^{\puiss(\dimp + 1)} \times \N^{(\puiss - 1) (\dimp - 1)}
  \text{ tel que }
  \lgr{p\mexp*} = \Dz*
  \text{ et }
  \lgr{q\mexpi*} = \Dz_{\puiss + \fcti}
  \pmm,
\end{equation}
où \( d \) est défini par~\eqref{e:def-d}.

% Note: on ne fait pas ce lemme dans sub:wemb
% car on utilise la définition de d pour simplifier le résultat
\begin{lem} \label{l:hmat-wemba}
  Avec les notations précédentes, l'application linéaire
  \begin{equation}
    \wemba \colon
    \cdn [\cmmh, \cmmhi]_\Dz
    \to
    \cdn [\cmmh]_\Dii
  \end{equation}
  est représentée dans les bases canoniques par une matrice de colonnes
  \(
  c_{p, q} = \wemba(\cmmh^p\cmmhi^q)
  = \sum_{\lgr s = \Dii - \lgr p} c_{p, q; s} \cmmh^{s+p}
  \)
  satisfaisant
  \begin{equation}
    \nv1{c_{p, q}}
    \le
    \left(
    \prod\fctirange
    \nv\infty \coi ^{2\wtis* - 1}
    \bigl(
    4^{g\wtis*} (\dimp + 1)^{\wtis* - 1}
    \bigr) ^\dv
    \right) ^\delta
    \pmm.
  \end{equation}
\end{lem}

\begin{proof}
  La définition de \( \wemba \) montre que \(
  \wemba( \cmmh^p \cmmhi^q ) = \cmmh^p \wemba(\cmmhi^q) \), on a donc \(
  \nv1 {c_{p, q}} = \nv1 {\wemba(\cmmhi^q)} \). Ainsi, avec le
  corollaire~\ref{c:addsub-form} et le fait~\ref{f:mult-form}, on a
  \begin{align}
    \nv1 {c_{p, q}}
    & \le
    \prod\fctirange \prod\indrange
    \nv1{ S^{-}_\ind\bigl( Q_{\wti*}(\cmh\fcti), \cmh\puiss \bigr) }
    ^{q\pexp\fcti[\ind]}
    \\ & \le
    \prod\fctirange \prod\indrange \left(
    \nv1{ S^{-}_\ind } \max\indirange \nv1{ Q_{\wti*, \indi} }^2
    \right) ^{q\pexp\fcti[\ind]}
    \\ & \le
    \prod\fctirange \left(
    \nv\infty \coi \cdot 4^{\genre \dv} \cdot \Bigl(
    \nv\infty \coi ^{\wtis*-1} (2^\genre \sqrt{\dimp+1})^{\dv \wtis*-1}
    \Bigr) ^{q\pexp\fcti[\ind]}
    \right)
    \\ & \le
    \left(
    \prod\fctirange
    \nv\infty \coi ^{2\wtis*-1} \bigl(
    4^{\genre\wtis*} (\dimp+1)^{\wtis*-1}
    \bigr) ^\dv
    \right) ^\delta
  \end{align}
  en remarquant que, par définition de \( d \), on a \( \lgr{q\mexpi*} =
    \delta \) pour tout \( \fcti \) (voir \eqref{e:def-dp}).
\end{proof}

Comme on l'a annoncé, le morphisme \( \rdiv \) revient à diviser par des
relations de dépendance intégrale. On énonce pour commencer un résultat de
réduction modulo de telles relations sous une forme un peu générale avant de
l'appliquer au cas qui nous intéresse.

\begin{lem}
  Pour \( \fct \in \{ 1, \dots, \puiss \} \) et \( \ind \in \{ \vdim* + 1,
  \dots \dimp \} \), on se donne :
  \begin{enumthm}
    \item \( \Delta\mexp* \in \N^* \) ;
    \item \( P\mexp*[\ind]
      \in
      \cdn [ \cmh\fct[0], \dots, \cmh\fct[\vdim*], \cmh\fct[\ind] ] \)
      homogène de degré \( \Delta\mexp* \) et unitaire en \( \cmh\fct[\ind]
      \).
  \end{enumthm}
  On note \( N_\fct = \max_\ind \nv1 { P\mexp*[\ind] } \) et \( \Ideal \)
  l'idéal engendré par les \( P\mexp*[\ind] \). En tout multidegré
  \( f \in \N^\puiss \), il existe une (unique) application linéaire
  \begin{equation}
    \rdiv^f \colon \cdn [\cmmh]_ f \to \cdn [\cmmh]_ f^{C'_\Delta}
  \end{equation}
  qui est l'identité modulo \( \Ideal \) (voir~\ref{e:C-spaces}
  et~\ref{e:C-i-iii} pour la définition de l'espace d'arrivée). De plus les
  colonnes \( c_p \) de la matrice de cette application satisfont la majoration
  de norme
  \begin{equation}
    \nv1{c_p}
    \le
    \prod\fctrange \bigl(
    N_\fct \cdot (2 \Delta\mexp*)^\dv
    \bigr) ^{f_\fct}
  \end{equation}
  pour tout \( p \in \N^{\puiss(\dimp+1)} \) de multilongueur \( f \), et
  l'image de \( \cdn [\cmmh]_ f ^{C''_\Delta} \) par \( \rdiv^f \) est
  contenue dans \( \cdn [\cmmh]_ f ^{C'''_\Delta} \).
\end{lem}

\begin{proof}
  C'est essentiellement une variante du lemme~2.5 de~\cite{remivg}, notre
  résultat étant formulé différemment et dans un cadre d'apparence un peu
  moins générale ; la preuve suivra en tout cas les mêmes lignes. On commence
  par décomposer chaque \( P\mexp*[\ind] \) de la façon suivante :
  \begin{equation}
    P\mexp*[\ind]
    =
    \sum _{\alpha=1}^{\Delta\mexp*}
    P\mexp*[\ind, \alpha] \cdot (\cmh\fct[\ind])^{\Delta\mexp* - \alpha}
    \pmm,
  \end{equation}
  où \( P\mexp*[\ind] \in \cdn [ \cmh\fct[0], \dots, \cmh\fct[\vdim*] ] \).
  On pose ensuite, pour tout \( \fct \), tout
  \( \ind \in \{ 0, \dots, \dimp \} \) et tout
  \( p \in \N^{\puiss(\dimp+1)} \),
  \begin{equation}
    \rho_{\fct, \ind, p}
    =
    \begin{dcases*}
      ( \cmh\fct[\ind] ) ^{p\mexp*[\ind]}
      & si \( \fct \le \vdim* \) ;
      \\
      \sum _{\alpha=1}^{\Delta\mexp*}
      U_{p, \alpha, \Delta\mexp*}
      ( P\mexp*[\ind, 1], \dots, P\mexp*[\ind, \Delta\mexp*] )
      & sinon,
    \end{dcases*}
  \end{equation}
  où les polynômes \( U \) sont donnés par le lemme~2.4 de \cite{remivg}. En
  particulier, \( \rho_{\fct, \ind, p} \) est toujours congru à
  \( ( \cmh\fct[\ind] ) ^{p\mexp*[\ind]} \) modulo \( \Ideal \) et on a
  l'estimation de norme
  \begin{equation}
    \nv1{ \rho_{\fct, \ind, p} }
    \le
    \left(
    \nv1{ P\mexp*[\ind] } (2\Delta\mexp*)^\dv
    \right) ^{p\mexp*[\ind]}
    \pmm.
  \end{equation}
  On définit alors \( \rdiv^f \) par son action sur les monômes, en posant
  \begin{equation}
    c_p
    = \rdiv^f(\cmmh^p)
    = \prod\fctrange \prod\indrange \rho_{\fct, \ind, p}
  \end{equation}
  et en prolongeant par linéarité. L'estimation de norme annoncée découle
  directement de la majoration précédente en prenant le produit. Par ailleurs,
  il est clair que si une forme ne fait intervenir que les variables \(
  \cmh\fct[\ind] \) pour \( \ind \le \vdim* + 1 \), il en est de même de son
  image.
\end{proof}

\begin{coro} \label{c:hmat-rdiv}
  Il existe des applications linéaires \( \rdiv^\Dii \) et \( \rdiv^\Diii \)
  faisant commuter le diagramme~\ref{e:diag-f-aux}, dont les matrices dans
  les bases monomiales canoniques ont des colonnes de normes \( \nv1\truc \)
  majorées par
  \begin{equation}
    \prod\fctrange \left(
    \nv1{ \varfc* }
    \cdot ( 2 \vdeg* )^\dv
    \right) ^{ \delta d'_\fct }
    \cdot \expb^{ o(\delta) }
  \end{equation}
\end{coro}

\begin{proof}
  Découle directement du lemme précédent en utilisant les relations de
  dépendance fixées au dernier paragraphe de la scholie~\ref{s:plong-adapt}.
\end{proof}

Intéressons-nous maintenant au morphisme \( \relim \).

\begin{lem}
  Soient, pour tout \( \fct \in \{ 1, \dots, \puiss \} \) et tout \( \ind \in
  \{ \vdim* + 1, \dots \dimp \} \), des formes
  \( S\mexp*[\ind] \in \cdn [ \cmh\fct[0], \dots, \cmh\fct[\vdim*] ] \) et
  \( T\mexp*[\ind] \in \cdn [ \cmh\fct[0], \dots, \cmh\fct[\vdim*+1] ] \)
  telles que \( \deg S\mexp*[\ind] + 1 = T\mexp*[\ind] \) et des entiers
  \( \Delta_\fct \). On note \( \Ideal_{S, T} \) l'idéal engendré par les
  \( S\mexp*[\ind] \cmh\fct[\ind] - T\mexp*[\ind] \).

  Il existe une forme \( R \) ne dépendant que des familles \( S \) et \(
  \Delta \), de degré noté \( r \), et, pour tout \( f \in \N^\puiss \), une
  application linéaire
  \begin{equation}
    \relim \colon
    \cdn [\cmmh]_ {f}^{C'_\Delta}
    \to
    \cdn [\cmmh]_ {f+r}^{C'_\Delta}
  \end{equation}
  qui est la multiplication par \( R \) modulo \( \Ideal_{S, T} \).

  De plus, on peut prendre
  \( R = \prod\fctrange \prod_{\ind = \vdim*+1}^{\dimp}
    ( S\mexp*[\ind] )^{\Delta_\fct} \) ; les colonnes \( c_q^{\relim} \) de la
  matrice de \( \relim \) dans la base canonique satisfont alors
  \begin{equation}
    \nv1{ c_q^{\relim} }
    \le
    \prod\fctrange N_\fct^{\Delta_\fct}
  \end{equation}
  pour tout \( q \) de multilongueur \( f \), où \( N_\fct \) majore
  \( \nv1{ S\mexp*[\ind] } \) et \( \nv1{ T\mexp*[\ind] } \) pour tout
  \( \ind \).
\end{lem}

\begin{proof}
  Soit \( R \) défini comme dans l'énoncé, et \( \cmmh^q \) un monôme de
  l'espace de départ. Par hypothèse, \( q\mexp*[\ind] \le \Delta_\fct \) pour
  tout \( \fct \) et \( \ind \le \vdim* + 1 \), de sorte que l'on peut poser
  \begin{equation}
    \relim( \cmmh^q )
    =
    \prod\fctrange \left(
    \prod_{\ind=1}^{\vdim*}
    (\cmh\fct[\ind])^{q\mexp*[\ind]}
    \prod_{\ind=\vdim*+1}^\dimp
    (T\mexp*[\ind])^{q\mexp*[\ind]}
    (S\mexp*[\ind])^{\Delta_\fct - q\mexp*[\ind]}
    \right)
  \end{equation}
  et prolonger par linéarité. On vérifie immédiatement que \( \relim(\cmmh^q)
  \) est congru à \( R \cdot \cmmh^q \) modulo \( \Ideal_{S, T} \), de même
  que l'estimation de norme annoncée.
\end{proof}

\begin{coro} \label{c:hmat-relim}
  Il existe une forme \( R \in \cdn [\cmmh]^{C''} \) ne dépendant que de \(
  \var \) et un morphisme \( \relim \) tels que le
  diagramme~\ref{e:diag-f-aux} commute. De plus, les colonnes de la matrice
  de \( \relim \) dans les bases canoniques ont leur norme \( \nv1\truc \)
  majorée par
  \begin{equation}
    \prod\fctrange
    \nv1{ N'_\fct }^{\vdeg*}
    \pmm,
  \end{equation}
  où \( N'_\fct \) est une constante ne dépendant que de \( \var* \).
\end{coro}

\begin{proof}
  Il suffit d'établir l'existence de familles \( S \) et \( T \) comme dans
  l'énoncé du lemme précédent, telles que \( \Ideal_{S, T} \subset \varid \) ;
  elle découle du fait qu'on a supposé le plongement adapté
  (scholie~\ref{s:plong-adapt}).
  En effet, d'après le fait~\ref{f:plong-adapt-gen}, pour tous \( \fct \) et
  \( \ind \), il existe des formes \( A\mexp*[\ind, \alpha] \) et \(
    B\mexp*[\ind, \alpha] \) dans
  \( \cdn [ \vmp*[0], \dots, \vmp*[\vdim*] ] \) telles que
  \begin{equation}
    \frac{ \vmp** }{ \vmp*[0] }
    =
    \sum_{\alpha = 0}^{\vdeg* - 1}
    \frac {A\mexp*[\ind, \alpha]} {B\mexp*[\ind, \alpha]}
    \left( \frac{ \vmp*[\vdim* + 1] }{ \vmp*[0] } \right) ^\alpha
    \mod \varid*
    \pmm.
  \end{equation}
  On obtient alors les familles \( S \) et \( T \) recherchées en multipliant
  les deux membres de l'égalité précédent par \( \vmp*[0] \) puis en réduisant
  au même dénominateur le membre de droite.
\end{proof}

Notons que pour construire la forme auxiliaire, on n'a absolument pas besoin
d'en savoir davantage sur \( R \) ni les \( N'_\fct \). On pourra néanmoins
vouloir expliciter une valeur acceptable de \( R \) plus tard.


\subsection{Construction finale de la forme auxiliaire}

Commençons par rappeler la version que nous utiliserons du classique lemme de
\bsc{Siegel}.

\begin{fact} \label{f:siegel}
  Pour toute matrice \( M \) de dimensions \( p \times q \) à coefficients
  dans un corps de nombres \( \cdn \) avec \( p < q \), il existe un vecteur
  \( x \in \cdn^q \) non nul tel que \( M x = 0 \) et satisfaisant
  \begin{equation}
    \hautl[\infty] x
    \le
    \frac q{p-q} \bigl( \hautl[\infty] M + \ln q \bigr) + c_\cdn
    \pmm,
  \end{equation}
  où \( c_\cdn \) est une constante ne dépendant que de \( \cdn \).
\end{fact}

\begin{proof} \later
  C'est le lemme de \bsc{Siegel} de \bsc{Bombieri} et \bsc{Vaaler}.
\end{proof}

Nous sommes maintenant en mesure de construire la forme auxiliaire.

\begin{prop} \label{p:build-aux}
  Sous les hypothèses et notations précédentes, si \( \epsii \le 1/2 \) et si
  \( \delta \) est suffisamment grand, il existe une forme \( F \in \cdn
    [\cmmh, \cmmhi] \), multihomogène de degré \( \delta d \), où l'on
  rappelle que
  \begin{equation}
    d = \bigl(
      \epsz \wts[1],
      \dots,
      \epsz \wts[\puiss],
      1, \dots, 1
    \bigr)
    \pmm,
  \end{equation}
  non nulle modulo \( \ideal{\wemb(\var)} \), telle que \( R \cdot \wemba(F)
    \in \vanish \) et
  \begin{equation}
    \hautl[\infty] F
    \le
    \delta \Bigr(
    (4\puiss + 2\epsz) \sum\fctrange \wts* \hautl[1]{\varfc*}
    + (1 + \epsiii)\wts[1] \cdot \cst{f-aux}
    \Bigr)
    + o(\delta)
    \pmm,
  \end{equation}
  où
  \begin{equation}
    \newcst[]{f-aux}
    \le
    2 \hautl[\infty] \coi
    + 2 \ln (\dimp + 1)
    + (4\puiss + 2\epsz) \max\fctrange \ln(2\vdeg*)
    \pmm.
  \end{equation}
\end{prop}

\begin{proof}
  On reprend le diagramme~\ref{e:diag-f-aux}, dont la première ligne est
  maintenant bien définie. On considère le produit \( M'' \) des matrices des
  applications \( \wemba \), \( \rdiv^\Dii \), \( \relim \) et \(
  \rdiv^\Diii \) composées dans cet ordre. Ses lignes sont naturellement
  indexées par l'ensemble des multiindices
  \begin{equation}
    \left\{
      \imp \in \prod\fctrange \bigl(
      \N^{\vdim* + 1}
      \times \{ 0, \dots, \vdeg* - 1 \}
      \times \{ 0 \}^{\dimp - \vdim* - 1}
      \bigr)
      \text{ tel que }
      \sum\indrange \imp** = \Diii
      \ \forall \fct
      \right\}
    \pmm.
  \end{equation}
  On extrait les lignes dont l'indice appartient à l'ensemble
  \( (\stairs)\Ciii \) défini par~\eqref{e:stairs-c3} pour former la matrice
  \( M' \), dont les colonnes sont naturellement indexées par les monômes de
  multidegré \( \Dz \) en \( (\cmmh, \cmmhi) \). On extrait alors de
  \( M' \) les colonnes indexées par les monômes constituant la base choisie
  de \( \Ideal_\Dz \) pour former la matrice \( M \) à laquelle nous allons
  appliquer le fait~\ref{f:siegel}, et dont on note \( p, q \) les dimensions.

  L'hypothèse faite sur \( \epsii \) et la relation~\eqref{e:good-codim}
  impliquent directement que \( p < q \) et que
  \begin{equation}
    \frac q {p-q}
    \le
    \frac 1 {\frac1\epsii - 1} + o(1)
    \le
    1 + o(1)
    \pmm.
  \end{equation}
  En remarquant de plus que \( \ln q = o(\delta) \), on voit qu'il existe une
  forme \( F \) satisfaisant aux conditions de l'énoncé mais de hauteur majorée
  par \( \hautl[\infty] M + o(\delta) \) ; il reste donc à estimer cette
  hauteur.

  On utilise le fait élémentaire suivant : si \( A \) et \( B \) sont deux
  matrices telles que le produit \( A \cdot B \) est défini, alors
  \( \hautl[\infty]{AB} \) est majorée par la somme de \( \hautl[\infty] A \)
  et d'un majorant de \( \hautl[1] c \) où \( c \) parcourt les colonnes
  de \( B \).

  Or, d'après le lemme~\ref{l:hmat-wemba}, la hauteur \( \Hautl[1] \) des
  colonnes de la matrice de \( \wemba \) est majorée par
  \begin{equation}
    \bigl( \delta + o(\delta) \bigr)
    \sum\fctirange \biggl(
    2\wtis* \hautl[\infty] \coi
    + \genre \wtis* \ln(4)
    + \wtis* \ln(\dimp + 1)
    \biggr)
    \pmm,
  \end{equation}
  d'après le corollaire~\ref{c:hmat-rdiv} (et en remarquant que \( d'_\fct \le
    2\puiss + \epsz \) pour tout \( \fct \)), celle des colonnes des matrices
  de \( \rdiv^\Dii \) et \( \rdiv^\Diii \) par
  \begin{equation}
    \bigl( \delta + o(\delta) \bigr)
    (2\puiss + \epsz)
    \sum\fctrange \wts* \biggl(
    \hautl[1]{\varfc*}
    + \ln(2\vdeg*)
    \biggr)
    \pmm,
  \end{equation}
  et d'après le corollaire~\ref{c:hmat-relim} celle des colonnes de la matrice
  de \( \relim \) est un \( o(\delta) \). En sommant (en comptant bien deux
  fois le majorant commun des hauteurs des deux \( \rdiv \)) et en tenant
  compte du fait que \( \puiss - 1 + \sum\fctirange \wtis* \le (1 + \epsiii)
  \wts[1] \), il vient
  \begin{equation}
    \hautl[\infty] M
    \le
    \delta \Bigr(
    (4\puiss + 2\epsz) \sum\fctrange \wts* \hautl[1]{\varfc*}
    + (1 + \epsiii)\wts[1] \cdot \cst{f-aux}
    \Bigr)
    + o(\delta)
    \pmm,
  \end{equation}
  avec
  \begin{equation}
    \cst{f-aux}
    \le
    2 \hautl[\infty] \coi + \genre \ln(4) + \ln(\dimp + 1)
    + (4\puiss + 2\epsz) \max\fctrange \ln(2\vdeg*)
    \pmm.
  \end{equation}
  On remarque alors que \( 4^\genre \le \dimp + 1 \) pour aboutir à la
  majoration annoncée.
\end{proof}

\begin{scho} \label{s:aux-co}
  On note désormais \( F \) une forme comme dans la conclusion de la
  proposition précédente, \( F' = \wemba(F) \) et \( G =
    \rdiv^{\delta d' + r}(\relim(\rdiv^{\delta d'}(F))) \). On a alors
  \begin{enumthm}
    \item \( G = RF' \mod \varid \) et \( G \in \vanish* \) ;
    \item \( \deg G = \deg RF' = \deg F' + o(\delta) \) ;
    \item \( \deg F' = \delta \bigl(
          \wts[1] (\frac94 + \epsz), \dots,
          \wts[\puiss*] (\frac94 + \epsz), 2\puiss - 2 + \epsz
        \bigr) \) ;
    \item \( \nv1{F'} \le \nv1 F \cdot \left(
          \nv\infty\coi^2 \bigl( 4^\genre \dimp** \bigr)^\dv
        \right)^{(1+\epsiii) \wts[1] \delta}
      \) ;
    \item \( \nv1{G} \le \nv1 F \cdot
        \left(
          \prod\fctirange
          \nv1{\varfc*}^{(4\puiss + 2\epsz) \wtis*}
          \cdot
          \cst{f-aux-loc}^{(1+\epsiii) \wts[1]}
        \right)^\delta
        \cdot
        2^{o(\delta)}
      \), où
      \begin{equation}
        \newcst{f-aux-loc}
        =
        \nv\infty\coi^2 \bigl(
          4^\genre \dimp** \cdot (2 \max_\fct \vdeg*)^{(4\puiss + 2\epsz)}
        \bigr)^\dv
      \end{equation}
  \end{enumthm}
\end{scho}

\begin{proof}
  Les trois premiers points sont donnés directement par la méthode utilisée
  pour construire \( F \). Le suivant résulte du lemme~\ref{l:hmat-wemba} et
  de la propriété~\ref{e:l1-subslin} de la norme utilisée en effectuant les
  mêmes simplifications que dans la démonstration de la proposition
  précédente.

  Pour le dernier point, on reprend la matrice \( M \) de cette démonstration
  et on en majore la norme locale de la même façon qu'on en a majoré la
  hauteur :
  \begin{align}
    \nv\infty M
    & \le
    \prod\fctirange \left(
      \nv1{\varfc*}^{(4\puiss + 2\epsz) \wtis*}
    \right)^\delta
    \\ & \qquad
    \cdot \left(
      \nv\infty\coi^2 \bigl(
        4^\genre \dimp** \cdot (2 \max_\fct \vdeg*)^{(4\puiss + 2\epsz)}
      \bigr)^\dv
    \right)^{(1+\epsiii) \wts[1] \delta}
    \cdot
    2^{o(\delta)}
  \end{align}
  On en déduit l'estimation annoncée en remarquant que \( G \) est l'image de
  \( F \) par l'application de matrice \( M \) et que \( \nv1{M F} \le
    \nv\infty M \cdot \nv1 F \cdot 2^{\dv o(\delta)} \) par les propriétés
  classiques des normes.
\end{proof}

\endinput

% vim: spell spelllang=fr
