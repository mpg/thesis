% !TEX root = main.tex

\chapter{Inégalité de \bsc{Vojta}} \label{chap:vojta}

\section{Cas particulier fondamental}
\label{sec:vojta-main}

Nous établissons d'abord une inégalité de \bsc{Vojta} effective dans le cas
particulier où la variété à approcher est, dans un plongement \( \vaemb \)
fixé (\lat{cf.} section~\vref{sec:vaemb}) l'hyperplan \( \vp[0] = 0 \) ; nous
verrons plus tard (section~\vref{sec:vojta-coro}) que ce cas implique le cas
général. On rappelle la remarque~\vref{r:ha-cdn} concernant le sens à donner à
la condition~\eqref{e:Vapx} dans l'énoncé suivant.

\begin{thm} \label{t:vojta-div}
  Soient \( \divi \) l'hyperplan défini par \( \vp[0] = 0 \) dans le
  plongement fixé et \( \eps > 0 \) un nombre réel.  Pour tout
  \( m \ge g + 1 \), il n'existe dans \( \va(\Qbar) \) aucune famille
  de points \( x_1, \dots, x_m \) satisfaisant simultanément aux
  conditions suivantes :
  \begin{align}
    0 < \distv{x_\fct}{\divi}
    & < \hautm2{x_\fct}^{-\wtapx \eps}
    \quad \forall v \in \placesapx
    \label{e:Vapx}
    \\
    \hautn{x_1} & > \Vbig
    \label{e:Vbig}
    \\
    \cos(x_\fct, x_\fcti) & > 1 - \Vcos
    \label{e:Vcos}
    \\
    \hautn{x_\fct} & > \Vfar \hautn{x_{\fct-1}}
    \label{e:Vfar}
  \end{align}
  avec
  \begin{align}
    \label{e:def-Vbig}
    \Vbig & = 4\cst{vs-ht} \Lambda^{(1 + \frac1m) f(m-1)}
    \\
    \label{e:def-Vfar}
    \Vfar & = 4 m \Lambda^{2m f(m)}
    \\
    \label{e:def-Vcos}
    \Vcos & =
    \frac1{ m }
    \left(
      \frac{ \eps }{ 85 \nclmaps \cdot 5^g }
    \right)^{ \frac{m}{m-g} }
    \\
    \label{e:def-Lambda}
    \Lambda
    & = \max \bigl(
      (\deg \va)^m,
      5 m^2
      \bigl(
        85 \nclmaps \cdot 5^g \cdot \eps^{-1}
      \bigr)^{ \frac{m}{m-g} }
      \deg \va,
      (\sqrt2 m g)^{mg}
    \bigr)
    \\ \label{e:def-f}
    f(\vdim) & = \prod_{i=\vdim+1}^{mg} (2i + 1)
    \\
    \newcst[\vaemb]{vs-ht}
    & = \max( \hautl1{ \chow\va }, \hlclab, \htcmp, n\ln(n + 1) )
  \end{align}
\end{thm}

La démonstration de ce théorème occupera la majeure partie de ce chapitre,
jusqu'à la section~\vref{sec:vojta-coro} non incluse.

Remarquons que dans l'énoncé ci-dessus, \( \divi \) n'est pas une sous-variété
de \( \va \), cependant il est clair que si dans la condition~\eqref{e:Vapx}
on remplace \( \divi \) par \( \divi \cap \va \), on obtient une condition
plus forte : \( \distv x \divi \le \distv x{ \divi\cap\va } \) et \( x \not\in
  \divi\cap\va \Leftrightarrow x \not\in \divi \) pour \( x \in \va \). Ainsi,
l'énoncé ci-dessus est légèrement plus fort que le même énoncé en remplaçant \( \divi \)
par \( \divi\cap\va \) dans la condition d'approximation.



\section{Réduction et autres préliminaires}
\label{sec:vojta-reduc}

On rappelle pour commencer que d'après la proposition~\vref{p:dv-hp}, on a
\( \distv x \divi = \av{x_0} / \nv2 x \), ce qui est une motivation
pour traiter en premier le cas où \( \divi = \zeros{X_0} \), l'autre raison
étant que l'indice d'annulation le long de \( \divi \) admet une définition
particulièrement commode à manipuler dans ce cas (section~\vref{sec:siegel}).

Notons pour commencer qu'on peut supposer \( \eps < 1 \), car sinon
l'inégalité de la taille appliquée à l'expression de la distance ci-dessus
montre qu'il n'existe aucun point satisfaisant~\eqref{e:Vapx}, sans même
utiliser les autres conditions. De même, on peut supposer que \( \divi \) ne
contient pas \( \va \) car sinon il n'existe aucun point
satisfaisant~\eqref{e:Vapx}.


\subsection{Poids associés à une famille d'approximations}
\label{sec:wt}

La démonstration procède par l'absurde : si le théorème est faux, fixons une
famille \( \ex[1], \dots, \ex[m] \) qui le contredit. Bien que cette
famille soit toujours supposée satisfaire à toutes les conditions du théorème,
nous préciserons dans les hypothèses de la plupart des énoncés suivants la ou
lesquelles de ces conditions nous utilisons, par souci de clarté.

Nous utiliserons des combinaisons linéaires des \( \ex* \) de petite hauteur.
Les lemmes suivants permettent de choisir les coefficients pour ces
combinaisons ; nous les prendrons entiers et n'ayant que \( 2 \) et \( 3 \)
pour diviseurs premiers, de sorte à disposer de représentations polynomiales
convenables des formes linéaires abéliennes associées
satisfaisant~\eqref{e:clab-norm} et~\eqref{e:clab-loc}.

\begin{lem} \label{l:wt-choose-gen}
  Soit \( \zeta > 0 \) un réel et supposons qu'aucun des \( \ex* \) n'est de
  torsion. Il existe des entiers \( \wt* \in 2^\N 3^\N \) tels que, pour tout
  \( \fct \in \set{1, \dots, m} \) :
  \begin{equation} \label{e:wt-ratio-gen}
    \frac1{1 + \zeta}
    \le
    \frac{ \wts* \hautn{\ex*} }{ \wts[1] \hautn{\ex[1]} }
    \le
    1 + \zeta
    \pmm.
  \end{equation}
\end{lem}

\begin{proof}
  L'hypothèse qu'aucun des \( \ex* \) n'est de torsion garanti que le quotient
  des \( \hautn{ \ex* } / \hautn{ \ex[1] } \) est bien défini et n'est jamais
  nul.  On commence par choisir des rationnels \( b_\fct = 2^{b_{\fct2}}
    3^{b_{\fct3}} \) tels que
  \begin{equation}
    \frac1{1 + \zeta}
    \le
    b_\fct^2 \frac{ \hautn{\ex*} }{ \hautn{\ex[1]} }
    \le
    1 + \zeta
    \pmm,
  \end{equation}
  soit en prenant les logarithmes et en divisant par deux :
  \begin{equation}
    \abs*{
      b_{\fct2} \log 2 + b_{\fct3} \log 3
      + \frac12 \log \frac{ \hautn{\ex*} }{ \hautn{\ex[1]} }
    }
    \le
    \frac12 \log (1 + \zeta)
    \pmm.
  \end{equation}
  Comme \( \Z \log2 + \Z \log 3 \) est dense dans \( \R \), il est
  certainement possible de choisir indépendamment pour chaque \( \fct \) deux
  entiers \( b_{\fct2} \) et \( b_{\fct3} \) tels que cette dernière condition
  soit satisfaite.

  Il ne reste plus qu'à définir \( \wt[1] \) comme un dénominateur commun des
  \( b_\fct \) puis à poser \( \wt* = \wt[1] b_\fct \) pour tout \( \fct > 1
  \). Notons que ceci nous permet de choisir les \( \wt* \) arbitrairement
  grands, car seuls leurs rapports comptent.
\end{proof}

En utilisant les inégalités élémentaires \( (1 + \zeta)^{-1} \ge 1 - \zeta \)
et \( (1 + \zeta)^{1/2} \le 1 + \zeta/2 \), on en déduit facilement que pour
toute famille \( (\wt*) \) satisfaisant à~\eqref{e:wt-ratio-gen}, on a
\begin{equation} \label{e:wt-diff-nnt-gen}
  \abs[\big]{ \wt* \nnt{\ex*} - \wt[1] \nnt{\ex[1]} }
  \le
  \frac\zeta2 \wt[1] \nnt{\ex[1]}
  \quad\text{et}\quad
  \abs[\big]{ \wt* \nnt{\ex*} - \wt[\fcti] \nnt{\ex[\fcti]} }
  \le
  \zeta \wt[1] \nnt{\ex[1]}
\end{equation}
pour tous \( \fct \) et \( \fcti \), où \( \nnt\truc = \hautn\truc^{1/2} \)
désigne la norme de \NT. Cette inégalité nous sera utile pour démontrer le
lemme suivant.

\begin{lem}
  Soient \( \zeta > 0 \) et \( (\wt*) \) une famille d'entiers satisfaisant
  à~\eqref{e:wt-ratio-gen} (on suppose qu'aucun \( \ex* \) n'est de torsion).
  Si de plus la famille \( (\ex*) \) satisfait à~\eqref{e:Vcos}, alors pour
  tout \( \fct \in \set{1, \dots, m} \) on a
  \(
    \hautn{\wt* \ex* - \wt** \ex**}
    \le
    \wts[1] \hautn{\ex[1]} \left(
      \zeta^2 + 2 \Vcos (1 + \zeta)
    \right)
  \).
\end{lem}

\begin{proof}
  En développant le membre de gauche, il vient successivement
  \begin{alignat*}{2}
    \nnt{\wt* \ex* - \wt** \ex**}^2
    & =
    \nnt{\wt* \ex*}^2 + \nnt{\wt** \ex**}^2
    - 2 \scalnt{\wt* \ex*}{\wt** \ex**}
    \\
    & = \wts* \nnt{\ex*}^2 + \wts** \nnt{\ex**}^2
    - 2 \wt* \wt** \scalnt{\ex*}{\ex**}
    \\
    & = \left( \wt* \nnt{\ex*} - \wt** \nnt{\ex**} \right)^2
    + 2 \wt* \wt** \left(
      \nnt{\ex*} \nnt{\ex**} - \scalnt{\ex*}{\ex**}
    \right)
    \\
    & \le \left( \wt* \nnt{\ex*} - \wt** \nnt{\ex**} \right)^2
    + 2 \wt* \wt** \nnt{\ex*} \nnt{\ex**} \Vcos
    && \text{d'après~\eqref{e:Vcos}}
    \\
    & \le \left( \zeta \wt[1] \nnt{\ex[1]} \right)^2
    + 2 \left( \wt[1] \nnt{\ex[1]} \sqrt{1 + \zeta} \right)^2 \Vcos
    && \text{d'après~\eqref{e:wt-diff-nnt-gen} et~\eqref{e:wt-ratio-gen}}
  \end{alignat*}
  qui donne le résultat annoncé en factorisant \( (\wt[1] \nnt{\ex[1]})^2 =
      \wts[1] \hautn{\ex[1]} \).
\end{proof}

\begin{nota} \label{n:wt-choose}
  On choisit désormais et jusqu'à la fin du chapitre une famille \( (\wt*) \)
  donnée par l'application du lemme~\vref{l:wt-choose-gen} avec \( \zeta =
    \sqrt{\Vcos} / 2 \) (en effet, \eqref{e:Vbig} et \eqref{e:Vfar}
  garantissent qu'aucun \( \ex* \) n'est de torsion).
\end{nota}

En remarquant que \( \Vcos < 1/2 \), le lemme précédent et la
relation~\eqref{e:wt-ratio-gen} donnent immédiatement les relations :
\begin{gather} \label{e:hautn-wt-diff}
  \hautn{\wt* \ex* - \wt** \ex**}
  \le
  3 \Vcos \wts[1] \hautn{\ex[1]}
  \\ \label{e:wt-ratio}
  \frac12
  \le
  \frac{ \wts* \hautn{\ex*} }{ \wts[1] \hautn{\ex[1]} }
  \le
  2
  \pmm.
\end{gather}
Les estimations de la deuxième ligne sont un peu larges mais suffiront
amplement en pratique.

Jusqu'à présent nous avons surtout exploité l'hypothèse~\eqref{e:Vcos},
pour~\eqref{e:hautn-wt-diff}. En tenant compte de l'hypothèse~\eqref{e:Vfar},
on voit de plus que la suite \( \wts[1], \dots, \wts** \) décroît au moins
comme une suite géométrique de raison inférieure à \( 1 \). Plus précisément,
dès que \( \ex \) satisfait~\eqref{e:Vfar}, on a
\begin{equation} \label{e:wt-geom}
  \wts*
  \le
  (1+\zeta) \cdot \frac{\wts[1]}{\Vfar^{\fct-1}}
\end{equation}
en appliquant directement la définition de \( \wt* \). On peut ainsi majorer
des sommes faisant intervenir les \( \wts* \) en fonction de \( \wts[1] \) ;
par exemple, l'énoncé suivant nous sera utile par la suite.

\begin{lem} \label{l:sum-wts}
  On a \(
    \sum_{\fct=1}^{m-1} (\wts* + \wts**)
    = \sum\fctrange \wtw* \wts*
    \le 2 \wts[1]
  \), où l'on a noté \( \wtw* = 1 \) si \( \fct < m \) et \( \wtw** =
    m - 1 \).
\end{lem}

\begin{proof}
  En effet, on écrit
  \begin{align}
    \frac1{\wts[1]} \sum_{\fct=1}^{m-1} (\wts* + \wts**)
    & = 1
    + \frac{(m-1) \wts**}{\wts[1]}
    + \sum_{\fct=2}^{m-1} \frac{\wts*}{\wts[1]}
    \\
    & \le 1
    + \frac{m(1+\zeta)}{\Vfar^{m-1}}
    + (1+\zeta) \sum_{\fct=2}^{m-1} \Vfar^{-\fct+1}
    \\
    & \le 1 + (1+\zeta) \left(
      \Bigl( \frac2\Vfar \Bigr)^{m-1}
      + \frac1{\Vfar-1}
    \right)
    \pmm.
  \end{align}
  Pour conclure, il suffit d'observer que \( \Vfar > 4 \) et que \( \Vcos <
    1/9 \).
\end{proof}


\subsection{Réduction à l'existence d'une forme motrice}
\label{sec:vojta-prop}

Nous regardons \( \ex = (\ex*) \) comme un point de \( \va^m \) plongée
dans \( (\projd)^m \), et introduisons l'ensemble \( \varset(\ex) \) des
sous-variétés produit \( \var = \var[1] \times \dots \times \var** \) de \(
  \va^m \) qui contiennent \( \ex \) et satisfont aux majorations
suivantes, où les notions de degré et de hauteurs s'entendent dans le
plongement \( \vaemb^m \) et où l'on rappelle la notation \( \wtw* = 1 \)
si \( \fct < m \) et \( \wtw** = m-1 \) :
\begin{align}
  \label{e:varset-deg}
  \max_\fct \vdeg*
  & \le B
  = \cst{vs-deg-prod}^{-1} \, \Lambda^{f(\vdim)}
  \\ \label{e:varset-deg-prod}
  \prod\fctrange \vdeg*
  & \le \Lambda^{f(\vdim)}
  \\ \label{e:varset-ht}
  \sum\fctrange \wtw* \wts* \hautl1{\varfc*}
  & \le \cst{vs-ht} \Lambda^{(1 + \frac1m) f(\vdim)} \wts[1]
  \pmm,
\end{align}
où l'on a noté \( \vdeg* = \deg \var* \) et \( \vdim = \dim \var \), ainsi que
\begin{align}
  \newcst[]{vs-deg-prod}
  & =
  5 m^2
  \bigl(
    85 \nclmaps \cdot 5^g \cdot \eps^{-1}
  \bigr)^{ \frac{m}{m-g} }
\end{align}
de sorte que l'ensemble \( \varset(\ex) \) ne dépend de \( \ex \) que
par la condition \( \ex \in \var \) et les rapports entre les \( \wt* \).

Pour commencer, remarquons que cet ensemble n'est pas vide puisqu'il contient
\( \va^m \). En effet, la définition de \( \Lambda \) (deux premiers
arguments du maximum) assure que~\eqref{e:varset-deg}
et~\eqref{e:varset-deg-prod} sont satisfaits ; le fait que \( \cst{vs-ht} \ge
  \hautl1\va \), que \( \Lambda > 2 \) et le lemme~\vref{l:sum-wts}
garantissent que la condition~\eqref{e:varset-ht} l'est aussi.

Nous allons voir qu'à partir d'une variété dans \( \varset(\ex) \) on peut en
fabriquer une plus petite, pour peu que la variété initiale n'ait aucun
facteur réduit à un point. Pour cela, on utilise la proposition suivante.

\begin{prop} \label{p:varset-notmin}
  Soit \( \var \in \varset(\ex) \) n'ayant aucun facteur de dimension nulle.
  Alors il existe un \( \fct \) et une forme \( T \) ne dépendant que de \(
    \vmp* \), s'annulant en \( \ex \) mais pas identiquement sur \( \var \),
  telle que
  \begin{align}
    \deg T
    & \le \Lambda^{2\vdim f(\vdim)}
    \\
    \wtw* \wts* \hautl\infty{ T }
    & \le
    g \Lambda^{2\vdim f(\vdim)}
    \Biggl(
      6 \sum_{\fcti=1}^m \wtw[\fcti] \wts[\fcti] \hautl1{\varfc[\fcti]}
      + 3 \wts[1] \cst{vs-ht} \Lambda^{(1 + \frac1m) f(\vdim)}
    \Biggr)
    \pmm.
  \end{align}
\end{prop}

Supposons un instant cet énoncé acquis et voyons comment on en déduit le
théorème~\vref{t:vojta-div} ; nous appelons « motrice » la forme \( T \) de la
proposition car c'est elle qui « fait marcher la machine » en permettant la
construction itérative dans la démonstration ci-dessus.

\begin{proof}[\proofname\ du théorème~\vref{t:vojta-div}]
  Voyons comment, en partant d'une variété \( \var \) satisfaisant aux
  hypothèses de la proposition, on peut fabriquer une variété \( \var' \) plus
  petite et appartenant toujours à \( \varset(\ex) \) : on utilise la forme
  \( T \) fournie par la proposition et on note
  \( \var' \) la variété égale à \( \var \) sur tous les facteurs sauf le \(
    \fct \)-ème où on choisit une composante irréductible de
  \( \var* \cap \zeros T \) qui contient \( \ex* \), de sorte que \( \var' \)
  est une variété strictement contenue dans \( \var \) et qui contient \( \ex
  \).

  Restent donc à vérifier les conditions~\eqref{e:varset-deg}
  à~\eqref{e:varset-ht} ; pour les deux premières (degrés), le théorème de
  \bsc{Bézout} donne
  \begin{equation}
    \deg \var*' \le \vdeg* \deg T
  \end{equation}
  ce qui montre que les quantités apparaissant dans les
  conditions~\eqref{e:varset-deg} et~\eqref{e:varset-deg-prod} sur les degrés
  sont au plus multipliées par \( \deg T \le \Lambda^{2\vdim f(\vdim)} \) en
  passant de \( \var \) à \( \var' \). Or, par définition de \( f \) on a
  \begin{equation} \label{e:ct-f}
    f(\vdim-1) = 2\vdim f(\vdim) + f(\vdim)
  \end{equation}
  donc ces conditions sont à nouveau satisfaites par \( \var' \).

  Pour la condition sur la hauteur, on utilise la formule suivante
  (\emph{arithmetic Bézout theorem}, \cite[p. 87]{phidg}) :
  \begin{equation}
    \hautl\htpph{ \chow{\var*'} }
    \le
    \deg \var* \cdot \hautl{\mathrm{m}} T
    + \deg T \cdot \hautl\htpph{ \varfc* }
  \end{equation}
  où \( \Hautl{\mathrm{m}} \) est la hauteur désignée par \( h_1 \) dans la
  référence citée et par \( h_m \) dans \cite{remstp}, dont le lemme~5.2
  assure par ailleurs que \( \hautl{\mathrm{m}} T \le \hautl\infty T + \sqrt n
  \). En utilisant~\eqref{e:ht-pph-1} pour se ramener aux hauteurs qui nous
  intéressent, le théorème de \bsc{Bézout} arithmétique s'écrit donc, dans
  notre cas :
  \begin{align}
    \hautl1{ \chow{\var*'} }
    & \le
    \deg \var* \bigl( \hautl\infty T + \sqrt n \bigr)
    + \deg T \Bigl(
      \hautl1{ \varfc* }
      + \ln(n + 1) (\vdim* + 1) \deg \var*
    \Bigr)
    \\ & \quad
    + \ln(n + 1) \vdim* \deg \var*'
    \\ & \le
    B \bigl( \hautl\infty T + n \bigr)
    + \deg T \Bigl(
      \hautl1{ \varfc* } + ng B
    \Bigr)
    + ng B \deg T
    \\ & \le
    \Lambda^{2\vdim f(\vdim)} \hautl1{ \varfc* }
    + B \hautl\infty T
    + 3 n g B \Lambda^{2\vdim f(\vdim)}
  \end{align}
  et finalement, en multipliant par \( \wtw* \wts* \) et en majorant dans le
  dernier terme cette quantité par \( \wts[1] \), puis en utilisant
  l'information sur la hauteur de \( T \) fournie par la proposition :
  \begin{align}
    \wtw* \wts* \hautl1{ \chow{\var*'} }
    & \le
    \wtw* \wts* \Lambda^{2\vdim f(\vdim)} \hautl1{ \varfc* }
    + 3 \wts[1] n g B \Lambda^{2\vdim f(\vdim)}
    + \wtw* \wts* B \hautl\infty T
    \\ & \le
    \wtw* \wts* \Lambda^{2\vdim f(\vdim)} \hautl1{ \varfc* }
    + 3 \wts[1] n g B \Lambda^{2\vdim f(\vdim)}
    \\ & \qquad
    + B g \Lambda^{2\vdim f(\vdim)}
    \Biggl(
      6 \sum_{\fcti=1}^m \wtw[\fcti] \wts[\fcti] \hautl1{\varfc[\fcti]}
      + 3 \wts[1] \cst{vs-ht} \Lambda^{(1 + \frac1m) f(\vdim)}
    \Biggr)
    \\ & \le
    \Lambda^{2\vdim f(\vdim)} \Biggl(
      \wtw* \wts* \hautl1{ \varfc* }
      + B g
      \Bigl(
        6 \sum_{\fcti=1}^m \wtw[\fcti] \wts[\fcti] \hautl1{\varfc[\fcti]}
        + 4 \wts[1] \cst{vs-ht} \Lambda^{(1 + \frac1m) f(\vdim)}
      \Bigr)
    \Biggr)
  \end{align}
  en remarquant que \( 3n \le \cst{vs-ht} \Lambda \) pour intégrer le
  deuxième terme de la deuxième ligne dans celui en \( \cst{vs-ht} \).

  Nous pouvons maintenant substituer cette estimation dans la somme pondérée
  des hauteurs :
  \begin{align}
    \sum_{\fcti=1}^m
    \wtw[\fcti] \wts[\fcti] \hautl1{\chow{\var[\fcti]'}}
    & =
    \sum_{\fcti \neq \fct}
    \wtw[\fcti] \wts[\fcti] \hautl1{\varfc[\fcti]}
    + \wtw* \wts* \hautl1{ \chow{\var*'} }
    \\ & \le
    \Lambda^{2\vdim f(\vdim)} \sum_{\fcti \neq \fct}
    \wtw[\fcti] \wts[\fcti] \hautl1{\varfc[\fcti]}
    + \Lambda^{2\vdim f(\vdim)} \wtw* \wts* \hautl1{ \varfc* }
    \\ & \qquad
    + B g \Lambda^{2\vdim f(\vdim)}
    \Bigl(
      6 \sum_{\fcti=1}^m \wtw[\fcti] \wts[\fcti] \hautl1{\varfc[\fcti]}
      + 4 \wts[1] \cst{vs-ht} \Lambda^{(1 + \frac1m) f(\vdim)}
    \Bigr)
    \\ & \le
    7 B g \Lambda^{2\vdim f(\vdim)}
    \Bigl(
      \sum_{\fcti=1}^m \wtw[\fcti] \wts[\fcti] \hautl1{\varfc[\fcti]}
      + \wts[1] \cst{vs-ht} \Lambda^{(1 + \frac1m) f(\vdim)}
    \Bigr)
    \\ & \le
    \frac12 \Lambda^{(2\vdim + 1) f(\vdim)}
    \Bigl(
      \sum_{\fcti=1}^m \wtw[\fcti] \wts[\fcti] \hautl1{\varfc[\fcti]}
      + \wts[1] \cst{vs-ht} \Lambda^{(1 + \frac1m) f(\vdim)}
    \Bigr)
    \label{e:vs-ht-rec}
  \end{align}
  en utilisant le fait que \( 7 B g \) est (largement) majoré par \(
    \Lambda^{f(\vdim)} / 2 \) d'après~\eqref{e:varset-deg}. Il ne reste plus
  qu'à appliquer l'hypothèse de hauteur sur \( \var \), à
  savoir~\eqref{e:varset-ht}, pour obtenir
  \begin{align}
    \sum_{\fcti=1}^m
    \wtw[\fcti] \wts[\fcti] \hautl1{\chow{\var[\fcti]'}}
    & \le
    \wts[1] \cst{vs-ht} \Lambda^{(2\vdim + 2 + \frac1m) f(\vdim)}
  \end{align}
  qui achève de montrer que \( \var' \in \varset(\ex) \) car
  \begin{equation}
    (2\vdim + 2 + \frac1m) f(\vdim)
    \le
    (2\vdim + 1)(1 + \frac1m) f(\vdim)
    =
    (1 + \frac1m) f(\vdim-1)
    \pmm.
  \end{equation}

  On peut ainsi, en partant de \( \va^m \), construire une suite
  d'éléments de \( \varset(\ex) \) de dimensions décroissantes, jusqu'au
  moment où l'un des facteurs est réduit à un point, c'est-à-dire que \( \var*
    = \set{\ex*} \) pour un certain \( \fct \) et en particulier \(
    \hautl1{\varfc*} = \hautl1{\ex*} \) vu l'expression de la forme de
  \bsc{Chow} d'un point. Par ailleurs, à ce stade on a \( \vdim \ge m - 1
  \) ; on en déduit alors, par \eqref{e:varset-ht} et~\eqref{e:comp-h-hn},
  \begin{align}
    \wts* \hautn{\ex*}
    & \le 2\cst{vs-ht} \Lambda^{(1 + \frac1m) f(m-1)} \wts[1]
    \intertext{puis, par \eqref{e:wt-ratio},}
    \wts[1] \hautn{\ex[1]}
    & \le 4\cst{vs-ht} \Lambda^{(1 + \frac1m) f(m-1)} \wts[1]
    \label{e:ct-Vbig-final}
  \end{align}
  qui contredit directement~\eqref{e:Vbig}. Cette contradiction prouve qu'il
  est absurde de supposer l'existence d'une famille satisfaisant simultanément
  toutes les conditions du théorème~\vref{t:vojta-div}.
\end{proof}

Il suffit donc d'établir la proposition \vref{p:varset-notmin} pour prouver le
théorème \vref{t:vojta-div} ; la démonstration de cette proposition nous
occupera le reste du chapitre. L'intérêt de l'énoncer immédiatement plutôt
qu'au moment où nous serons en mesure de l'établir est de pouvoir écarter
rapidement quelques cas particuliers pour lesquels la méthode qui suit ne
s'applique pas (mais qui sont heureusement immédiats). Nous énoncerons ces cas
à la fin de la sous-section suivante après avoir introduit le système de
coordonnées dans lequel ils s'expriment naturellement.

\begin{nota} \label{n:var&co}
  Désormais et jusqu'à la fin du chapitre, nous fixons une variété
  \( \var = \var[1] \times \dots \times \var[m] \) satisfaisant aux
  hypothèses de la proposition ; en particulier elle n'est contenue dans aucun
  hyperplan d'équation \( \vmp*[0] = 0 \) puisqu'elle contient
  \( \ex \) qui n'est d'après \eqref{e:Vapx} sur aucun de ces hyperplans, et
  aucun de ses facteurs n'est réduit à un point.

  On notera en outre \( \vdim* = \dim \var* \) et \( \vdim = \dim \var \) ;
  observons de suite que \( 1 \le \vdim* \le g \) et donc \( m \le
    \vdim \le mg \). Notons également
  \( \vdeg* = \deg \var* \) et \( B \) le majorant commun des \( \vdeg* \)
  donné par~\eqref{e:varset-deg}. Enfin, on note \( \varfc* \) une
  \footnote{\label{fn:varfc}Cette forme est unique à multiplication par un
    scalaire près ; au besoin on la supposera normalisée de sorte que toutes
    ses normes locales soient au moins \( 1 \), par exemple en faisant en
    sorte qu'un de ses coefficients soit \( 1 \).}
  forme de \bsc{Chow} de \( \var* \) ainsi que \( \varid \) l'idéal
  multihomogène saturé de \( \var \) et \( \varid* \) ceux de ses facteurs.
\end{nota}

La preuve de la proposition~\vref{p:varset-notmin} suit la méthode de
\TS : on construit d'abord une forme auxiliaire ayant un indice élevé le long
de \( \divi \), puis on montre qu'elle s'annule en \( \ex \) avec un indice
plus faible mais toujours élevé par rapport à son degré et on conclut en
appliquant une variante du théorème du produit pour obtenir la conclusion de
la proposition.

Auparavant, nous aurons besoin de préciser un système de coordonnées adapté à
\( \var \) et nous présenterons une écriture réduite des formes sur \( \var \)
dans ce système de coordonnées. Nous présenterons ensuite un plongement
éclatant associé aux poids \( \wt* \) qui permettra
d'exploiter~\eqref{e:hautn-wt-diff}.


\subsection{Plongements projectifs adaptés}
\label{sec:plong-adapt}

\begin{tdef} \label{d:plong-adapt}
  Suivant \cite{remivg}, si \( \anyvar \) est une variété de dimension
  \( \anydim \) et \( \projd \) un espace projectif muni de coordonnées
  homogènes \( \anyvp[0], \dots, \anyvp** \), on dit qu'un plongement
  \( \iota\colon \anyvar \embedin \projd \) est adapté si
  \begin{enumthm}
    \item \( \anyvar \cap \zeros{\anyvp[0], \dots, \anyvp[\anydim]}
        = \emptyset \) ;
    \item \( \korper{\anyvar} \) est engendré par
      \( \frac{\anyvp[1]}{\anyvp[0]}, \dots,
        \frac{\anyvp[\anydim+1]}{\anyvp[0]} \) ;
    \item \( \frac{\anyvp[\anydim+1]}{\anyvp[0]} \neq 0 \) dans \(
        \korper\anyvar \).
  \end{enumthm}
\end{tdef}

Le fait suivant, qui ne fait que rappeler \cite[partie~4.1, p.~114]{remivds},
explicite les principales propriétés d'un plongement adapté, qui est
une version plus précise de la mise en position de \bsc{Noether}.

\begin{fact} \label{f:plong-adapt-gen}
  Si \( \anyvar \) de dimension \( \anydim \)  est plongée dans \( \projd \)
  de façon adaptée, alors les fonctions rationnelles
  \( \frac{\anyvp[1]}{\anyvp[0]}, \dots, \frac{\anyvp[\anydim]}{\anyvp[0]} \)
  forment une base de transcendance de \( \korper\anyvar \) sur \( \cdn \). De
  plus, \( \frac{\anyvp[\anydim+1]}{\anyvp[0]} \) est un élément primitif de
  \( \korper\anyvar \) sur \( \cdn( \frac{\anyvp[1]}{\anyvp[0]}, \dots,
    \frac{\anyvp[\anydim]}{\anyvp[0]} ) \).

  La projection linéaire \( \anyvar \to \proj\anydim \) obtenue en ne gardant
  que les \( \anydim + 1 \) premières variables est un revêtement fini,
  éventuellement ramifié.
\end{fact}

On contrôle en fait des relations de dépendance intégrale des dernières
variables sur la base de transcendance choisie.

\begin{fact} \label{f:plong-adapt-dep}
  Si le plongement \( \iota \colon \anyvar \embedin \projd \) est adapté, il
  existe des formes homogènes \( \poldep[][\ind] \) pour \( \ind \in
    \set{\anydim+1, \dots, n} \) telles que :
  \begin{enumthm}
    \item \(
        \poldep[][\ind]
        \in
        \cdn [ \anyvp[0], \dots, \anyvp[\anydim], \anyvp[\ind] ]
        \cap \ideal\anyvar \) ;
    \item \( \poldep[][\ind] \) est unitaire en \( \anyvp[\ind] \), de degré
      \( \anydeg \) ;
    \item \( \nv1{ \poldep[][\ind] } \le \nv1{ \chow\anyvar } \) ;
  \end{enumthm}
  où \( \anydeg \) est le degré de \( \anyvar \) dans ce plongement.
\end{fact}

\begin{proof}
  Le lemme~4.1 de \cite{remivds} donne explicitement des formes satisfaisant
  les deux premières conditions.

  Seule l'assertion sur la norme n'y est pas énoncée sous cette forme mais
  elle vient en remarquant que \( \poldep_\ind \) est une spécialisation de
  \( \varfc \) qui annule certaines variables et remplace les autres par des
  monômes unitaires.
\end{proof}

Nous établissons maintenant une variante de la proposition~4.1, de
\cite{remivds} qui montre qu'il est possible de rendre adapté un plongement
donné tout en gardant fixe le diviseur \( \divi \), à peu de frais.

\begin{lem} \label{l:adapt-gen}
  Soit \( \iota \colon \anyvar \embedin \projd \) un sous-schéma fermé intègre
  de degré \( \anydeg \), non contenu dans l'hyperplan d'équation \( \anyvp[0]
    = 0 \).  Il existe une transformation linéaire \( \chi \in
    \GL_{n+1}(\Q) \), représentable par une matrice à coefficients entiers
  de valeur absolue (archimédienne) majorée par \( \max(\frac\anydeg2, 1) \),
  telle que \( \chi \circ \iota \) est un plongement adapté à \( \anyvar \) et
  que
  \( \anyvp[0] \) soit invariant par ce changement de coordonnées.
\end{lem}

\begin{proof}
  On reprend la preuve de la proposition citée (p.~116) ; au moment de choisir
  des formes linéaires \( L_0, \dots, L_n \) telles que
  \begin{equation*}
    \chow \anyvar (L_0, \dots, L_n) \neq 0
    \pmm,
  \end{equation*}
  on commence en fait par fixer \( L_0 = \anyvp[0] \). Le polynôme \(
    \varfc(L_0, \truc, \dots, \truc) \) est multihomogène de degré \( \anydeg
  \) en chaque variable ; vu l'hypothèse sur \( \anyvar \), il est non nul
  grâce au théorème fondamental de l'élimination. On peut donc choisir \( L_1,
    \dots L_\anydim \) comme dans \cite{remivds} puis continuer la preuve sans
  autre modification.
\end{proof}

Nous aurons également besoin de contrôler \( \chi^{-1} \). Le lemme suivant
établit un résultat général élémentaire sur l'inversion de matrices.

\begin{lem} \label{l:cramer}
  Soit \( M \) une matrice \( p \times p \) inversible à coefficients entiers.
  On a alors \( \hautm\infty{M^{-1}} \le (p-1)! \cdot
    \hautm\infty{M}^{p-1} \).
\end{lem}

\begin{proof}
  On peut supposer que les coefficients de \( M \) sont premiers entre eux,
  de sorte qu'ils sont tous majorés en valeur absolue par \( \hautm\infty M \).
  On utilise alors les formules de \bsc{Cramer}. On constate d'abord que la
  contribution de l'inverse du déterminant s'élimine en prenant le produit sur
  toutes les places, de sorte qu'il s'agit d'estimer la hauteur de la
  comatrice. Cette dernière est à coefficients entiers, donc de norme
  inférieur à \( 1 \) aux places finies, et la valeur absolue archimédienne
  des coefficients est majorée par \( (p-1)! \, \hautm\infty{M}^{p-1} \).
\end{proof}

\begin{nota}
On fixe désormais des transformations linéaires \( \vadapt* \) obtenues en
appliquant le lemme~\vref{l:adapt-gen} à chacun des \( \var* \), de sorte que
\( \vadapt* \circ \vaemb \) est adapté à \( \var* \) et on note \(
  \vdegp* = \max(\vdeg*/2, 1) \).
\end{nota}

La construction et le lemme précédent montrent alors que
\begin{equation} \label{e:vadapt-ht}
  \nv\infty{ \vadapt* }
  \le
  (\vdegp*)^\dv
  \qquad\text{et}\qquad
  \hautm\infty{\vadapt*^{-1}}
  \le
  n! \cdot (\vdegp*)^{n}
\end{equation}
On note \( \varida* \) l'idéal homogène saturé de \( \var* \) dans le
plongement \( \vadapt* \circ \vaemb \) et \( \varfca* \) une forme de
\bsc{Chow} de \( \var* \) dans ces coordonnées. Le théorème de l'élimination
montre qu'une telle forme s'obtient en composant \( \varfc* \) avec (la
transposée de) \( \vadapt* \) sur chacun des \( \vdim* + 1 \) groupes de
variables ; il est alors clair que
\begin{equation} \label{e:nv-varfca}
  \nv1{\varfca*} \le \nv1{\varfc*}
  \cdot \bigl( (n+1) \vdegp* \bigr)^{\vdeg* (\vdim* + 1) \dv}
  \pmm.
\end{equation}

Par ailleurs, pour \( \indi \in \set{\vdim*, \dots, n} \), on notera \(
  \poldep** \) \label{p:def-poldep} une relation de dépendance de \(
  \vmp*[\indi] \) sur \(
  \vmp*[0], \dots, \vmp*[\vdim*] \) telle que donnée par l'application du
fait~\vref{f:plong-adapt-dep} au facteur \( \var* \) dans le plongement \(
  \vadapt* \circ \vaemb \). L'estimation précédente et le fait cité donnent
une majoration des normes locales de ces polynômes, qu'on peut en fait
améliorer en
\begin{equation} \label{e:nv-poldep}
  \nv1{\poldep**} \le \nv1{\varfc*}
  \cdot ( 2 \vdegp* )^{\vdeg* (\vdim* + 1) \dv}
\end{equation}
en utilisant le début de la démonstration du lemme~4.2 de~\cite{remivds} (la
suite de la démonstration étant moins pertinente compte tenu des différences
de normes utilisées). Cependant, nous utiliserons souvent la majoration plus
évidente \( \nv1{\poldep**} \le \nv1{\varfca*} \) dans les estimations où
cette quantité apparaît déjà par ailleurs.

\begin{nota} \label{n:pden-cexa}
  On note \( \pden* \) la dérivée de \( \poldep*[\vdim*+1] \) par rapport à la
  dernière variable.  Par ailleurs, on fixe \( (\cex**)_\ind \) un système de
  coordonnées multihomogènes de \( \ex* \) dans le plongement \( \vaemb \) et
  on note \( (\cexa**)_\ind \) l'image de ce dernier par \( \vadapt* \).
\end{nota}

Nous pouvons maintenant énoncer les cas particuliers à exclure dans la
démonstration de la proposition~\vref{p:varset-notmin}.

\begin{sco} \label{s:part-cases}
  Dans la démonstration de la proposition~\vref{p:varset-notmin}, on peut
  supposer que :
  \begin{enumthm}
    \item \( \cexa** \neq 0 \) pour tous \( \fct \) et \( \ind \in \set{1,
          \dots, \vdim*} \) ;
    \item \( \pden*(\cexa*) \neq 0 \) pour tout \( \fct \).
  \end{enumthm}
  Remarquons qu'on a déjà \( \cexa*[0] \neq 0 \) d'après
  l'hypothèse~\eqref{e:Vapx}.
\end{sco}

\begin{proof}
  Si le premier point n'est pas satisfait, pour un certain \( (\fct, \ind) \),
  on peut prendre \( T = \vmp** \) dans la conclusion de la
  proposition ; en effet cette forme s'annulle en \( \ex \) mais pas
  identiquement sur \( \var* \) (car le plongement est adapté) et les
  conditions de degré et de hauteur sont largement satisfaites.

  Si le deuxième point est faux pour un certain \( \fct \) , on choisit cette
  fois \( T = \pden* \) qui convient également : en effet, le caractère adapté
  du plongement implique que \( \poldep*[\vdim*+1] \) est irréductible et de
  degré \( \vdeg* \) en \( \vmp*[\vdim*+1] \) donc \( \pden* \) étant de degré
  plus petit, ne peut pas s'annuler identiquement sur \( \var* \).
\end{proof}


\subsection{Réduction de formes sur une variété plongée de façon adaptée}
\label{sec:rfull}

Les propriétés des plongements adaptés permettent d'associer à chaque forme
homogène, à peu de choses près, une représentation canonique modulo \( \varida
\), définie par des restrictions de degrés en certaines variables.
Introduisons pour cela quelques notations : pour chaque \( C \in (\N \cup
  \set{ +\infty }) ^{m(n+1)} \), on notera
\begin{equation} \label{e:C-spaces}
  \cdn[\vmp]^C
  = \{
    H \in \cdn[\vmp]
    \text{ tel que }
    \deg_{\vmp*[k]} H \le C\pexp{\fct}[k]
    \quad \forall i, k
    \}
  \pmm.
\end{equation}
Pour tout \( \Delta \in \N^m \) on définit trois tels vecteurs \(
  C'_\Delta \), \( C'' \) et \( C'''_\Delta \) par :
\begin{gather} \label{e:C-i-iii}
  (C'_\Delta)\pexp\fct[\ind] =
  \begin{cases}
    +\infty & \text{si \( \ind \le \vdim* \)} \\
    \Delta_\fct - 1 & \text{sinon}
  \end{cases}
  \qquad
  (C'')\pexp\fct[\ind] =
  \begin{cases}
    +\infty & \text{si \( \ind \le \vdim* + 1 \)} \\
    0 & \text{sinon}
  \end{cases}
  \\[1em]
  (C'''_\Delta)\pexp\fct[\ind] = \min\bigl(
    (C'_\Delta)\pexp\fct[\ind], (C'')\pexp\fct[\ind]
  \bigr)
  \pmm,
\end{gather}
pour \( 0 \le \fct \le m \) et \( 0 \le k \le n \).  Par ailleurs, on
omettra l'indice \( \Delta \) lorsqu'il est égal à \( \vdeg \) (le vecteur des
degrés de \( \var \)) pour alléger les notations.

L'intérêt de \( \cdn[ \vmp ]^{C'''} \) est que, le plongement étant
adapté, son intersection avec \( \varida \) est réduite à \( 0 \) : en effet,
\( \vmp*[\vdim*+1] \) est de degré \( \vdeg* \) sur les variables précédentes,
il n'est donc pas possible qu'un polynôme de degré en \( \vmp*[\vdim*+1] \)
strictement inférieur à \( \vdeg* \) (et ne dépendant pas des variables
suivantes) soit dans \( \varida \) sans être nul.
Par ailleurs, on peut facilement vérifier que les dimensions des parties
homogènes de degré \( \beta \), notées \( \cdn[ \vmp ]^{C'''}_{\beta} \) et \(
  ( \cdn[\vmp] / \varida)_\beta \) sont données par des polynômes en \( \beta
\) ayant le même terme dominant, de sorte que le morphisme de réduction, qui
est injectif, n'est pas très loin d'être un isomorphisme en degré assez grand.

Nous allons maintenant expliciter, en chaque multidegré \( \beta \in \N^m
\), une application linéaire \( \rfull^\beta \) définie sur la partie homogène
\( \cdn[ \vmp ]_\beta \) et à valeurs dans \( \cdn[ \vmp ]^{C'''} \), qui
permet par exemple de déterminer l'appartenance à \( \varida \) d'une forme
homogène.

L'idée générale est la suivante : on peut exploiter les relations de
dépendance des dernières variables sur les premières données par le
fait~\vref{f:plong-adapt-dep} pour réduire le degré en les dernières
variables, et le deuxième point de la définition~\vref{d:plong-adapt} permet
même d'éliminer totalement les variables d'indice strictement supérieur à \(
  \vdim* + 1 \), à condition de multiplier par un certain dénominateur
intervenant dans ces relations de dépendance rationnelles.

Il sera essentiel par la suite que ce dénominateur puisse être choisi
indépendamment du degré considéré ; pour cela l'application \( \rfull \)
sera la composée des trois étapes suivantes : une application \( \rdiv \)
arrivant dans \( \cdn[ \vmp ]^{C'} \), c'est-à-dire faisant chuter le degré en
les dernières variables par division euclidienne, une application \( \relim \)
arrivant dans \( \cdn[ \vmp ]^{C''} \), c'est-à-dire éliminant
les toutes dernières variables en faisant éventuellement croître le degré en
\( \vmp*[\vdim*+1] \), et une dernière application \( \rdiv \) pour limiter à
nouveau ce degré et arriver dans \( \cdn[ \vmp ]^{C'''} \).

Avant de procéder à la construction des applications évoquées, soulignons
qu'il s'agit d'applications linéaires définies sur chaque partie homogène de
\( \cdn[ \vmp ] \) mais en aucun cas de morphismes d'algèbres, les espaces
d'arrivée n'étant eux-mêmes pas des algèbres.

Pour construire l'application \( \rdiv \) on commence par énoncer un résultat
de réduction modulo des relations de dépendance intégrale sous une forme un
peu générale avant de l'appliquer au cas qui nous intéresse.

\begin{lem}
  Pour \( \fct \in \{ 1, \dots, m \} \) et \( \ind \in \{ \vdim* + 1,
  \dots, n \} \), on se donne :
  \begin{enumthm}
    \item \( \Delta_\fct \in \N^* \) ;
    \item \( P\mexp*[\ind]
      \in
      \cdn [ \vmp*[0], \dots, \vmp*[\vdim*], \vmp*[\ind] ] \)
      homogène de degré \( \Delta_\fct \) et unitaire en \( \vmp*[\ind]
      \).
  \end{enumthm}
  On note \( N_\fct = \max_\ind \nv1 { P\mexp*[\ind] } \) et \( \Ideal \)
  l'idéal engendré par les \( P\mexp*[\ind] \). En tout multidegré
  \( \beta \in \N^m \), il existe une (unique) application linéaire
  \begin{equation}
    \rdiv^\beta \colon \cdn [\vmp]_ \beta \to \cdn [\vmp]_ \beta^{C'_\Delta}
  \end{equation}
  qui est l'identité modulo \( \Ideal \) (voir~\eqref{e:C-spaces}
  et~\eqref{e:C-i-iii} pour la définition de l'espace d'arrivée). De plus les
  colonnes \( c_p \) de la matrice de cette application dans les bases de
  monômes satisfont la majoration de norme
  \begin{equation}
    \nv1{c_p}
    \le
    \prod\fctrange \bigl(
    N_\fct \cdot (2 \Delta_\fct)^\dv
    \bigr) ^{\beta_\fct}
  \end{equation}
  pour tout \( p \in \N^{m(n+1)} \) de multilongueur \( \beta \), et
  l'image de \( \cdn [\vmp]_ \beta ^{C''} \) par \( \rdiv^\beta \) est
  contenue dans \( \cdn [\vmp]_ \beta ^{C'''_\Delta} \).
\end{lem}

\begin{proof}
  C'est essentiellement une variante du lemme~2.5 de~\cite{remivg}, notre
  résultat étant formulé différemment et dans un cadre d'apparence un peu
  moins générale ; la preuve suivra en tout cas les mêmes lignes. On commence
  par décomposer chaque \( P\mexp*[\ind] \) de la façon suivante :
  \begin{equation}
    P\mexp*[\ind]
    =
    (\vmp**)^{\Delta_\fct}
    +
    \sum _{\alpha=1}^{\Delta_\fct}
    P\mexp*[\ind, \alpha] \cdot (\vmp*[\ind])^{\Delta_\fct - \alpha}
    \pmm,
  \end{equation}
  où \( P\mexp*[\ind, \alpha] \in \cdn [ \vmp*[0], \dots, \vmp*[\vdim*] ] \).
  On a alors \(
    (\vmp**)^{\Delta_\fct}
    \equiv
    - \sum _{\alpha=1}^{\Delta_\fct}
    P\mexp*[\ind, \alpha] \cdot (\vmp*[\ind])^{\Delta_\fct - \alpha}
    \mod \Ideal \) et plus généralement, le lemme~2.4 de \cite{remivg} fournit
  pour tout \( q \in \N \) et tout \( \alpha \in \set{1, \dots,
      \Delta_\fct} \) un polynôme \( U_{q, \alpha, \Delta_\fct} \) tel que
  si l'on pose, pour \( \fct \in \set{1, \dots, m} \) et \(
    \ind \in \set{ 0, \dots, n } \) :
  \begin{equation}
    R_{\fct, \ind, q}
    =
    \begin{dcases*}
      ( \vmp*[\ind] ) ^q
      & si \( \ind \le \vdim* \) ;
      \\
      \sum _{\alpha=1}^{\Delta_\fct}
      U_{q, \alpha, \Delta_\fct}
      ( P\mexp*[\ind, 1], \dots, P\mexp*[\ind, \Delta\mexp*] )
      (\vmp**)^{\Delta_\fct - \alpha}
      & sinon,
    \end{dcases*}
  \end{equation}
  alors \( R_{\fct, \ind, q} \equiv ( \vmp*[\ind] ) ^q \mod \Ideal \) et
  on a l'estimation de norme
  \begin{equation}
    \nv1{ R_{\fct, \ind, q} }
    \le
    \left(
    \nv1{ P\mexp*[\ind] } (2\Delta_\fct)^\dv
    \right) ^q
    \pmm.
  \end{equation}
  On définit alors \( \rdiv^\beta \) par son action sur les monômes, en posant
  pour tout \( p \in \N^{m(n+1)} \) :
  \begin{equation}
    c_p
    = \rdiv^\beta(\vmp^p)
    = \prod\fctrange \prod\indrange R_{\fct, \ind, \gmp p**}
  \end{equation}
  et en prolongeant par linéarité. L'estimation de norme annoncée découle
  directement de la majoration précédente en prenant le produit.

  Par ailleurs, il est clair que si une forme ne fait intervenir que les
  variables \( \vmp*[\ind] \) pour \( \ind \le \vdim* + 1 \), il en est de
  même de son image. En effet, \( \rdiv^\beta \) consiste à substituer, dans
  chaque monôme, le facteur en \( \vmp** \) (pour \( \ind > \vdim* \)) par un
  polynôme en \( \vmp*[0], \dots, \vmp*[\vdim*] \) et ne peut donc pas
  introduire de variables \( \vmp** \) pour \( \ind > \vdim* \).
\end{proof}

\begin{coro} \label{c:hmat-rdiv}
  Pour tout \( \alpha \in \N^m \), il existe une application linéaire \(
    \rdiv^\alpha \) égale à l'identité modulo \( \varida \), dont la matrice
  dans la base monomiale canonique a des colonnes de normes \( \nv1\truc \)
  majorées par
  \begin{equation}
    \prod\fctrange \left(
    \nv1{ \varfc* }
    2 B^{(B(g+1) + 1)\dv}
  \right) ^{ \alpha_\fct }
  \pmm.
  \end{equation}
\end{coro}

\begin{proof}
  Découle directement du lemme précédent en utilisant~\eqref{e:nv-poldep} et
  en remarquant que
  \begin{equation}
    ( 2 \vdegp* )^{\vdeg* (\vdim* + 1)} \cdot 2 \vdeg*
    \le
    2 B^{(B(g+1) + 1)}
  \end{equation}
  car d'après~\eqref{e:varset-deg}, \( B \) est un majorant commun des \(
    \vdeg* \) et donc de \( 2\vdegp* \), et pour tout \( \fct \) on a \(
    \vdim* \le g \).
\end{proof}

Intéressons-nous maintenant au morphisme \( \relim \).

\begin{lem}
  Soient, pour tout \( \fct \in \{ 1, \dots, m \} \) et tout \( \ind \in
  \{ \vdim* + 1, \dots n \} \), des formes
  \( S\mexp*[\ind] \in \cdn [ \vmp*[0], \dots, \vmp*[\vdim*] ] \) et
  \( T\mexp*[\ind] \in \cdn [ \vmp*[0], \dots, \vmp*[\vdim*+1] ] \)
  telles que \( \deg S\mexp*[\ind] + 1 = \deg T\mexp*[\ind] \) et des entiers
  \( \Delta_\fct \). On note \( \Ideal_{S, T} \) l'idéal engendré par les
  \( S\mexp*[\ind] \vmp*[\ind] - T\mexp*[\ind] \).

  Posons
  \( R = \prod\fctrange \prod_{\ind = \vdim*+1}^{n}
    ( S\mexp*[\ind] )^{\Delta_\fct} \) et \( r = \deg R \) ; la multiplication
  par \( R \) modulo \( \Ideal_{S, T} \) est alors représentée en tout degré
  \( \alpha \in \N^m \) par une application linéaire
  \begin{equation}
    \relim^\alpha \colon
    \cdn [\vmp]_ {\alpha}^{C'_\Delta}
    \to
    \cdn [\vmp]_ {\alpha+r}^{C''}
  \end{equation}
  telle que les colonnes \( c_q^{\relim} \) de sa matrice dans les bases de
  monômes satisfont
  \begin{equation}
    \nv1{ c_q^{\relim} }
    \le
    \prod\fctrange N_\fct^{\Delta_\fct}
  \end{equation}
  pour tout \( q \) de multilongueur \( \alpha \), où \( N_\fct \) majore
  \( \nv1{ S\mexp*[\ind] } \) et \( \nv1{ T\mexp*[\ind] } \) pour tout
  \( \ind \).
\end{lem}

\begin{proof}
  Soit \( R \) défini comme dans l'énoncé, et \( \vmp^q \) un monôme de
  l'espace de départ. Par hypothèse, \( q\mexp*[\ind] < \Delta_\fct \) pour
  tout \( \fct \) et \( \ind \le \vdim* + 1 \), de sorte que l'on peut poser
  \begin{equation}
    \relim( \vmp^q )
    =
    \prod\fctrange \left(
    \prod_{\ind=1}^{\vdim*}
    (\vmp*[\ind])^{q\mexp*[\ind]}
    \prod_{\ind=\vdim*+1}^n
    (T\mexp*[\ind])^{q\mexp*[\ind]}
    (S\mexp*[\ind])^{\Delta_\fct - q\mexp*[\ind]}
    \right)
  \end{equation}
  et prolonger par linéarité. On vérifie immédiatement que \( \relim(\vmp^q)
  \) est congru à \( R \cdot \vmp^q \) modulo \( \Ideal_{S, T} \), de même
  que l'estimation de norme annoncée.
\end{proof}

\begin{coro} \label{c:hmat-relim}
  Il existe une forme \( R \in \cdn [\vmp]^{C''} \) ne dépendant que de \(
    \var \) et n'appartenant pas à \( \varida \), et une application linéaire
  \begin{equation}
    \relim^\alpha \colon
    \cdn [\vmp]_{\alpha}^{C'}
    \to
    \cdn [\vmp]_{\alpha+r}^{C''}
  \end{equation}
  qui est la multiplication par \( R \) modulo \( \varida \).  De plus, les
  colonnes de la matrice de \( \relim^\alpha \) dans les bases monomiales
  canoniques ont leur norme \( \nv1\truc \) majorée par
  \begin{equation}
    \prod\fctrange
    (N'_\fct)^{\vdeg*}
    \pmm,
  \end{equation}
  où \( N'_\fct \) est une constante ne dépendant pas de \( \alpha \).
\end{coro}

\begin{proof}
  Il suffit d'établir l'existence de familles \( S \) et \( T \) comme dans
  l'énoncé du lemme précédent, telles que \( \Ideal_{S, T} \subset \varida \)
  et \( S \notin \varida \) ;
  elle découle du fait qu'on a utilisé un plongement adapté.
  En effet, d'après le fait~\vref{f:plong-adapt-gen}, pour tous \( \fct \) et
  \( \ind \), il existe des formes \( A\mexp*[\ind, \beta] \) et \(
    B\mexp*[\ind, \beta] \) dans
  \( \cdn [ \vmp*[0], \dots, \vmp*[\vdim*] ] \) telles que \(
    B\mexp*[\ind, \beta] \notin \varida \) et
  \begin{equation}
    \frac{ \vmp** }{ \vmp*[0] }
    =
    \sum_{\beta = 0}^{\vdeg* - 1}
    \frac {A\mexp*[\ind, \beta]} {B\mexp*[\ind, \beta]}
    \left( \frac{ \vmp*[\vdim* + 1] }{ \vmp*[0] } \right) ^\beta
    \quad\text{dans \( \cdn(\var) = \mathrm{Frac}( \cdn[\vmp] / \varida ) \).}
  \end{equation}
  On obtient alors les familles \( S \) et \( T \) recherchées en multipliant
  les deux membres de l'égalité précédente par \( \vmp*[0] \) puis en
  réduisant au même dénominateur le membre de droite ; ce dénominateur commun
  \( S \) n'appartient évidemment pas à \( \varida \). La forme \( R \)
  recherchée est alors celle fournie par le lemme précédent.
\end{proof}

On note désormais \( R \) la forme donnée par le lemme précédent et \( r \in
  \N^m \) son multidegré. Notons qu'on a pas besoin d'en savoir plus sur
cette forme, à part le fait qu'elle n'appartient pas à \( \varida \) et ne
dépend que de \( \var \) mais pas du degré \( \alpha \) considéré. En effet,
en pratique on fera tendre ce dernier vers l'infini ; c'est en ce sens qu'il
faut comprendre la notation \( o(\alpha) \) dans le lemme suivant, qui résume
la construction de \( \rfull^\alpha \) à partir des briques précédentes.

\begin{lem} \label{l:rfull}
  Pour tout multidegré \( \alpha \in \N^m \) il existe une application
  linéaire
  \begin{equation}
    \rfull^\alpha \colon
    \cdn [\vmp]_ {\alpha}
    \to
    \cdn [\vmp]_ {\alpha+r}^{C'''}
  \end{equation}
  qui est égale à la multiplication par \( R \) modulo \( \varida \). De plus,
  les colonnes de sa matrice dans les bases monomiales canoniques sont de
  norme \( \nv\infty\truc \) majorée par
  \begin{equation}
    \prod\fctrange \left(
      \nv1{ \varfc* }
      2 B^{(B(g+1) + 1)\dv}
    \right) ^{ 2\alpha_\fct }
    \cdot \expb^{o(\alpha)}
    \pmm.
  \end{equation}
  En particulier, on a \( \ker \rfull^\alpha = \cdn[ \vmp ]_\alpha \cap
    \varida \).
\end{lem}

\begin{proof}
  Il suffit d'utiliser les résultats des lemmes précédents et de poser
  \begin{equation}
    \rfull^\alpha
    =
    \rdiv^{\alpha+r} \circ \relim^{\alpha} \circ \rdiv^\alpha
    \pmm.
  \end{equation}
  L'assertion sur la norme est immédiate en remarquant que, si \( M_1 \) et \(
    M_2 \) sont deux matrices, la norme \( \nv\infty\truc \) de leur produit
  est majorée par \( \nv\infty{M_1} \max(\nv1{c}) \) où \( c \) parcourt les
  colonnes de \( M_2 \). On absorbe par ailleurs les constantes ne dépendant
  pas de \( \alpha \), à savoir le \( r \) et la norme de \( \relim \), dans
  le \( o(\alpha) \).

  Pour le noyau, considérons une forme \( H \) de multidegré \( \alpha \). On
  constate d'une part que si \( \rfull^\alpha(H) = 0 \), alors \( R H \in
    \varida \) donc \( H \in \varida \) car ce n'est pas le cas de \( R \) et
  que \( \varida \) est premier. Réciproquement, si \( H \in \varida \), alors
  \( RH \in \varida \) et \( \rfull^\alpha(H) \in \varida \cap \cdn[ \vmp
    ]^{C'''} = \set0 \).
\end{proof}

Notons qu'en prenant la somme, on peut définir une application linéaire \(
  \rfull \) sur \( \cdn[\vmp] \) entier ; on utilisera cette notation quand il
ne sera pas utile de préciser le degré.


\subsection{Plongement abélien pondéré}
\label{sec:wemb}

Introduisons un plongement, dit \emph{éclatant} ou pondéré par \( \wt =
  (\wt[1], \dots, \wt**) \), défini par
\begin{alignat}{2} \label{e:def-wemb}
  \wemb \colon && \var
  & \longto \va^m \times \va^{m-1}
  = \va^{2m-1}
  \\ &&
  (x_1, \dots, x_m)
  & \longmapsto
  (x_1, \dots, x_m;
  \wt[1] x_1 - \wt** x_m, \dots,
  \wt[m-1] x_{m-1} - \wt** x_m)
  \pmm.
\end{alignat}
qui nous permettra d'exploiter la relation~\eqref{e:hautn-wt-diff}.

Nous allons représenter ce morphisme par des familles de polynômes. Pour
cela, commençons par préciser les plongements utilisés : au départ, chaque
facteur \( \var* \) est plongé par \( \vadapt* \circ \vaemb \) ; on utilise
ces mêmes plongements pour les \( m \) premiers facteurs \( \va \) de
l'espace d'arrivée et on garde le plongement \( \vaemb \) pour les \( m-1
\) facteurs \( \va \) restants.

On munit l'espace \( (\projd)^{2m-1} \) (dans lequel est plongé l'espace
d'arrivée) des coordonnées multihomogènes \( \vmp, \vmpi = \vmp[1], \dots,
  \vmp[m], \vmpi[1], \dots, \vmpi[m-1] \) ; dans ce contexte quand \(
  \fct \) et \( \fcti \) sont deux indices non précisés, on supposera
implicitement \( 1 \le \fct \le m \) et \( 1 \le \fcti \le m-1 \).

Une représentation de \( \wemb \) dans ces plongements, définie sur un ouvert
\( \clmap \) de \( \va^m \), est un morphisme \( \wemba \) tel que le
diagramme suivant, dont les flèches verticales sont les projections
canoniques, commute.
\begin{equation}
  \xymatrix{
    \cdn [\vmp, \vmpi]                      \ar [r] ^{\wemba}   \ar [d]
    & \cdn [\vmp]                                               \ar [d] ^\pi
    \\ \cdn [\vmp, \vmpi] / \idealim \ar [r] ^{\wemb^*}
    & ( \cdn [\vmp] / \varida )_\clmap
  }
\end{equation}
En bas, \( \idealim \) désigne l'idéal multihomogène de l'image
dans les coordonnées choisies et \( ( \cdn [\vmp] / \varida )_\clmap \)
désigne la localisation de l'anneau des coordonnées au départ correspondant à
l'ouvert \( \clmap \).

On dispose (section~\vref{sec:vaemb}) d'un atlas \( \clmaps \) de \( \va^2 \)
avec sur chaque carte une représentation locale de \( (x, y) \mapsto \alpha x
  - \beta y \) dans le plongement \( \vaemb \). Voyons comment en déduire un
atlas de \( \va^m \) et des représentations locales de \( \wemb \) dans
les plongements considérés.

Soit \( \clmap = (\clmap_1, \dots, \clmap_{m-1}) \in \clmaps^{m-1}
\). On lui associe un ouvert de \( \va^m \), que par abus on notera
encore \( \clmap \), défini par \( \bigcap\fctirange p_\fcti^{-1}(
  \clmap_\fcti ) \) où \( p_\fcti \) désigne la projection sur les facteurs \(
  \fcti \) et \( m \). On vérifie sans peine qu'on obtient ainsi un
atlas de \( \va^m \), qu'on notera encore \( \clmaps^{m-1} \).
On introduit alors pour chaque \( \fcti \) la forme
\begin{equation} \label{e:def-wembclm}
  \wembclm*
  =
  \wembclp{\clmap}{\fcti}
  \bigl( \vadapt[\fcti]^{-1}(\vmp[\fcti]), \vadapt**^{-1}(\vmp[m]) \bigr)
\end{equation}
où \( \wembclp{\clmap}{\fcti} \) est donnée par la
section~\vref{sec:vaemb}, de sorte que chaque application
\begin{align} \label{e:def-wemba}
     \wemba* \colon \cdn [\vmp, \vmpi]
  &  \to \cdn[\vmp]
  \\ \vmp*
  &  \mapsto \vmp*
  \\ \vmpi*
  &  \mapsto \wembclm*( \vmp*, \vmp[m] )
\end{align}
est une représentation locale de \( \wemb \) dans les plongements considérés,
valable sur l'ouvert \( \clmap \) (ou plus précisément, son image dans le
plongement utilisé).

Étudions maintenant l'action de \( \wemba* \) sur le degré et la hauteur.  Le
lemme suivant est une conséquence immédiate de la définition.

\begin{lem} \label{l:deg-wemba}
  Soit \( H \in \Qbar[\vmp, \vmpi] \) une forme multihomogène de multidegré \(
    (\alpha, \beta) \) où \( \alpha \in \N^m \) et \( \beta \in
    \N^{m-1} \). On a alors
  \begin{equation}
    \deg \wemba*(H)
    =
    \bigr(
    \alpha_1 + 2 \beta_1 \wts[1],
    \dots,
    \alpha_{m-1} + 2 \beta_{m-1} \wts[m-1],
    \alpha_m + 2 \lgr\beta \wts[m]
    \bigl)
    \pmm.
  \end{equation}
\end{lem}

Avant de passer aux estimations de hauteur, il est utile d'introduire quelques
paramètres\footnote{Paramètres dont les définitions peuvent paraître
  énigmatiques à ce stade ; voir la section~\vref{sec:vojta-adjust} pour une
  meilleure appréciation de ces choix.} qui seront utilisés tout au long de ce
chapitre et qui contrôlent notamment le degré en lequel nous établirons ces
estimations.

Introduisons un réel \( \epsi \) défini par :
\begin{equation} \label{e:def-epsi}
  \epsi
  =
  \eps^{\frac{g}{m-g}}
  \cdot \Bigl(
    N^{m-1}
    \, (85 \cdot 5^m)^g
  \Bigr)^{\frac{-1}{m-g}}
\end{equation}
et un rationnel \( \epsz \) tel que :
\begin{equation} \label{e:def-epsz}
  \frac{\eps \epsi}{33m}
  \le \epsz \le
  \frac{\eps \epsi}{32m}
  \pmm,
\end{equation}
et enfin un entier \( \delta \) (destiné à tendre vers l'infini) tel que \(
  \epsz \delta \) soit également entier. On pose alors
\begin{equation} \label{e:d-dp-def}
  \begin{aligned}
    d & = \bigl(
      \epsz \wts[1],
      \dots,
      \epsz \wts[m],
      1, \dots, 1
    \bigr) \in \Q^{2m-1}
    \\
    d' & = \bigl(
      \wts[1] (2 + \epsz),
      \dots,
      \wts[m-1] (2 + \epsz),
      \wts[m] (2m - 2 + \epsz)
    \bigr) \in \Q^m
  \end{aligned}
\end{equation}
de sorte que, si \( H \) est une forme de multidegré \( \Dz \) dans \(
  \cdn[\vmp, \vmpi] \), son image par \( \wemba* \) est de multidegré
exactement \( \Di \) d'après le lemme~\vref{l:deg-wemba}. Pour alléger les
notations par la suite, on introduit le vecteur \( \wtw = (1, \dots, 1, m
  - 1) \in \N^m \), de sorte qu'on a \( d'_\fct = \wts* (2 \wtw* + \epsz)
\) pour tout \( \fct \).

\medskip

Revenons donc au calcul de l'action de \( \wemba* \) sur la hauteur. Ce
morphisme étant homogène, il induit d'après la remarque ci-dessus une
application linéaire de \( \cdn [\vmp, \vmpi]_\Dz \) dans \( \cdn [\vmp]_\Di
\).  La base évidente de l'espace de départ est formée par les monômes \(
  \vmp^p \vmpi^q \) pour
\begin{equation}
  (p, q)
  \in \N^{m (n+1)} \times \N^{(m-1) (n+1)}
  \text{ tel que }
  \lgr{p\mexp*} = \Dz* = \delta \epsz \wts*
  \text{ et }
  \lgr{q\mexpi*} = \Dz_{m + \fcti} = \delta
  \pmm.
\end{equation}

\begin{lem} \label{l:hmat-wemba}
  Avec les notations précédentes, l'application linéaire
  \begin{equation}
    \wemba* \colon
    \cdn [\vmp, \vmpi]_\Dz
    \to
    \cdn [\vmp]_\Di
  \end{equation}
  est représentée dans les bases canoniques de monômes par une matrice dont
  les colonnes
  \(
  c_{p, q} = \wemba*(\vmp^p\vmpi^q)
  = \sum_{\lgr s = \Di - \lgr p} c_{p, q; s} \vmp^{s+p}
  \)
  satisfont
  \begin{align}
    \hautm1{c_{p, q}}
    & \le
    \bigl(
      \hmclab ((n+1)!)^2 \, B^{2n}
    \bigr) ^{2 \delta \wts[1]}
    \pmm,
  \end{align}
  où l'on rappelle que \( \hmclab \) est donné par la
  section~\vref{sec:vaemb}.
\end{lem}

\begin{proof}
  La définition de \( \wemba* \) montre que \(
  \wemba*( \vmp^p \vmpi^q ) = \vmp^p \wemba*(\vmpi^q) \), on a donc \(
  \nv1 {c_{p, q}} = \nv1 {\wemba*(\vmpi^q)} \). Ainsi, on a
  \begin{align}
    \nv1 {c_{p, q}}
    & \le
    \prod\fctirange \prod\indrange
    \nv1{ \wembclp{\clmap}{\fcti}[\ind] \bigl(
        \vadapt[\fcti]^{-1}(\vmp[\fcti]), \vadapt**^{-1}(\vmp[m])
      \bigr) }
    ^{q\pexp\fcti[\ind]}
    \\ & \le
    \prod\fctirange \prod\indrange \left(
      \nv1{ \wembclp{\clmap}{\fcti}[\ind] } \,
      \nv1{ \vadapt[\fcti]^{-1} }^{2\wtis*} \,
      \nv1{ \vadapt**^{-1} }^{2\wts**}
    \right) ^{q\pexp\fcti[\ind]}
    \\ & \le
    \prod\fctirange \left(
      \hmclab*^{\wtis* + \wts**} \,
      \nv1{ \vadapt[\fcti]^{-1} }^{2\wtis*} \,
      \nv1{ \vadapt**^{-1} }^{2\wts**}
    \right) ^\delta
  \end{align}
  en remarquant que, par définition de \( d \), on a \( \lgr{q\mexpi*} =
    \delta \) pour tout \( \fcti \), puis en
  utilisant la section~\vref{sec:vaemb}. En prenant le
  produit sur toutes les places (et en prenant la racine \( \delta \)-ième
  pour simplifier l'écriture) il vient, compte tenu de~\eqref{e:vadapt-ht} et
  de~\eqref{e:varset-deg}
  \begin{align}
    \hautm1{c_{p, q}}^{1/\delta}
    & \le
    \prod\fctirange
    \hmclab^{\wtis* + \wts**} \,
    \bigr( (n+1)! \, (\vdegp[\fcti] )^n) \bigl)^{2 \wts[\fcti]} \,
    \bigr( (n+1)! \, (\vdegp[m])^n) \bigl)^{2 \wts[m]}
    \\ & \le
    \prod\fctirange
    \bigr( \hmclab ((n+1)!)^2 \, B^{2n} \bigr)^{\wtis* + \wts**}
    \pmm.
  \end{align}
  Le résultat annoncé suit en invoquant le lemme~\vref{l:sum-wts}.
\end{proof}

Énonçons de suite une variante qui nous sera utile par la suite.

\begin{coro} \label{c:ht-wemba}
  Soient \( (\clmapv)_v \) une famille de cartes indexée
  par les places de \( \cdn \) et \( F \in \cdn[\vmp, \vmpi] \) une forme
  homogène. Alors
  \begin{equation}
    \prod_v \nv1{ \wemba[\clmapv](F) }^\degv
    \le
    \hautm1 F
    \bigl(
      \hmclab ((n+1)!)^2 \, B^{2n}
    \bigr) ^{2 \delta \wts[1]}
    \pmm.
  \end{equation}
\end{coro}

\begin{proof}
  Pour le cas où \( F \) est un monôme, on reprend point par point la preuve
  précédente, en remplaçant simplement \( \wembclp{\clmap}{\fcti}[\ind] \) par
  \( \wembclp{\clmapv}{\fcti}[\ind] \) au début, différence qui disparaît au
  moment où l'on majore la norme de ces formes par \( \hmclab* \), car cette
  majoration ne dépend pas de la carte utilisée.

  Le cas général se déduit de celui des monômes par linéarité.
\end{proof}



\section{Construction d'une forme auxiliaire} \label{sec:siegel}

L'objectif de cette section est de construire une forme non nulle sur \( \var
\), provenant d'une forme sur \( \wemb(\var) \), de degré prescrit, de hauteur
contrôlée, et d'indice élevé le long de \( \divi \) dans \( \var \).  Nous
commencerons par définir la notion d'indice utilisée et préciser les
propriétés voulues et la stratégie de construction, puis nous établirons les
estimations de dimension et de hauteur nécessaires avant de conclure en
appliquant un lemme de \TS.


\subsection{Stratégie de construction de la forme auxiliaire}
\label{sec:siegel-plan}

Commençons par définir l'indice d'annulation (par rapport à un vecteur de
poids \( b \in \N^m \)) d'une forme le long de \( \divi \) . (Une autre
notion d'indice, en un point cette fois-ci, sera définie de façon similaire au
début de la sous-section~\vref{sec:vojta-extrap-core}.) Pour tout \( \imp \in
  \N^{m(n+1)} \), notons
\begin{equation}
  \wtsum[b]*( \imp )
  =
  \frac {\imp[1][0]} { b_1 } + \dots
  + \frac {\imp[m][0]} { b_m }
  \pmm.
\end{equation}
Pour toute forme \( H = \sum h_\imp \vmp^\imp \) et tout \( \beta > 0 \), on
pose
\begin{equation} \label{e:def-pi-beta}
  \pi_\beta^b (H)
  =
  \sum_{\wtsum[b]*( \imp ) < \beta}
  h_\imp \vmp^\imp
\end{equation}
et on définit l'indice par
\begin{equation} \label{e:inda-def}
  \indg{b}* H
  =
  \inf \set{
    \beta \text{ tel que } \pi_\beta^b (H) \neq 0
  }
  \pmm,
\end{equation}
où la borne inférieure est en fait un minimum, sauf si \( H = 0 \), cas où
l'indice est infini. À partir de maintenant, nous utiliserons l'indice \(
  \inda* \) pondéré par le vecteur \( \wtw \wts = (\wtw[1] \wts[1], \dots,
  \wtw** \wts**) \), qui est essentiellement proportionnel à \( d' \), ce
qui sera crucial pour la section~\vref{sec:thm-prod}.

\medskip

Au début de cette section, nous avons dit vouloir construire une forme sur \(
  \var \) provenant d'une forme sur \( \wemb(\var) \) ; plus précisément nous
allons construire une forme \( \faux \) sur \( \wemb(\var) \) et la tirer
localement en arrière sur \( \var \) en une famille de formes \(
  \faux''_\clmap \) grâce aux différents morphismes \( \wemba* \) introduits à
la sous-section~\vref{sec:wemb}.

Mais on ne peut brutalement exiger
que leur indice, tel que défini précédemment, soit élevé, car le nombre de
conditions serait trop élevé (géométriquement, il s'agirait d'exiger une
annulation le long de \( \divi \) dans \( \projd \) alors que les degrés de
libertés sont donnés par \( \var \) qui est de dimension plus petite). Nous
allons donc utiliser le morphisme \( \rfull \) introduit à la
sous-section~\vref{sec:rfull} et exiger que \( \faux* = \rfull(\faux''_\clmap)
\) ait un indice élevé, ce qui est loisible car le nombre des équations
définissant un indicé élevé dans \( \cdn[\vmp]^{C'''} \) est convenable (le
nombre de variables libre dans cet espace correspondant à la dimension de \(
  \var \)) comme le montrera la prochaine sous-section.

Concernant le degré, on reprend les multidegrés définis par~\eqref{e:d-dp-def}
et on rappelle que \( \delta \) désigne un grand entier tel que \( \delta
  \epsz \) soit aussi entier. On impose alors à \( \faux \) d'être
de degré \( \Dz \), de sorte que les \( \faux''_\clmap \) seront de degré \(
  \Di \) et les \( \faux* \) de degré \( \Dir = \Di + o(\delta) \). Par
ailleurs, l'indice exigé sera \( \epsi \delta \), où \( \epsi \) est défini
par~\eqref{e:def-epsi} et la relation entre \( \epsz \) et \( \epsi \)
permettra d'ajuster l'exposant de \bsc{Dirichlet} du système auquel on
appliquera le lemme de \TS.

Choisissons donc dans \( \cdn[\vmp, \vmpi]_\Dz \) un supplémentaire de \(
  (\idealim)_\Dz \) engendré par des monômes, que l'on notera \( \fspace \) et
dans lequel on cherchera \( \faux \).  L'entier \( \delta \) sera choisi assez
grand pour que la dimension de \( \fspace \) coïncide avec la valeur du
polynôme de \bsc{Hilbert} de \( \idealim \) en \( \Dz \).

Reformulons l'objectif en introduisant un morphisme \( \sigma \) défini par
\begin{align} \label{e:def-ftarget}
  \sigma \colon \fspace
  & \to
  \ftarget =
  \bigoplus_{\clmap \in \clmaps^{m-1}} \bigl(
    \cdn[\vmp]^{C'''}_\Dir
    \cap \vect( (\vmp^\imp)_{\wtsum*(\imp) < \epsi\delta} )
  \bigr)
  \\
  \faux
  & \mapsto
  \bigl(
    \pi_{\epsi\delta}^{\wtw\wts} \circ \rfull^{\Di} \circ \wemba*(\faux)
  \bigr)_\clmap
\end{align}
de sorte qu'il s'agit en fait de trouver une forme non nulle dans le noyau de
\( \sigma \). Pour procéder nous devons donc :
\begin{enumerate}
  \item Estimer les dimensions des espaces de départ et d'arrivée de \( \sigma
    \) et s'assurer que la seconde est plus petite que la première ;
  \item Estimer la hauteur de la matrice de \( \sigma \) dans les bases
    monomiales évidentes.
\end{enumerate}
Ces estimations font l'objet des deux sous-sections suivantes.  Comme \(
  \delta \) est arbitrairement grand devant les autres paramètres, on
n'explicitera à chaque fois que le terme de plus haut degré en \( \delta \) ;
ainsi lorsqu'on utilisera la notation \( o(\truc) \) ou \( \sim \), il s'agira
d'équivalents quand \( \delta \) tend vers l'infini.


\subsection{Deux calculs de dimension} \label{sec:comp-dim}

La dimension de \( \fspace \) est donnée par le théorème de \bsc{Hilbert}
multihomogène dès qu'on connaît les différents multidegrés de
\( \wemb(\var) \). Il est \lat{a priori} difficile de tous les calculer, mais
comme il suffit en fait de minorer la dimension, le lemme suivant nous donne
tout ce qu'on aura besoin de savoir sur le degré.

\begin{lem}
  Avec les notations précédentes, on a
  \begin{equation}
    \deg_{(0, \dots, 0, \vdim[m]; \vdim[1], \dots, \vdim[m-1])}
    \bigl( \wemb(\var) \bigr)
    =
    \vdeg[m]
    \prod\fctirange
    \vdeg[\fcti] \wt[\fcti]^{2\vdim[\fcti]}
    \pmm.
  \end{equation}
\end{lem}

\begin{proof}
  Ce degré est donné par le cardinal de l'intersection de \( \wemb(\var) \)
  avec des hyperplans génériques choisis de la façon suivante : \(
    \vdim[m] \) provenant du \( m \)-ième facteur \( \projd \), et
  \( \vdim* \) hyperplans provenant du facteur \( m + \fct \) pour \(
    \fct \in \set{1, \dots, m-1} \).

  On commence par choisir les hyperplans sur le facteur \( m \) : on
  remarque qu'ils se remontent par \( \wemb \) en des hyperplans sur le
  dernier facteur de l'espace de départ \( (\projd)^m \). Ainsi, couper
  \( \wemb(\var) \) par ces hyperplans revient à imposer à \( x_m \)
  de parcourir un ensemble de cardinal \( \vdeg[m] \).

  Fixons maintenant un point \( p' \) dans cet ensemble et notons \( p = \wt**
    p' \).  On constate que \( \wemb(\var) \cap \zeros{\vmp[m] = p'} \)
  coïncide avec l'image de
  \begin{align}
    \wemb[\wt, p']'
    \colon
    \var[1] \times \dots \times \var[m-1] \times \set{p'}
    & \to
    \va^{2m-1}
    \\
    (x_1, \dots, x_{m-1}, p')
    & \mapsto
    (x_1, \dots, x_{m-1}, p';
    \wt[1] x_1 - p,
    \dots,
    \wt[m-1] x_{m-1} - p)
  \end{align}
  qui est le produit d'un point par des variétés de la forme \( \wemb[\wt*,
    p]''(\var*) \) pour \( \fct \) variant de \( 1 \) à \( m-1 \) en
  notant
  \begin{align}
    \wemb[\wt*, p]''
    \colon
    \var*
    & \to
    \va^2
    \\
    x
    & \mapsto
    (x, \wt* x - p)
  \end{align}
  Il suffit donc de calculer le degré de ces variétés. La translation par \( p
  \) n'ayant pas d'influence sur le degré, il suffit de regarder l'action de
  la multiplication par \( \wt* \). Or, celle-ci pouvant être représentée
  globalement par des formes de degré \( \wts* \), en tirant en arrière par
  \( \wemb[\wt*, p]'' \) une famille de \( \vdim* \) hyperplans
  génériques sur le second facteur, on obtient des hypersurfaces génériques de
  degré \( \wts* \) qui coupent donc \( \var* \) en \( \vdeg* \wt*^{2\vdim*}
  \) points.

  Le résultat suit en prenant le produit.
\end{proof}

\begin{lem} \label{l:dim-fspace}
  Avec les notations précédentes, on a
  \begin{align}
    \dim \fspace
    \ge
    \frac{ \epsz^{\vdim[m]}
      \prod\fctrange \vdeg* \, \wt* ^{2\vdim*}
      }{ \prod\fctrange \vdim* ! }
    \delta^\vdim
    + o( \delta^\vdim )
    \pmm.
  \end{align}
\end{lem}

\begin{proof}
  Il suffit d'appliquer le théorème de \bsc{Hilbert} multihomogène en
  utilisant le lemme précédent, car une somme de nombre positifs est minorée
  par chacun de ses termes. Il vient
  \begin{align}
    \dim \fspace
    & =
    \Biggl(
    \sum_{\substack{ t \in \N^{2m-1} \\ \vlg t = \vlg u }}
    \deg_t \wemb(\var) \frac{ d^t }{ t! }
    \Biggr)
    \delta^\vdim
    + o( \delta^\vdim )
    \\
    & \ge
    \deg_{(0, \dots, 0, \vdim[m]; \vdim[1], \dots, \vdim[m-1])}
    \bigl( \wemb(\var) \bigr)
    \cdot
    \frac { (\epsz \wts**)^{\vdim[m]} }{ \prod\fctrange \vdim* ! }
    \, \delta^\vdim
    + o( \delta^\vdim )
  \end{align}
  par définition de \( d \), voir~\eqref{e:d-dp-def}. On achève la preuve en
  combinant avec le résultat du lemme précédent.
\end{proof}

Il reste à majorer la dimension de \( \ftarget \).  On introduit à cet effet
l'ensemble suivant :
\begin{equation} \label{e:stairs-c3}
  \begin{split}
    \stairs
    & =
    \Biggl\{
      ( \imp[1], \dots, \imp[m] )
      \in
      \prod\fctrange \bigl(
        \N^{\vdim* + 1}
        \times \{ 0, \dots, \vdeg* - 1 \}
        \times \{ 0 \}^{n - \vdim* - 1}
      \bigr)
      \\ & \qquad
      \text{tel que }
      \wtsum*(\lambda) < \delta \epsi
      \text{ et }
      \lgr{\imp*}
      = \Dir* \quad \forall \fct
    \Biggr\}
    \pmm,
  \end{split}
\end{equation}
dont il s'agit de calculer le cardinal. En effet, la famille \(
  (\vmp^\imp)_{\imp \in \stairs} \) forme une base de
\(
  \cdn[\vmp]^{C'''}_\Dir
  \cap \vect( (\vmp^\imp)_{\wtsum*(\imp) < \epsi\delta} )
\)
et \( \ftarget \) est somme directe de tels espaces.

\begin{lem}
  Avec les notations précédentes,
  \begin{align}
    \card \stairs
    & \le
    \delta^\vdim
    \prod\fctrange \vdeg* \wt*^{2\vdim*}
    \cdot
    \frac {
      \epsi^m
      \, m^{\vdim[m]}
      \, 3^{\vdim - m}
    }{
      m!
      \, \prod\fctrange (\vdim* - 1)!
    }
    + o(\delta^\vdim)
  \end{align}
\end{lem}

\begin{proof}
  Pour choisir un indice \( \imp \) dans \( \stairs \), on peut commencer par
  choisir \( \imp*[\vdim*+1] \) entre \( 0 \) et \( \vdeg* - 1 \) pour tout \(
    \fct \), ce qui représente \( \prod\fctrange \vdeg* \) possibilités.

  On peut ensuite choisir des entiers \( \imp[1][0], \dots \imp[m][0] \)
  sujets à la seule condition
  \begin{equation}
    \wtsum*(\lambda) < \delta \epsi \pmm.
  \end{equation}
  Le lemme~2.14.5 de \cite{farhith} donne le nombre de choix possibles, qui
  est
  \begin{equation}
    \frac {\prodwt} {m !} (\delta\epsi)^m
    + o(\delta^m)
    \pmm.
  \end{equation}

  Il reste alors à choisir pour tout \( \fct \) un élément de l'ensemble
  \begin{equation}
    \left\{
      (\imp*[1],  \dots, \imp*[\vdim*])
      \in \N ^{\vdim*}
      \text{ tel que }
      \sum_{j=1}^{\vdim*} \imp*[j]
      =
      \Dir* - \imp*[0] - \imp*[\vdim*+1]
    \right\}
    \pmm,
  \end{equation}
  qui est de cardinal
  \begin{align}
    \binom {
      \Dir* - \imp*[0] - \imp*[\vdim*+1] + \vdim* - 1
      }{
      \vdim* - 1
      }
    & \le
    \binom {
      \Dir* + \vdim* - 1
      }{
      \vdim* - 1
      }
    \\
    & \le
    \frac {(d'_\fct)^{\vdim* - 1}} {(\vdim* - 1)!} \delta^{\vdim* - 1}
    + o( \delta^{\vdim* - 1} )
    \pmm.
  \end{align}
  On prend alors le produit :
  \begin{align}
    \card \stairs
    & \le
    \frac {\prodwt} {m !} (\delta\epsi)^m
    \cdot \prod\fctrange
    \frac {(d'_\fct)^{\vdim* - 1}} {(\vdim* - 1)!}
    \vdeg* \delta^{\vdim* - 1}
    + o( \delta^\vdim )
    \\ & \le
    \delta^\vdim
    \prod\fctrange \vdeg* \wt*^{2\vdim*}
    \cdot
    \frac {
      \epsi^m (m-1)
      (2m - 2 + \epsz) ^{\vdim[m]-1}
      (2 + \epsz) ^{\vdim - m - \vdim[m] + 1}
    }{
      m! \prod\fctrange (\vdim* - 1)!
    }
    + o(\delta^\vdim)
  \end{align}
  d'après la définition~\eqref{e:d-dp-def} de \( d' \). Le résultat annoncé
  suit en remarquant que \( 2 + \epsz \le 3 \) et \( 2m - 2 + \epsz \le
    3m \).
\end{proof}

Compte tenu de la définition~\eqref{e:def-ftarget} de \( \ftarget \) et du
fait que \( \card \clmaps^{m-1} = \nclmaps^{m-1} \), on a
immédiatement
\begin{equation}
  \dim \ftarget
  \le
  \delta^\vdim
  \prod\fctrange \vdeg* \wt*^{2\vdim*}
  \cdot
  \frac {
    \epsi^m
    \, \nclmaps^{m-1}
    \, m^{\vdim[m]}
    \, 3^{\vdim - m}
  }{
    m!
    \, \prod\fctrange (\vdim* - 1)!
  }
  + o(\delta^\vdim)
  \pmm.
\end{equation}
En combinant ce résultat avec le lemme~\vref{l:dim-fspace} et en observant que
\( \vdim* \le g \), on a
\begin{align}
  \frac {\dim \ftarget} {\dim \fspace}
  & \le
  \frac {
    \epsi^m
    \, \nclmaps^{m-1}
    \, m^{\vdim[m]}
    \, 3^{\vdim - m}
    \, \prod\fctrange \vdim*
  }{
    \epsz^{\vdim[m]}
    \, m!
  }
  + o(1)
  \\ & \le
  \frac {
    \epsi^m
    \, \nclmaps^{m-1}
    \, m ^{g}
    \, 3 ^{m (g-1)}
    \, g^m
  }{
    \epsz^g
    \, m!
  }
  + o(1)
  \pmm.
\end{align}
Or, nous allons prouver dans un instant que
\begin{equation} \label{e:ct-eps*-siegel}
  \frac {
    \epsi^m
    \, \nclmaps^{m-1}
    \, m ^{g}
    \, 3 ^{m (g-1)}
    \, g^m
  }{
    \epsz^g
    \, m!
  }
  \le
  \frac12
\end{equation}
ce qui nous donne immédiatement
\begin{equation} \label{e:good-codim}
  \frac {\dim \ftarget} {\dim \fspace}
  \le
  \frac12
  + o(1)
\end{equation}
et montre que le noyau de l'application \( \sigma \) définie
par~\eqref{e:def-ftarget} est de dimension strictement positive et permet même
de contrôler l'exposant de \bsc{Dirichlet} lorsque nous appliquerons le lemme
de \TS.

Prouvons maintenant notre assertion~\eqref{e:ct-eps*-siegel} :
on utilise la définition \eqref{e:def-epsz} de \( \epsz \) et la minoration
classique \( m! \ge m^m \expb^{-m} \) pour écrire
\begin{align}
  \frac {\dim \ftarget} {\dim \fspace}
  & \le
  \frac {
    \epsi^{m-g}
    \, \nclmaps^{m-1}
    \, m ^{g}
    \, 3 ^{m (g-1)}
    \, g^m
    \, (33m)^g
    \, \expb^m
  }{
    \eps^g
    \, m^m
  }
  + o(1)
  \\ & \le
  \frac 1 { m^{m-g} }
  \cdot
  \frac {
    \epsi^{m-g}
    \, \nclmaps^{m-1}
    \, 3 ^{m g}
    \, g^m
    \, (33m)^g
  }{
    \eps^g
  }
  + o(1)
  \\ & \le
  \frac 1 { m^{m-g} }
  \cdot
  \frac {
    \epsi^{m-g}
    \, \nclmaps^{m-1}
    \, (85 \cdot 5^m)^g
  }{
    \eps^g
  }
  + o(1)
\end{align}
où l'on a utilisé les faits élémentaires (quitte à vérifier numériquement pour
les petites valeurs) \( g \cdot 3^g \le (13/3)^g \) puis \( 33
  m \cdot (13/3)^m \le 85 \cdot 5^m \).  On remarque que la
définition \eqref{e:def-epsi} de \( \epsi \) signifie précisément que le
deuxième facteur vaut \( 1 \) ; il suffit maintenant d'observer que \( m
  \ge g + 1 \ge 2 \) pour conclure.


\subsection{Hauteur du système}
\label{sec:siegel-ht}

Il s'agit d'estimer la hauteur de la matrice (dans les bases monomiales
canoniques) \( M_\sigma \) de l'application \( \sigma \) définie
par~\eqref{e:def-ftarget}. Nous utiliserons ici la norme \( \nv\infty\truc \)
en chaque place, de sorte que la norme de la matrice de \( \sigma \) sera
majorée par tout majorant commun des normes des matrices des applications
\( \pi_{\epsi\delta}^{\wtw\wts} \circ \rfull^{\Di} \circ \wemba* \).

D'après la définition~\eqref{e:def-pi-beta} de \( \pi_{\epsi\delta}^{\wtw\wts}
\) il est clair que cette application ne peut que faire décroître la norme. On
pose alors \( \sigma'_\clmap = \rfull^{\Di} \circ \wemba* \) ; il s'agit donc
de majorer \( \hautl\infty{ M_{\sigma'_\clmap} } \) indépendamment de \(
  \clmap \).

On utilise le fait élémentaire suivant : si \( M_1 \) et \( M_2 \) sont deux
matrices, la hauteur \( \Hautl\infty\) de leur produit est majorée par \(
  \hautl\infty{M_1} + \max(\hautl1{c}) \) où \( c \) parcourt les colonnes
de \( M_2 \). Il suffit alors de combiner les estimations des
lemmes~\vref{l:rfull} et~\vref{l:hmat-wemba} pour obtenir
\begin{align}
  \hautl\infty{ M_{\sigma'_\clmap} }
  & \le
  2 \delta \wts[1]
  \bigl(
    \hlclab + 2\ln((n+1)!) + 2n \ln B
  \bigr)
  \\ & \qquad +
  \sum\fctrange 2\Di* \Bigl(
    \hautl1{ \varfc* }
    + \bigl( B(g+1) + 1 \bigr) \ln(B) + \ln(2)
  \Bigr)
  + o(\delta)
  \pmm.
\end{align}
On remarque pour commencer que, comme \( \epsz < 1 \) on a \( d'_\fct = 2\wtw*
  \wts* + \epsz \le 3 \wtw* \wts* \) ; le lemme~\vref{l:sum-wts} implique
alors que \( \sum_\fct d'_\fct \le 6 \wts[1] \). On utilise de plus la
définition~\eqref{e:cst-vs-ht} pour écrire
\begin{align}
  \hautl\infty{ M_{\sigma'_\clmap} }
  & \le
  4 \delta \wts[1]
  \Bigl(
    \cst{vs-ht}/2 + \ln((n+1)!) + n \ln B
    + 3 \ln(B) \bigl( B(g+1) + 1 \bigr) + 3 \ln(2)
  \Bigr)
  \\ & \qquad
  + 6 \delta \sum\fctrange \wtw* \wts* \hautl1{ \varfc* }
  + o(\delta)
  \pmm.
\end{align}
On invoque alors~\eqref{e:B-siegel} pour conclure :
\begin{equation} \label{e:hmat-sigma-p}
  \hautl\infty{ M_{\sigma'_\clmap} }
  \le
  \delta \wts[1] \cst{vs-ht} \Lambda^{(1 + \frac1m) f(\vdim)}
  + 6 \delta \sum\fctrange \wtw* \wts* \hautl1{ \varfc* }
  + o(\delta)
  \pmm.
\end{equation}
Compte tenu des remarques précédentes, on a donc aussi
\begin{equation} \label{e:hmat-sigma}
  \hautl\infty{ M_\sigma }
  \le
  \delta \wts[1] \cst{vs-ht} \Lambda^{(1 + \frac1m) f(\vdim)}
  + 6 \delta \sum\fctrange \wtw* \wts* \hautl1{ \varfc* }
  + o(\delta)
  \pmm.
\end{equation}
Remarquons que l'on pourrait invoquer l'hypothèse~\eqref{e:varset-ht} pour
simplifier cette dernière expression ; voire à ce sujet la remarque suivant le
scolie~\vref{s:aux-co}.


\subsection{Construction finale de la forme auxiliaire}

Commençons par rappeler la version que nous utiliserons du classique lemme de
\TS.

\begin{fact} \label{f:siegel}
  Pour toute matrice \( M \) de dimensions \( p \times q \) à coefficients
  dans un corps de nombres \( \cdn \) avec \( p < q \), il existe un vecteur
  \( x \in \cdn^q \) non nul tel que \( M x = 0 \) et satisfaisant
  \begin{equation}
    \hautl\infty x
    \le
    \frac p{q-p} \bigl( \hautl\infty M + \ln q \bigr)
    + \frac q{q-p} c_\cdn
    \pmm,
  \end{equation}
  où \( c_\cdn \) est une constante ne dépendant que de \( \cdn \).
\end{fact}

\begin{proof}
  C'est le lemme de \bsc{Siegel} de \bsc{Bombieri} tel qu'énoncé
  dans~\cite{bogf}. Notons que des versions plus précises sont disponibles,
  mais les améliorations ne concernent que des termes qui sont négligeables
  dans notre situation et n'ont donc pas d'intérêt ici.
\end{proof}

Nous sommes maintenant en mesure de construire la forme auxiliaire.

\begin{prop} \label{p:build-aux}
  Sous les hypothèses et notations précédentes et si
  \( \delta \) est assez grand, il existe une forme non nulle \( \faux \in
    \ker \sigma \) telle que
  \begin{align}
    \deg \faux
    & = \Dz
    = \bigl(
      \epsz \wts[1] \delta,
      \dots,
      \epsz \wts[m] \delta,
      \delta, \dots, \delta
    \bigr)
    \\
    \hautl\infty \faux
    & \le
    \delta \wts[1] \cst{vs-ht} \Lambda^{(1 + \frac1m) f(\vdim)}
    + 6 \delta \sum\fctrange \wtw* \wts* \hautl1{ \varfc* }
    + o(\delta)
    \pmm.
  \end{align}
\end{prop}

\begin{proof}
  On applique le fait~\vref{f:siegel} à la matrice \( M_\sigma \) introduite à
  la sous-section précédente, de dimensions \( p = \dim \ftarget \) et \( q =
    \dim \fspace \) estimées précédemment.  La relation~\eqref{e:good-codim}
  se lit alors \( \frac pq \le \frac12 + o(1) \), ce qui implique directement
  que \( p < q \) et que
  \begin{equation}
    \frac p {q-p}
    \le
    \frac 1 {\frac qp - 1}
    \le
    \frac 1 {1 - o(1)}
    \le
    1 + o(1)
    \pmm.
  \end{equation}
  De même, \( \frac q {q-p} c_\cdn = o(\delta) \) ;
  en remarquant de plus que \( \ln q = o(\delta) \), on voit qu'il existe une
  forme \( \faux \) satisfaisant aux conditions de l'énoncé mais de hauteur
  majorée par \( \hautl\infty{M_\sigma} + o(\delta) \).  Il ne reste plus
  qu'à invoquer~\eqref{e:hmat-sigma} pour conclure au résultat annoncé.
\end{proof}

Remarquons que dans la majoration de hauteur on peut en fait choisir la
hauteur utilisée, comme le montre le scolie suivant.

\begin{sco} \label{s:h1-aux}
  On a \( \hautl1 \faux \le \hautl\infty \faux + o(\delta) \).
\end{sco}

\begin{proof}
  La différence entre ces deux hauteurs est majorée par le logarithme du
  nombre de coefficients de \( \faux \), qui est au plus
  \begin{equation}
    \prod_{\fct=1}^{2m-1}
    \binom{\delta d_\fct + n}{n}
    \le
    \prod_{\fct=1}^{2m-1}
    \frac{ (\delta d_\fct + n)^n }{n!}
    \pmm.
  \end{equation}
  Cette quantité étant polynomiale en \( \delta \), son logarithme est
  négligeable devant~\( \delta \).
\end{proof}

Par la suite, nous travaillerons presqu'exclusivement avec les formes \(
  \faux* = \rfull \circ \wemba*(\faux) \), la forme \( F \) ne servant qu'à
assurer un lien global entre elles. Le scolie suivant résume toutes les
propriétés de ces formes dont nous aurons besoin.

\begin{sco} \label{s:aux-co}
  Pour toute carte \( \clmap \in \clmaps^{m-1} \), la forme \( \faux*
    \in \cdn[(\vmp*[0], \dots, \vmp*[\vdim+1])_\fct] \) satisfait :
  \begin{enumthm}
    \item \( \faux* = R(\vmp) \cdot \faux(\vmp, \wembclm(\vmp)) \mod \varida
      \) ;
    \item \( \inda* \faux* \ge \epsi \delta \) ;
    \item \( \deg \faux* = \delta d' + r \) avec
      \( d'_\fct = \wts* (2 \wtw* + \epsz) \) pour tout \( \fct \), et \( r \)
      indépendant de \( \delta \) ;
    \item \(
        \hautl1{\faux*}
        \le
        2 \delta \wts[1] \cst{vs-ht} \Lambda^{(1 + \frac1m) f(\vdim)}
        + 12 \delta \sum\fctrange \wtw* \wts* \hautl1{ \varfc* }
        + o(\delta)
      \) ;
  \end{enumthm}
  De plus, si \( (\clmapv)_v \) est une famille de cartes indexée
  par les places de \( \cdn \), on a
  \begin{equation} \label{e:aux-co-htv}
    \sum_v \degv \ln \nv1{ \faux[\clmapv] }
    \le
    14 \delta \wts[1] \cst{vs-ht} \Lambda^{(1 + \frac1m) f(\vdim)}
    + o(\delta)
    \pmm.
  \end{equation}
\end{sco}

\begin{proof}
  Les trois premiers points ne font que résumer la construction. Plus
  précisément, le premier découle de la définition de \( \faux* \), du
  lemme~\vref{l:rfull} et de la définition~\eqref{e:def-wemba} de \( \wemba*
  \) ; le deuxième est imposé par la construction et le troisième découle du
  lemme~\vref{l:deg-wemba} et des définitions~\eqref{e:d-dp-def} de
  \( d \) et \( d' \).

  Pour le quatrième point, on remarque que
  \( \faux* = \sigma'_\clmap( \faux ) \)
  dans les notations de la sous-section~\vref{sec:siegel-ht} de sorte qu'on a
  \begin{align}
    \hautl\infty{ \faux* }
    \le
    \hautl\infty{ M_{\sigma'_\clmap} }
    + \hautl1{ \faux }
    \le
    2 \delta \wts[1] \cst{vs-ht} \Lambda^{(1 + \frac1m) f(\vdim)}
    + 12 \delta \sum\fctrange \wtw* \wts* \hautl1{ \varfc* }
    + o(\delta)
  \end{align}
  en utilisant le résultat de la proposition précédente
  et~\eqref{e:hmat-sigma-p}. Le même argument qu'au scolie~\vref{s:h1-aux}
  montre que \( \hautl1{ \faux* } \) est en fait majoré par la même
  quantité.

  Pour la dernière assertion, on procède exactement comme pour majorer la
  hauteur de \( \faux* \) sauf qu'on utilise le corollaire~\vref{c:ht-wemba} à
  la place du lemme~\vref{l:hmat-wemba}, ce qui donne une majoration identique
  à la précédente :
  \begin{align}
    \sum_v \degv \ln \nv1{ \faux[\clmapv] }
    \le
    2 \delta \wts[1] \cst{vs-ht} \Lambda^{(1 + \frac1m) f(\vdim)}
    + 12 \delta \sum\fctrange \wtw* \wts* \hautl1{ \varfc* }
    + o(\delta)
  \end{align}
  Il ne reste qu'à appliquer l'hypothèse~\eqref{e:varset-ht} pour conclure.
\end{proof}

Remarquons qu'on a utilisé l'hypothèse~\eqref{e:varset-ht} pour simplifier la
dernière estimation contrairement à celle du quatrième point. En effet, cette
estimation de \( \hautl1{ \faux* } \) contribuera directement, en
section~\vref{sec:thm-prod}, à l'estimation de hauteur de la forme motrice
que nous cherchons à construire (proposition~\vref{p:varset-notmin}) et il est
donc intéressant de conserver la dépendance en \( \sum\fctrange \wtw* \wts*
  \hautl1{\varfc*} \) explicite jusqu'à la relation de
récurrence~\eqref{e:vs-ht-rec}.  L'autre estimation en revanche, ne sera
utilisée que dans la section suivante pour assurer que les termes ne dépendant
pas de \( \hautn{ \ex[1] } \) sont assez petits pour être couverts par cette
dernière quantité, compte tenu de l'hypothèse~\eqref{e:Vbig}.



\section{Extrapolation} \label{sec:vojta-extrap}

Le but de cette section est de montrer que la fonction auxiliaire que nous
venons de construire s'annulle avec un indice élevé en \( \ex \), pour une
définition de l'indice que nous préciserons.

Auparavant, nous étudierons les dérivées sur \( \var \) des fonctions
rationnelles obtenues en déshomogénéisant \( \faux* \) par \(
  (\vmp*[\ind_\fct])_\fct \) où \( 0 \le \ind_\fct \le \vdim* \), que nous
représenterons par des fractions rationnelles suffisamment bien contrôlées.
Pour commencer, nous effectuons ce travail sur une variété projective
quelconque (plongée de façon adaptée) avant de passer à une variété produit.


\subsection{Estimation de dérivées} \label{sec:vojta-param}

Soit \( \anyvar \) une variété projective de dimension \( \anydim \), plongée
de façon adaptée dans un espace projectif \( \projd \), de degré \( \anydeg \)
dans ce plongement. Alors \( \cdn(\anyvar) \) est une extension finie
de
\begin{equation}
  \cdn\Big(
    \frac{ \vp[1]         }{ \vp[0] }, \dots,
    \frac{ \vp[\anydim]   }{ \vp[0] }
  \Big)
\end{equation}
dont \( \frac{ \vp[\anydim+1] }{ \vp[0] } \) est un élément primitif.
Sur ce dernier corps, on dispose des dérivations standard définies par
\(
  \diff_\ind \frac{ \vp[\indi] }{ \vp[0] } = \delta_\ind^\indi
\)
qui forment une base de l'espace des dérivations, et s'étendent de façon
unique à \( \cdn(\anyvar) \) pour former une base de l'espace de ses
dérivations ; on pose alors
\begin{equation}
  \der[\derp]
  =
  \prod_{\ind = 0}^\anydim \frac1{\derp*!} \diff_\ind^{\derp*}
  \pmm.
\end{equation}

On cherche, pour tout \( \derp \) et pour certaines fonctions
rationnelles \( f \), à donner une représentation de \( \der[\derp] f \) sous
la forme \( G/H \) avec \( G \in \cdn[\vp] \) de degré et normes locales
contrôlés, et \( H \in \cdn[\vp] \setminus \ideal\anyvar \) totalement
explicite. Par \og représentation \fg, on entend qu'on
veut avoir \( \pi(G/H) = \der[\derp] f \) où \( \pi \) est la projection
naturelle de \( \cdn[\vp]_{(\ideal\anyvar)} \) sur \( \cdn(\anyvar) \). Pour
des raisons techniques, on construira en fait deux représentations, dont
on contrôlera respectivement les normes aux places finies ou infinies.

Pour l'application considérée, on peut se restreindre au cas où \( f \) est de
la forme \( P / \vp*^{\deg P} \) avec \( 0 \le \vp* \le \anydim \) et \( P \in
  \cdn[ \vp[0], \dots, \vp[\anydim+1] ] \). Par linéarité et en utilisant la
règle de \bsc{Leibniz}, il suffit pour commencer de considérer les fonctions
de la forme \( \vp[\indi] / \vp* \) pour \( 0 \le \indi \le \anydim+1 \) et \(
  0 \le \ind \le \anydim \). Le cas où \( \indi \neq \anydim + 1 \) est facile
et fait l'objet du lemme suivant, valable aussi bien dans \(
  \cdn[X]_{\ideal\anyvar} \) que dans \( \cdn(\var) \).

\begin{lem} \label{l:param-any-easy}
  Soit \( \derp \in \N^\anydim \) ; on note \( \supp \derp \) l'ensemble des
  indices \( \alpha \) tels que \( \derp[\alpha] \neq 0 \). Alors, pour tout
  \( (\ind, \indi) \in \set{0, \dots, \anydim}^2 \) tel que  \( \ind \neq
    \indi \) on a
  \begin{equation}
    \der[\derp] \frac{ \vp[\indi] }{ \vp* }
    =
    \begin{cases*}
      0
      & si \( \supp\derp \not\subset \set{\ind, \indi} \)
      ou \( \derp[\indi] > 1 \) ;
      \\
      (-1)^{\derp*}
      \left( \frac{ \vp[\indi] }{ \vp* } \right)^{1-\derp[\indi]}
      \left( \frac{ \vp[0] }{ \vp* } \right)^{\lgr\derp}
      & sinon.
    \end{cases*}
  \end{equation}
  Par ailleurs, si \( \ind = \indi \) alors
  \( \der[\derp] \frac{ \vp[\indi] }{ \vp* } \) vaut \( 0 \) si \( \derp \neq
    0 \) et \( 1 \) sinon.
\end{lem}

\begin{proof}
  Si \( \ind = 0 \), c'est la définition même de \( \der[\derp] \) ; si \(
    \ind = \indi \) le résultat est clair également. Sinon, en utilisant la
  règle de \bsc{Leibniz} et ses conséquences usuelles, on a facilement
  \begin{equation}
    \diff_\alpha
    \frac{ \vp[\indi] }{ \vp[0] }
    \left( \frac{ \vp* }{ \vp[0] } \right)^{-1}
    =
    \left( \frac{ \vp* }{ \vp[0] } \right)^{-2} \left(
      \frac{ \vp* }{ \vp[0] }
      \diff_\alpha \frac{ \vp[\indi] }{ \vp[0] }
      -
      \frac{ \vp[\indi] }{ \vp[0] }
      \diff_\alpha \frac{ \vp* }{ \vp[0] }
    \right)
    =
    \begin{cases*}
      - \frac{ \vp[\indi] }{ \vp[0] } \,
      \left( \frac{ \vp[0] }{ \vp* } \right)^2
      & si \( \alpha = \ind \) ;
      \\
      \frac{ \vp[0] }{ \vp* }
      & si \( \alpha = \indi \) ;
      \\
      0
      & sinon.
    \end{cases*}
  \end{equation}
  Si \( \supp\derp \not\subset \set{\ind, \indi} \), il existe un \( \alpha \)
  distinct de \( \ind \) et \( \indi \) tel que \( \der[\derp] \) contient un
  facteur \( \diff_\alpha \) et on a donc \(
    \der[\derp] \frac{ \vp[\indi] }{ \vp* } = 0
  \).
  De même, \( \derp[\indi] > 1 \) on a \(
    \der[\derp] \frac{ \vp[\indi] }{ \vp* } = 0
  \) car \(
    \diff_\indi \circ \diff_\indi \frac{ \vp[\indi] }{ \vp* }
    = \diff_\indi \frac{ \vp[0] }{ \vp* }
    = 0
  \).
  Ceci donne le premier cas ; pour le deuxième, on écrit
  \begin{align}
    \der[\derp] \frac{ \vp[\indi] }{ \vp[\ind] }
    & =
    \diff_\indi^{\derp_\indi} \frac{ \vp[\indi] }{ \vp[0] }
    \cdot
    \frac1{\derp*!}
    \diff_\ind^{\derp_\ind} \frac{ \vp[0] }{ \vp* }
    \\ & =
    \left( \frac{ \vp[\indi] }{ \vp[0] } \right)^{1 - \derp_\indi}
    \cdot
    (-1)^{\derp*}
    \left( \frac{ \vp[0] }{ \vp* } \right)^{\derp* + 1}
    =
    (-1)^{\derp*}
    \left( \frac{ \vp[\indi] }{ \vp[\ind] } \right)^{1 - \derp_\indi}
    \left( \frac{ \vp[0] }{ \vp* } \right)^{\derp[\ind] + \derp[\indi]}
  \end{align}
  qui donne bien le résultat annoncé.
\end{proof}

Pour le cas \( \indi = \anydim + 1 \), on utilisera des relations de
dépendance sur les autres variables, données par le fait que le plongement est
adapté. Le lemme suivant énonce de façon générale comment exploiter de telles
relations.

\begin{lem} \label{l:param-aff}
  Soient \( \vaf[1], \dots, \vaf[\anydim], \vaf[1]', \dots, \vaf[\anydim]',
    \vafi \) des variables et \( L \)
  une extension algébrique finie de \( \cdn(\vaf[1]', \dots, \vaf[\anydim]')
  \). On fixe \( \omega \) un élément de \( L \) ; on note \( \pi \) le
  morphisme de \( \cdn[\vaf[1], \dots, \vaf[\anydim], \vafi ] \) dans \( L \)
  qui envoie \( \vaf* \) sur \( \vaf*' \) et \( \vafi \) sur \( \omega \).
  Soit \( S \in \ker \pi \) ; on note \( R = \frac{ \partial S }{ \partial
      \vafi } \) et on suppose que \( R \notin \ker \pi \).

  On considère les dérivations standard \( \diff_\ind \) sur \( \cdn(
    \vaf[1]', \dots, \vaf[\anydim]' ) \) ainsi que leurs extensions à \( L \),
  et on note \( \der[\derp] = \frac1{\derp!} \prod_{\ind = 0}^\anydim
    \diff_\ind^{\derp*} \).  Il existe des polynômes \( P_\alpha^\derp \in
    \cdn[ \vaf[1], \dots, \vaf[\anydim], \vafi ] \), pour \( \derp \in
    \N^{\anydim} \minusset0 \) et \( \alpha \in \set{0, 1} \), tels que :
  \begin{enumthm}
    \item \( \der[\derp] \omega
        = \pi\left(
          \dfrac{ P_\dv^\derp }{ R^{2\lgr\dermp - 1} }
        \right)
      \) ;
    \item \( \deg_\vaf P_\alpha^\derp \le (2\lgr\derp - 1) \deg_\vaf S \) ;
    \item \( \deg_\vafi P_\alpha^\derp \le (2\lgr\derp - 1) \deg_\vafi S \) ;
    \item \( \nv1{ P_\dv^\derp }
        \le
        \nv1 S ^{2\lgr\derp - 1}
        \cdot \left(
          (8\anydim)^{\lgr\derp - 1} D^{3\lgr\derp -2}
        \right)^\dv \) où \( D = \max(\deg_\vaf S, \deg_\vafi S) \).
  \end{enumthm}
\end{lem}

\begin{proof}
  Il s'agit en fait de compléter\footnote{On contrôle en fait le développement
    autour d'un point générique, alors que \bsc{Rémond} l'étudie en un point
    fixé. Plus précisément, on pourrait retrouver l'énoncé de \bsc{Rémond} à
    partir de celui-ci par évaluation de \( P_\dv^\derp \) et \( R
    \) en un point convenable.} la preuve du lemme~6.1 de \cite{remivds}, en
  utilisant aux places finies une généralisation de la relation~2.3.1, p.~63
  de~\cite{farhith}.

  On va construire \( P_0^\derp \) et \( P_1^\derp \) indépendamment par
  récurrence sur la longueur de \( \derp \), en partant à chaque fois de \(
    P_\dv^\derp = - \frac{\partial S}{\partial \vaf[\ind_0]} \) quand \(
    \derp_{\ind} = \delta^\ind_{\ind_0} \) (cas \( \lgr\derp = 1 \)), car ce
  choix convient.  Pour la suite, on fixe un \( \dv \), un \( \derp \) de
  longueur au moins \( 2 \), et on suppose qu'on a choisi un \( P_\dv^{\derp'}
  \) convenable pour chaque \( \derp' \) de longueur strictement inférieure à
  celle de \( \derp \).

  On commence par le cas ultramétrique et on note donc provisoirement \(
    P^\derp = P_0^\derp \) pour alléger. Les polynômes recherchés sont
  caractérisés par la contrainte
  \begin{equation}
    S \left(
      \vaf[1] + \psp[1], \dots, \vaf[\anydim] + \psp[\anydim],
      \vafi + \sum_{ \derp \in \N^\anydim \minusset 0 }
      \frac {P^\derp} {R^{2\lgr\derp -1}} \psp^\derp
    \right)
    \in \ker \pi
    \pmm,
  \end{equation}
  où l'on a étendu \( \pi \) aux algèbres de séries en \( \psp[1], \dots,
    \psp[\anydim] \) au départ et à l'arrivée. Il suffit, pour satisfaire
  cette contrainte,  d'imposer que la série ci-dessus ait tous ses termes nuls
  sauf le premier qui est égal à \( S \). Le développement de \bsc{Taylor} de
  l'expression ci-dessus donne alors :
  \begin{align}
    0
    & =
    \sum_{ (\ip, \mu) \in \N^{\anydim+1} \minusset{(0, 0)} }
    \der[\ip, \mu] S
    \cdot \psp^\ip
    \cdot \left(
      \sum_{ \derp \in \N^\anydim \minusset 0 }
      \frac {P^\derp} {R^{2\lgr\derp - 1}} \psp^\derp
    \right)^\mu
    \\
    & =
    \sum_{\substack{ (\ip, \mu) \in \N^{\anydim+1} \minusset{(0, 0)}
        \\ \gmp\nu \in (\N^\anydim \minusset 0)^\mu }}
    \left(
      \der[\ip, \mu] S
      \cdot \prod_{\fct = 1}^\mu
      \frac {P^{\gmp\nu*}} {R^{2\lgr{\gmp\nu*} - 1}}
    \right)
    \psp^{\sum_\fct \gmp\nu* + \ip}
    \\
    & =
    \sum_{\derp \in \N^\anydim \minusset 0}
    \Biggl(
    \frac {P^\derp} {R^{2\lgr\derp - 2}}
    + \sum_{\substack{
        (\ip, \mu) \in \N^{\anydim+1} \minusset{(0, 0), (0, 1)}
        \\ \gmp\nu \in (\N^\anydim \minusset 0)^\mu
        \\ \sum_\fct \gmp\nu* + \ip = \derp }}
    \der[\ip, \mu] S
    \cdot \prod_{\fct = 1}^\mu
    \frac {P^{\gmp\nu*}} {R^{2\lgr{\gmp\nu*} - 1}}
    \Biggr)
    \psp^\derp
    \pmm,
  \end{align}
  où l'on a noté \( (\ip, \mu) = (\ip[1], \dots, \ip[\anydim], \mu) \) et \(
    \der[\ip, \mu] \) les dérivées divisées correspondantes dans \( \cdn[
    \vaf[1], \dots, \vaf[\anydim], \vafi ] \).
  On remarque alors que, sur l'ensemble de sommation, on a d'une part
  \( \sum_{\fct=1}^{\mu} (2 \lgr{\gmp\nu*} - 1) = 2 \lgr\derp - 2 \lgr\ip -
    \mu \) et d'autre part \( 2\lgr\lambda + \mu \ge 2 \) : en effet on a soit
  \( \lgr\lambda \ge 1 \) soit \( \mu \ge 2 \).

  On peut donc définir les polynômes \( P^\derp \)  par la relation de
  récurrence
  \begin{equation}
    - P^\derp
    =
    \sum_{\substack{
        (\ip, \mu) \in \N^{\anydim+1} \minusset{(0, 0), (0, 1)}
        \\ \gmp\nu* \in \N^\anydim \minusset 0
        \\ \sum_\fct \gmp\nu* + \ip = \derp }}
    \der[\ip, \mu] S
    \cdot \Bigl( \prod_{\fct = 1}^\mu P^{\gmp\nu*} \Bigr)
    \cdot R^{2\lgr\ip + \mu - 2}
    \pmm.
  \end{equation}
  On majore alors les degrés de \( P^\derp \) par récurrence :
  \begin{align}
    \deg_\vaf P^\derp
    & \le
    \deg_\vaf S - \lgr\ip + (2\lgr\derp - 2) \deg_\vaf S
    \le
    (2\lgr\derp - 1) \deg_\vaf S
    \\
    \deg_\vafi P^\derp
    & \le
    \deg_\vafi S - \mu + (2\lgr\derp - 2) \deg_\vafi S
    \le
    (2\lgr\derp - 1) \deg_\vafi S
    \pmm.
  \end{align}
  La majoration de norme locale est immédiate par analogie avec les degrés vu
  les propriétés de la norme aux places ultramétriques.

  Considérons maintenant le cas archimédien (désormais \( P^\derp = P^\derp_1
  \) pour alléger). On utilise la relation de récurrence suivante, établie
  dans la démonstration du lemme~6.1 de \cite{remivds} (haut de la page 139),
  avec \( Q_\derp = P^\derp \cdot \derp! \) et \( P = S \) où, rappelons-le,
  \( \derp' \) est tel que
  \( \derp[\ind_0] = \derp[\ind_0]' + 1 \) et \( \derp* = \derp*' \) sinon :
  \begin{align}
    Q_\derp
    & =
    R^2 \, \frac{ \partial Q_{\derp'} }{ \partial \vaf[\ind_0] }
    - R \, \frac{ \partial S }{ \partial \vaf[\ind_0] }
    \, \frac{ \partial Q_{\derp'} }{ \partial \vafi }
    \\ & \qquad
    + (2\lgr{\derp'} - 1) Q_{\derp'} \cdot \left(
      \frac{ \partial S }{ \partial \vaf[\ind_0] }
      \, \frac{ \partial R }{ \partial \vafi }
      - R \frac{ \partial R }{ \partial \vaf[\ind_0] }
    \right)
    \pmm.
  \end{align}
  On en déduit immédiatement les estimations de degré annoncées. En
  particulier, on utilise le fait que \( \max(\deg_\vaf Q_{\derp'}, \deg_\vafi
    Q_{\derp'}) \le (2\lgr\derp - 3) D \) pour l'estimation de norme :
  \begin{align}
    \nv1{ Q_\derp }
    & \le
    2 D^2 \nv1 S^2 \cdot (2\lgr\derp - 3) D \nv1{ Q_{\derp'} }
    + (2\lgr\derp - 3) \nv1{ Q_{\derp'} } \cdot 2 D^3 \nv1 S^2
    \\ & \le
    4 (2\lgr\derp - 3) D^3 \nv1 S^2 \nv1{ Q_{\derp'} }
    \le
    8 (\lgr\derp - 1) D^3 \nv1 S^2 \nv1{ Q_{\derp'} }
    \pmm.
  \end{align}
  On se souvient alors que dans le cas \( \lgr\derp = 1 \) on a \( \nv1{
      Q_\derp } \le \nv1 S D^\dv \) pour en déduire par récurrence que
  \(
    \nv1{Q_\derp}
    \le
    \nv1 S^{2\lgr\derp-1} 8^{\lgr\derp-1} D^{3\lgr\derp-2}
    (\lgr\derp - 1) !
  \) ;
  ceci implique le résultat annoncé vu la définition de \( Q_\derp \)
  ci-dessus et le fait que \( \binom{\lgr\derp}{\derp} \le
    \anydim^{\lgr\derp-1} \).
\end{proof}

\begin{nota} \label{n:pden-gen}
  On note \( \poldep[][\anydim+1] \) le polynôme donné par l'application du
  fait~\vref{f:plong-adapt-dep} à \( \anyvar \) avec \( \ind = \anydim+1 \) et
  \( Q = \frac{\partial \poldep[][\anydim+1]}{\partial \vp[\anydim+1]} \). Il
  est clair que \( \deg Q = \anydeg - 1 \) et
  \( \nv1 Q \le \nv1{ \chow\anyvar } \anydeg^\dv \).
\end{nota}

\begin{lem} \label{l:param-any+1}
  Avec la définition précédente, pour tous \( \ind \in \set{0, \dots, \anydim}
  \) et \( \derp \in \N^\anydim \minusset0 \), il existe des formes
  homogènes \( P_{\ind, \dv}^\derp \) telles que :
  \begin{enumthm}
    \item \(
        \der[\derp] \frac{ \vp[\anydim+1] }{ \vp* }
        =
        \frac{ P_{\ind, \dv}^\derp }{ (\vp*^{\anydeg+1} Q)^{2\lgr\derp} }
      \) dans \( \cdn(\anyvar) \) ;
    \item \(
        \deg P_{\ind, \dv}^\derp
        =
        4\anydeg \lgr\derp
      \) ;
    \item \(
        \nv1{ P_{\ind, \dv}^\derp }
        \le
        \nv1{ \chow\anyvar }^{2\lgr\derp}
        \cdot (8 \anydim \anydeg^3)^{\lgr\derp \dv}
      \).
  \end{enumthm}
\end{lem}

\begin{proof}
  Posons \( S_\ind = \poldep[][\anydim+1](1, \vaf[1], \dots, \vaf[\anydim],
    \vaf[\ind] \vafi) \) ; les degrés partiels en \( \vaf \) et \( \vafi \) de
  \( S_\ind \) sont tous deux majorés par \( \anydeg \), et la norme par celle
  de \( \chow\anyvar \).

  On souhaite appliquer le lemme précédent avec \( L = \cdn(\anyvar) \) en
  prenant pour \( \vaf[\indi]' \) et \( \omega \) les images respectives de \(
    \vp[\indi]/\vp[0] \) et \( \vp[\anydim+1]/\vp* \) dans ce dernier, ainsi
  que \( S = S_\ind \). Ceci est possible car on a
  \begin{align}
    \pi(R)
    & =
    \pi\left( \frac{ \diff S_\ind }{ \diff \vafi } \right)
    =
    \pi\bigl( \vaf* Q(1, \vaf[1], \dots, \vaf[\anydim], \vaf* \vafi) \bigr)
    \\ & =
    \frac{ \vp* }{ \vp[0] }
    Q\left(
      \frac{\vp[0]}{\vp[0]}, \dots, \frac{\vp[\anydim+1]}{\vp[0]}
    \right)
  \end{align}
  et qu'aucun de ces facteurs n'est nul dans \( \cdn(\anyvar) \) car le
  plongement est adapté.  On note \( \tilde P_{\ind, \dv}^\derp \) les
  polynômes obtenus.  On a alors, dans \( \cdn(\anyvar) \) :
  \begin{align}
    \der[\derp] \frac{ \vp[\anydim+1] }{ \vp* }
    & =
    \pi \left( \frac{
        \tilde P_{\ind, \dv}^\derp (\vaf[1], \dots, \vaf[\anydim], \vafi)
      }{
        R(\vaf[1], \dots, \vaf[\anydim], \vafi) ^{2\lgr\derp - 1}
      } \right)
    \\ & =
    \frac{
      \tilde P_{\ind, \dv}^\derp
      (\vp[1]/\vp[0], \dots, \vp[\anydim]/\vp[0], \vp[\anydim+1]/\vp*)
    }{
      \bigl(
        (\vp* / \vp[0])
        Q (\vp[0]/\vp[0], \dots, \vp[\anydim]/\vp[0], \vp[\anydim+1]/\vp[0])
      \bigr)^{2\lgr\derp - 1}
    }
    \cdot \left(
      \frac{ \vp[0] \vp* }{ \vp[0] \vp* }
    \right)^{(2\lgr\derp-1)\anydeg}
    \\ & =
    \frac{
      \bar P_{\ind, \dv}^\derp (\vp[0], \dots, \vp[\anydim+1])
    }{
      \bigl(
        \vp*^{\anydeg+1} Q(\vp[0], \dots, \vp[\anydim+1])
      \bigr)^{2\lgr\derp - 1}
    }
    =
    \frac{
      P_{\ind, \dv}^\derp (\vp[0], \dots, \vp[\anydim+1])
    }{
      \bigl(
        \vp*^{\anydeg+1} Q(\vp[0], \dots, \vp[\anydim+1])
      \bigr)^{2\lgr\derp}
    }
  \end{align}
  où \( \bar P_{\ind, \dv}^\derp \) est l'homogénéisé de \( \tilde P_{\ind,
      \dv}^\derp \) par rapport à \( \vp[0] \vp* \), de degré \(
    2\anydeg(2\lgr\derp-1) \),
  et \( P_{\ind, \dv}^\derp = \bar P_{\ind, \dv}^\derp \vp*^{\anydeg+1} Q \).
  Cette dernière opération n'a pour but que de simplifier les calculs
  ultérieurs en rendant le degré exactement linéaire en \( \lgr\dermp \).

  Par construction, \( P_{\ind, \dv}^\derp \) est homogène de degré \(
    4\anydeg \lgr\derp \) et satisfait le premier point ; de plus
  \begin{align}
    \nv1{ P_{\ind, \dv}^\derp }
    & \le
    \nv1{ \chow\anyvar }^{2\lgr\derp - 1}
    \cdot \left(
      (8\anydim)^{\lgr\derp - 1} \anydeg^{3\lgr\derp -2}
    \right)^\dv
    \cdot
    \nv1{ \chow\anyvar } \cdot \anydeg^\dv
  \end{align}
  qui donne immédiatement le dernier point.
\end{proof}

Nous sommes maintenant prêts à étudier les dérivées des monômes de la forme
\begin{equation}
  M^\ip_\ind
  =
  \prod_{\indi=0}^{\anydim+1}
  \left( \frac{\vp[\indi]}{\vp*} \right)^{\ip[\indi]}
  \pmm,
\end{equation}
qui font l'objet du lemme suivant.

\begin{lem} \label{l:par-anyvar-mono}
  Soient \( \ip \in \N^{\anydim+1} \) et \( M_\ind^\ip \) comme ci-dessus,
  pour \( \ind \in \set{0, \dots, \anydim} \). En conservant les notations
  ci-dessus, pour tout \( \derp \in \N^\anydim \), il existe des formes
  homogènes \( P^\derp_{\ip, \dv} \) telles que
  \begin{enumthm}
    \item \( \der^\derp M_\ind^\ip
        = \frac{
          P^\derp_{\ip, \ind, \dv}
        }{
          Q_\ind^{2\lgr\derp} \vp*^{2\lgr\derp(\anydeg+1) + \lgr\ip}
        }
      \) dans \( \cdn(\anyvar) \) ;
    \item \( \deg P^\derp_{\ip, \ind, \dv}
        =
        \lgr\ip + 4 \anydeg \lgr\derp
      \) ;
    \item \( \nv1{ P^\derp_{\ip, \ind, \dv} }
        \le
        \nv1{ \chow\anyvar }^{2\lgr\derp}
        \left(
          \bigl( 8 \anydim \anydeg^{3} \bigr) ^{ \lgr\derp }
          \cdot 2^{ \anydim \lgr\ip }
        \right)^\dv
      \).
  \end{enumthm}
  De plus, \( \ord_{\vp[0]} \bigl( P^\derp_{\ip, \ind, \dv} \bigr)
    \ge \ip[0]  - \derp[0] \).
\end{lem}

\begin{proof}
  \newcommand \indl { \gmp\nu[\indi, \alpha_\indi] }
  \newcommand \sumset { {\mathcal N} }
  On utilise la règle de \bsc{Leibniz} :
  \begin{equation}
    \der^\derp M_\ind^\ip
    =
    \der^\derp
    \prod_{\indi=0}^{\anydim+1}
    \prod_{ \alpha_\indi = 1 }^{ \ip[\indi] }
    \biggl( \frac{ \vp[\indi] }{ \vp* } \biggr)
    =
    \sum_{\gmp\nu \in \sumset}
    \prod_{\indi=0}^{\anydim+1}
    \prod_{ \alpha_\indi = 1 }^{ \ip[\indi] }
    \der^\indl \biggl( \frac{ \vp[\indi] }{ \vp* } \biggr)
    \pmm,
  \end{equation}
  où la somme est prise sur l'ensemble
  \begin{equation}
    \sumset = \left\{
      \gmp\nu \in \prod_{\indi=0}^{\anydim+1} (\N^\anydim)^{\ip[\indi]}
      \text{ tel que }
      \sum_{\indi=0}^{\anydim+1}
      \sum_{ \alpha_\indi = 1 }^{ \ip[\indi] }
      \indl
      = \derp
    \right\}
    \pmm.
  \end{equation}

  On pose alors \( P_{\ind, \dv}^{(0, \dots, 0)} = 1 \) et \( \indic0(\derp) =
    1 \) si \( \derp = (0, \dots, 0) \) et \( 0 \) sinon, de sorte que d'après
  le lemme précédent on a
  \begin{equation}
    \der[\derp] \frac{ \vp[\anydim+1] }{ \vp* }
    =
    \frac{ P_{\ind, \dv}^\derp }{ (\vp*^{\anydeg+1} Q)^{2\lgr\derp} }
    \cdot \left( \frac{ \vp[\anydim+1] }{ \vp* } \right)^{\indic0(\derp)}
  \end{equation}
  pour tout \( \derp \), même nul. En appliquant de plus le
  lemme~\vref{l:param-any-easy}, il vient :
  \begin{align}
    \der^\derp M_\ind^\ip
    =
    \sum_{\gmp\nu \in \sumset'} \Biggl(
      \prod_{\indi=0}^{\anydim}
      \ &
      \prod_{ \mathclap{\alpha_\indi = 1} }^{ \ip[\indi] }
      (-1)^{\indl[\ind]}
      \left( \frac{ \vp[\indi] }{ \vp* } \right)^{1-\indl[\indi]}
      \left( \frac{ \vp[0] }{ \vp* } \right)^{\lgr\indl}
    \Biggr)
    \\ \cdot &
    \renewcommand\indi{{\anydim+1}}
    \prod_{ \mathclap{\alpha_\indi = 1} }^{ \ip[\indi] } \
    \frac{ P_{\ind, \dv}^\indl }{ (\vp*^{\anydeg+1} Q)^{2\lgr\indl} }
    \cdot \left( \frac{ \vp[\anydim+1] }{ \vp* } \right)^{\indic0(\indl)}
    \pmm,
  \end{align}
  où la somme est prise sur l'ensemble
  \begin{equation}
    \sumset' = \left\{
      \gmp\nu \in \sumset
      \text{ tel que }
      \gmp\nu[\ind][\alpha_\ind] = 0
      \text{ et, \( \forall (\indi, \alpha_\indi) \), }
      \supp\indl \subset \set{\ind, \indi}
      \text{ et }
      \indl[\indi] \le 1
    \right\}
    \pmm.
  \end{equation}

  On remarque alors que
  \(
    \sum_{\indi=0}^{\anydim+1}
    \sum_{ \alpha_\indi = 1 }^{ \ip[\indi] }
    1
    =
    \lgr\ip
  \)
  de sorte que multiplier chaque facteur par \( \vp* \) dans l'expression
  ci-dessus revient à multiplier la somme par \( \vp*^{\lgr\ip} \). On est
  ainsi amené à poser
  \begin{align}
    P^\derp_{\ip, \ind, \dv}
    =
    \sum_{\gmp\nu \in \sumset'} \Biggl(
      \prod_{\indi=0}^{\anydim}
      & \prod_{ \alpha_\indi = 1 }^{ \ip[\indi] }
      (-1)^{\indl[\ind]}
      \vp[\indi] ^{1-\indl[\indi]} \vp* ^{\indl[\indi]}
      \cdot \vp[0] ^{\lgr\indl} \vp*^{\lgr\indl(2\anydeg + 1)}
      \cdot Q^{2\lgr\indl}
    \Biggr)
    \\
    \cdot
    & \renewcommand\indi{{\anydim+1}}
    \prod_{ \alpha_\indi = 1 }^{ \ip[\indi] }
    P_{\ind, \dv}^\indl
    \cdot
    \vp[\anydim+1]^{\indic0(\indl)}
    \vp*^{1 - \indic0(\indl)}
  \end{align}
  qui satisfait au premier point.

  Le calcul du degré est direct (par homogénéité, c'est celui du dénominateur)
  et on ne détaille donc que l'estimation de norme : chaque terme de la somme
  définissant \( P_{\ip, \ind, \dv}^\derp \) est majoré en norme par
  \begin{align}
    \Biggl(
      \prod_{\indi=0}^{\anydim}
      \prod_{ \alpha_\indi = 1 }^{ \ip[\indi] }
      \nv1{ Q }^{2\lgr\indl}
    \Biggr)
    \renewcommand\indi{{\anydim+1}}
    \prod_{ \alpha_\indi = 1 }^{ \ip[\indi] }
    \nv1{ P_{\ind, \dv}^\indl }
    \le
    \nv1{ \chow\anyvar }^{2\lgr\derp}
    \cdot (8 \anydim \anydeg^3)^{\lgr\derp \dv}
  \end{align}
  d'après le lemme précédent et le fait que
  \( \nv1 Q ^{2\lgr{\gmp\nu}} \le \nv1{ P_{\ind, \dv}^{\gmp\nu} } \) pour tout
  \( \gmp\nu \).  On remarque alors que l'ensemble de sommation \( \sumset' \)
  est contenu dans \( \sumset \) qui s'écrit aussi
  \begin{equation}
    \prod_{p = 1}^{\anydim} \left\{
      \gmp\nu[][p] \in \N^{\lgr\ip}
      \text{ tel que }
      \sum_{\indi=0}^{\anydim+1}
      \sum_{ \alpha_\indi = 1 }^{ \ip[\indi] }
      \indl[p]
      = \derp[p]
    \right\}
    \pmm.
  \end{equation}
  Chacun des facteurs de ce produit est de cardinal
  \(
    \binom{ \derp[p] + \lgr\ip - 1 }{ \lgr\ip - 1 }
    \le
    2^{ \derp[p] + \lgr\ip - 2 }
  \).
  L'estimation annoncée suit en prenant le produit.

  Enfin, on constate que chaque terme de la somme définissant
  \( P^\derp_{\ip, \ind, \dv} \) contient un facteur (pour \( \indi = 0 \)) de
  la forme
  \(
    \prod_{\alpha_0 = 1}^{\ip[0]} \vp[0] ^{1-\gmp\nu[0, \alpha_0][0]}
  \). De plus, on a \( \sum_{\alpha_0 = 1}^{\ip[0]} \gmp\nu[0, \alpha_0][0]
    \le \derp[0] \). Ainsi, il y a au moins \( \ip[0] - \derp[0] \) termes
  nuls dans cette somme, soit autant de facteurs égaux à \( \vp[0] \) dans le
  produit précédent. Ainsi, \( \vp[0]^{\ip[0] - \derp[0]} \) est en facteur de
  chaque terme de la somme définissant \( P^\derp_{\ip, \ind, \dv} \), ce qui
  prouve l'assertion sur l'ordre.
\end{proof}

Transposons maintenant ce résultat à notre situation multiprojective. On
rappelle auparavant que les \( \pden* \) sont donnés par la
notation~\vref{n:pden-cexa}, qui coïncide avec la notation~\vref{n:pden-gen}
appliquée successivement à chaque facteur \( \var* \).

\begin{lem} \label{l:par-var}
  Soit \( G \in \cdn[(\vmp*[0], \dots, \vmp*[\vdim+1])_\fct] \) une forme de
  multidegré \(
    \alpha = (\alpha_1, \dots, \alpha_m ) \) et \( g = G /
    \prod\fctrange (\vmp*[\indt*])^{\alpha_\fct} \) avec \( \indt \in
    \prod\fctrange \set{0, \dots, \vdim*} \). Alors, pour tout \( \dermp \in
    \prod\fctrange \N^{\vdim*} \), il existe une forme \( P_{G, \ind,
      \dv}^\dermp \) telle que
  \begin{enumthm}
    \item \(
        \der^\dermp g
        = \frac{
          P^\dermp_{G, \ind, \dv}
        }{
          \prod\fctrange
          \pden*^{2\lgr{\dermp*}}
          (\vmp*[\indt*])^{ 2\lgr{\dermp*}(\vdeg* + 1) + \alpha_\fct }
        }
      \) dans \( \cdn(\var) \) ;
    \item \(
        \deg_\fct P^\dermp_{G, \ind, \dv}
        =
        \alpha_\fct + 4\vdeg* \lgr{\dermp*}
      \) ;
    \item \(
        \nv1{ P^\dermp_{G, \ind, \dv} }
        \le
        \nv1{ G }
        \prod\fctrange
        \nv1{ \varfca* }^{2\lgr{\dermp*}}
        \left(
          \bigl( 8 \vdim* \vdeg*^{3} \bigr) ^{ \lgr{\dermp*} }
          \cdot 2^{ \vdim* \alpha_\fct }
        \right)^\dv
      \).
  \end{enumthm}
  De plus, \( \inda* \bigl( P^\dermp_{G, \ind, \dv} \bigr) \ge \inda*(G) -
    \wtsum*(\dermp) \).
\end{lem}

\begin{proof}
  Les trois premiers points s'obtiennent en remarquant que \( g \) est une
  combinaison linéaire de monômes en \( \vmp*[\indi] / \vmp** \) pour \( \fct
    \in \set{1, \dots, m} \) et \( \indi \in \set{0, \dots, \vdim* + 1}
  \), puis en écrivant que
  \begin{equation}
    \der^\dermp \Biggl(
      \prod\fctrange
      \prod_{\indi=0}^{\vdim*+1}
      \biggl( \frac{ \vmp*[\indi] }{ \vmp** } \biggr)^{\imp**}
    \Biggr)
    =
    \prod\fctrange
    \der^{\fct, \dermp*} \Biggl(
      \prod_{\indi=0}^{\vdim*+1}
      \biggl( \frac{ \vmp*[\indi] }{ \vmp** } \biggr)^{\imp**}
    \Biggr)
  \end{equation}
  et en appliquant le lemme~\vref{l:par-anyvar-mono} sur chaque facteur.

  Seul le point sur l'indice reste à vérifier. Pour cela, considérons \(
    \vmp^\imp \) un monôme apparaissant dans l'écriture \( G \), puis \(
    \vmp^{\gmp\nu} \) un monôme apparaissant dans \( P^\dermp_{G, \ind, \dv}
  \).  D'après le lemme cité, on alors \( \gmp\nu*[0] \ge \imp*[0] -
    \dermp*[0] \) d'où, en sommant, \( \wtsum*(\gmp\nu) \ge \wtsum*(\imp) -
    \wtsum*(\dermp) \) qui est équivalent à l'estimation annoncée vu la
  définition de l'indice.
\end{proof}


\subsection{Minoration de l'indice}
\label{sec:vojta-extrap-core}

Nous allons maintenant montrer que la forme auxiliaire construite précédemment
s'annule avec un indice élevé en \( \ex \). Commençons par préciser la notion
d'indice utilisée, similaire à celle définie au début de la
sous-section~\vref{sec:siegel-plan}, mais cette fois en un point. Par ailleurs
on introduit d'abord la notion d'indice pondéré par un vecteur \( b \in
  \N^m \) quelconque avant de spécialiser au vecteur de poids qui nous
intéresse, ce qui n'était pas utile précédemment mais le sera cette fois pour
énoncer la variante du théorème du produit que nous utiliserons. Posons
\begin{equation}
  \wtsum[b]( \dermp )
  =
  \frac {\lgr{\dermp[1]}} {b_1} + \dots
  + \frac {\lgr{\dermp[m]}} {b_m}
\end{equation}
pour tout \( \dermp \in \N^\vdim \).  Si \( \ratfi \) est une fraction
rationnelle et \( x \) un point où elle est définie, on définit son
indice en \( x \) comme
\begin{equation}
  \indg{b}[x] \ratfi
  =
  \inf \set{
    \wtsum[b](\dermp)
    \text{ pour }
    \dermp \text{ tel que } \der[\dermp] \ratfi(x) \neq 0
  }
  \pmm,
\end{equation}
où la borne inférieure est en fait un minimum, sauf si \( \ratfi = 0 \), cas
où l'indice est infini.

\begin{rem} \label{r:ind-basis}
  La définition ci-dessus fait usage d'une base de dérivations de \(
    \cdn(\var*) \) \lat{via} les applications \( \der[\dermp] \) (voir le début
  de la sous-section~\vref{sec:vojta-param}). Cependant, elle ne dépend pas de
  la base choisie, car seul compte l'ordre des opérateur différentiels sur
  chaque facteur (correspondant à \( \lgr{\dermp*} \) ci-dessus), ordre qui
  est indépendant de l'expression de ces opérateurs en fonction de telle ou
  telle base de dérivations.
\end{rem}

\begin{lem} \label{l:indice-inversible}
  Soient \( \ratfi_1 \), \( \ratfi_2 \) et \( \alpha \) des fonctions
  rationnelles telles que \( \ratfi_1 = \alpha \ratfi_2 \) et \( x \) un
  point où elles sont toutes les trois définies.
  \begin{enumthm}
    \item Si \( \dermp \) est tel que \( \der[\gmp\nu] \ratfi_2(x) = 0 \) dès
      que \( \gmp\nu < \dermp \) pour l'ordre produit sur \( \N^\vdim
      \), alors \( \der[\dermp] \ratfi_1(x) = \alpha(x) \,
        \der[\dermp] \ratfi_2(x) \).
    \item Si \( \alpha(x) \neq 0 \) on a \( \indg b[x](\ratfi_1) =
        \indg b[x](\ratfi_2) \).
  \end{enumthm}
\end{lem}

\begin{proof}
  Le premier point découle facilement de la formule de \bsc{Leibniz} :
  \begin{equation}
    \der[\dermp] \ratfi_1(x)
    =
    \sum_{\gmp\nu \le \dermp}
    \der[\dermp - \gmp\nu] \alpha(x) \,
    \der[\gmp\nu] \ratfi_2(x)
    \pmm.
  \end{equation}
  Or, par hypothèse, tous les termes de cette somme sont nuls sauf peut-être
  celui où \( \gmp\nu = \dermp \).

  Si \( \alpha(x) \neq 0 \) alors \( \alpha^{-1} \) est également définie
  en \( x \) et les deux autres fonctions jouent donc un rôle symétrique.
  Ainsi, si un indice \( \dermp \) est minimal pour la condition \(
    \der[\dermp] \ratfi_1(x) \neq 0 \), il l'est aussi pour la condition \(
    \der[\dermp] \ratfi_2(x) \neq 0 \) grâce au point précédent, ce qui
  prouve que les deux fonctions ont le même indice en \( x \).
\end{proof}

Si \( G \) est une forme multihomogène, on définit \( \indg b[x] G =
  \indg b[x] G/H \) où \( H \) est n'importe quelle forme multihomogène de
même degré ne s'annulant pas en \( x \). Le deuxième point du lemme
précédent montre que cette définition a bien un sens. On s'intéressera
désormais à l'indice pondéré par \( \wtw \wts = (\wts[1], \dots,
  \wts[m-1], (m-1) \wts**) \).

\medskip

On choisit dans l'atlas \( \clmaps^{m-1} \) présenté en
sous-section~\vref{sec:wemb} une carte \( \clmape \) contenant \( \ex \), et
on considère la forme multihomogène \( \faux** \) donnée par le
scolie~\vref{s:aux-co}. Le but de cette section est alors de montrer la
proposition suivante.

\begin{prop} \label{p:extra}
  On a \( \inda** \faux** \ge \frac{ \epsi \delta }{ \sigma } \) avec
  \( \sigma = \frac{128 (m-1) B}{\eps} \).
\end{prop}

La preuve consistera à évaluer les valeurs absolues locales des valeurs en \(
  \ex \) des dérivées successives de cette forme et à montrer qu'elles sont
suffisamment petites, tant que l'ordre de dérivation n'est pas trop élevé,
pour contredire la formule du produit à moins que la dérivée ne s'annule.
Avant de procéder, commençons par introduire quelques choix d'indices
adaptés à chaque place.

On rappelle que \( \cexa \) désigne un système de coordonnées multihomogènes
(dans le plongement adapté) du point \( \ex \) supposé contredire le
théorème~\vref{t:vojta-div}.  On choisit, pour tout \(
  \fct \), un indice \( \indv* \in \set{0, \dots, \vdim*} \) de sorte que \(
  \av{ \cexa*[\indv*] } \) soit maximal parmi \( \av{\cexa*[0]}, \dots,
  \av{\cexa*[\vdim*]} \).  Le lemme suivant montre que cette valeur absolue
représente à peu de chose près la norme de \( \cexa* \).

\begin{lem} \label{l:coord-norm}
  Avec les notations précédentes, on a
  \begin{equation}
    \av{ \cexa*[\indv*] }
    \ge
    \nv\infty{ \cexa }
    \cdot \nv1{ \varfc* }^{-1}
    \cdot ( 2 \vdegp* )^{- \vdeg* (\vdim* + 1) \dv}
    \pmm.
  \end{equation}
\end{lem}

\begin{proof}
  Il s'agit de montrer qu'on a
  \begin{equation}
    \av{ \cexa** }
    \le
    \av{ \cexa*[\indv*] }
    \cdot \nv1{ \varfc* }
    \cdot ( 2 \vdegp* )^{\vdeg* (\vdim* + 1) \dv}
  \end{equation}
  pour tout \( \ind \). D'après la note~\vref{fn:varfc},
  on a \( \nv1{ \varfc* } \ge 1 \), donc le résultat
  est acquis dès que \( \av{ \cexa** } \le \av{ \cexa*[\indv*] } \), ce qui
  est le cas pour \( \ind \le \vdim* \) par définition de \( \indv* \). Notons
  donc \( \ind > \vdim* \) un indice tel que \( \av{ \cexa*[\indv*] } < \av{
      \cexa** } \).

  On utilise alors le fait qu'on connaît, dans un plongement adapté, des
  relations \( \poldep** \) liant chacune des dernières coordonnées aux
  premières,  unitaires en la dernière variable (voir
  page~\pageref{p:def-poldep}). En décomposant \( \poldep**
  \) suivant les puissances de \( \vmp** \), on voit qu'il existe des
  polynômes \( \poldep*[\ind, \alpha] \in \cdn[ \vmp*[0], \dots, \vmp*[\vdim*]
    ] \) tels que :
  \begin{equation}
    (\vmp**) ^{ \vdeg* }
    =
    \sum_{ \alpha=1 }^{ \vdeg* }
    (\vmp**) ^{ \vdeg* - \alpha }
    \, \poldep*[\ind, \alpha]
    \quad\text{où }
    \deg \poldep*[\ind, \alpha] = \alpha
    \text{ et }
    \sum_{ \alpha=1 }^{ \vdeg* } \nv1{ \poldep*[\ind, \alpha] }
    \le \nv1{ \poldep** }
    \pmm.
  \end{equation}
  On spécialise alors en \( \cexa* / \cexa*[\indv*] \) et on prend les valeurs
  absolues :
  \begin{equation}
    \av[\Bigg]{ \frac{ \cexa** }{ \cexa*[\indv*] } }^{ \vdeg* }
    \le
    \sum_{ \alpha=1 }^{ \vdeg* }
    \: \av[\Bigg]{ \frac{ \cexa** }{ \cexa*[\indv*] } }^{ \vdeg* - \alpha }
    \: \av[\Bigg]{
      \poldep*[\ind, \alpha] \Biggl(
        \frac{ \cexa*[0] }{ \cexa*[\indv*] }, \dots,
        \frac{ \cexa*[\vdim*] }{ \cexa*[\indv*] }
      \Biggr)
    }
  \end{equation}
  puis on divise et on utilise les remarques précédentes pour obtenir :
  \begin{align}
    \av[\Bigg]{ \frac{ \cexa** }{ \cexa*[\indv*] } }
    \le
    \sum_{ \alpha=1 }^{ \vdeg* }
    \: \av[\Bigg]{ \frac{ \cexa** }{ \cexa*[\indv*] } }^{ 1 - \alpha }
    \: \nv1{ \poldep*[\ind, \alpha] }
    \le
    \sum_{ \alpha=1 }^{ \vdeg* }
    \nv1{ \poldep*[\ind, \alpha] }
    \le
    \nv1{ \poldep** }
  \end{align}
  Le résultat désiré est alors donné exactement par~\eqref{e:nv-poldep}.
\end{proof}

On introduit ensuite le point \( \exi \in \va^{m-1} \) tel que \(
  \wemb(\ex) = (\ex, \exi) \) et on note
\( \cexi* = \wembclg\clmape* (\cexa[\fcti], \cexa[m]) \), qui est un
système de coordonnées multihomogènes de \( \exi \) dans le plongement \(
  \vaemb^{m-1} \) (on rappelle que \( \wembclm* \) est défini
par~\eqref{e:def-wembclm}).  On choisit alors pour tout \( \fcti \) un indice
\( \indiv* \in \set{0, \dots, n} \) maximisant \( \av{ \cexi*[\indiv*] }
\), ainsi que \( \indig* \in \set{0, \dots, n} \) tel que \(
  \cexi*[\indig*] \neq 0 \). Il est alors clair que \( \av{ \cexi*[\indiv*] }
  \ge \nv1{ \cexi* } (n+1)^{-\dv} \).

\medskip

Introduisons maintenant une fonction rationnelle définie par
\begin{equation}
  \ratf =
  \frac{
    \faux**( \vmp )
  }{
    \prod\fctrange
    (\vmp*[0])^{\epsz \wts* \delta + r_\fct}
    \,
    \prod\fctirange
    \wembclg\clmape*[\indig*](\vmp) ^{ \delta }
  }
  \pmm.
\end{equation}
D'après l'hypothèse~\eqref{e:Vapx}, on a \( \cexa*[0] \neq 0 \) ; par
ailleurs le choix de \( \indig* \) assure que
\( \wembclg\clmape*[\indig*]( \cexa ) \neq 0 \), de sorte que le
dénominateur ne s'annule pas en \( \cexa \) et
que par définition on a \( \inda** \faux** = \inda** \ratf \).
Pour minorer cet indice, nous allons commencer par décomposer cette fonction
en des facteurs dont nous aurons un bon contrôle en chaque place.

Pour toute place \( v \) et tout indice \( \fcti \) on note désormais
\( \clmapv* \) la carte donnée par la section~\vref{sec:vaemb} avec \( (a, b,
  x, y) = (\wt*, \wt**, \ex*, \ex**) \), puis \( \clmapv = (\clmapv[1], \dots,
  \clmapv**) \) la carte correspondante dans \( \clmaps^{m-1} \).  On
pose alors pour simplifier \( \wembclv* = \wembclg\clmapv* \) et \( \fauxv =
  \faux[\clmapv] \).  En utilisant le premier point du scolie~\vref{s:aux-co}
et en remarquant que, pour chaque \( \fcti \), les vecteurs \( \wembclv*
  (\cexa[\fcti], \cexa[m]) \) sont non nuls et proportionnels entre eux
quand \( v \) varie, on a, pour tout \( v \) :
\begin{align}
  \ratf
  & =
  \frac{
    R(\vmp) \,
    \faux( \vmp; \wembclv[1](\vmp), \dots, \wembclv[m-1](\vmp) )
  }{
    \prod\fctrange
    (\vmp*[0])^{\epsz \wts* \delta + r_\fct}
    \cdot
    \prod\fctirange
    \wembclv*[\indig*](\vmp) ^{ \delta }
  }
\end{align}
par homogénéité de \( \faux \). On introduit alors nos indices locaux \( \indv
\) et \( \indiv \) :
\begin{align}
  \ratf
   & =
  \frac{
    R(\vmp) \,
    \faux( \vmp; \wembclv[1](\vmp), \dots, \wembclv[m-1](\vmp) )
  }{
    \prod\fctrange
    (\vmp*[\indv*])^{\epsz \wts* \delta + r_\fct}
    \cdot
    \prod\fctirange
    \wembclv*[\indiv*](\vmp) ^{ \delta }
  }
  \cdot \prod\fctirange
  \left(
    \frac{ \wembclv**(\vmp) }{ \wembclv*[\indig*](\vmp) }
  \right) ^{ \delta }
  \!\!\! \cdot \prod\fctrange
  \left(
    \frac{ \vmp*[\indv*] }{ \vmp*[0] }
  \right) ^{ \epsz \wts* \delta + r_\fct }
\end{align}
puis on utilise la définition de \( \Dir \) en scindant le premier facteur et
de nouveau la remarque sur les \( \wembclv* \) pour simplifier le second :
\begin{align}
  \ratf
  & =
  \frac{
    R(\vmp) \,
    \faux( \vmp; \wembclv[1](\vmp), \dots, \wembclv[m-1](\vmp) )
  }{
    \prod\fctrange (\vmp*[\indv*]) ^{ \Dir* }
  }
  \\ & \qquad
  \cdot \underbrace{
    \prod\fctirange \left(
      \frac{
        ( \vmp[\fcti ][{\indv[\fcti ]}] )^{ 2\wts[\fcti ] }
        ( \vmp[m][{\indv[m]}] )^{ 2\wts[m] }
      }{
        \wembclv**(\vmp[\fcti], \vmp[m])
      }
    \right)^\delta
  }_{\textstyle \ratfv2}
  \cdot \underbrace{
    \prod\fctirange \left(
      \frac{
        \wembclg\clmape*[\indiv*](\vmp) }{
        \wembclg\clmape*[\indig*](\vmp) }
    \right) ^{ \delta }
  }_{\textstyle \ratfv3}
  \cdot \underbrace{
    \prod\fctrange \left(
      \frac{ \vmp*[\indv*] }{ \vmp*[0] }
    \right) ^{ \epsz \wts* \delta + r_\fct }
  }_{\textstyle \ratfv4}
  \pmm.
\end{align}
On remarque que le numérateur du premier facteur est égal à \( \fauxv \)
modulo \( \varida \).

Soit maintenant \( \dermp \in \N^\vdim \) un indice minimal tel que \(
  \der[\dermp] \ratf(\ex) \neq 0 \). Comme les derniers facteurs de
l'écriture précédente sont tous définis et inversibles en \( \cex \) grâce aux
différents choix d'indices et au premier point du scolie~\vref{s:part-cases},
une nouvelle application du lemme~\vref{l:indice-inversible} montre que, pour
tout \( v \), on a l'expression suivante, où l'on rappelle que \( \fauxv
  = \faux[\clmapv] \) :
\begin{equation} \label{e:def-ratfv1}
  \der[\dermp] \ratf(\ex)
  =
  \underbrace{
    \der[\dermp]
    \frac{
      \fauxv( \vmp )
    }{
      \prod\fctrange (\vmp*[\indv*]) ^{ \Dir* }
    }
  }_{\textstyle \ratfv1} \null
  ( \cexa )
  \cdot \ratfv2(\cexa)
  \cdot \ratfv3(\cexa)
  \cdot \ratfv4(\cexa)
\end{equation}
c'est-à-dire que
\(
  \der[\dermp] \ratf(\ex)
  =
  \ratfv1(\cexa)
  \cdot \ratfv2(\cexa)
  \cdot \ratfv3(\cexa)
  \cdot \ratfv4(\cexa)
\).
Remarquons que chacun des facteurs du membre de droite dépend de \( v \),
mais pas leur produit.

Pour \( p \in \set{1, 2, 3, 4} \), posons alors \( \ratfh{p} = \sum_v
  \degv \ln \av{\ratfv{p}(\cexa)} \) où \( \degv \) désigne le degré local
divisé en \( v \) ; cette somme converge car chaque \( \ratfv p \)
appartient à un ensemble fini fixe de nombres algébriques (dont le cardinal
est majoré en fonction de \( n \), \( m \) et \( \nclmaps \)).  La
formule du produit donne alors
\begin{equation} \label{e:prod-lem}
  0 = \ratfh1 + \ratfh2 + \ratfh3 + \ratfh4
  \pmm.
\end{equation}
Supposons maintenant que, contrairement à la conclusion de la proposition, on
ait
\begin{equation} \label{e:extra-dermp}
  \wtsum( \dermp ) < \epsi \delta / \sigma
\end{equation}
et montrons qu'on contredit alors l'égalité ci-dessus. Pour cela, on majore
séparément chacun des termes, en procédant de droite à gauche (par ordre plus
ou moins croissant de difficulté) et on obtient une somme négative en
utilisant l'hypothèse ci-dessus, les hypothèses~\eqref{e:Vbig}
et~\eqref{e:Vcos} sur \( \ex \), et bien sûr les propriétés de la forme
auxiliaire.

Pour \( \ratfv4(\cexa) \) on commence par remarquer que le dénominateur
n'intervient pas dans le calcul de la hauteur, puis au numérateur que la
valeur absolue d'une coordonnée est majorée par la norme du vecteur :
\begin{equation}
  \prod_v
  \prod\fctrange
  \left(
    \frac{ \av{\cexa*[\indv*]} }{ \av{\cexa*[0]} }
  \right) ^{ (\epsz \delta \wts* + r_\fct) \degv }
  \le
  \prod\fctrange
  \hautm\infty{ \cexa* }^{\epsz \delta \wts* + r_\fct}
\end{equation}
puis on passe au logarithme :
\begin{align}
  \ratfh4
  & \le
  \epsz \delta \Bigl(
    \sum\fctrange \wts* \hautl\infty{ \cexa* }
  \Bigr) + o(\delta)
  \pmm.
\end{align}
Concentrons-nous sur le contenu de la parenthèse, que nous rencontrerons à
nouveau en évaluant \( \ratfv1(\cexa) \).  On utilise le fait que \( \cexa* =
  \vadapt*(\cex*) \), puis~\eqref{e:comp-h-hn} :
\begin{align}
  \sum\fctrange \wts* \hautl\infty{ \cexa* }
  & \le
  \sum\fctrange \wts* \bigl(
    \hautl1{ \cex* } + \hautl\infty{\vadapt*}
  \bigr)
  \\ & \le
  \sum\fctrange \wts* \bigl(
    \hautn{ \ex* } + \htcmp + \frac12\ln(n+1) + \hautl\infty{\vadapt*}
  \bigr)
  \pmm.
\end{align}
On remarque alors que \( \hautl\infty{\vadapt*} \le \ln B \)
d'après~\eqref{e:vadapt-ht} et~\eqref{e:varset-deg}. On regroupe alors les
termes avant d'utiliser d'une part~\eqref{e:wt-ratio} et d'autre part le
lemme~\vref{l:sum-wts} :
\begin{align}
  \sum\fctrange \wts* \hautl\infty{ \cexa* }
  & \le
  \sum\fctrange \bigl( \wts* \hautn{ \ex* } \bigr)
  + \bigl( \htcmp + \frac12\ln(n+1) + \ln B \bigr) \sum\fctrange \wts*
  \\ & \le
    2 m \wts[1] \hautn{ \ex[1] }
    + 2 \wts[1] \bigl( \frac32\cst{vs-ht} + \ln B \bigr)
  \pmm.
  \label{e:sum-h-cexa}
\end{align}
Au final on a donc
\begin{equation} \label{e:ratfh4}
  \ratfh4
  \le
  2 \epsz \delta \wts[1]
  \bigl( m \hautn{ \ex[1] } + \frac32\cst{vs-ht} + \ln B \bigr)
  \pmm.
\end{equation}

Pour \( \ratfh3 \) on commence avec les mêmes arguments pour écrire
\begin{equation}
  \prod_v
  \prod\fctirange
  \left(
    \frac{
      \av{\wembclg\clmape*[\indiv*](\cexa[\fcti], \cexa[m])} }{
      \av{\wembclg\clmape*[\indig*](\cexa[\fcti], \cexa[m])} }
  \right) ^{ \delta \degv }
  \le
  \prod\fctirange
  \hautm\infty{ \exi* }^\delta
\end{equation}
puis on passe au logarithme et on utilise encore~\eqref{e:comp-h-hn} :
\begin{align}
  \ratfh3
  & \le
  \delta \sum\fctirange \Bigl(
    \hautn{ \exi* } + \htcmp
  \Bigr)
  \intertext{mais on conclut cette fois en invoquant~\eqref{e:hautn-wt-diff} :}
  \ratfh3 \label{e:ratfh3}
  & \le
  \delta \bigl(
    3 \wts[1] (m-1) \Vcos \hautn{\ex[1]} + o(\wts[1])
  \bigr)
  \pmm.
\end{align}

Pour \( \ratfv2 \), on commence par minorer le dénominateur grâce à la
définition de \( \indiv* \) puis on utilise la majoration évidente \(
  \nv\infty{\cexa*} \le \nv\infty{\vadapt*} \, \nv1{\cex*} \) :
\begin{align}
  \frac{
    \av{ \cexa[\fcti ][{\indv[\fcti ]}] }^{ 2\wts[\fcti ] }
    \av{ \cexa[m][{\indv[m]}] }^{ 2\wts[m] }
  }{
    \av{ \wembclv**(\cexa[\fcti], \cexa[m]) }
  }
  & \le
  \frac{
    \nv\infty{ \cexa[\fcti ] }^{ 2\wts[\fcti ] }
    \nv\infty{ \cexa[m] }^{ 2\wts[m] }
  }{
    \nv1{ \wembclv*(\cexa[\fcti], \cexa[m]) }
  }
  (n+1)^\dv
  \\ & \le
  \frac{
    \nv1{ \cex[\fcti ] }^{ 2\wts[\fcti ] }
    \nv1{ \cex[m] }^{ 2\wts[m] }
  }{
    \nv1{ \wembclp\clmapv\fcti(\cex[\fcti], \cex[m]) }
  }
  (n+1)^\dv B^{2 \dv (\wts[\fcti] + \wts[m])}
  \\ & \le
  (n+1)^\dv
  \bigl( \hmclab* B^{2\dv} \bigr)^{\wts[\fcti] + \wts[m]}
\end{align}
compte tenu de~\eqref{e:clab-loc} et du choix des \( \clmapv \).  Il ne reste
alors plus qu'à prendre le logarithme dans l'estimation précédente, multiplier
par \( \delta \) puis sommer sur \( v \) et sur \( \fcti \) pour avoir
\begin{equation} \label{e:ratfh2}
  \ratfh2 \le
  \delta \bigl( 2 \wts[1] (\cst{vs-ht} + 2 \ln B) + o(\wts[1]) \bigr)
\end{equation}
en utilisant encore le lemme~\vref{l:sum-wts}.

\medskip

Nous allons maintenant expliciter \( \ratfv1 \).
Pour chaque place \( v \), on applique alors le
lemme~\vref{l:par-var} avec \( G = \fauxv \) et \( \ind = \indv \), et on
note \( P_v = P_{\fauxv, \indv, \dv}^\dermp \) la forme obtenue pour
l'indice \( \dermp \) fixé plus haut (plus précisément, celle des deux formes
obtenues qui correspond à la place considérée). Remarquons comme précédemment
que \( P_v \) parcourt un ensemble fini de formes à coefficients
algébriques quand \( v \) varie ; la famille \( P_v \) jouit de plus
des propriétés suivantes.

\begin{lem} \label{l:ratfv1}
  Dans les notations précédentes, on a :
  \begin{enumthm}
    \item \( \displaystyle
        \ratfv1
        =
        \frac{ P_v(\vmp) }{
          \prod\fctrange
          \pden*(\vmp*)^{2\lgr{\dermp*}}
          (\vmp*[\indv*])^{ 2\lgr{\dermp*}(\vdeg* + 1) + \Dir* }
        }
      \) ;
    \item \(
        \deg_\fct P_v
        \le
        3 \delta \wtw* \wts*
      \) ; \label{i:deg-p-der}
    \item \(
        \hautl1{ (P_v)_v }
        \le
        15 \delta \wts[1] \cst{vs-ht} \Lambda^{(1 + \frac1m) f(\vdim)}
    + o(\delta)
      \) ; \label{i:norm-p-der}
    \item \(
        \inda* \bigl( P_v \bigr)
        \ge
        \frac34 \epsi \delta
      \) ; \label{i:ind-p-der}
  \end{enumthm}
  où l'on a noté
  \(
    \hautl1{ (P_v)_v }
    =
    \sum_v \degv \ln \nv1{ P_v }
  \).
\end{lem}

\begin{proof}
  Chaque point découle du point correspondant du lemme~\vref{l:par-var}, en
  tenant compte des informations connues sur \( \fauxv \) (voir le
  scolie~\vref{s:aux-co}) et \( \dermp \) (voir~\eqref{e:extra-dermp}). Le
  premier est immédiat d'après la définition~\eqref{e:def-ratfv1} de \(
    \ratfv1 \) et le fait que \( \deg_\fct \fauxv = \Dir* \).

  Avant de passer au degré, remarquons que l'hypothèse~\eqref{e:extra-dermp}
  implique que
  \begin{equation} \label{e:dermp2wts}
    \lgr{ \dermp* } < \wtw* \wts* \epsi \delta / \sigma
    \pmm,
  \end{equation}
  ce qui nous servira également pour l'estimation de hauteur.  On écrit alors
  \begin{align}
    \deg_\fct P_v
    & =
    \Dir* + 4\vdeg* \lgr{\dermp*}
    \\ & \le
    \delta \wts* (2\wtw* + \epsz)
    + 4 \vdeg* \wtw* \wts* \epsi \delta / \sigma
    + o(\delta)
    \\ & \le
    \delta \wts* \wtw* (2 + \epsz + \frac{4B}{\sigma} \epsi)
    + o(\delta)
  \end{align}
  qui donne bien la majoration annoncée, compte tenu du fait que \( 4B/\sigma
    \le 1 \) par définition de \( \sigma \) (prop.~\vref{p:extra}) et qu'on a
  largement \( \epsz + \epsi \le 1 \).

  Pour la hauteur, dans le point correspondant du lemme, on prend le
  logarithme puis la somme (pondérée par les degrés locaux divisés) sur toutes
  les places :
  \begin{align}
    \hautl1{ (P_v)_v }
    & \le
    \hautl1{ (\fauxv)_v }
    \\ & \qquad
    + \sum\fctrange \bigl(
      2 \lgr{\dermp*} \hautl1{ \varfca* }
      + \lgr{\dermp*} ( \ln(8g) + 3 \ln B )
      + g (\Dir*) \ln 2
    \bigr)
  \\ & \le
    \hautl1{ (\fauxv)_v }
    + 2 \bigl( \sum\fctrange \lgr{\dermp*} \hautl1{ \varfc* } \bigr)
    + \delta g \ln 2 \sum\fctrange \wts* (2 \wtw* + \epsz)
    + o(\delta)
    \\ & \qquad
    + \bigl(
      2 g B ( \ln B + \ln( n + 1 ) )
      + \ln(8g) + 3 \ln B
    \bigr)
    \sum\fctrange \lgr{\dermp*}
  \end{align}
  où l'on a utilisé~\eqref{e:nv-varfca} et la définition de \( d' \). Nous
  allons maintenant majorer indépendamment chacun des termes de cette somme.
  Le premier est donné par~\eqref{e:aux-co-htv} ; nous verrons que c'est le
  terme dominant. Pour le deuxième, on a
  \begin{equation}
    \sum\fctrange \lgr{\dermp*} \hautl1{ \varfc* }
    \le
    \delta \epsi \sigma^{-1}
    \sum\fctrange \wts* \wtw* \hautl1{ \varfc* }
    \le
    \delta \wts[1] \epsi \sigma^{-1}
    \cst{vs-ht} \Lambda^{(1 + \frac1m) f(\vdim)}
  \end{equation}
  en utilisant la remarque précédente sur \( \dermp \) puis
  l'hypothèse~\eqref{e:varset-ht}.

  On écrit ensuite
  \begin{equation}
    g \ln 2 \sum\fctrange \wts* (2 \wtw* + \epsz)
    \le
    3 g \ln 2 \sum\fctrange \wts* \wtw*
    \le
    6 g \ln 2 \, \wts[1]
    \le
    \frac14 \Lambda \wts[1]
  \end{equation}
  en utilisant successivement le fait que \( \epsz < 1 \) puis le
  lemme~\vref{l:sum-wts} et enfin la définition~\eqref{e:def-Lambda} de \(
    \Lambda \) (deuxième argument du maximum et \( m \ge g \)).
  Enfin, on remarque que
  \begin{equation}
    \sum\fctrange \lgr{\dermp*}
    \le
    \delta \epsi \sigma^{-1} \sum\fctrange \wtw* \wts*
    \le
    2 \delta \wts[1] \epsi \sigma^{-1}
  \end{equation}
  en utilisant à nouveau la remarque précédente sur \( \dermp \) et le
  lemme~\vref{l:sum-wts}. On invoque alors~\eqref{e:ht-Pv-adhoc} pour majorer
  le dernier terme par
  \begin{equation}
    2 \delta \wts[1] \epsi \sigma^{-1}
    \cst{vs-ht} \Lambda^{(1 + \frac1m) f(\vdim)}
    \pmm.
  \end{equation}
  Il ne reste plus qu'à prendre la somme, en utilisant le fait que \( \epsi
    \sigma^{-1} \le 1/8 \) (majoration très large mais suffisante) pour
  aboutir à l'estimation annoncée.

  Enfin, le point sur l'indice est immédiat en remarquant que \(
    \wtsum*(\dermp) \le \wtsum(\dermp) \) par définition puis en utilisant les
  hypothèses :
  \begin{equation}
    \inda* \bigl( P_v \bigr)
    \ge
    \inda*(\fauxv) - \wtsum(\dermp)
    \ge
    \epsi \delta (1 - \sigma^{-1})
    \ge
    \frac34 \epsi \delta
  \end{equation}
  en utilisant la minoration large mais suffisante \( \sigma \ge 4 \).
\end{proof}

Pour estimer \( \ratfh1 \), commençons par scinder une nouvelle fois notre
fonction en deux facteurs :
\begin{equation}
  \ratfv1 =
  \frac{ P_v(\vmp) }{
    \prod\fctrange
    \pden*^{2\lgr{\dermp*}}
    (\vmp*[\indv*])^{ 2\lgr{\dermp*}(\vdeg* + 1) + \Dir* }
  }
  =
  \underbrace{
    \frac{ P_v(\vmp) }{
      \prod\fctrange (\vmp*[\indv*])^{\deg_\fct P_v }
    }
  }_{\textstyle \ratfv{1'}}
  \cdot
  \underbrace{
    \prod\fctrange
    \frac{
      (\vmp*[\indv*])^{ 2 \lgr{\dermp*} \cdot \deg \pden* }
    }{
      (\pden*)^{ 2 \lgr{\dermp*} }
    }
  }_{\textstyle \ratfv{1''}}
\end{equation}
Pour le second facteur, on commence par remarquer que le dénominateur ne
s'annulle pas en \( \cexa* \)
d'après le deuxième point du scolie~\vref{s:part-cases} et qu'on a donc
\begin{equation}
  \prod_v
  \frac{
    \av{ \cexa*[\indv] }^{ 2 \deg \pden* \lgr{\dermp*} \degv }
  }{
    \av{ \pden_\fct(\cexa*) }^{ \lgr{\dermp*} \degv }
  }
  \le
  \hautm\infty{ \cexa* }^{ 2 \deg \pden* \lgr{\dermp*} }
  \pmm.
\end{equation}
On note alors que par définition, \( \deg \pden* = \vdeg* - 1 \le B \), puis
on utilise~\eqref{e:dermp2wts}
\begin{align}
  \ratfh{1''}
  & \le
  2 B \sum\fctrange
  \hautl\infty{ \cexa* } \lgr{\dermp*}
  \\ & \le
  2 \delta B \epsi \sigma^{-1}
  \sum\fctrange \wtw* \wts* \hautl\infty{ \cexa* }
  \pmm.
\end{align}
En adaptant la démonstration de~\eqref{e:sum-h-cexa} on a facilement
\begin{equation}
  \sum\fctrange \wtw* \wts* \hautl\infty{ \cexa* }
  \le
  2 \wts[1] \bigl(
    2(m-1) \hautn{ \ex[1] } + \frac32\cst{vs-ht} + \ln B
  \bigr)
\end{equation}
en remarquant que \( \sum\fctrange \wtw* = 2(m-1) \). Au final, on a donc
\begin{align}
  \ratfh{1''}
  & \le
  4 \delta \wts[1] B \epsi \sigma^{-1} \bigl(
    2(m-1) \hautn{ \ex[1] } + \frac32\cst{vs-ht} + \ln B
  \bigr)
  \\ & \le \label{e:ratfh1b}
  \delta \wts[1] \bigl(
    8(m-1) B \epsi \sigma^{-1} \hautn{\ex[1]} + \frac32\cst{vs-ht} + \ln B
  \bigr)
  \pmm.
\end{align}
en utilisant le fait que \( 4 B \epsi \sigma^{-1} \le 1 \) d'après le choix de
\( \sigma \) pour les deux derniers termes, mais pas pour le premier car c'est
celui qui est crucial dans le choix de \( \sigma \) : on désire donc conserver
jusqu'au bout sa dépendance en \( \sigma \) explicite.

Pour l'autre partie, on écrit
\begin{equation}
  \ratfv{1'}(\cexa)
  =
  P_v \Biggl(
    \frac{ \cexa[1] } { \cexa[1][{\indv[1]}] }, \dots,
    \frac{ \cexa[m] } { \cexa[m][{\indv[m]}] }
  \Biggr)
  =
  \sum_{\imp} p_{v, \imp} \Biggl(
    \frac{ \cexa } { \cexa[][\indv] }
  \Biggr)^{\imp}
\end{equation}
où la dernière somme est prise sur les multiindices \( \imp \) tels que \(
  \lgr{\imp*} = \deg_\fct P_v \) et \( \wtsum*(\imp) \ge \frac34 \epsi
  \delta \). Étudions de plus près les valeurs
absolues des monômes intervenant dans cette écriture ; d'après le
lemme~\vref{l:coord-norm} on a :
\begin{align}
  \prod\fctrange \prod\indrange
  \Biggl(
    \frac{ \av{\cexa**} } { \av{\cexa*[\indv*]} }
  \Biggr)^{\imp**}
  & \le
  \prod\fctrange \prod\indrange
  \Biggl(
    \frac{ \av{\cexa**} } { \nv\infty{\cexa*} }
  \Biggr)^{\imp**}
  \cdot
  \prod\fctrange \Bigl(
    \nv1{ \varfc* }
    \cdot ( 2 \vdegp* )^{\vdeg* (\vdim* + 1) \dv}
  \Bigr)^{\deg_\fct P_v}
  \pmm.
\end{align}
Il est clair que le premier facteur est inférieur à \( 1 \) pour tout \(
  v \). Cependant, si \( v \in \placesapx \) on peut dire mieux en se
concentrant sur la partie
\begin{equation} \label{e:apx-temp}
  \prod\fctrange
  \Biggl(
    \frac{ \av{ \cexa*[0] } }{ \nv\infty{ \cexa* } }
  \Biggr)^{\imp*[0]}
\end{equation}
et en exploitant l'hypothèse principale.
On commence par utiliser le fait que \( \cex* = \vadapt*^{-1}( \cexa* ) \)
pour minorer le dénominateur, puis une comparaison classique de normes donne
\begin{equation}
  \frac{ \av{ \cexa*[0] } }{ \nv\infty{ \cexa* } }
  \le
  \frac{ \av{ \cexa*[0] } }{ \nv2{ \cex* } }
  \cdot (n+1)^{3\dv/2} \, \nv\infty{\vadapt*^{-1}}
  \pmm.
\end{equation}
Par construction de \( \vadapt* \) (lemme~\vref{l:adapt-gen}) on a \(
  \cexa*[0] = \cex*[0] \) et on remarque alors que le premier facteur dans
l'écriture ci-dessus n'est autre que \( \distv{\ex*}{\divi} \), ce qui nous
permet d'exploiter~\eqref{e:Vapx} :
\begin{align}
  \frac{ \av{ \cexa*[0] } }{ \nv\infty{ \cexa* } }
  & \le
  \hautm2{\ex*}^{-\wtapx \eps}
  \cdot (n+1)^{3\dv/2} \, \nv\infty{\vadapt*^{-1}}
  \\ & \le
  \expb^{ -\wtapx \eps \hautn{\ex*} }
  \cdot \expb^{\wtapx \eps \htcmp}
  \cdot (n+1)^{3\dv/2} \, \nv\infty{\vadapt*^{-1}}
\end{align}
où la dernière ligne découle comme d'habitude de~\eqref{e:comp-h-hn}.

En utilisant le fait que \( \imp*[0] \le \lgr{\imp*} = \deg_\fct P_v \),
on peut alors majorer le logarithme de~\eqref{e:apx-temp} par
\begin{align}
  - \wtapx\eps \sum\fctrange \imp*[0] \hautn{\ex*}
  + \sum\fctrange \deg_\fct P_v \left(
    \frac{3\dv}2 \ln(n+1) + \ln \nv\infty{\vadapt*^{-1}}
    + \wtapx\eps \htcmp
  \right)
\end{align}
et pour majorer le premier terme de cette somme on écrit en utilisant
encore~\eqref{e:wt-ratio}
\begin{equation}
  \sum\fctrange
  \imp*[0] \hautn{ \ex* }
  =
  \sum\fctrange
  \frac{ \imp*[0] }{ \wts* } \, \wts* \hautn{ \ex* }
  \ge
  \frac12 \wts[1] \hautn{ \ex[1] }
  \sum\fctrange
  \frac{ \imp*[0] }{ \wts* }
  \ge
  \frac38 \wts[1] \hautn{ \ex[1] }
  \, \delta \epsi
\end{equation}
où la dernière estimation découle du quatrième point du lemme~\vref{l:ratfv1}
et de la définition de l'indice.

En regroupant tous les termes qui constituent \( \ratfv{1'}( \cex ) \) puis en
sommant sur \( v \), on majore la hauteur logarithmique de ce facteur par
la quantité suivante, où l'on note \( \wtapx = 0 \) si \( v \notin
  \placesapx \) :
\begin{multline}
  \sum_v \degv \Biggl(
    - \frac38 \delta \wts[1] \hautn{\ex[1]} \wtapx \eps \epsi
    + \ln \nv1{P_v}
    + \sum\fctrange \deg_\fct P_v
    \biggl(
      \frac{3\dv}2 \ln(n+1)
      \\
      + \ln \nv\infty{\vadapt*^{-1}}
      + \wtapx\eps \htcmp
      + \ln \nv1{ \varfc* }
      + (\vdeg* (\vdim* + 1) \dv) \ln( 2 \vdegp* (n+1) )
    \biggl)
  \Biggr)
\end{multline}
puis en se souvenant que \( \sum_v \degv \wtapx = 1 \) et en
utilisant~\eqref{e:vadapt-ht} et le lemme~\vref{l:ratfv1} :
\begin{align}
  \ratfh{1'}
  & \le
  - \frac38 \delta \wts[1] \hautn{\ex[1]} \eps \epsi
  + 15 \delta \wts[1] \cst{vs-ht} \Lambda^{(1 + \frac1m) f(\vdim)}
  + 3\delta \sum\fctrange \wtw* \wts* \hautl1{ \varfc* }
  + o(\delta)
  \\ & \quad
  + 3\delta \Bigl(
    \frac{3}2 \ln(n+1)
    + n \ln(Bn)
    + \eps \htcmp
    + B (g + 1) \ln (B (n+1))
  \Bigr)
  \sum\fctrange \wtw* \wts*
\end{align}
On utilise alors l'hypothèse~\eqref{e:varset-ht}, le lemme~\vref{l:sum-wts} et
l'estimation~\eqref{e:ratfh1a-adhoc} pour conclure :
\begin{equation}
  \ratfh{1'}
  \le
  - \frac38 \delta \wts[1] \hautn{\ex[1]} \eps \epsi
  + 19 \delta \wts[1] \cst{vs-ht} \Lambda^{(1 + \frac1m) f(\vdim)}
  + o(\delta)
  \pmm.
\end{equation}

En substituant cette dernière estimation ainsi que \eqref{e:ratfh4},
\eqref{e:ratfh3}, \eqref{e:ratfh2}, \eqref{e:ratfh1b} dans \eqref{e:prod-lem},
il vient
\begin{align} \label{e:prod-lem-final}
  0
  & \le
  \delta \wts[1] \bigl(
    \alpha \hautn{\ex[1]} + \beta
  \bigr) + o(\delta \wts[1])
\end{align}
avec : % reporter dans la section « ajustement » !
\begin{align}
  \alpha
  & =
  2 m \epsz
  + 3 (m - 1) \Vcos
  + \frac{8 (m - 1) B \epsi}{\sigma}
  - \frac38 \eps \epsi
  \\
  \beta
  & =
  19 \cst{vs-ht} \Lambda^{(1 + \frac1m) f(\vdim)}
  + (\frac72 + 3\epsz) \cst{vs-ht} + (5 + 2\epsz) \ln B
  \\ & \le
  20 \cst{vs-ht} \Lambda^{(1 + \frac1m) f(\vdim)}
\end{align}
où la dernière majoration utilise~\eqref{e:extra-beta-adhoc} et le fait que \(
  2 \epsz \le 1 \).

Montrons que \( \alpha \) est négatif et même
inférieur à \( - \frac{3\eps \epsi}{16} \). La définition~\eqref{e:def-epsz}
de \( \epsz \) assure que
\begin{equation} \label{e:epsz-ct-extra}
  2 m \epsz
  \le \frac{\eps \epsi}{16}
  \pmm.
\end{equation}
Par ailleurs, en utilisant la définition~\eqref{e:def-epsi} de \( \epsi \)
puis celle~\eqref{e:def-Vcos} de \( \Vcos \), on a
\begin{align} \label{e:Vcos-ct-details}
  \frac{ \eps \epsi }{ 48 (m - 1) }
  & \ge
  \frac{
    \eps^{ \frac{m}{m-g} }
  }{
    48 (m - 1)
    \, \nclmaps^{ \frac{m-1}{m-g} }
    \, (85 \cdot 5^m)^{ \frac{g}{m-g} }
  }
  \\ & \ge
  \frac{
    \eps^{ \frac{m}{m-g} }
  }{
    m
    \nclmaps^{ \frac{m}{m-g} }
    \, (85 \cdot 5^g)^{ \frac{m}{m-g} }
  }
  =
  \Vcos
\end{align}
en remarquant que \( 48 \cdot 85^{ \frac{g}{m-g} } \le 85^{
    \frac{m}{m-g} } \). Ceci entraîne immédiatement que
\begin{equation} \label{e:Vcos-ct-extra}
  3 (m-1) \Vcos
  \le \frac{\eps \epsi}{16}
  \pmm.
\end{equation}
Enfin, le choix de \( \sigma \) dans l'énoncé de la proposition~\vref{p:extra}
assure que
\begin{equation} \label{e:ct-sigma}
  \frac{8 (m - 1) B \epsi}{\sigma}
  \le \frac{\eps \epsi}{16}
\end{equation}
et on a bien \( \alpha \le - \frac{3\eps \epsi}{16} < 0 \) comme annoncé.

Ainsi on a, en remarquant que \( \vdim \ge m \) et que \( f \) est
décroissante :
\begin{align} \label{e:Vbig-ct-extra}
  \alpha \hautn{\ex[1]} + \beta
  & \le
  - \frac{3\eps \epsi}{16} \hautn{\ex[1]}
  + 20 \cst{vs-ht} \Lambda^{(1 + \frac1m) f(m)}
  \\ & \le
  - \frac{3\eps \epsi}{16} \Bigl(
    \hautn{\ex[1]}
    - \frac{320}{3\eps \epsi}
    \cst{vs-ht} \Lambda^{(1 + \frac1m) f(m)}
  \Bigr)
  \pmm.
\end{align}
Or, en remarquant que \( \cst{vs-deg-prod} = 5 m / \Vcos \) d'après les
définitions respectives~\eqref{e:def-Vcos} et~\eqref{e:cst-vs-deg-prod} de ces
deux constantes, la comparaison~\eqref{e:Vcos-ct-details} entraîne
\begin{equation}
  \frac{320}{3\eps \epsi}
  \le
  \frac{ 48(m-1) \cdot 5m }{ \eps\epsi }
  \le
  \cst{vs-deg-prod}
  \le
  \Lambda
\end{equation}
par définition~\eqref{e:def-Lambda} de \( \Lambda \). Par ailleurs,
\( (1 + \frac1m) f(m) + 1 \le (1 + \frac1m) f(m-1) \)
d'après la définition~\eqref{e:def-f} de \( f \), donc en utilisant le fait
que \( m \ge 2 \), il vient :
\begin{align}
  \alpha \hautn{\ex[1]} + \beta
  & \le
  - \frac{3\eps \epsi}{16} \Bigl(
    \hautn{\ex[1]}
    - \cst{vs-ht} \Lambda^{(1 + \frac1m) f(m-1)}
  \Bigr)
  \le
  - \frac{9\eps \epsi}{16}
  \cst{vs-ht} \Lambda^{(1 + \frac1m) f(m-1)}
  <
  0
\end{align}
où l'on a utilisé l'hypothèse~\eqref{e:Vbig} et la
définition~\eqref{e:def-Vbig}. Au final, pour \( \delta \) et \( \wts[1] \)
assez grands, le membre de droite de~\eqref{e:prod-lem-final} est négatif,
contradiction qui achève la preuve de la proposition~\vref{p:extra}.



\section{Application du théorème du produit et conclusion}
\label{sec:thm-prod}

La section précédente a montré que \( \faux** \) était d'indice
élevé en \( \ex \). Nous allons maintenant en déduire l'existence d'une forme
\( T \) comme dans la conclusion de la proposition~\vref{p:varset-notmin}, en
utilisant le fait suivant, conséquence du théorème du produit.

\begin{fact} \label{f:thm-prod}
  Soient \( (x_1, \dots, x_m) \) un point rationnel de \( \proj{\vdim_1}
    \times \dots \times \proj{\vdim**} \) et \( \vdim = \vdim_1 + \dots +
    \vdim** \). On considère une forme \( G \) de degré \( b \in (\N
    \minusset 0)^m \) et on suppose qu'il existe \( \alpha > 0 \) tel
  que :
  \begin{enumthm}
    \item \( \indg b[x] G \ge \alpha \) ;
    \item \(
        \frac{b_\fcti}{b_{\fcti+1}}
        \ge
        \left( \frac m\alpha \right)^\vdim
      \)
      pour tout \( \fcti \in \set{1, \dots, m-1} \) ;
    \item \(
        \frac\alpha m < \left( \frac{\ln(\vdim+1)}{2\vdim^2}\right)^\vdim
      \).
  \end{enumthm}
  Il existe alors un \( \fct \in \set{1, \dots, m} \) et une forme
  \( T \in \cdn[ \vmp*[0], \dots, \vmp*[\vdim*] ] \) non nulle, telle que
  \begin{enumthm}
    \item \( x \in \zeros T  \) ;
    \item \( \deg T
        \le
        \left( \frac m\alpha \right)^\vdim
      \) ;
    \item la hauteur de \( T \) satisfait à
      \begin{align}
        b_\fct \, \hautl\infty T
        & \le
        \vdim*
        \left( \frac m\alpha \right)^\vdim
        \left(
          \hautl\infty G
          + \sum_{\fcti=1}^m \bigl(
            b_\fcti (\stoll{\vdim[\fcti]} + \ln 2) + \sqrt{\vdim[\fcti]}
          \bigr)
          + \frac{\vdim-1}2 \ln \lgr b
        \right)
        \\ & \qquad
        + b_\fct
        \left( \frac m\alpha \right)^\vdim (\vdim* + 1)
        \ln \left( \left( \frac m\alpha \right)^\vdim (\vdim* + 1) \right)
        + b_\fct \ln \binom{\deg T + \vdim*}{\vdim*}
      \end{align}
      où le nombre de \bsc{Stoll} est défini par~\eqref{e:def-stoll}.
  \end{enumthm}
\end{fact}

\begin{proof}
  C'est le théorème~7.1 page~149 de \cite{remivds}.
\end{proof}

Pour appliquer ce fait, nous devons fabriquer à partir de \( \faux**
\) une forme sur \( \proj{\vdim_1} \times \dots \times \proj{\vdim**} \) qui
conserve un indice comparable et dont on contrôlera degré et hauteur.
On considère à cet effet la projection linéaire \( \pi \) de \(
  (\projd)^m \) sur cet espace obtenue en
conservant les \( \vdim* + 1 \) premières coordonnées sur chaque facteur.
Cette projection fait apparaître \( \var \), ou plus précisément son image par
\( \vadapt* \circ \vaemb \), comme un revêtement (ramifié) de l'espace
d'arrivée, car ce plongement est adapté.  Algébriquement, ceci signifie que
l'anneau des coordonnées homogènes de \( \var \) est une extension de type
fini de \( \cdn[
  \vmp[1][0], \dots, \vmp[1][{\vdim[1]}]; \dots;
  \vmp[m][0], \dots, \vmp[m][{\vdim[m]}]
  ]
\). Le fait suivant montre que la norme \( N(\faux**) \) de \( \faux** \) dans
cette extension est une forme possédant les propriétés voulues.

\begin{fact} \label{f:nfaux}
  La forme \( N(\faux**) \in \cdn[
    \vmp[1][0], \dots, \vmp[1][{\vdim[1]}]; \dots;
    \vmp[m][0], \dots, \vmp[m][{\vdim[m]}]
    ] \) possède les propriétés suivantes :
  \begin{enumthm}
    \item \( \inda[\pi(\ex)] N(\faux**) \ge \inda** \faux** \) ;
    \item \( \deg N(\faux**) = (\prod\fctrange \vdeg*) \deg \faux** \) ;
    \item \(
        \hautl\infty{ N(\faux**) }
        \le
        (\prod\fctrange \vdeg*) \hautl\infty{ \faux** } + o(\delta)
      \).
  \end{enumthm}
\end{fact}

\begin{proof}
  C'est le résultat de la page~148 de~\cite{remivds}.
\end{proof}

On note désormais \( G = N(\faux**) \) puis \( \vdeg = \prod\fctrange \vdeg*
\) et \( b_\fct = D \delta \wts* (2\wtw* + \epsz) \).
On souhaite alors appliquer le fait~\vref{f:thm-prod} à \( G \) avec \( \alpha
  = m \Lambda^{-2f(u)} \) et \( x = \pi(\ex) \). Il s'agit tout d'abord
de vérifier que les hypothèses sont bien satisfaites.

Commençons avec l'indice : pour tout \( \imp \in (\N^{n+1})^m \) on a
\begin{align}
  \sum\fctrange \frac{ \lgr{\imp*} }{ b_\fct }
  & \ge
  \frac1{ 3 \delta \vdeg }
  \sum\fctrange
  \frac{ \lgr{\imp*} }{ \wtw* \wts* }
\end{align}
par définition de \( b \) et en observant que \( \epsz < 1 \), de sorte qu'en
utilisant successivement l'inégalité ci-dessus, le fait précédent et la
proposition~\vref{p:extra}, il vient :
\begin{equation}
  \indg b[x] G
  \ge
  \frac1{ 3 \delta \vdeg } \inda[\pi(\ex)] G
  \ge
  \frac1{ 3 \delta \vdeg } \inda** \faux**
  \ge
  \frac{ \eps \epsi }{ 384 (m-1) B D }
  \pmm.
\end{equation}
On tire maintenant parti de la définition~\eqref{e:def-epsi} de \( \epsi
\) puis de celle~\eqref{e:cst-vs-deg-prod} de \( \cst{vs-deg-prod} \) :
\begin{align}
  \frac{ 384 (m-1) m }{ \eps \epsi }
  & =
  \frac{
    384 (m-1) m
    \cdot N^{ \frac{m-1}{m-g} }
    \, (85 \cdot 5^m)^{ \frac{g}{m-g} }
  }{
    \eps^{ \frac{m}{m-g} }
  }
  \\ & \le
  \frac{
    5 m^2
    N^{ \frac{m}{m-g} }
    \, (85 \cdot 5^g)^{ \frac{m}{m-g} }
  }{
    \eps^{ \frac{m}{m-g} }
  }
  =
  \cst{vs-deg-prod}
  \label{e:ct-cst-vs-deg}
\end{align}
où l'on a utilisé le fait que \( 384 \le 5 \cdot 85 \).
Après avoir rassemblé les deux inégalités précédentes, il ne reste plus qu'à
exploiter les hypothèses~\eqref{e:varset-deg} et \eqref{e:varset-deg-prod}
pour obtenir
\begin{equation} \label{e:tp-ind}
  \indg b[x] G
  \ge
  \frac{ m }{ \cst{vs-deg-prod} B D }
  \ge
  \frac{ m }{ \Lambda^{2f(\vdim)} }
  =
  \alpha
\end{equation}
qui est précisément la première hypothèse à satisfaire.

Pour la deuxième, en utilisant la définition des \( \wtw* \) et le fait que \(
  \epsz < 1 \), puis deux fois~\eqref{e:wt-ratio} et enfin
l'hypothèse~\eqref{e:Vfar}, on écrit
\begin{equation} \label{e:ct-Vfar}
  \frac{ b_\fct }{ b_{\fct+1} }
  =
  \frac{
    (2\wtw* + \epsz) \wts*
  }{
    (2\wtw[\fct+1] + \epsz) \wts[\fct+1]
  }
  \ge
  \frac1m \,
  \frac{ \wts* }{ \wts[\fct+1] }
  \ge
  \frac1{4m} \,
  \frac{ \hautn{ \ex[\fct+1] } }{ \hautn{ \ex* } }
  \ge
  \frac{ \Vfar }{ 4m }
  \pmm.
\end{equation}
Par ailleurs, \( \bigl( \frac m\alpha \bigr)^\vdim = \Lambda^{2\vdim
    f(\vdim)} \) d'après le choix de \( \alpha \). Pour \( \vdim \) compris
entre \( m \) et \( mg \), cette quantité est majorée par \(
  \Lambda^{2m f(m)} \) vu la définition de \( f \). La
définition~\eqref{e:def-Vfar} de \( \Vfar \) implique alors directement que la
deuxième hypothèse est satisfaite.

Pour la troisième, on observe que l'expression
\begin{equation}
  \left( \frac{\ln(\vdim+1)}{2\vdim^2}\right)^\vdim
\end{equation}
est décroissante en \( \vdim \) tandis que \( \alpha / m =
  \Lambda^{-2f(\vdim)} \) est croissante que \( \vdim \), de sorte qu'il
suffit de vérifier que cette condition est satisfaite pour \( \vdim =
  mg \). C'est bien le cas puisque
\begin{equation} \label{e:tp-3}
  \left( \frac{\ln(mg + 1)}{2(mg)^2}\right)^{mg}
  \ge
  (\sqrt2 mg)^{-2mg}
  \ge
  \Lambda^{-2}
\end{equation}
d'après la définition~\eqref{e:def-Lambda} de \( \Lambda \) (troisième
argument du maximum) et \( f(mg) = 1 \).

On peut donc appliquer le fait~\vref{f:thm-prod} comme annoncé, on note \( T \)
la forme obtenue. On a alors \( \pi(\ex) \in \zeros T \), c'est-à-dire \(
  T(\cexa) = 0 \). On introduit une forme \( T'(\vmp*) = T\bigl(
    \vadapt*(\vmp*) \bigr) \) dont on va montrer qu'elle possède bien
les propriétés annoncés par la proposition~\vref{p:varset-notmin}.

Tout d'abord il est clair, par construction, que \( T'(\cex) = 0 \) et que \(
  T' \) ne s'annule pas identiquement sur \( \var \). Pour l'assertion sur le
degré, d'après le choix de \( \alpha \) on a
\begin{equation}
  \deg T' = \deg T \le \Lambda^{2\vdim f(\vdim)}
\end{equation}
qui est bien la majoration annoncée.

Concernant la hauteur, on commence par majorer \( \hautl\infty T \) en
fonction de \( \hautl\infty{ \faux** } \) ; le fait~\vref{f:thm-prod} donne
immédiatement la majoration suivante, compte tenu du fait~\vref{f:nfaux} et du
choix de \( \alpha \), et en remarquant que \( \ln \lgr b \) et \(
  \sqrt{\vdim*} \) sont négligeables devant \( \delta \) :
\begin{align}
  D \delta \wts* (2\wtw* + \epsz) \hautl\infty T
  & \le
  \vdim*
  \Lambda^{2\vdim f(\vdim)}
  \left(
    D \hautl\infty{ \faux** }
    + \sum_{\fcti=1}^m
    D \delta \wts[\fcti] (2\wtw[\fcti] + \epsz)
    (\stoll{\vdim[\fcti]} + \ln 2)
  \right)
  \\ & \qquad
  + D \delta \wts* (2\wtw* + \epsz)
  \Lambda^{2\vdim f(\vdim)}
  (\vdim* + 1)
  \ln \left(
    \Lambda^{2\vdim f(\vdim)}
    (\vdim* + 1)
  \right)
  \\ & \qquad
  + D \delta \wts* (2\wtw* + \epsz)
  \ln \binom{\Lambda^{2\vdim f(\vdim)} + \vdim*}{\vdim*}
  + o(\delta)
  \pmm.
\end{align}
On utilise alors l'estimation élémentaire
\begin{equation}
  \ln \binom{\Lambda^{2\vdim f(\vdim)} + \vdim*}{\vdim*}
  \le
  \vdim* \ln(\Lambda^{2\vdim f(\vdim)} + 1)
  \le
  \vdim* \Lambda^{2\vdim f(\vdim)}
\end{equation}
et le fait que \( 2 \le 2\wtw* + \epsz \le 2m \), puis on simplifie par
\( 2 \delta D \) et on met \( \vdim* \Lambda^{2\vdim f(\vdim)} \) en facteur
(en majorant \( \vdim* + 1 \) par \( 2\vdim* \) le cas échéant) pour obtenir
\begin{align*}
  \wtw* \wts* \hautl\infty T
  & \le
  \vdim*
  \Lambda^{2\vdim f(\vdim)}
  \Biggl(
    \frac1{2\delta} \hautl\infty{ \faux** }
    \\ & \qquad
    + m \biggl(
      \sum_{\fcti=1}^m \wts[\fcti] (\stoll{\vdim[\fcti]} + \ln 2)
      + \wts* \Bigl(
        2 \ln \bigl( \Lambda^{2\vdim f(\vdim)} (\vdim* + 1) \bigr) + 1
      \Bigr)
    \biggr)
  \Biggr)
  + o(1)
\end{align*}
On utilise alors~\eqref{e:maj-stoll} et
le fait que \( \vdim[\fcti] \le g \le m - 1 \) pour écrire
\( \stoll{\vdim[\fcti]} \le m \ln m \).  On utilise de plus le
lemme~\vref{l:sum-wts} et le fait que \( \wts* \le \wts[1] \) pour estimer le
terme de la dernière ligne ci-dessus par
\begin{align}
  2m \wts[1] \left(
    m \ln m + \ln 2 + 2\vdim \ln \Lambda^{f(\vdim)}
    + \ln m + \frac12
  \right)
\end{align}
puis on invoque~\eqref{e:ht-T-adhoc} pour obtenir
\begin{align}
  \wtw* \wts* \hautl\infty T
  & \le
  \vdim* \Lambda^{2\vdim f(\vdim)}
  \Biggl(
    \frac1{2\delta} \hautl\infty{ \faux** }
    + \frac14 \wts[1] \Lambda^{(1 + \frac1m) f(\vdim)}
  \Biggr)
  + o(1)
\end{align}
comme estimation finale de la hauteur de \( T \) en fonction de celle de \(
  \faux** \).

Utilisons maintenant le scolie~\vref{s:aux-co} pour exprimer cette hauteur en
fonction de \( \sum\fctrange \wtw* \wts* \hautl1{\varfc*} \).
\begin{align}
  \wtw* \wts* \hautl\infty T
  & \le
  \vdim* \Lambda^{2\vdim f(\vdim)}
  \Biggl(
    6 \sum_{\fcti=1}^m \wtw[\fcti] \wts[\fcti] \hautl1{\varfc[\fcti]}
    + \frac54 \wts[1] \cst{vs-ht} \Lambda^{(1 + \frac1m) f(\vdim)}
  \Biggr)
  + o(1)
  \\ & \le
  \vdim* \Lambda^{2\vdim f(\vdim)}
  \Biggl(
    6 \sum_{\fcti=1}^m \wtw[\fcti] \wts[\fcti] \hautl1{\varfc[\fcti]}
    + 2 \wts[1] \cst{vs-ht} \Lambda^{(1 + \frac1m) f(\vdim)}
  \Biggr)
  \label{e:ht-T-var}
\end{align}
où l'on a supposé \( \delta \) assez grand pour que le
terme en \( o(1) \) de la première ligne soit plus petit que la quantité
perdue en remplaçant \( \frac54 \) par \( 2 \) dans le terme précédent, qui ne
dépend pas de \( \delta \).

Passons maintenant à \( T' \). Par construction, en utilisant les propriétés
classiques de la hauteur, on a
\begin{align*}
  \hautl\infty{ T' }
  & \le
  \hautl\infty{ T }
  + \left(
    \deg T \Bigl( \hautl\infty{ \vadapt* } + \ln(n + 1) \Bigr)
    + \ln \binom{ \deg T + n + 1 }{ n + 1 }
  \right)
  \\ & \le
  \hautl\infty{ T }
  + \left(
    \Lambda^{2\vdim f(\vdim)} \Bigl( \ln B + \ln(n + 1) \Bigr)
    + 2\vdim (n + 1) \ln \Lambda^{f(\vdim)}
  \right)
  \pmm.
\end{align*}
On invoque alors~\eqref{e:ht-T'-adhoc}, puis on multiplie les deux membres par
\( \wtw* \wts* \) et, en remarquant que cette dernière quantité est majorée
par \( \wts[1] \), il vient :
\begin{equation}
  \wtw* \wts* \hautl\infty{ T' }
  \le
  \wtw* \wts* \hautl\infty{ T }
  +
  \wts[1] \cst{vs-ht} \Lambda^{(2\vdim + 1 + \frac1m) f(\vdim)}
  \pmm.
\end{equation}
Il ne reste plus qu'à substituer~\eqref{e:ht-T-var} dans l'estimation
précédente pour avoir
\begin{align*}
  \wtw* \wts* \hautl\infty{ T' }
  & \le
  \vdim* \Lambda^{2\vdim f(\vdim)}
  \Biggl(
    6 \sum_{\fcti=1}^m \wtw[\fcti] \wts[\fcti] \hautl1{\varfc[\fcti]}
    + 3 \wts[1] \cst{vs-ht} \Lambda^{(1 + \frac1m) f(\vdim)}
  \Biggr)
  \pmm,
\end{align*}
ce qui achève la preuve de la proposition~\vref{p:varset-notmin} en remarquant
que \( \vdim* \le g \).



\section{Valeurs des paramètres et estimations reliées}
\label{sec:vojta-adjust}

Les sections précédentes suivent l'ordre logique de la démonstration :
réduction du théorème à l'existence d'une forme motrice, puis construction
de celle-ci par la méthode de \TS : construction d'une forme auxiliaire,
extrapolation et application d'un théorème de multiplicité. Cependant, cet
ordre n'est pas celui dans lequel les valeurs des différents paramètres sont
déterminées ; la présente section a pour but de clarifier la façon de choisir
ces valeurs qui autrement risquent de paraître un peu « magiques » au moment
où elles sont introduites dans les sections précédentes.

Par ailleurs, nous établissons ici quelques majorations \lat{ad hoc} de
quantités apparaissant au cours de la preuve ; ces estimations sont plus ou
moins intimement liées aux valeurs choisies pour les différents paramètres, ou
un peu fastidieuses quoiqu'élémentaires, c'est pourquoi nous les avons
regroupées ici pour ne pas perturber le cours des sections précédentes.


\subsection{Méthode d'ajustement des paramètres}

Tout d'abord, \( \eps \) est fixé par l'énoncé, et la famille \( \ex \)
supposée contredire le théorème est fixée en premier, ainsi que la famille \(
  (\wt*)_\fct \) comme indiqué à la sous-section~\vref{sec:wt}.

Nous arrivons maintenant aux conditions définissant \( \varset(\ex) \). En
fait, celles-ci sont déterminées de façon à ce que \( \va^m \in
  \varset(\ex) \) et qu'on puisse, en coupant par des formes données par la
proposition~\vref{p:varset-notmin} produire d'autres formes qui restent dans \(
  \varset(\ex) \). On peut voir cette partie comme la construction d'une
suite de sous-variétés emboîtées, en estimant à chaque étape le degré et la
hauteur, qui dépendent des estimations à l'étape précédente ; \( \varset(\ex)
\) est défini de sorte à contenir tous les éléments de cette suite. Le plan
général de la construction est résumé par la figure~\vref{fig:vojta}.
\afterpage{% !TEX root = main.tex

\begin{landscape} \centering

\vspace*{\stretch{1}}

\begin{tikzpicture}[
  auto,
  hypo/.style={
    text height=0.7\baselineskip, text depth=0.3\baselineskip,
    draw,
  },
  result/.style={
    text height=0.7\baselineskip, text depth=0.3\baselineskip,
  },
  result2/.style={
    text height=1.7\baselineskip, text depth=0.3\baselineskip,
    align=center,
  },
  stop/.style={
    text height=0.7\baselineskip, text depth=0.3\baselineskip,
    draw, thick, rounded rectangle,
  },
  section/.style={
    draw, rounded corners,
    inner sep=0.5cm,
  },
  iter/.style={
    draw, rounded corners, thick, dashed,
    inner xsep=0.3cm, inner ysep=1cm,
  },
  dimdep/.style={
    draw, dotted, ->,
  },
  dep/.style={
    draw, ->,
  },
  dimloop/.style={
    to path={-- ++(0, #1) -| (\tikztotarget)}, rounded corners,
  },
  puiss/.style={
    draw, decorate, decoration={
      snake,
      amplitude=.4mm, segment length=2mm, 
      pre length=.5mm, post length=.5mm,
    },
  },
  ]

  % matrice pour les étapes de base
  \matrix [row sep=1cm, column sep=2.4cm] {
      \node[result]  (var-deg)   { \( \deg \var \) };
    & \node[result2] (aux-ex)    { Existence \\ et degrés };
    & \node[result2] (ind-ok)    { Indice \\ extrapolé };
    & \node[result2] (fo-ex)     { Existence \\ et degré };
    & \node[result]  (next-deg)  { \( \deg \var' \) };
    \\
      \node[result] (var-ht)     { \( \hautl{}{ \var } \) };
    & \node[result] (aux-ht)     { Hauteur };
    & \node[result] (ind-ht)     { À \( \hautl{}{ \ex[1] } \) fixé };
    & \node[result] (fo-ht)      { Hauteur };
    & \node[result] (next-ht)    { \( \hautl{}{ \var' } \) };
    \\
  };

  % nœuds supplémentaires
  \node[hypo] (in-deg)     [left=of var-deg] { \( \deg \va \) };
  \node[hypo] (in-ht)      [left=of var-ht]  { \( \hautl{}{ \va } \) };

  \node[hypo] (eps-zgti)   [above=of aux-ex] { \( \epsz \gg \epsi \) };

  \node[hypo] (Vcos)       [above=of ind-ok] { \( \Vcos \ll 1 \) };
  \begin{scope}[node distance=2mm]
    \node[hypo] (eps)      [right=of Vcos]   { \( \eps > 0 \) };
    \node[hypo] (eps-igtz) [left=of Vcos]    { \( \epsi \gg \epsz \) };
  \end{scope}

  \node[hypo] (Vfar)       [above=of fo-ex]  { \( \Vfar \gg 1 \) };

  \begin{scope}[node distance=3cm]
    \node[result] (out-ht) [below=of var-ht] { \( \hautl{}{ \ex* } \) };
    \node[stop]   (end)    [below=of aux-ht] { Contradiction };
    \node[hypo]   (Vbig)   [below=of ind-ht] { \( \Vbig \gg 1 \) };
  \end{scope}

  % dépendences
  \draw[dimdep] (in-deg)     to                  (var-deg);
  \draw[dimdep] (in-ht)      to                  (var-ht);

  \draw[dep]    (eps-zgti)   to                  (aux-ex);
  \draw[dep]    (aux-ex)     to                  (aux-ht);
  \draw[dep]    (var-deg)    to                  (aux-ht);
  \draw[dep]    (var-ht)     to                  (aux-ht);

  \draw[dep]    (aux-ex)     to                  (ind-ok);
  \draw[dep]    (var-deg)    to [bend left=18]   (ind-ok);
  \draw[dep]    (eps-igtz)   to                  (ind-ok);
  \draw[dep]    (Vcos)       to                  (ind-ok);
  \draw[dep]    (eps)        to                  (ind-ok);
  \draw[dep]    (ind-ok)     to                  (ind-ht);
  \draw[dep]    (aux-ht)     to                  (ind-ht);
  \draw[dep]    (Vbig.west)  to [bend left=30]   (ind-ht.190);

  \draw[dep]    (ind-ok)     to                  (fo-ex);
  \draw[dep]    (Vfar)       to                  (fo-ex);
  \draw[dep]    (fo-ex)      to                  (fo-ht);
  \draw[dep]    (aux-ht)     to [bend left=12]   (fo-ht);

  \draw[dep]    (fo-ex)      to                  (next-deg);
  \draw[dep]    (fo-ex)      to                  (next-ht);
  \draw[dep]    (fo-ht)      to                  (next-ht);
  \draw[dep]    (var-ht)     to [bend right=5]   (next-ht.190);

  \draw[dimdep] (next-deg)   to [dimloop=+2.6cm] (var-deg);
  \draw[dimdep] (next-ht)    to [dimloop=-1.5cm] (var-ht.280);

  \draw[dimdep] (var-ht.260) to                  (out-ht);

  \draw[dep]    (out-ht)     to                  (end);
  \draw[dep]    (Vbig)       to                  (end);

  \draw[puiss]  (eps-zgti)   to node [swap] {
    \( \scriptstyle m > g \) } (eps-igtz);

  % cadres et leurs étiquettes
  \node[section] (aux) [fit=(aux-ex)(aux-ht)] {};
  \node[section] (ind) [fit=(ind-ok)(ind-ht)] {};
  \node[section] (fo)  [fit=(fo-ex)(fo-ht)]   {};

  \node[below] (laux) at (aux.south) {Forme auxiliaire};
  \node[below] (lind) at (ind.south) {Extrapolation};
  \node[below] (lfo)  at (fo.south)  {Forme motrice};

  \node[iter] (rec) [fit=(var-ht)(var-deg)(Vcos)(next-deg)(next-ht)(laux)]
  {\hfill\null};
  \node[above] at (rec.north)  {
    Partie itérée.
    Entrée : \( \var = \va^m \) ;
    sortie : \( \var* = \set{ \ex* } \) pour un certain \( \fct \).
  };

\end{tikzpicture}

\vspace*{\stretch{1}}

\captionof{figure}{%
  Dépendances entre les différentes quantités en jeu.
  Toutes les quantités dans le cadre en traitillé qui dépendent indirectement
  de \( \deg Z \) en dépendent aussi directement ; les flèches correspondantes
  sont omises pour alléger.
}
\label{fig:vojta}

\vspace*{\stretch{1}}

\end{landscape}

\endinput
}

Remarquons qu'on pourrait tout à fait dans un premier temps se concentrer
uniquement sur le degré puisqu'on peut l'estimer indépendamment de la hauteur,
et qu'au contraire cette dernière dépend de façon cruciale du degré. Lors de
la construction de la fonction auxiliaire, on se contenterait de savoir que la
borne de hauteur ne dépend pas de \( \ex \), ce qui permettrait ensuite de
faire l'extrapolation « pour \( \hautn\ex \) assez grand » sans préciser, et
d'obtenir un indice extrapolé permettant d'appliquer le théorème du produit,
donnant ainsi l'estimation de degré au cran suivant.

Plus précisément, pour la construction de la fonction auxiliaire on introduit
les deux paramètres \( \epsz \) (contrôlant certains degrés de la fonction
auxiliaire) et \( \epsi \) (contrôlant son indice de construction). À ce
stade on a seulement besoin que \( \epsz \) soit suffisamment grand devant \(
  \epsi \) pour avoir~\eqref{e:ct-eps*-siegel} ; notons que cette condition ne
fait pas intervenir les degrés de \( \var \), uniquement ceux de la
fonction auxiliaire. L'estimation de hauteur de la fonction auxiliaire en
revanche dépend du maximum des degrés de \( \var \) et de ses hauteurs.

On peut alors mener les calculs de la sous-section~\vref{sec:vojta-extrap-core}
et aboutir à~\eqref{e:prod-lem-final} : la quantité qui y est notée \(
  \alpha \) ne dépend que du maximum des degrés de \( \var \) tandis que \(
  \beta \) dépend de ce maximum et de la hauteur de la forme auxiliaire, donc
en définitive de celle de \( \var \). Dans un premier temps, en attendant les
estimations finales, on peut tout à fait ne pas achever le calcul de ce \(
  \beta \) car son seul impact sera de fixer une valeur minimale pour \( \Vbig
\), paramètre qui peut être fixé en dernier.

Rappelons que pour pouvoir extrapoler, il est essentiel que cette
quantité \( \alpha \) soit négative. Rappelons sa valeur :
\begin{equation}
  \alpha
  =
  2 m \epsz
  + 3 (m - 1) \Vcos
  + \frac{8 (m - 1) n B \epsi}{\sigma}
  - \frac38 \eps \epsi
\end{equation}
et essayons de la relier à la structure de la fonction auxiliaire (on
considère ici la forme \( F \in \cdn[\vmp, \vmpi] \)). Le premier terme
provient du premier groupe de variables et peut être rendu petit en
diminuant les degrés en ces variables, contrôlés par \( \epsz \) ; le deuxième
correspond au deuxième groupe et peut être rendu petit en diminuant la hauteur
de \( \ex' \), contrôlée par \( \Vcos \) (une fois les \( \wt* \)
convenablement choisis). Le troisième terme correspond à ce qu'on perd en
dérivant, il dépend du degré de \( \var \) car c'est le long de cette variété
qu'on dérive ; la seule façon de le rendre petit est d'augmenter \( \sigma \)
c'est-à-dire de ne pas dériver à un ordre trop élevé. Le dernier terme enfin
est déterminé par la condition d'approximation~\eqref{e:Vapx} ; le seul moyen
de le rendre grand en valeur absolue est que l'indice de construction de la
forme auxiliaire, contrôlé par \( \epsi \), soit assez grand.

On se retrouve à ce stade avec deux conditions « allant en sens contraire »
sur \( \epsz \) et \( \epsi \) : le premier doit être, par rapport au second,
suffisamment grand pour permettre la construction de la forme auxiliaire, mais
suffisamment petit pour pouvoir extrapoler, plus précisément on veut
avoir~\eqref{e:ct-eps*-siegel} et~\eqref{e:epsz-ct-extra}. Un calcul
élémentaire montre qu'il est possible de satisfaire simultanément ces deux
contraintes pour peu que \( m > \max \vdim* = g \) (exposants
apparaissant dans la première condition).  C'est uniquement pour répondre à
cette exigence qu'on demande \( m > g \) dans l'énoncé du théorème ;
de même les deux conditions citées déterminent seules les définitions de \(
  \epsz \) et \( \epsi \) données en~\eqref{e:def-epsz} et~\eqref{e:def-epsi}.
(On utilisera souvent par ailleurs le fait que \( m \ge 2 \) et que \(
  \epsz < 1 \) pour simplifier certaines estimations, mais aucune de ces
inégalités n'est cruciale pour la bonne marche de la preuve.)

Maintenant que \( \epsi \) est fixé, on en déduit les valeurs souhaitables de
\( \Vcos \) et \( \sigma \) : le premier est choisi dans le seul but
d'avoir~\eqref{e:Vcos-ct-extra}, ce qui est assuré
par~\eqref{e:Vcos-ct-details} et le second est seulement déterminé
par~\eqref{e:ct-sigma}.

Nous arrivons maintenant à l'application du théorème du produit : l'indice de
la forme à laquelle nous souhaitons l'appliquer dépend d'une part de l'indice
extrapolé de la fonction auxiliaire, donc du maximum des degrés de \( \var \)
et d'autre part du produit de ces degrés. C'est principalement cet indice qui
détermine la valeur de la quantité notée \( \alpha \) dans la
section~\vref{sec:thm-prod}\footnote{Distincte de la quantité également notée
  \( \alpha \) dans la section précédente, que le lecteur veuille bien nous
  pardonner ce recyclage de notations entre plusieurs sections.}, et donc le
degré de la forme motrice.

La raison pour laquelle on introduit \( \cst{vs-deg-prod} \) dans la
condition~\eqref{e:varset-deg} est qu'on veut pouvoir écrire~\eqref{e:tp-ind}
et c'est ceci, plus précisément~\eqref{e:ct-cst-vs-deg}, qui détermine
sa valeur. Il est plus naturel d'absorber
les facteurs supplémentaires dans la majoration de \( B \) que celle de \( D
\) car la première quantité est \lat{a priori} plus petite (ceci se vérifie
sur leurs valeurs initiales : \( \deg \va \) et \( (\deg \va)^m \)
respectivement, ensuite les deux quantités seront multipliées à chaque cran
par le même facteur, à savoir \( \deg T \)).

Nous sommes maintenant en mesure de choisir \( \Lambda \) : les deux premiers
arguments dans le maximum assurent que \( \va^m \) satisfait bien les
conditions~\eqref{e:varset-deg} et~\eqref{e:varset-deg-prod}. Le troisième a
pour but de satisfaire la troisième hypothèse dans l'énoncé du théorème du
produit, voir~\eqref{e:tp-3}.

Par ailleurs, la volonté de satisfaire la deuxième hypothèse du théorème du
produit motive la définition de \( \Vfar \) : \eqref{e:ct-Vfar} est la seule
contrainte sur \( \Vfar \).

Penchons-nous à présent sur la démonstration du théorème à partir de la
proposition~\vref{p:varset-notmin} ; elle explique la définition de la fonction
\( f \) apparaissant en exposant dans les estimations, qui procède par
récurrence décroissante en partant de \( f(mg) = 1 \) et en
descendant avec la relation~\eqref{e:ct-f}. Nous avons à ce stade pleinement
expliqué (forme générale et définition de chaque quantité intervenant) les
estimations de degré~\eqref{e:varset-deg} et~\eqref{e:varset-deg-prod}.

Muni de ces informations, nous pouvons à présent remplacer les degrés
intervenant dans les estimations de hauteur. On estime donc successivement la
hauteur de la forme auxiliaire puis celle de la forme motrice en fonction
des hauteurs de \( \var \), et au final on obtient la relation de
récurrence~\eqref{e:vs-ht-rec}. Il ne reste plus qu'à dérouler cette
récurrence (en partant rappelons-le de \( Z = \va^m \)) pour obtenir la
forme finale de~\eqref{e:varset-ht}.

Enfin, de cette dernière relation on déduit~\eqref{e:ct-Vbig-final} qui
motive à choisir \( \Vbig \) assez grand pour contredire cette relation. On se
souvient alors que lors de l'extrapolation on devait également avoir \( \Vbig
\) assez grand pour que la quantité~\eqref{e:Vbig-ct-extra} soit négative.
Il suffit donc maintenant de choisir \( \Vbig \) répondant à ces deux
contraintes, ce qui n'est pas problématique puisqu'elles vont dans le même
sens : en fait, la première contrainte est la plus forte et implique
facilement la deuxième, comme le montrent les calculs
suivant~\eqref{e:Vbig-ct-extra}.


\subsection{Estimations diverses}

Commençons par quelques estimations générales ; on rappelle pour commencer que
\( n \ge 2 \), \( N \ge 3 \) et \( \deg \va \ge 3 \) d'après la
remarque~\vref{r:vaemb}, puis \( g \ge 1 \) et \( m \ge g + 1
  \ge 2 \). Ainsi, le dernier argument du maximum dans la
définition~\eqref{e:cst-vs-ht} de \( \cst{vs-ht} \) donne
\( \cst{vs-ht} \ge n \ln 3 \ge 2 \).
De plus, la définition~\eqref{e:def-Lambda} de \( \Lambda \) implique
immédiatement, en considérant le deuxième argument du maximum, que
\begin{equation}
  \Lambda
  \ge
  5 \cdot m^2 \cdot 85 \cdot 3 \cdot 5^g \cdot 3
  \ge
  3\,825 \, m^2 \, 5^g
  \ge
  19\,125 \, m^2
  \ge
  76\,500
  \pmm.
\end{equation}
Remarquons au passage que si le deuxième argument dans le maximum domine
largement pour \( m \) et \( g \) petits, le troisième croît en
revanche bien plus vite quand ces deux quantités augmentent.

On fera fréquemment usage des deux estimations suivantes :
\begin{equation}
  \ln B
  \le
  \ln(\Lambda^{f(\vdim)})
  \le
  m \Lambda^{\frac1m\,f(\vdim)}
  \qquad\text{et}\qquad
  \ln B
  \le
  B
  \le
  \frac{ \Lambda^{f(\vdim)} }{ 1\,275 \cdot m^2 \cdot 5^g }
  \pmm,
\end{equation}
qui découlent de~\eqref{e:varset-deg} et dont la première utilise l'inégalité
élémentaire \( \ln(x) \le \alpha \, x^{1/\alpha} \), valable pour tout
\( \alpha > 0 \).

Avant de passer aux différentes estimations \lat{ad hoc}, établissons encore
une estimation qui nous sera utile plusieurs fois :
\begin{align}
  B (g + 1) \ln (B (n+1))
  & \le
  B (g + 1) \bigl(
    m \Lambda^{\frac1m\,f(\vdim)}
    + \cst{vs-ht}
  \bigr)
  \\ & \le
  m^2 B \cst{vs-ht} \Lambda^{\frac1m\,f(\vdim)}
  \\ & \le \label{e:adhoc-gen}
  \frac1{1\,275} \cst{vs-ht} \Lambda^{(1 + \frac1m) f(\vdim)}
  \pmm.
\end{align}
Signalons enfin qu'en général on ne cherchera pas à obtenir la plus grande
précision possible sur le facteur constant, on se contentera de la valeur
utile pour chacune des situations considérées.

\medskip

Intéressons-nous maintenant à notre première estimation \lat{ad hoc} :
\begin{equation} \label{e:B-siegel}
  \cst{vs-ht}/2 + \ln((n+1)!) + n \ln B
  + 3 \ln(B) \bigl( B(g+1) + 1 \bigr) + 3 \ln(2)
  \le
  \frac14 \cst{vs-ht} \Lambda^{(1 + \frac1m) f(\vdim)}
  \pmm.
\end{equation}
Pour le deuxième terme, on a par définition :
\(
  \ln((n+1)!)
  \le
  n \ln (n+1)
  \le
  \cst{vs-ht}
\)
de sorte que la somme des deux premiers termes est au plus \( \frac32
  \cst{vs-ht} \).
En utilisant de plus l'estimation~\eqref{e:adhoc-gen} ci-dessus, il vient :
\begin{align}
  & \cst{vs-ht}/2 + \ln((n+1)!)
  + 3 \ln(B) B(g+1) + n \ln B + 3\ln B + 3 \ln(2)
  \\ & \qquad \le
  \frac{3\Lambda}{2\cdot1\,275} \cst{vs-ht}
  + \frac3{1\,275} \cst{vs-ht} \Lambda^{(1 + \frac1m) f(\vdim)}
  + \cst{vs-ht} \cdot \frac{ \Lambda^{f(\vdim)} }{ 1\,275 }
  + 3 \cdot \frac{ \Lambda^{f(\vdim)} }{ 1\,275 }
  + \frac{ \Lambda }{ 1\,275 }
\end{align}
qui implique largement l'estimation annoncée.

\medskip

La prochaine majoration \lat{ad hoc} que nous montrons est
\begin{equation} \label{e:ht-T-adhoc}
  2m \left(
    m \ln m + \ln 2 + 2\vdim \ln \Lambda^{f(\vdim)}
    + \ln m + \frac12
  \right)
  \le
  \frac14 \Lambda^{ (1 + \frac1m) f(\vdim) }
  \pmm.
\end{equation}
On commence par utiliser le fait que \( \Lambda \ge (\sqrt2
  mg)^{mg} \) pour écrire
\begin{equation}
  m \ln \Lambda
  \ge
  m^2 g \ln( \sqrt2 m g )
  \ge
  m^2 \ln m + \frac{ m^2 }2 \ln 2
  \ge
  m \ln m + \ln m + \frac12 + \ln 2
\end{equation}
où l'on utilise le fait que \( m \ge 2 \) pour la dernière minoration.
Comme de plus \( \vdim \ge m \), la parenthèse dans~\eqref{e:ht-T-adhoc}
est majorée par \( 3\vdim \ln \Lambda^{f(\vdim)} \) et comme de plus \( \vdim
  \le mg \) le membre de gauche de~\eqref{e:ht-T-adhoc} est majoré
successivement par
\begin{equation}
  6 m^2 g \ln \Lambda^{f(\vdim)}
  \le
  6 m^3 g \Lambda^{\frac1m f(\vdim)}
  \le
  \frac14 \Lambda^{(1 + \frac1m) f(\vdim)}
  \cdot \frac{ 24 m^4 }{ \Lambda }
\end{equation}
qui implique l'estimation annoncée pour peu que le dernier facteur soit
inférieur à \( 1 \). Or on a vu que \( \Lambda \ge 19\,125 m^2 \) et un
simple calcul montre que cette quantité est supérieure à \( 24 m^4 \)
tant que \( m \le 28 \) ; par ailleurs pour \( m > 28 \) on a
immédiatement \( (\sqrt2 \, m)^m \ge m^5 \ge 24 m^4 \).

\medskip

Prouvons maintenant que
\begin{equation} \label{e:ht-T'-adhoc}
  \Lambda^{2\vdim f(\vdim)} \Bigl( \ln B + \ln(n + 1) \Bigr)
  + 2\vdim (n + 1) \ln \Lambda^{f(\vdim)}
  \le
  \cst{vs-ht} \Lambda^{(2\vdim + 1) f(\vdim)}
  \pmm.
\end{equation}
Pour commencer, on a
\begin{align} \label{e:extra-beta-adhoc}
  \ln B + \ln(n + 1)
  \le
  \ln B + \cst{vs-ht}
  \le
  \frac{\Lambda^{f(\vdim)}}{1\,275}
  + \cst{vs-ht} \frac\Lambda{1\,275}
  \le
  \frac17 \cst{vs-ht} \Lambda^{f(\vdim)}
  \pmm.
\end{align}
Par ailleurs
\begin{equation}
  2 \vdim (n + 1) \ln \Lambda^{f(\vdim)}
  \le
  4 \cst{vs-ht} m^2 g \Lambda^{\frac1m f(\vdim)}
  \le
  \frac12 \cst{vs-ht} \Lambda^{(1 + \frac1m) f(\vdim)}
\end{equation}
qui aboutit largement à l'estimation annoncée.

\medskip

On a également, en utilisant~\eqref{e:adhoc-gen} :
\begin{align}
  & 2 g B ( \ln B + \ln( n + 1 ) ) + \ln(8g) + 3 \ln B
  \\ & \qquad \le
  \frac2{1\,275} \cst{vs-ht} \Lambda^{(1 + \frac1m) f(\vdim)}
  + \frac{ 4 \Lambda^{f(\vdim)} }{1\,275}
  \\ & \qquad \le \label{e:ht-Pv-adhoc}
  \cst{vs-ht} \Lambda^{(1 + \frac1m) f(\vdim)}
  \pmm.
\end{align}
Par ailleurs, en utilisant à nouveau~\eqref{e:adhoc-gen} :
\begin{align}
  & \frac32 \ln(n+1)
  + n \ln(Bn)
  + \htcmp
  + B (g + 1) \ln (B (n+1))
  \\ & \qquad \le
  \frac32 \cst{vs-ht}
  + \cst{vs-ht} + \cst{vs-ht} \ln B
  + \cst{vs-ht} 
  + \frac1{1\,275} \cst{vs-ht} \Lambda^{(1 + \frac1m) f(\vdim)}
  \\ & \qquad \le
  4 \cst{vs-ht} \cdot \frac\Lambda{1\,275}
  + \cst{vs-ht} \frac{ \Lambda^{f(\vdim)} }{ 1\,275 }
  + \frac1{1\,275} \cst{vs-ht} \Lambda^{(1 + \frac1m) f(\vdim)}
  \\ & \qquad \le \label{e:ratfh1a-adhoc}
  \frac16 \cst{vs-ht} \Lambda^{(1 + \frac1m) f(\vdim)}
  \pmm.
\end{align}


\section{Déduction du cas général}
\label{sec:vojta-coro}

Le théorème~\vref{t:vojta-div} fait l'hypothèse que la variété dont on étudie
les approximations exceptionnelles est l'hyperplan \( \vp[0] = 0 \) dans le
plongement considéré. Nous allons en déduire un résultat similaire sans cette
hypothèse, par des changements de plongement successifs.  Si \( \divi \) est
un hyperplan quelconque, on se ramène au cas précédent par un changement de
coordonnées linéaire. Si c'est une hypersurface quelconque, on se ramène à un
hyperplan par un plongement de \bsc{Veronese}. Enfin, si c'est une variété
quelconque, on se ramène au cas précédent en considérant une hypersurface
convenable la contenant.


\subsection{Hyperplans quelconques}

On commencera par un hyperplan presque quelconque, en excluant seulement un
cas particulier. On montrera ensuite qu'on peut toujours l'éviter, avec de
plus un certaine marge de sécurité évitant l'explosion de certains constantes,
quitte à changer l'ensemble des places considérées dans l'hypothèse
d'approximation. En attendant, le lemme suivant donne un changement de base
dans le cas favorable et estime son coût sur les différentes quantités en jeu.

\begin{lem} \label{l:plin}
  Soient \( \vaemb \colon \va \to \projd \) un plongement projectif,
  \( p_0 \) le point de coordonnées homogènes \( (1 {:} 0 {:} \cdots {:} 0) \)
  et \( \divi \) un hyperplan ne contenant pas \( p_0 \).  Il existe un
  plongement \( \vaemb' \colon \va \to \projd \) dans lequel \( \divi \) admet
  \( \vp[0] = 0 \) comme équation et tel que :
  \begin{enumthm}
    \item \(
        \distv[v, \vaemb']{x}{\divi}
        \le
        \distv[v, \vaemb]{x}{\divi}
        \cdot \distv[v, \vaemb]{\divi}{p_0}^{-2}
        \, (n + 1)^{\dv/2}
      \) ; \label{i:plin-dv}
    \item \(
        \hautm{2, \vaemb'}{x}
        \le
        \hautm{2, \vaemb}{x}
        \cdot \hautm{2, \vaemb}{E}
        \sqrt{n + 1}
      \) ; \label{i:plin-ht-pt}
    \item \(
        \hautl{1, \vaemb'}{\va}
        \le
        \hautl{1, \vaemb}{\va}
        + \hautl{1, \vaemb}{ E } (g+1) \deg \va
      \) ; \label{i:plin-ht-va}
    \item \(
        \htcmp[\vaemb']
        =
        \htcmp[\vaemb]
        + \hautl{2, \vaemb} E
        + \frac12 \ln(n + 1)
      \) ; \label{i:plin-htcmp}
    \item \(
        \hlclab[\vaemb']
        =
        \hlclab[\vaemb]
        + \frac52 \hautl{1, \vaemb} E
      \).
  \end{enumthm}
  Par ailleurs le cardinal \( \nclmaps \) de l'ensemble de cartes introduit
  à la section\vref{sec:vaemb} et la dimension ambiante sont inchangés.
\end{lem}

\begin{proof}
  Notons \( L \) une équation de \( E \), qu'on peut supposer de la forme
  \( L(\vp) = \vp[0] + l_1 \vp[1] + \dots + l_n \vp[n] \). Remarquons
  de suite que \( \distv[v, \vaemb]{\divi}{p_0} = \nv2 L ^{-1} \).
  Par ailleurs, posons
  \begin{equation}
    M = \begin{pmatrix}
      1      & -l_1   & -l_2   & \dots  & -l_n \\
      0      & 1      & 0      & \dots  & 0 \\
      \vdots & \ddots & \ddots & \ddots & \vdots \\
      \vdots &        & \ddots & 1      & 0 \\
      0      & \hdotsfor2      & 0      & 1 \\
    \end{pmatrix}
    \quad\text{soit}\quad
    M^{-1} = \begin{pmatrix}
      1      & l_1    & l_2    & \dots   & l_n \\
      0      & 1      & 0      & \dots  & 0 \\
      \vdots & \ddots & \ddots & \ddots & \vdots \\
      \vdots &        & \ddots & 1      & 0 \\
      0      & \hdotsfor2      & 0      & 1 \\
    \end{pmatrix}
  \end{equation}
  de sorte que toutes les colonnes de \( M \) sauf la première sont les
  coordonnées homogènes (dans le plongement \( \vaemb \)) de
  points de \( \divi \). Plus précisément, en relevant dans \( \cdn^{n+1}
  \), ces vecteurs constituent une base du relevé de \( \divi \). On remarque
  que les normes des lignes et des colonnes de \( M \) comme de \( M^{-1} \)
  sont majorées par la norme de \( L \).

  On note alors \( \chi \in \GL_{n+1}(\cdn) \) la transformation linéaire
  associée à \( M^{-1} \) (dans la base canonique) et on note de même la
  transformation linéaire de \( \projd \) qui en découle en passant au
  quotient. On pose alors \( \vaemb' = \chi \circ \vaemb \). On
  note \( L'(\vp) = L(M(\vp)) = \vp[0] \) ; par construction \( L' \)
  est une équation de \( \divi \) dans ce nouveau plongement, lequel
  satisfait donc à la première condition.

  Soit maintenant \( x \) un point de \( \va \), dont on note \( C_x \in
    \Qbar^{n+1} \) les coordonnées dans le plongement \( \vaemb \) et \(
    C'_x = M^{-1} C_x \) les coordonnées dans le plongement \( \vaemb' \). Vu
  les expressions de \( M \) et \( M^{-1} \) on a immédiatement
  \begin{align} \label{e:plin-norm1}
    \nv1{ C _x } & \le \nv1{ L } \nv1{ C'_x }
    &
    \nv1{ C'_x } & \le \nv1{ L } \nv1{ C _x }
    \\ \label{e:plin-norm}
    \nv\infty{ C _x } & \le \nv2{ L } \nv2{ C'_x }
    &
    \nv\infty{ C'_x } & \le \nv2{ L } \nv2{ C _x }
  \end{align}
  et la dernière inégalité donne aussitôt le point~\vref{i:plin-ht-pt} en
  comparant \( \nv\infty\truc \) à \( \nv2\truc \).

  Par ailleurs, en appliquant la définition de la distance d'un point à un
  hyperplan, il vient
  \[
    \distv[v, \vaemb']{x}{\divi}
    =
    \frac{
      \av{ L'( C'_x ) }
    }{
      \nv2{ L' } \nv2{ C'_x }
    }
    =
    \distv[v, \vaemb]{x}{\divi}
    \frac{ \nv2 L }{ \nv2{ L' } }
    \frac{ \nv2{ C_x } }{ \nv2{ C'_x } }
  \]
  qui donne le point~\vref{i:plin-dv} en remarquant que \( \nv2{ L' } = 1 \)
  par construction et en utilisant la comparaison ci-dessus pour majorer le
  dernier facteur.

  Par ailleurs, si \( \chow\va \) est une forme de \bsc{Chow} de \( \va \)
  dans le plongement \( \vaemb \), on obtient une forme de \bsc{Chow} \(
    \chow\va' \) de \( \va \) dans \( \vaemb' \) en composant \( \chow\va \)
  avec (la transposée de) \( M^{-1} \) sur chacun des \( g + 1 \)
  groupes de \( n + 1 \) variables, de sorte que,
  si \( \deg \va \) désigne le degré de \( \va \) en tant que sous-variété de
  \( \projd \) (qui est évidemment le même dans les plongements \( \vaemb \)
  et \( \vaemb' \)) c'est-à-dire le degré de \( \chow\va \) en chacun des
  groupes de variables, on a
  \begin{equation}
    \nv1{ \chow\va' }
    \le
    \nv1{ \chow\va }
    \nv1{ L }^{(g+1) \deg \va}
  \end{equation}
  qui établit~\vref{i:plin-ht-va}.

  Par ailleurs, les comparaisons~\eqref{e:plin-norm} donnent immédiatement
  \begin{equation}
    \abs{ \hautl{2, \vaemb} x - \hautl{2, \vaemb'} x }
    \le
    \hautl2 L + \frac12 \ln(n + 1)
  \end{equation}
  d'où, en constatant que les hauteurs normalisées sont les mêmes dans \(
    \vaemb \) et \( \vaemb' \)
  \begin{equation}
    \abs{ \hautl{2, \vaemb'} x - \hautn x }
    \le
    \htcmp[\vaemb] + \hautl2 L + \frac12 \ln(n + 1)
  \end{equation}
  qui donne directement~\vref{i:plin-htcmp}.

  Enfin, si \( (\rmclab{a}{b}{\clmap}[\vaemb])_{\clmap \in \clmaps} \) est
  une famille de \( (n+1) \)-uplets de formes donnée par le
  lemme~\vref{l:hmclab}, on pose
  \begin{equation}
    \rmclab{a}{b}{\clmap}[\vaemb']
    =
    M^{-1} \rmclab{a}{b}{\clmap}[\vaemb](MX, MY)
  \end{equation}
  de sorte que la famille
  \( (\rmclab{a}{b}{\clmap}[\vaemb])_{\clmap \in \clmaps} \) satisfait aux
  mêmes hypothèses dans le plongement \( \vaemb' \), sauf
  peut-être~\eqref{e:clab-norm} et~\eqref{e:clab-loc}. Montrons que ces deux
  conditions sont satisfaites avec \( \hmclab[\vaemb']* = \hmclab[\vaemb]*
    \cdot \nv1 L ^{5/2} \), ce qui impliquera le dernier point de l'énoncé.

  Pour la première, en utilisant les propriétés de la norme \( 1 \), on a
  immédiatement
  \begin{align}
    \nnv1{ \rmclab{a}{b}{\clmap}[\vaemb'] }
    \le
    \nv1{ L }
    \nnv1{ \rmclab{a}{b}{\clmap}[\vaemb] }
    \nv1{ L }^{2(a^2 + b^2)}
    \le
    \hmclab[\vaemb]*^{a^2 + b^2} \nv1{ L }^{\frac52(a^2 + b^2)}
    =
    \hmclab[\vaemb']*^{a^2 + b^2}
  \end{align}
  en remarquant que pour \( (a, b) \ge (1, 1) \) on a
  \( 1 + 2(a^2 + b^2) \le \frac52 (a^2 + b^2) \).
  Pour la deuxième, en utilisant~\eqref{e:plin-norm1}, on écrit de même
  \begin{align}
    \frac{
      \nv1{ \rmclab{a}{b}{\clmap}[\vaemb'](C'_x, C'_y) }
    }{
      \nv1{C'_x}^{2a^2} \nv1{C'_y}^{2b^2}
    }
    & =
    \frac{
      \nv1{ M^{-1} \rmclab{a}{b}{\clmap}[\vaemb](C_x, C_y) }
    }{
      \nv1{C_x}^{2a^2} \nv1{C_y}^{2b^2}
    }
    \left( \frac{ \nv1{C_x} }{ \nv1{C'_x} } \right)^{2a^2}
    \left( \frac{ \nv1{C_y} }{ \nv1{C'_y} } \right)^{2b^2}
    \\ & \ge
    \frac{
      \nv1{ \rmclab{a}{b}{\clmap}[\vaemb](C_x, C_y) }
    }{
      \nv1{C_x}^{2a^2} \nv1{C_y}^{2b^2}
    }
    \, \frac1{
      \nv1 L ^{1 + 2(a^2 + b^2)}
    }
    \ge
    \frac{
      \hmclab[\vaemb]*^{-(a^2 + b^2)}
    }{
      \nv1 L ^{\frac52(a^2 + b^2)}
    }
    \pmm.
  \end{align}

  Enfin, il est clair que \( N = \card \clmaps \) reste inchangé car
  l'ensemble de cartes considéré est le même (les cartes sont des ouverts de
  \( \va^2 \), pas de son image dans un plongement) ; par ailleurs \( n \)
  n'est bien évidemment pas modifié par un changement de coordonnées linéaire.
\end{proof}

On en déduit une version un peu plus générale du théorème~\vref{t:vojta-div}.

\begin{coro} \label{c:vojta-div-dv}
  Soient \( \vaemb \colon \va \embedin \projd \) un plongement projectif,
  \( \divi \) un hyperplan de \( \projd \), \( p_0 \) le point de \( \projd \)
  de coordonnées homogènes \( (1 {:} 0 {:} \cdots {:} 0) \) et \( \eps > 0 \)
  un nombre réel.  Il n'existe dans \( \va(\Qbar) \) aucune famille de points
  \( x_1, \dots, x_m \) avec \( m \ge g + 1 \) satisfaisant
  simultanément aux conditions suivantes, où les notions de distance et de
  hauteur sont relatives au plongement \( \vaemb \) :
  \begin{align}
    0 < \distv{x_i}{\divi}
    & <
    \frac{ \distv{p_0}{\divi}^2 }{ (n + 1)^{\dv/2} }
    \hautm2{x_i}^{-\wtapx \eps}
    \quad \forall v \in \placesapx
    \\
    \hautn{x_1}
    & > 4\cst{ht-v-div-dv} \Lambda_1^{(1 + \frac1m)
      \prod_{i=m}^{mg} (2i + 1)}
    \\
    \hautn{x_i} & > \hautn{x_{i-1}} \cdot
    4 m \Lambda_1^{2m \prod_{i=m+1}^{mg} (2i + 1)}
    \\
    \cos(x_i, x_j) & > 1 -
    \frac1{ m }
    \left(
      \frac{ \eps }{ 86 \nclmaps \cdot 5^g }
    \right)^{ \frac{m}{m-g} }
  \end{align}
  avec
  \begin{align}
    \Lambda_1
    & = \max \bigl(
      (\deg \va)^m,
      5 m^2
      \bigl(
        86 \nclmaps \cdot 5^g \cdot \eps^{-1}
      \bigr)^{ \frac{m}{m-g} }
      \deg \va,
      (\sqrt2 m g)^{mg}
    \bigr)
    \\
    \newcst[\vaemb]{ht-v-div-dv}
    & = \max \bigl(
      \hautl1{\va} + \hautl1{ E } (g+1) \deg \va,
      \hlclab + \frac52 \hautl1 E,
      \\ & \qquad
      \htcmp + \hautl2 E + \frac12 \ln(n + 1),
      n\ln(n + 1)
    \bigr)
    \pmm.
  \end{align}
\end{coro}

Avant de passer à la démonstration, examinons les différences entre cet énoncé
et le théorème~\vref{t:vojta-div} : les deux principales sont une hypothèse
moins restrictive sur \( \divi \) compensée par une hypothèse d'approximation
plus contraignante, en particulier si \( \divi \) est proche de \( p_0 \).
Les seuls autres changements concernent les valeurs des constantes :
\( \cst{vs-ht} \) est remplacée par \( \cst{ht-v-div-dv} \) avec une
définition différente faisant intervenir la hauteur de \( E \), et le nombre
\( 85 \) est remplacé par \( 86 \) partout où il apparaît en même temps que \(
  \eps \).

\begin{proof}
  On remarque pour commencer que le résultat est vide dans le cas où \( p_0
    \in E \), car la première condition n'est jamais satisfaite. Dans le cas
  contraire, on considère le plongement \( \vaemb' \) donné par le
  lemme~\vref{l:plin}
  et on prévoit d'appliquer le théorème~\vref{t:vojta-div} à \( E \)
  et \( \vaemb' \) avec \( \eps' = 85 \eps / 86 \) pour une raison qui
  deviendra bientôt claire. Il s'agit de vérifier que les hypothèses du
  corollaire impliquent bien celles du théorème.

  Pour l'hypothèse principale~\eqref{e:Vapx}, on utilise d'abord le
  point~\vref{i:plin-dv} du lemme, puis l'hypothèse principale du présent
  corollaire, la définition de \( \eps' \) et enfin le
  point~\vref{i:plin-ht-pt} du lemme :
  \begin{align}
    \distv[v, \vaemb']{x_i}{\divi}
    & \le
    \distv[v, \vaemb]{x_i}{\divi}
    \cdot \distv[v, \vaemb]{\divi}{p_0}^{-2}
    \, (n + 1)^{\dv/2}
    \\ & \le
    \hautm{2, \vaemb}{x_i}^{-\wtapx \eps}
    \\ & \le
    \hautm{2, \vaemb}{x_i}^{-\wtapx \eps'}
    \cdot \hautm{2, \vaemb}{x_i}^{-\wtapx \eps/86}
    \\ & \le
    \hautm{2, \vaemb'}{x_i}^{-\wtapx \eps'}
    \cdot \hautm{2, \vaemb}{E}^{\wtapx \eps'}
    (n + 1)^{\wtapx \eps'/2}
    \cdot \hautm{2, \vaemb}{x_i}^{-\wtapx \eps/86}
    \\ & \le
    \hautm{2, \vaemb'}{x_i}^{-\wtapx \eps'}
    \Bigl(
      \hautm{2, \vaemb}{E}^{85}
      (n + 1)^{85/2}
      /
      \hautm{2, \vaemb}{x_i}
    \Bigr)^{\wtapx \eps/86}
    \pmm,
  \end{align}
  qui donne le résultat voulu pour peu que la quantité dans la dernière
  parenthèse soit inférieure à \( 1 \). Or, la deuxième hypothèse entraîne
  largement cette inégalité, par exemple :
  \(
    \hautl{2, \vaemb}{x_i}
    \ge
    2 \cst{ht-v-div-dv} \Lambda_1
    \ge
    85 ( \hautl{1, \vaemb}{E} + \frac12 \ln (n + 1) )
  \)
  en utilisant de diverses façon la définition de \( \cst{ht-v-div-dv} \) et
  le fait que \( \Lambda_1 \ge 85 \).

  Les autres hypothèses sont plus faciles, car un changement de coordonnées
  linéaire ne modifie en rien la hauteur normalisée, c'est-à-dire la forme
  quadratique définissant la géométrie de l'espace de \MoW. Les conditions
  s'écrivant dans cet espace sont donc inchangées en passant de \( \vaemb \) à
  \( \vaemb' \).  Ainsi, les hypothèses~\eqref{e:Vcos} et~\eqref{e:Vfar} sont
  satisfaites car les hypothèses correspondantes du corollaire sont les mêmes,
  compte tenu du changement de \( \eps \). Il en est de même
  pour~\eqref{e:Vbig} en remarquant de plus que \( \cst{ht-v-div-dv} >
    \cst{vs-ht}[\vaemb'] \) par définition et en utilisant les points restants
  du lemme précédent.

  On peut donc appliquer le théorème~\vref{t:vojta-div} comme prévu et en
  déduire qu'il n'existe pas de famille de points satisfaisant aux hypothèses
  du présent énoncé.
\end{proof}

Le défaut du résultat précédent est que l'hypothèse d'approximation est
d'autant plus forte (et donc le résultat d'autant plus faible) que \( p_0 \)
est proche de \( \divi \), jusqu'à obtenir une hypothèse impossible à
satisfaire (donc un résultat vide) lorsque \( p_0 \in \divi \). Le corollaire
suivant propose deux options pour contourner ce problème : la première
consiste à considérer des permutations de coordonnées pour éviter que cette
condition ne devienne trop restrictive, mais a un coût sur les autres
constantes (sauf si \( \card \placesapx = 1 \)) ; la deuxième consiste à
remarquer que les différentes distances en question, dès qu'elles sont non
nulles, sont liées à la hauteur de \( \divi \) -- cette version prendra
plus de sens lorsque l'hypothèse d'approximation sera modifiée en utilisant un
produit sur toutes les places de \( \placesapx \) à la
section~\vref{sec:ha-prod}.

\begin{coro} \label{c:vojta-div-any}
  Soient \( \vaemb \colon \va \embedin \projd \) un plongement projectif,
  \( \divi \) un hyperplan de \( \projd \) et \( \eps > 0 \) un nombre réel.
  Il n'existe dans \( \va(\Qbar) \) aucune famille de points \( x_1,
    \dots, x_m \) avec \( m \ge g + 1 \) satisfaisant
  simultanément aux conditions suivantes :
  \begin{align}
    0 < \distv{x_i}{\divi}
    & <
    \alpha_v^{-1}
    \hautm2{x_i}^{-\wtapx \eps}
    \quad \forall v \in \placesapx
    \\
    \hautn{x_1}
    & > 4\cst{ht-v-div-dv} \Lambda_2^{(1 + \frac1m)
      \prod_{i=m}^{mg} (2i + 1)}
    \\
    \hautn{x_i} & > \hautn{x_{i-1}} \cdot
    4 m \Lambda_2^{2m \prod_{i=m+1}^{mg} (2i + 1)}
    \\
    \cos(x_i, x_j) & > 1 -
    \frac1{ m }
    \left(
      \frac{ \eps }{ 86 \nclmaps \cdot 5^g s }
    \right)^{ \frac{m}{m-g} }
  \end{align}
  avec
  \begin{align}
    \Lambda_2
    & = \max \bigl(
      (\deg \va)^m,
      5 m^2
      \bigl(
        86 \nclmaps \cdot 5^g \cdot \eps^{-1} s
      \bigr)^{ \frac{m}{m-g} }
      \deg \va,
      (\sqrt2 m g)^{mg}
    \bigr)
    \\
    \cst{ht-v-div-dv}
    & = \max \bigl(
      \hautl1{\va} + \hautl1{ E } (g+1) \deg \va,
      \hlclab + \frac52 \hautl1 E,
      \\ & \qquad
      \htcmp + \hautl2 E + \frac12 \ln(n + 1),
      n\ln(n + 1)
    \bigr)
  \end{align}
  et au choix :
  \begin{enumthm}
  \item \( s = \min(n+1, \card \placesapx) \) et \( \alpha_v = (n+1)^{3\dv/2}
    \) ;  ou bien
  \item \( s = 1 \) et \( (\alpha_v)_v \) une certaine famille de réels tous
    supérieurs ou égaux à \( 1 \), satisfaisant
    \( \prod\placerange \alpha_v^\degv \le \hautm2 \divi ^2 \cdot \sqrt{n+1} \).
  \end{enumthm}
\end{coro}

\begin{proof}
  Commençons par la deuxième option, qui est la plus simple.  On considère les
  points \( p_i \) définis comme \( p_0 \) mais avec le \( 1 \) en \( i \)-ème
  position ; on peut certainement choisir un \( i \) tel que \( p_i \not\in
    \divi \).  On pose alors
  \begin{equation}
    \alpha_v
    =
    \frac{ (n + 1)^{\dv/2} }{ \distv{p_i}{\divi}^2 }
    \quad \forall v \in \placesapx
    \pmm,
  \end{equation}
  ce qui a un sens car le dénominateur n'est pas nul, et on a bien \( \alpha_v
    \ge 1 \). Par ailleurs, on a
  \begin{equation}
    \prod\placerange  \distv{p_i}{\divi}
    \le
    \prod_{v \in M_\cdn} \distv{p_i}{\divi}
    =
    \hautm2 \divi ^{-1}
  \end{equation}
  vu l'expression de la distance à un hyperplan. On applique alors le
  corollaire précédent au plongement déduit de \( \vaemb \) en échangeant les
  coordonnées d'indices \( i \) et \( 0 \), ce qui ne change aucune des
  quantité en jeu.

  Pour la première option, soit \( L = l_0 \vp[0] + \dots + l_n \vp[n] \) une
  équation de \( \divi \). Pour chaque place \( v \), on note \( i_v \) un
  indice tel que \( \av{ l_{i_v} } \) soit maximal, de sorte que
  \( \av{ l_{i_v} } \ge \nv2 L (n + 1)^{-\dv/2} \).  On a immédiatement
  \( \distv{\divi}{p_{i_v}} \ge (n + 1)^{-\dv/2} \).

  On définit maintenant pour chaque \( i \) un ensemble de places
  \begin{equation}
    \placesapx_i
    =
    \set{
      v \in \placesapx \text{ telle que }
      \distv{\divi}{p_{i_v}} \ge (n + 1)^{-\dv/2}
    }
    \pmm.
  \end{equation}
  Le paragraphe précédent montre que \( \placesapx = \placesapx_0 \cup
    \cdots \cup \placesapx_n \). (Cette réunion n'est pas nécessairement
  disjointe, mais il n'est pas utile qu'elle le soit.) On associe à chacun de
  ces ensembles le nombre réel \( \sigma_i = \sum_{v \in \placesapx_i}
    \wtapx \degv \) où \( \degv \) désigne comme d'habitude le degré local en
  \( v \) divisé par le degré global. Par hypothèse, on a
  \( 1 = \sum\placerange \wtapx \degv \le \sum_{i=0}^n \sigma_i \), or cette
  somme comporte au plus \( s = \min(n+1, \card \placesapx) \) termes non
  nuls, donc il existe (au moins) un indice \( i \) tel que \( \sigma_i
    \ge 1 / s \). On fixe un tel indice, désormais désigné par \( i \).

  On note alors \( \placesapx' = \placesapx_i \) et, pour tout \( v \in
    \placesapx' \), on pose \( \wtapx' = \wtapx / \sigma_i \) pour avoir \(
    \sum_{v \in \placesapx'} \wtapx' \degv = 1 \).  Enfin, on pose \(
    \eps' = \eps \sigma_i \) de sorte qu'on a
  \begin{equation}
    (n + 1)^{-3\dv/2}
    \hautm2{x_i}^{-\wtapx \eps}
    \le
    \frac{ \distv{p_{i_v}}{\divi}^2 }{ (n + 1)^{\dv/2} }
    \hautm2{x_i}^{-\wtapx' \eps'}
    \quad \forall v \in \placesapx'
    \pmm.
  \end{equation}
  On considère pour finir le plongement \( \vaemb' \) obtenu à partir de \(
    \vaemb \) en échangeant les coordonnées d'indices \( i \) et \( 0 \), ce
  qui n'a aucune incidence sur la distance ni la hauteur. On peut alors
  appliquer le corollaire~\vref{c:vojta-div-dv} avec le plongement \( \vaemb'
  \), le réel \( \eps' \), l'ensemble de places \( \placesapx' \) et les
  poids \( \wtapx' \), car ses hypothèses sont impliquées par celle du présent
  corollaire compte tenu de la comparaison ci-dessus et du fait que \(
    \eps' \ge \eps / s \).
\end{proof}


\subsection{Hypersurfaces}

Pour se ramener au cas précédent, on utilise un plongement de \bsc{Veronese}
remodelé, tel que défini dans \cite[p. 14]{jadotth} ou \cite[p. 102]{remgdmp},
dont on rappelle la définition (sur les coordonnées homogènes) :
\begin{align}
  \chi_\adeg \colon \projd & \to \proj{n'} \\
  (x_0 : \dots : x_n) & \mapsto
  \bigl( \sbin\adeg\alpha^{1/2} x^\alpha \bigr)_\alpha
\end{align}
où \( n' = \binom{n + \adeg}{\adeg} - 1 \) et \( \alpha \) parcourt
l'ensemble des \( (n+1) \)-uplets d'entiers de longueur \( \adeg \).
L'intérêt de ce plongement par rapport au plongement de \bsc{Veronese}
classique réside dans le fait qu'on a, aux places archimédiennes :
\begin{equation}
  \nv2{ \chi_\adeg(x) }
  =
  \left(
    \sum_\alpha \binom\adeg\alpha \av{x}^{2\alpha}
  \right)^{1/2}
  =
  \nv2{ x }^\adeg
\end{equation}
d'après la formule multinomiale. Par ailleurs, pour chaque place
ultramétrique, si l'on choisit un indice \( \indv \) tel que \( \av{ x_\indv }
\) soit maximale, il n'est pas difficile de constater que \( x_\indv^\adeg \)
est une coordonnée de \( \chi_\adeg(x) \) qui est aussi de valeur absolue
maximale. On a ainsi
\begin{equation} \label{e:vero-norm}
  \nv2{ \chi_\adeg(x) }
  =
  \nv2{ x }^\adeg
  \quad \forall v
  \pmm.
\end{equation}
Dans les estimations de normes et hauteurs à venir, on aura aussi besoin du
lemme suivant.

\begin{lem} \label{l:norm-mn}
  Soient \( n \) et \( \adeg \) deux entiers et \( \alpha \in
    \N^{n+1} \) tel que \( \alpha_0 + \dots + \alpha_n = \adeg \).
  Pour toute place \( v \) d'un corps de nombres contenant \(
    \binom\adeg\alpha^{-1/2} \), on pose \( \dpv = 1 \) si \( v \)
  divise un premier inférieur ou égal à \( \adeg \) et \( 0 \) sinon. On a
  alors
  \begin{equation}
    \av*{ \binom\adeg\alpha^{-1/2} }
    \le
    \adeg^{\dpv n / 2}
    \pmm.
  \end{equation}
  De plus, cette valeur absolue vaut \( 1 \) si \( v \) divise un premier
  strictement supérieur à \( \adeg \).

  Par ailleurs, on a \( \sum_v \degv \dpv \le 1.26 \adeg / \ln \adeg
  \).  En particulier si on note \( b \) le vecteur dont les composantes sont
  les \( \binom\adeg\alpha^{-1/2} \) quand \( \alpha \) varie, on a
  \( \hautl\infty b \le \adeg n \).
\end{lem}

\begin{proof}
  Le cas des places archimédiennes est clair. Pour les places finies, on note
  \( p \) tel que \( \av\truc \) prolonge \( \av\truc[p] \) et on
  utilise la définition du multinomial en termes de factorielles et la formule
  (de \bsc{Legendre}) donnant la valuation d'une factorielle :
  \begin{equation}
    v_p( \binom\adeg\alpha )
    =
    \sum_{i = 1}^{\floor{\ln_p\adeg}}
    \floor*{ \frac{\adeg}{p^i} }
    - \floor*{ \frac{\alpha_0}{p^i} }
    - \dots
    - \floor*{ \frac{\alpha_n}{p^i} }
  \end{equation}
  qui donne immédiatement le cas d'égalité, car la somme est vide si \(
    p > \adeg \). Par ailleurs, chacun des termes vaut au plus \( n \) donc
  en sommant sur \( i \) on a \( v_p( \binom\adeg\alpha ) \le n \ln_p(\adeg)
  \) qui donne bien l'estimation de valeur absolue annoncée.

  La somme pondérée des \( \dpv \) est par définition le nombre de premiers
  inférieurs ou égaux à \( \adeg \), qui est au plus
  \( 1.25506 \adeg / \ln \adeg \) d'après le corollaire~1 page~69
  de~\cite{rosch}. L'estimation de hauteur annoncée en découle immédiatement.
\end{proof}

Le lemme suivant précise l'effet d'un plongement de \bsc{Veronese} remodelé
sur les différents paramètres en jeu dans notre situation.

\begin{lem} \label{l:hs-vero}
  Soient \( \vaemb \colon \va \to \projd \) un plongement projectif et \( \hs \)
  une hypersurface de \( \projd \), de degré \( \adeg \). On note \( \chi \)
  le plongement de \bsc{Veronese} remodelé de degré \( \adeg \), qui va donc
  de \( \projd \) dans \( \proj{n'} \) où \( n' = \binom{n +
      \adeg}{\adeg} - 1 \), et \( \vaemb' = \chi \circ \vaemb \). Dans ce
  plongement, l'image de \( \hs \) est découpée par un hyperplan \( \divi \) et
  on a :
  \begin{enumthm}
    \item \(
        \hautl{1, \vaemb'} \divi
        \le
        \hautl{1, \vaemb} \hs
        + \adeg n
      \) ;
    \item \(
        \distv[v, \vaemb']{x}{\divi}
        \le
        \distv[v, \vaemb]{x}{\hs}
        \cdot \adeg^{\dv/2}
      \) ; \label{i:hs-vero-dv}
    \item \(
        \hautl{2, \vaemb'}{x}
        =
        \hautl{2, \vaemb}{x}
        \cdot \adeg
      \) ; \label{i:hs-vero-ht-pt}
    \item \(
        \deg_{\vaemb'} \va
        =
        \deg_{\vaemb} \va
        \cdot \adeg^g
      \) ; \label{i:hs-vero-deg-va}
    \item \(
        \hautl{1, \vaemb'}{\va}
        \le
        \adeg^{g+1} \hautl{1, \vaemb}{\va}
        + 2 \adeg^{g+1} \ln(n + 1) (g + 1) \deg_{\vaemb} \va
      \) ; \label{i:hs-vero-ht-va}
    \item \(
        \htcmp[\vaemb']
        =
        \htcmp[\vaemb]
        \cdot \adeg
      \) ; \label{i:hs-vero-htcmp}
    \item \(
        \hlclab[\vaemb']
        =
        \hlclab[\vaemb]
        \cdot \adeg
        + 2 \adeg n
      \).
  \end{enumthm}
  Par ailleurs le cardinal \( \nclmaps \) de l'ensemble de cartes introduit
  à la section~\vref{sec:vaemb} est inchangé.
\end{lem}

\begin{proof}
  On utilise la remarque de~\cite[p.  102]{remgdmp} pour interpréter les
  distances et hauteurs dans le plongement de \bsc{Veronese} remodelé comme
  des distances et hauteurs d'indice \( (\adeg, \dots, \adeg) \) dans le
  plongement initial.

  On trouve alors les comparaisons de degré et de hauteurs qui nous
  intéressent dans~\cite[p. 85-86]{phidg} : les
  points~\vref{i:hs-vero-deg-va} et~\vref{i:hs-vero-ht-pt} en découlent
  directement et le point~\vref{i:hs-vero-htcmp} est une conséquence immédiate
  de ce dernier (pour lequel ou pouvait aussi utiliser~\eqref{e:vero-norm} en
  fait). Par ailleurs, la référence citée donne \( \hautl{\htpph, \vaemb'} \va
    = \adeg^{g+1} \hautl{\htpph, \vaemb} \va \) (en notant \( \Hautl\htpph \)
  la hauteur utilisée dans cette référence), dont on déduit le
  point~\vref{i:hs-vero-ht-va} en comparant les hauteurs utilisées grace
  à~\eqref{e:ht-pph-1} :
  \begin{align}
    \hautl{1, \vaemb'} \va
    & \le
    \hautl{\htpph, \vaemb'} \va
    + \ln(n' + 1) (g + 1) \deg_{\vaemb'} \va
    \\ & \le
    \adeg^{g+1} \hautl{\htpph, \vaemb} \va
    + \adeg^{g+1} \ln(n + 1) (g + 1) \deg_{\vaemb} \va
    \\ & \le
    \adeg^{g+1} \hautl{1, \vaemb} \va
    + 2 \adeg^{g+1} \ln(n + 1) (g + 1) \deg_{\vaemb} \va
    \pmm.
  \end{align}

  Pour~\vref{i:hs-vero-dv}, pour chaque place, introduisons un point
  \( y_v \in \hs \) tel que \( \distv[v, \vaemb] x \hs = \distv[v,
    \vaemb] x{y_v} \) ; par définition on a \( \distv[v, \vaemb'] x
    \divi \le \distv[v, \vaemb'] x{y_v} \). Pour majorer cette
  dernière quantité, aux places infinies on utilise la première
  formule en haut de la page~89 de~\cite{phidg} qu'on interprète ainsi :
  \begin{equation}
    \distv[v, \vaemb'] x{y_v}
    =
    \sqrt{ 1 - \bigl( 1 - \distv[v, \vaemb] x{y_v}^2 \bigr)^\adeg }
    \pmm.
  \end{equation}

  On conclut
  \pagebreak[2]
  en appliquant l'inégalité élémentaire \( 1 - (1 - z)^\adeg \le
    \adeg z \) avec \( z = \distv[v, \vaemb] x{y_v}^2 \). Aux places
  finies, on se base sur la définition de la distance : aux dénominateurs on
  utilise~\eqref{e:vero-norm} et au numérateur on voit facilement par le
  même type d'argument que \( \nv{}{ \chi(x) \wedge \chi(y) } \le \nv{}{ x
      \wedge y } \) de sorte que
  \(
    \distv[v, \vaemb'] x{y_v}
    \le
    \distv[v, \vaemb] x{y_v}
  \) comme annoncé.

  Pour la hauteur de \( \divi \), on remarque pour commencer que les résultats
  de~\cite[p. 85-86]{phidg} utilisés ci-dessus ne s'appliquent pas, car ce
  dernier n'est pas l'image de \( \hs \) dans le plongement ; en fait on a \(
    \chi(\hs) = \divi \cap \chi(\projd) \). Fixons plutôt une équation de \( \hs
  \) dans le plongement initial : c'est un polynômes homogène de degré \(
    \adeg \) en \( X = (X_0, \dots, X_n) \) mais on peut aussi le voir
  comme une forme linéaire en \( Z = (Z_\alpha)_\alpha \) (où \( \alpha \)
  parcourt l'ensemble des \( (n + 1) \)-uplets de longueur \( \adeg \)),
  en faisant le changement de variable \( Z_\alpha = X^\alpha \). On note \( L
    = \sum l_\alpha Z_\alpha \) la forme linéaire obtenue, puis \( L' =
    l_\alpha \binom\adeg\alpha^{-1/2} Z_\alpha \). Par construction, \( L' \)
  est bien l'équation dans le nouveau plongement d'un hyperplan \( \divi \)
  contenant \( \chi(\hs) \). Par ailleurs, le lemme précédent donne l'estimation
  de hauteur annoncée.

  Pour chaque \( (a, b, \clmap) \) on considère la famille
  \( (\rmclab{a}{b}{\vaemb, \clmap}[\ind])\indrange \) donnée par la
  section~\vref{sec:vaemb}. Pour tout \( \beta \in \N^{n+1} \) de longueur
  \( \adeg \), on regarde \( \prod\indrange
    (\rmclab{a}{b}{\clmap}[\ind])^{\beta_\ind} \), qui est une forme
  bihomogène de bidegré \( (2\adeg a^2, 2\adeg b^2) \) en \( (\vmp[1],
    \vmp[2]) \), comme une forme bihomogène de \( (2a^2, 2b^2) \) en la
  famille des monômes de bidegré \( (\adeg, \adeg) \) en \( (\vmp[1], \vmp[2])
  \), image qu'on note provisoirement \( K_\beta \). La famille \(
    (K_\beta)_\beta \) représente le morphisme \( (x, y) \mapsto ax - by \)
  sur l'ouvert \( \clmap \) dans le plongement \( \vaemb' \).

  La famille correspondante dans le plongement remodelé, notée \( (\rmclab
    ab\clmap[\vaemb', \beta])_\beta \),  s'en déduit en
  multipliant chacun des coefficients de \( K_\beta \) par un monôme de degré
  \( 2(a^2 + b^2) \) en les \( \binom\adeg\alpha^{-1/2} \) puis en
  multipliant la forme obtenue par \( \binom\adeg\beta^{1/2} \).
  D'après le lemme précédent, on a donc
  \begin{equation}
    \nv1{ \rmclab ab\clmap[\vaemb', \beta] }
    \le
    \av{ \textstyle \binom\adeg\beta^{1/2} }
    \cdot
    \nnv1{ \rmclab ab\clmap }^\adeg
    \cdot
    \adeg^{n (a^2 + b^2) \dpv}
  \end{equation}
  puis en sommant sur \( \beta \) :
  \begin{equation}
    \nnv1{ \rmclab ab\clmap[\vaemb'] }
    \le
    (n + 1)^{\adeg \, \dv}
    \cdot
    \nnv1{ \rmclab ab\clmap }^\adeg
    \cdot
    \adeg^{n (a^2 + b^2) \dpv}
    \le
    \hmclab[\vaemb']*^{a^2 + b^2}
  \end{equation}
  en posant
  \(
    \hmclab[\vaemb']* = \hmclab*^\adeg
    \cdot (n + 1)^{5\adeg \dv/4} \cdot \adeg^{n \dpv}
  \) (l'exposant de \( n + 1 \) n'est pas optimal ici mais sera utile au
  paragraphe suivant).

  Par ailleurs, on sait que si \( x \) désigne des coordonnées homogènes d'un
  point de \( \projd \) et \( x' \) l'image de ces coordonnées par \( \chi \),
  on a en toute place \( \nv2{ x' } = \nv2{ x }^\adeg \). Ainsi, vu les
  définitions ci-dessus, on a
  \begin{equation}
    \frac{
      \nv2{ \rmclab{a}{b}{ \clmap }[\vaemb'](x', y') }
    }{
      \nv2{ x' }^{2a^2} \nv2{ y' }^{2b^2}
    }
    =
    \left(
      \frac{
        \nv2{ \rmclab{a}{b}{ \clmap }[\vaemb](x, y) }
      }{
        \nv2 x ^{2a^2} \nv2 y ^{2b^2}
      }
    \right)^\adeg
  \end{equation}
  puis, en utilisant des comparaisons classiques de normes :
  \begin{align}
    \frac{
      \nv1{ \rmclab{a}{b}{ \clmap }[\vaemb'](x', y') }
    }{
      \nv1{ x' }^{2a^2} \nv1{ y' }^{2b^2}
    }
    & \ge
    \frac{
      \nv2{ \rmclab{a}{b}{ \clmap }[\vaemb'](x', y') }
    }{
      \nv2{ x' }^{2a^2} \nv2{ y' }^{2b^2}
    }
    \, (n' + 1)^{-(a^2 + b^2)\dv}
    \\ & \ge
    \left(
      \frac{
        \nv2{ \rmclab{a}{b}{ \clmap }[\vaemb](x, y) }
      }{
        \nv2 x ^{2a^2} \nv2 y ^{2b^2}
      }
      (n + 1)^{-(a^2 + b^2)\dv}
    \right)^\adeg
    \\ & \ge
    \left(
      \frac{
        \nv1{ \rmclab{a}{b}{ \clmap }[\vaemb](x, y) }
      }{
        \nv1 x ^{2a^2} \nv1 y ^{2b^2}
      }
      (n + 1)^{-(a^2 + b^2 + 1/2)\dv}
    \right)^\adeg
    \\ & \ge
    \bigl( \hmclab[\vaemb]* (n + 1)^{5\dv / 4} \bigr)^{-\adeg(a^2 + b^2)}
    \ge
    \hmclab[\vaemb']*^{-(a^2 + b^2)}
    \pmm.
  \end{align}
  En utilisant à nouveau le lemme précédent il vient
  \begin{equation}
    \hlclab[\vaemb']
    =
    \hlclab[\vaemb] \cdot \adeg
    + \frac54\adeg \ln(n + 1) + 1.26 \adeg n
  \end{equation}
  dont le dernier terme se déduit en remarquant que \( \frac54 \ln(n +
    1) / n \le \frac58 \ln 3 \le 0.69 \) car \( n \ge 2 \).
\end{proof}

\begin{coro} \label{c:vojta-hs}
  Soient \( \vaemb \colon \va \embedin \projd \) un plongement projectif,
  \( \hs \) une hypersurface de \( \projd \) de
  degré \( \adeg \) et \( \eps > 0 \) un nombre réel.
  Il n'existe dans \( \va(\Qbar) \) aucune famille de points \( x_1,
    \dots, x_m \) avec \( m \ge g + 1 \) satisfaisant
  simultanément aux conditions suivantes :
  \begin{align}
    0 < \distv{x_i}{\hs}
    & <
    \alpha_v^{-1}
    \hautm2{x_i}^{-\wtapx \eps}
    \quad \forall v \in \placesapx
    \\
    \hautn{x_1}
    & > \frac4\adeg \cst{ht-v-hs} \Lambda_3^{(1 + \frac1m)
      \prod_{i=m}^{mg} (2i + 1)}
    \\
    \hautn{x_i} & > \hautn{x_{i-1}} \cdot
    4 m \Lambda_3^{2m \prod_{i=m+1}^{mg} (2i + 1)}
    \\
    \cos(x_i, x_j) & > 1 -
    \frac1{ m }
    \left(
      \frac{ \eps }{
        86 \nclmaps \cdot 5^g \adeg s
      }
    \right)^{ \frac{m}{m-g} }
  \end{align}
  avec
  \begin{align}
    \Lambda_3 \label{e:deg-v-hs}
    & = \max \bigl(
      (\adeg^g \deg \va)^m,
      5 m^2
      \bigl(
        86 \nclmaps \cdot 5^g
        \adeg s \, \eps^{-1}
      \bigr)^{ \frac{m}{m-g} }
      \adeg^g \deg \va,
      (\sqrt2 m g)^{mg}
    \bigr)
    \qquad\null
    \\
    \newcst[\vaemb]{ht-v-hs}
    & =
    \adeg \max \bigl(
      \adeg^g \hautl1{\va}, \hlclab, \htcmp
    \bigr)
    + (g + 1) \deg \va
    \bigl(
      \adeg^g \hautl1 \hs
      + (\adeg+1)^n \ln(\adeg+1) n
    \bigr)
  \end{align}
  et au choix :
  \begin{enumthm}
  \item \( s = \min(\sbin{\adeg+n}{n}, \card \placesapx) \) et
    \( \alpha_v = \bigl( \adeg \sbin{\adeg+n}{n}^3 \bigr)^{\dv/2} \) ;  ou
    bien
  \item \( s = 1 \) et \( (\alpha_v)_v \) une certaine famille de réels tous
    supérieurs ou égaux à \( 1 \), satisfaisant
    \(
      \prod\placerange \alpha_v^\degv
      \le
      \hautm1 \hs ^2 \, \expb^{3 \adeg n}
    \).
  \end{enumthm}
\end{coro}

\begin{proof}
  On considère le plongement \( \vaemb' \) et l'hyperplan \( \divi \) fournis
  par le lemme~\vref{l:hs-vero} précédent et on prévoit d'appliquer le
  corollaire~\vref{c:vojta-div-any} dans ce cadre avec \( \eps' =
    \eps/\adeg \) comme exposant d'approximation. Il s'agit donc de
  vérifier que nos conditions impliquent celles du corollaire en question.

  Concernant l'hypothèse d'approximation, avec le premier choix pour \( (s,
    \alpha) \),
  \begin{align}
    \distv[v, \vaemb']{ x_i }{\divi}
    \le
    \distv[v, \vaemb]{ x_i }{\hs}
    \cdot \adeg^{\dv/2}
    & \le
    \bigl( \adeg \sbin{\adeg+n}{n}^3 \bigr)^{-\dv/2}
    \hautm{2, \vaemb}{x_i}^{-\wtapx \eps}
    \cdot \adeg^{\dv/2}
    \\ & \le
    (n' + 1)^{-3\dv/2}
    \hautm{2, \vaemb'}{x_i}^{-\wtapx \eps'}
  \end{align}
  ce qui justifie le choix fait pour \( \eps' \).

  Avec le deuxième choix, si l'on note \( \alpha' \) la famille du
  corollaire~\vref{c:vojta-div-any}, on voit qu'il s'agit de poser \( \alpha_v
    = \alpha'_v \cdot \adeg^{\dv/2} \). On a alors
  \begin{equation}
    \prod\placerange \alpha_v^\degv
    \le
    \adeg^{1/2} \prod\placerange \alpha'_v
    \le
    \adeg^{1/2} \hautm{2, \vaemb'} \divi ^2 (n' + 1)^{1/2}
    \le
    \adeg^{1/2}
    \hautm{1, \vaemb} \hs ^2 \expb^{2\adeg n}
    \sbin{\adeg+n}{n}^{1/2}
  \end{equation}
  qui donne bien la majoration annoncée.

  On adapte alors les constantes pour tenir compte des variations de degrés,
  hauteurs et dimension ambiante ainsi que du choix de \( \eps \).
  Détaillons le cas de \( \cst{ht-v-hs} \). En utilisant abondamment le lemme
  précédent, on a
  \begin{align}
    & \hautl{1, \vaemb'}{\va}
    + \hautl{1, \vaemb'}{ \divi } (g+1) \deg_{\vaemb'} \va
    \\ & \quad \le
    \adeg^{g+1} \hautl{1, \vaemb}{\va}
    + (g + 1) \deg_{\vaemb} \va \Bigl(
      \adeg^{g+1} \ln(n + 1)
      + \adeg^g \bigl( \hautl{1, \vaemb} \hs + \adeg n \bigr)
    \Bigr)
  \end{align}
  puis
  \begin{equation}
    \hlclab[\vaemb'] + \frac52 \hautl{1, \vaemb'} E
    \le
    \adeg \hlclab[\vaemb] + 2 \adeg n
    + \frac52 ( \hautl{1, \vaemb} \hs + \adeg n)
  \end{equation}
  et
  \begin{equation}
    \htcmp[\vaemb'] + \hautl{2, \vaemb'} E + \frac12 \ln(n' + 1)
    \le
    \adeg \htcmp[\vaemb]
    + \hautl{1, \vaemb} \hs + \adeg n
    + \frac12 \adeg \ln(n+1)
  \end{equation}
  ce qui prouve que
  \begin{align}
    \cst{ht-v-div-dv}[\vaemb']
    & \le \max \Bigl(
      \adeg^{g+1} \hautl{1, \vaemb}{\va}
      + \adeg^g (g + 1) \deg_{\vaemb} \va
      \bigl( \hautl{1, \vaemb} \hs + 3\adeg n \bigr),\
      \\ & \qquad
      \adeg \hlclab[\vaemb] + \frac52 \hautl{1, \vaemb} \hs
      + \frac92 \adeg n,\
      \adeg \htcmp[\vaemb] + \hautl{1, \vaemb} \hs + \frac32 \adeg n,\
      (\adeg+1)^n \ln(\adeg+1) n
    \Bigr)
    \\ & \le
    \adeg \max \bigl(
      \adeg^g \hautl{1, \vaemb}{\va}, \hlclab[\vaemb], \htcmp[\vaemb]
    \bigr)
    \\ & \qquad
    + (g + 1) \deg_{\vaemb} \va
    \bigl(
      \adeg^g \hautl{1, \vaemb} \hs
      + (\adeg+1)^n \ln(\adeg+1) n
    \bigr)
  \end{align}
  en prenant le maximum terme par terme, assez largement pour le dernier.

  La condition suivante vient alors en remarquant que la hauteur normalisée
  est multipliée par \( \adeg \) en changeant de plongement.  Enfin, les
  rapports de hauteurs normalisées et les angles dans l'espace de \MoW étant
  invariants sous le changement de plongement considéré, les deux dernières
  conditions sont inchangées.  On peut donc appliquer le
  corollaire~\vref{c:vojta-div-any} comme prévu et conclure.
\end{proof}


\subsection{Variétés quelconques}

Il s'agit de choisir une hypersurface contenant la variété et ne passant par
aucun des points considérés, puis de lui appliquer le corollaire précédent. Le
lemme suivant donne un choix possible d'hypersurface de degré et hauteur
contrôlés.

\begin{lem} \label{l:hs-choice}
  Soient \( \avar \) une sous-variété de \( \projd \) de degré \( \adeg \) et de
  dimension \( \adim \), et \( x_1, \dots, x_m \) des points dans \(
    \projd(\Qbar) \setminus \avar(\Qbar) \). Il existe une hypersurface \( \hs
  \) de \( \projd \) contenant \( \avar \) mais aucun des \( x_i \), de degré
  \( \deg \avar \) et de hauteur
  \begin{equation}
    \hautl1 \hs
    \le
    \hautl1 \avar
    + (\adim + 2) \ln (\adeg + 1) ( \adeg + n + 1 )
    + \ln(m/2)
    \pmm.
  \end{equation}
\end{lem}

\begin{proof}
  On se base sur la proposition~6.1 (p. 543) de~\cite{remdcl}, qui fournit une
  famille de formes \( P_1, \dots, P_s \) de degré \( \adeg \) définissant \(
    \avar \) (ensemblistement), avec une variante : si \( \adeg = 1 \) on
  choisit pour \( \mathcal M_2 \) (dans les notations de la preuve de
  \lat{loc. cit.}) l'ensemble des matrices dont les coefficients sont \( 0 \)
  ou \( 1 \), de sorte que pour tout \( \adeg \) on a
  \begin{equation}
    s = \card \mathcal M_2 \le (\adeg + 1)^{(\adim+2)(n+1)}
    \pmm.
  \end{equation}
  Par ailleurs, on utilisera plutôt les hauteurs \( \Hautl1 \), pour
  lesquelles on obtient la majoration suivante
  \begin{equation}
    \hautl1{ P_i }
    \le
    \hautl1 \avar + (\adim + 1) \adeg \ln(\adeg + 1)
  \end{equation}
  en remarquant que les \( P_i \) sont une spécialisation de la forme de
  \bsc{Chow} de \( \avar \) en des points (formes linéaires) de hauteur \(
    \Hautm1 \) au plus \( \adeg + 1 \) : au plus deux coefficients non
  nuls, tous deux dans \( I_\adeg \).

  Si \( m = 1 \), il suffit de choisir un \( P_i \) ne s'annulant pas en
  \( x_1 \). On suppose donc désormais que \( m \ge 2 \).

  Pour chaque \( i \in \set{1, \dots, m} \) on fixe un \( j_i
  \) tel que \( P_{j_i}(x_i) \neq 0 \). On considère alors le polynôme
  \( \mathcal P \in \Qbar[T_1, \dots, T_s] \) défini par
  \begin{equation}
    \mathcal P
    =
    \prod\fctrange \bigl( P_1(x_i) T_1 + \dots + P_s(x_i) T_s \bigr)
  \end{equation}
  qui est donc homogène de degré \( m \), et n'est pas le polynôme nul,
  car un de ses coefficients est \( \prod\fctrange P_{j_i}(x_i) \) qui
  n'est pas nul. On peut donc choisir un point \( \lambda \in \Z^s \) avec \(
    \abs{\lambda_j} \le m/2 \) en lequel \( \mathcal P \) ne
  s'annule pas.

  On pose alors \( Q = \lambda_1 P_1 + \dots + \lambda_s P_s \) et \( \hs =
    \zeros Q \). Par construction, il est clair que \( \hs \) est de degré \(
    \adeg \) et contient \( \avar \) mais aucun des \( x_i \) (en effet, \(
    \prod\fctrange Q(x_i) = \mathcal P(\lambda) \neq 0 \)).  Par ailleurs, \(
    \hautl1{\lambda} \le \ln s + \ln(m/2) \), ce qui donne
  immédiatement
  \begin{equation}
    \hautl1 \hs
    \le
    \hautl1 \avar + (\adim + 1) \adeg \ln(\adeg + 1)
    + (\adim+2)(n+1) \ln (\adeg + 1)
    + \ln(m/2)
  \end{equation}
  qui implique l'estimation annoncée.
\end{proof}

\begin{coro} \label{c:vojta-gen1}
  Soient \( \vaemb \colon \va \embedin \projd \) un plongement projectif,
  \( \avar \) une variété de \( \projd \) de degré \( \adeg \) et de dimension
  \( \adim \), et \( \eps > 0 \) un nombre réel.  Il n'existe dans \(
    \va(\Qbar) \) aucune famille de points \( x_1, \dots, x_m \) avec \(
    m \ge g + 1 \) satisfaisant simultanément aux conditions suivantes :
  \begin{align}
    0 < \distv{x_i} \avar
    & <
    \alpha_v^{-1}
    \hautm2{x_i}^{-\wtapx \eps}
    \quad \forall v \in \placesapx
    \\
    \hautn{x_1}
    & > \frac4\adeg \cst{ht-v-hs} \Lambda_3^{(1 + \frac1m)
      \prod_{i=m}^{mg} (2i + 1)}
    \\
    \hautn{x_i} & > \hautn{x_{i-1}} \cdot
    4 m \Lambda_3^{2m \prod_{i=m+1}^{mg} (2i + 1)}
    \\
    \cos(x_i, x_j) & > 1 -
    \frac1{ m }
    \left(
      \frac{ \eps }{
        86 \nclmaps \cdot 5^g \adeg s
      }
    \right)^{ \frac{m}{m-g} }
  \end{align}
  avec \( \Lambda_3 \) comme en~\eqref{e:deg-v-hs},
  \begin{align}
    \newcst[\vaemb]{ht-v-gen1}
    & =
    \adeg \max \bigl(
      \adeg^g \hautl1{\va}, \hlclab, \htcmp
    \bigr)
    + (g + 1) \deg \va
    \Bigl(
      (\adeg+1)^n \ln(\adeg+1) n
    \\ & \qquad
      + \adeg^g \bigl (
        \hautl1 \avar
        + (\adim + 2) \ln (\adeg + 1) ( \adeg + n + 1 )
        + \ln(m/2)
      \bigr)
    \Bigr)
  \end{align}
  et au choix :
  \begin{enumthm}
  \item \( s = \min(\sbin{\adeg+n}{n}, \card \placesapx) \) et
    \( \alpha_v = \bigl( \adeg \sbin{\adeg+n}{n}^3 \bigr)^{\dv/2} \) ;  ou
    bien
  \item \( s = 1 \) et \( (\alpha_v)_v \) une certaine famille de réels tous
    supérieurs ou égaux à \( 1 \), satisfaisant
    \(
      \prod\placerange \alpha_v^\degv
      \le
      \hautm1 \avar ^2
      \cdot (\adeg+1)^{ 2 (\adim+2) (\adeg + n+1) } \cdot \frac{m^2}4
      \cdot \expb^{3 \adeg n}
    \).
  \end{enumthm}
\end{coro}

\begin{proof}
  Découle immédiatement du corollaire~\vref{c:vojta-hs} appliqué à
  l'hypersurface produite par le lemme précédent.
\end{proof}

On énonce maintenant une version du précédent corollaire avec des constantes
légèrement simplifiées.

\begin{coro} \label{c:vojta-gen2}
  Soient \( \vaemb \colon \va \embedin \projd \) un plongement projectif,
  \( \avar \) une variété de \( \projd \) de degré \( \adeg \) et de dimension
  \( \adim \), et \( 0 < \eps < 1 \) un nombre réel.  Il n'existe dans \(
    \va(\Qbar) \) aucune famille de points \( x_1, \dots, x_m \) avec \(
    m \ge g + 1 \) satisfaisant simultanément aux conditions suivantes :
  \begin{align}
    0 < \distv{x_i} \avar
    & <
    \alpha_v^{-1}
    \hautm2{x_i}^{-\wtapx \eps}
    \quad \forall v \in \placesapx
    \\
    \hautn{x_1}
    & > \cst{ht-v-gen1} \Lambda_4^{(2mg)^{mg}}
    \\
    \hautn{x_i} & > \hautn{x_{i-1}}
    \cdot \Lambda_4^{(2mg)^{mg}}
    \\
    \cos(x_i, x_j) & > 1 -
    \frac1{ m \, \cst{v-gen} }
  \end{align}
  avec
  \begin{align}
    \newcst[]{v-gen}
    & =
    \bigl(
      86 \nclmaps \cdot 5^g \adeg s \, \eps^{-1}
    \bigr)^{ \frac{m}{m-g} }
    \\
    \Lambda_4
    & =
    \cst{v-gen}
    \bigl( (\sqrt2 m g\adeg)^g \deg \va \bigr)^m
    \\
    \cst{ht-v-gen1}
    & =
    \adeg \max \bigl(
      \adeg^g \hautl1{\va}, \hlclab, \htcmp
    \bigr)
    + (g + 1) \deg \va
    \Bigl(
      (\adeg+1)^n \ln(\adeg+1) n
    \\ & \qquad
      + \adeg^g \bigl (
        \hautl1 \avar
        + (\adim + 2) \ln (\adeg + 1) ( \adeg + n + 1 )
        + \ln(m/2)
      \bigr)
    \Bigr)
  \end{align}
  et au choix :
  \begin{enumthm}
  \item \( s = \min(\sbin{\adeg+n}{n}, \card \placesapx) \) et \( \alpha_v =
      \bigl( \adeg \sbin{\adeg+n}{n}^3 \bigr)^{\dv/2} \) ; ou bien
  \item \( s = 1 \) et \( (\alpha_v)_v \) une certaine famille de réels tous
    supérieurs ou égaux à \( 1 \), satisfaisant \(
      \prod\placerange \alpha_v^\degv
      \le
      \hautm1 \avar ^2
      \cdot (\adeg+1)^{ 2 (\adim+2) (\adeg + n+1) } \cdot \frac{m^2}4
      \cdot \expb^{3 \adeg n}
    \).
  \end{enumthm}
\end{coro}

\begin{proof}
  La définition de \( \cst{ht-v-gen1} \) n'a pas changé, elle est simplement
  rappelée pour référence, et l'apparition de \( \cst{v-gen} \) ne change
  rien. La définition de \( \Lambda_4 \) a été simplifiée en observant que \(
    5m^2 \deg \va \le (\deg \va)^m (\sqrt2 mg)^{mg} \) compte tenu du fait que
  \( m \ge 2 \) et \( \deg \va \ge 3 \) (c'est évident pour \( m \) assez
  grand et on le vérifie numériquement pour les petites valeurs).

  On montre maintenant que \( \prod_{i=m}^{mg} (2i + 1) \le
    \frac12 (2mg)^{mg} \). C'est clair si \( g = 1 \)
  et dans le cas contraire on remarque qu'il s'agit d'un produit de \(
    mg - m + 1 \) facteurs tous inférieurs à \( mg \)
  sauf le dernier qu'on regroupe avec le premier car \( (2m +
    1)(2mg + 1) \le (2mg)^2 \) pour \( g \ge 2 \) (et
  donc \( m \ge 3 \)).

  On en déduit que la deuxième condition ci-dessus implique celle du
  corollaire précédent en remarquant que \( \frac4\adeg \le \Lambda_4^{1/4}
  \) puis que \( (1 + \frac1m) \prod_{i=m}^{mg} (2i + 1) +
    \frac14 \le (2mg)^{mg} \). On procède de façon
  similaire pour la troisième condition en remarquant que cette fois le
  produit dans l'exposant commence à \( i = m + 1 \).
\end{proof}

Signalons enfin qu'une version simplifiée de l'énoncé précédent, dans le cas
où \( \va \) est plongée dans \( \projd \) par un plongement de \bsc{Mumford}
modifié associé à une polarisation principale, est donnée par le
corollaire~\vref{c:vojta-gen3}.

\cleardoublepage
\endinput

% vim: spell spelllang=fr
