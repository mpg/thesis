% !TEX root = main.tex

\chapter{Inégalité de \bsc{Vojta}} \label{chap:vojta}


\section{Stratégie de la preuve}

Nous établissons ici une inégalité de \bsc{Vojta} effective dans le cas
particulier où la variété à approcher est le diviseur \( \divi \) découpé par
l'hyperplan d'équation \( X_0 = 0 \), supposé ne pas contenir \( \va \). Nous
verrons plus tard que ce cas particulier implique le cas général.
\nomuse \divi [chap] {Le diviseur découpé par \( \vp[0] = 0 \), dont on étudie
  les approximations}
Le but du chapitre est donc de prouver le théorème suivant.

\begin{thm} \label{t:vojta-div}
  Dans \( \va(\Qbar) \), il n'existe pas de famille de points \( \ex[1],
    \dots, \ex[\puiss] \) satisfaisant simultanément aux conditions
  suivantes :
  \begin{align}
    0 < \distv{\ex*}{\divi}
    & < \hautm{\ex*}^{-\wtapx \expapx}
    \quad \forall \place \in \placesapx
    \label{e:Vapx}
    \\
    \hautn{\ex[1]} & > \Vbig
    \label{e:Vbig}
    \\
    \hautn{\ex*} & > \Vfar \hautn{\ex[\fct-1]}
    \label{e:Vfar}
    \\
    \cos(\ex*, \ex[\fcti]) & > 1 - \Vcos
    \label{e:Vcos}
  \end{align}
  avec
  \nomuse \Vbig {Dans le théorème~\ref{t:vojta-div}, minore la hauteur du plus
  petit point}
  \nomuse \Vfar {Dans le théorème~\ref{t:vojta-div}, minore l'écart
    multiplicatif entre les points}
  \nomuse \Vcos {Dans le théorème~\ref{t:vojta-div}, contrôle l'angle entre
    les points}
  \nomuse \puiss [chap] {cardinal de la famille d'approximations \( (\ex*) \)
    considérée}
  \begin{align}
    \Vbig & = \dots (\puiss)
    \\
    \Vfar & = \dots (\puiss)
    \\
    \Vcos & = \dots (\puiss)
  \end{align}
\end{thm}

\nomuse[\ex]{(\ex*)}[chap]{Famille d'approximations de \( \divi \) dont
  l'existence contredirait le théorème~\ref{t:vojta-div}}
La démonstration procède par l'absurde : si le théorème est faux, fixons une
famille \( \ex[1], \dots, \ex[\puiss] \) qui le contredit. Bien que cette
famille soit toujours supposée satisfaire à toutes les conditions du théorème,
nous préciserons dans les hypothèses de la plupart des énoncés la ou
lesquelles de ces conditions nous utilisons, par souci de clarté.

Nous utiliserons des combinaisons linéaires des \( \ex* \) de petite hauteur.
Les lemmes suivants permettent de choisir les coefficients pour ces
combinaisons ; nous les prendrons entiers et n'ayant que \( 2 \) et \( 3 \)
pour diviseurs premiers, de sorte à disposer de représentations polynomiales
convenables des formes linéaires abéliennes associées données par
l'annexe~\ref{sec:form-ab-alt}.

\begin{lem} \label{l:wt-choose-gen}
  Soit \( \zeta \) un réel positif. Il existe des entiers \( \wt* \in 2^\N
    3^\N \) tels que, pour tout \( \fct \in \set{1, \dots, \puiss} \) :
  \begin{equation} \label{e:wt-ratio}
    \frac1{1 + \zeta}
    \le
    \frac{ \wts* \hautn{\ex*} }{ \wts[1] \hautn{\ex[1]} }
    \le
    1 + \zeta
    \pmm.
  \end{equation}
\end{lem}

\begin{proof}
  On commence par choisir des rationnels \( b_\fct = 2^{b_{\fct2}}
    3^{b_{\fct3}} \) tels que
  \begin{equation}
    \frac1{1 + \zeta}
    \le
    b_\fct^2 \frac{ \hautn{\ex*} }{ \hautn{\ex[1]} }
    \le
    1 + \zeta
    \pmm,
  \end{equation}
  soit en prenant les logarithmes et en divisant par deux :
  \begin{equation}
    \abs*{
      b_{\fct2} \log 2 + b_{\fct3} \log 3
      - \frac12 \log \frac{ \hautn{\ex*} }{ \hautn{\ex[1]} }
    }
    \le
    \frac12 \log (1 + \zeta)
    \pmm.
  \end{equation}
  Comme \( \Z \log2 + \Z \log 3 \) est dense dans \( \R \), il est
  certainement possible de choisir indépendamment pour chaque \( \fct \) deux
  entiers \( b_{\fct2} \) et \( b_{\fct3} \) tels que cette dernière condition
  soit satisfaite.

  Il reste plus qu'à définir \( \wt[1] \) comme le plus petit dénominateur
  commun des \( b_\fct \) puis à poser \( \wt* = \wt[1] b_\fct \) pour tout \(
    \fct > 1 \).
\end{proof}

\begin{scho} \label{s:wt-diff-small}
  Pour toute famille \( (\wt*) \) satisfaisant à~\eqref{e:wt-ratio}, on a
  \begin{align} \label{e:wt-diff-hautn}
    \abs[\big]{ \wts* \hautn{\ex*} - \wts[1] \hautn{\ex[1]} }
    & \le \zeta \wts[1] \hautn{\ex[1]}
    \\ \label{e:wt-diff-normn}
    \abs[\big]{ \wt* \normn{\ex*} - \wt[1] \normn{\ex[1]} }
    & \le \frac\zeta2 \wt[1] \normn{\ex[1]}
  \end{align}
\end{scho}

Ces inégalités découlent immédiatement de l'encadrement donné en utilisant les
comparaisons classiques \( (1 + \zeta)^{-1} \ge 1 - \zeta \) et \( (1 +
  \zeta)^{1/2} \le 1 + \zeta/2 \) et le fait que \( \hautn\truc =
  \normn\truc^2 \) par définition de la norme de \NT.

\begin{lem}
  Soit \( \zeta > 0 \) et \( (\wt*) \) une famille d'entiers satisfaisant
  à~\eqref{e:wt-ratio}. Si de plus la famille \( (\ex*) \) satisfait
  à~\eqref{e:Vcos}, alors pour tout \( \fct \in \set{1, \dots, \puiss} \) on a
  \begin{equation} \label{e:hautn-wt-diff-gen}
    \hautn{\wt* \ex* - \wt** \ex**}
    \le
    \wts[1] \hautn{\ex[1]} \left(
      \frac{\zeta^2}4 + 2 \Vcos (1 + \zeta)
    \right)
    \pmm.
  \end{equation}
\end{lem}

\begin{proof}
  En développant le membre de gauche, il vient successivement
  \begin{alignat}{2}
    \normn{\wt* \ex* - \wt** \ex**}^2
    & =
    \normn{\wt* \ex*}^2 + \normn{\wt** \ex**}^2
    - 2 \scalarn{\wt* \ex*}{\wt** \ex**}
    \\
    & = \wts* \normn{\ex*}^2 + \wts** \normn{\ex**}^2
    - 2 \wt* \wt** \scalarn{\ex*}{\ex**}
    \\
    & = \left( \wt* \normn{\ex*} - \wt** \normn{\ex**} \right)^2
    + 2 \wt* \wt** \left(
      \normn{\ex*} \normn{\ex**} - \scalarn{\ex*}{\ex**}
    \right)
    \\
    & \le \left( \wt* \normn{\ex*} - \wt** \normn{\ex**} \right)^2
    + 2 \wt* \wt** \normn{\ex*} \normn{\ex**} \Vcos
    && \text{d'après~\eqref{e:Vcos}}
    \\
    & \le \left( \frac\zeta2 \wt[1] \normn{\ex[1]} \right)^2
    && \text{d'après~\eqref{e:wt-diff-normn}}
    \\
    & \qquad + 2 \left( \wt[1] \normn{\ex[1]} \sqrt{1 + \zeta} \right)^2 \Vcos
    && \text{d'après~\eqref{e:wt-ratio}}
  \end{alignat}
  qui donne le résultat annoncé en factorisant.
\end{proof}

\begin{scho} \label{s:wt-choose}
  \nomuse[\wt]{(\wt*)}[chap]{Poids associés à \( (\ex*) \) et fixés par le
    scolie~\ref{s:wt-choose}}
  \nomdirect{g}{zeta}{\zeta}[chap]{Constante proche de zéro fixée par le
    scolie~\ref{s:wt-choose}}
  On choisit désormais et jusqu'à la fin du chapitre une famille \( (\wt*) \)
  donnée par l'application du lemme~\ref{l:wt-choose-gen} avec \( \zeta =
    \sqrt{\Vcos} \) de sorte qu'on a
  \begin{equation} \label{e:hautn-wt-diff}
    \hautn{\wt* \ex* - \wt** \ex**}
    \le
    3 \Vcos \wts[1] \hautn{\ex[1]}
    \pmm.
  \end{equation}
\end{scho}

Cette dernière égalité découle directement de~\eqref{e:hautn-wt-diff-gen} en
tenant compte du fait que \( 1 + \zeta \le 5/4 \) car \( \Vcos \le 1/16 \).
\label{ct:Vcos<1/16}

Étudions maintenant quelques propriétés de ces poids, en sus de celles déjà
énoncées, en exploitant le fait que les \( \ex* \) satisfont à
\eqref{e:Vbig}. Pour commencer, la suite \( \wts[1], \dots, \wts** \) décroît
au moins comme une suite géométrique de raison inférieure à $1$. Plus
précisément, dès que \( \ex \) satisfait~\eqref{e:Vfar}, on a
\begin{equation} \label{e:wt-geom}
  \wts*
  \le
  (1+\zeta) \cdot \frac{\wts[1]}{\Vfar^{\fct-1}}
\end{equation}
en appliquant directement la définition de \( \wt* \). On peut ainsi majorer
des sommes faisant intervenir les \( \wts* \) essentiellement par \( \wts[1]
\) ; par exemple, l'énoncé suivant nous sera utile par la suite.

\begin{lem}
  On a \(
    \sum_{\fct=1}^\puiss (\wts* + \wts**) \le \wts[1] (1 + \epsiii)
  \) avec
  \nomuse \epsiii [chap] {Un petit réel positif, défini
    par~\eqref{e:def-epsiii}}
  \begin{equation} \label{e:def-epsiii}
    \epsiii = \frac{ 3 (1+\zeta) }{ \Vfar - 1 }
    \pmm.
  \end{equation}
\end{lem}

\begin{proof}
  En effet, on a
  \begin{align}
    \frac1{\wts[1]} \sum_{\fct=1}^\puiss (\wts* + \wts**)
    & = 1
    + \frac{\puiss \wts**}{\wts[1]}
    + \sum_{\fct=2}^\puiss \frac{\wts*}{\wts[1]}
    \\
    & \le 1
    + \frac{\puiss(1+\zeta)}{\Vfar^{\puiss*}}
    + (1+\zeta) \sum_{\fct=2}^\puiss \Vfar^{-\fct+1}
    \\
    & \le 1 + (1+\zeta) \left(
      \Bigl( \frac2\Vfar \Bigr)^{\puiss-1}
      + \frac1{\Vfar-1}
    \right)
    \pmm.
  \end{align}
  Pour conclure, il suffit d'observer que \( \frac2\Vfar < 1 \).
  \label{ct:Vfar>2}
\end{proof}

Pour la suite du chapitre, on notera \( \epsiii \) le réel défini
par~\eqref{e:def-epsiii}.

\subsection{Réduction à l'existence d'une forme obstructrice}

Nous regardons \( \ex = (\ex*) \) comme un point de \( \va^\puiss \) plongée
dans \( (\projd)^\puiss \), et introduisons l'ensemble \( \varset(\ex) \) des
sous-variétés produit \( \var = \var[1] \times \dots \times \var** \) de \(
  \va^\puiss \) qui contiennent \( \ex \) et satisfont aux majorations
suivantes :
\nomuse {\varset(\ex)} [chap] {Ensemble des variétés produit contenant \( \ex
  \) et satisfaisant aux conditions~\eqref{e:varset-deg}
  à~\eqref{e:varset-ht}}
\begin{align}
  \label{e:varset-deg}
  \vdeg* & \le \Lambda^{f_1(u)} \quad \forall \fct
  \\ \label{e:varset-deg-prod}
  \prod\fctrange \vdeg* & \le (?) \Lambda^{f_1(u)}
  \\ \label{e:varset-ht}
  \sum\fctrange \wts* \hautl{\var*}
  & \le (\dots) \Lambda^{f_2(u)} \wts[1]
  \pmm,
\end{align}
où l'on a noté \( \vdeg* = \deg \var* \) et \( \vdim = \dim \var \), ainsi que
\begin{align}
  \Lambda & = \text{ne dépend que des degrés et dimensions ambiantes}
  \\
  f_1 & = \text{est une fonction décroissante}
  \\
  f_2 & = \text{est une fonction décroissante}
\end{align}
qui ne dépendent pas du choix de \( \ex \), de sorte que l'ensemble \(
  \varset(\ex) \) ne dépend de \( \ex \) que \lat{via} la condition \( \ex \in
  \var \).

Si \( \ex \) satisfait \eqref{e:Vbig}, cet ensemble ne contient que des
variétés dont tous les facteurs sont de dimension au moins \( 1 \). En effet,
dans le cas contraire, on aurait \( \var* = \ex* \) pour un certain \( \fct
\), ce dont on déduirait successivement
\begin{alignat}{2}
  \wts* \hautl{\ex*}
  & \le (\dots) \Lambda^{f_2(u)} \wts[1]
  &\quad& \text{par \eqref{e:varset-ht},}
  \\
  \wts[1] \hautl{\ex[1]}
  & \le (1+\zeta) (\dots) \Lambda^{f_2(u)} \wts[1]
  && \text{par \eqref{e:wt-ratio},}
\end{alignat}
qui contredirait précisément \eqref{e:Vbig}.

Or cet ensemble \( \varset(\ex) \) n'est certainement pas vide, puisqu'il
contient au moins \( \va^\puiss \) ; il possède donc au moins un élément
minimal. La proposition suivante implique que cet élément a au moins un
facteur réduit à un point, ce qui, nous venons de le voir, fournit la
contradiction prouvant le théorème \ref{t:vojta-div}.

\begin{prop} \label{p:varset-notmin}
  Soit \( \var \in \varset(\ex) \) n'ayant aucun facteur de dimension nulle.
  Alors il existe un \( \fct \) et une forme \( T \) ne dépendant que de \(
    \vmp* \), s'annulant en \( \ex \) mais pas identiquement sur \( \var \),
  telle que
  \begin{align}
    \deg T
    & \le \Lambda^{f_1(u)}
    \\
    \wts* \hautl T
    & \le (\dots) \Lambda^{f_3(u)} \wts[1]
    \pmm.
  \end{align}
\end{prop}

En effet, on note alors \( \var' \) la variété égale à \( \var \) sur tous les
facteurs sauf le \( \fct \)-ème où on choisit une composante irréductible de
\( \var* \cap \zeros T \) qui contient \( \ex* \), de sorte que \( \var' \)
est une variété strictement contenue dans \( \var \) et contenant \( \ex \).
Le théorème de \bsc{Bézout} donne alors
\worknote{LNM7, thm 3.4, coro 3.6 ou LNM6 thm, lemme 4.1}
\begin{equation}
  \deg \var*' \le \vdeg* \deg T
\end{equation}
et sa version arithmétique indique que
\begin{equation}
  \hautl{\var*'} \le \vdeg* \hautl{T} + \hautl{\var*} \deg T
  + \text{cte(dimension)} \vdeg* \deg T
  \pmm,
\end{equation}
ce qui montre que \( \var' \in \varset(\ex) \) (au vu des constantes qui
seront déterminées par ce point) et que \( \var \) n'était pas minimale.

\medskip

Il suffit donc d'établir la proposition \ref{p:varset-notmin} pour prouver le
théorème \ref{t:vojta-div} ; la démonstration de cette proposition nous
occupera le reste du chapitre. L'intérêt de l'énoncer immédiatement plutôt
qu'au moment où nous serons en mesure de l'établir et de pouvoir écarter
d'emblée quelques cas particuliers pour lesquels la méthode qui suit ne
s'applique pas (mais qui sont heureusement immédiats). Auparavant nous
introduirons quelques hypothèses commodes sur les coordonnées projectives
utilisées.

Désormais et jusqu'à la fin du chapitre, nous fixons une variété
\nomuse \var [chap]{Une variété \( \var[1] \times \dots \times \var[\puiss] \)
  appartenant à \( \varset(\ex) \), dont aucun facteur n'est de dimension
  nulle} \( \var = \var[1] \times \dots \times \var[\puiss] \)
satisfaisant aux hypothèses de la proposition ; en particulier elle n'est
contenue dans aucun hyperplan d'équation $\vmp*[0] = 0$ puisqu'elle contient
\( \ex \) qui n'est d'après \eqref{e:Vapx} sur aucun de ces hyperplans.  On
notera en outre $\vdim* = \dim \var*$ et $\vdim = \dim \var$ puis $\vdeg*
= \deg \var*$, et $\varfc*$ une forme de \bsc{Chow} de $\var*$. On note enfin
$\varid$ l'idéal multihomogène saturé de $\var$ et $\varid*$ ceux de ses
facteurs.
\nomuse{\vdeg*}[chap]{Le degré de \( \var* \)}
\nomuse{\vdim*}[chap]{La dimension de \( \var* \)}
\nomuse{\vdim }[chap]{La dimension de \( \var \)}
\nomuse{\varfc*}[chap]{La forme de \bsc{Chow} de \( \var* \) normalisée par le
  scolie~\ref{s:plong-adapt}}
\nomuse{\varid}[chap]{L'idéal homogène saturé de \( \var \)}

De façon générale, si $A$ est une algèbre graduée et $\Ideal$ un idéal
homogène, on notera $A_d$ et $\Ideal_d$ leur partie homogène de degré $d$ ; on
utilisera la même notation pour les algèbres et idéaux multigradués, où $d$
désignera une famille d'entiers.


\subsection{Plongements projectifs adaptés} \label{sec:plong-adapt}

\begin{tdef} \label{d:plong-adapt}
  Suivant \cite{remivg}, on dit qu'un plongement
  \(
    \iota\colon \anyvar \embedin \proj\dimp
  \)
  d'une variété de dimension \( \anydim \) est adapté si
  \begin{enumthm}
    \item \( \anyvar \cap \zeros{\anyvp[0], \dots, \anyvp[\anydim]}
        = \emptyset \) ;
    \item \( \korper{\anyvar} \) est engendré par
      \( \frac{\anyvp[1]}{\anyvp[0]}, \dots,
        \frac{\anyvp[\anydim+1]}{\anyvp[0]} \) ;
    \item \( \frac{\anyvp[\anydim+1]}{\anyvp[0]} \neq 0 \) dans \(
        \korper\anyvar \).
  \end{enumthm}
\end{tdef}

Le fait suivant, qui ne fait que rappeler \cite[partie~4.1, p.~114]{remivds},
explicite les principales propriétés d'une plongement adapté, qui est en fait
une version plus précise de la mise en position de \bsc{Noether}.

\begin{fact} \label{f:plong-adapt-gen}
  Si \( \anyvar \) de dimension \( \anydim \)  est plongée dans \( \projd \)
  de façon adaptée, alors les fonctions rationnelles
  \( \frac{\anyvp[1]}{\anyvp[0]}, \dots, \frac{\anyvp[\anydim]}{\anyvp[0]} \)
  forment une base de transcendance de \( \korper\anyvar \) sur \( \cdn \). De
  plus, \( \frac{\anyvp[\anydim+1]}{\anyvp[0]} \) est un élément primitif de
  \( \korper\anyvar \) sur \( \cdn( \frac{\anyvp[1]}{\anyvp[0]}, \dots,
    \frac{\anyvp[\anydim]}{\anyvp[0]} ) \).

  La projection linéaire \( \anyvar \to \proj\anydim \) obtenue en ne gardant
  que les \( \anydim + 1 \) variables est un revêtement ramifié.
\end{fact}

On contrôle en fait des relations de dépendance intégrale des dernières
variables sur la base de transcendance choisie.

\begin{fact} \label{f:plong-adapt-dep}
  Si le plongement \( \anyvar \embedin \projd \) est adapté, il existe des
  formes homogènes \( \poldep[][\ind] \) pour \( \ind \in \set{\anydim+1,
      \dots, \dimp} \) telles que :
  \begin{enumthm}
    \item \(
        \poldep[][\ind]
        \in
        \cdn [ \anyvp[0], \dots, \anyvp[\anydim], \anyvp[\ind] ]
        \cap \ideal\anyvar \) ;
    \item \( \poldep[][\ind] \) est unitaire de degré \( \anydeg \) en \(
        \anyvp[\ind] \) ;
    \item \( \deg \poldep[][\ind] = \anydeg \) ;
    \item \( \nv1{ \poldep[][\ind] } \le \nv1{ \chow\anyvar } \) ;
  \end{enumthm}
  où \( \anydeg \) est le degré de \( \anyvar \) dans ce plongement.
\end{fact}

\begin{proof}
  Le lemme~4.1 de \cite{remivds} donne explicitement des formes satisfaisant
  les trois premières conditions.

  Seule l'assertion sur la norme n'y est pas énoncée sous cette forme mais
  elle vient en remarquant que \( \poldep_\indi \) est une spécialisation de
  \( \varfc \) qui annule certaines variables et remplace les autres par des
  monômes unitaires, argument qui est utilisé la proposition~4.2 de la
  référence citée pour obtenir une estimation similaire dans un contexte
  légèrement différent.
\end{proof}

Nous montrons maintenant qu'il est possible de rendre adapté un plongement
donné tout en gardant fixe le diviseur \( \divi \), à peu de frais.

\begin{lem}
  Soit $\iota \colon \anyvar \embedin \projd$ un sous-schéma fermé intègre de
  degré $\anydeg$, non contenu dans l'hyperplan d'équation $\anyvp[0] = 0$.
  Il existe une matrice $M \in \GL_{\dimp*}(\Q)$, à coefficients entiers de
  valeur absolue (archimédienne) majorée par $\max(\frac\anydeg2, 1)$, telle
  que, si $e_M$ est l'automorphisme linéaire de $\proj\dimp$ associé à $M$,
  alors $e_M \circ \iota$ est un plongement adapté et que $\anyvp[0]$ soit
  invariant par ce changement de coordonnées.
\end{lem}

\begin{proof} \later
  On reprend la preuve de la proposition 2.2 de \cite{remivg} (p.~469). Au
  moment de choisir des formes linéaires $M_0, \dots, M_n$ telles que
  \begin{equation*}
    \chow \anyvar (M_0, \dots, M_\dimp) \neq 0
    \pmm,
  \end{equation*}
  on commence en fait par fixer $M_0 = \anyvp[0]$. Le polynôme $\varfc(M_0,
  \truc, \dots, \truc)$ est multihomogène de degré $\anydeg$ en chaque
  variable ; vu l'hypothèse sur $\anyvar$, il est non nul grâce au théorème
  fondamental de l'élimination. On peut donc choisir $M_1, \dots M_n$ comme
  dans \cite{remivg} et continuer la preuve sans autre modification.
\end{proof}

\begin{lem}
  Soit $\anyvar \colon \embedin \proj\dimp$ un sous-schéma fermé intègre, et
  $M \in \GL_{\dimp*}(\Qbar)$ à coefficients entiers dans l'intervalle $[-B,
  B]$, où $B$ est un réel positif fixé, et $e_M$ l'automorphisme linéaire de
  $\proj\dimp$ associé à $M$. Alors
  \begin{equation}
    \hautm[1]{e_M(\anyvar)}
    \le
    \hautm[1]{\anyvar}
    \cdot \bigl( \dimp** B \bigr)^{\anydeg\dimp**}
    \pmm.
  \end{equation}
\end{lem}

\begin{proof} \later
  Inégalité classique sur la norme $L_1$ et la spécialisation des polynômes.
\end{proof}

\begin{scho} \label{s:plong-adapt}
  Vu les hypothèses sur \( \var \) on peut sans perte de généralité supposer
  de plus que chaque plongement $\var* \embedin \projd$ est adapté, quitte à
  multiplier la hauteur de \( \var* \) par
  \( \bigl (\dimp** \max (\frac{\vdeg*}2, 1) \bigr)^{\vdeg*\dimp**} \).
  \worknote[later]{Il faut aussi estimer le coût sur les formules de
    multiplication et soustraction.}

  On supposera donc désormais que chaque $\var*$ est plongé de façon adaptée
  dans son facteur $\proj\dimp$. Ceci implique que
  $\varfc*(\vmp*[0], \dots,\vmp*[\vdim*])$
  est non nul pour tout $\fct$ ; on normalisera donc notre choix de $\varfc*$
  (qui est unique à une constante multiplicative près) en imposant que cette
  quantité vaille $1$.

  On notera par ailleurs \( \poldep** \) une relation de dépendance
  de \( \vmp*[\indi] \) sur \( \vmp*[0], \dots, \vmp*[\vdim*] \)
  telle que donnée par l'application du fait~\ref{f:plong-adapt-dep} au
  facteur \( \var* \).
\end{scho}

Nous pouvons maintenant énoncer les cas particuliers à exclure dans la
démonstration de la proposition~\ref{p:varset-notmin}.

\begin{scho} \label{s:part-cases}
  On peut supposer que :
  \begin{enumthm}
    \item \( \cex** \neq 0 \) pour tous \( \fct \) et \( \ind \) ;
    \item \( \pden_\fct(\cex*) \neq 0 \) pour tout \( \fct \).
  \end{enumthm}
\end{scho}

En effet, dans ces cas la conclusion de la proposition est immédiate.


\subsection{Plongement abélien pondéré, notations et stratégie}
\label{sec:wemb}

Nous utiliserons un plongement, dit \emph{éclatant} ou pondéré par \( \wt \)
et défini par
\nomuse \wemb [chap] {Plongement éclatant introduit à la
  section~\ref{sec:wemb}}
\begin{align} \label{e:def-wemb}
  \wemb \colon \var
  & \longto \va^\puiss \times \va^{\puiss-1}
  = \va^{2\puiss-1} \subset (\projd)^{2\puiss-1}
  \\
  (\pmp_1, \dots, \pmp_\puiss)
  & \longmapsto
  (\pmp_1, \dots, \pmp_\puiss;
  \wt[1] \pmp_1 - \wt** \pmp_\puiss, \dots,
  \wt[\puiss-1] \pmp_{\puiss-1} - \wt** \pmp_\puiss)
  \pmm.
\end{align}
qui nous permettra d'exploiter le scolie~\ref{s:wt-diff-small}. Il nous faudra
représenter ce morphisme par des familles de polynômes dans les plongements
utilisés.

On munit l'espace d'arrivée $(\projd)^{2\puiss-1}$ des coordonnées
multihomogènes $\vmp, \vmpi = \vmp[1], \dots, \vmp[\puiss], \vmpi[1], \dots,
\vmpi[\puiss*]$ ; dans ce contexte quand $\fct$ et $\fcti$ sont deux indices
non précisés, on supposera implicitement $1 \le \fct \le \puiss$ et $1 \le
\fcti \le \puiss-1$.

\nomuse \wemba [chap] {Représentation polynomiale du plongement éclatant}
Un représentation locale de \( \wemb \) au voisinage de \( \ex \), est
constituée d'un ouvert \( \ex \in \opdef \subset \var \) et d'une famille de
formes multihomogènes définissant un morphisme \( \wemba \) tel que le
diagramme suivant, dont les flèches verticales sont les projections
canoniques, commute :
\begin{equation}
  \xymatrix{
    \cdn [\vmp, \vmpi]                          \ar [r] ^{\wemba}   \ar [d]
    & \cdn [\vmp]                                                   \ar [d]
    \\ \cdn [\vmp, \vmpi] / \ideal{\wemb(\var)} \ar [r] ^{\wemb^*}
    & ( \cdn [\vmp] / \varid )_\opdef
  }
\end{equation}
et en particulier, que \( \pi \circ \wemba \) soit bien défini. On peut en
fait choisir
\begin{align}
     \wemba \colon \cdn [\vmp, \vmpi]
  &  \to \cdn[\vmp]
  \\ \vmp*
  &  \mapsto \vmp*
  \\ \vmpi*
  &  \mapsto L\pexp{\wt[\fcti], \wt[\puiss]}(\vmp[\fcti], \vmp[\puiss])
\end{align}
où les \( L\pexp{\wt[\fcti], \wt[\puiss]} \) sont prises parmi les familles
de formes données par le lemme~\ref{l:hclab} de sorte que leur ouvert de
définition \( \opdef \) contienne \( \ex \). Le choix de ces familles pourra
être précisé ultérieurement ; pour l'instant on utilisera seulement les
estimations de degré et hauteurs qui sont uniformément valables. Par exemple,
on a toujours le lemme suivant, qui est immédiat.

\begin{lem} \label{l:deg-wemba}
  Soient $\wemba$ un morphisme d'algèbre comme ci-dessus et $F \in
  \Qbar[\vmp, \vmpi]$ une forme multihomogène de multidegré $(\alpha,
  \beta)$ où $\alpha \in \N^\puiss$ et $\beta \in \N^{\puiss-1}$. On a alors
  \begin{equation}
    \deg \wemba(F)
    =
    \bigr(
    \alpha_1 + 2 \beta_1 \wts[1],
    \dots,
    \alpha_{\puiss-1} + 2 \beta_{\puiss-1} \wts[\puiss-1],
    \alpha_\puiss + 2 \lgr\beta \wts[\puiss]
    \bigl)
    \pmm.
  \end{equation}
\end{lem}

\medskip

Avant de passer aux estimations de hauteur, il est utile d'introduire quelques
paramètres qui seront utilisés tout au long de ce chapitre et qui contrôlent
notamment le degré en lequel nous établirons ces estimations.

Introduisons deux réels \( \epsz, \epsi \) strictement positifs tels que :
\begin{gather} \label{e:epsz-epsi-siegel}
  \frac {
    \epsi^\puiss (\puiss-1)
    (2\puiss + \epsz) ^{\vdim[\puiss]-1}
    (\frac94 + \epsz) ^{\lgr\vdim - \puiss - \vdim[\puiss] + 1}
    \prod\fctrange \vdim*
  }{
    \puiss! \epsz^{\vdim[\puiss]}
  }
  \le
  \frac12
  \\ \label{e:epsz-epsi-extra}
  \epsz \le \frac {\eps \epsi} {8\puiss}
\end{gather}
On choisira en fait \( \epsz \) rationnel, et on introduit un entier \( \delta
\) (destiné à tendre vers l'infini) tel que \( \epsz \delta \) soit également
entier. On pose alors
\begin{equation} \label{e:d-dp-def}
  \begin{aligned}
    d & = \bigl(
      \epsz \wts[1],
      \dots,
      \epsz \wts[\puiss],
      1, \dots, 1
    \bigr) \in \Q^{2\puiss-1}
    \\
    d' & = \bigl(
      \wts[1] (2 + \epsz),
      \dots,
      \wts[\puiss*] (2 + \epsz),
      \wts[\puiss] (2\puiss - 2 + \epsz)
    \bigr) \in \Q^\puiss
  \end{aligned}
\end{equation}
de sorte que, si $F$ est une forme de multidegré $\delta d$ dans $\cdn[\vmp,
\vmpi]$, son image par $\wemba$ est de multidegré exactement $\delta d'$
d'après le lemme précédent.



\section{Construction d'une forme auxiliaire} \label{sec:siegel}

L'objectif de cette section est de construire une forme non nulle sur \( \var
\), provenant d'une forme sur \( \wemb(\var) \), de degré prescrit, de
hauteur contrôlée, et d'indice élevé le long de \( \divi \) dans \( \var \).

Nous commençons par définir la notion d'indice utilisée, puis précisons la
stratégie de construction de la forme recherchée. Nous établissons ensuite les
estimations de dimension et de hauteur nécessaires avant de conclure en
appliquant un lemme de \bsc{Siegel}.


\subsection{Idéal d'annulation}

Pour tout \( \imp \in \N^{\puiss\dimp**} \), notons
\nomuse {\wtsum*} [chap] {Somme pondérée servant à définir l'indice
  d'annulation le long de \( \divi \)}
\begin{equation}
  \wtsum*( \imp )
  =
  \frac {\imp[1][0]} {\wts[1]} + \dots
  + \frac {\imp[\puiss*][0]} {\wts[\puiss*]}
  + \frac {\imp[\puiss][0]} {\puiss*}
\end{equation}
On définit l'indice le long de \( \divi \) d'une forme \( H = \sum h_\imp
  \vmp^\imp \) par :
\nomuse {\inda*} {Indice d'annulation le long de \( \divi \), défini
  par~\ref{e:inda-def}}
\begin{equation} \label{e:inda-def}
  \inda*(H)
  =
  \max \set{
    \alpha \in \R
    \text{ tels que }
    \wtsum*(\imp) \le \alpha \implies h_\imp = 0
  }
  \pmm.
\end{equation}
On introduit alors, pour \( \beta > 0 \), les idéaux
\nomuse {\vanish[\beta]} [chap] {Idéal des formes s'annulant le long de \(
    \divi \) avec un indice (pondéré par \( \wt \)) supérieur à \( \beta \),
  défini par~\ref{e:vanish-def}}
\nomuse {\vanish*[\beta]} [chap] {Idéal d'annulation définir par \(
    \vanish*[\beta] + \varid \)}
\begin{equation} \label{e:vanish-def}
  \vanish*[\beta]
  = \left\{
    H \in \cdn [\vmp]
    \text{ tels que }
    \inda*(H) \ge \beta
  \right\}
\end{equation}
et $\vanish[\beta] = \vanish*[\beta] + \varid$. Par abus, on notera de même
l'image de \( \vanish[\beta] \) dans $\ring\var = \cdn [\vmp] / \varid$, ce
qui est légitime puisqu'on a fait en sorte que $\varid \subset
\vanish[\beta]$. On dira alors d'une forme qu'elle s'annule le long de \(
  \divi \) dans \( \var \) avec un indice au moins \( \beta \) si elle
appartient à \( \vanish[\beta] \).  Par ailleurs, on notera également
\nomuse {\stairs[\beta]} [chap] {Ensembles des multiindices tels que \(
    \wtsum*(\imp) \le \beta \), défini par~\ref{e:stairs-def}}
\begin{equation} \label{e:stairs-def}
  \stairs[\beta] = \set{
    \imp \in \N^{\puiss\dimp**}
    \text{ tels que }
    \wtsum*(\imp) \le \beta
  }
\end{equation}
l'ensemble des indices correspondant aux coefficients qui doivent être nuls
pour assurer l'appartenance à \( \vanish*[\beta] \).


\subsection{Stratégie de construction de la forme auxiliaire}

Il s'agit de trouver une forme multihomogène non nulle sur $\var$, de degrés
prescrits et de hauteur contrôlée, provenant d'une forme sur $\wemb(\var)$ et
ayant indice élevé le long de \( \divi \) dans \( \var \).  Plus précisément,
en conservant les notations du paragraphe précédent et
notamment~\ref{e:d-dp-def}, on cherche une forme
\begin{equation}
  F'
  \in \cdn[\vmp]_{\delta d'}
  \cap \vanish
  \cap \wemba(\cdn[\vmp, \vmpi]_{\delta d})
\end{equation}
non nulle modulo l'idéal de $\wemb(\var)$, ou de façon équivalente (avec $F' =
\wemba(F)$), on cherche une forme
\begin{equation}
  F
  \in \cdn[\vmp, \vmpi]_{\delta d}
  \cap \wemba^{-1}(\vanish)
\end{equation}
non nulle modulo l'idéal de $\var$, de hauteur relativement petite.

Ce problème est un cas classique d'application du lemme de \bsc{Siegel}.
Cependant, il est difficile d'expliciter un système essentiellement minimal
d'équations linéaires exprimant l'appartenance à $\vanish$. Il est en
revanche immédiat d'expliciter un tel système pour $\vanish*$, mais
il comporte beaucoup trop d'équations (géométriquement, il correspond à une
condition d'annulation beaucoup trop forte : avoir un indice élevé le long de
$\divi$ dans $\projd$, alors qu'on travaille avec des formes sur $\var$).

Nous allons donc devoir procéder à quelques réductions pour pouvoir écrire le
système d'équations. L'idée fondamentale est la suivante : ramener $F'$ dans
un sous-espace de $\cdn[\vmp]_{\delta d'}$ dans lequel $\vanish$
est de codimension suffisamment petite pour pouvoir demander que l'image de
$F$ par ces réductions soit dedans, sans pour autant avoir trop d'équations.

Pour commencer, on choisit dans $\cdn[\vmp, \vmpi]_{\delta d}$ un
supplémentaire de $(\ideal{\wemb(\var)})_{\delta d}$ engendré par des
monômes, que l'on notera $\Ideal_{\delta d}$, et dans lequel on cherchera
$F$.

On considérera des sous-espaces vectoriels de partie homogènes de
$\cdn[\vmp]$ définis par des restrictions de degré ; introduisons à cet effet
quelques notations. Pour chaque $C \in (\N \cup \{ +\infty
  \})^{\puiss(\dimp+1)}$, on notera
\begin{equation} \label{e:C-spaces}
  \cdn[\vmp]^C
  = \{
    F \in \cdn[\vmp]
    \text{ tels que }
    \deg_{\vmp*[k]} F \le C\pexp{\fct}[k]
    \quad \forall i, k
    \}
  \pmm.
\end{equation}
On utilisera trois tels vecteurs $C'$, $C''$, $C'''$, définis respectivement
par
\begin{gather} \label{e:C-i-iii}
  (C')\pexp\fct[k] =
  \begin{cases}
    +\infty & \text{si $k \le \vdim*$} \\
    \vdeg* - 1 & \text{sinon}
  \end{cases}
  \qquad
  (C'')\pexp\fct[k] =
  \begin{cases}
    +\infty & \text{si $k \le \vdim* + 1$} \\
    0 & \text{sinon}
  \end{cases}
  \\
  C''' = \min(C', C'')
  \pmm,
\end{gather}
pour $0 \le \fct \le \puiss$ et $0 \le k \le \dimp$, et avec l'ordre produit
évident pour la dernière définition. Par ailleurs, on notera $C'_\Delta$ et
$C'''_\Delta$ les vecteurs obtenus en remplaçant $D$ par un certain $\Delta$
dans la définition précédente.

Le plan général de l'écriture du système se résume alors par le diagramme
commutatif suivant, où l'on note
\( \ring{\wemb(\var)} = \cdn[\vmp, \vmpi] / \ideal{\wemb(\var)} \)
et
\( \ring{\opdef} = (\cdn[\vmp])_\opdef \)
et où les flèches verticales sont les morphismes de réduction.

\begin{equation} \label{e:diag-f-aux}
  \xymatrix{
    \Ideal_\Dz            \ar [r] ^{\wemba}        \ar [d] _{\sim}
    & \cdn [\vmp]_\Dii   \ar [r] ^{\rdiv^\Dii}    \ar [d]
    & \cdn [\vmp]\Ci     \ar [r] ^{\relim}        \ar [d]
    & \cdn [\vmp]\Cii    \ar [r] ^{\rdiv^\Diii}   \ar [d]
    & \cdn [\vmp]\Ciii                            \ar [d]
    \\ \ring{\wemb(\var)} \ar [r] ^{\wemb^*}
    & \ring\opdef         \ar [r] ^{\I}
    & \ring\opdef         \ar [r] ^{R \cdot \null}
    & \ring\opdef         \ar [r] ^{\I}
    & \ring\opdef
  }
\end{equation}

Les morphismes $\rdiv$ et $\relim$ seront précisés ultérieurement ; le premier
représente une réduction par division euclidienne par certaines relations de
dépendance intégrale, les second permet d'éliminer les variables en utilisant
des relations linéaires dans $\korper\var$ et correspond modulo $\varid$ à
une multiplication par une certaine forme $R$ de degré $r$ ne dépendant que de
$\var$, de façon à éliminer les dénominateurs apparaissant dans ces relations
linéaires. Notons qu'il est essentiel d'effectuer $\rdiv$ avant $\relim$ pour
que le degré en les variables à éliminer ne dépende pas de $\delta$, ce qui
assure que $R$ ne dépend en effet que de $\var$.

Compte tenu de la présence de ce dénominateur $R$, révisons une dernière fois
notre objectif : il s'agira en fait de construire une forme $F \in
\Ideal_{\delta d} \setminus \{0\}$ telle que
\begin{equation}
  \rdiv^{\delta d' + r}(\relim(\rdiv^{\delta d'}(\wemba(F))))
  \in \vanish*
  \pmm,
\end{equation}
ce qui implique que $R \cdot \wemba(F) \in \vanish$ car
\begin{equation}
  R \cdot \wemba(F)
  - \rdiv^{\delta d' + r}(\relim(\rdiv^{\delta d'}(\wemba(F))))
  \in \varid
  \subset \vanish
  \pmm.
\end{equation}

Au final, le système d'équations auquel on appliquera le lemme de \bsc{Siegel}
sera composé des équations définissant $\vanish$ dans
\( \cdn[\vmp]_{\delta d' + r'}^{C'''} \),
tirées en arrière sur $\Ideal_{\delta d}$ par
\( \rdiv^{\delta d'} \circ \relim \circ \rdiv^{\delta d' + r} \circ \wemba \).
Pour pouvoir procéder, nous devons donc :
\begin{enumerate}
  \item calculer la codimension de $\Ideal_{\delta d}$ ;
  \item calculer la codimension de $\vanish*$ dans
    $\cdn[\vmp]_{\delta d' + r}^{C'''}$ et s'assurer qu'elle est plus
    petite que la dimension précédente ;
  \item expliciter les morphismes de la première ligne du diagramme et majorer
    la hauteur de leurs matrices dans les bases monomiales évidentes.
\end{enumerate}

Dans tous ces calculs, $\delta$ sera moralement arbitrairement grand devant
les autres termes, on n'explicitera donc à chaque fois que le terme de plus
haut degré en $\delta$ ; ainsi lorsqu'on utilisera la notation $o(\truc)$ ou
$\sim$, il s'agira d'équivalents quand $\delta$ tend vers l'infini.


\subsection{Deux calculs de dimension} \label{sec:comp-dim}

La dimension de $\Ideal_{\delta d}$ est donnée par le théorème de
\bsc{Hilbert} multihomogène dès qu'on connaît les différents multidegrés de
$\wemb(\var)$. Il est \lat{a priori} difficile de tous les calculer, mais il
comme suffit en fait de minorer la dimension, le lemme suivant nous donne tout
ce qu'on aura besoin de savoir sur le degré.

\begin{lem}
  Avec les notations précédentes, on a
  \begin{equation}
    \deg_{(0, \dots, 0, \vdim[\puiss]; \vdim[1], \dots, \vdim[\puiss-1])}
    \bigl( \wemb(\var) \bigr)
    =
    \prod\fctrange
    \vdeg* \wt ^{2\vdim*}
    \pmm.
  \end{equation}
\end{lem}

\begin{proof}
  Ce degré est donné par le cardinal de l'intersection de $\wemb(\var)$ avec
  des hyperplans génériques choisis de la façon suivante : $\vdim[\puiss]$
  provenant du $\puiss$-ième facteur $\projd$, et $\vdim*$ hyperplans
  provenant du facteur $\puiss + \fct$ pour $\fct \in \{1, \dots, \puiss-1\}$.

  On commence par choisir les hyperplans sur le facteur $\puiss$ : on remarque
  qu'ils se remontent par $\wemb$ en des hyperplans sur le dernier facteur de
  l'espace de départ $(\projd)^\puiss$. Ainsi, couper $\wemb(\var)$ par ces
  hyperplans revient à imposer à $\point_\puiss$ de parcourir un ensemble de
  cardinal $\vdeg[\puiss]$.

  Fixons maintenant un point $p$ dans cet ensemble. On constate que
  $\wemb(\var) \cap \zeros{\vmp** = p}$ coïncide avec l'image de
  \begin{align}
    \wemb[\wt, p]'
    \colon
    \var[1] \times \dots \times \var[\puiss]
    & \to
    \va^{2\puiss-1}
    \\
    (\point_1, \dots, \point_{\puiss-1})
    & \mapsto
    (\point_1, \dots, \point_{\puiss-1}, p;
    \wt[1] \point_1 - p,
    \dots,
    \wt[\puiss-1] \point_{\puiss-1} - p)
  \end{align}
  qui est le produit d'un point par des variétés de la forme
  $\wemb[\wt*, p]''(\var*)$ pour $\fct$ variant de $1$ à $\puiss-1$ en
  notant
  \begin{align}
    \wemb[\wt*, p]''
    \colon
    \var*
    & \to
    \va^2
    \\
    \point
    & \mapsto
    (\point, \wt* \point - p)
  \end{align}
  Il suffit donc de calculer le degré de ces variétés. La translation par $p$
  n'ayant pas d'influence sur le degré, il suffit de regarder l'action de la
  multiplication par $\wt*$. Or, celle-ci pouvant être représentée
  globalement par des formes de degré $\wts*$, en tirant en arrière par
  $\wemb[\wt*, p]''(\var*)$ une famille de $\vdim*$ hyperplans
  génériques sur le second facteur, on obtient des hypersurfaces génériques de
  degré $\wts*$ qui coupent donc $\var*$ en $\vdeg*
  \wt*^{2\vdim*}$ points.

  Le résultat suit en prenant le produit et en se rappelant que
  $\wt[\puiss] = 1$.
\end{proof}

\begin{lem}
  Avec les notations précédentes, on a
  \begin{align}
    \dim \Ideal_{\delta d}
    \ge
    \frac{ \epsz^{\vdim[\puiss]}
      \prod\fctrange \vdeg* \, \wt* ^{2\vdim*}
      }{ \vdim ! }
    \delta^{\vlg\vdim}
    + o( \delta^{\vlg\vdim} )
    \pmm.
  \end{align}
\end{lem}

\begin{proof}
  Il suffit d'appliquer le théorème de \bsc{Hilbert} multihomogène en
  utilisant le lemme précédent, car une somme de nombre positifs est minorée
  par chacun de ses termes. Il vient
  \begin{align}
    \dim \Ideal_{\delta d}
    & =
    \Biggl(
    \sum_{\substack{ t \in \N^{2\puiss-1} \\ \vlg t = \vlg u }}
    \deg_t \wemb(\var) \frac{ d^t }{ t! }
    \Biggr)
    \delta^{\vlg\vdim}
    + o( \delta^{\vlg\vdim} )
    \\
    & \ge
    \deg_{(0, \dots, 0, \vdim[\puiss]; \vdim[1], \dots, \vdim[\puiss-1])}
    \bigl( \wemb(\var) \bigr)
    \cdot
    \frac { \epsz^{\vdim[\puiss]} }{ \vdim ! }
    \delta^{\vlg\vdim}
    + o( \delta^{\vlg\vdim} )
  \end{align}
  par définition de \( d \), voir~\eqref{e:d-dp-def}.
\end{proof}

Nous allons maintenant calculer la codimension de $\vanish*$ dans
$\cdn[\vmp]\Ciii$. On introduit à cet effet le sous-ensemble de $\stairs$
suivant :

\begin{equation} \label{e:stairs-c3}
  \begin{split}
    (\stairs)\Ciii
    & =
    \Biggl\{
      ( \imp[1], \dots \imp[\puiss] )
      \in
      \prod\fctrange \bigl(
        \N^{\vdim* + 1}
        \times \{ 0, \dots, \vdeg* - 1 \}
        \times \{ 0 \}^{\dimp - \vdim* - 1}
      \bigr)
      \\ & \quad
      \text{tel que }
      \wtsum*(\lambda) \le \delta \epsi
      \text{ et }
      \lgr{\imp*}
      = \delta d' + r \quad \forall \fct
    \Biggr\}
    \pmm,
  \end{split}
\end{equation}
dont il s'agit de calculer le cardinal. En effet,
\( \vanish* \cap \cdn [\vmp]\Ciii \)
est le sous-espace engendré par les monômes dont les exposants ne sont pas
dans \( (\stairs)\Ciii \).

\begin{lem}
  Avec les notations précédentes,
  \begin{align}
    \card (\stairs)\Ciii
    & \le
    \bigl(\delta^{\lgr\vdim} + o(\delta^{\lgr\vdim})\bigr)
    \prod\fctrange \vdeg* \wt*^{2\vdim*}
    \\ & \qquad \cdot
    \frac {
      \epsi^\puiss (\puiss-1)
      (2\puiss + \epsz) ^{\vdim[\puiss]-1}
      (\frac94 + \epsz) ^{\lgr\vdim - \puiss - \vdim[\puiss] + 1}
      }{
      \puiss! \prod\fctrange (\vdim* - 1)!
      }
  \end{align}
\end{lem}

\begin{proof}
  Pour choisir un point dans \( (\stairs)\Ciii \), on peut commencer par
  choisir \( \imp*[\vdim*] \) entre \( 0 \) et \( \vdeg* \)
  pour tout \( \fct \), ce qui représente \( \prod\fctrange \vdeg* \)
  possibilités.

  On peut ensuite choisir \( \imp[1][0], \dots \imp[\puiss][0] \)
  sujets à la seule condition
  \begin{equation}
    \wtsum*(\lambda) \le \delta \epsi \pmm.
  \end{equation}
  Le lemme~2.14.5 de \cite{farhith} donne le nombre de choix possibles, qui
  est
  \begin{equation}
    \frac {\prodwt} {\puiss !} (\delta\epsi)^\puiss
    + o(\delta^\puiss)
    \pmm.
  \end{equation}

  Il reste alors à choisir pour tout \( \fct \) un élément de l'ensemble
  \begin{equation}
    \left\{
      (\imp*[1],  \dots, \imp*[\vdim*])
      \in \N ^{\vdim*}
      \text{ tels que }
      \sum_{j=1}^{\vdim*} \imp*[j]
      =
      \delta d'_\fct + r_\fct - \imp*[0] - \imp*[\vdim*]
    \right\}
    \pmm,
  \end{equation}
  qui est de cardinal
  \begin{align}
    \binom {
      \delta d'_\fct + r_\fct - \imp*[0] - \imp*[\vdim*] - 1
      }{
      \vdim* - 1
      }
    & \le
    \binom {
      \delta d'_\fct + r_\fct - 1
      }{
      \vdim* - 1
      }
    \\
    & \le
    \frac {(d'_\fct)^{\vdim* - 1}} {(\vdim* - )!} \delta^{\vdim* - 1}
    + o( \delta^{\vdim* - 1} )
  \end{align}
  On prend alors le produit :
  \begin{equation}
    \card (\stairs)\Ciii
    \le
    \frac {\prodwt} {\puiss !} (\delta\epsi)^\puiss
    \cdot \prod\fctrange
    \frac {(d'_\fct)^{\vdim* - 1}} {(\vdim* - )!}
    \vdeg* \delta^{\vdim* - 1}
    + o( \delta^{\lgr\vdim} )
  \end{equation}
  et le résultat suit en remplaçant \( d'_\fct \) par sa valeur : \( 2\puiss +
  \epsz \) si \( \fct = \puiss \) et \( \wts* (\frac94 + \epsz) \) sinon.
\end{proof}

Au vu des deux lemmes précédents et de~\ref{e:epsz-epsi-siegel}, on a
\begin{equation} \label{e:good-codim}
  \frac {\card (\stairs)\Ciii} {\dim \Ideal_{\delta d}}
  \le
  \frac12 + o(1)
  \pmm.
\end{equation}


\subsection{Trois lemmes de hauteurs}

Nous allons maintenant expliciter les différents morphismes intervenant dans
l'écriture du système et contrôler la hauteur de leurs matrices, en commençant
par le morphisme \( \wemba \) qui est déjà bien défini.

On étudie en fait le prolongement
\begin{equation}
  \wemba \colon
  \cdn [\vmp, \vmpi]_\Dz
  \to
  \cdn [\vmp]_\Dii
\end{equation}
donné par les mêmes formules. La base évidente de ce nouvel espace de départ
est formée par les monômes \( \vmp^p \vmpi^q \) pour
\begin{equation}
  (p, q)
  \in \N^{\puiss(\dimp + 1)} \times \N^{(\puiss - 1) (\dimp - 1)}
  \text{ tel que }
  \lgr{p\mexp*} = \Dz*
  \text{ et }
  \lgr{q\mexpi*} = \Dz_{\puiss + \fcti}
  \pmm,
\end{equation}
où \( d \) est défini par~\eqref{e:d-dp-def}.

% Note: on ne fait pas ce lemme dans sub:wemb
% car on utilise la définition de d pour simplifier le résultat
\begin{lem} \label{l:hmat-wemba}
  Avec les notations précédentes, l'application linéaire
  \begin{equation}
    \wemba \colon
    \cdn [\vmp, \vmpi]_\Dz
    \to
    \cdn [\vmp]_\Dii
  \end{equation}
  est représentée dans les bases canoniques par une matrice de colonnes
  \(
  c_{p, q} = \wemba(\vmp^p\vmpi^q)
  = \sum_{\lgr s = \Dii - \lgr p} c_{p, q; s} \vmp^{s+p}
  \)
  satisfaisant
  \begin{equation}
    \nv1{c_{p, q}}
    \le
    \left(
    \prod\fctirange
    \nv\infty \coi ^{2\wtis* - 1}
    \bigl(
    4^{g\wtis*} (\dimp + 1)^{\wtis* - 1}
    \bigr) ^\dv
    \right) ^\delta
    \pmm.
  \end{equation}
\end{lem}

\begin{proof}
  La définition de \( \wemba \) montre que \(
  \wemba( \vmp^p \vmpi^q ) = \vmp^p \wemba(\vmpi^q) \), on a donc \(
  \nv1 {c_{p, q}} = \nv1 {\wemba(\vmpi^q)} \). Ainsi, avec le
  corollaire~\ref{c:addsub-form} et le fait~\ref{f:mult-form}, on a
  \begin{align}
    \nv1 {c_{p, q}}
    & \le
    \prod\fctirange \prod\indrange
    \nv1{ S^{-}_\ind\bigl( Q_{\wti*}(\vmp[\fcti]), \vmp** \bigr) }
    ^{q\pexp\fcti[\ind]}
    \\ & \le
    \prod\fctirange \prod\indrange \left(
    \nv1{ S^{-}_\ind } \max\indirange \nv1{ Q_{\wti*, \indi} }^2
    \right) ^{q\pexp\fcti[\ind]}
    \\ & \le
    \prod\fctirange \left(
    \nv\infty \coi \cdot 4^{\genre \dv} \cdot \Bigl(
    \nv\infty \coi ^{\wtis*-1} (2^\genre \sqrt{\dimp+1})^{\dv \wtis*-1}
    \Bigr) ^{q\pexp\fcti[\ind]}
    \right)
    \\ & \le
    \left(
    \prod\fctirange
    \nv\infty \coi ^{2\wtis*-1} \bigl(
    4^{\genre\wtis*} (\dimp+1)^{\wtis*-1}
    \bigr) ^\dv
    \right) ^\delta
  \end{align}
  en remarquant que, par définition de \( d \), on a \( \lgr{q\mexpi*} =
    \delta \) pour tout \( \fcti \) (voir \eqref{e:d-dp-def}).
\end{proof}

Comme on l'a annoncé, le morphisme \( \rdiv \) revient à diviser par des
relations de dépendance intégrale. On énonce pour commencer un résultat de
réduction modulo de telles relations sous une forme un peu générale avant de
l'appliquer au cas qui nous intéresse.

\begin{lem}
  Pour \( \fct \in \{ 1, \dots, \puiss \} \) et \( \ind \in \{ \vdim* + 1,
  \dots \dimp \} \), on se donne :
  \begin{enumthm}
    \item \( \Delta\mexp* \in \N^* \) ;
    \item \( P\mexp*[\ind]
      \in
      \cdn [ \vmp*[0], \dots, \vmp*[\vdim*], \vmp*[\ind] ] \)
      homogène de degré \( \Delta\mexp* \) et unitaire en \( \vmp*[\ind]
      \).
  \end{enumthm}
  On note \( N_\fct = \max_\ind \nv1 { P\mexp*[\ind] } \) et \( \Ideal \)
  l'idéal engendré par les \( P\mexp*[\ind] \). En tout multidegré
  \( f \in \N^\puiss \), il existe une (unique) application linéaire
  \begin{equation}
    \rdiv^f \colon \cdn [\vmp]_ f \to \cdn [\vmp]_ f^{C'_\Delta}
  \end{equation}
  qui est l'identité modulo \( \Ideal \) (voir~\eqref{e:C-spaces}
  et~\eqref{e:C-i-iii} pour la définition de l'espace d'arrivée). De plus les
  colonnes \( c_p \) de la matrice de cette application satisfont la majoration
  de norme
  \begin{equation}
    \nv1{c_p}
    \le
    \prod\fctrange \bigl(
    N_\fct \cdot (2 \Delta\mexp*)^\dv
    \bigr) ^{f_\fct}
  \end{equation}
  pour tout \( p \in \N^{\puiss(\dimp+1)} \) de multilongueur \( f \), et
  l'image de \( \cdn [\vmp]_ f ^{C''_\Delta} \) par \( \rdiv^f \) est
  contenue dans \( \cdn [\vmp]_ f ^{C'''_\Delta} \).
\end{lem}

\begin{proof}
  C'est essentiellement une variante du lemme~2.5 de~\cite{remivg}, notre
  résultat étant formulé différemment et dans un cadre d'apparence un peu
  moins générale ; la preuve suivra en tout cas les mêmes lignes. On commence
  par décomposer chaque \( P\mexp*[\ind] \) de la façon suivante :
  \begin{equation}
    P\mexp*[\ind]
    =
    \sum _{\alpha=1}^{\Delta\mexp*}
    P\mexp*[\ind, \alpha] \cdot (\vmp*[\ind])^{\Delta\mexp* - \alpha}
    \pmm,
  \end{equation}
  où \( P\mexp*[\ind] \in \cdn [ \vmp*[0], \dots, \vmp*[\vdim*] ] \).
  On pose ensuite, pour tout \( \fct \), tout
  \( \ind \in \{ 0, \dots, \dimp \} \) et tout
  \( p \in \N^{\puiss(\dimp+1)} \),
  \begin{equation}
    \rho_{\fct, \ind, p}
    =
    \begin{dcases*}
      ( \vmp*[\ind] ) ^{p\mexp*[\ind]}
      & si \( \fct \le \vdim* \) ;
      \\
      \sum _{\alpha=1}^{\Delta\mexp*}
      U_{p, \alpha, \Delta\mexp*}
      ( P\mexp*[\ind, 1], \dots, P\mexp*[\ind, \Delta\mexp*] )
      & sinon,
    \end{dcases*}
  \end{equation}
  où les polynômes \( U \) sont donnés par le lemme~2.4 de \cite{remivg}. En
  particulier, \( \rho_{\fct, \ind, p} \) est toujours congru à
  \( ( \vmp*[\ind] ) ^{p\mexp*[\ind]} \) modulo \( \Ideal \) et on a
  l'estimation de norme
  \begin{equation}
    \nv1{ \rho_{\fct, \ind, p} }
    \le
    \left(
    \nv1{ P\mexp*[\ind] } (2\Delta\mexp*)^\dv
    \right) ^{p\mexp*[\ind]}
    \pmm.
  \end{equation}
  On définit alors \( \rdiv^f \) par son action sur les monômes, en posant
  \begin{equation}
    c_p
    = \rdiv^f(\vmp^p)
    = \prod\fctrange \prod\indrange \rho_{\fct, \ind, p}
  \end{equation}
  et en prolongeant par linéarité. L'estimation de norme annoncée découle
  directement de la majoration précédente en prenant le produit. Par ailleurs,
  il est clair que si une forme ne fait intervenir que les variables \(
  \vmp*[\ind] \) pour \( \ind \le \vdim* + 1 \), il en est de même de son
  image.
\end{proof}

\begin{coro} \label{c:hmat-rdiv}
  Il existe des applications linéaires \( \rdiv^\Dii \) et \( \rdiv^\Diii \)
  faisant commuter le diagramme~\eqref{e:diag-f-aux}, dont les matrices dans
  les bases monomiales canoniques ont des colonnes de normes \( \nv1\truc \)
  majorées par
  \begin{equation}
    \prod\fctrange \left(
    \nv1{ \varfc* }
    \cdot ( 2 \vdeg* )^\dv
    \right) ^{ \delta d'_\fct }
    \cdot \expb^{ o(\delta) }
  \end{equation}
\end{coro}

\begin{proof}
  Découle directement du lemme précédent en utilisant les relations de
  dépendance fixées au dernier paragraphe du scolie~\ref{s:plong-adapt}.
\end{proof}

Intéressons-nous maintenant au morphisme \( \relim \).

\begin{lem}
  Soient, pour tout \( \fct \in \{ 1, \dots, \puiss \} \) et tout \( \ind \in
  \{ \vdim* + 1, \dots \dimp \} \), des formes
  \( S\mexp*[\ind] \in \cdn [ \vmp*[0], \dots, \vmp*[\vdim*] ] \) et
  \( T\mexp*[\ind] \in \cdn [ \vmp*[0], \dots, \vmp*[\vdim*+1] ] \)
  telles que \( \deg S\mexp*[\ind] + 1 = T\mexp*[\ind] \) et des entiers
  \( \Delta_\fct \). On note \( \Ideal_{S, T} \) l'idéal engendré par les
  \( S\mexp*[\ind] \vmp*[\ind] - T\mexp*[\ind] \).

  Il existe une forme \( R \) ne dépendant que des familles \( S \) et \(
  \Delta \), de degré noté \( r \), et, pour tout \( f \in \N^\puiss \), une
  application linéaire
  \begin{equation}
    \relim \colon
    \cdn [\vmp]_ {f}^{C'_\Delta}
    \to
    \cdn [\vmp]_ {f+r}^{C'_\Delta}
  \end{equation}
  qui est la multiplication par \( R \) modulo \( \Ideal_{S, T} \).

  De plus, on peut prendre
  \( R = \prod\fctrange \prod_{\ind = \vdim*+1}^{\dimp}
    ( S\mexp*[\ind] )^{\Delta_\fct} \) ; les colonnes \( c_q^{\relim} \) de la
  matrice de \( \relim \) dans la base canonique satisfont alors
  \begin{equation}
    \nv1{ c_q^{\relim} }
    \le
    \prod\fctrange N_\fct^{\Delta_\fct}
  \end{equation}
  pour tout \( q \) de multilongueur \( f \), où \( N_\fct \) majore
  \( \nv1{ S\mexp*[\ind] } \) et \( \nv1{ T\mexp*[\ind] } \) pour tout
  \( \ind \).
\end{lem}

\begin{proof}
  Soit \( R \) défini comme dans l'énoncé, et \( \vmp^q \) un monôme de
  l'espace de départ. Par hypothèse, \( q\mexp*[\ind] \le \Delta_\fct \) pour
  tout \( \fct \) et \( \ind \le \vdim* + 1 \), de sorte que l'on peut poser
  \begin{equation}
    \relim( \vmp^q )
    =
    \prod\fctrange \left(
    \prod_{\ind=1}^{\vdim*}
    (\vmp*[\ind])^{q\mexp*[\ind]}
    \prod_{\ind=\vdim*+1}^\dimp
    (T\mexp*[\ind])^{q\mexp*[\ind]}
    (S\mexp*[\ind])^{\Delta_\fct - q\mexp*[\ind]}
    \right)
  \end{equation}
  et prolonger par linéarité. On vérifie immédiatement que \( \relim(\vmp^q)
  \) est congru à \( R \cdot \vmp^q \) modulo \( \Ideal_{S, T} \), de même
  que l'estimation de norme annoncée.
\end{proof}

\begin{coro} \label{c:hmat-relim}
  Il existe une forme \( R \in \cdn [\vmp]^{C''} \) ne dépendant que de \(
  \var \) et un morphisme \( \relim \) tels que le
  diagramme~\eqref{e:diag-f-aux} commute. De plus, les colonnes de la matrice
  de \( \relim \) dans les bases canoniques ont leur norme \( \nv1\truc \)
  majorée par
  \begin{equation}
    \prod\fctrange
    \nv1{ N'_\fct }^{\vdeg*}
    \pmm,
  \end{equation}
  où \( N'_\fct \) est une constante ne dépendant que de \( \var* \).
\end{coro}

\begin{proof}
  Il suffit d'établir l'existence de familles \( S \) et \( T \) comme dans
  l'énoncé du lemme précédent, telles que \( \Ideal_{S, T} \subset \varid \) ;
  elle découle du fait qu'on a supposé le plongement adapté
  (scolie~\ref{s:plong-adapt}).
  En effet, d'après le fait~\ref{f:plong-adapt-gen}, pour tous \( \fct \) et
  \( \ind \), il existe des formes \( A\mexp*[\ind, \alpha] \) et \(
    B\mexp*[\ind, \alpha] \) dans
  \( \cdn [ \vmp*[0], \dots, \vmp*[\vdim*] ] \) telles que
  \begin{equation}
    \frac{ \vmp** }{ \vmp*[0] }
    =
    \sum_{\alpha = 0}^{\vdeg* - 1}
    \frac {A\mexp*[\ind, \alpha]} {B\mexp*[\ind, \alpha]}
    \left( \frac{ \vmp*[\vdim* + 1] }{ \vmp*[0] } \right) ^\alpha
    \mod \varid*
    \pmm.
  \end{equation}
  On obtient alors les familles \( S \) et \( T \) recherchées en multipliant
  les deux membres de l'égalité précédent par \( \vmp*[0] \) puis en réduisant
  au même dénominateur le membre de droite.
\end{proof}

Notons que pour construire la forme auxiliaire, on n'a absolument pas besoin
d'en savoir davantage sur \( R \) ni les \( N'_\fct \). On pourra néanmoins
vouloir expliciter une valeur acceptable de \( R \) plus tard.


\subsection{Construction finale de la forme auxiliaire}

Commençons par rappeler la version que nous utiliserons du classique lemme de
\bsc{Siegel}.

\begin{fact} \label{f:siegel}
  Pour toute matrice \( M \) de dimensions \( p \times q \) à coefficients
  dans un corps de nombres \( \cdn \) avec \( p < q \), il existe un vecteur
  \( x \in \cdn^q \) non nul tel que \( M x = 0 \) et satisfaisant
  \begin{equation}
    \hautl[\infty] x
    \le
    \frac q{p-q} \bigl( \hautl[\infty] M + \ln q \bigr) + c_\cdn
    \pmm,
  \end{equation}
  où \( c_\cdn \) est une constante ne dépendant que de \( \cdn \).
\end{fact}

\begin{proof} \later
  C'est le lemme de \bsc{Siegel} de \bsc{Bombieri} et \bsc{Vaaler}.
\end{proof}

Nous sommes maintenant en mesure de construire la forme auxiliaire.

\begin{prop} \label{p:build-aux}
  Sous les hypothèses et notations précédentes et si
  \( \delta \) est assez grand, il existe une forme \( F \in \cdn [\vmp,
    \vmpi] \), multihomogène de degré \( \delta d \) (où l'on rappelle que \(
    d \) est défini par~\eqref{e:d-dp-def}), non nulle modulo \(
    \ideal{\wemb(\var)} \), telle que \( R \cdot \wemba(F) \in \vanish \) et
  \begin{equation}
    \hautl[\infty] F
    \le
    \delta \Bigr(
    (4\puiss + 2\epsz) \sum\fctrange \wts* \hautl[1]{\varfc*}
    + (1 + \epsiii)\wts[1] \cdot \cst{f-aux}
    \Bigr)
    + o(\delta)
    \pmm,
  \end{equation}
  où
  \begin{equation}
    \newcst[]{f-aux}
    \le
    2 \hautl[\infty] \coi
    + 2 \ln (\dimp + 1)
    + (4\puiss + 2\epsz) \max\fctrange \ln(2\vdeg*)
    \pmm.
  \end{equation}
\end{prop}

\begin{proof}
  On reprend le diagramme~\eqref{e:diag-f-aux}, dont la première ligne est
  maintenant bien définie. On considère le produit \( M'' \) des matrices des
  applications \( \wemba \), \( \rdiv^\Dii \), \( \relim \) et \(
  \rdiv^\Diii \) composées dans cet ordre. Ses lignes sont naturellement
  indexées par l'ensemble des multiindices
  \begin{equation}
    \left\{
      \imp \in \prod\fctrange \bigl(
      \N^{\vdim* + 1}
      \times \{ 0, \dots, \vdeg* - 1 \}
      \times \{ 0 \}^{\dimp - \vdim* - 1}
      \bigr)
      \text{ tel que }
      \sum\indrange \imp** = \Diii
      \ \forall \fct
      \right\}
    \pmm.
  \end{equation}
  On ne garde de \( M \) que les lignes dont l'indice appartient à l'ensemble
  \( (\stairs)\Ciii \) défini par~\eqref{e:stairs-c3} pour former la matrice
  \( M' \), dont les colonnes sont naturellement indexées par les monômes de
  multidegré \( \Dz \) en \( (\vmp, \vmpi) \). On extrait alors de
  \( M' \) les colonnes indexées par les monômes constituant la base choisie
  de \( \Ideal_\Dz \) pour former la matrice \( M \) à laquelle nous allons
  appliquer le fait~\ref{f:siegel}, et dont on note \( p, q \) les dimensions.

  L'hypothèse faite sur \( \epsii \) et la relation~\eqref{e:good-codim}
  impliquent directement que \( p < q \) et que
  \begin{equation}
    \frac q {p-q}
    \le
    \frac 1 {\frac1\epsii - 1} + o(1)
    \le
    1 + o(1)
    \pmm.
  \end{equation}
  En remarquant de plus que \( \ln q = o(\delta) \), on voit qu'il existe une
  forme \( F \) satisfaisant aux conditions de l'énoncé mais de hauteur majorée
  par \( \hautl[\infty] M + o(\delta) \) ; il reste donc à estimer cette
  hauteur.

  On utilise le fait élémentaire suivant : si \( A \) et \( B \) sont deux
  matrices telles que le produit \( A \cdot B \) est défini, alors
  \( \hautl[\infty]{AB} \) est majorée par la somme de \( \hautl[\infty] A \)
  et d'un majorant de \( \hautl[1] c \) où \( c \) parcourt les colonnes
  de \( B \).

  Or, d'après le lemme~\ref{l:hmat-wemba}, la hauteur \( \Hautl[1] \) des
  colonnes de la matrice de \( \wemba \) est majorée par
  \begin{equation}
    \bigl( \delta + o(\delta) \bigr)
    \sum\fctirange \biggl(
    2\wtis* \hautl[\infty] \coi
    + \genre \wtis* \ln(4)
    + \wtis* \ln(\dimp + 1)
    \biggr)
    \pmm,
  \end{equation}
  d'après le corollaire~\ref{c:hmat-rdiv} (et en remarquant que \( d'_\fct \le
    2\puiss + \epsz \) pour tout \( \fct \)), celle des colonnes des matrices
  de \( \rdiv^\Dii \) et \( \rdiv^\Diii \) par
  \begin{equation}
    \bigl( \delta + o(\delta) \bigr)
    (2\puiss + \epsz)
    \sum\fctrange \wts* \biggl(
    \hautl[1]{\varfc*}
    + \ln(2\vdeg*)
    \biggr)
    \pmm,
  \end{equation}
  et d'après le corollaire~\ref{c:hmat-relim} celle des colonnes de la matrice
  de \( \relim \) est un \( o(\delta) \). En sommant (en comptant bien deux
  fois le majorant commun des hauteurs des deux \( \rdiv \)) et en tenant
  compte du fait que \( \puiss - 1 + \sum\fctirange \wtis* \le (1 + \epsiii)
  \wts[1] \), il vient
  \begin{equation}
    \hautl[\infty] M
    \le
    \delta \Bigr(
    (4\puiss + 2\epsz) \sum\fctrange \wts* \hautl[1]{\varfc*}
    + (1 + \epsiii)\wts[1] \cdot \cst{f-aux}
    \Bigr)
    + o(\delta)
    \pmm,
  \end{equation}
  avec
  \begin{equation}
    \cst{f-aux}
    \le
    2 \hautl[\infty] \coi + \genre \ln(4) + \ln(\dimp + 1)
    + (4\puiss + 2\epsz) \max\fctrange \ln(2\vdeg*)
    \pmm.
  \end{equation}
  On remarque alors que \( 4^\genre \le \dimp + 1 \) pour aboutir à la
  majoration annoncée.
\end{proof}

\begin{scho} \label{s:aux-co}
  On note désormais \( F \) une forme comme dans la conclusion de la
  proposition précédente, \( F' = \wemba(F) \) et \( F'' =
    \rdiv^{\delta d' + r}(\relim(\rdiv^{\delta d'}(F))) \). On a alors
  \begin{enumthm}
    \item \( F'' = RF' \mod \varid \) et \( F'' \in \vanish* \) ;
    \item \( \deg F'' = \deg RF' = \deg F' + o(\delta) \) ;
    \item \( \deg F' = \delta \bigl(
          \wts[1] (\frac94 + \epsz), \dots,
          \wts[\puiss*] (\frac94 + \epsz), 2\puiss - 2 + \epsz
        \bigr) \) ;
    \item \( \nv1{F'} \le \nv1 F \cdot \left(
          \nv\infty\coi^2 \bigl( 4^\genre \dimp** \bigr)^\dv
        \right)^{(1+\epsiii) \wts[1] \delta}
      \) ;
    \item \( \nv1{F''} \le \nv1 F \cdot
        \left(
          \prod\fctirange
          \nv1{\varfc*}^{(4\puiss + 2\epsz) \wtis*}
          \cdot
          \cst{f-aux-loc}^{(1+\epsiii) \wts[1]}
        \right)^\delta
        \cdot
        2^{o(\delta)}
      \), où
      \begin{equation}
        \newcst{f-aux-loc}
        =
        \nv\infty\coi^2 \bigl(
          4^\genre \dimp** \cdot (2 \max_\fct \vdeg*)^{(4\puiss + 2\epsz)}
        \bigr)^\dv
      \end{equation}
  \end{enumthm}
\end{scho}

\begin{proof}
  Les trois premiers points sont donnés directement par la méthode utilisée
  pour construire \( F \). Le suivant résulte du lemme~\ref{l:hmat-wemba} et
  de la propriété~\eqref{e:l1-subslin} de la norme utilisée en effectuant les
  mêmes simplifications que dans la démonstration de la proposition
  précédente.

  Pour le dernier point, on reprend la matrice \( M \) de cette démonstration
  et on en majore les normes locales de la même façon qu'on en a majoré la
  hauteur :
  \begin{align}
    \nv\infty M
    & \le
    \prod\fctirange \left(
      \nv1{\varfc*}^{(4\puiss + 2\epsz) \wtis*}
    \right)^\delta
    \\ & \qquad
    \cdot \left(
      \nv\infty\coi^2 \bigl(
        4^\genre \dimp** \cdot (2 \max_\fct \vdeg*)^{(4\puiss + 2\epsz)}
      \bigr)^\dv
    \right)^{(1+\epsiii) \wts[1] \delta}
    \cdot
    2^{o(\delta)}
  \end{align}
  On en déduit l'estimation annoncée en remarquant que \( F'' \) est l'image de
  \( F \) par l'application de matrice \( M \) et que \( \nv1{M F} \le
    \nv\infty M \cdot \nv1 F \cdot 2^{\dv o(\delta)} \) par les propriétés
  classiques des normes.
\end{proof}



\section{Extrapolation} \label{sec:vojta-extrap}

Le but de cette section est de montrer que la fonction auxiliaire que nous
venons de construire s'annulle avec un indice élevé en \( \ex \), pour une
définition de l'indice que nous préciserons.

Auparavant, nous aurons besoin de représenter les dérivées de certaines
fonctions rationnelles sur \( \var \) par des fractions rationnelles de
dénominateur explicite dont on contrôle le degré et la hauteur du numérateur.
Pour commencer, nous effectuons ce travail sur une variété projective
quelconque (plongée de façon adaptée) avant de passer à une variété produit.


\subsection{Estimation de dérivées} \label{sec:vojta-param}

Soit \( \anyvar \) une variété projective de dimension \( \anydim \), plongée
de façon adaptée dans un espace projectif \( \projd \), de degré \( \anydeg \)
dans ce plongement. Alors \( \cdn(\anyvar) \) est une extension finie de
\begin{equation}
  \cdn\Big(
    \frac{ \vp[0]           }{ \vp[\anydim] }, \dots,
    \frac{ \vp[\anydim-1]   }{ \vp[\anydim] }
  \Big)
\end{equation}
dont \( \frac{ \vp[\anydim+1] }{ \vp[\anydim] } \) est un élément primitif.
Sur ce dernier corps, on dispose des dérivations standard définies par
\(
  \diff_\ind \frac{ \vp[\indi] }{ \vp[\anydim] } = \delta_\ind^\indi
\)
qui forment une base de l'espace des dérivations, et s'étendent de façon
unique à \( \cdn(\anyvar) \) pour former une base de son espace de
dérivations. Les propriétés classiques des dérivations montrent alors que
l'application
\begin{equation}
  \begin{aligned}
    \pmor \colon \cdn(Z)
    & \to \cdn(Z)\series\psp
    \\
    f
    & \mapsto
    \sum_{\derp \in \N^\anydim} \der[\derp] \!f \, \psp^\derp
  \end{aligned}
  \qquad \text{où }
  \der[\derp]
  =
  \frac1{\derp!}
  \prod_{\ind = 0}^\anydim \diff_\ind^{\derp*}
  \pmm.
\end{equation}
est un morphisme de \( \cdn \)-algèbres.

Il est intéressant de pouvoir le représenter par un morphisme \( \pmor* \)
faisant commuter le diagramme
\begin{equation} \label{e:pmor}
  \begin{aligned}[c]
    \xymatrix{
      \cdn[\vp]_{(\ideal\anyvar)}               \ar[d]^\pi \ar@{.>}[r]^{\pmor*}
      & \cdn[\vp]_{(\ideal\anyvar)} \series\psp \ar[d]^\pi
      \\ \cdn(\anyvar)                                     \ar[r]^{\pmor}
      & \cdn(\anyvar)\series\psp
    }
  \end{aligned}
\end{equation}
où les flèches verticales sont les projections canoniques. Posons
\( \pmor* = \sum_\derp \psp^\derp \pdiff*^\derp \) où les \( \pdiff*^\derp \)
sont des applications linéaires sur \( \cdn[X]_{(\ideal\anyvar)} \) vérifiant
la règle de \bsc{Leibniz} pour les dérivées divisées. Pour chaque fraction
rationnelle \( f = F/G \), nous exhiberons (au moins) un polynôme \( R_\derp
\), dépendant éventuellement de la représentation \( F/G \) choisie, tel que
\( R_\derp \cdot \pdiff*^\derp(f) \) soit un polynôme de degré et normes
locales contrôlées.

En pratique, nous construirons en fait deux morphismes, \( \pmor*_0 \) et \(
  \pmor*_1 \), et deux familles d'applications \( \pdiff^\derp \) telles que
l'on contrôlera la norme \( \place \)-adique de \( R_\derp \cdot \pdiff^\derp
\), où l'on rappelle que \( \dv \) vaut \( 1 \) si \( \place \) est
archimédienne et \( 0 \) sinon.

Pour définir \( \pmor*_\dv \), il suffit de définir les images des différents
\( \vp* / \vp[\anydim] \). En effet, ceci donne un morphisme de \( \cdn[ \vp /
  \vp[\anydim] ] \) dans \( \cdn[X]_{(\ideal\anyvar)} \series\psp \). Si l'on
impose de plus que \( \pdiff^0 \) soit l'identité, on constate que l'image du
complémentaire de \( \ideal\anyvar \) ne contient que des séries inversibles
(car leur terme constant l'est), ce qui permet d'étendre le morphisme à \(
  \cdn[X]_{(\ideal\anyvar)} \).

Pour tout \( \ind \in \set{0, \dots, \anydim-1} \), on peut poser \(
  \pmor*_\dv(\vp*/\vp[\anydim]) = \vp*/\vp[\anydim] + \psp* \). Pour \( \ind >
  \anydim \), nous utiliserons le lemme suivant, que l'on énonce dans un cadre
affine avant de l'appliquer à notre situation projective.

\begin{lem} \label{l:param-aff}
  Soient \( \anyvp[1], \dots, \anyvp[\anydim], Y \) des variables et \( L \)
  une algébrique extension finie de \( \cdn(\anyvp[1], \dots, \anyvp[\anydim])
  \). On fixe \( y \) un élément de \( L \) ; on note \( \pi \) le
  morphisme de \( \cdn[\anyvp[1], \dots, \anyvp[\anydim], Y ] \) dans \( L \)
  qui laisse stable les \( \anyvp* \) et envoie \( Y \) sur \( y \). Soit \(
    \Pi \in \ker \pi \) tel que \( \pden = \diff_Y \Pi \notin \ker \pi \).

  On considère les dérivations standard \( \diff_\ind \) sur \( \cdn[
    \anyvp[1], \dots, \anyvp[\anydim] ] \) ainsi que leurs extensions à \( L
  \), et on note \( \der[\derp] = \frac1{\derp!} \prod_{\ind = 0}^\anydim
    \diff_\ind^{\derp*} \) les dérivations réduites.  Il existe des polynômes
  \( P_\dv^\derp \), pour
  \( \derp \in \N^{\anydim} \minusset0 \), tels que :
  \begin{enumthm}
    \item \( \der[\derp] y
        = \pi\left(
          \frac{ P_\dv^\derp }{ \pden^{2\lgr\dermp - 1} }
        \right)
      \) ;
    \item \( \deg P_\dv^\derp \le (\deg \Pi - 1) (2\lgr\derp - 1) \) ;
    \item \( \nv1{ P_\dv^\derp }
        \le \nv1 \poldep ^{2\lgr\derp - 1}
        \cdot \left(
          (4\anydim)^{\lgr\derp -1} (\deg\Pi)^{3\lgr\derp -2}
        \right)^\dv \).
  \end{enumthm}
\end{lem}

\begin{proof}
  Il s'agit en fait de compléter\footnote{On contrôle en fait le développement
    autour d'un point générique, alors que \bsc{Rémond} l'étudie en un point
    fixé.} la preuve du lemme~6.1 de \cite{remivds}, en
  utilisant aux places archimédiennes une généralisation de
  \cite[relation~2.3.1, p.~63]{farhith}.

  On va construire $P_0^\derp$ et $P_1^\derp$ indépendamment par récurrence
  sur la longueur de $\derp$, en partant à chaque fois de $P_\dv^\derp = -
  \diff_{\ind_0} \Pi$ quand $\derp_{\ind_0} = 1$ et $\derp_\ind = 0$ sinon
  (cas $\lgr\derp = 1$), car ce choix convient. Pour la suite, on fixe un
  $\dv$, un $\derp$ de longueur au moins $2$, et on suppose qu'on a choisi un
  $P_\dv^{\derp'}$ convenable pour chaque $\derp'$ de longueur strictement
  inférieure à celle de $\derp$.

  On commence par le cas ultramétrique et on note donc provisoirement $P^\derp
  = P_0^\derp$ pour alléger. Les polynômes recherchés sont caractérisés par la
  relation
  \begin{equation}
    \Pi \left(
      \anyvp[1] + \psp[1], \dots, \anyvp[\anydim] + \psp[\anydim],
      Y + \sum_{ \derp \in \N^\anydim \minusset 0 }
      \frac {P^\derp} {\pden^{2\lgr\derp -1}} \psp^\derp
    \right)
    = 0 \mod (\Pi)
    \pmm.
  \end{equation}
  On remplace alors \( \Pi \) par son développement de \bsc{Taylor}, pour
  obtenir les égalités suivantes modulo \( \Pi \) :
  \begin{align}
    0
    & =
    \sum_{(\ip, \mu) \in \N^{\anydim+1}}
    \der[\ip, \mu] \Pi
    \cdot \psp^\ip
    \cdot \left(
      \sum_{ \derp \in \N^\anydim \minusset 0 }
      \frac {P^\derp} {\pden^{2\lgr\derp - 1}} \psp^\derp
    \right)^\mu
    \\
    & =
    \sum_{\substack{ (\ip, \mu) \in \N^{\anydim+1} \minusset{(0, 0)}
        \\ \gmp\nu* \in \N^\anydim \minusset 0 }}
    \left(
      \der[\ip, \mu] \Pi
      \cdot \prod_{\fct = 1}^\mu
      \frac {P^{\gmp\nu*}} {\pden^{2\lgr{\gmp\nu*} - 1}}
    \right)
    \psp^{\sum_\fct \gmp\nu* + \ip}
    \\
    & =
    \sum_{\derp \in \N^\anydim \minusset 0}
    \Biggl(
    \frac {P^\derp} {\pden^{2\lgr\derp - 2}}
    + \sum_{\substack{
        (\ip, \mu) \in \N^{\anydim+1} \minusset{(0, 0), (0, 1)}
        \\ \gmp\nu* \in \N^\anydim \minusset 0
        \\ \sum_\fct \gmp\nu* + \ip = \derp }}
    \der[\ip, \mu] \Pi
    \cdot \prod_{\fct = 1}^\mu
    \frac {P^{\gmp\nu*}} {\pden^{2\lgr{\gmp\nu*} - 1}}
    \Biggr)
    \psp^\derp
    \pmm,
  \end{align}
  où l'on a noté \( (\ip, \mu) = (\ip[1], \dots, \ip[\anydim], \mu) \).
  Il suffit donc de définir \( P^\derp \) par la relation de récurrence
  \begin{equation}
    - P^\derp
    =
    \sum_{\substack{
        (\ip, \mu) \in \N^{\anydim+1} \minusset{(0, 0), (0, 1)}
        \\ \gmp\nu* \in \N^\anydim \minusset 0
        \\ \sum_\fct \gmp\nu* + \ip = \derp }}
    \der[\ip, \mu] \Pi
    \cdot \pden^{2\lgr\derp - 2}
    \cdot \prod_{\fct = 1}^\mu
    \frac {P^{\gmp\nu*}} {\pden^{2\lgr{\gmp\nu*} - 1}}
  \end{equation}
  qui consiste à imposer que chaque terme de la série précédente soit nul, ce
  qui assure bien sa nullité modulo \( \Pi \).

  On majore alors le degré de \( P^\derp \) par récurrence :
  \begin{align}
    \deg P^\derp
    & \le
    \deg\Pi - \lgr\ip - \mu + (\deg\Pi - 1) (2\lgr\derp - 2)
    \\ & \le
    1 - \lgr\ip - \mu + (\deg\Pi - 1) (2\lgr\derp - 1)
    \\ & \le
    (\deg\Pi - 1) (2\lgr\derp - 1)
    \pmm,
  \end{align}
  car \( \ip \) et \( \mu \) ne sont pas simultanément nuls.  La majoration de
  norme locale est immédiate par analogie avec le degré vu les propriétés de
  la norme aux places ultramétriques.

  Considérons maintenant le cas archimédien (désormais \( P^\derp = P^\derp_1
  \) pour alléger). On utilise la relation de récurrence suivante, établie
  dans la démonstration de \cite[lemme~6.1]{remivds}, avec \( Q_\derp =
    P^\derp \cdot \derp! \) et où, rappelons-le, \( \derp' \) est tel que
  \( \derp[\ind_0] = \derp[\ind_0]' + 1 \) et \( \derp* = \derp*' \) sinon :
  \begin{equation}
    Q_\derp
    =
    \pden^2 \, \diff_{\ind_0} Q_{\derp'}
    - \pden \, \diff_{\ind_0} P \, \diff_{\anydim+1} Q_{\derp'}
    + (2\lgr{\derp'} - 1)
    (\diff_{\ind_0} P \, \diff_Y \pden - \pden \diff_{\ind_0} \pden)
    Q_{\derp'}
    \pmm.
  \end{equation}
  On en déduit immédiatement l'estimation de degré suivante :
  \begin{equation}
    \deg P^\derp
    \le 2 (\deg\Pi - 1) + \deg P^{\derp'}
    \le (\deg\Pi - 1) (2\lgr\derp - 1)
    \pmm.
  \end{equation}
  Pour la norme, on prouve que \(
    \nv1{Q_\derp}
    \le
    \nv1\Pi^{2\lgr\derp-1} 4^{\lgr\derp-1} (\deg\Pi)^{3\lgr\derp-2}
    (\lgr\derp - 1) !
  \)
  (ce qui implique le résultat annoncé vu que \( \binom{\lgr\derp}{\derp}
    \le \anydim^{\lgr\derp-1} \)) en exploitant la majoration de degré
  sous la forme \( \deg P^\derp \le 2\lgr\derp \deg\Pi \).
  \begin{align}
    \nv1{ Q_\derp }
    & \le
    2(\deg\Pi)^2 \cdot \deg Q_{\derp'} \cdot \nv1\Pi^2 \nv1{ Q_{\derp'} }
    + 2 (2\lgr{\derp'} - 1) (\deg\Pi)^3 \nv1\Pi^2
    \\ & \le
    \nv1\Pi^2 \nv1{ Q_{\derp'} } \cdot 4 (\deg\Pi)^{3\lgr{\derp'}}
    \\ & \le
    \nv1\Pi^{2\lgr\derp-1} \cdot 4^{\lgr\derp-1} (\deg\Pi)^{3\lgr\derp-2}
    (\lgr\derp - 1) !
    \qedhere
  \end{align}
\end{proof}

Nous sommes maintenant en mesure de définir les morphismes \( \pmor*_\dv \).

\begin{lem} \label{l:def-pmor}
  Dans les notations précédentes, on peut définir \( \pmor*_\dv \) faisant
  commuter le diagramme~\eqref{e:pmor} par :
  \begin{enumthm}
    \item \( \pmor*(\frac{ \vp* }{ \vp[\anydim] })
        = \frac{ \vp* }{ \vp[\anydim] } + \psp* \)
      pour tout \( \ind \in \set{0, \dots, \anydim-1} \) ;
    \item \( \pmor*(\frac{ \vp* }{ \vp[\anydim] })
        = \frac{ \vp* }{ \vp[\anydim] }
        + \sum_{\derp \neq 0}
        \frac{ P^\derp_{\ind, \dv} }{ U_\ind^{\lgr\derp} }
        \psp^\derp
      \)
  \end{enumthm}
  où les polynômes \( P^\derp_{\ind, \dv} \) et \(  U_\ind \) satisfont
  \begin{enumthm}
    \item \( \deg  U_\ind   =   2 (\anydeg - 1) \) ;
    \item \( \nv1{ U_\ind } \le \nv1{ \chow\anyvar }^2 4^{(\anydeg-1)\dv} \) ;
    \item \( \deg  P^\derp_{\ind, \dv}   =   2 (\anydeg - 1) \lgr\derp \) ;
    \item \( \nv1{ P^\derp_{\ind, \dv} } \le \left(
          \nv1{ \chow\anyvar }^2
          \bigl( 4^\anydeg \anydim \anydeg^{3} \bigr)^\dv
        \right)^{\lgr\derp} \).
  \end{enumthm}
\end{lem}

\begin{proof}
  Fixons un \( \ind > \anydim \) et notons \( \tilde\Pi = P_\ind(\anyvp[1],
    \dots, \anyvp[\anydim], 1, Y ) \), où \( P_\ind \) est donné par le
  fait~\ref{f:plong-adapt-dep}, puis \( \Pi \) une des dérivées successives de
  \( \tilde\Pi \) par rapport à la dernière variable, telle que \( \diff_Y \Pi
    \notin \ideal\anyvar \). On pose alors
  \begin{equation}
    \pdiff^\derp \left( \frac{ \vp* }{ \vp[\anydim] } \right)
    =
    \frac {
      P_\dv^\derp ( \vp[0] / \vp[\anydim], \dots,
      \vp[\anydim-1] / \vp[\anydim], \vp* / \vp[\anydim] )
    }{
      \pden ( \vp[0] / \vp[\anydim], \dots,
      \vp[\anydim-1] / \vp[\anydim], \vp* / \vp[\anydim] )
      ^{2\lgr\derp - 1}
    }
    \pmm,
  \end{equation}
  où les \( P_\dv^\derp \) sont donnés par le lemme précédent. Quitte à
  multiplier par une puissance convenable de \( \vp[\anydim] \), on peut
  supposer que le degré de \( \pden \) est exactement \( 2\anydeg - 1 \) et
  que celui de \( P_\dv^\derp \) est exactement \( (2\anydeg - 1) (2\lgr\derp
    - 1) \), de sorte que par homogénéité :
  \begin{equation}
    \pdiff^\derp \left( \frac{ \vp* }{ \vp[\anydim] } \right)
    =
    \frac {
      P_\dv^\derp ( \vp[0], \dots, \vp[\anydim-1], \vp* )
    }{
      \pden ( \vp[0], \dots, \vp[\anydim-1], \vp* )
      ^{2\lgr\derp - 1}
    }
    =
    \frac {
      P_\dv^\derp
    }{
      \pden
      ^{2\lgr\derp - 1}
    }
  \end{equation}
  On pose finalement\footnote{Ceci n'a pour but que de simplifier les calculs
    ultérieurs en rendant le degré exactement linéaire en \( \lgr\derp \).} \(
    P_{\ind, \dv}^\derp = \pden \cdot P_\dv^\derp \) et \( \pden_\ind =
    \pden^2 \). Les estimations de norme annoncées sont alors immédiates.
\end{proof}

Les morphismes \( \pmor*_\dv \) s'étendent à \( \cdn[X]_{(\ideal\anyvar)} \)
comme expliqué au deuxième paragraphe précédant le lemme~\ref{l:param-aff}.
Nous allons maintenant étudier (dénominateur, degré et hauteur des
coefficients) les images de certaines fonctions, en commençant par celles des
monômes en \( \vp / \vp[\anydim] \).

\begin{lem} \label{l:par-anyvar-mono}
  Soit \( \ip \in \N^{\dimp*} \) et \( M_\ip = \prod\limits\indrange
    \bigl( \frac{\vp*}{\vp[\anydim]} \bigr)^{\ip*} \).  Posons \( \pden =
    \prod_{\ind = \anydim + 1}^{\dimp} \pden_\ind \) où les \( \pden_\ind \)
  sont donnés par le lemme précédent. Alors :
  \begin{enumthm}
    \item \( P^\derp_{\ip, \dv}
        = \pdiff^\derp(M_\ip)
        \cdot \vp[\anydim]^{\lgr\ip} \pden^{\lgr\derp} \) est un polynôme ;
    \item \( \deg P^\derp_{\ip, \dv}
        = \lgr\ip + 2 (\anydeg - 1) (\dimp - \anydim) \lgr\derp \) ;
    \item \( \nv1{ P^\derp_{\ip, \dv} }
        \le
        \nv1{ \chow\anyvar }^{2 (\dimp - \anydim) \lgr\derp}
        \left(
          \bigl( 4^{\anydeg+1} \anydim \anydeg^{3} \bigr) ^{
            (\dimp - \anydim) \lgr\derp }
          \cdot 2^{ \anydim \lgr\ip }
        \right)^\dv
      \).
  \end{enumthm}
  De plus, \( \ord_{\vp[0]} \bigl( P^\derp_{\ip, \dv} \bigr)
    \ge \ip[0]  - \derp[0] \).
\end{lem}

\begin{proof}
  \newcommand \indl {{ \gmp\nu[\ind, \indi_\ind] }}
  On utilise la règle de \bsc{Leibniz} :
  \begin{equation}
    \pdiff^\derp(M_\ip) =
    \sum_{\gmp\nu \in N}
    \prod\indrange
    \prod_{ \indi_\ind = 1 }^{ \ip* }
    \pdiff^\indl \biggl( \frac{ \vp* }{ \vp[\anydim] } \biggr)
    \pmm,
  \end{equation}
  où la somme est prise sur l'ensemble
  \begin{equation}
    N = \left\{
      \gmp\nu \in (\N^\anydim)^{\lgr\ip}
      \text{ tels que }
      \sum\indrange \sum_{\indi_\ind = 1}^{\ip*} \indl = \derp
    \right\}
    \pmm.
  \end{equation}

  On vérifie alors que
  \begin{equation}
    P^\derp_{\ip, \dv}
    =
    \sum_{\gmp\nu \in N}
    \prod\indrange
    \prod_{ \indi_\ind = 1 }^{ \ip* }
    \pdiff^\indl \biggl( \frac{ \vp* }{ \vp[\anydim] } \biggr)
    \pden_\ind^{\lgr\indl}
    \vp[\anydim]
  \end{equation}
  est bien un polynôme : pour chaque facteur de chaque terme, si \( \indl = 0
  \) alors \(
    \pdiff^\indl ( \frac{ \vp* }{ \vp[\anydim] } )
    \vp[\anydim]
  \) est un polynôme, sinon \(
    \pdiff^\indl ( \frac{ \vp* }{ \vp[\anydim] } )
    \pden_\ind^{\lgr\indl}
  \) en est un, d'après le lemme précédent.

  Le calcul du degré est direct et on ne détaille donc que l'estimation de
  norme : chaque terme de la somme est majoré en norme par
  \begin{align}
    \prod_{\ind = \anydim + 1}^{\dimp}
    \prod_{ \indi_\ind = 1 }^{ \ip* }
    \left(
      \nv1{ \chow\anyvar }^2
      \bigl( 4^\anydeg \anydim \anydeg^{3} \bigr)^\dv
    \right)^{\lgr\indl}
    \le
    \nv1{ \chow\anyvar }^{2 (\dimp - \anydim) \lgr\derp}
    \bigl( 4^\anydeg \anydim \anydeg^{3} \bigr)
    ^{(\dimp - \anydim) \lgr\derp \dv}
  \end{align}
  On remarque alors que l'ensemble de sommation \( N \) s'écrit aussi
  \begin{equation}
    \prod_{p = 0}^{\anydim - 1} \left\{
      \gmp\nu[][p] \in \N^{\lgr\ip}
      \text{ tels que }
      \sum\indrange \sum_{\indi_\ind = 1}^{\ip*}
      \gmp\nu[\ind, \indi_\ind][p]
      = \derp[p]
    \right\}
    \pmm.
  \end{equation}
  Chacun des facteurs de ce produit est de cardinal
  \(
    \binom{ \derp[p] + \lgr\ip - 1 }{ \lgr\ip - 1 }
    \le
    2^{ \derp[p] + \lgr\ip - 2 }
  \).
  L'estimation annoncée suit en prenant le produit en majorant assez largement
  certains facteurs.

  Enfin, on constate que chaque terme de la somme définissant
  \( P^\derp_{\ip, \dv} \) contient un facteur de la forme
  \(
    \prod_{\indi_0 = 1}^{\ip[0]}
    \pmnum_{1, \dv}^{\gmp\nu[0, \indi_0]}( \frac{ \vp[0] }{ \vp[\anydim] } )
    \vp[\anydim]
  \). On peut supposer que \( \gmp\nu[0, \indi_0] \in \N \times
    \set{0}^{\anydim-1} \) car sinon ce facteur, donc le terme correspondant,
  est nul. De plus, on a \( \sum_{\indi_0 = 1}^{\ip[0]} \gmp\nu[0, \indi_0][0]
    \le \derp[0] \). Ainsi, il y a au moins \( \ip[0] - \derp[0] \) termes
  nuls dans cette somme, donc \( \vp[0]^{\ip[0] - \derp[0]} \) est en
  facteur de chaque terme non nul de la somme définissant \( P^\derp_{\ip,
      \dv} \), ce qui prouve l'assertion sur l'ordre.
\end{proof}

On souhaite maintenant étudier le développement en série des fonctions
rationnelles sur notre variété produit \( \var \).  On étend les notations de
précédentes au cas multi-homogène de la façon évidente. En particulier, on a
un morphisme \( \pmor \) de développement en série, qu'on représente par des
morphismes \( \pmor*_\dv \) faisant commuter le diagramme
\begin{equation} \label{pmor*-prod}
  \xymatrix{
    \cdn[\vmp]_{(\varid)}                 \ar[d]^\pi  \ar@{.>}[r]^{\pmor*_\dv}
    & \cdn[\vmp]_{(\varid)} \series\psmp  \ar[d]^\pi
    \\ \cdn(\var)                                     \ar[r]^{\pmor}
    & \cdn(\var)\series\psmp
  }
\end{equation}
et qu'on écrira encore sous la forme \( \pmor*_\dv = \sum_\dermp \psmp^\dermp
  \cdot \pdiff^\dermp \).

Plus précisément, pour chaque \( \fct \in \set{1, \dots, \puiss} \), on a un
morphisme \( \pmor*\pexp\fct \) obtenu en appliquant le lemme~\ref{l:def-pmor}
avec \( V = \var* \) et \( \vp = \vmp* \), et on définit \( \pmor \) comme le
produit tensoriel de ces morphismes. On note par ailleurs \( \pden_\fct =
  \prod_{\ind=\vdim*+1}^{\dimp} \gmp\pden** \) où les \( \gmp\pden** \) sont
donnés par le lemme cité.

Nous allons maintenant contrôler l'image par \( \pmor* \) de fonctions
particulières, en commençant par celles qui admettent un monôme en les \(
  \vmp*[\vdim*] \) comme dénominateur.

\begin{lem} \label{l:par-var}
  Soit \( G \) une forme multi-homogène de multidegré \( \alpha = (\alpha_1,
    \dots, \alpha_\puiss ) \) et \( g = G / \vmp[][\vdim]^{\alpha} \) où
  l'on a noté \( \vmp[][\vdim]^{\alpha} = \prod\fctrange
    (\vmp*[\vdim*])^{\alpha_\fct} \). Alors :
  \begin{enumthm}
    \item \( P^\dermp_{G, \dv}
        = \pdiff^\dermp(g)
        \cdot \vmp[][\vdim]^{\alpha}
        \prod\fctrange \pden_\fct^{\lgr{\dermp*}}
      \) est un polynôme ;
    \item \( \deg_\fct P^\dermp_{G, \dv}
        = \alpha_\fct + 2 (\vdeg* - 1) (\dimp - \vdim*) \lgr{\dermp*} \) ;
    \item \( \nv1{ P^\dermp_{G, \dv} }
        \le
        \nv1{ G }
        \prod\fctrange
        \nv1{ \varfc* }^{2 (\dimp - \vdim*) \lgr{\dermp*}}
        \left(
          \bigl( 4^{\vdeg*+1} \vdim* \vdeg*^{3} \bigr) ^{
            (\dimp - \vdim*) \lgr{\dermp*} }
          \cdot 2^{ \vdim* \alpha_\fct }
        \right)^\dv
      \).
  \end{enumthm}
  De plus, \( \inda* \bigl( P^\derp_{G, \dv} \bigr) \ge \inda*(G) -
    \wtsum*(\dermp) \).
\end{lem}

\begin{proof}
  Les trois premiers points s'obtiennent en remarquant que \( g \) est une
  combinaison linéaire de monômes en \( \vmp / \vmp[][\vdim] \), que l'image
  d'un tel monôme par \( \pdiff^\dermp \) est donnée par
  \begin{equation}
    \pdiff^\dermp \Biggl(
      \prod\fctrange \prod\indrange
      \biggl( \frac{ \vmp** }{ \vmp*[\vdim*] } \biggr)^{\imp**}
    \Biggr)
    =
    \prod\fctrange
    \pdiff^{\fct, \dermp*} \Biggl(
      \prod\indrange
      \biggl( \frac{ \vmp** }{ \vmp*[\vdim*] } \biggr)^{\imp**}
    \Biggr)
  \end{equation}
  et en appliquant le lemme~\ref{l:par-anyvar-mono}.

  Seul le point sur l'indice reste à vérifier. Pour cela, considérons \(
    \vmp^\imp \) un monôme apparaissant dans l'écriture \( G \), puis \(
    \vmp^{\gmp\nu} \) un monôme apparaissant dans \( P^\dermp_{G, \dv} \).
  D'après le lemme cité, on alors \( \gmp\nu*[0] \ge \imp*[0] - \dermp*[0] \)
  d'où, en sommant, \( \wtsum*(\gmp\nu) \ge \wtsum*(\imp) - \wtsum*(\dermp) \)
  qui est équivalent à l'estimation annoncée vu la définition de l'indice.
\end{proof}

Notons que les estimations obtenues ne valent que pour des fonctions admettant
un monôme en \( \vmp[][\vdim] \) comme dénominateur.  On pourrait aisément en
déduire des estimations pour des fonctions rationelles de dénominateur
quelconque en les écrivant comme un quotient de deux telles fonctions, mais
ce n'est pas utile ici.


\subsection{Minoration de l'indice}

Nous allons maintenant montrer que la forme auxiliaire construite
précédemment s'annule avec un indice élevé en \( \ex \). Plus précisément,
posons
\begin{equation}
  \wtsum( \dermp )
  =
  \frac {\lgr{\dermp[1]}} {\wts[1]} + \dots
  + \frac {\lgr{\dermp[\puiss*]}} {\wts[\puiss*]}
  + \frac {\lgr{\dermp[\puiss]}} {\puiss*}
\end{equation}
pour tout \( \dermp \in \N^{\lgr\vdim} \).  Si \( h \) est une forme
multihomogène et \( \point \) où elle est définie, on définit son indice en \(
  \point \), noté \( \inda[\point](h) \),  comme le minimum des \(
  \wtsum*(\dermp) \) pour \( \dermp \) tel que \( \der[\dermp] h(\point) \neq 0
\).

On reprend les notations introduites par le scolie~\ref{s:aux-co} et on pose
\( f_1 = F' / \vmp[][\vdim]^\Di \). Le but de cette section est alors de montrer
la proposition suivante.

\begin{prop}
  On a \( \inda**(f_2) \ge \epsi \delta / \sigma \), où \( \sigma \) est tel
  que \dots
\end{prop}

Commençons par un lemme facile sur l'indice.

\begin{lem} \label{l:indice-inversible}
  Soient \( g_1 \), \( g_2 \) et \( \alpha \) des fonctions rationnelles
  telles que \( g_1 = \alpha g_2 \) et \( \point \) un point où elles sont
  toutes les trois définies.
  \begin{enumthm}
    \item Si \( \dermp \) est tel que \( \der[\gmp\nu] g_2(x) = 0 \) dès que
      \( \gmp\nu < \dermp \) pour l'ordre produit sur \( \N^{\lgr\vdim} \),
      alors \( \der[\dermp] g_1(\point) = \alpha(\point) \, \der[\dermp]
        g_2(\point) \).
    \item Si \( \alpha(\point) \neq 0 \) on a \( \inda**(g_1) = \inda**(g_2)
      \).
  \end{enumthm}
\end{lem}

\begin{proof}
  Le premier point découle facilement de la formule de \bsc{Leibniz} :
  \begin{equation}
    \der[\dermp] g_1(\point)
    =
    \sum_{\gmp\nu \le \dermp}
    \der[\dermp - \gmp\nu] \alpha(\point) \,
    \der[\gmp\nu] g_2(\point)
    \pmm.
  \end{equation}
  Or, par hypothèse, tous les termes de cette somme sont nuls sauf peut-être
  celui où \( \gmp\nu = \dermp \).

  Si \( \alpha(\point) \neq 0 \) alors \( \alpha^{-1} \) est également définie
  en \( \point \) et les deux autres fonctions jouent donc un rôle symétrique.
  Ainsi, si un indice \( \dermp \) est minimal pour la condition \(
    \der[\dermp] g_1(\point) \neq 0 \), il l'est aussi pour la condition \(
    \der[\dermp] g_2(\point) \neq 0 \) grâce au point précédent, ce qui prouve
  que les deux fonctions ont le même indice en \( \point \).
\end{proof}

On rappelle que \( \cex \) désigne un système de coordonnées multihomogènes du
point \( \ex \) supposé contredire le théorème~\ref{t:vojta-div}. On peut
supposer, d'après le scolie~\ref{s:part-cases}, qu'aucune des formes \(
  \pden_\fct \) s'annulle en \( \cex* \) et qu'aucun des \( \cex** \) n'est
nul ; ci-dessous on utilisera le fait que \( \cex*[\vdim*] \neq 0 \).  De
même, on peut supposer que la forme \( R \) donnée par le
corollaire~\ref{c:hmat-relim} ne s'annulle pas en \( \cex \).  \todo[Vérifier
  que c'est bien le cas.]

On choisit, pour tout \( \fct \), un indice \( \indv* \in \set{0, \dots,
    \vdim*} \) de sorte que \( \av{ \cex*[\indv*] } \) soit maximal sous cette
condition. De même, on choisit \( \indiv* \in \set{0, \dots, \dimp} \)
maximisant \( \av{ \wembcl*[\indiv*](\cex) } \), ainsi que \( \indig* \in
  \set{0, \dots, \dimp} \) tel que \( \wembcl*[\indig*](\cex) \neq 0 \). Notons
que \( \indig \) et \( \indiv \) ne dépendent pas du choix des formules \(
  \wembcl \) utilisées, et que \( \indig \) ne dépend pas de \( \place \).
\todo[Introduire la notation \( \wembcl \) et d'ailleurs choisir une autre
  lettre.]

Il est alors clair que \( \av{ \wembcl*[\indiv*](\cex) } \ge \nv1{
    \wembcl*(\cex) } \dimp**^{-\dv} \).  Le lemme suivant fournit un analogue
pour les \( \indv* \).

\begin{lem} \label{l:coord-norm}
  Avec les notations précédentes, on a
  \begin{equation}
    \av{ \cex*[\indv*] }
    \ge
    \nv1{ \cex* } \cdot \nv1{ \varfc* }^{-1} \dimp**^{-\dv}
    \pmm.
  \end{equation}
\end{lem}

\begin{proof}
  Il suffit de montrer qu'on a \( \av{ \cex** } \le \nv1{ \varfc* } \av{
      \cex*[\indv] } \) pour tout \( \ind \in \set{\vdim* + 1, \dots, \dimp}
  \). Notons \( \poldep** \) la relation dépendance intégrale donnée par le
  scolie~\ref{s:plong-adapt} ; en décomposant suivant les puissance de \(
    \vmp** \) on a
  \begin{equation}
    (\vmp**) ^{ \vdeg* }
    =
    \sum_{ \alpha=1 }^{ \vdeg* }
    (\vmp**) ^{ \vdeg* - \alpha }
    \poldep*[\ind, \alpha]
    \quad\text{où }
    \deg \poldep*[\ind, \alpha] = \alpha
    \text{ et }
    \sum_{ \alpha=1 }^{ \vdeg* } \nv1{ \poldep*[\ind, \alpha] }
    \le \nv1{ \varfc* }
  \end{equation}
  En passant aux valeurs absolues et en divisant, il vient
  \begin{equation}
    \av{ \cex** }
    \le
    \sum_{ \alpha=1 }^{ \vdeg* }
    \poldep*[\ind, \alpha]
    \left(
      \frac{ \av{\cex*[\indv]} }{ \av{\cex**} }
    \right) ^{ \vdeg* - \alpha }
  \end{equation}
  Si le quotient apparaissant dans le membre de droite est inférieur à \( 1
  \), notre assertion initiale est vérifiée. Sinon, elle l'est aussi, car
  \( \varfc* \) est normalisé (scolie cité) de façon à ce qu'un de ses
  coefficients soit \( 1 \) ce qui assure \( \nv1{ \varfc* } \ge 1 \) en toute
  place.
\end{proof}

Introduisons maintenant une nouvelle fonction rationnelle définie par
\begin{equation}
  f_2 =
  \frac{ R(\vmp) }{ \vmp[][\vdim]^r }
  \cdot
  \frac{
    F( \vmp, \wembcl(\vmp) )
  }{
    \vmp[][\vdim] ^{\epsz \wts \delta}
    \cdot
    \wembcl[][\indig](\vmp) ^{ \delta }
  }
  \pmm,
\end{equation}
où au dénominateur on a noté
\begin{equation}
  \wembcl[][\indig](\vmp) ^{ \delta }
  =
  \prod\fctirange \wembcl*[\indig](\vmp) ^{ \delta }
  \pmm.
\end{equation}
D'après le lemme~\ref{l:indice-inversible}, \( f_1 \) et \( f_2 \) ont le même
indice en \( \cex \). Nous allons maintenant décomposer \( f_2 \) en facteurs
qui seront plus faciles à contrôler.

On commence par choisir pour chaque \( \place \) des formes \( \wembcl[\place,
    \fcti] \) en appliquant le lemme~\ref{l:hclab} avec \( (\pp, \ppi) =
  (\ex[\fcti], \ex[\puiss]) \) et \( (\alpha, \beta) = (\wti*, \wt**)
\). On écrit alors
\begin{align}
  f_2
  & =
  \frac{ R(\vmp) }{ \vmp[][\vdim]^r }
  \cdot
  \frac{
    F( \vmp, \wembcl[\place](\vmp) )
  }{
    \vmp[][\vdim] ^{\epsz \wts \delta}
    \cdot
    \wembcl[\place][\indig](\vmp) ^{ \delta }
  }
  \\ & =
  \frac{ R(\vmp) }{ \vmp[][\indv]^r }
  \cdot
  \frac{
    F( \vmp, \wembcl[\place](\vmp) )
  }{
    \vmp[][\indv] ^{\epsz \wts \delta}
    \cdot
    \wembcl[\place][\indiv](\vmp) ^{ \delta }
  }
  \left(
    \frac{ \wembcl[\place][\indiv](\vmp) }{ \wembcl[\place][\indig](\vmp) }
  \right) ^{ \delta }
  \left(
    \frac{ \vmp[][\indv] }{ \vmp[][\vdim] }
  \right) ^{ \epsz \wts \delta + r }
  \\ & =
  \frac{
    R(\vmp) \cdot F( \vmp, \wembcl[\place](\vmp) )
  }{
    \vmp[][\indv] ^{ \Diii }
  }
  \left(
    \prod\fctirange
    \frac{
      ( \vmp[\fcti ][{\indv[\fcti ]}] )^{ 2\wts[\fcti ] }
      ( \vmp[\puiss][{\indv[\puiss]}] )^{ 2\wts[\puiss] }
    }{
      \wembcl[\place, \fcti][\indiv*]( \vmp )
    }
  \right)^\delta
  \left(
    \frac{ \wembcl[][\indiv](\vmp) }{ \wembcl[][\indig](\vmp) }
  \right) ^{ \delta }
  \left(
    \frac{ \vmp[][\indv] }{ \vmp[][\vdim] }
  \right) ^{ \epsz \wts \delta + r }
\end{align}

Soit maintenant \( \dermp \in \N^{\lgr\vdim} \) un indice minimal tel que \(
  \der[\dermp] f_2 \neq 0 \). Comme les derniers facteurs de l'écriture
précédente sont tous inversibles en \( \cex \), le
lemme~\ref{l:indice-inversible} montre que \( \der[\dermp] f_2 \) est égal à

\begin{equation}
  \der[\dermp]
    \frac{
      R(\vmp) \cdot F( \vmp, \wembcl[\place](\vmp) )
    }{
      \vmp[][\indv] ^{ \Diii }
    }
  ( \cex )
  \cdot
  \left(
    \prod\fctirange
    \frac{
      ( \cex[\fcti ][{\indv[\fcti ]}] )^{ 2\wts[\fcti ] }
      ( \cex[\puiss][{\indv[\puiss]}] )^{ 2\wts[\puiss] }
    }{
      \wembcl[\place, \fcti][\indiv*]( \cex )
    }
  \right)^\delta
  \left(
    \frac{ \wembcl[][\indiv](\cex) }{ \wembcl[][\indig](\cex) }
  \right) ^{ \delta }
  \left(
    \frac{ \cex[][\indv] }{ \cex[][\vdim] }
  \right) ^{ \epsz \wts \delta + r }
\end{equation}
Nous allons maintenant estimer la valeur absolue de chacun des facteurs, en
distinguant selon que \( \place \in \placesapx \) ou pas pour le premier
facteur, et montrer que l'on contredit la formule du produit si \( \wtsum(
  \dermp ) < \epsi \delta / \sigma \). Rappelons qu'on note \( \degv \) le
degré local de \( \cdn \) en \( \place \).

Pour le facteur le plus à droite, on a facilement
\begin{equation}
  \prod_\place
  \prod\fctrange
  \left(
    \frac{ \av{\cex[][\indv]} }{ \av{\cex[][\vdim]} }
  \right) ^{ (\epsz \delta \wts* + r_\fct) \degv }
  \le
  \prod\fctrange
  \hautm[\infty]{ \ex* }^{\epsz \delta \wts* + r_\fct}
  \le
  \hautm[1]{ \ex[1] }^{2\puiss \epsz \delta \wts[1]} \cdot o(\expb^\delta)
\end{equation}
De plus, en notant \( \exi \) l'image de \( \ex \) par la deuxième
partie du plongement éclatant, on a\worknote{\bsc{Farhi} p. 96 en haut}
\begin{equation}
  \prod_\place
  \prod\fctirange
  \left(
    \frac{ \av{\wembcl[][\indiv](\cex)} }{ \av{\wembcl[][\indig](\cex)} }
  \right) ^{ \delta \degv }
  \le
  \prod\fctirange
  \hautm[1]{ \exi* }^\delta
  \le
  \hautm[1]{ \ex[1] }^{2\puiss \epsiv \wts[1] \delta}
  \cdot \cst{exci}^{\wts[1] \delta}
\end{equation}
Enfin, grâce au choix des formules \( \wembcl[\place] \), on a
\begin{equation}
  \prod_\place
  \left(
    \prod\fctirange
    \frac{
      \nv1{ \wembcl[\place, \fcti][\indiv*] }
      ( \cex[\fcti ][{\indv[\fcti ]}] )^{ 2\wts[\fcti ] }
      ( \cex[\puiss][{\indv[\puiss]}] )^{ 2\wts[\puiss] }
    }{
      \wembcl[\place, \fcti][\indiv*]( \cex )
    }
  \right)^{ \delta \degv }
  \le
  \hclab^{2\wts[1] \delta}
\end{equation}

Passons maintenant au premier facteur. Remarquons pour commencer que
\begin{equation}
  \nv1{ R(\vmp) \cdot F( \vmp, \wembcl[\place](\vmp) ) }
  \le
  \nv1{ F } \nv1{ \wembcl }^\delta \cdot o(\expb^\delta)
\end{equation}
et que le facteur en \( \nv1{ \wembcl } \) a déjà été intégré à l'estimation
précédente. Appliquons maintenant le lemme~\ref{l:par-var} à cette forme :
\worknote{À adapter pour remplacer \( \vdim \) par une autre coordonnée ?}
\begin{equation}
  \der[\dermp]
    \frac{
      R(\vmp) \cdot F( \vmp, \wembcl[\place](\vmp) )
    }{
      \vmp[][\indv] ^{ \Diii }
    }
  ( \cex )
  =
  \frac{ P^\dermp_\dv(\cex) }{ \cex[][\indv]^\Diii \pden(\cex)^{\vlg\dermp} }
  =
  \frac{ P^\dermp_\dv(\cex) }{ \cex[][\indv]^{\deg P^\dermp_\dv } }
  \cdot
  \frac{
    \cex[][\indv]^{\vlg\dermp \cdot \deg \pden}
  }{
    \pden(\cex)^{\vlg\dermp}
  }
  \pmm.
\end{equation}
On estime facilement le second facteur :
\begin{equation}
  \prod_\place
  \prod\fctrange
  \frac{
    \av{ \cex*[\indv] }^{ 2 (\vdeg*-1) (\dimp-\vdim*) \lgr{\dermp*} \degv }
  }{
    \av{ \pden_\fct(\cex*) }^{ \lgr{\dermp*} \degv }
  }
  =
  \prod\fctrange
  \hautm[\infty]{ \cex* }^{ 2 (\vdeg*-1) (\dimp-\vdim*) \lgr{\dermp*} }
  \le
  \hautm[1]{ \cex[1] }^{ \newcst[]{degdim} \wts[1] \epsi\delta/\sigma }
\end{equation}
Le premier facteur est majoré en valeur absolue par \( \nv1{ P^\dermp_\dv }
\), qui est à peu près \( \nv1{ F } \cdot \nv1{ \varfc }^{\wts[1] \epsi\delta
    / \sigma } \). Cependant, aux places de \( \placesapx \), on peut obtenir
une estimation plus fine.

En effet, chaque monôme divisé intervenant dans le déshomogénéisé de \(
  P^\dermp_\dv \) est composé d'un facteur des valeurs absolue inférieure à \(
  1 \) et d'un facteur de la forme
\begin{equation}
  \prod\fctrange
  \left(
    \frac{ \av{ \cex*[0] } }{ \av{ \cex*[\indv*] } }
  \right)^{\imp*[0]}
  \pmm,
\end{equation}
où \( \imp \) est tel que \( \wtsum*(\imp) \ge \epsi (1 - \frac1\sigma)
  \delta \). Or, par le lemme~\ref{l:coord-norm} et l'hypothèse principale, on
a \todo[Ne pas prendre le produit sur \( \place \) si tôt.]
\begin{equation}
  \prod\placerange
  \left(
    \frac{ \av{ \cex*[0] } }{ \av{ \cex*[\indv*] } }
  \right)^{\degv}
  \le
  \left(
    \frac{ \av{ \cex*[0] } }{ \nv1{ \cex* } }
    \nv1{ \varfc* } \dimp**^\dv
  \right)^{\degv}
  \le
  \hautm[2]{ \ex* }^{-\eps}
  \hautm[1]{ \varfc* } \dimp**^\dv
\end{equation}
Ainsi, \todo en négligeant pour l'instant les partie archimédiennes des
constantes, on a
\begin{align}
  \prod\placerange
  \prod\fctrange
  \left(
    \frac{ \av{ \cex*[0] } }{ \av{ \cex*[\indv*] } }
  \right)^{\imp*[0] \degv}
  & \le
  \prod\fctrange
  \left(
    \hautm[2]{ \ex* }^{-\eps}
    \hautm[1]{ \varfc* } \dimp**^\dv
  \right)^{ \imp*[0] }
  \\ & \le
  \hautm[1]{ \ex[1] }^{ -\eps \epsi(1 - \frac1\sigma) \delta \wts[1] }
  \hautm[1]{ \varfc }^{2 \wts[1] \delta }
\end{align}

En appliquant la formule du produit d'un côté et en mettant bout à bout les
estimations ci-dessus de l'autre, il vient, après avoir pris les logarithmes :
\begin{align}
  0
  & \le
  \delta \Bigl(
    2\puiss \epsz \wts[1] \hautl{ \ex[1] }
    +
    2\puiss \epsiv \wts[1] \hautl{ \ex[1] } + \wts[1] \cst{exci}
    +
    2\wts[1] \log\hclab
    +
    \cst{degdim} \wts[1] \epsi \sigma^{-1} \hautl{ \ex[1] }
    \\ & \qquad +
    \wts[1] \epsi \sigma^{-1} (\hautl{ F } + \hautl{ \varfc })
    +
    2 \wts[1] \hautl{ \varfc }
    -
    \eps \epsi(1 - \sigma^{-1}) \wts[1] \hautl{ \ex[1] }
  \Bigr)
  + o(\delta)
  \\ & \le
  \delta \Bigl(
    \wts[1] \hautl{ \ex[1] }
    \bigl(
      2\puiss \epsz + 2\puiss \epsiv + \cst{degdim} \epsi \sigma^{-1}
      -
      \eps \epsi(1 - \sigma^{-1})
    \bigr)
    + o( \wts[1] \hautl{ \ex[1] } )
  \Bigr)
  + o(\delta)
\end{align}
ce qui devient absurde dès que \( \delta \) et \( \wts[1] \hautl{ \ex[1] }
\) sont assez grands, pour peu que
\begin{equation}
  \frac{
    2\puiss \epsz + 2\puiss \epsiv + \cst{degdim} \epsi \sigma^{-1}
  }{
    \epsi(1 - \sigma^{-1})
  }
  \le
  \eps
\end{equation}

\subsection{Appendice provisoire}

\begin{fact} \todo
  On a
  \begin{equation}
  \prod\fctirange
  \hautm[1]{ \exi* }^\delta
  \le
  \hautm[1]{ \ex[1] }^{2\puiss \epsiv \delta} \cdot \cst{exci}^{\delta}
  \end{equation}
  avec \( \newcst[]{exci} = \dots \)
\end{fact}



\section{Application du théorème du produit et conclusion}

\section{Contraintes sur les paramètres}

\begin{enumerate}
  \item Page~\pageref{ct:Vcos<1/16} : \( \Vcos \le 1/16 \) pour
    avoir~\eqref{e:hautn-wt-diff}.
  \item Page~\pageref{ct:Vfar>2} : \( \Vfar \ge 2 \) pour la preuve
    du lemme sur \( \epsiii \).
  \item Page~\pageref{e:def-epsiii} : \( \epsiii \) défini
    par~\eqref{e:def-epsiii}.
\end{enumerate}

\endinput

% vim: spell spelllang=fr
