% !TEX root = main.tex

\chapter{Inégalité de \bsc{Vojta}} \label{chap:vojta}

Nous établissons ici une inégalité de \bsc{Vojta} effective dans le cas
particulier où la variété à approcher est un hyperplan (ne contenant pas \(
  \va \)) ; nous verrons plus tard que ce cas implique le cas général.
Le but du chapitre est donc de prouver le théorème suivant, où les notions de
distances et de hauteurs découlent d'un plongement \( \Theta \colon \va
  \embedin \projd \) fixé, et \( \divi \) est l'hyperplan défini par \( \vp[0]
  = 0 \) dans ce plongement.
\nomuse \divi [chap] {L'hyperplan à approcher,  défini par \( \vp[0] = 0 \)}

\begin{thm} \label{t:vojta-div}
  Pour tout réel \( 0 < \eps < 1 \), dans \( \va(\Qbar) \), il n'existe pas de
  famille de points \( \ex[1], \dots, \ex[\puiss] \) satisfaisant
  simultanément aux conditions suivantes :
  \begin{align}
    0 < \distv{\ex*}{\divi}
    & < \hautm[2]{\ex*}^{-\wtapx \expapx}
    \quad \forall \place \in \placesapx
    \label{e:Vapx}
    \\
    \hautn{\ex[1]} & > \Vbig
    \label{e:Vbig}
    \\
    \hautn{\ex*} & > \Vfar \hautn{\ex[\fct-1]}
    \label{e:Vfar}
    \\
    \cos(\ex*, \ex[\fcti]) & > 1 - \Vcos
    \label{e:Vcos}
  \end{align}
  avec
  \nomuse \Vbig {Dans le théorème~\ref{t:vojta-div}, minore la hauteur du plus
  petit point}
  \nomuse \Vfar {Dans le théorème~\ref{t:vojta-div}, minore l'écart
    multiplicatif entre les points}
  \nomuse \Vcos {Dans le théorème~\ref{t:vojta-div}, contrôle l'angle entre
    les points}
  \nomuse \puiss [chap] {cardinal de la famille d'approximations \( (\ex*) \)
    considérée}
  \begin{align}
    \label{e:def-Vbig}
    \Vbig & \ge \stuffout{\cst{vs-ht} \Lambda^{2\,f(\puiss-1)}}
    \text{ pour la conclusion de la construction}
    \\ & \ge \stuffout{
      \frac{16}{3\expapx \epsi}
      \Lambda^{3f(\puiss)} \bigl(
        ( 3 \cst{vs-deg-prod}^{-1} + 2\dimp ) \cst{vs-ht}
        + 4\dimp^2 \cst{vs-deg-prod}^{-2}
      \bigr)
    } \text{ pour \eqref{e:Vbig-ct-extra}}
    \\
    \label{e:def-Vfar}
    \Vfar
    & = 2 \puiss \Lambda^{2\puiss f(\puiss)}
    \text{ pour la section~\ref{sec:thm-prod}}
    \\
    \label{e:def-Vcos}
    \Vcos & = \stuffout{\frac{\eps\epsi}{48(\puiss-1)}}
    \text{ pour \eqref{e:Vcos-ct-extra}}
    \\
    \puiss & \ge \stuffout{g+1}
  \end{align}
\end{thm}


\section{Réduction et autres préliminaires}
\label{sec:vojta-reduc}

Notons pour commencer qu'on peut supposer \( \eps < 1 \), car sinon
l'inégalité de \bsc{Liouville} (qui dans ce cas déroule directement de la
formule du produit vu la définition de la distance à un hyperplan) montre
qu'il n'existe aucun point satisfaisant~\eqref{e:Vapx}, sans même utiliser les
autres conditions.


\subsection{Poids associés à une famille d'approximations}
\label{sec:wt}

\nomuse[\ex]{(\ex*)}[chap]{Famille d'approximations de \( \divi \) dont
  l'existence contredirait le théorème~\ref{t:vojta-div}}
La démonstration procède par l'absurde : si le théorème est faux, fixons une
famille \( \ex[1], \dots, \ex[\puiss] \) qui le contredit. Bien que cette
famille soit toujours supposée satisfaire à toutes les conditions du théorème,
nous préciserons dans les hypothèses de la plupart des énoncés suivants la ou
lesquelles de ces conditions nous utilisons, par souci de clarté.

Nous utiliserons des combinaisons linéaires des \( \ex* \) de petite hauteur.
Les lemmes suivants permettent de choisir les coefficients pour ces
combinaisons ; nous les prendrons entiers et n'ayant que \( 2 \) et \( 3 \)
pour diviseurs premiers, de sorte à disposer de représentations polynomiales
convenables des formes linéaires abéliennes associées données par
l'annexe~\ref{sec:form-ab-alt}.

\begin{lem} \label{l:wt-choose-gen}
  Soit \( \zeta \) un réel positif. Il existe des entiers \( \wt* \in 2^\N
    3^\N \) tels que, pour tout \( \fct \in \set{1, \dots, \puiss} \) :
  \begin{equation} \label{e:wt-ratio-gen}
    \frac1{1 + \zeta}
    \le
    \frac{ \wts* \hautn{\ex*} }{ \wts[1] \hautn{\ex[1]} }
    \le
    1 + \zeta
    \pmm.
  \end{equation}
\end{lem}

\begin{proof}
  On commence par choisir des rationnels \( b_\fct = 2^{b_{\fct2}}
    3^{b_{\fct3}} \) tels que
  \begin{equation}
    \frac1{1 + \zeta}
    \le
    b_\fct^2 \frac{ \hautn{\ex*} }{ \hautn{\ex[1]} }
    \le
    1 + \zeta
    \pmm,
  \end{equation}
  soit en prenant les logarithmes et en divisant par deux :
  \begin{equation}
    \abs*{
      b_{\fct2} \log 2 + b_{\fct3} \log 3
      - \frac12 \log \frac{ \hautn{\ex*} }{ \hautn{\ex[1]} }
    }
    \le
    \frac12 \log (1 + \zeta)
    \pmm.
  \end{equation}
  Comme \( \Z \log2 + \Z \log 3 \) est dense dans \( \R \), il est
  certainement possible de choisir indépendamment pour chaque \( \fct \) deux
  entiers \( b_{\fct2} \) et \( b_{\fct3} \) tels que cette dernière condition
  soit satisfaite.

  Il ne reste plus qu'à définir \( \wt[1] \) comme une dénominateur commun des
  \( b_\fct \) puis à poser \( \wt* = \wt[1] b_\fct \) pour tout \( \fct > 1
  \). Notons que ceci nous permet de choisir les \( \wt* \) arbitrairement
  grands, car seuls leurs rapports compte.
\end{proof}

En utilisant les inégalités élémentaires \( (1 + \zeta)^{-1} \ge 1 - \zeta \)
et \( (1 + \zeta)^{1/2} \le 1 + \zeta/2 \), on en déduit facilement que pour
toute famille \( (\wt*) \) satisfaisant à~\eqref{e:wt-ratio}, on a
\begin{equation} \label{e:wt-diff-normn-gen}
  \abs[\big]{ \wt* \normn{\ex*} - \wt[1] \normn{\ex[1]} }
  \le
  \frac\zeta2 \wt[1] \normn{\ex[1]}
  \quad\text{et}\quad
  \abs[\big]{ \wt* \normn{\ex*} - \wt[\fcti] \normn{\ex[\fcti]} }
  \le
  \zeta \wt[1] \normn{\ex[1]}
\end{equation}
pour tous \( \fct \) et \( \fcti \), où \( \hautn\truc = \normn\truc^2 \)
désigne la norme de \NT. Cette inégalité nous sera utile pour démontrer le
lemme suivant.

\begin{lem}
  Soit \( \zeta > 0 \) et \( (\wt*) \) une famille d'entiers satisfaisant
  à~\eqref{e:wt-ratio}. Si de plus la famille \( (\ex*) \) satisfait
  à~\eqref{e:Vcos}, alors pour tout \( \fct \in \set{1, \dots, \puiss} \) on a
  \begin{equation}
    \hautn{\wt* \ex* - \wt** \ex**}
    \le
    \wts[1] \hautn{\ex[1]} \left(
      \zeta^2 + 2 \Vcos (1 + \zeta)
    \right)
    \pmm.
  \end{equation}
\end{lem}

\begin{proof}
  En développant le membre de gauche, il vient successivement
  \begin{alignat}{2}
    \normn{\wt* \ex* - \wt** \ex**}^2
    & =
    \normn{\wt* \ex*}^2 + \normn{\wt** \ex**}^2
    - 2 \scalarn{\wt* \ex*}{\wt** \ex**}
    \\
    & = \wts* \normn{\ex*}^2 + \wts** \normn{\ex**}^2
    - 2 \wt* \wt** \scalarn{\ex*}{\ex**}
    \\
    & = \left( \wt* \normn{\ex*} - \wt** \normn{\ex**} \right)^2
    + 2 \wt* \wt** \left(
      \normn{\ex*} \normn{\ex**} - \scalarn{\ex*}{\ex**}
    \right)
    \\
    & \le \left( \wt* \normn{\ex*} - \wt** \normn{\ex**} \right)^2
    + 2 \wt* \wt** \normn{\ex*} \normn{\ex**} \Vcos
    && \text{d'après~\eqref{e:Vcos}}
    \\
    & \le \left( \zeta \wt[1] \normn{\ex[1]} \right)^2
    && \text{d'après~\eqref{e:wt-diff-normn-gen}}
    \\
    & \qquad + 2 \left( \wt[1] \normn{\ex[1]} \sqrt{1 + \zeta} \right)^2 \Vcos
    && \text{d'après~\eqref{e:wt-ratio}}
  \end{alignat}
  qui donne le résultat annoncé en factorisant \( (\wt[1] \normn{\ex[1]})^2 =
      \wts[1] \hautn{\ex[1]} \).
\end{proof}

\begin{tdef} \label{d:wt-choose}
  \nomuse[\wt]{(\wt*)}[chap]{Poids associés à \( (\ex*) \) et fixés par le
    scolie~\ref{d:wt-choose}}
  \nomdirect{g}{zeta}{\zeta}[chap]{Constante proche de zéro fixée par le
    scolie~\ref{d:wt-choose}}
  On choisit désormais et jusqu'à la fin du chapitre une famille \( (\wt*) \)
  donnée par l'application du lemme~\ref{l:wt-choose-gen} avec \( \zeta =
    \sqrt{\Vcos} / 2 \).
\end{tdef}

En remarquant que \( \Vcos < 1 \), le lemme précédent et la
relation~\eqref{e:wt-ratio-gen} donnent immédiatement les relations :
\begin{gather} \label{e:hautn-wt-diff}
  \hautn{\wt* \ex* - \wt** \ex**}
  \le
  3 \Vcos \wts[1] \hautn{\ex[1]}
  \\ \label{e:wt-ratio}
  \frac12
  \le
  \frac{ \wts* \hautn{\ex*} }{ \wts[1] \hautn{\ex[1]} }
  \le
  2
  \pmm.
\end{gather}
Les estimations de la deuxième ligne sont un peu larges mais suffiront
amplement en pratique.

Jusqu'à présent nous n'avons exploité que l'hypothèse~\eqref{e:Vcos},
pour~\eqref{e:hautn-wt-diff}. En tenant compte de l'hypothèse~\eqref{e:Vfar}, on
voit de plus que la suite \( \wts[1], \dots, \wts** \) décroît
au moins comme une suite géométrique de raison inférieure à $1$. Plus
précisément, dès que \( \ex \) satisfait~\eqref{e:Vfar}, on a
\begin{equation} \label{e:wt-geom}
  \wts*
  \le
  (1+\zeta) \cdot \frac{\wts[1]}{\Vfar^{\fct-1}}
\end{equation}
en appliquant directement la définition de \( \wt* \). On peut ainsi majorer
des sommes faisant intervenir les \( \wts* \) en fonction de \( \wts[1] \) ;
par exemple, l'énoncé suivant nous sera utile par la suite.

\begin{lem} \label{l:sum-wts}
  On a \(
    \sum\fctrange (\wts* + \wts**)
    = \sum\fctrange \wtw* \wts*
    \le 2 \wts[1]
  \), où l'on a noté \( \wtw* = 1 \) si \( \fct < \puiss \) et \( \wtw** =
    \puiss - 1 \).
\end{lem}

\begin{proof}
  En effet, on écrit
  \begin{align}
    \frac1{\wts[1]} \sum_{\fct=1}^\puiss (\wts* + \wts**)
    & = 1
    + \frac{\puiss \wts**}{\wts[1]}
    + \sum_{\fct=2}^\puiss \frac{\wts*}{\wts[1]}
    \\
    & \le 1
    + \frac{\puiss(1+\zeta)}{\Vfar^{\puiss-1}}
    + (1+\zeta) \sum_{\fct=2}^\puiss \Vfar^{-\fct+1}
    \\
    & \le 1 + (1+\zeta) \left(
      \Bigl( \frac2\Vfar \Bigr)^{\puiss-1}
      + \frac1{\Vfar-1}
    \right)
    \pmm.
  \end{align}
  Pour conclure, il suffit d'observer que \( \Vfar > 4 \) et que \( \Vcos <
    1/9 \).
\end{proof}


\subsection{Réduction à l'existence d'une forme obstructrice}
\label{sec:vojta-prop}

Nous regardons \( \ex = (\ex*) \) comme un point de \( \va^\puiss \) plongée
dans \( (\projd)^\puiss \), et introduisons l'ensemble \( \varset(\ex) \) des
sous-variétés produit \( \var = \var[1] \times \dots \times \var** \) de \(
  \va^\puiss \) qui contiennent \( \ex \) et satisfont aux majorations
suivantes, où les notions de degré et de hauteurs s'entendent dans le
plongement \( \Theta^\puiss \) et où l'on rappelle la notation \( \wtw* = 1 \)
si \( \fct < \puiss \) et \( \wtw** = \puiss-1 \) :
\nomuse {\varset(\ex)} [chap] {Ensemble des variétés produit contenant \( \ex
  \) et satisfaisant aux conditions~\eqref{e:varset-deg}
  à~\eqref{e:varset-ht}}
\begin{align}
  \label{e:varset-deg}
  \vdeg*
  & \le B
  = \cst{vs-deg-prod}^{-1} \, \Lambda^{f(u)} \quad \forall \fct
  \\ \label{e:varset-deg-prod}
  \prod\fctrange \vdeg*
  & \le \Lambda^{f(u)}
  \\ \label{e:varset-ht}
  \sum\fctrange \wtw \wts* \hautl[1]{\var*}
  & \le \cst{vs-ht} \Lambda^{2\,f(u)} \wts[1]
  \pmm,
\end{align}
où l'on a noté \( \vdeg* = \deg \var* \) et \( \vdim = \dim \var \), ainsi que
\worknote{\( \epsi \) défini plus tard, à remplacer ici après avoir vérifié si
  ça vaut la peine de supprimer la dépendance en \( \dimp \)}
\begin{align}
  \label{e:def-Lambda}
  \Lambda
  & = \max \bigl(
    (\deg \va)^\puiss,
    \frac{ 96\puiss^2 }{ \eps \epsi } \, \deg \va,
    (\sqrt2 \puiss \genre)^{\puiss\genre}
  \bigr)
  \\
  f(u) & = \prod_{i=u+1}^{mg} (2i + 1) \le m^{2m^2 - u}
  \\
  \newcst[]{vs-deg-prod}
  & = \frac{ 96\puiss^2 }{ \eps \epsi }
  \\
  \newcst[\va]{vs-ht} & = \stuffout{\max(n^{m^2}, \vaht, \hlclab)}
\end{align}
de sorte que l'ensemble \( \varset(\ex) \) ne dépend de \( \ex \) que
par la condition \( \ex \in \var \) et les rapports entre les \( \wt* \).

Si \( \ex \) satisfait \eqref{e:Vbig}, cet ensemble ne contient que des
variétés dont tous les facteurs sont de dimension au moins \( 1 \). En effet,
dans le cas contraire, on aurait \( \var* = \ex* \) pour un certain \( \fct
\), ce dont on déduirait successivement
\begin{alignat}{2}
  \wts* \hautn{\ex*}
  & \le 2\cst{vs-ht} \Lambda^{2\,f(u)} \wts[1]
  &\quad& \text{par \eqref{e:varset-ht} et~\ref{f:comp-h-hn},}
  \\
  \wts[1] \hautn{\ex[1]}
  & \le 3\cst{vs-ht} \Lambda^{2\,f(u)} \wts[1]
  && \text{par \eqref{e:wt-ratio},}
\end{alignat}
qui contredirait précisément \eqref{e:Vbig}.

Or cet ensemble \( \varset(\ex) \) n'est certainement pas vide, puisqu'il
contient au moins \( \va^\puiss \) ; il possède donc au moins un élément
minimal. La proposition suivante implique que cet élément a au moins un
facteur réduit à un point, ce qui, nous venons de le voir, fournit la
contradiction prouvant le théorème \ref{t:vojta-div}.

\begin{prop} \label{p:varset-notmin}
  Soit \( \var \in \varset(\ex) \) n'ayant aucun facteur de dimension nulle.
  Alors il existe un \( \fct \) et une forme \( T \) ne dépendant que de \(
    \vmp* \), s'annulant en \( \ex \) mais pas identiquement sur \( \var \),
  telle que \worknote{Rémond prop. 4.2 p. 118}
  \begin{align}
    \deg T
    & \le \stuffout{\Lambda^{2u f(u)}}
    \\
    \wts* \hautl[1] T
    & \le \stuffout{\newcst[]{vs-div-ht} \cst{vs-ht}
      \Lambda^{(3u + 1/m) f(u)} \wts[1]}
    \pmm.
  \end{align}
\end{prop}

En effet, on note alors \( \var' \) la variété égale à \( \var \) sur tous les
facteurs sauf le \( \fct \)-ème où on choisit une composante irréductible de
\( \var* \cap \zeros T \) qui contient \( \ex* \), de sorte que \( \var' \)
est une variété strictement contenue dans \( \var \) et contenant \( \ex \).
Le théorème de \bsc{Bézout} donne alors
\worknote{LNM7, thm 3.4, coro 3.6 ou LNM6 thm, lemme 4.1}
\begin{equation}
  \deg \var*' \le \vdeg* \deg T
\end{equation}
et sa version arithmétique indique que
\begin{equation}
  \stuffout{
    \hautl[\mathrm{proj}]{\var*'}
    \le
    \vdeg* \hautl[1]{T}
    + \hautl[\mathrm{proj}]{\var*} \deg T
  }
  \pmm,
\end{equation}
ce qui montre que \( \var' \in \varset(\ex) \) (au vu des constantes qui
seront déterminées par ce point) et que \( \var \) n'était pas minimale.

\medskip

Il suffit donc d'établir la proposition \ref{p:varset-notmin} pour prouver le
théorème \ref{t:vojta-div} ; la démonstration de cette proposition nous
occupera le reste du chapitre. L'intérêt de l'énoncer immédiatement plutôt
qu'au moment où nous serons en mesure de l'établir est de pouvoir écarter
rapidement quelques cas particuliers pour lesquels la méthode qui suit ne
s'applique pas (mais qui sont heureusement immédiats). Nous énoncerons ces cas
à la fin de la section suivante après avoir introduit le système de
coordonnées dans lequel ils s'expriment naturellement.

\begin{tdef} \label{d:var&co}
  \nomuse \var [chap]{Une variété de \( \varset(\ex) \),
    définition~\ref{d:var&co}}
  \nomuse{\vdeg*}[chap]{Le degré de \( \var* \), définition~\ref{d:var&co}}
  \nomuse{\vdim*}[chap]{La dimension de \( \var* \),
    définition~\ref{d:var&co}}
  \nomuse{\vdim }[chap]{La dimension de \( \var \), définition~\ref{d:var&co}}
  \nomuse{\varfc*}[chap]{Une forme de \bsc{Chow} de \( \var* \),
    définition~\ref{d:var&co}}
  \nomuse{\varid}[chap]{L'idéal homogène saturé de \( \var \),
    définition~\ref{d:var&co}}
  Désormais et jusqu'à la fin du chapitre, nous fixons une variété
  \( \var = \var[1] \times \dots \times \var[\puiss] \) satisfaisant aux
  hypothèses de la proposition ; en particulier elle n'est contenue dans aucun
  hyperplan d'équation \( \vmp*[0] = 0 \) puisqu'elle contient
  \( \ex \) qui n'est d'après \eqref{e:Vapx} sur aucun de ces hyperplans, et
  aucun de ces facteurs n'est réduit à un point.

  On notera en outre \( \vdim* = \dim \var* \) et \( \vdim = \dim \var \) puis
  \( \vdeg* = \deg \var* \), et \( \varfc* \) une
  \footnote{\label{fn:varfc}Cette forme est unique à multiplication par un
    scalaire près ; au besoin on la supposera normalisée de sorte que toutes
    ses normes locales soient au moins \( 1 \), par exemple en faisant en
    sorte qu'un de ses coefficients soit \( 1 \).}
  forme de \bsc{Chow} de \( \var* \). On note enfin \( \varid \) l'idéal
  multihomogène saturé de $\var$ et $\varid*$ ceux de ses facteurs.
\end{tdef}

De façon générale, si $A$ est une algèbre graduée et $\Ideal$ un idéal
homogène, on notera $A_d$ et $\Ideal_d$ leur partie homogène de degré $d$ ; on
utilisera la même notation pour les algèbres et idéaux multigradués, où $d$
désignera une famille d'entiers.

\medskip

La preuve de la proposition~\ref{p:varset-notmin} suit la méthode de
\TS : on construit d'abord une forme auxiliaire ayant un indice élevé le long
de \( \divi \), puis on montre qu'elle s'annule en \( \ex \) avec un indice
plus faible mais toujours élevé par rapport à son degré et on conclut en
appliquant une variante du théorème du produit pour obtenir la conclusion de
la proposition.

Auparavant, nous aurons besoin de préciser un système de coordonnées adapté à
\( \var \) et nous présenterons une écriture réduite des formes sur \( \var \)
dans ce système de coordonnées. Nous présenterons ensuite un plongement
éclatant associé aux poids \( \wt* \) qui permettra par la suite
d'exploiter~\eqref{e:hautn-wt-diff}.


\subsection{Plongements projectifs adaptés}
\label{sec:plong-adapt}

\begin{tdef} \label{d:plong-adapt}
  Suivant \cite{remivg}, si \( \anyvar \) est une variété de dimension
  \( \anydim \) et \( \projd \) un espace projectif muni de coordonnées
  homogènes \( \anyvp[0], \dots, \anyvp** \), on dit qu'un plongement
  \( \iota\colon \anyvar \embedin \proj\dimp \) est adapté si
  \begin{enumthm}
    \item \( \anyvar \cap \zeros{\anyvp[0], \dots, \anyvp[\anydim]}
        = \emptyset \) ;
    \item \( \korper{\anyvar} \) est engendré par
      \( \frac{\anyvp[1]}{\anyvp[0]}, \dots,
        \frac{\anyvp[\anydim+1]}{\anyvp[0]} \) ;
    \item \( \frac{\anyvp[\anydim+1]}{\anyvp[0]} \neq 0 \) dans \(
        \korper\anyvar \).
  \end{enumthm}
\end{tdef}

Le fait suivant, qui ne fait que rappeler \cite[partie~4.1, p.~114]{remivds},
explicite les principales propriétés d'un plongement adapté, qui est
une version plus précise de la mise en position de \bsc{Noether}.

\begin{fact} \label{f:plong-adapt-gen}
  Si \( \anyvar \) de dimension \( \anydim \)  est plongée dans \( \projd \)
  de façon adaptée, alors les fonctions rationnelles
  \( \frac{\anyvp[1]}{\anyvp[0]}, \dots, \frac{\anyvp[\anydim]}{\anyvp[0]} \)
  forment une base de transcendance de \( \korper\anyvar \) sur \( \cdn \). De
  plus, \( \frac{\anyvp[\anydim+1]}{\anyvp[0]} \) est un élément primitif de
  \( \korper\anyvar \) sur \( \cdn( \frac{\anyvp[1]}{\anyvp[0]}, \dots,
    \frac{\anyvp[\anydim]}{\anyvp[0]} ) \).

  La projection linéaire \( \anyvar \to \proj\anydim \) obtenue en ne gardant
  que les \( \anydim + 1 \) premières variables est un revêtement ramifié.
\end{fact}

On contrôle en fait des relations de dépendance intégrale des dernières
variables sur la base de transcendance choisie.

\begin{fact} \label{f:plong-adapt-dep}
  Si le plongement \( \iota \colon \anyvar \embedin \projd \) est adapté, il
  existe des formes homogènes \( \poldep[][\ind] \) pour \( \ind \in
    \set{\anydim+1, \dots, \dimp} \) telles que :
  \begin{enumthm}
    \item \(
        \poldep[][\ind]
        \in
        \cdn [ \anyvp[0], \dots, \anyvp[\anydim], \anyvp[\ind] ]
        \cap \ideal\anyvar \) ;
    \item \( \poldep[][\ind] \) est unitaire de degré \( \anydeg \) en \(
        \anyvp[\ind] \) ;
    \item \( \deg \poldep[][\ind] = \anydeg \) ;
    \item \( \nv1{ \poldep[][\ind] } \le \nv1{ \chow\anyvar } \) ;
  \end{enumthm}
  où \( \anydeg \) est le degré de \( \anyvar \) dans ce plongement.
\end{fact}

\begin{proof}
  Le lemme~4.1 de \cite{remivds} donne explicitement des formes satisfaisant
  les trois premières conditions.

  Seule l'assertion sur la norme n'y est pas énoncée sous cette forme mais
  elle vient en remarquant que \( \poldep_\ind \) est une spécialisation de
  \( \varfc \) qui annule certaines variables et remplace les autres par des
  monômes unitaires.
\end{proof}

Nous établissons maintenant une variante de la proposition~4.1, de
\cite{remivds} qui montre qu'il est possible de rendre adapté un plongement
donné tout en gardant fixe le diviseur \( \divi \), à peu de frais.

\begin{lem} \label{l:adapt-gen}
  Soit $\iota \colon \anyvar \embedin \projd$ un sous-schéma fermé intègre de
  degré $\anydeg$, non contenu dans l'hyperplan d'équation $\anyvp[0] = 0$.
  Il existe une transformation linéaire $\chi \in \GL_{\dimp+1}(\Q)$,
  représentable par une matrice à coefficients entiers de
  valeur absolue (archimédienne) majorée par $\max(\frac\anydeg2, 1)$, telle
  que $\chi \circ \iota$ est un plongement adapté à \( \anyvar \) et que
  $\anyvp[0]$ soit invariant par ce changement de coordonnées.
\end{lem}

\begin{proof}
  On reprend la preuve de la proposition citée (p.~116) ; au moment de choisir
  des formes linéaires $L_0, \dots, L_n$ telles que
  \begin{equation*}
    \chow \anyvar (L_0, \dots, L_\dimp) \neq 0
    \pmm,
  \end{equation*}
  on commence en fait par fixer $L_0 = \anyvp[0]$. Le polynôme $\varfc(L_0,
  \truc, \dots, \truc)$ est multihomogène de degré $\anydeg$ en chaque
  variable ; vu l'hypothèse sur $\anyvar$, il est non nul grâce au théorème
  fondamental de l'élimination. On peut donc choisir $L_1, \dots L_\anydim$
  comme dans \cite{remivg} puis continuer la preuve sans autre modification.
\end{proof}

Nous aurons également besoin de contrôler \( \chi^{-1} \). Le lemme suivant
établit un résultat général élémentaire sur l'inversion de matrices.

\begin{lem} \label{l:cramer}
  Soit \( M \) une matrice \( p \times p \) inversible à coefficients entiers
  de valeur absolue inférieure ou égale à \( C \).  On a alors \(
    \hautm[\infty]{M^{-1}} \le (p-1)! \cdot C^{p-1} \).
\end{lem}

\begin{proof}
  On utilise les formules de \bsc{Cramer}. On constante d'abord que la
  contribution de l'inverse du déterminant s'élimine en prenant le produit sur
  toutes les places, de sorte qu'il s'agit d'estimer la hauteur de la
  comatrice. Cette dernière est à coefficients entiers, donc de norme
  inférieur à \( 1 \) aux places finies, et la valeur absolue archimédienne
  des coefficients est majorée par \( (p-1)! \, C^{p-1} \).
\end{proof}

\begin{tdef}
On fixe désormais des transformations linéaires \( \vadapt* \) obtenues en
appliquant le lemme~\ref{l:adapt-gen} à chacun des \( \var* \), de sorte que
\( \vadapt* \circ \Theta \) est adapté à \( \var* \) et on note \(
  \vdegp* = \max(\vdeg*/2, 1) \).
\end{tdef}

La construction et le lemme précédent montrent alors que
\begin{equation} \label{e:vadapt-ht}
  \hautm[\infty]{\vadapt*}
  \le
  \vdegp*
  \qquad\text{et}\qquad
  \hautm[\infty]{\vadapt*^{-1}}
  \le
  \dimp! \cdot (\vdegp*)^{\dimp}
\end{equation}
On note \( \varida* \) l'idéal homogène saturé de \( \var* \) dans le
plongement \( \vadapt* \circ \Theta \) et \( \varfca* \) une forme de
\bsc{Chow} de \( \var* \) dans ces coordonnées. Le théorème de l'élimination
montre qu'une telle forme s'obtient en composant \( \varfc* \) avec la
transposée de \( \vadapt* \) sur chacun des \( \vdim* + 1 \) groupes de
variables ; il est alors clair que
\begin{equation} \label{e:nv-varfca}
  \nv1{\varfca*} \le \nv1{\varfc*}
  \cdot \bigl( (\dimp+1) \vdegp* \bigr)^{\vdeg* (\vdim* + 1)}
\end{equation}

Par ailleurs, pour \( \indi \in \set{\vdim*, \dots, \dimp} \), on notera \(
  \poldep** \) une relation de dépendance de \( \vmp*[\indi] \) sur \(
  \vmp*[0], \dots, \vmp*[\vdim*] \) telle que donnée par l'application du
fait~\ref{f:plong-adapt-dep} au facteur \( \var* \) dans le plongement \(
  \vadapt* \circ \Theta \). L'estimation précédente et le fait cité donnent
une majoration des normes locales de ces polynômes, qu'on peut en fait
améliorer en
\begin{equation} \label{e:nv-poldep}
  \nv1{\poldep**} \le \nv1{\varfc*}
  \cdot ( 2 \vdegp* )^{\vdeg* (\vdim* + 1) \dv}
\end{equation}
en utilisant le début de la démonstration du lemme~4.2 de~\cite{remivds} (la
suite de la démonstration étant moins pertinente compte tenu des différences
de normes utilisées). Cependant, nous utiliserons souvent la majoration plus
évidente \( \nv1{\poldep**} \le \nv1{\varfca*} \) dans les estimations où
cette quantité apparaît déjà par ailleurs.

On introduit ensuite \( \pden** \) la dérivée de \( \poldep** \) par rapport à
la dernière variable et \( \pdenp* = \prod_{\indi=\vdim*+1}^\dimp \pden** \).
Enfin, on notera \( (\cex**)_\ind \) un système de coordonnées
multihomogène de \( \ex* \) dans le plongement \( \Theta \) et \(
  (\cexa**)_\ind \) l'image de ce dernier par \( \vadapt* \).  Nous pouvons
maintenant énoncer les cas particuliers à exclure dans la démonstration de la
proposition~\ref{p:varset-notmin}.

\begin{scho} \label{s:part-cases}
  Dans la démonstration de la proposition~\ref{p:varset-notmin}, on peut
  supposer que :
  \begin{enumthm}
    \item \( \cexa** \neq 0 \) pour tous \( \fct \) et \( \ind \in \set{1,
          \dots, \vdim*} \) ;
    \item \( \pdenp*(\cexa*) \neq 0 \) pour tout \( \fct \).
  \end{enumthm}
  Remarquons qu'on a déjà \( \cexa*[0] \neq 0 \) d'après
  l'hypothèse~\eqref{e:Vapx}.
\end{scho}

\begin{proof}
  Si le premier point n'est pas satisfait, pour un certain \( (\fct, \ind) \),
  on peut prendre \( T = \vmp** \) dans la conclusion de la
  proposition ; en effet cette forme ne s'annulle en \( \ex \) mais pas
  identiquement sur \( \var \) (car le plongement est adapté) et les
  conditions de degré et de hauteur sont largement satisfaites.

  \later
  Si le deuxième point est faux, on a aussi la conclusion de la proposition en
  procédant comme au lemme~4.3 de \cite{remivds}.
\end{proof}


\subsection{Réduction de formes sur une variété plongée de façon adaptée}
\label{sec:rfull}

Les propriétés des plongements adaptés permettent d'associer à chaque forme
homogène, à peu de chose près, une représentation canonique modulo \( \varida
\), définie par des restrictions de degrés en certains variables.
Introduisons pour cela quelques notations : pour chaque $C \in (\N \cup \set{
  +\infty }) ^{\puiss(\dimp+1)}$, on notera
\begin{equation} \label{e:C-spaces}
  \cdn[\vmp]^C
  = \{
    H \in \cdn[\vmp]
    \text{ tel que }
    \deg_{\vmp*[k]} H \le C\pexp{\fct}[k]
    \quad \forall i, k
    \}
  \pmm.
\end{equation}
On utilisera trois tels vecteurs $C'$, $C''$, $C'''$, définis respectivement
par
\begin{gather} \label{e:C-i-iii}
  (C')\pexp\fct[\ind] =
  \begin{cases}
    +\infty & \text{si $\ind \le \vdim*$} \\
    \vdeg* - 1 & \text{sinon}
  \end{cases}
  \qquad
  (C'')\pexp\fct[\ind] =
  \begin{cases}
    +\infty & \text{si $\ind \le \vdim* + 1$} \\
    0 & \text{sinon}
  \end{cases}
  \\
  (C''')\pexp\fct[\ind] = \min\bigl(
    (C')\pexp\fct[\ind], (C'')\pexp\fct[\ind]
  \bigr)
  \pmm,
\end{gather}
pour $0 \le \fct \le \puiss$ et $0 \le k \le \dimp$.  Par ailleurs, on notera
$C'_\Delta$ et $C'''_\Delta$ les vecteurs obtenus en remplaçant $D$ par un
certain $\Delta$ dans la définition précédente.

L'intérêt de \( \cdn[ \vmp ]^{C'''} \) est que, le plongement étant
adapté, son intersection avec \( \varida \) est réduite à \( 0 \) : en effet,
\( \vmp*[\vdim*] \) est de degré \( \vdeg* \) sur les autres variables, il
n'est donc pas possible qu'un polynôme de degré inférieur en \( \vmp*[\vdim*]
\) soit dans \( \varida \) sans être nul. Par ailleurs, on peut facilement
vérifier que les dimensions de \( \cdn[ \vmp ]^{C'''}_{\delta} \) et \(
  ( \cdn[\vmp] / \varida )_\delta \) sont équivalentes quand \( \delta \) tend
vers l'infini, de sorte que le morphisme de réduction, qui est injectif, n'est
pas très loin d'être un isomorphisme en degré assez grand.

Nous allons maintenant expliciter, en chaque multidegré \( \beta \in \N^\puiss
\), une application linéaire \( \rfull^\beta \) définie sur la partie homogène
\( \cdn[ \vmp ]_\beta \) et à valeurs dans \( \cdn[ \vmp ]^{C'''} \), qui
permet par exemple de déterminer l'appartenance à \( \varida \) d'une forme
homogène.

L'idée générale est la suivante : on peut exploiter les relations de
dépendance des dernières variables sur les premières données par le
fait~\ref{f:plong-adapt-dep} pour réduire le degré en les dernières variables,
et le deuxième point de la définition~\ref{d:plong-adapt} permet même
d'éliminer totalement les variables d'indice strictement supérieur à \( \vdim*
  + 1 \), à condition de multiplier par un certain dénominateur intervenant
dans ces relations de dépendance rationnelles.

Il sera essentiel par la suite que ce dénominateur puisse être choisi
indépendamment du degré considéré ; pour cela l'application \( \rfull \)
sera la composée des trois étapes suivantes : une application \( \rdiv \)
arrivant dans \( \cdn[ \vmp ]^{C'} \), c'est-à-dire faisant chuter le degré ne
les dernières variables par division euclidienne, une application \( \rdiv \)
arrivant dans \( \cdn[ \vmp ]^{C''} \), c'est-à-dire éliminant
les toutes dernières variables en faisant éventuellement croître le degré en
\( \vmp*[\vdim*] \), et une dernière application \( \relim \) pour limiter à
nouveau ce degré et arriver dans \( \cdn[ \vmp ]^{C'''} \).

Avant de procéder à la construction des applications évoquées, soulignons
qu'il s'agit d'applications linéaires définies sur chaque partie homogène de
\( \cdn[ \vmp ] \) mais en aucun cas de morphismes d'algèbres, les espaces
d'arrivée n'étant eux-mêmes pas des algèbres.

Pour construire l'application \( \rdiv \) on commence par énoncer un résultat
de réduction modulo des relations de dépendance intégrale sous une forme un
peu générale avant de l'appliquer au cas qui nous intéresse.

\begin{lem}
  Pour \( \fct \in \{ 1, \dots, \puiss \} \) et \( \ind \in \{ \vdim* + 1,
  \dots \dimp \} \), on se donne :
  \begin{enumthm}
    \item \( \Delta\mexp* \in \N^* \) ;
    \item \( P\mexp*[\ind]
      \in
      \cdn [ \vmp*[0], \dots, \vmp*[\vdim*], \vmp*[\ind] ] \)
      homogène de degré \( \Delta\mexp* \) et unitaire en \( \vmp*[\ind]
      \).
  \end{enumthm}
  On note \( N_\fct = \max_\ind \nv1 { P\mexp*[\ind] } \) et \( \Ideal \)
  l'idéal engendré par les \( P\mexp*[\ind] \). En tout multidegré
  \( \alpha \in \N^\puiss \), il existe une (unique) application linéaire
  \begin{equation}
    \rdiv^\alpha \colon \cdn [\vmp]_ \alpha \to \cdn [\vmp]_ \alpha^{C'_\Delta}
  \end{equation}
  qui est l'identité modulo \( \Ideal \) (voir~\eqref{e:C-spaces}
  et~\eqref{e:C-i-iii} pour la définition de l'espace d'arrivée). De plus les
  colonnes \( c_p \) de la matrice de cette application satisfont la majoration
  de norme
  \begin{equation}
    \nv1{c_p}
    \le
    \prod\fctrange \bigl(
    N_\fct \cdot (2 \Delta\mexp*)^\dv
    \bigr) ^{\alpha_\fct}
  \end{equation}
  pour tout \( p \in \N^{\puiss(\dimp+1)} \) de multilongueur \( \alpha \), et
  l'image de \( \cdn [\vmp]_ \alpha ^{C''_\Delta} \) par \( \rdiv^\alpha \) est
  contenue dans \( \cdn [\vmp]_ \alpha ^{C'''_\Delta} \).
\end{lem}

\begin{proof}
  C'est essentiellement une variante du lemme~2.5 de~\cite{remivg}, notre
  résultat étant formulé différemment et dans un cadre d'apparence un peu
  moins générale ; la preuve suivra en tout cas les mêmes lignes. On commence
  par décomposer chaque \( P\mexp*[\ind] \) de la façon suivante :
  \begin{equation}
    P\mexp*[\ind]
    =
    \sum _{\alpha=1}^{\Delta\mexp*}
    P\mexp*[\ind, \alpha] \cdot (\vmp*[\ind])^{\Delta\mexp* - \alpha}
    \pmm,
  \end{equation}
  où \( P\mexp*[\ind] \in \cdn [ \vmp*[0], \dots, \vmp*[\vdim*] ] \).
  On pose ensuite, pour tout \( \fct \), tout
  \( \ind \in \{ 0, \dots, \dimp \} \) et tout
  \( p \in \N^{\puiss(\dimp+1)} \),
  \begin{equation}
    \rho_{\fct, \ind, p}
    =
    \begin{dcases*}
      ( \vmp*[\ind] ) ^{p\mexp*[\ind]}
      & si \( \fct \le \vdim* \) ;
      \\
      \sum _{\alpha=1}^{\Delta\mexp*}
      U_{p, \alpha, \Delta\mexp*}
      ( P\mexp*[\ind, 1], \dots, P\mexp*[\ind, \Delta\mexp*] )
      & sinon,
    \end{dcases*}
  \end{equation}
  où les polynômes \( U \) sont donnés par le lemme~2.4 de \cite{remivg}. En
  particulier, \( \rho_{\fct, \ind, p} \) est toujours congru à
  \( ( \vmp*[\ind] ) ^{p\mexp*[\ind]} \) modulo \( \Ideal \) et on a
  l'estimation de norme
  \begin{equation}
    \nv1{ \rho_{\fct, \ind, p} }
    \le
    \left(
    \nv1{ P\mexp*[\ind] } (2\Delta\mexp*)^\dv
    \right) ^{p\mexp*[\ind]}
    \pmm.
  \end{equation}
  On définit alors \( \rdiv^\alpha \) par son action sur les monômes, en posant
  \begin{equation}
    c_p
    = \rdiv^\alpha(\vmp^p)
    = \prod\fctrange \prod\indrange \rho_{\fct, \ind, p}
  \end{equation}
  et en prolongeant par linéarité. L'estimation de norme annoncée découle
  directement de la majoration précédente en prenant le produit. Par ailleurs,
  il est clair que si une forme ne fait intervenir que les variables \(
  \vmp*[\ind] \) pour \( \ind \le \vdim* + 1 \), il en est de même de son
  image.
\end{proof}

\begin{coro} \label{c:hmat-rdiv}
  Pour tout \( \alpha \in \N^\puiss \), il existe une application linéaire \(
    \rdiv^\alpha \) égale à l'identité modulo \( \varida \), dont la matrice
  dans la base monomiale canonique a des colonnes de normes \( \nv1\truc \)
  majorées par
  \begin{equation}
    \prod\fctrange \left(
    \nv1{ \varfc* }
    2 B^{(B(\genre+1) + 1)\dv}
  \right) ^{ \alpha_\fct }
  \pmm.
  \end{equation}
\end{coro}

\begin{proof}
  Découle directement du lemme précédent en utilisant~\eqref{e:nv-poldep} et
  en remarquant que
  \begin{equation}
    ( 2 \vdegp* )^{\vdeg* (\vdim* + 1)} \cdot 2 \vdeg*
    \le
    2 B^{(B(\genre+1) + 1)}
  \end{equation}
  car par définition \( B \) est un majorant commun des \( \vdeg* \) et donc
  de \( 2\vdegp* \), et pour tout \( \fct \) on a \( \vdim* \le \genre \).
\end{proof}

Intéressons-nous maintenant au morphisme \( \relim \).

\begin{lem}
  Soient, pour tout \( \fct \in \{ 1, \dots, \puiss \} \) et tout \( \ind \in
  \{ \vdim* + 1, \dots \dimp \} \), des formes
  \( S\mexp*[\ind] \in \cdn [ \vmp*[0], \dots, \vmp*[\vdim*] ] \) et
  \( T\mexp*[\ind] \in \cdn [ \vmp*[0], \dots, \vmp*[\vdim*+1] ] \)
  telles que \( \deg S\mexp*[\ind] + 1 = T\mexp*[\ind] \) et des entiers
  \( \Delta_\fct \). On note \( \Ideal_{S, T} \) l'idéal engendré par les
  \( S\mexp*[\ind] \vmp*[\ind] - T\mexp*[\ind] \).

  Il existe une forme \( R \) ne dépendant que des familles \( S \) et \(
    \Delta \), de degré noté \( r \), et, pour tout \( \alpha \in \N^\puiss \),
  une application linéaire
  \begin{equation}
    \relim^\alpha \colon
    \cdn [\vmp]_ {\alpha}^{C'_\Delta}
    \to
    \cdn [\vmp]_ {\alpha+r}^{C'_\Delta}
  \end{equation}
  qui est la multiplication par \( R \) modulo \( \Ideal_{S, T} \).

  De plus, on peut prendre
  \( R = \prod\fctrange \prod_{\ind = \vdim*+1}^{\dimp}
    ( S\mexp*[\ind] )^{\Delta_\fct} \) ; les colonnes \( c_q^{\relim} \) de la
  matrice de \( \relim \) dans la base canonique satisfont alors
  \begin{equation}
    \nv1{ c_q^{\relim} }
    \le
    \prod\fctrange N_\fct^{\Delta_\fct}
  \end{equation}
  pour tout \( q \) de multilongueur \( \alpha \), où \( N_\fct \) majore
  \( \nv1{ S\mexp*[\ind] } \) et \( \nv1{ T\mexp*[\ind] } \) pour tout
  \( \ind \).
\end{lem}

\begin{proof}
  Soit \( R \) défini comme dans l'énoncé, et \( \vmp^q \) un monôme de
  l'espace de départ. Par hypothèse, \( q\mexp*[\ind] \le \Delta_\fct \) pour
  tout \( \fct \) et \( \ind \le \vdim* + 1 \), de sorte que l'on peut poser
  \begin{equation}
    \relim( \vmp^q )
    =
    \prod\fctrange \left(
    \prod_{\ind=1}^{\vdim*}
    (\vmp*[\ind])^{q\mexp*[\ind]}
    \prod_{\ind=\vdim*+1}^\dimp
    (T\mexp*[\ind])^{q\mexp*[\ind]}
    (S\mexp*[\ind])^{\Delta_\fct - q\mexp*[\ind]}
    \right)
  \end{equation}
  et prolonger par linéarité. On vérifie immédiatement que \( \relim(\vmp^q)
  \) est congru à \( R \cdot \vmp^q \) modulo \( \Ideal_{S, T} \), de même
  que l'estimation de norme annoncée.
\end{proof}

\begin{coro} \label{c:hmat-relim}
  Il existe une forme \( R \in \cdn [\vmp]^{C''} \) ne dépendant que de \(
    \var \) et n'appartenant pas à \( \varida \), et
  une application linéaire
  \begin{equation}
    \relim^\alpha \colon
    \cdn [\vmp]_ {\alpha}^{C'}
    \to
    \cdn [\vmp]_ {\alpha+r}^{C'}
  \end{equation}
  qui est la multiplication par \( R \) modulo \( \varida \).  De plus, les
  colonnes de la matrice de \( \relim^\alpha \) dans les bases monomiales
  canoniques ont leur norme \( \nv1\truc \) majorée par
  \begin{equation}
    \prod\fctrange
    \nv1{ N'_\fct }^{\vdeg*}
    \pmm,
  \end{equation}
  où \( N'_\fct \) est une constante ne dépendant que de \( \var* \).
\end{coro}

\begin{proof}
  Il suffit d'établir l'existence de familles \( S \) et \( T \) comme dans
  l'énoncé du lemme précédent, telles que \( \Ideal_{S, T} \subset \varida \)
  et \( S \notin \varida \) ;
  elle découle du fait qu'on a utilise un plongement adapté.
  En effet, d'après le fait~\ref{f:plong-adapt-gen}, pour tous \( \fct \) et
  \( \ind \), il existe des formes \( A\mexp*[\ind, \beta] \) et \(
    B\mexp*[\ind, \beta] \) dans
  \( \cdn [ \vmp*[0], \dots, \vmp*[\vdim*] ] \) telles que
  \begin{equation}
    \frac{ \vmp** }{ \vmp*[0] }
    =
    \sum_{\beta = 0}^{\vdeg* - 1}
    \frac {A\mexp*[\ind, \beta]} {B\mexp*[\ind, \beta]}
    \left( \frac{ \vmp*[\vdim* + 1] }{ \vmp*[0] } \right) ^\beta
    \mod \varida*
    \pmm.
  \end{equation}
  On obtient alors les familles \( S \) et \( T \) recherchées en multipliant
  les deux membres de l'égalité précédente par \( \vmp*[0] \) puis en
  réduisant au même dénominateur le membre de droite.
\end{proof}

On note désormais \( R \) la forme donnée par le lemme précédent et \( r \in
  \N^\puiss \) son multidegré. Notons qu'on a pas besoin d'en savoir plus sur
cette forme, à part le fait qu'elle n'appartient pas à \( \varida \) et ne
dépend que de \( \var \) mais pas du degré \( \alpha \) considéré. En effet,
en pratique on fera tendre ce dernier vers l'infini ; c'est en ce sens qu'il
faut comprendre la notation \( o \) dans le lemme suivant, qui résume la
construction de \( \rfull^\alpha \) à partir des briques précédentes.

\begin{lem} \label{l:rfull}
  Pour tout multidegré \( \alpha \in \N^\puiss \) il existe une application
  linéaire
  \begin{equation}
    \rfull^\alpha \colon
    \cdn [\vmp]_ {\alpha}
    \to
    \cdn [\vmp]_ {\alpha+r}^{C'''}
  \end{equation}
  qui est égale à la multiplication par \( R \) modulo \( \varida \). De plus,
  les colonnes de sa matrice dans les bases monomiales canoniques sont de
  norme \( \nv\infty\truc \) majorée par
  \begin{equation}
    \prod\fctrange \left(
      \nv1{ \varfc* }
      2 B^{(B(\genre+1) + 1)\dv}
    \right) ^{ 2\alpha_\fct }
    \cdot \expb^{o(\alpha)}
    \pmm.
  \end{equation}
  En particulier, on a \( \ker \rfull^\alpha = \cdn[ \vmp ]_\alpha \cap
    \varida \).
\end{lem}

\begin{proof}
  Il suffit d'utiliser les résultats des lemmes précédents et de poser
  \begin{equation}
    \rfull^\alpha
    =
    \rdiv^{\alpha+r} \circ \relim^{\alpha} \circ \rdiv^\alpha
    \pmm.
  \end{equation}
  L'assertion sur la norme est immédiate en remarquant que, si \( M_1 \) et \(
    M_2 \) sont deux matrices, la norme \( \nv\infty\truc \) de leur produit
  est majorée par \( \nv\infty{M_1} \max(\nv1{c}) \) où \( c \) parcourt les
  colonnes de \( M_2 \). On absorbe par ailleurs les constantes ne dépendant
  pas de \( \alpha \), à savoir le \( r \) et la norme de \( \relim \), dans le
  \( o(\alpha) \).

  Pour le noyau, considérons une forme \( H \) de multidegré \( \alpha \). On
  constate d'une part que si \( \rfull^\alpha(H) = 0 \), alors \( R H \in
    \varida \) donc \( H \in \varida \) car ce n'est pas le cas de \( R \) et
  que \( \varida \) est premier. Réciproquement, si \( H \in \varida \), alors
  \( RH \in \varida \) et \( \rfull^\alpha(H) \in \varida \cap \cdn[ \vmp
    ]^{C'''} = \set0 \).
\end{proof}

Notons qu'en prenant la somme, on peut définir une application linéaire \(
  \rfull \) sur \( \cdn[\vmp] \) entier ; on utilisera cette notation quand il
ne sera pas utile de préciser le degré.


\subsection{Plongement abélien pondéré}
\label{sec:wemb}

Introduisons un plongement, dit \emph{éclatant} ou pondéré par \( \wt \),
défini par
\nomuse \wemb [chap] {Plongement éclatant introduit à la
  section~\ref{sec:wemb}}
\begin{alignat}{2} \label{e:def-wemb}
  \wemb \colon && \var
  & \longto \va^\puiss \times \va^{\puiss-1}
  = \va^{2\puiss-1}
  \\ &&
  (\pmp_1, \dots, \pmp_\puiss)
  & \longmapsto
  (\pmp_1, \dots, \pmp_\puiss;
  \wt[1] \pmp_1 - \wt** \pmp_\puiss, \dots,
  \wt[\puiss-1] \pmp_{\puiss-1} - \wt** \pmp_\puiss)
  \pmm.
\end{alignat}
qui nous permettra d'exploiter la relation~\eqref{e:hautn-wt-diff}.

Nous allons représenter ce morphisme par des familles de polynômes. Pour
cela, commençons par préciser les plongements utilisés : au départ, chaque
facteur \( \var* \) est plongé par \( \vadapt* \circ \Theta \) ; on utilise
ces mêmes plongements pour les \( \puiss \) premiers facteurs \( \va \) de
l'espace d'arrivée et on garde le plongement \( \Theta \) pour les \( \puiss-1
\) facteurs \( \va \) restants.

On munit l'espace \( (\projd)^{2\puiss-1} \) (dans lequel est plongé l'espace
d'arrivée) des coordonnées multihomogènes \( \vmp, \vmpi = \vmp[1], \dots,
  \vmp[\puiss], \vmpi[1], \dots, \vmpi[\puiss-1] \) ; dans ce contexte quand \(
  \fct \) et \( \fcti \) sont deux indices non précisés, on supposera
implicitement \( 1 \le \fct \le \puiss \) et \( 1 \le \fcti \le \puiss-1 \).

\nomuse \wemba [chap] {Représentation polynomiale du plongement éclatant}
Une représentation de \( \wemb \) dans ces plongements, définie sur un ouvert
\( \clmap \) de \( \va^\puiss \), est un morphisme \( \wemba \) tel que le
diagramme suivant, dont les flèches verticales sont les projections
canoniques, commute.
\begin{equation}
  \xymatrix{
    \cdn [\vmp, \vmpi]                      \ar [r] ^{\wemba}   \ar [d]
    & \cdn [\vmp]                                               \ar [d] ^\pi
    \\ \cdn [\vmp, \vmpi] / \idealim \ar [r] ^{\wemb^*}
    & ( \cdn [\vmp] / \varida )_\clmap
  }
\end{equation}
En bas, \( \idealim \) désigne l'idéal multihomogène de l'image
dans les coordonnées choisies et \( ( \cdn [\vmp] / \varida )_\clmap \)
désigne la localisation de l'anneau des coordonnées au départ correspondant à
l'ouvert \( \clmap \).

Le lemme~\ref{l:hclab} fournit un atlas \( \clmaps \) de \( \va^2 \) avec
sur chaque carte une représentation locale de \( (x, y) \mapsto \alpha x -
  \beta y \) dans le plongement \( \Theta \). Voyons comment en déduire un
atlas de \( \va^\puiss \) et des représentations locales de \( \wemb \) dans
les plongements considérés.

Soit \( \clmap = (\clmap_1, \dots, \clmap_{\puiss-1}) \in \clmaps^{\puiss-1}
\). On lui associe un ouvert de \( \va^\puiss \), que par abus on notera
encore \( \clmap \), défini par \( \bigcap\fctirange p_\fcti^{-1}(
  \clmap_\fcti ) \) où \( p_\fcti \) désigne la projection sur les facteurs \(
  \fcti \) et \( \fcti + 1 \). On introduit alors pour chaque \( \fcti \) la
forme
\begin{equation} \label{e:def-wembclm}
  \wembclm*
  = L\pexp{\wti*, \wti**, \clmap_\fcti}
  \bigl( \vadapt[\fcti]^{-1}(\vmp[\fcti]), \vadapt**^{-1}(\vmp[\puiss]) \bigr)
\end{equation}
où \( L\pexp{\wti*, \wti**, \clmap_\fcti} \) est donnée par le
lemme~\ref{l:hclab}, de sorte que chacune des applications
\begin{align} \label{e:def-wemba}
     \wemba* \colon \cdn [\vmp, \vmpi]
  &  \to \cdn[\vmp]
  \\ \vmp*
  &  \mapsto \vmp*
  \\ \vmpi*
  &  \mapsto \wembcl*_{\clmap_\fcti}
\end{align}
est une représentation locale de \( \wemb \) valable sur l'ouvert \( \clmap
\).  Dans la suite de cette section, la carte \( \clmap \) choisie n'a pas
d'importance, on notera donc simplement \( \wemba \) l'un des \( \wemba*
\) pour alléger.

Étudions maintenant l'action de \( \wemba \) sur le degré et la hauteur.  Le
lemme suivant est une conséquence immédiate de la définition.

\begin{lem} \label{l:deg-wemba}
  Soit \( H \in \Qbar[\vmp, \vmpi] \) une forme multihomogène de multidegré \(
    (\alpha, \beta) \) où \( \alpha \in \N^\puiss \) et \( \beta \in
    \N^{\puiss-1} \). On a alors
  \begin{equation}
    \deg \wemba(H)
    =
    \bigr(
    \alpha_1 + 2 \beta_1 \wts[1],
    \dots,
    \alpha_{\puiss-1} + 2 \beta_{\puiss-1} \wts[\puiss-1],
    \alpha_\puiss + 2 \lgr\beta \wts[\puiss]
    \bigl)
    \pmm.
  \end{equation}
\end{lem}

Avant de passer aux estimations de hauteur, il est utile d'introduire quelques
paramètres qui seront utilisés tout au long de ce chapitre et qui contrôlent
notamment le degré en lequel nous établirons ces estimations.

Introduisons un réel \( \epsi \) défini par :
\begin{equation} \label{e:def-epsi}
  \epsi =
  \eps^{\frac{\genre}{\puiss-\genre}}
  \cdot \Bigl(
    (\dimp + 1)^{\puiss+1}
    \, (33\puiss)^\genre
  \Bigr)^{\frac{-1}{\puiss-\genre}}
\end{equation}
et un rationnel \( \epsz \) tel que :
\begin{equation} \label{e:def-epsz}
  \frac{\eps \epsi}{33\puiss}
  \le \epsz \le
  \frac{\eps \epsi}{32\puiss}
  \pmm,
\end{equation}
et enfin un entier \( \delta \) (destiné à tendre vers l'infini) tel que \(
  \epsz \delta \) soit également entier. On pose alors
\begin{equation} \label{e:d-dp-def}
  \begin{aligned}
    d & = \bigl(
      \epsz \wts[1],
      \dots,
      \epsz \wts[\puiss],
      1, \dots, 1
    \bigr) \in \Q^{2\puiss-1}
    \\
    d' & = \bigl(
      \wts[1] (2 + \epsz),
      \dots,
      \wts[\puiss-1] (2 + \epsz),
      \wts[\puiss] (2\puiss - 2 + \epsz)
    \bigr) \in \Q^\puiss
  \end{aligned}
\end{equation}
de sorte que, si \( H \) est une forme de multidegré \( \Dz \) dans \(
  \cdn[\vmp, \vmpi] \), son image par \( \wemba \) est de multidegré
exactement \( \Di \) d'après le lemme~\ref{l:deg-wemba}. Pour alléger les
notations par la suite, on introduit le vecteur \( \wtw = (1, \dots, 1, \puiss
  - 1) \in \N^\puiss \), de sorte qu'on a \( d'_\fct = \wts* (2 \wtw* + \epsz)
\) pour tout \( \fct \).

\medskip

Revenons donc au calcul de l'action de \( \wemba \) sur la hauteur. Ce
morphisme étant homogène, il induit d'après la remarque ci-dessus une
application linéaire de \( \cdn [\vmp, \vmpi]_\Dz \) dans \( \cdn [\vmp]_\Di
\).  La base évidente de l'espace de départ est formée par les monômes \(
  \vmp^p \vmpi^q \) pour
\begin{equation}
  (p, q)
  \in \N^{\puiss (\dimp+1)} \times \N^{(\puiss-1) (\dimp+1)}
  \text{ tel que }
  \lgr{p\mexp*} = \Dz* = \delta \epsz \wts*
  \text{ et }
  \lgr{q\mexpi*} = \Dz_{\puiss + \fcti} = \delta
  \pmm.
\end{equation}

\begin{lem} \label{l:hmat-wemba}
  Avec les notations précédentes, l'application linéaire
  \begin{equation}
    \wemba \colon
    \cdn [\vmp, \vmpi]_\Dz
    \to
    \cdn [\vmp]_\Di
  \end{equation}
  est représentée dans les bases canoniques par une matrice de colonnes
  \(
  c_{p, q} = \wemba(\vmp^p\vmpi^q)
  = \sum_{\lgr s = \Di - \lgr p} c_{p, q; s} \vmp^{s+p}
  \)
  satisfaisant
  \begin{equation}
    \hautm[1]{c_{p, q}}
    \le
    \bigl(
      \hmclab (\dimp+1)!^2 \, B^{2\dimp}
    \bigr) ^{2 \delta \wts[1]}
    \pmm,
  \end{equation}
  où l'on rappelle que \( \hmclab \) est donné par la
  définition~\ref{d:hclab}.
\end{lem}

\begin{proof}
  La définition de \( \wemba \) montre que \(
  \wemba( \vmp^p \vmpi^q ) = \vmp^p \wemba(\vmpi^q) \), on a donc \(
  \nv1 {c_{p, q}} = \nv1 {\wemba(\vmpi^q)} \). Ainsi, on a
  \begin{align}
    \nv1 {c_{p, q}}
    & \le
    \prod\fctirange \prod\indrange
    \nv1{ \wembcl** \bigl(
        \vadapt[\fcti]^{-1}(\vmp[\fcti]), \vadapt**^{-1}(\vmp[\puiss])
      \bigr) }
    ^{q\pexp\fcti[\ind]}
    \\ & \le
    \prod\fctirange \prod\indrange \left(
      \nv1{ \wembcl** } \,
      \nv1{ \vadapt[\fcti]^{-1} }^{2\wtis*} \,
      \nv1{ \vadapt**^{-1} }^{2\wts**}
    \right) ^{q\pexp\fcti[\ind]}
    \\ & \le
    \prod\fctirange \left(
      \hmclab*^{\wtis* + \wts**} \,
      \nv1{ \vadapt[\fcti]^{-1} }^{2\wtis*} \,
      \nv1{ \vadapt**^{-1} }^{2\wts**}
    \right) ^\delta
  \end{align}
  en remarquant que, par définition de \( d \), on a \( \lgr{q\mexpi*} =
    \delta \) pour tout \( \fcti \), puis en
  utilisant le lemme~\ref{l:hclab}. En prenant le
  produit sur toutes les places (et en prenant la racine \( \delta \)-ième
  pour simplifier l'écriture) il vient, compte tenu de~\eqref{e:vadapt-ht} et
  de~\eqref{e:varset-deg}
  \begin{align}
    \hautm[1]{c_{p, q}}^{1/\delta}
    & \le
    \prod\fctirange
    \hmclab^{\wtis* + \wts**} \,
    \bigr( (\dimp+1)! (\vdegp[\fcti])^\dimp) \bigl)^{2 \wts[\fcti]} \,
    \bigr( (\dimp+1)! (\vdegp[\puiss])^\dimp) \bigl)^{2 \wts[\puiss]}
    \\ & \le
    \prod\fctirange
    \bigr( \hmclab (\dimp+1)!^2 \, B^{2\dimp} \bigr)^{\wtis* + \wts**} \,
  \end{align}
  Le résultat annoncé suit en invoquant le lemme~\ref{l:sum-wts}.
\end{proof}



\section{Construction d'une forme auxiliaire} \label{sec:siegel}

L'objectif de cette section est de construire une forme non nulle sur \( \var
\), provenant d'une forme sur \( \wemb(\var) \), de degré prescrit, de hauteur
contrôlée, et d'indice élevé le long de \( \divi \) dans \( \var \).  Nous
commencerons par définir la notion d'indice utilisée et préciser les
propriétés voulues et la stratégie de construction, puis nous établirons les
estimations de dimension et de hauteur nécessaires avant de conclure en
appliquant un lemme de \TS.


\subsection{Stratégie de construction de la forme auxiliaire}
\label{sec:siegel-plan}

Commençons par définir l'indice d'annulation d'une forme la long de
\( \divi \). (Une autre notion d'indice, en un point cette fois-ci, sera
définie de façon similaire au début de la
sous-section~\ref{sec:vojta-extrap-core}.) Pour tout \( \imp \in
  \N^{\puiss(\dimp+1)} \), notons
\nomuse {\wtsum*} [chap] {Somme pondérée servant à définir l'indice
  d'annulation le long de \( \divi \)}
\begin{equation}
  \wtsum*( \imp )
  =
  \frac {\imp[1][0]} {\wts[1]} + \dots
  + \frac {\imp[\puiss-1][0]} {\wts[\puiss-1]}
  + \frac {\imp[\puiss][0]} {\wts**(\puiss-1)}
  \pmm.
\end{equation}
Les poids dans cette somme sont essentiellement proportionnels à \( d' \), ce
qui sera crucial pour la section~\ref{sec:thm-prod}.
Pour toute forme \( H = \sum h_\imp \vmp^\imp \), on pose
\begin{equation} \label{e:def-pi-beta}
  \pi_\beta (H)
  =
  \sum_{\wtsum*( \imp ) < \beta}
  h_\imp \vmp^\imp
\end{equation}
et on défini l'indice par
\nomuse {\inda*} {Indice d'annulation le long de \( \divi \), défini
  par~\eqref{e:inda-def}}
\begin{equation} \label{e:inda-def}
  \inda* H
  =
  \inf \set{
    \beta \text{ tel que } \pi_\beta (H) \neq 0
  }
  \pmm,
\end{equation}
où la borne inférieure est en fait un minimum, sauf si \( H = 0 \), cas où
l'indice est infini.

\medskip

Au début de cette section, nous avons dit vouloir construire une forme sur \(
  \var \) provenant d'une forme sur \( \wemb(\var) \) ; plus précisément nous
allons construire une forme \( \faux \) sur \( \wemb(\var) \) et la tirer
localement en arrière sur \( \var \) en une famille de formes \( \faux'_\clmap \)
grâce aux différents morphismes \( \wemba* \) introduits à la
sous-section~\ref{sec:wemb}.

Revenons maintenant aux \( \faux'_\clmap \) : on ne peut brutalement exiger que
leur indice, tel que défini précédemment, soit élevé, car le nombre de
conditions serait trop élevé (géométriquement, il s'agirait d'exiger une
annulation le long de \( \divi \) dans \( \projd \) alors que les degrés de
libertés sont donnés par \( \var \) qui est de dimension plus petite). Nous
allons donc utiliser le morphisme \( \rfull \) introduit à la
sous-section~\ref{sec:rfull} et exiger que \( \faux* = \rfull(\faux'_\clmap)
\) ait un indice élevé, ce qui est loisible car le nombre de d'équations
définissant un indicé élevé dans \( \cdn[\vmp]^{C'''} \) est convenable (le
nombre de variables libre dans cet espace correspondant à la dimension de \(
  \var \)) comme le montrera la prochaine sous-section.

Concernant le degré, on reprend les multidegrés définis par~\eqref{e:d-dp-def}
et on rappelle que \( \delta \) désigne un grand entier tel que \( \delta
  \eps_0 \) soit aussi entier. On impose alors à \( \faux \) d'être
de degré \( \Dz \), de sorte que les \( \faux'_\clmap \) seront de degré \( \Di \)
et les \( \faux* \) de degré \( \Dir = \Di + o(\delta) \). Par ailleurs,
l'indice exigé sera \( \epsi \delta \), où \( \epsi \) est défini
par~\eqref{e:def-epsi} et la relation entre \( \epsz \) et \( \epsi \)
permettra d'ajuster l'exposant de \bsc{Dirichlet} du système auquel on
appliquera le lemme de \TS.

Choisissons alors dans \( \cdn[\vmp, \vmpi]_\Dz \) un supplémentaire de \(
  (\idealim)_\Dz \) engendré par des monômes, que l'on notera \( \fspace \) et
dans lequel on cherchera \( \faux \).  L'entier \( \delta \) sera choisi assez
grand pour que la dimension de \( \fspace \) coïncide avec la valeur du
polynôme de \bsc{Hilbert} de \( \idealim \) en \( \Dz \).

Reformulons l'objectif en introduisant un morphisme \( \sigma \) défini par
\begin{align} \label{e:def-ftarget}
  \sigma \colon \fspace
  & \to
  \ftarget =
  \bigoplus_{\clmap \in \clmaps^{\puiss-1}} \bigl(
    \cdn[\vmp]^{C'''}_\Dir
    \cap \vect( (\vmp^\imp)_{\wtsum*(\imp) < \epsi\delta} )
  \bigr)
  \\
  \faux
  & \mapsto
  \bigl(
    \pi_{\epsi\delta} \circ \rfull^{\Di} \circ \wemba*(\faux)
  \bigr)_\clmap
\end{align}
de sorte qu'il s'agit en fait de trouver une forme non nulle dans le noyau de
\( \sigma \). Pour procéder nous devons donc :
\begin{enumerate}
  \item Estimer les dimensions des espaces de départ et d'arrivée de \( \sigma
    \) et s'assurer que la seconde est plus petite que la première ;
  \item Estimer la hauteur de la matrice de \( \sigma \) dans les bases
    monomiales évidentes.
\end{enumerate}
Ces estimations font l'objet des deux sous-sections suivantes.  Comme \(
  \delta \) est arbitrairement grand devant les autres paramètres, on
n'explicitera à chaque fois que le terme de plus haut degré en \( \delta \) ;
ainsi lorsqu'on utilisera la notation \( o(\truc) \) ou \( \sim \), il s'agira
d'équivalents quand \( \delta \) tend vers l'infini.


\subsection{Deux calculs de dimension} \label{sec:comp-dim}

La dimension de $\fspace$ est donnée par le théorème de
\bsc{Hilbert} multihomogène dès qu'on connaît les différents multidegrés de
$\wemb(\var)$. Il est \lat{a priori} difficile de tous les calculer, mais
comme il suffit en fait de minorer la dimension, le lemme suivant nous donne
tout ce qu'on aura besoin de savoir sur le degré.

\begin{lem}
  Avec les notations précédentes, on a
  \begin{equation}
    \deg_{(0, \dots, 0, \vdim[\puiss]; \vdim[1], \dots, \vdim[\puiss-1])}
    \bigl( \wemb(\var) \bigr)
    =
    \vdeg[\puiss]
    \prod_{\fct=1}^{\puiss-1}
    \vdeg[\fcti] \wts[\fcti]
    \pmm.
  \end{equation}
\end{lem}

\begin{proof}
  Ce degré est donné par le cardinal de l'intersection de $\wemb(\var)$ avec
  des hyperplans génériques choisis de la façon suivante : $\vdim[\puiss]$
  provenant du $\puiss$-ième facteur $\projd$, et $\vdim*$ hyperplans
  provenant du facteur $\puiss + \fct$ pour $\fct \in \set{1, \dots,
    \puiss-1}$.

  On commence par choisir les hyperplans sur le facteur $\puiss$ : on remarque
  qu'ils se remontent par $\wemb$ en des hyperplans sur le dernier facteur de
  l'espace de départ $(\projd)^\puiss$. Ainsi, couper $\wemb(\var)$ par ces
  hyperplans revient à imposer à $\point_\puiss$ de parcourir un ensemble de
  cardinal $\vdeg[\puiss]$.

  Fixons maintenant un point $p'$ dans cet ensemble et notons $p = \wt** p'$.
  On constate que $\wemb(\var) \cap \zeros{\vmp[\puiss] = p'}$ coïncide avec
  l'image de
  \begin{align}
    \wemb[\wt, p']'
    \colon
    \var[1] \times \dots \times \var[\puiss-1] \times \set{p'}
    & \to
    \va^{2\puiss-1}
    \\
    (\point_1, \dots, \point_{\puiss-1}, p')
    & \mapsto
    (\point_1, \dots, \point_{\puiss-1}, p';
    \wt[1] \point_1 - p,
    \dots,
    \wt[\puiss-1] \point_{\puiss-1} - p)
  \end{align}
  qui est le produit d'un point par des variétés de la forme
  $\wemb[\wt*, p]''(\var*)$ pour $\fct$ variant de $1$ à $\puiss-1$ en
  notant
  \begin{align}
    \wemb[\wt*, p]''
    \colon
    \var*
    & \to
    \va^2
    \\
    \point
    & \mapsto
    (\point, \wt* \point - p)
  \end{align}
  Il suffit donc de calculer le degré de ces variétés. La translation par $p$
  n'ayant pas d'influence sur le degré, il suffit de regarder l'action de la
  multiplication par $\wt*$. Or, celle-ci pouvant être représentée
  globalement par des formes de degré $\wts*$, en tirant en arrière par
  $\wemb[\wt*, p]''(\var*)$ une famille de $\vdim*$ hyperplans
  génériques sur le second facteur, on obtient des hypersurfaces génériques de
  degré $\wts*$ qui coupent donc $\var*$ en $\vdeg*
  \wt*^{2\vdim*}$ points.

  Le résultat suit en prenant le produit.
\end{proof}

\begin{lem} \label{l:dim-fspace}
  Avec les notations précédentes, on a
  \begin{align}
    \dim \fspace
    \ge
    \frac{ \epsz^{\vdim[\puiss]}
      \prod\fctrange \vdeg* \, \wt* ^{2\vdim*}
      }{ \prod\fctrange \vdim* ! }
    \delta^\vdim
    + o( \delta^\vdim )
    \pmm.
  \end{align}
\end{lem}

\begin{proof}
  Il suffit d'appliquer le théorème de \bsc{Hilbert} multihomogène en
  utilisant le lemme précédent, car une somme de nombre positifs est minorée
  par chacun de ses termes. Il vient
  \begin{align}
    \dim \fspace = \dim \ringa{\wemb(\var)}
    & =
    \Biggl(
    \sum_{\substack{ t \in \N^{2\puiss-1} \\ \vlg t = \vlg u }}
    \deg_t \wemb(\var) \frac{ d^t }{ t! }
    \Biggr)
    \delta^\vdim
    + o( \delta^\vdim )
    \\
    & \ge
    \deg_{(0, \dots, 0, \vdim[\puiss]; \vdim[1], \dots, \vdim[\puiss-1])}
    \bigl( \wemb(\var) \bigr)
    \cdot
    \frac { (\epsz \wts**)^{\vdim[\puiss]} }{ \prod\fctrange \vdim* ! }
    \, \delta^\vdim
    + o( \delta^\vdim )
  \end{align}
  par définition de \( d \), voir~\eqref{e:d-dp-def}. On achève la preuve en
  combinant avec le résultat du lemme précédent.
\end{proof}

Il reste à majorer la dimension de \( \ftarget \).  On introduit à cet effet
l'ensemble suivant :
\begin{equation} \label{e:stairs-c3}
  \begin{split}
    \stairs
    & =
    \Biggl\{
      ( \imp[1], \dots, \imp[\puiss] )
      \in
      \prod\fctrange \bigl(
        \N^{\vdim* + 1}
        \times \{ 0, \dots, \vdeg* - 1 \}
        \times \{ 0 \}^{\dimp - \vdim* - 1}
      \bigr)
      \\ & \qquad
      \text{tel que }
      \wtsum*(\lambda) < \delta \epsi
      \text{ et }
      \lgr{\imp*}
      = \Dir* \quad \forall \fct
    \Biggr\}
    \pmm,
  \end{split}
\end{equation}
dont il s'agit de calculer le cardinal. En effet, la famille \(
  (\vmp^\imp)_{\imp \in \stairs} \) forme une base de
\(
  \cdn[\vmp]^{C'''}_\Dir
  \cap \vect( (\vmp^\imp)_{\wtsum*(\imp) < \epsi\delta} )
\)
et \( \ftarget \) est somme directe de tels espaces.

\begin{lem}
  Avec les notations précédentes,
  \begin{align}
    \card \stairs
    & \le
    \bigl(\delta^\vdim + o(\delta^\vdim)\bigr)
    \prod\fctrange \vdeg* \wt*^{2\vdim*}
    \cdot
    \frac {
      \epsi^\puiss
      \, \puiss^{\vdim[\puiss]}
      \, 3^{\vdim - \puiss}
      }{
      \puiss!
      \, \prod\fctrange (\vdim* - 1)!
      }
  \end{align}
\end{lem}

\begin{proof}
  Pour choisir un indice \( \imp \) dans \( \stairs \), on peut commencer par
  choisir \( \imp*[\vdim*] \) entre \( 0 \) et \( \vdeg* \) pour tout \( \fct
  \), ce qui représente \( \prod\fctrange \vdeg* \) possibilités.

  On peut ensuite choisir \( \imp[1][0], \dots \imp[\puiss][0] \)
  sujets à la seule condition
  \begin{equation}
    \wtsum*(\lambda) \le \delta \epsi \pmm.
  \end{equation}
  Le lemme~2.14.5 de \cite{farhith} donne le nombre de choix possibles, qui
  est
  \begin{equation}
    \frac {\prodwt} {\puiss !} (\delta\epsi)^\puiss
    + o(\delta^\puiss)
    \pmm.
  \end{equation}

  Il reste alors à choisir pour tout \( \fct \) un élément de l'ensemble
  \begin{equation}
    \left\{
      (\imp*[1],  \dots, \imp*[\vdim*])
      \in \N ^{\vdim*}
      \text{ tel que }
      \sum_{j=1}^{\vdim*} \imp*[j]
      =
      \Dir* - \imp*[0] - \imp*[\vdim*]
    \right\}
    \pmm,
  \end{equation}
  qui est de cardinal
  \begin{align}
    \binom {
      \Dir* - \imp*[0] - \imp*[\vdim*] + \vdim* - 1
      }{
      \vdim* - 1
      }
    & \le
    \binom {
      \Dir* + \vdim* - 1
      }{
      \vdim* - 1
      }
    \\
    & \le
    \frac {(d'_\fct)^{\vdim* - 1}} {(\vdim* - 1)!} \delta^{\vdim* - 1}
    + o( \delta^{\vdim* - 1} )
  \end{align}
  On prend alors le produit :
  \begin{align}
    \card \stairs
    & \le
    \frac {\prodwt} {\puiss !} (\delta\epsi)^\puiss
    \cdot \prod\fctrange
    \frac {(d'_\fct)^{\vdim* - 1}} {(\vdim* - 1)!}
    \vdeg* \delta^{\vdim* - 1}
    + o( \delta^\vdim )
    \\ & \le
    \bigl(\delta^\vdim + o(\delta^\vdim)\bigr)
    \prod\fctrange \vdeg* \wt*^{2\vdim*}
    \cdot
    \frac {
      \epsi^\puiss (\puiss-1)
      (2\puiss - 2 + \epsz) ^{\vdim[\puiss]-1}
      (2 + \epsz) ^{\vdim - \puiss - \vdim[\puiss] + 1}
      }{
      \puiss! \prod\fctrange (\vdim* - 1)!
      }
  \end{align}
  d'après la définition~\eqref{e:d-dp-def} de \( d' \). Le résultat annoncé
  suit en remarquant que \( 2 + \epsz \le 3 \) et \( 2\puiss - 2 + \epsz \le
    3\puiss \).
\end{proof}

Compte tenu de la définition~\eqref{e:def-ftarget} de \( \ftarget \) et du fait
que \( \card \clmaps^{\puiss-1} = (\dimp + 1)^{\puiss-1} \) d'après le
lemme~\ref{l:hclab}, on a immédiatement
\begin{equation}
  \dim \ftarget
  \le
  \bigl(\delta^\vdim + o(\delta^\vdim)\bigr)
  \prod\fctrange \vdeg* \wt*^{2\vdim*}
  \cdot
  \frac {
    \epsi^\puiss
    \, (\dimp + 1)^{\puiss-1}
    \, \puiss^{\vdim[\puiss]}
    \, 3^{\vdim - \puiss}
  }{
    \puiss!
    \, \prod\fctrange (\vdim* - 1)!
  }
  \pmm.
\end{equation}
En combinant ce résultat avec le lemme~\ref{l:dim-fspace} et en observant que
\( \vdim* \le \genre \), on a
\begin{align}
  \frac {\dim \ftarget} {\dim \fspace}
  & \le
  \frac {
    \epsi^\puiss
    \, (\dimp + 1)^{\puiss-1}
    \, \puiss^{\vdim[\puiss]}
    \, 3^{\vdim - \puiss}
    \, \prod\fctrange \vdim*
  }{
    \epsz^{\vdim[\puiss]}
    \, \puiss!
  }
  + o(1)
  \\ & \le
  \frac {
    \epsi^\puiss
    \, (\dimp + 1)^{\puiss-1}
    \, \puiss ^{\genre}
    \, 3 ^{\puiss (\genre-1)}
    \, \genre^\puiss
  }{
    \epsz^\genre
    \, \puiss!
  }
  + o(1)
  \pmm.
\end{align}
On utilise alors la définition \eqref{e:def-epsz} de \( \epsz \) et la
minoration classique \( \puiss! \ge \puiss^\puiss \expb^{-\puiss} \) pour
écrire
\begin{align}
  \frac {\dim \ftarget} {\dim \fspace}
  & \le
  \frac {
    \epsi^{\puiss-\genre}
    \, (\dimp + 1)^{\puiss-1}
    \, \puiss ^{\genre}
    \, 3 ^{\puiss (\genre-1)}
    \, \genre^\puiss
    \, (33\puiss)^\genre
    \, \expb^\puiss
  }{
    \eps^\genre
    \, \puiss^\puiss
  }
  + o(1)
  \\ & \le
  \frac 1 { \puiss^{\puiss-\genre} }
  \cdot
  \frac {
    \epsi^{\puiss-\genre}
    \, (\dimp + 1)^{\puiss-1}
    \, 3 ^{\puiss \genre}
    \, \genre^\puiss
    \, (33\puiss)^\genre
  }{
    \eps^\genre
  }
  + o(1)
  \\ & \le
  \frac 1 { \puiss^{\puiss-\genre} }
  \cdot
  \frac {
    \epsi^{\puiss-\genre}
    \, (\dimp + 1)^{\puiss+1}
    \, (33\puiss)^\genre
  }{
    \eps^\genre
  }
  + o(1)
\end{align}
om l'on a utilisé le fait que \( \dimp + 1 \ge 4^\genre \).  On remarque que
la définition \eqref{e:def-epsi} de \( \epsi \) signifie précisément que le
deuxième facteur vaut \( 1 \) ; en utilisant alors le fait que \( \puiss \ge
  \genre + 1 \ge 2 \), il vient enfin
\begin{equation} \label{e:good-codim}
  \frac {\dim \ftarget} {\dim \fspace}
  \le
  \frac12
  + o(1)
  \pmm.
\end{equation}


\subsection{Hauteur du système}

Il s'agit d'estimer la hauteur de la matrice (dans les bases monomiales
canoniques) \( M_\sigma \) de l'application \( \sigma \) définie
par~\eqref{e:def-ftarget}. Nous utiliserons ici la norme \( \nv\infty\truc \)
en chaque place, de sorte que la norme de la matrice de \( \sigma \) sera
majorée par tout majorant commun des normes des matrices des applications
\( \pi_{\epsi\delta} \circ \rfull^{\Di} \circ \wemba* \).

D'après la définition~\eqref{e:def-pi-beta} de \( \pi_{\epsi\delta} \) il est
clair que cette application ne peut que faire décroître la norme. Par
ailleurs, on se souvient que, si \( M_1 \) et \( M_2 \) sont deux matrices, la
hauteur \( \Hautl[\infty]\) de leur produit est majorée par \(
  \hautl[\infty]{M_1} + \max(\hautl[1]{c}) \) où \( c \) parcourt les colonnes
de \( M_2 \). Il ne reste plus qu'à combiner les estimations des
lemmes~\ref{l:rfull} et~\ref{l:hmat-wemba} pour obtenir
\begin{align}
  \hautl[\infty]{M_\sigma}
  & \le
  2 \delta \wts[1]
  \bigl(
    \hlclab + 2\ln((\dimp+1)!) + 2\dimp \ln B
  \bigr)
  \\ & \qquad +
  \sum\fctrange 2\Di* \left(
    \hautl[1]{ \varfc* }
    + (B(\genre+1) + 1) \ln(2B)
  \right)
  + o(\delta)
  \pmm.
\end{align}
On majore alors \( \hlclab + 2\ln((\dimp+1)!) \) par \( 3\cst{vs-ht} \)
d'après~\eqref{e:cst-vs-ht} ; on observe ensuite que \( \sum\fctrange 2\Di*
  \le 12 \wts[1] \) d'après le lemme~\ref{l:sum-wts}, la
définition~\eqref{e:d-dp-def} de \( d' \) et le fait que \( \epsz \le 1 \). On
majore ensuite \( 2\dimp \ln B + 12 (B(\genre+1) + 1) \ln(2B) \) assez
largement par \( 12 \dimp B^2 = 12 \dimp \cst{vs-deg-prod}^{-2}
  \Lambda^{2f(\vdim)} \). Enfin, on utilise~\eqref{e:varset-ht} pour conclure
que
\begin{equation} \label{e:hmat-sigma}
  \hautl[\infty]{M_\sigma}
  \le
  2 \delta \wts[1] \Lambda^{2f(\vdim)} \left(
    4 \cst{vs-ht} + 12 \dimp \cst{vs-deg-prod}^{-2}
  \right) + o(\delta)
  \pmm.
\end{equation}


\subsection{Construction finale de la forme auxiliaire}

Commençons par rappeler la version que nous utiliserons du classique lemme de
\TS.

\begin{fact} \label{f:siegel}
  Pour toute matrice \( M \) de dimensions \( p \times q \) à coefficients
  dans un corps de nombres \( \cdn \) avec \( p < q \), il existe un vecteur
  \( x \in \cdn^q \) non nul tel que \( M x = 0 \) et satisfaisant
  \begin{equation}
    \hautl[\infty] x
    \le
    \frac p{q-p} \bigl( \hautl[\infty] M + \ln q \bigr)
    + \frac q{q-p} c_\cdn
    \pmm,
  \end{equation}
  où \( c_\cdn \) est une constante ne dépendant que de \( \cdn \).
\end{fact}

\begin{proof}
  C'est le lemme de \bsc{Siegel} de \bsc{Bombieri} et \bsc{Vaaler} tel
  qu'énoncé dans~\cite{bogf}. Notons que des versions plus précises
  existent, mais les améliorations ne concernent que des termes qui sont
  négligeables dans notre situation et n'ont donc pas d'intérêt ici.
\end{proof}

Nous sommes maintenant en mesure de construire la forme auxiliaire.

\begin{prop} \label{p:build-aux}
  Sous les hypothèses et notations précédentes et si
  \( \delta \) est assez grand, il existe une forme non nulle \( \faux \in
    \ker \sigma \) telle que
  \begin{align}
    \deg \faux
    & = \Dz
    = \bigl(
      \epsz \wts[1] \delta,
      \dots,
      \epsz \wts[\puiss] \delta,
      \delta, \dots, \delta
    \bigr)
    \\
    \hautl[\infty] \faux
    & \le
    2 \delta \wts[1] \Lambda^{2f(\vdim)} \left(
      4 \cst{vs-ht} + 12 \dimp \cst{vs-deg-prod}^{-2}
    \right) + o(\delta)
    \pmm.
  \end{align}
\end{prop}

\begin{proof}
  On applique le fait~\ref{f:siegel} à la matrice \( M_\sigma \) introduite à
  la sous-section précédente, de dimensions \( p = \dim \ftarget \) et \( q =
    \dim \fspace \) estimées précédemment.  La relation~\eqref{e:good-codim}
  se lit alors \( \frac pq \le \frac12 + o(1) \), ce qui implique directement
  que \( p < q \) et que
  \begin{equation}
    \frac p {q-p}
    \le
    \frac 1 {\frac qp - 1}
    \le
    \frac 1 {1 - o(1)}
    \le
    1 + o(1)
    \pmm.
  \end{equation}
  De même, \( \frac q {q-p} c_\cdn = o(\delta) \) ;
  en remarquant de plus que \( \ln q = o(\delta) \), on voit qu'il existe une
  forme \( \faux \) satisfaisant aux conditions de l'énoncé mais de hauteur majorée
  par \( \hautl[\infty]{M_\sigma} + o(\delta) \).  Il ne reste plus qu'à
  invoquer~\eqref{e:hmat-sigma} pour conclure au résultat annoncé.
\end{proof}

Remarquons que dans la majoration de hauteur on peut en fait choisir la
hauteur utilisée, comme le montre le scolie suivant.

\begin{scho} \label{s:h1-aux}
  On a \( \hautl[1] \faux \le \hautl[\infty] \faux + o(\delta) \).
\end{scho}

\begin{proof}
  La différence entre ces deux hauteurs est majorée par le logarithme du
  nombre de coefficients de \( \faux \), qui est au plus
  \begin{equation}
    \prod_{\fct=1}^{2\puiss-1}
    \binom{\delta d_\fct + \dimp}{\dimp}
    \le
    \prod_{\fct=1}^{2\puiss-1}
    \frac{ (\delta d_\fct + \dimp)^\dimp }{\dimp!}
    \pmm.
  \end{equation}
  Cette quantité est un polynôme en \( \delta \), son logarithme est donc
  négligeable devant ce dernier.
\end{proof}

En pratique, nous utiliserons beaucoup les formes \( \faux* = \rfull \circ
  \wemba*(\faux) \), résumons donc ici leurs principales propriétés.

\begin{scho} \label{s:aux-co}
  Pour toute carte \( \clmap \in \clmaps^{\puiss-1} \), la forme \( \faux* \)
  satisfait :
  \begin{enumthm}
    \item \( \faux* = R \cdot \faux(\vmp, \wembclm(\vmp)) \mod \varida
      \) ;
    \item \( \inda* \faux* \ge \epsi \delta \) ;
    \item \( \deg \faux* = \Dir \) avec
      \(
        d' = \bigl(
          \wts[1] (2 + \epsz),
          \dots,
          \wts[\puiss-1] (2 + \epsz),
          \wts[\puiss] (2\puiss - 2 + \epsz)
        \bigr)
      \) ;
    \item \(
        \nv1{\faux*}
        \le
        \nv1{\faux}
        \prod\fctrange \left(
          \nv1{ \varfc* }
          2 B^{(B(\genre+1) + 1)\dv}
        \right) ^{ 2\Di* }
        \prod\fctirange
        \nv1{ \wembclm* }^\delta
        \cdot \expb^{o(\delta)}
      \).
  \end{enumthm}
\end{scho}

\begin{proof}
  Le premier point est une conséquence directe de la définition de \( \faux*
  \), du lemme~\ref{l:rfull} et de la définition~\ref{e:def-wemba} de \(
    \wemba* \) ; le deuxième est donné par la construction et
  le troisième découle du lemme~\ref{l:deg-wemba} et des
  définitions~\ref{e:d-dp-def} de \( d \) et \( d' \).  Pour le quatrième
  point, on utilise le lemme~\ref{l:rfull} ainsi que des propriétés classiques
  de la norme.
\end{proof}



\section{Extrapolation} \label{sec:vojta-extrap}

Le but de cette section est de montrer que la fonction auxiliaire que nous
venons de construire s'annulle avec un indice élevé en \( \ex \), pour une
définition de l'indice que nous préciserons.

Auparavant, nous aurons besoin de représenter les dérivées de certaines
fonctions rationnelles sur \( \var \) par des fractions rationnelles de
dénominateur explicite dont on contrôle le degré et la hauteur du numérateur.
Pour commencer, nous effectuons ce travail sur une variété projective
quelconque (plongée de façon adaptée) avant de passer à une variété produit.


\subsection{Estimation de dérivées} \label{sec:vojta-param}

\worknote{Tout réécrire en remplaçant \( \anydim \) par \( 0 \) et, dans les
  trucs d'ordre ou indice, \( 0 \) par un indice quelconque. Pfiou\dots}
Soit \( \anyvar \) une variété projective de dimension \( \anydim \), plongée
de façon adaptée dans un espace projectif \( \projd \), de degré \( \anydeg \)
dans ce plongement. Alors \( \cdn(\anyvar) \) est une extension finie
de\footnote{\label{fn:coord-deshom}Dans toute cette section on prendra \(
    \vp[\anydim] \) (puis \( \vmp*[\vdim*] \) en multiprojectif) comme
  variable de déshomogénéisation pour simplifier les notations, mais il est
  clair qu'on pourrait utiliser n'importe quel indice entre \( 0 \)  et \(
    \anydim \) (et \( \vdim* \) en multiprojectif).  Les morphismes et formes
  obtenus dépendront bien sûr de ce choix, mais pas leurs propriétés.}
\begin{equation}
  \cdn\Big(
    \frac{ \vp[0]           }{ \vp[\anydim] }, \dots,
    \frac{ \vp[\anydim-1]   }{ \vp[\anydim] }
  \Big)
\end{equation}
dont \( \frac{ \vp[\anydim+1] }{ \vp[\anydim] } \) est un élément primitif.
Sur ce dernier corps, on dispose des dérivations standard définies par
\(
  \diff_\ind \frac{ \vp[\indi] }{ \vp[\anydim] } = \delta_\ind^\indi
\)
qui forment une base de l'espace des dérivations, et s'étendent de façon
unique à \( \cdn(\anyvar) \) pour former une base de son espace de
dérivations. Les propriétés classiques des dérivations montrent alors que
l'application
\begin{equation}
  \begin{aligned}
    \pmor \colon \cdn(Z)
    & \to \cdn(Z)\series\psp
    \\
    f
    & \mapsto
    \sum_{\derp \in \N^\anydim} \der[\derp] \!f \, \psp^\derp
  \end{aligned}
  \qquad \text{où }
  \der[\derp]
  =
  \frac1{\derp!}
  \prod_{\ind = 0}^\anydim \diff_\ind^{\derp*}
  \pmm.
\end{equation}
est un morphisme de \( \cdn \)-algèbres.

Il est intéressant de pouvoir le représenter par un morphisme \( \pmor* \)
faisant commuter le diagramme
\begin{equation} \label{e:pmor}
  \begin{aligned}[c]
    \xymatrix{
      \cdn[\vp]_{(\ideal\anyvar)}               \ar[d]^\pi \ar@{.>}[r]^{\pmor*}
      & \cdn[\vp]_{(\ideal\anyvar)} \series\psp \ar[d]^\pi
      \\ \cdn(\anyvar)                                     \ar[r]^{\pmor}
      & \cdn(\anyvar)\series\psp
    }
  \end{aligned}
\end{equation}
où les flèches verticales sont les projections canoniques. Posons
\( \pmor* = \sum_\derp \psp^\derp \pdiff*^\derp \) où les \( \pdiff*^\derp \)
sont des applications linéaires sur \( \cdn[X]_{(\ideal\anyvar)} \) vérifiant
la règle de \bsc{Leibniz} pour les dérivées divisées. Pour chaque fraction
rationnelle \( f = G/H \), nous exhiberons (au moins) un polynôme \( R_\derp
\), dépendant éventuellement de la représentation \( G/H \) choisie, tel que
\( R_\derp \cdot \pdiff*^\derp(f) \) soit un polynôme de degré et normes
locales contrôlées.

En pratique, nous construirons en fait deux morphismes, \( \pmor*_0 \) et \(
  \pmor*_1 \), et deux familles d'applications \( \pdiff^\derp \) telles que
l'on contrôlera la norme \( \place \)-adique de \( R_\derp \cdot \pdiff^\derp
\), où l'on rappelle que \( \dv \) vaut \( 1 \) si \( \place \) est
archimédienne et \( 0 \) sinon.

Pour définir \( \pmor*_\dv \), il suffit de définir les images des différents
\( \vp* / \vp[\anydim] \). En effet, ceci donne un morphisme de \( \cdn[ \vp /
  \vp[\anydim] ] \) dans \( \cdn[X]_{(\ideal\anyvar)} \series\psp \). Si l'on
impose de plus que \( \pdiff^0 \) soit l'identité, on constate que l'image du
complémentaire de \( \ideal\anyvar \) ne contient que des séries inversibles
(car leur terme constant l'est), ce qui permet d'étendre le morphisme à \(
  \cdn[X]_{(\ideal\anyvar)} \).

Pour tout \( \ind \in \set{0, \dots, \anydim-1} \), on peut poser \(
  \pmor*_\dv(\vp*/\vp[\anydim]) = \vp*/\vp[\anydim] + \psp* \). Pour \( \ind >
  \anydim \), nous utiliserons le lemme suivant, que l'on énonce dans un cadre
affine avant de l'appliquer à notre situation projective.

\begin{lem} \label{l:param-aff}
  Soient \( \anyvp[1], \dots, \anyvp[\anydim], Y \) des variables et \( L \)
  une algébrique extension finie de \( \cdn(\anyvp[1], \dots, \anyvp[\anydim])
  \). On fixe \( y \) un élément de \( L \) ; on note \( \pi \) le
  morphisme de \( \cdn[\anyvp[1], \dots, \anyvp[\anydim], Y ] \) dans \( L \)
  qui laisse stable les \( \anyvp* \) et envoie \( Y \) sur \( y \). Soit \(
    \Pi \in \ker \pi \) tel que \( \pden = \diff_Y \Pi \notin \ker \pi \).

  On considère les dérivations standard \( \diff_\ind \) sur \( \cdn[
    \anyvp[1], \dots, \anyvp[\anydim] ] \) ainsi que leurs extensions à \( L
  \), et on note \( \der[\derp] = \frac1{\derp!} \prod_{\ind = 0}^\anydim
    \diff_\ind^{\derp*} \) les dérivations réduites.  Il existe des polynômes
  \( P_\dv^\derp \), pour
  \( \derp \in \N^{\anydim} \minusset0 \), tels que :
  \begin{enumthm}
    \item \( \der[\derp] y
        = \pi\left(
          \frac{ P_\dv^\derp }{ \pden^{2\lgr\dermp - 1} }
        \right)
      \) ;
    \item \( \deg P_\dv^\derp \le (\deg \Pi - 1) (2\lgr\derp - 1) \) ;
    \item \( \nv1{ P_\dv^\derp }
        \le \nv1 \poldep ^{2\lgr\derp - 1}
        \cdot \left(
          (4\anydim)^{\lgr\derp -1} (\deg\Pi)^{3\lgr\derp -2}
        \right)^\dv \).
  \end{enumthm}
\end{lem}

\begin{proof}
  Il s'agit en fait de compléter\footnote{On contrôle en fait le développement
    autour d'un point générique, alors que \bsc{Rémond} l'étudie en un point
    fixé.} la preuve du lemme~6.1 de \cite{remivds}, en
  utilisant aux places archimédiennes une généralisation de
  \cite[relation~2.3.1, p.~63]{farhith}.

  On va construire $P_0^\derp$ et $P_1^\derp$ indépendamment par récurrence
  sur la longueur de $\derp$, en partant à chaque fois de $P_\dv^\derp = -
  \diff_{\ind_0} \Pi$ quand $\derp_{\ind_0} = 1$ et $\derp_\ind = 0$ sinon
  (cas $\lgr\derp = 1$), car ce choix convient. Pour la suite, on fixe un
  $\dv$, un $\derp$ de longueur au moins $2$, et on suppose qu'on a choisi un
  $P_\dv^{\derp'}$ convenable pour chaque $\derp'$ de longueur strictement
  inférieure à celle de $\derp$.

  On commence par le cas ultramétrique et on note donc provisoirement $P^\derp
  = P_0^\derp$ pour alléger. Les polynômes recherchés sont caractérisés par la
  relation
  \begin{equation}
    \Pi \left(
      \anyvp[1] + \psp[1], \dots, \anyvp[\anydim] + \psp[\anydim],
      Y + \sum_{ \derp \in \N^\anydim \minusset 0 }
      \frac {P^\derp} {\pden^{2\lgr\derp -1}} \psp^\derp
    \right)
    = 0 \mod (\Pi)
    \pmm.
  \end{equation}
  On remplace alors \( \Pi \) par son développement de \bsc{Taylor}, pour
  obtenir les égalités suivantes modulo \( \Pi \) :
  \begin{align}
    0
    & =
    \sum_{(\ip, \mu) \in \N^{\anydim+1}}
    \der[\ip, \mu] \Pi
    \cdot \psp^\ip
    \cdot \left(
      \sum_{ \derp \in \N^\anydim \minusset 0 }
      \frac {P^\derp} {\pden^{2\lgr\derp - 1}} \psp^\derp
    \right)^\mu
    \\
    & =
    \sum_{\substack{ (\ip, \mu) \in \N^{\anydim+1} \minusset{(0, 0)}
        \\ \gmp\nu* \in \N^\anydim \minusset 0 }}
    \left(
      \der[\ip, \mu] \Pi
      \cdot \prod_{\fct = 1}^\mu
      \frac {P^{\gmp\nu*}} {\pden^{2\lgr{\gmp\nu*} - 1}}
    \right)
    \psp^{\sum_\fct \gmp\nu* + \ip}
    \\
    & =
    \sum_{\derp \in \N^\anydim \minusset 0}
    \Biggl(
    \frac {P^\derp} {\pden^{2\lgr\derp - 2}}
    + \sum_{\substack{
        (\ip, \mu) \in \N^{\anydim+1} \minusset{(0, 0), (0, 1)}
        \\ \gmp\nu* \in \N^\anydim \minusset 0
        \\ \sum_\fct \gmp\nu* + \ip = \derp }}
    \der[\ip, \mu] \Pi
    \cdot \prod_{\fct = 1}^\mu
    \frac {P^{\gmp\nu*}} {\pden^{2\lgr{\gmp\nu*} - 1}}
    \Biggr)
    \psp^\derp
    \pmm,
  \end{align}
  où l'on a noté \( (\ip, \mu) = (\ip[1], \dots, \ip[\anydim], \mu) \).
  Il suffit donc de définir \( P^\derp \) par la relation de récurrence
  \begin{equation}
    - P^\derp
    =
    \sum_{\substack{
        (\ip, \mu) \in \N^{\anydim+1} \minusset{(0, 0), (0, 1)}
        \\ \gmp\nu* \in \N^\anydim \minusset 0
        \\ \sum_\fct \gmp\nu* + \ip = \derp }}
    \der[\ip, \mu] \Pi
    \cdot \pden^{2\lgr\derp - 2}
    \cdot \prod_{\fct = 1}^\mu
    \frac {P^{\gmp\nu*}} {\pden^{2\lgr{\gmp\nu*} - 1}}
  \end{equation}
  qui consiste à imposer que chaque terme de la série précédente soit nul, ce
  qui assure bien sa nullité modulo \( \Pi \).

  On majore alors le degré de \( P^\derp \) par récurrence :
  \begin{align}
    \deg P^\derp
    & \le
    \deg\Pi - \lgr\ip - \mu + (\deg\Pi - 1) (2\lgr\derp - 2)
    \\ & \le
    1 - \lgr\ip - \mu + (\deg\Pi - 1) (2\lgr\derp - 1)
    \\ & \le
    (\deg\Pi - 1) (2\lgr\derp - 1)
    \pmm,
  \end{align}
  car \( \ip \) et \( \mu \) ne sont pas simultanément nuls.  La majoration de
  norme locale est immédiate par analogie avec le degré vu les propriétés de
  la norme aux places ultramétriques.

  Considérons maintenant le cas archimédien (désormais \( P^\derp = P^\derp_1
  \) pour alléger). On utilise la relation de récurrence suivante, établie
  dans la démonstration de \cite[lemme~6.1]{remivds}, avec \( Q_\derp =
    P^\derp \cdot \derp! \) et où, rappelons-le, \( \derp' \) est tel que
  \( \derp[\ind_0] = \derp[\ind_0]' + 1 \) et \( \derp* = \derp*' \) sinon :
  \begin{equation}
    Q_\derp
    =
    \pden^2 \, \diff_{\ind_0} Q_{\derp'}
    - \pden \, \diff_{\ind_0} P \, \diff_{\anydim+1} Q_{\derp'}
    + (2\lgr{\derp'} - 1)
    (\diff_{\ind_0} P \, \diff_Y \pden - \pden \diff_{\ind_0} \pden)
    Q_{\derp'}
    \pmm.
  \end{equation}
  On en déduit immédiatement l'estimation de degré suivante :
  \begin{equation}
    \deg P^\derp
    \le 2 (\deg\Pi - 1) + \deg P^{\derp'}
    \le (\deg\Pi - 1) (2\lgr\derp - 1)
    \pmm.
  \end{equation}
  Pour la norme, on prouve que \(
    \nv1{Q_\derp}
    \le
    \nv1\Pi^{2\lgr\derp-1} 4^{\lgr\derp-1} (\deg\Pi)^{3\lgr\derp-2}
    (\lgr\derp - 1) !
  \)
  (ce qui implique le résultat annoncé vu que \( \binom{\lgr\derp}{\derp}
    \le \anydim^{\lgr\derp-1} \)) en exploitant la majoration de degré
  sous la forme \( \deg P^\derp \le 2\lgr\derp \deg\Pi \).
  \begin{align}
    \nv1{ Q_\derp }
    & \le
    2(\deg\Pi)^2 \cdot \deg Q_{\derp'} \cdot \nv1\Pi^2 \nv1{ Q_{\derp'} }
    + 2 (2\lgr{\derp'} - 1) (\deg\Pi)^3 \nv1\Pi^2
    \\ & \le
    \nv1\Pi^2 \nv1{ Q_{\derp'} } \cdot 4 (\deg\Pi)^{3\lgr{\derp'}}
    \\ & \le
    \nv1\Pi^{2\lgr\derp-1} \cdot 4^{\lgr\derp-1} (\deg\Pi)^{3\lgr\derp-2}
    (\lgr\derp - 1) !
    \qedhere
  \end{align}
\end{proof}

Nous sommes maintenant en mesure de définir les morphismes \( \pmor*_\dv \).

\begin{lem} \label{l:def-pmor}
  Dans les notations précédentes, on peut définir \( \pmor*_\dv \) faisant
  commuter le diagramme~\eqref{e:pmor} par :
  \begin{enumthm}
    \item \( \pmor*(\frac{ \vp* }{ \vp[\anydim] })
        = \frac{ \vp* }{ \vp[\anydim] } + \psp* \)
      pour tout \( \ind \in \set{0, \dots, \anydim-1} \) ;
    \item \( \pmor*(\frac{ \vp* }{ \vp[\anydim] })
        = \frac{ \vp* }{ \vp[\anydim] }
        + \sum_{\derp \neq 0}
        \frac{ P^\derp_{\ind, \dv} }{ U_\ind^{\lgr\derp} }
        \psp^\derp
      \)
  \end{enumthm}
  où les polynômes \( P^\derp_{\ind, \dv} \) et \(  U_\ind \) satisfont
  \begin{enumthm}
    \item \( \deg  U_\ind   =   2 (\anydeg - 1) \) ;
    \item \( \nv1{ U_\ind } \le \nv1{ \chow\anyvar }^2 4^{(\anydeg-1)\dv} \) ;
    \item \( \deg  P^\derp_{\ind, \dv}   =   2 (\anydeg - 1) \lgr\derp \) ;
    \item \( \nv1{ P^\derp_{\ind, \dv} } \le \left(
          \nv1{ \chow\anyvar }^2
          \bigl( 4^\anydeg \anydim \anydeg^{3} \bigr)^\dv
        \right)^{\lgr\derp} \).
  \end{enumthm}
\end{lem}

\begin{proof}
  Fixons un \( \ind > \anydim \) et notons \( \tilde\Pi = P_\ind(\anyvp[1],
    \dots, \anyvp[\anydim], 1, Y ) \), où \( P_\ind \) est donné par le
  fait~\ref{f:plong-adapt-dep}, puis \( \Pi \) une des dérivées successives de
  \( \tilde\Pi \) par rapport à la dernière variable, telle que \( \diff_Y \Pi
    \notin \ideal\anyvar \). On pose alors
  \begin{equation}
    \pdiff^\derp \left( \frac{ \vp* }{ \vp[\anydim] } \right)
    =
    \frac {
      P_\dv^\derp ( \vp[0] / \vp[\anydim], \dots,
      \vp[\anydim-1] / \vp[\anydim], \vp* / \vp[\anydim] )
    }{
      \pden ( \vp[0] / \vp[\anydim], \dots,
      \vp[\anydim-1] / \vp[\anydim], \vp* / \vp[\anydim] )
      ^{2\lgr\derp - 1}
    }
    \pmm,
  \end{equation}
  où les \( P_\dv^\derp \) sont donnés par le lemme précédent. Quitte à
  multiplier par une puissance convenable de \( \vp[\anydim] \), on peut
  supposer que le degré de \( \pden \) est exactement \( 2\anydeg - 1 \) et
  que celui de \( P_\dv^\derp \) est exactement \( (2\anydeg - 1) (2\lgr\derp
    - 1) \), de sorte que par homogénéité :
  \begin{equation}
    \pdiff^\derp \left( \frac{ \vp* }{ \vp[\anydim] } \right)
    =
    \frac {
      P_\dv^\derp ( \vp[0], \dots, \vp[\anydim-1], \vp* )
    }{
      \pden ( \vp[0], \dots, \vp[\anydim-1], \vp* )
      ^{2\lgr\derp - 1}
    }
    =
    \frac {
      P_\dv^\derp
    }{
      \pden
      ^{2\lgr\derp - 1}
    }
  \end{equation}
  On pose finalement\footnote{Ceci n'a pour but que de simplifier les calculs
    ultérieurs en rendant le degré exactement linéaire en \( \lgr\derp \).} \(
    P_{\ind, \dv}^\derp = \pden \cdot P_\dv^\derp \) et \( \pden_\ind =
    \pden^2 \). Les estimations de norme annoncées sont alors immédiates.
\end{proof}

Les morphismes \( \pmor*_\dv \) s'étendent à \( \cdn[X]_{(\ideal\anyvar)} \)
comme expliqué au deuxième paragraphe précédant le lemme~\ref{l:param-aff}.
Nous allons maintenant étudier (dénominateur, degré et hauteur des
coefficients) les images de certaines fonctions, en commençant par celles des
monômes en \( \vp / \vp[\anydim] \).

\begin{lem} \label{l:par-anyvar-mono}
  Soit \( \ip \in \N^{\dimp+1} \) et \( M_\ip = \prod\limits\indrange
    \bigl( \frac{\vp*}{\vp[\anydim]} \bigr)^{\ip*} \).  Posons \( \pden =
    \prod_{\ind = \anydim + 1}^{\dimp} \pden_\ind \) où les \( \pden_\ind \)
  sont donnés par le lemme précédent. Alors :
  \begin{enumthm}
    \item \( P^\derp_{\ip, \dv}
        = \pdiff^\derp(M_\ip)
        \cdot \vp[\anydim]^{\lgr\ip} \pden^{\lgr\derp} \) est un polynôme ;
    \item \( \deg P^\derp_{\ip, \dv}
        = \lgr\ip + 2 (\anydeg - 1) (\dimp - \anydim) \lgr\derp \) ;
    \item \( \nv1{ P^\derp_{\ip, \dv} }
        \le
        \nv1{ \chow\anyvar }^{2 (\dimp - \anydim) \lgr\derp}
        \left(
          \bigl( 4^{\anydeg+1} \anydim \anydeg^{3} \bigr) ^{
            (\dimp - \anydim) \lgr\derp }
          \cdot 2^{ \anydim \lgr\ip }
        \right)^\dv
      \).
  \end{enumthm}
  De plus, \( \ord_{\vp[0]} \bigl( P^\derp_{\ip, \dv} \bigr)
    \ge \ip[0]  - \derp[0] \).
\end{lem}

\begin{proof}
  \newcommand \indl {{ \gmp\nu[\ind, \indi_\ind] }}
  On utilise la règle de \bsc{Leibniz} :
  \begin{equation}
    \pdiff^\derp(M_\ip) =
    \sum_{\gmp\nu \in N}
    \prod\indrange
    \prod_{ \indi_\ind = 1 }^{ \ip* }
    \pdiff^\indl \biggl( \frac{ \vp* }{ \vp[\anydim] } \biggr)
    \pmm,
  \end{equation}
  où la somme est prise sur l'ensemble
  \begin{equation}
    N = \left\{
      \gmp\nu \in (\N^\anydim)^{\lgr\ip}
      \text{ tel que }
      \sum\indrange \sum_{\indi_\ind = 1}^{\ip*} \indl = \derp
    \right\}
    \pmm.
  \end{equation}

  On vérifie alors que
  \begin{equation}
    P^\derp_{\ip, \dv}
    =
    \sum_{\gmp\nu \in N}
    \prod\indrange
    \prod_{ \indi_\ind = 1 }^{ \ip* }
    \pdiff^\indl \biggl( \frac{ \vp* }{ \vp[\anydim] } \biggr)
    \pden_\ind^{\lgr\indl}
    \vp[\anydim]
  \end{equation}
  est bien un polynôme : pour chaque facteur de chaque terme, si \( \indl = 0
  \) alors \(
    \pdiff^\indl ( \frac{ \vp* }{ \vp[\anydim] } )
    \vp[\anydim]
  \) est un polynôme, sinon \(
    \pdiff^\indl ( \frac{ \vp* }{ \vp[\anydim] } )
    \pden_\ind^{\lgr\indl}
  \) en est un, d'après le lemme précédent.

  Le calcul du degré est direct et on ne détaille donc que l'estimation de
  norme : chaque terme de la somme est majoré en norme par
  \begin{align}
    \prod_{\ind = \anydim + 1}^{\dimp}
    \prod_{ \indi_\ind = 1 }^{ \ip* }
    \left(
      \nv1{ \chow\anyvar }^2
      \bigl( 4^\anydeg \anydim \anydeg^{3} \bigr)^\dv
    \right)^{\lgr\indl}
    \le
    \nv1{ \chow\anyvar }^{2 (\dimp - \anydim) \lgr\derp}
    \bigl( 4^\anydeg \anydim \anydeg^{3} \bigr)
    ^{(\dimp - \anydim) \lgr\derp \dv}
  \end{align}
  On remarque alors que l'ensemble de sommation \( N \) s'écrit aussi
  \begin{equation}
    \prod_{p = 0}^{\anydim - 1} \left\{
      \gmp\nu[][p] \in \N^{\lgr\ip}
      \text{ tel que }
      \sum\indrange \sum_{\indi_\ind = 1}^{\ip*}
      \gmp\nu[\ind, \indi_\ind][p]
      = \derp[p]
    \right\}
    \pmm.
  \end{equation}
  Chacun des facteurs de ce produit est de cardinal
  \(
    \binom{ \derp[p] + \lgr\ip - 1 }{ \lgr\ip - 1 }
    \le
    2^{ \derp[p] + \lgr\ip - 2 }
  \).
  L'estimation annoncée suit en prenant le produit en majorant assez largement
  certains facteurs.

  Enfin, on constate que chaque terme de la somme définissant
  \( P^\derp_{\ip, \dv} \) contient un facteur de la forme
  \(
    \prod_{\indi_0 = 1}^{\ip[0]}
    \pmnum_{1, \dv}^{\gmp\nu[0, \indi_0]}( \frac{ \vp[0] }{ \vp[\anydim] } )
    \vp[\anydim]
  \). On peut supposer que \( \gmp\nu[0, \indi_0] \in \N \times
    \set{0}^{\anydim-1} \) car sinon ce facteur, donc le terme correspondant,
  est nul. De plus, on a \( \sum_{\indi_0 = 1}^{\ip[0]} \gmp\nu[0, \indi_0][0]
    \le \derp[0] \). Ainsi, il y a au moins \( \ip[0] - \derp[0] \) termes
  nuls dans cette somme, donc \( \vp[0]^{\ip[0] - \derp[0]} \) est en
  facteur de chaque terme non nul de la somme définissant \( P^\derp_{\ip,
      \dv} \), ce qui prouve l'assertion sur l'ordre.
\end{proof}

\begin{rem}
  L'assertion sur l'ordre vaut aussi si la variable de déshomogénéisation
  (voir note~\ref{fn:coord-deshom} page~\pageref{fn:coord-deshom}) est
  \( \vp[0] \), même si elle paraît \lat{a priori} avoir moins de sens dans ce
  cas. En effet, on remarque que le dernier paragraphe de la démonstration
  utilise seulement le fait que \( \pdiff^0 \) est l'idendité ; tout le reste
  en découle formellement.
\end{rem}

On souhaite maintenant étudier le développement en série des fonctions
rationnelles sur notre variété produit \( \var \).  On étend les notations
précédentes au cas multi-homogène de la façon évidente. En particulier, on a
un morphisme \( \pmor \) de développement en série, qu'on représente par des
morphismes \( \pmor*_\dv \) faisant commuter le diagramme
\begin{equation}
  \xymatrix{
    \cdn[\vmp]_{(\varida)}                 \ar[d]^\pi  \ar@{.>}[r]^{\pmor*_\dv}
    & \cdn[\vmp]_{(\varida)} \series\psmp  \ar[d]^\pi
    \\ \cdn(\var)                                     \ar[r]^{\pmor}
    & \cdn(\var)\series\psmp
  }
\end{equation}
et qu'on écrira encore sous la forme \( \pmor*_\dv = \sum_\dermp \psmp^\dermp
  \cdot \pdiff^\dermp \).

Plus précisément, pour chaque \( \fct \in \set{1, \dots, \puiss} \), on a un
morphisme \( \pmor*\pexp\fct \) obtenu en appliquant le lemme~\ref{l:def-pmor}
avec \( V = \var* \) et \( \vp = \vmp* \), et on définit \( \pmor \) comme le
produit tensoriel de ces morphismes. On remarque que les formes \( \pden** \)
données par ces applications du lemme cité sont en fait celles définies au
parapgraphe précédent le scolie~\ref{s:part-cases}.  Nous étudions maintenant
l'image par \( \pmor* \) de certaines fonctions rationnelles.

\begin{lem} \label{l:par-var}
  Soit \( G \) une forme multi-homogène de multidegré \( \alpha = (\alpha_1,
    \dots, \alpha_\puiss ) \) et \( g = G / \vmp[][\vdim]^{\alpha} \) où
  l'on a noté \( \vmp[][\vdim]^{\alpha} = \prod\fctrange
    (\vmp*[\vdim*])^{\alpha_\fct} \). Alors :
  \begin{enumthm}
    \item \( P^\dermp_{G, \dv}
        = \pdiff^\dermp(g)
        \cdot \vmp[][\vdim]^{\alpha}
        \prod\fctrange \pden_\fct^{\lgr{\dermp*}}
      \) est un polynôme ;
    \item \( \deg_\fct P^\dermp_{G, \dv}
        = \alpha_\fct + 2 (\vdeg* - 1) (\dimp - \vdim*) \lgr{\dermp*} \) ;
    \item \( \nv1{ P^\dermp_{G, \dv} }
        \le
        \nv1{ G }
        \prod\fctrange
        \nv1{ \varfca* }^{2 (\dimp - \vdim*) \lgr{\dermp*}}
        \left(
          \bigl( 4^{\vdeg*+1} \vdim* \vdeg*^{3} \bigr) ^{
            (\dimp - \vdim*) \lgr{\dermp*} }
          \cdot 2^{ \vdim* \alpha_\fct }
        \right)^\dv
      \).
  \end{enumthm}
  De plus, \( \inda* \bigl( P^\derp_{G, \dv} \bigr) \ge \inda*(G) -
    \wtsum*(\dermp) \).
\end{lem}

\begin{proof}
  Les trois premiers points s'obtiennent en remarquant que \( g \) est une
  combinaison linéaire de monômes en \( \vmp / \vmp[][\vdim] \), que l'image
  d'un tel monôme par \( \pdiff^\dermp \) est donnée par
  \begin{equation}
    \pdiff^\dermp \Biggl(
      \prod\fctrange \prod\indrange
      \biggl( \frac{ \vmp** }{ \vmp*[\vdim*] } \biggr)^{\imp**}
    \Biggr)
    =
    \prod\fctrange
    \pdiff^{\fct, \dermp*} \Biggl(
      \prod\indrange
      \biggl( \frac{ \vmp** }{ \vmp*[\vdim*] } \biggr)^{\imp**}
    \Biggr)
  \end{equation}
  et en appliquant le lemme~\ref{l:par-anyvar-mono}.

  Seul le point sur l'indice reste à vérifier. Pour cela, considérons \(
    \vmp^\imp \) un monôme apparaissant dans l'écriture \( G \), puis \(
    \vmp^{\gmp\nu} \) un monôme apparaissant dans \( P^\dermp_{G, \dv} \).
  D'après le lemme cité, on alors \( \gmp\nu*[0] \ge \imp*[0] - \dermp*[0] \)
  d'où, en sommant, \( \wtsum*(\gmp\nu) \ge \wtsum*(\imp) - \wtsum*(\dermp) \)
  qui est équivalent à l'estimation annoncée vu la définition de l'indice.
\end{proof}

Notons que les estimations obtenues ne valent que pour des fonctions admettant
un monôme en \( \vmp[][\vdim] \) comme dénominateur.  On pourrait aisément en
déduire des estimations pour des fonctions rationnelles de dénominateur
quelconque en les écrivant comme un quotient de deux telles fonctions, mais
ce n'est pas utile ici.


\subsection{Minoration de l'indice}
\label{sec:vojta-extrap-core}

Nous allons maintenant montrer que la forme auxiliaire construite précédemment
s'annule avec un indice élevé en \( \ex \). Commençons par préciser la notion
d'indice utilisée, similaire à celle définie au début de la
sous-section~\ref{sec:siegel-plan}, mais cette fois en un point. Par ailleurs
on introduit d'abord la notion d'indice pondéré par un vecteur \( b \in
  \N^\puiss \) quelconque avant de spécialiser au vecteur de poids qui nous
intéresse, ce qui n'était pas utile précédemment mais le sera cette fois pour
énoncer la variante du théorème du produit que nous utiliserons. Posons
\begin{equation}
  \wtsum[b]( \dermp )
  =
  \frac {\lgr{\dermp[1]}} {b_1} + \dots
  + \frac {\lgr{\dermp[\puiss]}} {b_\puiss}
\end{equation}
pour tout \( \dermp \in \N^\vdim \).  Si \( \ratfi \) est une fraction
rationnelle et \( \point \) un point où elle est définie, on définit son
indice en \( \point \) comme
\begin{equation}
  \inda[\point] \ratfi
  =
  \inf \set{
    \wtsum[b](\imp)
    \text{ pour }
    \imp \text{ tel que } \der[\imp] \ratfi(\point) \neq 0
  }
  \pmm,
\end{equation}
où la borne inférieure est en fait un minimum, sauf si \( \ratfi = 0 \), cas
où l'indice est infini.

\begin{lem} \label{l:indice-inversible}
  Soient \( \ratfi_1 \), \( \ratfi_2 \) et \( \alpha \) des fonctions
  rationnelles telles que \( \ratfi_1 = \alpha \ratfi_2 \) et \( \point \) un
  point où elles sont toutes les trois définies.
  \begin{enumthm}
    \item Si \( \dermp \) est tel que \( \der[\gmp\nu] \ratfi_2(x) = 0 \) dès
      que \( \gmp\nu < \dermp \) pour l'ordre produit sur \( \N^\vdim
      \), alors \( \der[\dermp] \ratfi_1(\point) = \alpha(\point) \,
        \der[\dermp] \ratfi_2(\point) \).
    \item Si \( \alpha(\point) \neq 0 \) on a \( \indg b[\point](\ratfi_1) =
        \indg b[\point](\ratfi_2) \).
  \end{enumthm}
\end{lem}

\begin{proof}
  Le premier point découle facilement de la formule de \bsc{Leibniz} :
  \begin{equation}
    \der[\dermp] \ratfi_1(\point)
    =
    \sum_{\gmp\nu \le \dermp}
    \der[\dermp - \gmp\nu] \alpha(\point) \,
    \der[\gmp\nu] \ratfi_2(\point)
    \pmm.
  \end{equation}
  Or, par hypothèse, tous les termes de cette somme sont nuls sauf peut-être
  celui où \( \gmp\nu = \dermp \).

  Si \( \alpha(\point) \neq 0 \) alors \( \alpha^{-1} \) est également définie
  en \( \point \) et les deux autres fonctions jouent donc un rôle symétrique.
  Ainsi, si un indice \( \dermp \) est minimal pour la condition \(
    \der[\dermp] \ratfi_1(\point) \neq 0 \), il l'est aussi pour la condition \(
    \der[\dermp] \ratfi_2(\point) \neq 0 \) grâce au point précédent, ce qui
  prouve que les deux fonctions ont le même indice en \( \point \).
\end{proof}

Si \( G \) est une forme multihomogène, on défini \( \indg b[\point] G =
  \indg b[\point] G/H \) où \( H \) est n'importe quelle forme multihomogène de
même degré ne s'annulant pas en \( \point \). Le deuxième point du lemme
précédent montre que cette définition a bien un sens. On s'intéressera
désormais à l'indice pondéré par \( \wtw \wts = (\wts[1], \dots,
  \wts[\puiss-1], (\puiss-1) \wts**) \).

\medskip

On choisit dans l'atlas \( \clmaps^{\puiss-1} \) présenté en
sous-section~\ref{sec:wemb} une carte \( \clmap_\ex \) contenant \( \ex \), et
on considère la forme multihomogène \( \faux** \) donnée par le
scolie~\ref{s:aux-co}. Le but de cette section est alors de montrer la
proposition suivante.

\begin{prop} \label{p:extra}
  \worknote{Corriger, et éventuellement fixer la définition de \(
      \cst{vs-deg-prod} \)}
  On a \( \inda** \faux** \ge \frac{ \epsi \delta }{ \sigma } \) avec
  \( \sigma = \frac{16 (\puiss-1)}{\eps} \).
\end{prop}

La preuve consistera à évaluer les valeurs absolues locales des valeurs en \(
  \ex \) des dérivées successives de cette forme et à montrer qu'elles sont
suffisamment petites, tant que l'ordre de dérivation n'est pas trop élevé,
pour contredire la formule du produit à moins que la dérive ne s'annulle.
Avant de procéder, commençons par introduire quelques choix d'indices
adaptés à chaque place.

On rappelle que \( \cexa \) désigne un système de coordonnées multihomogènes
(dans le plongement adapté) du point \( \ex \) supposé contredire le
théorème~\ref{t:vojta-div}.  On choisit, pour tout \(
  \fct \), un indice \( \indv* \in \set{0, \dots, \vdim*} \) de sorte que \(
  \av{ \cexa*[\indv*] } \) soit maximal parmi \( \av{\cexa*[0]}, \dots,
  \av{\cexa*[\vdim*]} \).  Le lemme suivant montre que cette valeur absolue
représente à peu de chose près la norme de \( \cexa* \).

\begin{lem} \label{l:coord-norm}
  Avec les notations précédentes, on a
  \begin{equation}
    \av{ \cexa*[\indv*] }
    \ge
    \nv\infty{ \cexa }
    \cdot \nv1{ \varfc* }^{-1}
    \cdot ( 2 \vdegp* )^{- \vdeg* (\vdim* + 1) \dv}
    \pmm.
  \end{equation}
\end{lem}

\begin{proof}
  Il s'agit de montrer qu'on a
  \begin{equation}
    \av{ \cexa** }
    \le
    \av{ \cexa*[\indv] }
    \cdot \nv1{ \varfc* }
    \cdot ( 2 \vdegp* )^{\vdeg* (\vdim* + 1) \dv}
  \end{equation}
  pour tout \( \ind \). D'après la note~\ref{fn:varfc}
  page~\pageref{fn:varfc}, on a \( \nv1{ \varfc* } \ge 1 \), donc le résultat
  est acquis dès que \( \av{ \cexa** } \le \av{ \cexa*[\indv] } \), ce qui est
  le cas pour \( \ind \le \vdim* \) par définition de \( \indv \). Notons donc
  \( \ind > \vdim* \) un indice tel que \( \av{ \cexa*[\indv] } > \av{ \cexa**
    } \).

  On utilise alors le fait qu'on connaît, dans un plongement adapté, des
  relations \( \poldep** \) liant chacune des dernières coordonnées aux
  premières,  unitaires en la dernière variable. En décomposant \( \poldep**
  \) suivant les puissances de \( \vmp** \), on voit qu'il existe des
  polynômes \( \poldep*[\ind, \alpha] \in \cdn(\vmp*[0], \dots, \vmp*[\vdim*]
  \) tels que :
  \begin{equation}
    (\vmp**) ^{ \vdeg* }
    =
    \sum_{ \alpha=1 }^{ \vdeg* }
    (\vmp**) ^{ \vdeg* - \alpha }
    \, \poldep*[\ind, \alpha]
    \quad\text{où }
    \deg \poldep*[\ind, \alpha] = \alpha
    \text{ et }
    \sum_{ \alpha=1 }^{ \vdeg* } \nv1{ \poldep*[\ind, \alpha] }
    \le \nv1{ \poldep** }
  \end{equation}
  On spécialise alors en \( \cexa* / \cexa*[\indv] \) et on prend les valeurs
  absolues :
  \begin{equation}
    \av[\Bigg]{ \frac{ \cexa** }{ \cexa*[\indv] } }^{ \vdeg* }
    \le
    \sum_{ \alpha=1 }^{ \vdeg* }
    \: \av[\Bigg]{ \frac{ \cexa** }{ \cexa*[\indv] } }^{ \vdeg* - \alpha }
    \: \av[\Bigg]{
      \poldep*[\ind, \alpha] \Biggl(
        \frac{ \cexa*[0] }{ \cexa*[\indv] }, \dots,
        \frac{ \cexa*[\vdim*] }{ \cexa*[\indv] }
      \Biggr)
    }
  \end{equation}
  puis on divise et on utilise les remarques précédentes pour obtenir :
  \begin{align}
    \av[\Bigg]{ \frac{ \cexa** }{ \cexa*[\indv] } }
    \le
    \sum_{ \alpha=1 }^{ \vdeg* }
    \: \av[\Bigg]{ \frac{ \cexa** }{ \cexa*[\indv] } }^{ 1 - \alpha }
    \: \nv1{ \poldep*[\ind, \alpha] }
    \le
    \sum_{ \alpha=1 }^{ \vdeg* }
    \nv1{ \poldep*[\ind, \alpha] }
    \le
    \nv1{ \poldep** }
  \end{align}
  Le résultat désiré est alors donné exactement par~\eqref{e:nv-poldep}.
\end{proof}

On introduit ensuite le point \( \exi \in \va^{\puiss-1} \) tel que \(
  \wemb(\ex) = (\ex, \exi) \) et on note \( \cexi* = \wembcl[\clmap_\ex,
  \fcti]
  (\cexa[\fcti], \cexa[\puiss]) \) un système de coordonnées multihomogènes de
\( \exi \) dans le plongement \( \Theta^{\puiss-1} \). (Ci-dessous on notera
\( \wembcl* = \wembcl*_{\clmap_\ex} \) pour alléger.)
On choisit alors pour tout \(
  \fcti \) un indice \( \indiv* \in \set{0, \dots, \dimp} \) maximisant \(
  \av{ \cexi*[\indiv*] } \), ainsi que \( \indig* \in \set{0, \dots, \dimp} \)
tel que \( \cexi*[\indig*] \neq 0 \). Il est alors clair que \( \av{
    \cexi*[\indiv*] } \ge \nv1{ \cexi* } (\dimp+1)^{-\dv} \).

\medskip

Introduisons maintenant une fonction rationnelle définie par
\begin{equation}
  \ratf =
  \frac{
    \faux**( \vmp )
  }{
    \prod\fctrange
    (\vmp*[\vdim*])^{\epsz \wts* \delta, + r_\fct}
    \,
    \prod\fctirange
    \wembcl*[\indig*](\vmp) ^{ \delta }
  }
  \pmm.
\end{equation}
D'après le scolie~\ref{s:part-cases}, on a \( \cexa*[\vdim*] \neq 0 \) ; par
ailleurs le choix de \( \indig* \) assure que \( \wembcl*[\indig*]( \cexa )
  \neq 0 \), de sorte que le dénominateur ne s'annule pas en \( \cexa \) et
que par définition on a \( \inda** \faux** = \inda** \ratf \).
Pour minorer cet indice, nous allons commencer par décomposer cette fonction
en des facteurs dont nous aurons un bon contrôle en chaque place.

Pour tout place \( \place \) on note désormais \( \wembclv* =
  \wembcl[\clmap_{\wts*, \wts**, \ex*, \ex**, \place}, \fcti] \) où \(
  \wembclm* \)
est défini par~\eqref{e:def-wembclm} et \( \clmap_{\wts*, \wts**, \ex*, \ex**,
    \place} \) donné par le lemme~\ref{l:hclab}. De même, on note \(
  \fauxv = \faux[\clmap_{\wts*, \wts**, \ex*, \ex**, \place}] \).
En utilisant le premier point du scolie~\ref{s:aux-co} et en remarquant que,
pour chaque \( \fcti \), les vecteurs \( \wembclv* (\cexa[\fcti],
  \cexa[\puiss]) \) sont proportionnels entre eux quand \( \place \) varie, on
a, pour tout \( \place \) :
\begin{align}
  \ratf
  & =
  \frac{
    R(\vmp) \,
    \faux( \vmp; \wembclv[1](\vmp), \dots, \wembclv[\puiss-1](\vmp) )
  }{
    \prod\fctrange
    (\vmp*[\vdim*])^{\epsz \wts* \delta + r_\fct}
    \cdot
    \prod\fctirange
    \wembclv*[\indig](\vmp) ^{ \delta }
  }
  \intertext{par homogénéité de \( \faux \). On introduit alors nos indices locaux
    \( \indv \) et \( \indiv \) :}
   & =
  \frac{
    R(\vmp) \,
    \faux( \vmp; \wembclv[1](\vmp), \dots, \wembclv[\puiss-1](\vmp) )
  }{
    \prod\fctrange
    (\vmp*[\indv*])^{\epsz \wts* \delta + r_\fct}
    \cdot
    \prod\fctirange
    \wembclv*[\indiv](\vmp) ^{ \delta }
  }
  \cdot \prod\fctirange
  \left(
    \frac{ \wembclv**(\vmp) }{ \wembclv*[\indig*](\vmp) }
  \right) ^{ \delta }
  \!\!\! \cdot \prod\fctrange
  \left(
    \frac{ \vmp[\indv*] }{ \vmp[\vdim*] }
  \right) ^{ \epsz \wts* \delta + r_\fct }
  \intertext{puis on utilise la définition de \( \Dir \) en scindant le
    premier facteur et de nouveau la remarque sur les \( \wembclv* \) pour
    simplifier le second :}
  & =
  \frac{
    R(\vmp) \,
    \faux( \vmp; \wembclv[1](\vmp), \dots, \wembclv[\puiss-1](\vmp) )
  }{
    \prod\fctrange (\vmp*[\indv*]) ^{ \Dir* }
  }
  \\ & \qquad
  \cdot \underbrace{
    \prod\fctirange \left(
      \frac{
        ( \vmp[\fcti ][{\indv[\fcti ]}] )^{ 2\wts[\fcti ] }
        ( \vmp[\puiss][{\indv[\puiss]}] )^{ 2\wts[\puiss] }
      }{
        \wembclv**(\vmp[\fcti], \vmp[\puiss])
      }
    \right)^\delta
  }_{\textstyle \ratfv2}
  \cdot \underbrace{
    \prod\fctirange \left(
      \frac{ \wembcl*[\indiv*](\vmp) }{ \wembcl*[\indig*](\vmp) }
    \right) ^{ \delta }
  }_{\textstyle \ratfv3}
  \cdot \underbrace{
    \prod\fctrange \left(
      \frac{ \vmp[\indv*] }{ \vmp[\vdim*] }
    \right) ^{ \epsz \wts* \delta + r_\fct }
  }_{\textstyle \ratfv4}
  \pmm.
\end{align}
On remarque que le dénominateur du premier facteur est égale à \( \fauxv
\) modulo \( \varida \).

Soit maintenant \( \dermp \in \N^\vdim \) un indice minimal tel que \(
  \der[\dermp] \ratf(\ex) \neq 0 \). Comme les derniers facteurs de
l'écriture précédente sont tous définis et inversibles en \( \cex \) grâce aux
différents choix d'indices et au premier point du scolie~\ref{s:part-cases},
une nouvelle application du lemme~\ref{l:indice-inversible} montre que, pour
tout \( \place \) :
\begin{equation}
  \der[\dermp] \ratf(\ex)
  =
  \underbrace{
    \der[\dermp]
    \frac{
      \fauxv( \vmp )
    }{
      \prod\fctrange (\vmp*[\indv*]) ^{ \Dir* }
    }
  }_{\textstyle \ratfv1} \null
  ( \cexa )
  \cdot \ratfv2(\cexa)
  \cdot \ratfv3(\cexa)
  \cdot \ratfv4(\cexa)
\end{equation}
Remarquons que chacun des facteurs du membre de droite dépend de \( \place \),
mais pas leur produit.
Rappelons qu'on note \( \degv \) le degré local en \(
  \place \) et posons \( \ratfh{p} = \sum_\place \degv
  \ln \av{\ratfv{p}(\cexa)} \) ; cette somme converge car \( \ratfv p \)
parcourt en fait un ensemble fini (en bijection avec l'atlas choisi en
sous-section~\ref{sec:wemb}) lorsque \( \place \) parcourt l'ensemble des
places de \( \cdn \).  La formule du produit donne alors
\begin{equation} \label{e:prod-lem}
  0 = \ratfh1 + \ratfh2 + \ratfh3 + \ratfh4
\end{equation}
Supposons maintenant que \( \wtsum( \dermp ) < \epsi \delta / \sigma \)
contrairement à la conclusion de la proposition, et montrons qu'on contredit
alors l'égalité ci-dessus.  Pour cela, on majore séparément chacun des termes,
en procédant de droite à gauche (par ordre plus ou moins croissant de
difficulté). Dans ces estimations, nous noterons \( B \) le majorant commun
des \( \vdeg* \) donné par \eqref{e:varset-deg} afin d'alléger l'écriture.

Pour \( \ratfv4 \) on commence par remarquer que le dénominateur n'intervient
pas dans le calcule de la hauteur, puis au numérateur que la hauteur d'une
coordonnée est majorée par la norme du vecteur :
\begin{equation}
  \prod_\place
  \prod\fctrange
  \left(
    \frac{ \av{\cexa*[\indv*]} }{ \av{\cexa*[\vdim*]} }
  \right) ^{ (\epsz \delta \wts* + r_\fct) \degv }
  \le
  \prod\fctrange
  \hautm[\infty]{ \cexa* }^{\epsz \delta \wts* + r_\fct}
\end{equation}
puis on passe au logarithme et on utilise le fait que \( \cexa*
  = \vadapt*(\cex*) \)
\begin{align}
  \ratfh4
  & \le
  \epsz \delta \Bigl(
    \sum\fctrange \wts* \hautl[\infty]{ \cexa* }
  \Bigr) + o(\delta)
  \\ & \le
  \epsz \delta \Bigl(
    \sum\fctrange \wts* \bigl(
      \hautl[\infty]{ \cex* } + \hautl[1]{\vadapt*}
    \bigr)
  \Bigr) + o(\delta)
  \intertext{puis, par le fait~\ref{f:comp-h-hn} :}
  & \le
  \epsz \delta \Bigl(
    \sum\fctrange \wts* \bigl(
      \hautn{ \ex* } + \htcmp + \hautl[1]{\vadapt*}
    \bigr)
  \Bigr) + o(\delta)
  \intertext{et en utilisant~\eqref{e:vadapt-ht} et la notation \( B \)
    ci-dessus puis en regroupant les termes :}
  & \le
  \epsz \delta \Bigl(
    \sum\fctrange \bigl( \wts* \hautn{ \ex* } \bigr)
    + \bigl( \htcmp + \ln(\dimp+1) + \ln B \bigr) \sum\fctrange \wts*
  \Bigr) + o(\delta)
  \intertext{et enfin, en invoquant~\eqref{e:wt-ratio}, le
    lemme~\ref{l:sum-wts} et la définition de \( \cst{vs-ht} \) :}
  \label{e:ratfh4}
  \ratfh4
  & \le
  \epsz \delta \Bigl(
    2 \puiss \wts[1] \hautn{ \ex[1] }
    + 2 \wts[1] \bigl( \cst{vs-ht} + \ln B \bigr)
  \Bigr) + o(\delta)
  \pmm.
\end{align}

Pour \( \ratfh3 \) on commence avec les mêmes arguments pour écrire
\begin{equation}
  \prod_\place
  \prod\fctirange
  \left(
    \frac{
      \av{\wembcl*[\indiv*](\cexa[\fcti], \cexa[\puiss])} }{
      \av{\wembcl*[\indig*](\cexa[\fcti], \cexa[\puiss])} }
  \right) ^{ \delta \degv }
  \le
  \prod\fctirange
  \hautm[\infty]{ \exi* }^\delta
\end{equation}
puis on passe au logarithme et on utilise encore le fait~\ref{f:comp-h-hn} :
\begin{align}
  \ratfh3
  & \le
  \delta \sum\fctirange \Bigl(
    \hautn{ \exi* } + \htcmp
  \Bigr)
  \intertext{mais on conclut cette fois en invoquant~\eqref{e:hautn-wt-diff} :}
  \ratfh3 \label{e:ratfh3}
  & \le
  \delta \bigl(
    3 \wts[1] (\puiss-1) \Vcos \hautn{\ex[1]} + o(\wts[1])
  \bigr)
  \pmm.
\end{align}

Pour \( \ratfv2 \), on commence par minorer le dénominateur grâce à la
définition de \( \indiv* \) puis on utilise la majoration évidente \(
  \nv\infty{\cexa*} \le \nv\infty{\vadapt*} \, \nv1{\cex*} \) :
\begin{align}
  \frac{
    \av{ \cexa[\fcti ][{\indv[\fcti ]}] }^{ 2\wts[\fcti ] }
    \av{ \cexa[\puiss][{\indv[\puiss]}] }^{ 2\wts[\puiss] }
  }{
    \av{ \wembclv**(\cexa[\fcti], \cexa[\puiss]) }
  }
  & \le
  \frac{
    \nv\infty{ \cexa[\fcti ] }^{ 2\wts[\fcti ] }
    \nv\infty{ \cexa[\puiss] }^{ 2\wts[\puiss] }
  }{
    \nv1{ \wembclv*(\cexa[\fcti], \cexa[\puiss]) }
  }
  (\dimp+1)^{2 \dv (\wts[\fcti] + \wts[\puiss])}
  \\ & \le
  \frac{
    \nv1{ \cex[\fcti ] }^{ 2\wts[\fcti ] }
    \nv1{ \cex[\puiss] }^{ 2\wts[\puiss] }
  }{
    \nv1{ \wembclv*(\cexa[\fcti], \cexa[\puiss]) }
  }
  ((\dimp+1) B)^{2 \dv (\wts[\fcti] + \wts[\puiss])}
  \\ & \le
  \frac{
    \bigl( \hmclab* ((\dimp+1) B)^{2\dv} \bigr)^{\wts[\fcti] + \wts[\puiss]}
  }{
    \nnv1{ \wembclv* }
  }
\end{align}
où la dernière ligne n'est autre que la conclusion du lemme~\ref{l:hclab}
compte tenu de la définition de \( \wembclv* \) ci-dessus. On note désormais
\begin{equation} \label{e:c-wembcl}
  C_\wembcl = \prod_\place \prod\fctirange \nnv1{ \wembclv* }^{\degv}
  \pmm,
\end{equation}
(somme qui converge bien car \( \wembclv \) parcourt en fait un ensemble fini
quand \( \place \) varie) de sorte qu'en prenant le logarithme dans
l'estimation précédente, puis en sommant sur \( \place \) et sur \( \fcti \),
compte tenu du lemme~\ref{l:sum-wts}, on a facilement
\begin{equation} \label{e:ratfh2}
  \ratfh2 \le
  \delta \Bigl(
    - \ln C_\wembcl
    + 2 \wts[1] \bigl( \cst{vs-ht} + \ln B \bigr)
  \Bigr)
  \pmm.
\end{equation}
Remarquons avant de continuer que le premier terme, qui semble négatif, va se
simplifier avec son opposé qui apparaît dans la hauteur du prochain facteur,
de sorte qu'il ne contribue pas à faire diminuer la hauteur finale et qu'on a
pas besoin de l'expliciter davantage.

\medskip

Pour expliciter \( \ratfv1 \), on applique le lemme~\ref{l:par-var} aux
formes \( \fauxv \).  Comme l'indique la note~\ref{fn:coord-deshom}
page~\pageref{fn:coord-deshom}, il est loisible d'utiliser sur chaque facteur
\( \vmp*[\indv*] \) plutôt que \( \vmp*[0] \) comme variable de
déshomogénéisation vu que \( 0 \le \indv* \le \vdim* \) ; simplement, les
polynômes fournis par le lemme dépendent de ce choix, donc en définitive de \(
  \place \), ce qui est déjà le cas.  Par ailleurs, chaque application du
lemme fournit deux familles de polynômes, satisfaisant respectivement des
majorations de normes aux places finies ou infinies ; on ne garde que celui
correspondant à l'indice \( \dermp \) fixé ci-dessus et la nature
archimédienne ou non de la place considérée, de façon à obtenir une famille
indexée par \( \place \) de polynômes \( P_\place \) et jouissant des
propriétés suivantes :
\begin{enumthm}
  \item \( \displaystyle
      \der[\dermp]
      \frac{
        \fauxv(\vmp)
      }{
        \prod\fctrange (\vmp*[\indv*]) ^{ \Dir* }
      }
      =
      \frac{ P_\place(\vmp) }{
        \prod\fctrange (\vmp*[\indv*]) ^{ \Dir* }
        \, \pdenp*(\vmp*)^{\lgr{\dermp*}}
      }
    \) ;
  \item \(
      \deg_\fct P_\place
      =
      \Dir* + 2 (\vdeg* - 1) (\dimp - \vdim*) \lgr{\dermp*}
    \) ; \label{i:deg-p-der}
  \item \(
      \nv1{ P_\place }
      \le
      \nv1{ \fauxv }
      \prod\fctrange
      \nv1{ \varfca* }^{2 (\dimp - \vdim*) \lgr{\dermp*}}
      \left(
        \bigl( 4^{\vdeg*+1} \vdim* \vdeg*^{3} \bigr) ^{
          (\dimp - \vdim*) \lgr{\dermp*} }
        \cdot 2^{ \vdim* (\Dir*) }
      \right)^\dv
    \) ; \label{i:norm-p-der}
  \item \(
      \inda* \bigl( P_\place \bigr)
      \ge
      \inda*(\fauxv) - \wtsum*(\dermp)
    \).
\end{enumthm}
Remarquons de suite que tous les \( P_\place \) sont de même degré, qu'on
notera parfois \( \deg P \) par abus, et que \( P_\place \) parcourt un
ensemble fini quand \( \place \) varie, car c'est déjà le cas de \( \fauxv
\) et de \( \wembclv \).

Rappelons également que l'on a supposé que \( \wtsum(\dermp) \le \epsi \delta
  / \sigma \) ; en remarquant que \( \inda*(\fauxv) \ge \epsi \delta \)
par construction (scolie~\ref{s:aux-co})  et que \( \wtsum*(\dermp) \le
  \wtsum(\dermp) \) par définition, le dernier point donne
\begin{equation} \label{e:ind-Pv}
  \inda*( P_\place ) \ge \epsi \delta (1 - \sigma^{-1})
  \pmm.
\end{equation}

Pour estimer \( \ratfh1 \), commençons par scinder une nouvelle fois notre
fonction en deux facteurs :
\begin{equation}
  \ratfv1 =
  \frac{ P_\place(\vmp) }{
    \prod\fctrange (\vmp*[\indv*]) ^{ \Dir* }
    \, \pdenp*(\vmp*)^{\lgr{\dermp*}}
  }
  =
  \underbrace{
    \frac{ P_\place(\vmp) }{
      \prod\fctrange (\vmp*[\indv*])^{\deg_\fct P_\place }
    }
  }_{\textstyle \ratfv{1'}}
  \cdot
  \underbrace{
    \prod\fctrange
    \frac{
      (\vmp*[\indv*])^{\lgr{\dermp*} \cdot \deg \pdenp*}
    }{
      \pdenp*(\vmp*)^{\lgr{\dermp*}}
    }
  }_{\textstyle \ratfv{1''}}
\end{equation}
Pour le second facteur, on commence par remarquer que le dénominateur n'est
pas nul d'après le deuxième point du scolie~\ref{s:part-cases} et qu'on a donc
\begin{equation}
  \prod_\place
  \frac{
    \av{ \cexa*[\indv] }^{ \deg \pdenp* \lgr{\dermp*} \degv }
  }{
    \av{ \pden_\fct(\cexa*) }^{ \lgr{\dermp*} \degv }
  }
  \le
  \hautm[\infty]{ \cexa* }^{ \deg \pdenp* \lgr{\dermp*} }
  \pmm.
\end{equation}
On note alors que par définition, \( \deg \pdenp* = 2 (\vdeg*-1) (\dimp-\vdim*)
  \le 2B\dimp \), puis en utilisant des comparaisons de hauteurs désormais
habituelles, on a :
\begin{align}
  \ratfh{1''}
  & \le
  2B\dimp \sum\fctrange
  \hautl[\infty]{ \cexa* } \lgr{\dermp*}
  \\ & \le
  2B\dimp \sum\fctrange
  \bigl( \hautn{ \ex* } + \cst{vs-ht} + \ln B \bigr)
  \lgr{\dermp*}
  \pmm.
\end{align}
On se souvient ensuite que \( \wts[1] \ge \wts* \) et même \( \wts[1] \ge
  (\puiss-1) \wts[\puiss] \) (par exemple d'après le lemme~\ref{l:sum-wts})
pour écrire
\begin{equation}
  \sum\fctrange \lgr{\dermp*}
  \le
  \wts[1] \, \wtsum(\dermp)
  \le
  \wts[1] \, \epsi \delta / \sigma
  \pmm.
\end{equation}
Par ailleurs, en utilisant~\eqref{e:wt-ratio}, on a
\begin{equation}
  \sum\fctrange \hautn{ \ex* } \lgr{\dermp*}
  \le
  (\puiss-1) \sum\fctrange
  \wts* \hautn{ \ex* } \, \frac{ \lgr{\dermp*} }{ \wts* }
  \le
  2 (\puiss-1) \wts[1] \hautn{ \ex[1] } \, \epsi \delta / \sigma
\end{equation}
et au final :
\begin{equation} \label{e:ratfh1b}
  \ratfh{1''} \le
  \frac{ \delta \wts[1] \epsi }{ \sigma } \Bigl(
    2 (\puiss-1) \hautn{ \ex[1] }
    + \cst{vs-ht} + \ln B
  \Bigr)
  \pmm.
\end{equation}

Pour l'autre partie, on écrit
\begin{equation}
  \ratfv{1'}(\cexa)
  =
  P_\place \Biggl(
    \frac{ \cexa[1] } { \cexa[1][{\indv[1]}] }, \dots,
    \frac{ \cexa[\puiss] } { \cexa[\puiss][{\indv[\puiss]}] }
  \Biggr)
  =
  \sum_{\imp} p_{\place, \imp} \Biggl(
    \frac{ \cexa } { \cexa[][\indv] }
  \Biggr)^{\imp}
\end{equation}
où la dernière somme est prise sur les multiindices \( \imp \) tels que \(
  \lgr{\imp*} = \deg_\fct P_\place \). Étudions de plus près les valeurs
absolues des monômes intervenant dans cette écriture ; d'après le
lemme~\ref{l:coord-norm} on a :
\begin{align}
  \prod\fctrange \prod\indrange
  \Biggl(
    \frac{ \av{\cexa**} } { \av{\cexa*[\indv*]} }
  \Biggr)^{\imp**}
  & \le
  \prod\fctrange \prod\indrange
  \Biggl(
    \frac{ \av{\cexa**} } { \nv\infty{\cexa*} }
  \Biggr)^{\imp**}
  \cdot
  \prod\fctrange \Bigl(
    \nv1{ \varfc* }
    \cdot ( 2 \vdegp* )^{\vdeg* (\vdim* + 1) \dv}
  \Bigr)^{\deg_\fct P_\place}
  \pmm.
\end{align}
Il est clair que le premier facteur est inférieur à \( 1 \) pour tout \(
  \place \). Cependant, si \( \place \in \placesapx \) on peut dire mieux en se
concentrant sur la partie
\begin{equation} \label{e:apx-temp}
  \prod\fctrange
  \Biggl(
    \frac{ \av{ \cexa*[0] } }{ \nv\infty{ \cexa* } }
  \Biggr)^{\imp*[0]}
\end{equation}
et en exploitant l'hypothèse principale.
On commence par utiliser le fait que \( \cex* = \vadapt*^{-1}( \cexa* ) \)
pour minorer le dénominateur, puis une comparaison classique de normes donne
\begin{equation}
  \frac{ \av{ \cexa*[0] } }{ \nv\infty{ \cexa* } }
  \le
  \frac{ \av{ \cexa*[0] } }{ \nv2{ \cex* } }
  \cdot \sqrt{\dimp+1}^\dv \, \nv1{\vadapt^{-1}}
  \pmm.
\end{equation}
Par construction de \( \vadapt \) (lemme~\ref{l:adapt-gen}) on a \(
  \cexa*[0] = \cex*[0] \) et on remarque alors que le premier facteur dans
l'écriture ci-dessus n'est autre que \( \distv{\ex*}{\divi} \), ce qui nous
permet d'exploiter~\eqref{e:Vapx} :
\begin{align}
  \frac{ \av{ \cexa*[0] } }{ \nv\infty{ \cexa* } }
  & \le
  \hautm[2]{\ex*}^{-\wtapx \expapx}
  \cdot \sqrt{\dimp+1}^\dv \, \nv1{\vadapt^{-1}}
  \\ & \le
  \expb^{ -\wtapx \expapx \hautn{\ex*} }
  \cdot \expb^{\wtapx \expapx \htcmp}
  \cdot \sqrt{\dimp+1}^\dv \, \nv1{\vadapt^{-1}}
\end{align}
où la dernière ligne découle comme d'habitude du fait~\ref{f:comp-h-hn}.

On peut alors majorer le logarithme de~\eqref{e:apx-temp} par
\begin{align}
  -\wtapx\expapx \sum\fctrange \imp*[0] \hautn{\ex*}
  + \sum\fctrange \deg_\fct P^\dermp_\place \left(
    \frac{3\dv}2 \ln(\dimp+1) + \ln \nv\infty{\vadapt^{-1}}
    + \wtapx\expapx \htcmp
  \right)
\end{align}
et pour majorer le premier terme de cette somme on écrit en utilisant
encore~\eqref{e:wt-ratio}
\begin{equation}
  \sum\fctrange
  \imp*[0] \hautn{ \ex* }
  \ge
  \sum\fctrange
  \frac{ \imp*[0] }{ \wts* } \, \wts* \hautn{ \ex* }
  \ge
  \frac12 \wts[1] \hautn{ \ex[1] }
  \sum\fctrange
  \frac{ \imp*[0] }{ \wts* }
  \ge
  \frac12 \wts[1] \hautn{ \ex[1] }
  \, \delta \epsi (1 - \sigma^{-1})
\end{equation}
où le dernier point découle de~\eqref{e:ind-Pv} et de la définition de
l'indice.

En regroupant tous les termes qui constituent \( \ratfv{1'}( \cex ) \) puis en
sommant sur \( \place \), on majore la hauteur logarithmique de ce facteur par
la quantité suivante, où l'on note \( \wtapx = 0 \) si \( \place \notin
  \placesapx \) :
\begin{multline}
  \sum_\place \degv \Biggl(
    - \delta \wts[1] \hautn{\ex[1]}
    \wtapx\expapx \epsi (1 - \sigma^{-1}) /2
    + \ln \nv1{P_\place}
    + \sum\fctrange \deg_\fct P_\place
    \biggl(
      \frac{3\dv}2 \ln(\dimp+1)
      \\
      + \ln \nv\infty{\vadapt^{-1}}
      + \wtapx\expapx \htcmp
      + \ln \nv1{ \varfc* }
      + (\vdeg* (\vdim* + 1) \dv) \ln( 2 \vdegp* (\dimp+1) )
    \biggl)
  \Biggr)
\end{multline}
puis en se souvenant que \( \sum_\place \degv \wtapx = 1 \) et \( \sigma >
  4 \) et en utilisant~\eqref{e:vadapt-ht} :
\begin{equation} \label{e:apx-1a-split}
  - \frac38 \delta \wts[1] \hautn{\ex[1]} \expapx \epsi
  + \hautl[1]{(P_\place)_\place}
  + \sum\fctrange \deg_\fct P \cdot C_\fct
\end{equation}
où l'on a noté \( \deg_\fct P \) le degré commun des \( P_\place \) et
\begin{align}
  C_\fct
  & \le
  \frac{3}2 \ln(\dimp+1)
  + \dimp \ln B + \dimp \ln \dimp
  + \expapx \htcmp
  + \hautl[1]{ \varfc* }
  + B (\genre + 1)  \bigl( \ln B + \ln (\dimp+1) \bigr)
  \\ & \le
  \hautl[1]{ \varfc* }
  + \cst{vs-ht}
  + \dimp B^2
\end{align}
en majorant assez largement sur la dernière ligne. Explicitions maintenant un
majorant du degré commun des \( P_\place \), donné par le
point~\ref{i:deg-p-der} page~\pageref{i:deg-p-der}, d'abord pour \( \fct <
  \puiss \) :
\begin{align}
  \deg_\fct P
  & \le
  \delta (2 + \epsz) \wts* + 2B\dimp \frac{\epsi \delta}{\sigma} \wts*
  + o(\delta)
  \\ & \le
  \delta \wts* \left( 3 + \frac{B\dimp}{2} \right)
  + o(\delta)
  \le
  \delta \wts* B\dimp
  + o(\delta)
\end{align}
\worknote{Si \( \epsi B\dimp > 1 \) on peut conserver le \( \epsi \) en
  facteur, ça serait intéressant.}
où l'on a utilisé la définition~\eqref{e:d-dp-def} de \( d' \), puis le fait
que \( \wtsum(\dermp) \le \epsi\delta / \sigma \) et les majorations larges \(
  \epsi < 1 \) et \( \sigma^{-1} \le 1/4 \). Pour \( \fct = \puiss \) on
obtient de même
\begin{equation}
  \deg_\puiss P
  \le
  \delta \wts[\puiss] B\dimp (\puiss - 1)
  \pmm.
\end{equation}
On utilise alors~\eqref{e:varset-ht} d'une part et le lemme~\ref{l:sum-wts}
d'autre part pour obtenir
\begin{align}
  \sum\fctrange \deg_\fct P \cdot C_\fct
  & \le
  \delta \wts[1] B\dimp
  \left(
    \cst{vs-ht} \Lambda^{2\,f(u)}
    + 2 \bigl(
      \cst{vs-ht} + \dimp B^2
    \bigr)
  \right)
  \\ & \le \label{e:apx-1a-sum}
  \delta \wts[1] \left(
    \Lambda^{3f(\vdim)} \bigl(
      3 \cst{vs-ht} \cdot \cst{vs-deg-prod}^{-1}
      + 2\dimp^2 \cst{vs-deg-prod}^{-2}
    \bigr)
  \right)
  \pmm.
\end{align}

Intéressons-nous maintenant à la hauteur de la famille \( (P_\place)_\place \),
dont les normes locales sont données par le point~\ref{i:norm-p-der}
page~\pageref{i:norm-p-der}. Compte tenu de \eqref{e:nv-varfca} et du dernier
point du scolie~\ref{s:aux-co}, il vient
\todo[un terme oublié en reportant le résultat du scolie]
\begin{align}
  \hautl[1]{ (P_\place)_\place }
  & \le
  \hautl[1]{ (\fauxv)_\place }
  + \sum\fctrange
  \bigl( 2 (\dimp - \vdim*) \lgr{\dermp*} \bigr) \Bigl(
    \hautl[1]{ \varfc* }
    + \vdeg* (\vdim* + 1) \ln( \vdegp* (\dimp+1) )
    \\ & \quad \qquad + 2(\vdeg*+1) \ln 2 + \ln \vdim* + 3 \ln \vdeg*
  \Bigr)
  + \vdim* (\Dir*) \ln 2
  \\ & \le
  \hautl[1] \faux + \delta C_\wembcl
  + \sum\fctrange
  2 \dimp \frac{\epsi \delta}{\sigma} \wts* (\puiss-1)^{\delta^\fct_\puiss}
  \Bigl(
    \hautl[1]{ \varfc* }
    + B (\genre + 1) \ln( B (\dimp+1) )
    \\ & \quad \qquad + 2 (B + 1) \ln 2 + 3 \ln B + \ln \genre
  \Bigr)
  + 3 \delta \wts* (\puiss-1)^{\delta^\fct_\puiss} \genre \ln 2
  + o(\delta)
\end{align}
en utilisant l'hypothèse \( \wtsum(\dermp) \le \epsi\delta / \sigma \) et la
définition~\eqref{e:d-dp-def} de \( d' \), ainsi que \( \epsz < 1 \) et \(
  \vdim* \le \genre \). On utilise alors d'une part le scolie~\ref{s:h1-aux}
et la proposition~\ref{p:build-aux}, d'autre part
l'hypothèse~\eqref{e:varset-ht} et par ailleurs le lemme~\ref{l:sum-wts} pour
écrire
\begin{align}
  \frac{ \hautl[1]{ (P_\place)_\place } }{\delta + o(\delta)}
  & \le
  \wts[1] \Lambda^{2\,f(u)} \bigl(
    5\puiss \cst{vs-ht}
    + 2 (\dimp + \puiss^2) \cst{vs-deg-prod}^{-2}
  \bigr)
  + C_\wembcl
  + 2 \dimp \frac{\epsi}{\sigma} \cst{vs-ht} \Lambda^{2\,f(u)} \wts[1]
  + 2 \wts[1] \dimp B^2
  \\ & \le \label{e:apx-1a-ht}
  \wts[1] \Lambda^{2\,f(u)} \bigl(
    (5\puiss + 2\dimp \epsi / \sigma) \cst{vs-ht}
    + 2 (2\dimp + \puiss^2) \cst{vs-deg-prod}^{-2}
  \bigr)
  + C_\wembcl
  \pmm.
\end{align}
Il ne reste alors plus qu'à substituer~\eqref{e:apx-1a-ht}
et~\eqref{e:apx-1a-sum} dans~\eqref{e:apx-1a-split} pour obtenir
\begin{align}
  \frac{ \ratfh{1'} }{ \delta+o(\delta) }
  & \le
  \wts[1] \Bigl(
    - \frac38 \hautn{\ex[1]} \expapx \epsi
    + \Lambda^{2\,f(u)} \bigl(
      (5\puiss + 2\dimp \epsi / \sigma) \cst{vs-ht}
      + 2 (2\dimp + \puiss^2) \cst{vs-deg-prod}^{-2}
    \bigr)
    \\ & \qquad
    + \Lambda^{3f(\vdim)} \bigl(
      3 \cst{vs-ht} \cdot \cst{vs-deg-prod}^{-1}
      + 2\dimp^2 \cst{vs-deg-prod}^{-2}
    \bigr)
  \Bigr)
  + C_\wembcl
  \\ & \le \label{e:ratfh1a}
  \wts[1] \Bigl(
    - \frac38 \hautn{\ex[1]} \expapx \epsi
    + \Lambda^{3f(\vdim)} \bigl(
      ( 1 + 3 \cst{vs-deg-prod}^{-1} + \frac{2\dimp \epsi}{\sigma} )
      \cst{vs-ht}
      + 3\dimp^2 \cst{vs-deg-prod}^{-2}
    \bigr)
  \Bigr)
  + C_\wembcl
\end{align}

En substituant cette dernière estimation ainsi que \eqref{e:ratfh4},
\eqref{e:ratfh3}, \eqref{e:ratfh2}, \eqref{e:ratfh1b} dans \eqref{e:prod-lem},
il vient
\begin{align} \label{e:prod-lem-final}
  0
  & \le
  \delta
    \wts[1] \Bigl(
      \alpha \hautn{\ex[1]} + \beta
    \Bigr) + o(\delta \wts[1])
\end{align}
avec :
\begin{align}
  \alpha
  & =
  2 \puiss \epsz
  + 3 (\puiss - 1) \Vcos
  + \frac{2 (\puiss - 1) \epsi}{\sigma}
  - \frac38 \expapx \epsi
  \\
  \beta
  & =
  (2\epsz + 2 + \frac\epsi\sigma) (\cst{vs-ht} + \ln B)
  + \Lambda^{3f(\vdim)} \bigl(
    ( 1 + 3 \cst{vs-deg-prod}^{-1} + \frac{2\dimp \epsi}{\sigma} ) \cst{vs-ht}
    + 3\dimp^2 \cst{vs-deg-prod}^{-2}
  \bigr)
  \\ & \le
  \Lambda^{3f(\vdim)} \bigl(
    ( 3 \cst{vs-deg-prod}^{-1} + 2\dimp ) \cst{vs-ht}
    + 4\dimp^2 \cst{vs-deg-prod}^{-2}
  \bigr)
  \pmm.
\end{align}

Montrons que le terme dominant noté \( \alpha \) ci-dessus est négatif et même
inférieur à \( - \frac{3\expapx \epsi}{16} \). La définition~\eqref{e:def-epsz}
de \( \epsz \) assure que
\begin{equation} \label{e:epsz-ct-extra}
  2 \puiss \epsz
  \le \frac{\expapx \epsi}{16}
  \pmm,
\end{equation}
tandis que la définition~\eqref{e:def-Vcos} de \( \Vcos \) assure que
\begin{equation} \label{e:Vcos-ct-extra}
  3 (\puiss-1) \Vcos
  \le \frac{\expapx \epsi}{16}
  \pmm.
\end{equation}
Enfin, le choix de \( \sigma \) dans l'énoncé de la proposition~\ref{p:extra}
assure que
\begin{equation}
  \frac{2 (\puiss - 1) \epsi}{\sigma}
  \le \frac{\expapx \epsi}{16}
  \pmm.
\end{equation}
Par ailleurs, en remarquant que \( \vdim > \puiss \), la
définition~\eqref{e:def-Vbig} de \( \Vbig \) montre que
\begin{equation} \label{e:Vbig-ct-extra}
  \frac{3\expapx \epsi}{16} \hautn{\ex[1]}
  \ge
  \Lambda^{3f(\vdim)} \bigl(
    ( 3 \cst{vs-deg-prod}^{-1} + 2\dimp ) \cst{vs-ht}
    + 4\dimp^2 \cst{vs-deg-prod}^{-2}
  \bigr)
\end{equation}
et que le membre de droite de~\eqref{e:prod-lem-final} est négatif,
contradiction qui achève la preuve de la proposition~\ref{p:extra}.



\section{Application du théorème du produit et conclusion}
\label{sec:thm-prod}

La section précédente a montré que \( \faux** \) était d'indice
élevé en \( \ex \). Nous allons maintenant en déduire l'existence d'une forme
\( T \) comme dans la conclusion de la proposition~\ref{p:varset-notmin}, en
utilisant le fait suivant (théorème~7.1 page~149 de \cite{remivds}),
conséquence du théorème du produit .

\begin{fact} \label{f:thm-prod}
  Soient \( x_1, \dots, x_\puiss \) un point rationnel de \( \proj{\vdim_1}
    \times \dots \times \proj{\vdim**} \) et \( \vdim = \vdim_1 + \dots +
    \vdim** \). On considère une forme \( G \) de degré \( b \in (\N
    \minusset 0)^\puiss \) et on suppose que
  \begin{enumthm}
    \item \( \indg b[x] G \ge \alpha \) ;
    \item \(
        \frac{b_\fcti}{b_{\fcti+1}}
        \ge
        \left( \frac\puiss\alpha \right)^\vdim
      \)
      pour tout \( \fcti \in \set{1, \dots, \puiss-1} \) ;
    \item \(
        \frac\alpha\puiss < \left( \frac{\ln(\vdim+1)}{2\vdim^2}\right)^\vdim
      \).
  \end{enumthm}
  Il existe alors un \( \fct \in \set{1, \dots, \puiss} \) et une forme
  \( T \in \cdn[ \vmp*[0], \dots, \vmp*[\vdim*] ] \) non nulle, telle que
  \begin{enumthm}
    \item \( x \in \zeros T  \) ;
    \item \( \deg T
        \le
        \left( \frac\puiss\alpha \right)^\vdim
      \) ;
    \item la hauteur de \( T \) satisfait à
      \begin{align}
        b_\fct \, \hautl[\infty] T
        & \le
        \vdim*
        \left( \frac\puiss\alpha \right)^\vdim
        \left(
          \hautl[\infty] G
          + \sum_{\fcti=1}^\puiss \bigl(
            b_\fcti (\stoll{\vdim[\fcti]} + \ln 2) + \sqrt{\vdim[\fcti]}
          \bigr)
          + \frac{\vdim-1}2 \ln \lgr b
        \right)
        \\ & \qquad
        + b_\fct
        \left( \frac\puiss\alpha \right)^\vdim (\vdim* + 1)
        \ln \left( \left( \frac\puiss\alpha \right)^\vdim (\vdim* + 1) \right)
        + b_\fct \ln \binom{\deg T + \vdim*}{\vdim*}
      \end{align}
      où l'on a noté \( \stoll{\vdim[\fcti]} = \frac12
        \sum_{\ind=1}^{\vdim[\fcti]} \sum_{\indi=1}^{\ind} \frac1\indi \) le
      nombre de \bsc{Stoll}.
  \end{enumthm}
\end{fact}

Pour appliquer ce fait, nous devons fabriquer à partir de \( \faux**
\) une forme sur \( \proj{\vdim_1} \times \dots \times \proj{\vdim**} \).
On considère à cet effet la projection linéaire \( \pi \) obtenue en
conservant les \( \vdim* + 1 \) premières coordonnées sur chaque facteur dans
le plongement \( \vadapt* \circ \Theta \). Ce plongement étant adapté, cette
projection fait apparaître \( \var \) comme un revêtement ramifié du produit
d'espaces projectif considéré. Algébriquement, ceci signifie que l'anneau des
coordonnées homogènes de \( \var \) est une extension de type fini de \( \cdn[
  \vmp[1][0], \dots, \vmp[1][\vdim*]; \dots; \vmp[\puiss][0], \dots,
  \vmp[\puiss][\vdim*]
  ]
\). Le fait suivant, qui résume le résultat de la page~148 de~\cite{remivds},
montre que la norme de \( \faux** \) dans cette extension est un choix
raisonnable.

\begin{fact}
  La forme \( N(\faux**) \in \cdn[
    \vmp[1][0], \dots, \vmp[1][\vdim*]; \dots;
    \vmp[\puiss][0], \dots, \vmp[\puiss][\vdim*]
    ] \) possède les propriétés suivantes :
  \begin{enumthm}
    \item \( \inda[\pi(\ex)] N(\faux**) \ge \inda** \faux** \) ;
    \item \( \deg N(\faux**) = (\prod\fctrange \vdeg*) \deg \faux** \) ;
    \item \(
        \hautl[\infty] N(\faux**)
        \le
        (\prod\fctrange \vdeg*) \hautl[\infty]{ \faux** } + o(\delta)
      \).
  \end{enumthm}
\end{fact}

On note désormais \( G = N(\faux**) \) puis \( \vdeg = \prod\fctrange \vdeg*
\) et \( b = \bigl( D \delta \wts* (2\wtw* + \epsz) \bigr)_\fct \).
On souhaite alors appliquer le fait~\ref{f:thm-prod} à \( G \) avec \( \alpha
  = \puiss \Lambda^{-2f(u)} \) et \( x = \pi(\ex) \). Il s'agit tout d'abord
de vérifier que les hypothèses sont bien satisfaites.

Commençons avec l'indice : pour tout \( \imp \in (\N^{\dimp+1})^\puiss \) on a
\begin{align}
  \sum\fctrange \frac{ \lgr{\imp*} }{ b_\fct }
  & \ge
  \frac1{ 3 \delta \vdeg }
  \sum\fctrange
  \frac{ \lgr{\imp*} }{ \wtw* \wts* }
\end{align}
par définition de \( b \) et en observant que \( \epsz < 1 \), de sorte qu'en
utilisant successivement la proposition~\ref{p:extra} puis les
hypothèses~\ref{e:varset-deg} et~\ref{e:varset-deg-prod} il vient :
\begin{equation}
  \indg b[x] G
  \ge
  \frac1{ 3 \delta \vdeg } \inda[x] G
  \ge
  \frac{ \eps \epsi }{ 96 (\puiss-1) B D }
  \ge
  \alpha
\end{equation}
qui est justement la première hypothèse à satisfaire.

Pour la deuxième, en utilisant la définition de \( \wtw \) et le fait que \(
  \epsz < 1 \), puis deux fois~\eqref{e:wt-ratio} et enfin
l'hypothèse~\eqref{e:Vfar}, on écrit
\begin{equation}
  \frac{ b_\fct }{ b_{\fct+1} }
  =
  \frac{
    (2\wtw* + \epsz) \wts*
  }{
    (2\wtw[\fct+1] + \epsz) \wts[\fct+1]
  }
  \ge
  \frac1\puiss \,
  \frac{ \wts* }{ \wts[\fct+1] }
  \ge
  \frac1{4\puiss} \,
  \frac{ \hautn{ \ex[\fct+1] } }{ \hautn{ \ex* } }
  \ge
  \frac{ \Vfar }{ 4\puiss }
  \pmm.
\end{equation}
Par ailleurs, \( \bigl( \frac\puiss\alpha \bigr)^\vdim = \Lambda^{2\vdim
    f(\vdim)} \) d'après le choix de \( \alpha \). Pour \( \vdim \) compris
entre \( \puiss \) et \( \puiss\genre \), cette quantité est majorée par \(
  \Lambda^{2\puiss f(\puiss)} \) vu la définition de \( f \). La
définition~\eqref{e:def-Vfar} de \( \Vfar \) implique alors directement que la
deuxième hypothèse est satisfaite.

Pour la troisième, on observe que pour tout entier \( \vdim \ge \puiss\genre
\) on a
\begin{equation}
  \left( \frac{\ln(\vdim+1)}{2\vdim^2}\right)^\vdim
  \ge
  \left( \frac{\ln(\puiss\genre + 1)}{2(\puiss\genre)^2}\right)^{\puiss\genre}
  \ge
  (\sqrt2 \puiss\genre)^{-2\puiss\genre}
  \ge
  \Lambda^{-2}
\end{equation}
d'après la définition~\eqref{e:def-Lambda} de \( \Lambda \) (troisième
argument du maximum). Comme \( f(\vdim) \ge 1 \) pour tout \( \vdim \), la
troisième hypothèse est également satisfaite.




\endinput

% vim: spell spelllang=fr
