% !TEX root = main.tex

\chapter{Inégalité de \bsc{Mumford}} \label{chap:mumford}

\section{Énoncé principal}

On prouve ci-dessous une inégalité de \bsc{Mumford} explicite dans le cas
particulier où \( \varapx \) est une sous-variété abélienne de \( \va \),
qu'on notera donc désormais \( \vai \). Le résultat principal de ce chapitre
est le suivant.

\begin{thm} \label{t:mumford}
  Soient \( \vai \) une sous-variété abélienne de \( \va \), \( \p x \) et \(
    \p y \) deux points de \( \va(\Qbar) \) satisfaisant la condition
  d'approximation (\eqref{e:HA}) ci-dessus. On note \( d = \deg \vai \), \( l
    = \dim\vai + 1 \) et on choisit \( \phi > 0 \) et \( \rho > 0 \) tels que
  \( ld^2((\rho^2/4) + 2\phi + \rho\phi)< \eps \). On pose
  \( B = (4^g-1)h(\va) + 3g\log(2) \) et
  \(
    C = 3\cdot4^gh(\va)
    + \log\Bigl(
      10\sqrt2\cdot 2^{7g/2} (n+1)^6
      \bigl(\frac{16e\sqrt 2}{30}(n+1)^{9/2}\bigr)^{ld}
    \Bigr)
  \). On suppose en outre :
  \begin{enumerate}
    \item \(
        \cos(\p{x},\p{y}) \ge 1-\phi
      \) ; \label{i:m-cone}
    \item \(
        \Hautn(\p{x}) \le \Hautn(\p{y}) \le (1+\rho)\Hautn(\p{x})
      \) ; \label{i:m-sect}
    \item \(
        \Hautn(\p{x})
        >
        [h(\vai) + C + B(ld + \eps/d)]
        \cdot [\eps/d - ld((\rho^2/4) + 2\phi + \rho\phi)]^{-1}
      \). \label{i:m-loin}
  \end{enumerate}
  On a alors \( \p x - \p y \in \vai(\Qbar) \).
\end{thm}

Les conditions sur \( \p{x} \) et \( \p{y} \) ont une interprétation
géométrique très simple dans l'espace de \bsc{Mordell-Weil}. En effet
\ref{i:m-cone} demande que les points se trouvent dans un même demi-cône
d'angle petit, \ref{i:m-sect} précise qu'ils doivent se trouver dans un même
secteur de petite largeur exponentielle et \ref{i:m-loin} que ce secteur est
suffisamment loin de l'origine. Le théorème~affirme alors qu'il y a, modulo \(
  \vai(\Qbar) \), au plus une approximation exceptionnelle dans un secteur de
cône d'angle \( \arccos(1-\phi) \), de largeur exponentielle \( 1+\rho \),
extérieur à une sphère de rayon adéquat centrée en l'origine.



\section{Comportement métrique des opérations}

La situation d'approximation considérée met en jeu plusieurs types de
géométrie sur \( \va \) : la géométrie euclidienne de son espace de
\bsc{Mordell-Weil}, la géométrie algébrique et projective de la variété
plongée, une géométrie métrique, locale, en chaque place de \( S \). On
s'attend à ce que les métriques locales sur \( \va \) se comportent de façon
agréable vis-à-vis des structures géométrique et arithmétique, par exemple que
les opérations soient uniformément continues, voire lipschitziennes. On montre
ici que c'est bien le cas et on explicite une valeur admissible pour la
constante de \bsc{Lipschitz}.

\begin{prop} \label{MetricOp}
  Il existe une constante \( \cst{CMetricOp} \) telle que, pour tous \( \p x
  \), \( \p x' \), \( \p y \), \( \p y' \) dans \( \va(\C_v) \), on a :
  \begin{equation}
  \max\bigl(\Distv(\p x \pm \p y, \p x' \pm \p y')\bigr) \le \cst{CMetricOp}
  \cdot \max \bigl(\Distv(\p x, \p x'), \Distv(\p y, \p y')\bigr) \pmm,
  \end{equation}
  où dans le membre de gauche il faut prendre le même signe sur les deux
  termes.  De plus, on peut choisir
  \begin{equation}
    \newcst{CMetricOp}
    =
    c_v(\va)^3 \av
    2^{-5g/2} \bigl( 5\sqrt2\cdot2^{7g/2}(n+1)^{9/2} \bigr)^\dv
    \pmm.
  \end{equation}
\end{prop}

\startout[À intégrer avec le paragraphe suivant]
On sera naturellement amené à considérer des situations multiprojectives. Dans
ce cas, plusieurs notions de distances peuvent être considérées, dont seules
les deux plus élémentaires nous intéressent ici : il s'agit du maximum des
distances sur chaque facteur d'une part, et de la distance déduite d'un
plongement de \bsc{Segre} d'autre part. Ces distances se comparent bien en
général ; on écrit ici la comparaison dans le cas de \( (\Proj^n)^2 \) (cas \(
  q = 2 \), \( d=(1,1) \) de \cite[lemme~4.3]{remgdmp}) :
\begin{equation}
  \max_{i \in \{1, 2\}}
  \Distv(\p x^{(i)}, \p y^{(i)})
  \le
  \Distv(s(\p x), s(\p y))
  \le
  \sqrt 2^{\,\dv} \max_{i \in \{1, 2\}} \Distv(\p x^{(i)}, \p y^{(i)})
  \pmm,
\end{equation}
où \( \p x = (\p x^{(1)}, \p x^{(2)}) \in (\Proj^n)^2 \) et \( s : (\Proj^n)^2
  \to \Proj^{n^2+2n} \) est le plongement de \bsc{Segre}.
\stopout

Pour établir ce résultat, on va en fait étudier les propriétés métriques du
morphisme \( \xi : \va^2 \to \va^2,\ (\p x, \p y) \mapsto (\p x + \p y, \p x -
  \p y) \). Ce dernier présente en effet l'avantage de pouvoir être représenté
globalement par une famille \( F \) de polynômes totalement explicite (en
particulier, de degré et hauteur contrôlés) à condition toutefois de regarder
\( \va^2 \) comme plongée dans \( \Proj^N \) (où \( N = n^2+2n \)) par un
plongement de \bsc{Segre}. Si l'on note \( \p z \) (resp. \( \p z' \)) l'image
de \( (\p x, \p y) \) (resp. \( (\p x', \p y') \)) par ce plongement, on est
ainsi ramené à montrer que \( \Distv(\xi \p z, \xi \p z') \le \cst{CMetricOp}
  \sqrt2^{\,-\dv} \Distv(\p z, \p z') \). Par ailleurs on pourra supposer \(
  \Distv( \p z, \p z') \le \sqrt2\, \cst{CMetricOp}^{-1} \), car sinon la
conclusion du lemme~est vide, compte tenu du fait que la distance est bornée
par \( 1 \). On utilisera alors un développement de \bsc{Taylor} des formes \(
  F \) représentant \( \xi \) au voisinage de \( \p z \).

En comparant les termes d'ordres \( 1 \) et \( 2 \) au terme principal dans ce
développement, va apparaître une quantité \( \Onv{F(z)} / (\Onnv{F}\Onv{z}^2)
  \le 1 \), qui détermine en grande partie le rayon sur lequel le
développement de \bsc{Taylor} donne une majoration utile. Il est donc
indispensable de pouvoir minorer cette quantité uniformément en \( \p z \). Le
lemme~suivant donne une telle minoration en fonction de la constante locale \(
  c_v(\va) \). Notons que cette minoration est la seule étape de la preuve où
intervient le fait que le morphisme \( \xi \) est donné par les opérations de
\( \va \), le reste s'appliquant en fait à n'importe quel morphisme d'espaces
projectifs.

\begin{lem} \label{TvIndepF}
  Il existe un choix de \( F \) tel que pour tout \( \p z \in \va(\C_v) \) on
  ait
  \begin{equation}
    \frac{\Onv{F(z)}}{\Onnv{F}\Onv{z}^2} \ge \cst{Tv} \pmm,
  \end{equation}
  avec \(
    \newcst{Tv}
    =
    \av 2^{5g/2} 2^{-7g\dv/2} (n+1)^{-3\dv/2} c_v(\va)^{-3}
  \).
\end{lem}

\begin{proof}[Démonstration (du lemme~\ref{TvIndepF}).]
  On utilise le formulaire de la section~\ref{sec:form-ab2}.

  On commence par estimer \( \Onnv{F} \). Chaque \( F_{ij} \) s'écrit comme
  une somme de \( 4^g \) termes de norme \( \Onv[v, \infty]{\truc} \) au plus
  \( \av 2^{-g} \Onv \vai^2 \) ; on a donc \( \Onv {F_{ij}} \le 2^{g\dv} \av 2
    ^{-g} \Onv \vai ^2 \) et \( \Onnv F \le  \sqrt{N+1}^\dv 2^{g\dv} \av 2
    ^{-g} \Onv \vai ^2 \). Pour la suite, on se base sur la minoration
  (\ref{NormDupl}) et le fait que \( \xi^2 \) est le morphisme de
  multiplication par \( 2 \) sur les deux facteurs. On a ainsi
  \begin{equation}
    F^2(\tilde\Delta)
    = (\xi^2)^*\tilde\Delta
    = [2]^* \tilde\Delta = G(\Delta) \otimes G(\Delta)
    \pmm.
  \end{equation}
  En appliquant ceci en un point \( \p z = (\p x, \p y) \), on peut, par
  homogénéité remplacer \( \Delta(\p x) \) par n'importe quel système de
  coordonnées \( x \) de \( \p x \) et faire de même pour \( \p y \), puis
  choisir \( z = x \otimes y \) comme représentant de \( \p z \). Il vient
  alors :
  \begin{equation}
    \frac {\Onv{F^2(z)}} {\Onv z ^4}
    = \frac {\Onv{G(x)} \Onv{G(y)}} {\Onv x ^4 \Onv y ^4}
    \ge \av 2 ^{2g} 4^{-2g\dv} \Onv\coa ^{-6}
    \pmm.
  \end{equation}
  Or,
  \begin{equation}
    \frac {\Onv{F^2(z)}} {\Onv z ^4}
    \le \frac {\Onnv F \Onv{F(z)}^2} {\Onv z ^4}
    = \Onnv F ^3 \left( \frac {\Onv{F(z)}} {\Onnv F \Onv z ^2} \right) ^2
    \pmm,
  \end{equation}
  d'où (en se rappelant que \( N+1 = (n+1)^2 \))
  \begin{align*}
    \left(\frac {\Onv{F(z)}} {\Onnv F \Onv z ^2} \right)^2
    & \ge \av2^{2g} 4^{-2g\dv} \Onv\coa^{-6} \Onnv F ^{-3} \\
    & \ge \av2^{5g} 2^{-7d\dv} (n+1)^{-3\dv} c_v(\va)
  \end{align*}
  et le lemme.
\end{proof}

\begin{proof}[Démonstration (de la proposition~\ref{MetricOp}).]
  Il s'agit de montrer que \( \Onv{F(z')} \) n'est pas trop petit devant \(
    \Onv {F(z)} \) et que \( \Onv{F(z) \wedge F(z')} \) n'est pas trop grand
  devant \( \Onv {F(z)}^2 \). Comme annoncé, on supposera que \( \Distv (\p z,
    \p z') < \sqrt2^\dv\, \cst{CMetricOp}^{-1} \), c'est-à-dire, vu que \(
    \cst{CMetricOp} = (5\sqrt2\cdot N\sqrt{N+1})^\dv / \cst{Tv} \), que \(
    \Distv (\p z, \p z') = \cst{Tv}\eps_1/(5 N \sqrt{N+1})^\dv \) pour un
  certain \( \eps_1 < 1 \). On peut en particulier appliquer le
  lemme~\ref{l:dv-common-i} à \( z \) et \( z' \) (en remplaçant \( n \) par \( N \))
  et supposer, quitte à renuméroter les coordonnées, que \( \av{z'_0} \ge
    (N+1)^{-\dv/2}\Onv{z'} \) et \( z_0 \neq 0 \). On écrit alors le
  développement de \bsc{Taylor} homogène de \( F \) comme suit :
  \begin{equation}
    F(z')
    =
    \frac{ (z'_0)^2 }{ z_0^2 } F(z)
    + \sum_{1 \le i \le N} R_i
    + \sum_{1 \le i, j \le N} R_{ij}
    \pmm,
  \end{equation}
  où
  \begin{equation}
    R_i
    =
    \frac{z'_0}{z_0} \frac{\partial F}{\partial Z_i}(z)
    \left( z'_i - \frac{z'_0}{z_0} z_i \right)
  \end{equation}
  et
  \begin{equation}
    R_{ij}
    =
    \frac12 \frac{ \partial^2 F }{ \partial Z_i \partial Z_j }(z)
    \left( z'_i - \frac{z'_0}{z_0} z_i \right)
    \left( z'_j - \frac{z'_0}{z_0} z_j \right)
    \pmm.
  \end{equation}
  On utilise alors l'hypothèse sous la forme
  \begin{equation}
    \av{z'_i - \frac{z'_0}{z_0} z_i}
    \le
    \Onv{z \wedge z'} / \av{z_0}
    \le
    \frac{\cst{Tv}\,\eps_1}{(5N\sqrt{N+1})^\dv}
    \frac{\Onv z \Onv{z'}}{\av{z_0}}
    \pmm.
  \end{equation}
  Il vient alors, grâce au lemme~\ref{TvIndepF},
  \begin{align*}
    \frac{ \Onv{R_i} }{ \Onv{F(z) \cdot {z_0'}^2/z_0^2} }
    & \le
    \frac{ \Onv{z'} }{ \av{z'_0} }
    \frac{ \Onnv{F} \Onv z^2 }{ \Onv{F(z)} }
    \frac{ 2^\dv \cst{Tv}\,\eps_1 }{ (5N\sqrt{N+1})^\dv }
    \\ & \le
    \left( \frac2{ 5N\sqrt{N+1} } \right)^\dv
    \frac{ \Onv{z'} }{ \av{z'_0} } \eps_1
    \le
    \left( \frac2{5N} \right)^\dv \eps_1
    \pmm.
  \end{align*}
  On obtient de même
  \begin{equation}
    \frac{ \Onv{R_{ij}} }{ \Onv{F(z) \cdot {z_0'}^2/z_0^2} }
    \le
    \left( \frac1{25N^2} \right)^\dv \eps_1
    \pmm,
  \end{equation}
  puis en sommant : \( \Onv{\sum R_i + \sum R_{ij}} \le (1/2)^\dv \Onv{F(z)
      \cdot {z_0'}^2/z_0^2} \eps_1 \). Enfin, les inégalités triangulaire aux
  places infinies et ultramétrique aux places finies donnent \( \Onv{F(z')}
    \ge (1/2)^\dv \Onv{F(z) \cdot {z_0'}^2/z_0^2} \). Afin d'estimer la
  distance, il reste à majorer \( \Onv{F(z) \wedge F(z')} \). Pour ce faire,
  on développe le second facteur, on remarque que le premier terme du produit
  est nul et on majore brutalement le terme restant par le produit des normes
  ; il vient ainsi :
  \begin{align*}
    \Distv(\xi(\p z), \xi(\p z'))
    & =
    \frac{ \Onv{F(z) \wedge F(z')} }{ \Onv{F(z)} \Onv{F(z')} }
    \\ & \le
    \frac{
      2^\dv \Onv{\sum R_i + \sum R_{ij}} \Onv{F(z)}
    }{
      \Onv{F(z)} \Onv{F(z) \cdot {z_0'}^2/z_0^2}
    }
    \\ & \le
    \eps_1
    \pmm,
  \end{align*}
  qui, vu la définition de \( \eps_1 \), équivaut à la conclusion de la
  proposition.
\end{proof}


\section{Inégalité de \bsc{Liouville}}

On établit ici un analogue de l'inégalité bien connue de \bsc{Liouville}.
L'énoncé suivant est une application directe de la formule du produit :

\begin{prop} \label{PLiouvilleMal}
  Soit \( V \) une sous-variété de \( \projd \) définie sur un corps de
  nombres \( \cdn \) et \( \placesapx \) un sous-ensemble fini de \( M(\cdn)
  \). Pour tout point \( x \in \Proj^n(\Qbar) \), on a soit \( x \in V(\Qbar)
  \), soit
  \begin{equation}
    \prod_{v \in \placesapx} \distalgv x V ^\degv
    \ge
    \frac1{ \hautm[2] V \hautm[2] x ^{ld} }
    \pmm,
  \end{equation}
  où \( l = \dim V + 1 \), \( d = \deg V \) et \( \degv = [\cdn_v : \Q_v] /
  [\cdn : \Q] \).
\end{prop}

\begin{proof}
  On remarque en effet que
  \begin{equation}
    \prod_{v \in M(\cdn)} \distalgv x V ^\degv
    =
    \prod_{v \in M(\cdn)} \frac{ \mvp{\md_x f} }{ \mvp f \nv2 x^{ld} }
    =
    \frac{ \hautm[2]{ \md_x f } }{ \hautm[2] V \hautm[2] x ^{ld} }
    \pmm.
  \end{equation}
  Or, d'une part cette quantité est majorée par
  \( \prod_{v \in \placesapx} \distalgv x V ^\degv \) (la distance est bornée
  par \( 1 \)) et d'autre part \( \hautm[2]{ \md_x f } \ge 1 \) par la formule
  du produit dès que \( x \not\in V(\Qbar) \).
\end{proof}

Cette version évidente de l'inégalité présente l'inconvénient de faire
apparaître la dimension de \( V \) dans l'exposant de \( \hautm[2] x \), ce
qui n'est pas naturel.




\section{Inégalité de \bsc{Mumford}}

La stratégie de la preuve est la suivante : si on a deux approximations
exceptionnelles, suffisamment proches dans l'espace de \bsc{Mordell-Weil} (au
sens des hypothèses \ref{i:m-cone} et \ref{i:m-sect} du
théorème~\ref{t:mumford}) et de hauteur assez grande, on fabrique, (en prenant
leur différence, sous l'hypothèse que \( \vai \) est un sous-groupe) une
approximation de qualité telle qu'elle appartiendra nécessairement à \( \vai
\), d'après une des inégalités de \bsc{Liouville} énoncées à la section
précédente.

\begin{lem} \label{Precis}
  Si \( \p z = \p x - \p y \), où \( \p x \) et \( \p y \) satisfont
  l'hypothèse (\eqref{e:HA}) du théorème~\ref{t:mumford}, on a \( \prod_{v \in
      S} \Distv(\p z, \vai)^{e_v} \le \cst{CPrecis}H^{-\eps/d} \), avec
  \begin{equation}
    \newcst[]{CPrecis}
    =
    H(\va)^{3\cdot4^g}
    10\sqrt2\,(n+1)^6
    2^{7g/2}
    \left(
      \frac{(2n^2+1)(n+1)^2}{2n}
      e^{ \gamma_{\frac{(n+1)n}{2}} + \gamma_{n+1} }
    \right)^{ld}
  \end{equation}
  et \( H = \min\bigl(H(\p x), H(\p y)\bigr) \).
\end{lem}

\begin{proof}
  On a par hypothèse \( \max(\Distv(\p x, \vai), \Distv(\p y, \vai)) \le
    H^{-\lambda_v \eps} \).  On sait alors grâce au fait~\ref{f:closest-point}
  qu'il existe des points \( \p x',\p y' \in \vai(\C_v) \) tels que
  \begin{equation}
    \max(\Distv(\p x, \p x'), \Distv(\p y, \p y'))
    \le
    H^{-\lambda_v \eps/d} e^{\dv \gamma_{n+1}}
    \pmm.
  \end{equation}
  On a alors en utilisant successivement le lemme~\ref{l:dv-p-alg-gen} et la
  proposition~\ref{MetricOp} :
  \begin{align*}
    \Distv(\p z, \vai)
    & \le
    \cst{CPonctuelAlg}\, \Distv(\p x - \p y, \p x' - \p y')
    \\ & \le
    \cst{CPonctuelAlg}\, \cst{CMetricOp}\,
    \max(\Distv(\p x, \p x'), \Distv(\p y, \p y'))
    \\ & \le
    \cst{CPonctuelAlg}\, \cst{CMetricOp}
    \, e^{\dv \gamma_{n+1}}H^{-\lambda_v \eps/d}
    \pmm.
  \end{align*}
  On prend alors le produit sur \( v \in S \), en supposant (c'est le cas
  défavorable) que \( S \) contient toutes les places divisant \( 2 \) ou \(
    \infty \).  Comme \( c_v(\va) \ge 1 \), on a \( \prod_{v \in S} c_v(\va)
    \le \prod_{v \in M(\cdn)} c_v(\va) \le H(\va)^{4^g} \). Vu la
  normalisation des \( \lambda_v \), on a ainsi la conclusion du lemme.
\end{proof}

\begin{lem} \label{Petit}
  Soient \( \p{x} \) et \( \p{y} \) satisfaisant aux hypothèses \ref{i:m-cone}
  et \ref{i:m-sect} du théorème~\ref{t:mumford}, notons \( \p z \) leur
  différence. On a alors \( \Hautn(\p z) \le (\rho^2/4 + 2\phi + \rho\phi)
    \Hautn(\p{x}) \).
\end{lem}

\begin{proof}
  On note \( \langle\truc; \truc \rangle \) le produit scalaire et \(
    \Onv[N]{\truc} = \sqrt{\Hautn(\truc)} \) la norme dans l'espace de
  \bsc{Mordell-Weil}. On remarque de plus que l'hypothèse \ref{i:m-sect}
  implique
  \( \Onv[N]{\p{y}} \le (1+\rho)^{1/2}\Onv[N]{\p{x}} \le (1+\rho/2)
    \Onv[N]{\p{x}} \). Il vient alors :
  \begin{align*}
    \Hautn(\p{z})
    & =
    \Onv[N]{\p{x}-\p{y}}^2
    \\ & =
    \Onv[N]{\p{x}}^2 + \Onv[N]{\p{y}}^2 - 2\langle \p{x}; \p{y} \rangle
    \\ & =
    \left( \Onv[N]{\p{y}} - \Onv[N]{\p{x}} \right)^2
    + 2 \Onv[N]{\p{x}} \Onv[N]{\p{y}} \left( 1- \cos(\p{x}, \p{y}) \right)
    \\ & \le
    \left( \frac{\rho}{2}\Onv[N]{\p{x}} \right)^2
    + 2\left( 1+\frac{\rho}{2} \right)
    \Onv[N]{\p{x}}^2 \cdot \phi
    \\ & \le
    \bigl( (\rho^2/4) + 2\phi + \rho\phi \bigr)
    \Hautn(\p{x})
    \pmm.
    \qedhere
  \end{align*}
\end{proof}

\begin{proof}[Démonstration (du théorème~\ref{t:mumford}).]
  En utilisant les conclusions des deux lemmes précédents, on montre que \( \p
    z \) ne satisfait pas à la conclusion la proposition~\ref{PLiouvilleMal}
  dès que \( \p x \) satisfait l'hypothèse \ref{i:m-loin} du
  théorème~\ref{t:mumford}.  On a besoin de pouvoir comparer les hauteurs
  projective et normalisée. On utilise à cet effet le lemme~3.9 de
  \cite{daphimhva2} et la remarque subséquente : \( \lvert\Hautn(\truc) -
    h(\truc)\rvert \le B \), avec \( B \) comme dans l'énoncé du théorème. En
  injectant ceci dans le lemme~\ref{Precis}, on a \( \sum_{v \in S} e_v
    \log\Distv(\p z, \vai) \le -(\eps/d) \Hautn(\p x) + B\eps/d +
    \cst{CPrecis} \). De même, le lemme~\ref{Petit} donne \( h(\p z) \le
    ((\rho^2/4) + 2\phi + \rho\phi) \Hautn(\p x) + B \). Ces estimations
  contredisent la conclusion de la proposition~\ref{PLiouvilleMal} dès que
  \begin{equation}
    \bigl(\eps/d - ld((\rho^2/4) + 2\phi + \rho\phi)\bigr)
    \Hautn(\p x)
    >
    h(\vai) + \cst{CPrecis} + B(ld + \eps/d)
    \pmm.
  \end{equation}
  Or, un calcul facile montre que la constante \( C \) du théorème majore \(
    \cst{CPrecis} \) et que la relation précédente est impliquée par
  l'hypothèse \ref{i:m-loin} du théorème ; cette dernière oblige donc \( \p z
  \) à être sur \( \vai \).
\end{proof}

\begin{rem}
  En utilisant la proposition~\ref{pLiouvilleBien} en lieu et place de la
  proposition~\ref{PLiouvilleMal} ci-dessus, on obtient une variante du
  théorème, avec la même conclusion mais en affaiblissant la condition
  \( ld^2((\rho^2/4) + 2\phi + \rho\phi) < \eps \) en
  \( d^2((\rho^2/4) + 2\phi + \rho\phi) < \eps \) et en remplaçant la condition
  3 par
  \begin{equation}
    \Hautn(\p{x})
    >
    \bigl( h(\vai) + C + \log \cst{cLiouville} + B(d + \eps/d) \bigr)
    \cdot \bigl(\eps/d - d((\rho^2/4) + 2\phi + \rho\phi) \bigr)^{-1}
    \pmm.
  \end{equation}
\end{rem}


\cleardoublepage
\endinput

% vim: spell spelllang=fr
