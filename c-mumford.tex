% !TEX root = main.tex

\chapter{Inégalité de \bsc{Mumford}} \label{chap:mumford}

\section{Énoncés principaux}

On prouve ici deux inégalités de \bsc{Mumford} explicites : la première dans
le cas particulier où \( \avar \) est un translaté d'une sous-variété
abélienne de \( \va \), la deuxième dans le cas général. À part être plus
simple à démonter et servir d'échauffement pour le cas général, la version
particulière présente deux intérêts majeurs : elle s'énonce sur \( \Qbar \)
sans devoir se restreindre à un sous-groupe de rang fini de \( \va(\Qbar) \)
et les constantes obtenues y sont sensiblement meilleures.

\begin{thm} \label{t:mumford-grp}
  Soient \( \vai \) une sous-variété abélienne de \( \va \) et \( \avar \)
  un translaté de \( \vai \) par un point algébrique ;
  on note \( \adeg = \deg \vai \) et \( \adim = \dim\vai \).
  On choisit \( \eps > 0 \) puis \( \phi > 0 \) et \( \rho > 0 \) tels que
  \begin{equation} \label{e:rho-phi-grp}
    \frac{ \rho^2 }{ 4 } + \rho\phi + 2\phi
    \le
    \frac{ \eps }{ 2\adeg }
    \pmm.
  \end{equation}
  Si \( x \) et \( y \) sont deux points de \( \va(\Qbar) \) tels que
  \begin{align}
    0
    & <
    \distv z \avar
    \le
    \hautm2 z ^{-\wtapx\eps}
    \quad \forall v \in \placesapx
    \quad \text{où \( z \) est \( x \) ou \( y \)}
    \label{e:Mapx}
    \\
    \hautn x
    & \ge
    \frac2\eps
    \adeg (\adim + 1)
    \Bigl(
      \ln(\adeg)
      + (2 + \frac\eps\adeg) \htcmp
      + \frac12 \hlclab
      + 5 \ln(n+1)
    \Bigr)
    \quad
    \label{e:Mbig}
    \\
    \cos(x, y)
    & \ge
    1 - \phi
    \label{e:Mcos}
    \\
    \hautn x
    & \le
    \hautn y \le (1+\rho) \hautn x
    \label{e:Mclose}
  \end{align}
  alors \( x - y \in \vai(\Qbar) \).
\end{thm}

On remarquera que la conclusion obtenue est plus faible que celle qu'on
pourrait attendre naïvement, à savoir \( x = y \), qui ne semble pas
accessible pour le moment. Cette obstruction sera discutée plus en détails à
la section~\vref{sec:obstruction}.

Pour le cas général, l'énoncé est le suivant.

\begin{thm} \label{t:mumford-gen}
  Soit \( \avar \) une sous-variété de \( \va \) ; on note \( \adeg = \deg
    \avar \) et \( \adim = \dim\avar \). On fixe un sous-groupe \( \grp
    \subset \va(\Qbar) \) de rang fini \( r \), puis on pose
  \begin{equation}
    m
    =
    \bigl(
    2^{34} \cdot [\cdn : \Q] \hlclab \cdot \adeg
    \bigr)^{ (r+1) g^{ 5(\adim + 1)^2 } }
    + 1
  \end{equation}
  où \( \cdn \) est un corps de définition de \( \va \).
  On choisit alors \( \eps > 0 \) puis \( \phi > 0 \) et \( \rho > 0 \)
  tels que
  \begin{equation} \label{e:rho-phi-gen}
    \frac{ \rho^2 }{ 4 } + \rho\phi + 2\phi
    \le
    \frac{ \eps }{ \adeg^m (2m)^{(m+1)\adim} }
    \pmm.
  \end{equation}
  Si \( x_1, \dots, x_m \) est une famille de points de \( \grp \)
  telle que, pour tous \( i \) et \( j \) entre \( 1 \) et \( m \) :
  \begin{align}
    0
    & <
    \distv{ x_i }\avar
    \le
    \hautm2{ x_i }^{-\wtapx\eps}
    \quad \forall v \in \placesapx
    \label{e:Mapx-gen}
    \\
    \hautn{ x_m }
    & \ge
    \frac2\eps
    \adeg^{m-1} (2m)^{ (m+1)\adim + 1}
    \Bigl(
      8 \hautl\htpph \avar
      + 4 \adeg m^2 \bigl( \ln \adeg + \hlclab + \htcmp + \ln(n+1) \bigr)
    \Bigr)
    \quad
    \label{e:Mbig-gen}
    \\
    \cos(x_i, x_j)
    & \ge
    1 - \phi
    \label{e:Mcos-gen}
    \\
    \hautn{ x_m }
    & \le
    \hautn{ x_i }
    \le
    (1+\rho) \hautn{ x_m }
    \label{e:Mclose-gen}
  \end{align}
  alors il existe une sous-variété abélienne \( \vai \subset \va \) dont un
  translaté est contenu dans \( \avar \) et deux indices distincts \( i \)
  et \( j \) tels que \( x_i - x_j \in \vai \).
\end{thm}

Avant d'attaquer les preuves proprement dites, on démontre une inégalité de
\bsc{Liouville}, qui est intéressante en elle-même et sera un ingrédient
crucial dans les inégalités de \bsc{Mumford}, puis on explicite quelques
aspects métriques des opérations de \( \va \) : cette étape sera utile non
seulement pour la preuve des inégalités de \bsc{Mumford}, mais aussi pour
expliciter l'obstruction à la version naïve de cette inégalité.

Une fois réunis ces ingrédients techniques, on peut conclure directement dans
le cas des translatés de sous-variété abéliennes ; pour le cas général, on
étudie au préalable un morphisme « des différences » qui joue un rôle central
dans la preuve.


\section{Inégalité de \bsc{Liouville}} \label{sec:liouville}

On établit ici l'analogue suivant de la classique inégalité de
\bsc{Liouville}.

\begin{prop} \label{p:liouville}
  Soit \( V \) une sous-variété de \( \projd \), de dimension \( \adim \) et
  de degré \( \adeg \). Pour tout point \( x \in \projd(\Qbar) \), on a soit
  \( x \in V(\Qbar) \) soit
  \begin{equation}
    \prod_{v \in \placesapx} \distv x V ^\degv
    \ge
    \frac1{
      (n+1)^{3/2}
      (3\adeg)^{\adeg (\adim+1)}
      \, \hautm1{ \chow V }
      \, \hautm2 x ^\adeg
    }
    \pmm.
  \end{equation}
\end{prop}

Ce type d'inégalité se prouve de façon naturelle avec la distance algébrique
(section~\vref{sec:distv-cmp}). Cependant, si l'on traite directement le cas
général en utilisant la définition, on obtient \( \adeg (\adim+1) \) comme
exposant de \( \hautm2 x \). On commence donc par le cas d'une hypersurface,
où la distance algébrique admet une expression plus simple, puis on remarque
que toute variété est une intersection d'hypersurfaces.

On pourrait se ramener au cas d'un hyperplan, où la distance admet une
expression simple, par un plongement de \bsc{Veronese} (remodelé) comme dans
le lemme~\vref{l:hs-vero}, mais on obtient en fait de meilleurs constantes en
utilisant la propriété~\vref{p:dv-p2alg-hs}.

\begin{lem} \label{l:liou-hs}
  Soit \( V \) une hypersurface de \( \projd \), d'équation \( F \) et de
  degré \( \adeg \). Pour tout point \( x \in \projd(\Qbar) \), on a soit \( x
    \in V(\Qbar) \) soit
  \begin{equation}
    \prod_{v \in \placesapx} \distv x V ^\degv
    \ge
    \frac1{
      (17/8)^\adeg (n+1)^{3/2}
      \, \hautm2 F
      \, \hautm2 x ^\adeg
    }
    \pmm.
  \end{equation}
\end{lem}

\begin{proof}
  Plutôt que la proposition~\vref{p:dv-p2alg-hs}, on utilise en fait
  l'inégalité~\eqref{e:dv-p2alg-hs} établie au cours de sa preuve afin
  d'éviter une comparaison norme-mesure superflue. On remarque également qu'on
  peut supposer \( m = 1 \) dans cette comparaison. On a alors :
  \begin{align}
    \prod_{v \in \placesapx}
    \distv x V ^\degv
    & \ge
    \prod_{v \in M(\cdn)}
    \distv x V ^\degv
    \\ & \ge
    \prod_{v \in M(\cdn)}
    \left(
      \left(
        \frac{27}{26} (n+1)^{3/2}
        \left( \frac{164}{81} \right)^{\adeg}
      \right)^{-\dv}
      \frac{ \av{G(x)} }{ \nv2 x^{\adeg} \nv2 G }
    \right)^\degv
  \end{align}
  qui implique le résultat annoncé, \lat{via} une simple comparaison
  numérique pour la constante.
\end{proof}

\begin{proof}[\proofname{} de la proposition~\vref{p:liouville}]
  Si \( x \not\in V(\Qbar) \), la proposition~6.1 de~\cite{remdcl}
  donne une forme \( F \) homogène de degré \( \adeg \), contenant \( V \)
  mais pas \( x \), telle que
  \begin{equation}
    \hautm1 F
    \le
    \hautm1 V \cdot (\adeg + 1)^{ \adeg (\adim + 1) }
  \end{equation}
  en tenant compte des différences de hauteurs utilisées, comme observé dans
  la preuve du lemme~\vref{l:hs-choice}. Il suffit alors d'appliquer le
  lemme précédent à \( \zeros F \) et de remarquer que \( 17/8 \cdot
    (\adeg+1)^2 \le (3\adeg)^2 \) pour conclure.
\end{proof}



\section{Comportement métrique des opérations}

La situation d'approximation considérée met en jeu plusieurs types de
géométrie sur \( \va \) : la géométrie euclidienne de son espace de
\bsc{Mordell-Weil}, la géométrie algébrique et projective de la variété
plongée, une géométrie métrique, locale, en chaque place de \( \placesapx \).
On s'attend à ce que les métriques locales sur \( \va \) se comportent de
façon agréable vis-à-vis des structures géométrique et arithmétique, par
exemple que les opérations soient uniformément continues, voire
lipschitziennes. On montre ici que c'est bien le cas et on explicite une
valeur admissible pour la constante de \bsc{Lipschitz}.

\begin{prop} \label{p:addsub-dv}
  Pour tous \( x \), \( x' \), \( y \), \( y' \) dans \( \va(\C_v) \), on a :
  \begin{equation}
  \Distv(x \pm y, x' \pm y')
  \le
  \max \bigl( \Distv(x, x'), \Distv(y, y') \bigr)
  \cdot \hmclab* \bigl( 5 \sqrt2 (n + 1)^3 \bigr)^\dv
  \pmm,
  \end{equation}
  où dans le membre de gauche il faut prendre le même signe des deux côtés.
\end{prop}

\begin{proof}
  On considère le plongement de \bsc{Segre} \( s \colon (\projd)^2 \to
    \proj{n'} \) avec \( n' = n^2 + 2n \). D'après le lemme~4.3 (p.~121) de
  \cite{remgdmp}, appliqué avec \( q = 2 \) et \( \adeg = (1,1) \), ainsi que
  le paragraphe~2.3 (p.~103 notamment) de cette référence donnant le lien
  entre indices (d'une forme de \bsc{Chow} ou, par extension, d'une distance)
  et plongements de \bsc{Segre} (et de \bsc{Veronese} remodelé), on a
  \begin{equation} \label{e:dv-segre}
    1
    \le
    \frac{
      \Distv(s(x, x'), s(y, y'))
    }{
      \max \bigl( \Distv(x, x'), \Distv(y, y') \bigr)
    }
    \le
    2^{\dv/2}
  \end{equation}
  dans le cas où le dénominateur est différent de zéro (dans le cas contraire,
  on a \( x = x' \) et \( y = y' \) et le résultat est immédiat).

  On considère alors le morphisme
  \( \xi : \va^2 \to \va^2,\ (x, y) \mapsto (x + y, x - y) \), qui
  présente l'avantage de pouvoir être représenté globalement
  par une famille \( F \) de formes de degré \( 2 \) dans le plongement de
  \bsc{Segre} (voir section~\vref{sec:vaemb}). Si l'on note \( z = s(x, y) \)
  et \( z' = s(x', y') \) les images dans ce plongement, on est ainsi ramené à
  montrer que
  \begin{equation} \label{e:addsub-dv}
    \Distv(\xi z, \xi z')
    \le
    \Distv(z, z')
    \cdot \hmclab* \bigl( 5 (n + 1)^3 \bigr)^\dv
  \end{equation}
  en appliquant deux fois~\eqref{e:dv-segre}. On peut donc supposer
  \( \Distv( z, z') \le \bigl( 5 (n + 1)^3 \bigr)^{-\dv} \), car sinon
  l'inégalité précédente est banale.

  On va utiliser un développement de \bsc{Taylor} au voisinage de \( z \) des
  formes représentant \( \xi \) pour montrer que \( \nv2{F(z')} \) n'est pas
  trop petit devant \( \nv2 {F(z)} \) puis que \( \nv2{F(z) \wedge F(z')} \)
  n'est pas trop grand devant \( \nv2 {F(z)}^2 \). L'hypothèse faite sur la
  distance permet d'appliquer le lemme~\vref{l:dv-common-i} à \( z \) et \( z'
  \) (en remplaçant \( n \) par \( n' = n^2 + 2n \)) et donc de supposer,
  quitte à renuméroter les coordonnées, que \( z_0 = z'_0 = 1 \) et \( \nv2{
      z' } \le (n')^{\dv/2} \).  Le développement de
  \bsc{Taylor} de \( F \) s'écrit alors :
  \begin{equation}
    F(z')
    =
     F(z)
    + \sum_{1 \le i \le n'}
    \underbrace{
      \frac{\partial F}{\partial Z_i}(z)
      \cdot ( z'_i - z_i )
    }_{\textstyle R_i}
    + \sum_{1 \le i, j \le n'}
    \underbrace{
      \frac12 \frac{ \partial^2 F }{ \partial Z_i \partial Z_j }(z)
      \cdot ( z'_i - z_i ) ( z'_j - z_j )
    }_{\textstyle R_{ij}}
    \pmm.
  \end{equation}
  On remarque alors que
  \(
    \av{z'_i - z_i}
    \le
    \nv2{z \wedge z'}
    \le
    \nv2 z \nv2{z'}
    \distv{ z }{ z' }
  \)
  pour estimer \( R_i \) en utilisant de plus~\eqref{e:addsub-loc} :
  \begin{align}
    \nv2{R_i}
    & \le
    2^\dv \nnv2 F \nv2 z
    \cdot \nv2 z \nv2{ z' }
    \distv{ z }{ z' }
    \\ & \le
    \hmclab* \nv2{ F(z) }
    (2 \sqrt{n'})^\dv
    \distv{ z }{ z' }
    \pmm.
  \end{align}
  On obtient de même
  \(
    \nv2{R_{ij}}
    \le
    \hmclab* \nv2{ F(z) }
    (n')^\dv
    \distv{ z }{ z' }^2
    \pmm.
  \)
  En prenant la somme, on obtient un facteur \( (n')^\dv \) supplémentaire
  pour les \( R_i \), et son carré pour les \( R_{ij} \). Au final on a :
  \begin{align}
    \nv*2{\sum R_i + \sum R_{ij}}
     & \le
    \nv2{F(z)}
    \cdot \hmclab* \left( \frac52 \cdot (n')^{3/2} \right)^\dv
    \distv{ z }{ z' }
     <
    \left( \frac12 \right)^\dv
    \nv2{F(z)}
  \end{align}
  en remarquant aux places archimédiennes que \( 2t + t^2 \le 5t/2 \) pour \(
    0 \le t \le 1/2 \), ce qui est le cas de \( t = (n')^{3/2}
    \distv{ z }{ z' } \) vu l'hypothèse sur la distance et le fait que \( n'
    < (n+1)^2 \).

  Les inégalités triangulaire aux places infinies et ultramétrique aux places
  finies donnent alors \( \nv2{F(z')} \ge (1/2)^\dv \nv2{F(z)} \).
  Afin d'estimer la distance, il reste à majorer \( \nv2{F(z) \wedge F(z')}
  \). Pour ce faire, on développe le second facteur, on remarque que le
  premier terme du produit est nul et on majore brutalement le terme restant
  par le produit des normes ; il vient ainsi :
  \begin{align}
    \Distv(\xi(z), \xi(z'))
    & =
    \frac{ \nv2{F(z) \wedge F(z')} }{ \nv2{F(z)} \nv2{F(z')} }
    \\ & \le
    \frac{
      \nv2{F(z)} \nv2{\sum R_i + \sum R_{ij}}
    }{
      \nv2{F(z)} \nv2{F(z)} \cdot 2^{-\dv}
    }
    \\ & \le
    \hmclab* \bigl( 5 (n')^{3/2} \bigr)^\dv
    \distv{ z }{ z' }
  \end{align}
  qui donne bien~\eqref{e:addsub-dv} et achève la preuve.
\end{proof}



\section{Cas des translatés de sous-variétés abéliennes}
\label{sec:mumford-grp}

La stratégie de la preuve est la suivante : si on a deux approximations
exceptionnelles, suffisamment proches dans l'espace de \bsc{Mordell-Weil} et
de hauteur assez grande, on fabrique, (en prenant leur différence, sous
l'hypothèse que \( \avar \) est un translaté du sous-groupe \( \vai \)) une
approximation de \( \vai \) de qualité telle qu'elle contredit l'inégalité de
\bsc{Liouville} à moins d'appartenir à \( \vai \).

\begin{lem} \label{l:diff-apx}
  Soient \( x \) et \( y \) dans \( \proj(\Qbar) \), on note \( z = x - y \).
  Si \( x \) et \( y \) satisfont à~\eqref{e:Mapx} et à l'inégalité de gauche
  de~\eqref{e:Mclose}, on a
  \begin{equation}
    \prod\placerange
    \Distv(z, \vai)^\degv
    \le
    \hautm2 x ^{-\eps}
    \cdot \expb^{\eps\htcmp}
    \cdot 5 \sqrt2 (n + 1)^3 \hmclab
    \pmm.
  \end{equation}
\end{lem}

\begin{proof}
  En chaque place \( v \), il existe des points \( x'_v \) et \( y'_v \) dans
  \( \avar(\Cv) \) tels que \( \distv x \avar = \distv{ x }{ x'_v } \) et
  \( \distv y \avar = \distv{ y }{ y'_v } \). Les hypothèses et une
  comparaison de hauteur donnent alors
  \begin{equation}
    \max \bigl( \Distv(x, x'_v), \Distv(y, y'_v) \bigr)
    \le
    \exp(-\wtapx \eps \hautn{ x } + \htcmp)
    \pmm.
  \end{equation}
  Comme \( \avar \) est translaté d'un sous-groupe \( \vai \), on a
  \( x'_v - y'_v \in \vai(\Cv) \), donc la proposition~\vref{p:addsub-dv}
  donne
  \begin{equation}
    \distv z \vai
    \le
    \distv z {x'_v - y'_v}
    \le
    \exp(-\wtapx \eps \hautn{ x } + \htcmp)
    \cdot \hmclab* \bigl( 5 \sqrt2 (n + 1)^3 \bigr)^\dv
    \pmm.
  \end{equation}
  On prend alors le produit sur \( v \in \placesapx \), en supposant (c'est le
  cas défavorable) que \( \placesapx \) contient toutes les places
  archimédiennes, et la normalisation \( \sum\placerange \wtapx \degv = 1 \)
  permet de conclure.
\end{proof}

\begin{lem} \label{l:diff-small}
  Soient \( x \) et \( y \) satisfaisant aux hypothèses \eqref{e:Mcos}
  et \eqref{e:Mclose} du théorème~\vref{t:mumford-grp}, notons \( z \) leur
  différence. On a alors \( \hautl2 z \le (\rho^2/4 + 2\phi + \rho\phi)
    \hautn x + \htcmp \).
\end{lem}

\begin{proof}
  On note \( \scalnt \truc\truc \) le produit scalaire et \(
    \nnt\truc = \sqrt{\hautn\truc} \) la norme dans l'espace de
  \bsc{Mordell-Weil}. On remarque de plus que l'hypothèse \eqref{e:Mclose}
  implique
  \( \nnt x \le \nnt y \le (1+\rho)^{1/2}\nnt x \le (1+\rho/2)
    \nnt x \). Il vient alors :
  \begin{align}
    \hautn z
    & =
    \nnt{x-y}^2
    \\ & =
    \nnt x ^2 + \nnt y ^2 - 2 \scalnt x y
    \\ & =
    \left( \nnt y  - \nnt x  \right)^2
    + 2 \nnt x  \nnt y  \left( 1- \cos(x, y) \right)
    \\ & \le
    \left( \frac{\rho}{2}\nnt x  \right)^2
    + 2\left( 1+\frac{\rho}{2} \right)
    \nnt x ^2 \cdot \phi
    \\ & \le
    \bigl( (\rho^2/4) + 2\phi + \rho\phi \bigr)
    \hautn x
  \end{align}
  et une simple comparaison de hauteurs permet de conclure.
\end{proof}

\begin{proof}[Démonstration du théorème~\vref{t:mumford-grp}.]
  \label{page:demo-mumgrp}
  Le lemme~\vref{l:diff-apx} donne, en prenant le logarithme :
  \begin{equation} \label{e:close-log}
    \sum\placerange
    \degv \ln\Distv(z, \vai)
    \le
    - \eps \hautn x
    + \eps \htcmp
    + \ln(5\sqrt2) + 3 \ln(n + 1) + \hlclab
    \pmm.
  \end{equation}
  Par ailleurs, le lemme~\vref{l:diff-small} et
  l'hypothèse~\eqref{e:rho-phi-grp} donnent
  \begin{equation} \label{e:h2-diff}
    \hautl2 z
    \le
    (\rho^2/4 + 2\phi + \rho\phi)
    \hautn x
    + \htcmp
    \le
    \frac{ \eps }{ 2\adeg }
    \hautn x
    + \htcmp
    \pmm.
  \end{equation}

  Supposons maintenant que \( z \not\in \vai(\Qbar) \) et montrons qu'on
  contredit~\eqref{e:Mbig}. En effet, l'inégalité de \bsc{Liouville}
  (proposition~\vref{p:liouville}) appliquée à \( z \) et \( \vai \) donne
  \begin{align}
    \sum\placerange
    \degv \ln\Distv(z, \vai)
    & \ge
    - \adeg \hautl2 z
    - \hautl1{ \vai }
    - \frac32 \ln(n + 1)
    - (\adim + 1)\adeg \ln(3\adeg)
    \\ & \ge
    - \frac{ \eps }{ 2 } \hautn x
    - \adeg \htcmp
    - \hautl1{ \chow\vai }
    - \frac32 \ln(n + 1)
    - (\adim + 1)\adeg \ln(3\adeg)
  \end{align}
  où la deuxième ligne vient en substituant~\eqref{e:h2-diff}.
  En comparant avec~\eqref{e:close-log}, il vient :
  \begin{equation}
    \frac\eps2 \hautn x
    \le
    \hautl1{ \vai }
    + (\adim + 1)\adeg \ln(3\adeg)
    + (\adeg + \eps) \htcmp
    + \hlclab
    + \frac92 \ln(n + 1)
    + \ln(5\sqrt2)
    \pmm.
  \end{equation}
  Comme \( \vai \) est une sous-variété abélienne de \( \va \), sa hauteur
  normalisée est nulle \cite[prop. 9]{phiha1} ; ainsi, en
  utilisant~\eqref{e:comp-h-hn-var} et une comparaison de normes, on a
  \begin{equation}
    \hautl1{ \vai }
    \le
    \hautl2 \vai
    + \frac12 \adeg (\adim + 1) \ln(n+1)
    \le
    \htcmp \adeg (\adim + 1)
    + \frac12 \adeg (\adim + 1) \ln(n+1)
    \pmm.
  \end{equation}
  En substituant dans l'inégalité précédente et en utilisant le fait que \(
    \adeg \ge 2 \), il vient :
  \begin{equation}
    \frac\eps2 \hautn x
    \le
    \adeg (\adim + 1) \Bigl(
      \ln(\adeg)
      + (2 + \frac\eps\adeg) \htcmp
      + \frac12 \hlclab
      + \frac{11}4 \ln(n+1)
      + \frac12 \ln(5\sqrt2)
      + \ln(3)
    \Bigl)
  \end{equation}
  qui, en observant de plus que \( \ln(5\sqrt2)/2 \le \ln(3) \le \ln(n+1) \),
  contredit bien~\eqref{e:Mbig}, achevant ainsi la preuve.
\end{proof}



\section{Étude du morphisme des différences}

Comme dans l'énoncé du théorème, on choisit \( \avar \) une sous-variété de \(
  \va \) ; on note \( \adeg = \deg\avar \) et \( \adim = \dim\avar \).
Remarquons de suite que le degré de \( \avar \) est nécessairement au moins \(
  2 \), car si \( \avar \) était linéaire, elle contiendrait une droite
projective, c'est-à-dire une courbe de genre \( 0 \), or il est impossible de
plonger une telle courbe dans une variété abélienne. On supposera de plus
\( \adim \ge 1 \) car l'étude qui suit est sans intérêt sinon.

Dans le cas des translatés de sous-variétés abéliennes, la preuve reposait sur
le morphisme
\begin{align}
  s_2 \colon \va^2 & \to \va \\
  (x, y) & \mapsto x - y
  \pmm.
\end{align}
Si \( \avar = z + \vai \), l'image de
\( \avar^2 \) par ce morphisme est exactement \( \vai \), qui a même dimension
et même degré que \( \avar \) et dont la hauteur est majorée en fonction de
celle de \( \va \). On montre alors que si \( (x, y) \) satisfait les
hypothèses du théorème, son image est dans \( s_2(\avar^2) \), ce qui permet
de conclure.

Dans le cas général, pour tout entier \( m \ge 2 \), on définit un morphisme
\begin{align}
  s_m \colon \va^m & \to \va^{m-1} \\
  (x_1, \dots, x_m) & \mapsto (x_1 - x_m, \dots, x_{m-1} - x_m)
\end{align}
et on note \( Z = s_m(\avar^m) \) plongée dans \( \proj N \), où
\( N = (n+1)^{m-1} - 1 \) par un plongement de \bsc{Segre}. Le but de cette
section est de contrôler le degré et la hauteur de \( Z \) dans ce plongement
en fonction du degré et de la hauteur de \( \avar \).

On commence par représenter localement le morphisme \( s_m \) par une famille
de formes de degrés et hauteur contrôlées.

\begin{lem} \label{l:repr-sm}
  Pour tout \( x \in \avar^m \), il existe une famille de \( N + 1 \) formes
  multihomogènes de multidegré \( (2, \dots, 2, 2(m-1)) \) et telle que \(
    \nnv1 S \le \hmclab* ^{m-1} \) en toute place \( v \).
\end{lem}

\begin{proof}
  La section~\vref{sec:vaemb} fournit, pour chaque \( i \in \set{1, \dots,
      m-1} \) une famille de formes bihomogènes de bidegré \( (2, 2) \)
  représentant la soustraction de \( \va^2 \) dans \( \va \) au voisinage de
  \( (x_i, x_m) \), que l'on notera \( (L\pexp i[k])_{k \in \set{0,
        \dots, n}} \), telle que \( \nnv1{ L\pexp i } \le \hmclab* \).
  Pour chaque \( l \in \set{0, \dots, n}^{m-1} \) on pose
  \begin{equation}
    S_l(\vmp[1], \dots, \vmp[m])
    =
    \prod_{i=1}^{m-1} L\pexp i[l_i]( \vmp[i], \vmp[m] )
  \end{equation}
  de sorte que la famille \( (S_l)_l \) représente \( s_m \) au voisinage de
  \( x \) et que \( \deg S_l = (2, \dots, 2, 2(m-1)) \).  On a de plus \(
    \nnv1{ S } \le \hmclab* ^{m-1} \) par les propriétés de la norme \( L_1
  \).
\end{proof}

On en déduit immédiatement l'action de \( s_m \) sur la hauteur des points.

\begin{coro} \label{c:ht-sm-p}
  Pour tout \( x \in \avar^m \), on a
  \begin{equation}
    \hautl2{ s_m(x) }
    \le
    4 (m-1) \max_{1 \le i \le m} \bigl( \hautl2{ x_i } \bigr)
    + (m-1) \hlclab
    \pmm.
  \end{equation}
\end{coro}

\begin{proof}
  Vu le lemme précédent, les propriétés de la norme euclidienne donnent
  \begin{equation}
    \hautl2{ S(x) }
    \le
    \hautl2{ S }
    + 2 \hautl2{ x_1 } + \dots + 2 \hautl2{ x_{m-1} }
    + 2 (m-1) \hautl2{ x_m }
  \end{equation}
  qui implique le résultat annoncé.
\end{proof}

Ce corollaire sera utile pour estimer la hauteur de \( Z \) en passant par le
minimum essentiel ; pour l'instant nous devons commencer par estimer le degré.

\begin{lem} \label{l:sm-deg}
  On a \( \deg Z \le \adeg^m \, 2^{mu} \, m^{(m+1)u - 1} \).
\end{lem}

\begin{proof}
  On note \( \adim' = \dim Z \le m \adim \) et on choisit \( u' \) hyperplans
  de \( \proj N \) en position générale, qu'on notera \( E_1, \dots,
    E_{\adim'} \) de sorte que \( F = \bigcap_i E_i \) est un ensemble fini de
  points, de cardinal \( \deg Z \).  On note donc \( F = \set{y_1, \dots, y_p}
  \) et notre but est de majorer \( p = \deg Z \).

  Pour chaque \( j \in \set{1, \dots, p} \), on choisit un point \( x_j \in
    s_m^{-1}(y_j) \cap \avar^m \) et une famille de formes \( (S_{j,l})_l \)
  représentant \( s_m \) au voisinage de \( x_j \) donnée par le
  lemme~\vref{l:repr-sm}. Clairement, il existe une combinaison linéaire de
  ces familles qui représente \( s_m \) sur un ouvert contenant tous les \(
    x_j \) : on la note \( S \). Maintenant, pour chaque \( i \in \set{1,
      \dots, \adim'} \), on choisit une équation \( L_i \) de \( E_i \), on
  pose \( \tilde L_i = L_i(S) \) puis on note \( \tilde E_i \) l'hypersurface
  de \( (\projd)^m \) définie par \( \tilde L_i \).

  Par construction, il est clair que pour tout \( j \in \set{1, \dots, p} \),
  la fibre \( s_m^{-1}(y_j) \cap \avar^m \) est une composante isolée de
  l'intersection \( \avar^m \cap \bigcap_i \tilde E_i \) : en effet, les
  fibres sont deux à deux d'intersection vide et les autres composantes
  éventuelles de l'intersection sont celles provenant du lieu des zéros commun
  des formes de \( S \), qui ne contient aucune des fibres en question grâce
  au choix de cette famille.

  On introduit alors les notations suivantes (voir p.~364 et p.~362 de
  \cite{philz}), pour tout fermé \( X \) de \( (\projd)^m \) et tout
  multidegré \( d_1, \dots, d_m \) :
  \begin{equation}
    SH(X ; d_1, \dots, d_m)
    =
    \sum_Y H(Y ; d_1, \dots, d_m)
  \end{equation}
  où la somme est prise sur l'ensemble des composantes de \( X \) et
  \begin{equation}
    H(Y ; d_1, \dots, d_m)
    =
    \sum_{\lgr\alpha = \dim Y}
    \deg_\alpha Y
    \, \frac{(\dim Y)!}{ \alpha_1 ! \dots \alpha_m ! }
    d_1^{\alpha_1} \cdots d_m^{\alpha_m}
    \pmm.
  \end{equation}
  On invoque alors la proposition~3.3, p.~365 de la référence citée pour
  obtenir
  \begin{equation}
    SH( \avar^m \cap \bigcap_i \tilde E_i \, ; 2, \dots, 2, 2(m-1) )
    \le
    SH( \avar^m ; 2, \dots, 2, 2(m-1) )
    \pmm.
  \end{equation}
  On remarque que le seul multidegré non nul de \( \avar^m \) est celui
  d'indice \( (\adim, \dots, \adim) \) et qu'il vaut \( \adeg^m \),  de sorte
  que l'on a
  \begin{equation}
    SH( \avar^m ; 2, \dots, 2, 2(m-1) )
    =
    \adeg^m
    \frac{ (mu)! }{ (\adim!)^m }
    \, 2^{mu} (m-1)^\adim
    \le
    \adeg^m
    (2m)^{mu} (m-1)^{\adim-1}
  \end{equation}
  en majorant le coefficient multinomial qui apparaît dans cette expression
  par \( m^{mu-1} \). Par ailleurs, on a
  \begin{equation} \label{e:sh-inter}
    SH( \avar^m \cap \bigcap_i \tilde E_i \, ; 2, \dots, 2, 2(m-1) )
    \ge
    \sum_{j=1}^p H( s_m^{-1}(y_j) \cap \avar ; 2, \dots, 2, 2(m-1) )
    \ge
    p
  \end{equation}
  d'après le paragraphe précédent, ce qui achève la preuve.
\end{proof}

On déduit assez aisément des deux résultats précédents une majoration de la
hauteur de \( Z \).

\begin{coro}
  On a \(
    \hautl\htpph Z
    \le
    \bigl( 8 \adeg^{m-1}\hautl\htpph \avar + \adeg^m \hlclab)
    (2m)^{ (m+1)\adim + 1}
  \).
\end{coro}

\begin{proof}
  Il suffit de savoir majorer le minimum essentiel de \( Z \) en fonction de
  celui de \( \avar \), en vertu des comparaison suivantes, données par le
  théorème 3.1 de \cite{daphimhva1} :
  \begin{equation}
    \miness(Z) \deg Z
    \le
    \hautl\htpph Z
    \le
    \miness(Z) \deg Z (\dim Z + 1)
  \end{equation}
  où \( \miness \) désigne le minimum essentiel, défini en utilisant la
  hauteur \( \Hautl2 \) pour les points et \( \Hautl\htpph \) est la hauteur
  projective.

  Notons \( F \) l'ensemble des points de \( x \in \avar \) tels que \(
    \hautl2 x \le 2 \miness(\avar) \) ; par définition \( F \) est dense
  dans \( \avar \), donc \( F^m \) est dense dans \( \avar^m \) et \( s_m(F^m)
  \) est dense dans \( Z \). On applique alors le corollaire~\ref{c:ht-sm-p}
  pour majorer la hauteur des points de \( s_m(F^m) \) et obtenir
  \begin{equation}
    \miness(Z)
    \le
    8 (m-1) \miness(\avar) + (m-1) \hlclab
    \pmm.
  \end{equation}

  Il est alors aisé de passer aux hauteurs grâce à l'encadrement rappelé
  ci-dessus, en utilisant le fait que \( \dim Z \le mu \) et le
  lemme~\ref{l:sm-deg}
  \begin{align}
    \hautl\htpph Z
    & \le
    \miness(Z) \deg Z (\dim Z + 1)
    \\ & \le
    (8 (m-1) \miness(\avar) + (m-1) \hlclab)
    \, \adeg^m \, 2^{mu} \, m^{(m+1)u - 1}
    \, mu
    \\ & \le
    (8  \hautl\htpph \avar / \adeg +  \hlclab)
    \adeg^m (2m)^{ (m+1)\adim + 1}
  \end{align}
  qui donne bien la majoration annoncée.
\end{proof}

Nous avons maintenant tous les ingrédients utiles pour continuer la preuve de
l'inégalité de \bsc{Mumford}. Néanmoins, nous concluons cette section par une
étude des fibres de \( s_m \), qui pourrait être utile pour discuter
d'améliorations éventuelles de l'inégalité de \bsc{Mumford}. Nous noterons \(
  S_\avar \) le stabilisateur de \( \avar \).

\begin{lem}
  Pour tous \( m \ge 2 \) et \( x \in \avar^m \), on a
  \begin{equation}
    x + \delta(S_\avar)
    \subset
    s_m^{-1} ( s_m(x) ) \cap \avar^m
    \subset
    x + \delta\Bigl( \bigcap_{i=1}^m (\avar - x_i) \Bigr)
  \end{equation}
  où \( \delta \colon \avar \to \avar^m \) est le plongement diagonal. En
  particulier, si \( S_\avar = \bigcap_{i=1}^m (\avar - x_i) \), ces inclusions
  sont des égalités.
\end{lem}

\begin{proof}
  Par définition de \( s_m \), il est clair que \( x + \delta(A) \subset
    s_m^{-1} ( s_m(x) ) \). Par ailleurs, la définition de \( S_\avar \) donne
  clairement \( x + \delta(S_\avar) \subset \avar \), ce qui établit la
  première inclusion annoncée.

  Dans l'autre sens, pour tout \( y \in s_m^{-1} ( s_m(x) ) \), on a
  \( y_m - x_m = y_i - x_i \) pour tout \( i \) donc \( y = x +
    \delta(y_m - x_m) \). Si de plus \( y \in \avar^m \) on a \( y_m - x_m \in
    \bigcap_{i=1}^m (\avar - x_i) \).
\end{proof}

Étudions dans quel cas l'égalité peut être obtenue. Par définition du
stabilisateur, on a (ensemblistement) :
\begin{equation}
  S_\avar = \bigcap_{x \in \avar(\Qbar)} (\avar - x)
  \pmm.
\end{equation}
Par noethérieneté, il est clair qu'on peut en fait prendre une intersection
finie ; le lemme suivant précise ce résultat.

\begin{lem}
  Posons \( M = 2 \adeg^{\adim+1} - \adeg \) ; pour tout \( m \ge M \) il
  existe \( (x_1, \dots, x_m) \in \avar^m(\Qbar) \) tel que
  \( S_\avar = \bigcap_{i=1}^m (\avar - x_i) \).
\end{lem}

\begin{proof}
  En partant d'un \( \avar - x_1 \) pour \( x_1 \) quelconque, on va couper
  successivement par des \( \avar - x_i \) choisis de sorte à faire chuter à
  chaque fois la dimension d'au moins une des composantes de dimension
  maximale de l'intersection partielle qui n'est pas une composante de \(
    S_\avar \).  Clairement, ce processus se termine et l'intersection obtenue
  est \( S_\avar \).

  Plus précisément, on note \( X_j = \bigcap_{i=1}^{j+1} (\avar - x_i) \) le
  fermé de \bsc{Zarisiki} obtenu après \( j \) intersections.  Pour nous aider
  à compter les composantes à chaque étape, on introduit la quantité définie
  pour tout fermé \( X \) et tout entier \( D \) par :
  \begin{equation}
    SH(X, D)
    =
    \sum_Y \deg Y \, D^{\dim Y}
  \end{equation}
  où la somme est prise sur l'ensemble des composantes de \( X \). Cette
  définition coïncide avec celle donnée p.~364 de \cite{philz} (voir aussi
  ($*$) p.~362 de cette référence) ; on est ici dans le cas homogène
  c'est-à-dire \( p = 1 \) dans les notations de cette référence.

  Il est clair que \( SH(X_1, \adeg) = SH(\avar, \adeg) = \adeg^{\adim+1} \).
  Par ailleurs, comme chaque \( X_j \) peut être défini (ensemblistement) par
  des équations de degré au plus \( \adeg \), on peut appliquer à ces
  équations la proposition~3.3 de la référence citée, ce qui garantit que \(
    SH(X_{j+1}, \adeg) \le SH(X_j, \adeg) \) et donne immédiatement par
  récurrence \( SH(X_j, \adeg) \le \adeg^{\adim+1} \) pour tout \( j \).

  Par définition de \( SH \), cette inégalité montre que pour tout \( k \), le
  nombre de composantes de \( X_j \) de dimension \( \adim - k \) est au plus
  \( \adeg^{k+1} \).  On peut être un peu plus précis pour \( X_1 \) et voir
  que toutes ses composantes sont de dimension au plus \( \adim - 1 \) car \(
    \avar - x_1 \) n'a qu'une composante. Ainsi, en choisissant successivement
  les \( x_i \) de façon à ce que \( \avar - x_{j+2} \) coupe strictement une
  des composantes de \( X_j \) de dimension maximale parmi celles qui ne sont
  pas composantes de \( S_\avar \), on voit qu'il existe un \( j_1 \le 1 +
    \adeg^2 \) tel que \( X_{j_1} \) n'ait plus de composante de dimension \(
    \adim - 1 \) à part peut-être celles de \( S_\avar \), et en continuant
  jusqu'à avoir éliminé les composantes indésirables de dimension \( 0 \),
  qu'il existe un \( j_\adim \le 1 + \adeg^2 + \adeg^3 + \dots +
    \adeg^{\adim+1} \) tel que \( X_{j_\adim} \) n'ait plus aucune composante
  autre que celles de
  \( S_\avar \), c'est-à-dire tel que \( X_{j_\adim} = S_\avar \).

  Le nombre de points utilisés est alors
  \begin{equation}
    j_\adim + 1
    \le
    2 + \adeg^2 \sum_{l=0}^{\adim-1} \adeg^l
    =
    2 + \frac{\adeg}{\adeg-1} (\adeg^{\adim+1} - \adeg)
    \le
    2 + 2 (\adeg^{\adim+1} - \adeg)
    \pmm,
  \end{equation}
  en utilisant le fait que \( \adeg \ge 2 \).

  Par ailleurs, on constate qu'une fois qu'on a choisi \( x_1, \dots,
    x_{j_\adim+1} \) tels que \( X_{j_\adim} = S_\avar \), on peut rajouter
  des \( x_i \) arbitraires pour \( i > j_\adim + 1 \) et on aura toujours \(
    S_\avar = \bigcap_i (\avar - x_i) \).
\end{proof}

On suppose désormais \( m \ge 2 \adeg^{\adim+1} - \adeg \) jusqu'à la fin de
cette section.  On peut voir qu'il existe un ouvert dense de \( \avar^m \) où
\( (x_1, \dots, x_m) \) « définissent le stabilisateur » au sens ci-dessus,
car la condition correspondante est ouverte et le lemme précédent affirme que
l'ouvert n'est pas vide. On en déduit immédiatement que les fibres de \(
  {s_m}_{|\avar} \) sont génériquement des translatés de \( \delta(S_\avar) \)
; en particulier, \( \dim s_m(\avar^m) = m \adim - \dim S_\avar \).

On peut alors préciser l'estimation du degré de l'image :
\begin{equation}
  \deg Z
  \le
  \frac{ \adeg^m }{ \deg S_\avar }
  2^{m\adim - 2\dim S_\avar} m^{(m+1)\adim - 1 - \dim S_\avar}
\end{equation}
En effet, on peut s'arranger pour que les \( s_m^{-1}(y_j) \cap \avar \)
apparaissant dans~\eqref{e:sh-inter} soient tous des translatés de \(
  \delta(S_\avar) \) et on voit assez facilement que tous les multidegrés de
\( \delta(S_\avar) \) sont égaux à \( \deg S_\avar \), ce qui donne
\begin{align}
  H( \delta(S_\avar); 2, \dots, 2, 2(m-1) )
  & =
  \ \sum_{ \mathclap{\lgr\alpha = \dim S_\avar} } \
  \deg S_\avar
  \, \frac{ (\dim S_\avar)! }{ \alpha_1! \cdots \alpha_m! }
  \, 2^{\dim S_\avar} (m-1)^{\alpha_m}
  \\ & =
  \deg S_\avar
  (2)^{2\dim S_\avar} (m-1)^{\dim S_\avar}
\end{align}
d'après la formule multinomiale.


\section{Cas général}

Dans toute cette section, on fixe \( m \ge 2 \) et une famille \( x = (x_1,
  \dots, x_m) \in \va(\Qbar)^m \) (et non pas \( \avar^m \)) et on note \( z =
  s_m(x) \). On suppose de plus \( \adim \ge 1 \) car sinon le
théorème~\vref{t:mumford-gen} est une conséquence directe du
théorème~\vref{t:mumford-grp}.  Les trois premiers énoncés sont des
conséquences ou analogues directs de ceux de la
section~\vref{sec:mumford-grp}.

\begin{lem} \label{l:img-small}
  Sous les hypothèses~\eqref{e:Mcos-gen} et~\eqref{e:Mclose-gen} du
  théorème~\vref{t:mumford-gen}, on a
  \begin{equation}
    \hautl2 z
    \le
    (\rho^2/4 + 2\phi + \rho\phi) \hautn{ x_m }
    + (m-1) \htcmp
    \pmm.
  \end{equation}
\end{lem}

\begin{proof}
  Pour chaque \( i \in \set{1, \dots, m-1} \), on applique le lemme
  \vref{l:diff-small} avec \( x = x_m \) et \( y = x_i \), ce qui montre que
  \( \hautl2{z_i} \le (\rho^2/4 + 2\phi + \rho\phi) \hautn{ x_m } + \htcmp \).
  On remarque ensuite que \( \hautl2{z} = \sum_{i=1}^{m-1} \hautl2{z_i} \)
  pour aboutir au résultat annoncé.
\end{proof}

\begin{lem} \label{l:img-apx}
  Si \( x \) satisfait l'hypothèse~\eqref{e:Mapx-gen} et la première inégalité
  de l'hypothèse~\eqref{e:Mclose-gen} du théorème~\vref{t:mumford-gen}, on a
  \begin{equation}
    \prod_\placerange \distv z Z ^\degv
    \le
    \expb^{-\eps \hautn{ x_m }}
    \cdot \expb^{\eps\htcmp} \hmclab
    \cdot 5 \sqrt{2m-2} \, (n + 1)^3
  \end{equation}
\end{lem}

\begin{proof}
  C'est une variante du lemme~\vref{l:diff-apx} ; la démonstration est
  identique sauf pour l'étape finale.  Pour chaque \( i \in \set{1, \dots,
      m} \) et chaque \( v \in \placesapx \), la définition de la distance
  permet de choisir un point \( y_{i, v} \in \avar(\Cv) \) tel que
  \begin{equation}
    \distv{ x_i }{ y_{i, v} }
    =
    \distv{ x_i }\avar
    \le
    \exp(-\wtapx \eps \hautn{ x_m } + \wtapx\eps \htcmp)
  \end{equation}
  où l'inégalité découle directement des hypothèses et d'une comparaison entre
  hauteurs.
  La proposition~\vref{p:addsub-dv} donne alors
  \begin{equation}
    \max_i \distv{ x_i - x_m }{ y_{i, v} - y_{m, v} }
    \le
    \exp(-\wtapx \eps \hautn{ x_m } + \wtapx\eps \htcmp)
    \cdot \hmclab* \bigl( 5 \sqrt2 (n + 1)^3 \bigr)^\dv
    \pmm.
  \end{equation}
  Or, d'après le lemme~4.3, p.~121 de \cite{remgdmp} (interprété à la lumière
  des remarques p.~102 de la même référence concernant le lien entre indices
  d'une distance et plongement de \bsc{Segre}), en notant \( y_v = (y_{1, v},
    \dots, y_{m, v}) \in \avar^m(\Cv) \), on a
  \begin{equation}
    \distv{ s_m(x) }{ s_m(y_v) }
    \le
    (m-1)^{\dv/2}
    \, \max_i \distv{ x_i - x_m }{ y_{i, v} - y_{m, v} }
    \pmm.
  \end{equation}
  Or par définition le membre de gauche majore \( \distv{ z, Z } \). Il suffit
  alors de prendre le produit sur \( v \) pour conclure.
\end{proof}

\begin{lem}
  Sous les hypothèses du théorème~\vref{t:mumford-gen} (sauf \( x_i \in \grp
  \) et la définition de \( m \)), on a \( z \in Z \).
\end{lem}

\begin{proof}
  On procède comme en page~\pageref{page:demo-mumgrp} pour la preuve du
  théorème~\ref{t:mumford-grp}. Si \( z \not\in Z \), l'inégalité de
  \bsc{Liouville} (proposition~\vref{p:liouville}) appliquée à \( z \) et \( Z
  \) donne, en prenant les logarithmes :
  \begin{align} \label{e:liou}
    \sum\placerange \degv \ln\distv z Z
    & \ge
    - (\deg Z) \hautl2 z
    - \hautl1{ Z }
    \\ & \qquad
    - \frac32 \ln(N + 1)
    - (\dim Z + 1) (\deg Z) \ln(3 \deg Z)
    \pmm.
  \end{align}
  On va maintenant estimer une par une les quantités apparaissant dans le
  membre de droite.

  Dans la démonstration du lemme~3.3, p. 111 de \cite{remgdmp}, il est établit
  que si \( P \) est une forme multihomogène de multidegré \( \delta \) en
  plusieurs groupes de \( n + 1 \) variables, dont on note \( p_\alpha \)
  les coefficients, on a aux places archimédiennes
  \begin{equation}
    \abs{ p_\alpha }
    \le
    \binom \delta \alpha
    \mahler P
    \le
    \binom \delta \alpha
    \mespph P
  \end{equation}
  ce qui en sommant sur les multimultiindices \( \alpha \) tels que \(
    \vlg\alpha = \delta \), donne immédiatement, par la formule multinomiale :
  \begin{equation}
    \normlun P
    \le
    (n+1)^{\lgr\delta}
    \mespph P
    \pmm.
  \end{equation}
  En appliquant ceci à la forme de \bsc{Chow} de \( Z \), qui est
  multihomogène de degré \( \deg Z \) en \( \dim Z + 1 \) groupes de \( N + 1
  \) variables, puis en utilisant les estimations de la section précédente, il
  vient :
  \begin{align}
    \hautl1 Z
    & \le
    \hautl\htpph Z
    + (\deg Z) (\dim Z + 1) \ln(N+1)
    \\ & \le
    \bigl( 8 \adeg^{m-1}\hautl\htpph \avar + \adeg^m \hlclab)
    (2m)^{ (m+1)\adim + 1}
    \\ & \qquad
    + \adeg^m \, 2^{mu} \, m^{(m+1)u - 1} (m \adim + 1) (m-1)\ln(n+1)
    \\ & \le
    (2m)^{ (m+1)\adim + 1} \Bigl(
    8 \adeg^{m-1} \hautl\htpph \avar
    + \adeg^m \bigl( \hlclab + \ln(n+1) \bigr)
    \Bigr)
    \label{e:liou-h1-img}
    \pmm.
  \end{align}

  Par ailleurs, ces mêmes estimations donnent :
  \begin{align}
    (\dim Z + 1) \ln(3 \deg Z)
    & \le
    (m \adim + 1) \bigl(
    m\ln \adeg + m \adim \ln 2 + (m+1) \adim \ln m + \ln 3
    \bigr)
    \\ & \le
    \adim^2 (m+1) \bigl(
    m \ln 2 + (m+1) \ln m + \ln 3
    \bigr)
    + m^2 (\adim + 1) \ln \adeg
    \notag
    \\ & \le
    2^{\adim+2} m^3
    + 2^\adim m^2 \ln \adeg
    \pmm,
  \end{align}
  où la dernière ligne utilise le fait que \( \adim^2 \le 2^{\adim+1} \) et
  que la partie en \( m \) est majorée par \( 2m^3 \) (vérification numérique
  pour les petites valeurs). Au final on a
  \begin{align}
    (\dim Z + 1) (\deg Z) \ln(3 \deg Z)
    & \le
    \adeg^m 2^{mu} m^{(m+1)\adim - 1}
    \cdot \bigl( 2^{\adim+2} m^3 + 2^\adim m^2 \ln \adeg \bigr)
    \\ & \le
    \adeg^m (2m)^{(m+1)\adim + 2} (1 + \ln\adeg / 8)
    \\ & \le
    \adeg^{m} (2m)^{(m+1)\adim + 2} \cdot 2 \ln \adeg
    \label{e:liou-cst2}
    \pmm.
  \end{align}

  Enfin, le lemme~\vref{l:img-small} et l'hypothèse~\eqref{e:rho-phi-gen}
  donnent
  \begin{align}
    (\deg Z) \hautl2 z
    & \le
    \frac\eps2 \hautn{ x_m }
    + \adeg^m (2m)^{(m+1)\adim} \htcmp
    \pmm.
  \end{align}
  On peut maintenant substituer cette estimation ainsi
  que~\eqref{e:liou-h1-img} et~\eqref{e:liou-cst2} dans l'inégalité de
  \bsc{Liouville} écrite en~\eqref{e:liou} :
  \begin{align}
    \sum\placerange \degv \ln\distv z Z
    & \ge
    - \frac\eps2 \hautn{ x_m }
    - \adeg^m (2m)^{(m+1)\adim} \htcmp
    \\ & \qquad
    - (2m)^{ (m+1)\adim + 1 } \Bigl(
    8 \adeg^{m-1} \hautl\htpph \avar
    + \adeg^m \bigl( \hlclab + \ln(n+1) \bigr)
    \Bigr)
    \\ & \qquad
    - 2(m-1) \ln(n+1)
    - \adeg^{m} (2m)^{(m+1)\adim + 2} \cdot 2 \ln \adeg
    \\ & \ge
    - \frac\eps2 \hautn{ x_m }
    - 8 \adeg^{m-1} (2m)^{ (m+1)\adim + 1} \, \hautl\htpph \avar
    \\ & \qquad
    - \adeg^m (2m)^{ (m+1)\adim + 2}
    \bigl( 2\ln \adeg + \hlclab + \htcmp + \ln(n+1) \bigr)
    \pmm.
  \end{align}

  Par ailleurs, le lemme~\vref{l:img-apx} dit que
  \begin{equation}
    \sum\placerange \degv \ln\distv z Z
    \le
    -\eps \hautn{ x_m }
    + \hlclab + \eps\htcmp
    + \ln 5 + \frac12 \ln(2m-2) + 3 \ln(n+1)
  \end{equation}
  Les deux dernières égalités mises ensemble, en supposant \( \eps < 2\adeg \)
  (ce qu'on peut faire car sinon l'inégalité de \bsc{Liouville} implique
  qu'aucun point ne satisfait les hypothèses~\eqref{e:Mapx-gen}
  et~\eqref{e:Mbig-gen}), donnent
  \begin{equation}
    \frac\eps2 \hautn{ x_m }
    \le
    8 \adeg^{m-1} (2m)^{ (m+1)\adim + 1 } \, \hautl\htpph \avar
    + \adeg^m (2m)^{ (m+1)\adim + 3}
    \bigl( \ln \adeg + \hlclab + \htcmp + \ln(n+1) \bigr)
  \end{equation}
  qui contredit précisément~\eqref{e:Mbig-gen}, achevant ainsi la preuve.
\end{proof}

On rappelle la version quantitative de l'ex-conjecture de \bsc{Mordell-Lang}
obtenue par \bsc{Rémond}.

\begin{fact}
  Soient \( \avar \subset \va \) une sous-variété, \( \grp \subset
    \va(\Qbar) \) un sous-groupe de rang fini \( r \) et \( \cdn \) un corps
  de définition de \( \va \).  Alors il existe un entier naturel
  \begin{equation}
    S
    \le
    \bigl(
    2^{34} \cdot [\cdn : \Q] \hlclab \cdot \adeg
    \bigr)^{ (r+1) g^{ 5(\adim + 1)^2 } }
    \pmm,
  \end{equation}
  des points \( x_1, \dots, x_S \) de \( \avar(\Qbar) \cap \grp \) et des
  sous-variétés abéliennes \( \vai_1, \dots, \vai_s \) de \( \va \) telles que
  \( x_i + \vai_i \subset \avar \) pour tout \( i \in \set{1, \dots, S} \) et
  \begin{equation}
    \avar(\Qbar) \cap \grp
    =
    \bigcup_{i=1}^{S} (x_i + \vai_i)(\Qbar) \cap \grp
    \pmm.
  \end{equation}
\end{fact}

\begin{proof}
  C'est une version du théorème de~2.1, p.~517 de \cite{remdcl} où la fonction
  notée \( f \) dans ledit énoncé est explicitée, citée dans \cite{daphimhva2}
  p.~643. Par ailleurs, la quantité notée \( h_0(\va) \) dans cette dernière
  référence y est définie (théorème~1.4 p.~641) comme \( [\cdn:\Q] \max(1,
    h(\va) \), où la quantité notée \( h(\va) \) (qui n'est pas celle que nous
  notons \( \hautl{}\va \)) est utilisée pour contrôler la hauteur des
  formules d'addition, duplication, etc. de \( \va \) : on peut donc la
  remplacer par la quantité que nous notons \( \hlclab \), que nous avons
  supposée supérieure ou égale à \( 1 \).
\end{proof}

Nous sommes maintenant prêts à conclure.

\begin{proof}[\proofname{} du théorème~\vref{t:mumford-gen}.]
  Soit \( x = (x_1, \dots, x_m) \) une famille de points comme dans l'énoncé.
  D'après le lemme précédent, il existe un point \( y \in \avar^m(\Qbar) \)
  tel que \( s_m(x) = s_m(y) \), c'est-à-dire \( x_i - x_m = y_i - y_m \) pour
  tout \( i \). En particulier, pour tout \( i \) on a
  \begin{equation}
    y_i - y_m
    \in
    (\avar - y_m)(\Qbar) \cap \grp
    =
    \bigcup_{k=1}^{S} (z_k + \vai_k)(\Qbar) \cap \grp
  \end{equation}
  où l'égalité de droite découle du fait précédent appliqué à \( \avar - y_m
  \), avec \( z_k \in \avar - y_m \) et \( z_k + \vai_k \subset \avar - y_m
  \), où \( \vai_k \subset \va \) est une sous-variété abélienne, pour tout \(
    k \), et \( S < m \).

  Ainsi, il existe un indice \( k \in \set{1, \dots, S} \) et deux indices
  distincts \( i \) et \( j \) tels que \( y_i - y_m \in z_k + \vai_k \) et \(
    y_j - y_m \in z_k + \vai_k \). Par différence, on a alors
  \begin{equation}
    y_i - y_j = x_i - x_j \in \vai_k
  \end{equation}
  et \( z_k + y_m + \vai_k \subset \avar \), ce qui achève la preuve.
\end{proof}


\cleardoublepage
\endinput

% vim: spell spelllang=fr
