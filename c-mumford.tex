% !TEX root = main.tex

\chapter{Inégalité de \bsc{Mumford}} \label{chap:mumford}

\section{Énoncé principal}

On prouve ci-dessous une inégalité de \bsc{Mumford} explicite dans le cas
particulier où \( \varapx \) est une sous-variété abélienne de \( \va \),
qu'on notera donc désormais \( \vai \). Le résultat principal de ce chapitre
est le suivant.

\begin{thm} \label{t:mumford}
  Soient \( \vai \) une sous-variété abélienne de \( \va \) et \( \varapx \)
  un translaté de \( \vai \) par un point algébrique ;
  on note \( d = \deg \vai \) et \( l = \dim\vai + 1 \).
  On choisit \( \expapx > 0 \) puis \( \phi > 0 \) et \( \rho > 0 \) tels que
  \begin{equation} \label{e:rho-phi}
    \frac{ \rho^2 }{ 4 } + \rho\phi + 2\phi
    \le
    \frac{ \expapx }{ 2d }
    \pmm.
  \end{equation}
  Si \( x \) et \( y \) sont deux points de \( \va(\Qbar) \) tels que
  \begin{align}
    0 < \distv z \varapx
    & \le
    \hautm[2] z ^{-\wtapx\expapx}
    \quad \forall \place \in \placesapx
    \quad \text{où \( z \) est \( x \) ou \( y \)}
    \label{e:Mapx}
    \\
    \hautn x
    & \ge
    \frac2\expapx \bigl(
      \hautl[1]{ \chow\vai }
      + ld \ln(3d)
      + (d + 1) \htcmp
      + \hlclab
      + 6 \ln(n + 1)
    \bigr)
    \label{e:Mbig}
    \\
    \cos(x, y)
    & \ge
    1 - \phi
    \label{e:Mcos}
    \\
    \hautn x
    & \le
    \hautn y \le (1+\rho) \hautn x
    \label{e:Mclose}
  \end{align}
  alors \( x - y \in \vai(\Qbar) \).
\end{thm}

La preuve consiste à montrer que la différence \( x - y \) est encore très
proche de \( \vai \) tout en étant de hauteur relativement petite, de sorte
qu'une inégalité de \bsc{Liouville} la contraint à être sur \( \vai \). On
commence donc par établir une inégalité de \bsc{Liouville}, puis on étudie
l'action des opérations sur la distance, avant d'achever la preuve du
théorème.



\section{Inégalité de \bsc{Liouville}} \label{sec:liouville}

On établit ici l'analogue suivant de la classique inégalité de
\bsc{Liouville}.

\begin{prop} \label{p:liouville}
  Soit \( V \) une sous-variété de \( \projd \), de dimension \( u \) et de
  degré \( d \). Pour tout point \( x \in \projd(\Qbar) \), on a soit \( x \in
  V(\Qbar) \) soit
  \begin{equation}
    \prod_{v \in \placesapx} \distv x V ^\degv
    \ge
    \frac1{
      (n+1)^{3/2}
      (3d)^{d (u+1)}
      \, \hautm[1]{ \chow V }
      \, \hautm[2] x ^d
    }
    \pmm.
  \end{equation}
\end{prop}

Ce type d'inégalité se prouve de façon naturelle avec la distance algébrique
(section~\ref{sec:distv-cmp}). Cependant, si l'on traite directement le cas
général en utilisant la définition, on obtient \( d (u+1) \) comme exposant de
\( \hautm[2] x \). On commence donc par le cas d'une hypersurface, où la
distance algébrique admet une expression plus simple, puis on remarque que
tout variété est une intersection d'hypersurfaces.

On pourrait se ramener au cas d'un hyperplan, où la distance admet une
expression simple, par un plongement de \bsc{Veronese} (remodelé) comme dans
le lemme~\ref{l:hs-vero}, mais on obtient en fait de meilleurs constantes en
utilisant la propriété~\ref{p:dv-p2alg-hs}.

\begin{lem} \label{l:liou-hs}
  Soit \( V \) une hypersurface de \( \projd \), d'équation \( F \) et de
  degré \( d \). Pour tout point \( x \in \projd(\Qbar) \), on a soit \( x \in
  V(\Qbar) \) soit
  \begin{equation}
    \prod_{v \in \placesapx} \distv x V ^\degv
    \ge
    \frac1{
      (17/8)^d (n+1)^{3/2}
      \, \hautm[2] G
      \, \hautm[2] x ^d
    }
    \pmm.
  \end{equation}
\end{lem}

\begin{proof}
  Plutôt que la proposition~\ref{p:dv-p2alg-hs}, on utilise en fait
  l'inégalité~\eqref{e:dv-p2alg-hs} établie au cours de sa preuve afin
  d'éviter une comparaison norme-mesure superflue. On remarque également qu'on
  peut supposer \( m = 1 \) dans cette comparaison. On a alors :
  \begin{align}
    \prod_{v \in \placesapx}
    \distv x V ^\degv
    & \ge
    \prod_{v \in M(\cdn)}
    \distv x V ^\degv
    \\ & \ge
    \prod_{v \in M(\cdn)}
    \left(
      \left(
        \frac{27}{26} (n+1)^{3/2}
        \left( \frac{164}{81} \right)^{d}
      \right)^{-\dv}
      \frac{ \av{G(x)} }{ \nv2 x^{d} \nv2 G }
    \right)^\degv
  \end{align}
  qui implique le résultat annoncé, \lat{via} une simple comparaison
  numérique pour la constante.
\end{proof}

\begin{proof}[\proofname{} de la proposition~\ref{p:liouville}]
  Si \( x \not\in V(\Qbar) \), la proposition~6.1 de~\cite{remdcl}
  donne une forme \( G \) homogène de degré \( d \), contenant \( V \) mais
  pas \( x \), telle que
  \begin{equation}
    \hautm[1] G
    \le
    \hautm[1] V \cdot (d + 1)^{ d (u + 1) }
  \end{equation}
  en tenant compte des différences de hauteurs utilisées, comme observé dans
  la preuve du lemme~\ref{l:hs-choice}. Il suffit alors d'appliquer le
  lemme précédent à \( \zeros G \) et de remarquer que \( 17/8 \cdot (d+1)^2
    \le (3d)^2 \) pour conclure.
\end{proof}



\section{Comportement métrique des opérations}

La situation d'approximation considérée met en jeu plusieurs types de
géométrie sur \( \va \) : la géométrie euclidienne de son espace de
\bsc{Mordell-Weil}, la géométrie algébrique et projective de la variété
plongée, une géométrie métrique, locale, en chaque place de \( \placesapx \). On
s'attend à ce que les métriques locales sur \( \va \) se comportent de façon
agréable vis-à-vis des structures géométrique et arithmétique, par exemple que
les opérations soient uniformément continues, voire lipschitziennes. On montre
ici que c'est bien le cas et on explicite une valeur admissible pour la
constante de \bsc{Lipschitz}.

\begin{prop} \label{p:addsub-dv}
  Pour tous \( x \), \( x' \), \( y \), \( y' \) dans \( \va(\C_v) \), on a :
  \begin{equation}
  \Distv(x \pm y, x' \pm y')
  \le
  \max \bigl( \Distv(x, x'), \Distv(y, y') \bigr)
  \cdot \hmclab* \bigl( 5 \sqrt2 (n + 1)^3 \bigr)^\dv
  \pmm,
  \end{equation}
  où dans le membre de gauche il faut prendre le même signe des deux côtés.
\end{prop}

\begin{proof}
  On considère le plongement de \bsc{Segre} \( s \colon (\projd)^2 \to
    \proj{n'} \) avec \( n' = n^2 + 2n \). D'après le lemme~4.3 (p.~121) de
  \cite{remgdmp}, appliqué avec \( q = 2 \) et \( d = (1,1) \), ainsi que le
  paragraphe~2.3 (p.~103 notamment) de cette référence donnant le lien entre
  indices (d'une forme de \bsc{Chow} ou, par extension, d'une distance) et
  plongements de \bsc{Segre} (et de \bsc{Veronese} remodelé), on a
  \begin{equation} \label{e:dv-segre}
    1
    \le
    \frac{
      \Distv(s(x, x'), s(y, y'))
    }{
      \max \bigl( \Distv(x, x'), \Distv(y, y') \bigr)
    }
    \le
    2^{\dv/2}
  \end{equation}
  dans le cas où le dénominateur est différent de zéro (dans le cas contraire,
  on a \( x = x' \) et \( y = y' \) et le résultat est immédiat).

  On considère alors le morphisme
  \( \xi : \va^2 \to \va^2,\ (x, y) \mapsto (x + y, x - y) \), qui
  présente l'avantage de pouvoir être représenté globalement
  par une famille \( F \) de formes de degré \( 2 \) dans le plongement de
  \bsc{Segre} (voir section~\ref{sec:vaemb}). Si l'on note \( z = s(x, y) \)
  et \( z' = s(x', y') \) les images dans ce plongement, on est ainsi ramené à
  montrer que
  \begin{equation} \label{e:addsub-dv}
    \Distv(\xi z, \xi z')
    \le
    \Distv(z, z')
    \cdot \hmclab* \bigl( 5 (n + 1)^3 \bigr)^\dv
  \end{equation}
  en appliquant deux fois~\eqref{e:dv-segre}. On peut donc supposer
  \( \Distv( z, z') \le \bigl( 5 (n + 1)^3 \bigr)^{-\dv} \), car sinon
  l'inégalité précédente est banale.

  On va utiliser un développement de \bsc{Taylor} au voisinage de \( z \) des
  formes représentant \( \xi \) pour montrer que \( \nv2{F(z')} \) n'est pas
  trop petit devant \( \nv2 {F(z)} \) puis que \( \nv2{F(z) \wedge F(z')} \)
  n'est pas trop grand devant \( \nv2 {F(z)}^2 \). L'hypothèse faite sur la
  distance permet d'appliquer le lemme~\ref{l:dv-common-i} à \( z \) et \( z'
  \) (en remplaçant \( n \) par \( n' = n^2 + 2n \)) et donc de supposer,
  quitte à renuméroter les coordonnées, que \( z_0 = z'_0 = 1 \) et \( \nv2{
      z' } \le (n')^{\dv/2} \).  Le développement de
  \bsc{Taylor} de \( F \) s'écrit alors :
  \begin{equation}
    F(z')
    =
     F(z)
    + \sum_{1 \le i \le n'}
    \underbrace{
      \frac{\partial F}{\partial Z_i}(z)
      \cdot ( z'_i - z_i )
    }_{\textstyle R_i}
    + \sum_{1 \le i, j \le n'}
    \underbrace{
      \frac12 \frac{ \partial^2 F }{ \partial Z_i \partial Z_j }(z)
      \cdot ( z'_i - z_i ) ( z'_j - z_j )
    }_{\textstyle R_{ij}}
    \pmm.
  \end{equation}
  On remarque alors que
  \(
    \av{z'_i - z_i}
    \le
    \nv2{z \wedge z'}
    \le
    \nv2 z \nv2{z'}
    \distv{ z }{ z' }
  \)
  pour estimer \( R_i \) en utilisant de plus~\eqref{e:addsub-loc} :
  \begin{align}
    \nv2{R_i}
    & \le
    2^\dv \nnv2 F \nv2 z
    \cdot \nv2 z \nv2{ z' }
    \distv{ z }{ z' }
    \\ & \le
    \hmclab* \nv2{ F(z) }
    (2 \sqrt{n'})^\dv
    \distv{ z }{ z' }
    \pmm.
  \end{align}
  On obtient de même
  \(
    \nv2{R_{ij}}
    \le
    \hmclab* \nv2{ F(z) }
    (n')^\dv
    \distv{ z }{ z' }^2
    \pmm.
  \)
  En prenant la somme, on obtient un facteur \( (n')^\dv \) supplémentaire
  pour les \( R_i \), et son carré pour les \( R_{ij} \). Au final on a :
  \begin{align}
    \nv*2{\sum R_i + \sum R_{ij}}
     & \le
    \nv2{F(z)}
    \cdot \hmclab* \left( \frac52 \cdot (n')^{3/2} \right)^\dv
    \distv{ z }{ z' }
     <
    \left( \frac12 \right)^\dv
    \nv2{F(z)}
  \end{align}
  en remarquant aux places archimédiennes que \( 2t + t^2 \le 5t/2 \) pour \(
    0 \le t \le 1/2 \), ce qui est le cas de \( t = (n')^{3/2}
    \distv{ z }{ z' } \) vu l'hypothèse sur la distance et le fait que \( n'
    < (n+1)^2 \).

  Les inégalités triangulaire aux places infinies et ultramétrique aux places
  finies donnent alors \( \nv2{F(z')} \ge (1/2)^\dv \nv2{F(z)} \).
  Afin d'estimer la distance, il reste à majorer \( \nv2{F(z) \wedge F(z')}
  \). Pour ce faire, on développe le second facteur, on remarque que le
  premier terme du produit est nul et on majore brutalement le terme restant
  par le produit des normes ; il vient ainsi :
  \begin{align}
    \Distv(\xi(z), \xi(z'))
    & =
    \frac{ \nv2{F(z) \wedge F(z')} }{ \nv2{F(z)} \nv2{F(z')} }
    \\ & \le
    \frac{
      \nv2{F(z)} \nv2{\sum R_i + \sum R_{ij}}
    }{
      \nv2{F(z)} \nv2{F(z)} \cdot 2^{-\dv}
    }
    \\ & \le
    \hmclab* \bigl( 5 (n')^{3/2} \bigr)^\dv
    \distv{ z }{ z' }
  \end{align}
  qui donne bien~\eqref{e:addsub-dv} et achève la preuve.
\end{proof}



\section{Inégalité de \bsc{Mumford}}

La stratégie de la preuve est la suivante : si on a deux approximations
exceptionnelles, suffisamment proches dans l'espace de \bsc{Mordell-Weil} (au
sens des hypothèses \eqref{e:Mcos} et \eqref{e:Mclose} du
théorème~\ref{t:mumford}) et de hauteur assez grande, on fabrique, (en prenant
leur différence, sous l'hypothèse que \( \vai \) est un sous-groupe) une
approximation de qualité telle qu'elle appartiendra nécessairement à \( \vai
\), d'après l'inégalité de \bsc{Liouville}.

\begin{lem} \label{l:diff-apx}
  Soient \( x \) et \( y \) dans \( \proj(\Qbar) \), on pose \( z = x -
    y \) et \( H = \min\bigl( H(x), H(y) \bigr) \).
  Si \( x \) et \( y \) satisfont à~\eqref{e:Mapx}, on a
  \begin{equation}
    \prod\placerange
    \Distv(z, \vai)^\degv
    \le
    H^{-\expapx}
    \cdot 5 \sqrt2 (n + 1)^3 \hmclab
    \pmm.
  \end{equation}
\end{lem}

\begin{proof}
  En chaque place \( v \), il existe des points \( x'_v \) et \( y'_v \) dans
  \( \varapx(\Cv) \) tels que \( \distv x \varapx = \distv{ x }{ x'_v } \) et
  \( \distv y \varapx = \distv{ y }{ y'_v } \). L'hypothèse~\eqref{e:Mapx}
  donne alors
  \begin{equation}
    \max \bigl( \Distv(x, x'_v), \Distv(y, y'_v) \bigr)
    \le
    H^{-\wtapx \expapx}
    \pmm.
  \end{equation}
  Comme \( \varapx \) est translaté d'un sous-groupe \( \vai \), on a
  \( x'_v - y'_v \in \vai(\Cv) \), donc la proposition~\ref{p:addsub-dv} donne
  \begin{equation}
    \distv z \vai
    \le
    \distv z {x'_v - y'_v}
    \le
    H^{-\wtapx \expapx}
    \cdot \hmclab* \bigl( 5 \sqrt2 (n + 1)^3 \bigr)^\dv
    \pmm.
  \end{equation}
  On prend alors le produit sur \( v \in \placesapx \), en supposant (c'est le
  cas défavorable) que \( \placesapx \) contient toutes les places
  archimédiennes et toutes celles où \( \hmclab* > 1 \).  La normalisation \(
    \sum\placerange \wtapx \degv = 1 \) permet alors de conclure.
\end{proof}

\begin{lem} \label{l:diff-small}
  Soient \( x \) et \( y \) satisfaisant aux hypothèses \eqref{e:Mcos}
  et \eqref{e:Mclose} du théorème~\ref{t:mumford}, notons \( z \) leur
  différence. On a alors \( \hautn z \le (\rho^2/4 + 2\phi + \rho\phi)
    \hautn x \).
\end{lem}

\begin{proof}
  On note \( \langle\truc; \truc \rangle \) le produit scalaire et \(
    \nnt{\truc} = \sqrt{\Hautn(\truc)} \) la norme dans l'espace de
  \bsc{Mordell-Weil}. On remarque de plus que l'hypothèse \eqref{e:Mclose}
  implique
  \( \nnt x \le \nnt y \le (1+\rho)^{1/2}\nnt x \le (1+\rho/2)
    \nnt x \). Il vient alors :
  \begin{align}
    \Hautn(z)
    & =
    \nnt{x-y}^2
    \\ & =
    \nnt x ^2 + \nnt y ^2 - 2\langle x; y \rangle
    \\ & =
    \left( \nnt y  - \nnt x  \right)^2
    + 2 \nnt x  \nnt y  \left( 1- \cos(x, y) \right)
    \\ & \le
    \left( \frac{\rho}{2}\nnt x  \right)^2
    + 2\left( 1+\frac{\rho}{2} \right)
    \nnt x ^2 \cdot \phi
    \\ & \le
    \bigl( (\rho^2/4) + 2\phi + \rho\phi \bigr)
    \Hautn(x)
    \pmm.
    \qedhere
  \end{align}
\end{proof}

\begin{proof}[Démonstration (du théorème~\ref{t:mumford}).]
  Compte tenu de l'inégalité de gauche de~\eqref{e:Mclose} et de la
  comparaison entre hauteurs projective et normalisée, le
  lemme~\ref{l:diff-apx} donne, en prenant le logarithme :
  \begin{equation} \label{e:close-log}
    \sum\placerange
    \degv \ln\Distv(z, \vai)
    \le
    - \expapx \hautn x
    + \expapx \htcmp
    + \ln(5\sqrt2) + 3 \ln(n + 1) + \hlclab
    \pmm.
  \end{equation}
  Par ailleurs, le lemme~\ref{l:diff-small} et l'hypothèse~\eqref{e:rho-phi}
  donnent
  \begin{equation} \label{e:h2-diff}
    \hautl[2] z
    \le
    \hautn z + \htcmp
    \le
    (\rho^2/4 + 2\phi + \rho\phi)
    \hautn x
    + \htcmp
    \le
    \frac{ \expapx }{ 2d }
    \hautn x
    + \htcmp
    \pmm.
  \end{equation}

  Supposons maintenant que \( z \not\in \vai(\Qbar) \) et montrons qu'on
  contredit~\eqref{e:Mbig}. En effet, l'inégalité de \bsc{Liouville}
  (proposition~\ref{p:liouville}) appliquée à \( z \) et \( \vai \) donne
  \begin{align}
    \sum\placerange
    \degv \ln\Distv(z, \vai)
    & \ge
    - d \hautl[2] z
    - \hautl[1]{ \chow\vai }
    - \frac32 \ln(n + 1)
    - ld \ln(3d)
    \\ & \ge
    - \frac{ \expapx }{ 2 } \hautn x
    - d \htcmp
    - \hautl[1]{ \chow\vai }
    - \frac32 \ln(n + 1)
    - ld \ln(3d)
  \end{align}
  où la deuxième ligne vient en substituant~\eqref{e:h2-diff}.
  En comparant avec~\eqref{e:close-log}, il vient :
  \begin{equation}
    \frac\expapx2 \hautn x
    \le
    \hautl[1]{ \chow\vai }
    + ld \ln(3d)
    + (d + \expapx) \htcmp
    + \hlclab
    + \frac92 \ln(n + 1)
    + \ln(5\sqrt2)
  \end{equation}
  qui contredit bien~\eqref{e:Mbig}, achevant ainsi la preuve.
\end{proof}


\cleardoublepage
\endinput

% vim: spell spelllang=fr
