% !TEX root = main.tex

\chapter{Inégalité de \bsc{Mumford}} \label{chap:mumford}

\section{Énoncé principal}

On prouve ci-dessous une inégalité de \bsc{Mumford} explicite dans le cas
particulier où \( \varapx \) est une sous-variété abélienne de \( \va \),
qu'on notera donc désormais \( \vai \). Le résultat principal de ce chapitre
est le suivant.

\begin{thm} \label{t:mumford}
  Soient \( \vai \) une sous-variété abélienne de \( \va \) ;
  on note \( d = \deg \vai \), \( l = \dim\vai + 1 \).
  On choisit \( \phi > 0 \) et \( \rho > 0 \) tels que
  \( ld^2((\rho^2/4) + 2\phi + \rho\phi)< \eps \). On pose
  \( B = (4^g-1)h(\va) + 3g\log(2) \) et
  \(
    C = 3\cdot4^gh(\va)
    + \log\Bigl(
      10\sqrt2\cdot 2^{7g/2} (n+1)^6
      \bigl(\frac{16e\sqrt 2}{30}(n+1)^{9/2}\bigr)^{ld}
    \Bigr)
  \).

  Si \( x \) et \( y \) sont deux points de \( \va(\Qbar) \) tels que
  \begin{align}
    \label{e:Mapx}
    0 < \distv z \varapx
    & <
    \hautm[2] z ^{-\wtapx\expapx}
    \quad \forall \place \in \placesapx
    \quad \text{où \( z \) est \( x \) ou \( y \)}
    \\
    \label{e:Mbig}
    \Hautn(x)
    & >
    [h(\vai) + C + B(ld + \eps/d)]
    \cdot [\eps/d - ld((\rho^2/4) + 2\phi + \rho\phi)]^{-1}
    \\
    \label{e:Mcos}
    \cos(x, y)
    & >
    1 - \phi
    \\
    \label{e:Mclose}
    \Hautn(x)
    & <
    \Hautn(y) \le (1+\rho)\Hautn(x)
  \end{align}
  alors \( x - y \in \vai(\Qbar) \).
\end{thm}

La preuve consiste à montrer que la différence \( x - y \) est encore très
proche de \( \vai \) tout en étant de hauteur relativement petite, de sorte
qu'une inégalité de \bsc{Liouville} la contraint à être sur \( \vai \). On
commence donc par étudier l'action des opérations sur la distance, puis on
établit une inégalité de \bsc{Liouville}, avant d'achever la preuve du
théorème.



\section{Comportement métrique des opérations}

La situation d'approximation considérée met en jeu plusieurs types de
géométrie sur \( \va \) : la géométrie euclidienne de son espace de
\bsc{Mordell-Weil}, la géométrie algébrique et projective de la variété
plongée, une géométrie métrique, locale, en chaque place de \( S \). On
s'attend à ce que les métriques locales sur \( \va \) se comportent de façon
agréable vis-à-vis des structures géométrique et arithmétique, par exemple que
les opérations soient uniformément continues, voire lipschitziennes. On montre
ici que c'est bien le cas et on explicite une valeur admissible pour la
constante de \bsc{Lipschitz}.

\begin{prop} \label{p:addsub-dv}
  Il existe une constante \( \cst{Cp:addsub-dv} \) telle que, pour tous \( x
  \), \( x' \), \( y \), \( y' \) dans \( \va(\C_v) \), on a :
  \begin{equation}
  \max \bigl( \Distv(x \pm y, x' \pm y') \bigr)
  \le
  :\cst{Cp:addsub-dv}
  \cdot \max \bigl( \Distv(x, x'), \Distv(y, y') \bigr)
  \pmm,
  \end{equation}
  où dans le membre de gauche il faut prendre le même signe sur les deux
  termes. De plus, on peut choisir
  \begin{equation}
    \newcst{Cp:addsub-dv}
    =
    c_v(\va)^3 \av
    2^{-5g/2} \bigl( 5\sqrt2\cdot2^{7g/2}(n+1)^{9/2} \bigr)^\dv
    \pmm.
  \end{equation}
\end{prop}

\begin{proof}
  On sera naturellement amené à considérer des situations multiprojectives.
  Dans ce cas, plusieurs notions de distances peuvent être considérées, dont
  seules les deux plus élémentaires nous intéressent ici : il s'agit du
  maximum des distances sur chaque facteur d'une part, et de la distance
  déduite d'un plongement de \bsc{Segre} d'autre part. Ces distances se
  comparent bien en général ; on écrit ici la comparaison dans le cas de \(
    (\Proj^n)^2 \) (cas \( q = 2 \), \( d=(1,1) \) de
  \cite[lemme~4.3]{remgdmp}) :
  \begin{equation}
    \max_{i \in \{1, 2\}}
    \Distv(x^{(i)}, y^{(i)})
    \le
    \Distv(s(x), s(y))
    \le
    \sqrt 2^{\,\dv} \max_{i \in \{1, 2\}} \Distv(x^{(i)}, y^{(i)})
    \pmm,
  \end{equation}
  où \( x = (x^{(1)}, x^{(2)}) \in (\Proj^n)^2 \) et \( s : (\Proj^n)^2
    \to \Proj^{n^2+2n} \) est le plongement de \bsc{Segre}.

  Pour établir ce résultat, on va en fait étudier les propriétés métriques du
  morphisme \( \xi : \va^2 \to \va^2,\ (x, y) \mapsto (x + y, x - y) \). Ce
  dernier présente en effet l'avantage de pouvoir être représenté globalement
  par une famille \( F \) de polynômes totalement explicite (en particulier,
  de degré et hauteur contrôlés) à condition toutefois de regarder
  \( \va^2 \) comme plongée dans \( \Proj^N \) (où \( N = n^2+2n \)) par un
  plongement de \bsc{Segre}. Si l'on note \( z \) (resp. \( z' \)) l'image de
  \( (x, y) \) (resp. \( (x', y') \)) par ce plongement, on est ainsi ramené à
  montrer que \( \Distv(\xi z, \xi z') \le \cst{Cp:addsub-dv} \sqrt2^{\,-\dv}
    \Distv(z, z') \). Par ailleurs on pourra supposer \( \Distv( z, z') \le
    \sqrt2\, \cst{Cp:addsub-dv}^{-1} \), car sinon la conclusion du lemme~est
  vide, compte tenu du fait que la distance est bornée par \( 1 \). On
  utilisera alors un développement de \bsc{Taylor} des formes \( F \)
  représentant \( \xi \) au voisinage de \( z \).

  En comparant les termes d'ordres \( 1 \) et \( 2 \) au terme principal dans
  ce développement, va apparaître une quantité \( \Onv{F(z)} /
    (\Onnv{F}\Onv{z}^2) \le 1 \), qui détermine en grande partie le rayon sur
  lequel le développement de \bsc{Taylor} donne une majoration utile. Il est
  donc indispensable de pouvoir minorer cette quantité uniformément en \( z
  \). Le lemme~suivant donne une telle minoration en fonction de la constante
  locale \( c_v(\va) \). Notons que cette minoration est la seule étape de la
  preuve où intervient le fait que le morphisme \( \xi \) est donné par les
  opérations de
  \( \va \), le reste s'appliquant en fait à n'importe quel morphisme
  d'espaces projectifs.

  Il s'agit de montrer que \( \Onv{F(z')} \) n'est pas trop petit devant \(
    \Onv {F(z)} \) et que \( \Onv{F(z) \wedge F(z')} \) n'est pas trop grand
  devant \( \Onv {F(z)}^2 \). Comme annoncé, on supposera que \( \Distv (z,
    z') < \sqrt2^\dv\, \cst{Cp:addsub-dv}^{-1} \), c'est-à-dire, vu que \(
    \cst{Cp:addsub-dv} = (5\sqrt2\cdot N\sqrt{N+1})^\dv / \cst{Tv} \), que \(
    \Distv (z, z') = \cst{Tv}\eps_1/(5 N \sqrt{N+1})^\dv \) pour un
  certain \( \eps_1 < 1 \). On peut en particulier appliquer le
  lemme~\ref{l:dv-common-i} à \( z \) et \( z' \) (en remplaçant \( n \) par \( N \))
  et supposer, quitte à renuméroter les coordonnées, que \( \av{z'_0} \ge
    (N+1)^{-\dv/2}\Onv{z'} \) et \( z_0 \neq 0 \). On écrit alors le
  développement de \bsc{Taylor} homogène de \( F \) comme suit :
  \begin{equation}
    F(z')
    =
    \frac{ (z'_0)^2 }{ z_0^2 } F(z)
    + \sum_{1 \le i \le N} R_i
    + \sum_{1 \le i, j \le N} R_{ij}
    \pmm,
  \end{equation}
  où
  \begin{equation}
    R_i
    =
    \frac{z'_0}{z_0} \frac{\partial F}{\partial Z_i}(z)
    \left( z'_i - \frac{z'_0}{z_0} z_i \right)
  \end{equation}
  et
  \begin{equation}
    R_{ij}
    =
    \frac12 \frac{ \partial^2 F }{ \partial Z_i \partial Z_j }(z)
    \left( z'_i - \frac{z'_0}{z_0} z_i \right)
    \left( z'_j - \frac{z'_0}{z_0} z_j \right)
    \pmm.
  \end{equation}
  On utilise alors l'hypothèse sous la forme
  \begin{equation}
    \av{z'_i - \frac{z'_0}{z_0} z_i}
    \le
    \Onv{z \wedge z'} / \av{z_0}
    \le
    \frac{\cst{Tv}\,\eps_1}{(5N\sqrt{N+1})^\dv}
    \frac{\Onv z \Onv{z'}}{\av{z_0}}
    \pmm.
  \end{equation}
  Il vient alors, grâce au lemme~\ref{TvIndepF},
  \begin{align*}
    \frac{ \Onv{R_i} }{ \Onv{F(z) \cdot {z_0'}^2/z_0^2} }
    & \le
    \frac{ \Onv{z'} }{ \av{z'_0} }
    \frac{ \Onnv{F} \Onv z^2 }{ \Onv{F(z)} }
    \frac{ 2^\dv \cst{Tv}\,\eps_1 }{ (5N\sqrt{N+1})^\dv }
    \\ & \le
    \left( \frac2{ 5N\sqrt{N+1} } \right)^\dv
    \frac{ \Onv{z'} }{ \av{z'_0} } \eps_1
    \le
    \left( \frac2{5N} \right)^\dv \eps_1
    \pmm.
  \end{align*}
  On obtient de même
  \begin{equation}
    \frac{ \Onv{R_{ij}} }{ \Onv{F(z) \cdot {z_0'}^2/z_0^2} }
    \le
    \left( \frac1{25N^2} \right)^\dv \eps_1
    \pmm,
  \end{equation}
  puis en sommant : \( \Onv{\sum R_i + \sum R_{ij}} \le (1/2)^\dv \Onv{F(z)
      \cdot {z_0'}^2/z_0^2} \eps_1 \). Enfin, les inégalités triangulaire aux
  places infinies et ultramétrique aux places finies donnent \( \Onv{F(z')}
    \ge (1/2)^\dv \Onv{F(z) \cdot {z_0'}^2/z_0^2} \). Afin d'estimer la
  distance, il reste à majorer \( \Onv{F(z) \wedge F(z')} \). Pour ce faire,
  on développe le second facteur, on remarque que le premier terme du produit
  est nul et on majore brutalement le terme restant par le produit des normes
  ; il vient ainsi :
  \begin{align*}
    \Distv(\xi(z), \xi(z'))
    & =
    \frac{ \Onv{F(z) \wedge F(z')} }{ \Onv{F(z)} \Onv{F(z')} }
    \\ & \le
    \frac{
      2^\dv \Onv{\sum R_i + \sum R_{ij}} \Onv{F(z)}
    }{
      \Onv{F(z)} \Onv{F(z) \cdot {z_0'}^2/z_0^2}
    }
    \\ & \le
    \eps_1
    \pmm,
  \end{align*}
  qui, vu la définition de \( \eps_1 \), équivaut à la conclusion de la
  proposition.
\end{proof}



\section{Inégalité de \bsc{Liouville}} \label{sec:liouville}

On établit ici l'analogue suivant de la classique inégalité de
\bsc{Liouville}.

\begin{prop} \label{p:liouville}
  Soit \( V \) une sous-variété de \( \projd \), de dimension \( u \) et de
  degré \( d \). Pour tout point \( x \in \projd(\Qbar) \), on a soit \( x \in
  V(\Qbar) \) soit
  \begin{equation}
    \prod_{v \in \placesapx} \distv x V ^\degv
    \ge
    \frac1{
      (n+1)^{3/2}
      (3d)^{d (u+1)}
      \, \hautm[1]{ \chow V }
      \, \hautm[2] x ^d
    }
    \pmm.
  \end{equation}
\end{prop}

Ce type d'inégalité se prouve de façon naturelle avec la distance algébrique
(section~\ref{sec:distv-cmp}). Cependant, si l'on traite directement le cas
général en utilisant la définition, on obtient \( d (u+1) \) comme exposant de
\( \hautm[2] x \). On commence donc par le cas d'une hypersurface, où la
distance algébrique admet une expression plus simple, puis on remarque que
tout variété est une intersection d'hypersurfaces.

On pourrait se ramener au cas d'un hyperplan, où la distance admet une
expression simple, par un plongement de \bsc{Veronese} (remodelé) comme dans
le lemme~\ref{l:hs-vero}, mais on obtient en fait de meilleurs constantes en
utilisant la propriété~\ref{p:dv-p2alg-hs}.

\begin{lem} \label{l:liou-hs}
  Soit \( V \) une hypersurface de \( \projd \), d'équation \( F \) et de
  degré \( d \). Pour tout point \( x \in \projd(\Qbar) \), on a soit \( x \in
  V(\Qbar) \) soit
  \begin{equation}
    \prod_{v \in \placesapx} \distv x V ^\degv
    \ge
    \frac1{
      (17/8)^d (n+1)^{3/2}
      \, \hautm[2] G
      \, \hautm[2] x ^d
    }
    \pmm.
  \end{equation}
\end{lem}

\begin{proof}
  Plutôt que la proposition~\ref{p:dv-p2alg-hs}, on utilise en fait
  l'inégalité~\eqref{e:dv-p2alg-hs} établie au cours de sa preuve afin
  d'éviter une comparaison norme-mesure superflue. On remarque également qu'on
  peut supposer \( m = 1 \) dans cette comparaison. On a alors :
  \begin{align}
    \prod_{v \in \placesapx}
    \distv x V ^\degv
    & \ge
    \prod_{v \in M(\cdn)}
    \distv x V ^\degv
    \\ & \ge
    \prod_{v \in M(\cdn)}
    \left(
      \frac{27}{26} (n+1)^{3/2}
      \left( \frac{164}{81} \right)^{d}
    \right)^{-\dv}
    \frac{ \av{G(x)} }{ \nv2 x^{d} \nv2 G }
  \end{align}
  qui implique le résultat annoncé, \lat{via} une simple comparaison
  numérique pour la constante.
\end{proof}

\begin{proof}[\proofname{} de la proposition~\ref{p:liouville}]
  Si \( x \not\in V(\Qbar) \), la proposition~6.1 de~\cite{remdcl}
  donne une forme \( G \) homogène de degré \( d \), contenant \( V \) mais
  pas \( x \), telle que
  \begin{equation}
    \hautm[1] G
    \le
    \hautm[1] V \cdot (d + 1)^{ d (u + 1) }
  \end{equation}
  en tenant compte des différences de hauteurs utilisées, comme observé dans
  la preuve du lemme~\ref{l:hs-choice}. Il suffit alors d'appliquer la
  propriété précédente à \( \zeros G \) et de remarquer que \( 17/8 \cdot
  (d+1)^2 \le (3d)^2 \) pour conclure.
\end{proof}



\section{Inégalité de \bsc{Mumford}}

La stratégie de la preuve est la suivante : si on a deux approximations
exceptionnelles, suffisamment proches dans l'espace de \bsc{Mordell-Weil} (au
sens des hypothèses \ref{e:Mcos} et \ref{e:Mclose} du
théorème~\ref{t:mumford}) et de hauteur assez grande, on fabrique, (en prenant
leur différence, sous l'hypothèse que \( \vai \) est un sous-groupe) une
approximation de qualité telle qu'elle appartiendra nécessairement à \( \vai
\), d'après l'inégalité de \bsc{Liouville}.

\begin{lem} \label{l:diff-close}
  Si \( z = x - y \), où \( x \) et \( y \) satisfont
  l'hypothèse \eqref{e:Mapx} du théorème~\ref{t:mumford}, on a \( \prod_{v \in
      S} \Distv(z, \vai)^{e_v} \le \cst{diff-close} H^{-\eps/d} \), avec
  \begin{equation}
    \newcst[]{diff-close}
    =
    H(\va)^{3\cdot4^g}
    10\sqrt2\,(n+1)^6
    2^{7g/2}
    \left(
      \frac{(2n^2+1)(n+1)^2}{2n}
      e^{ \gamma_{\frac{(n+1)n}{2}} + \gamma_{n+1} }
    \right)^{ld}
  \end{equation}
  et \( H = \min\bigl( H(x), H(y) \bigr) \).
\end{lem}

\begin{proof}
  On a par hypothèse \( \max(\Distv(x, \vai), \Distv(y, \vai)) \le
    H^{-\lambda_v \eps} \).  On sait alors grâce au fait~\ref{f:closest-point}
  qu'il existe des points \( x',y' \in \vai(\C_v) \) tels que
  \begin{equation}
    \max(\Distv(x, x'), \Distv(y, y'))
    \le
    H^{-\lambda_v \eps/d} e^{\dv \gamma_{n+1}}
    \pmm.
  \end{equation}
  On a alors en utilisant successivement la proposition~\ref{p:dv-p2alg} et la
  proposition~\ref{p:addsub-dv} :
  \begin{align*}
    \Distv(z, \vai)
    & \le
    \cst{CPonctuelAlg}\, \Distv(x - y, x' - y')
    \\ & \le
    \cst{CPonctuelAlg}\, \cst{Cp:addsub-dv}\,
    \max(\Distv(x, x'), \Distv(y, y'))
    \\ & \le
    \cst{CPonctuelAlg}\, \cst{Cp:addsub-dv}
    \, e^{\dv \gamma_{n+1}}H^{-\lambda_v \eps/d}
    \pmm.
  \end{align*}
  On prend alors le produit sur \( v \in S \), en supposant (c'est le cas
  défavorable) que \( S \) contient toutes les places divisant \( 2 \) ou \(
    \infty \).  Comme \( c_v(\va) \ge 1 \), on a \( \prod_{v \in S} c_v(\va)
    \le \prod_{v \in M(\cdn)} c_v(\va) \le H(\va)^{4^g} \). Vu la
  normalisation des \( \lambda_v \), on a ainsi la conclusion du lemme.
\end{proof}

\begin{lem} \label{l:diff-small}
  Soient \( x \) et \( y \) satisfaisant aux hypothèses \ref{e:Mcos}
  et \ref{e:Mclose} du théorème~\ref{t:mumford}, notons \( z \) leur
  différence. On a alors \( \Hautn(z) \le (\rho^2/4 + 2\phi + \rho\phi)
    \Hautn(x) \).
\end{lem}

\begin{proof}
  On note \( \langle\truc; \truc \rangle \) le produit scalaire et \(
    \Onv[N]{\truc} = \sqrt{\Hautn(\truc)} \) la norme dans l'espace de
  \bsc{Mordell-Weil}. On remarque de plus que l'hypothèse \ref{e:Mclose}
  implique
  \( \Onv[N]{y} \le (1+\rho)^{1/2}\Onv[N]{x} \le (1+\rho/2)
    \Onv[N]{x} \). Il vient alors :
  \begin{align*}
    \Hautn(z)
    & =
    \Onv[N]{x-y}^2
    \\ & =
    \Onv[N]{x}^2 + \Onv[N]{y}^2 - 2\langle x; y \rangle
    \\ & =
    \left( \Onv[N]{y} - \Onv[N]{x} \right)^2
    + 2 \Onv[N]{x} \Onv[N]{y} \left( 1- \cos(x, y) \right)
    \\ & \le
    \left( \frac{\rho}{2}\Onv[N]{x} \right)^2
    + 2\left( 1+\frac{\rho}{2} \right)
    \Onv[N]{x}^2 \cdot \phi
    \\ & \le
    \bigl( (\rho^2/4) + 2\phi + \rho\phi \bigr)
    \Hautn(x)
    \pmm.
    \qedhere
  \end{align*}
\end{proof}

\begin{proof}[Démonstration (du théorème~\ref{t:mumford}).]
  En utilisant les conclusions des deux lemmes précédents, on montre que \(
    z \) ne satisfait pas à la conclusion la proposition~\ref{p:liouville}
  dès que \( x \) satisfait l'hypothèse \ref{e:Mbig} du
  théorème~\ref{t:mumford}.  On a besoin de pouvoir comparer les hauteurs
  projective et normalisée. On utilise à cet effet le lemme~3.9 de
  \cite{daphimhva2} et la remarque subséquente : \( \lvert\Hautn(\truc) -
    h(\truc)\rvert \le B \), avec \( B \) comme dans l'énoncé du théorème. En
  injectant ceci dans le lemme~\ref{l:diff-close}, on a \( \sum_{v \in S} e_v
    \log\Distv(z, \vai) \le -(\eps/d) \Hautn(x) + B\eps/d +
    \cst{diff-close} \). De même, le lemme~\ref{l:diff-small} donne \( h(z) \le
    ((\rho^2/4) + 2\phi + \rho\phi) \Hautn(x) + B \). Ces estimations
  contredisent la conclusion de la proposition~\ref{p:liouville} dès que
  \begin{equation}
    \bigl(\eps/d - ld((\rho^2/4) + 2\phi + \rho\phi)\bigr)
    \Hautn(x)
    >
    h(\vai) + \cst{diff-close} + B(ld + \eps/d)
    \pmm.
  \end{equation}
  Or, un calcul facile montre que la constante \( C \) du théorème majore \(
    \cst{diff-close} \) et que la relation précédente est impliquée par
  l'hypothèse \ref{e:Mbig} du théorème ; cette dernière oblige donc \( z
  \) à être sur \( \vai \).
\end{proof}


\cleardoublepage
\endinput

% vim: spell spelllang=fr
