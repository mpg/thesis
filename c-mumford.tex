% !TEX root = main.tex

\chapter{Inégalité de \bsc{Mumford}} \label{chap:mumford}

\section{Énoncés principaux}

On prouve ici deux inégalités de \bsc{Mumford} explicites : la première dans
le cas particulier où \( \varapx \) est un translaté d'une sous-variété
abélienne de \( \va \), la deuxième dans le cas général. À part être plus
simple à démonter et servir d'échauffement pour le cas général, la version
particulière présente deux intérêts majeurs : elle s'énonce sur \( \Qbar \)
sans devoir se restreindre à un sous-groupe de type fini de \( \va(\Qbar) \),
et les constantes obtenus y sont sensiblement meilleures.

\begin{thm} \label{t:mumford-grp}
  Soient \( \vai \) une sous-variété abélienne de \( \va \) et \( \varapx \)
  un translaté de \( \vai \) par un point algébrique ;
  on note \( d = \deg \vai \) et \( l = \dim\vai + 1 \).
  On choisit \( \expapx > 0 \) puis \( \phi > 0 \) et \( \rho > 0 \) tels que
  \begin{equation} \label{e:rho-phi-grp}
    \frac{ \rho^2 }{ 4 } + \rho\phi + 2\phi
    \le
    \frac{ \expapx }{ 2d }
    \pmm.
  \end{equation}
  Si \( x \) et \( y \) sont deux points de \( \va(\Qbar) \) tels que
  \begin{align}
    0 < \distv z \varapx
    & \le
    \hautm[2] z ^{-\wtapx\expapx}
    \quad \forall \place \in \placesapx
    \quad \text{où \( z \) est \( x \) ou \( y \)}
    \label{e:Mapx}
    \\
    \hautn x
    & \ge
    \frac2\expapx \bigl(
      \hautl[1]{ \chow\vai }
      + ld \ln(3d)
      + (d + 1) \htcmp
      + \hlclab
      + 6 \ln(n + 1)
    \bigr)
    \label{e:Mbig}
    \\
    \cos(x, y)
    & \ge
    1 - \phi
    \label{e:Mcos}
    \\
    \hautn x
    & \le
    \hautn y \le (1+\rho) \hautn x
    \label{e:Mclose}
  \end{align}
  alors \( x - y \in \vai(\Qbar) \).
\end{thm}

On remarquera que la conclusion obtenue est plus faible que celle qu'on
pourrait attendre naïvement, à savoir \( x = y \), qui ne semble pas
accessible pour le moment. Cette obstruction sera discutée plus en détails à
la section~\vref{sec:obstruction}.

Pour le cas général, l'énoncé est le suivant.

\begin{thm} \label{t:mumford-gen}
  Soit \( \varapx \) une sous-variété de \( \va \) ; on note \( d = \deg
    \varapx \) et \( l = \dim\varapx + 1 \). On fixe un sous-groupe \( \Gamma
    \subset \va(\Qbar) \) de type fini dont on note \( r \) le rang, puis on
  pose \( m = \todots \). On choisit alors \( \expapx > 0 \) puis \( \phi > 0
  \) et \( \rho > 0 \) tels que
  \begin{equation} \label{e:rho-phi-gen}
    \frac{ \rho^2 }{ 4 } + \rho\phi + 2\phi
    \le
    \frac{ \expapx }{ 2d^m }
    \pmm.
  \end{equation}
  Si \( x_1, \dots, x_m \) est une famille de points de \( \Gamma \)
  telle que, pour tous \( i \) et \( j \) entre \( 1 \) et \( m \) :
  \begin{align}
    0 < \distv{ x_i }\varapx
    & \le
    \hautm[2]{ x_i }^{-\wtapx\expapx}
    \quad \forall \place \in \placesapx
    \label{e:Mapx-gen}
    \\
    \hautn{ x_m }
    & \ge
    \dots
    \label{e:Mbig-gen}
    \\
    \cos(x_i, x_j)
    & \ge
    1 - \phi
    \label{e:Mcos-gen}
    \\
    \hautn{ x_m }
    & \le
    \hautn{ x_i }
    \le
    (1+\rho) \hautn{ x_m }
    \label{e:Mclose-gen}
  \end{align}
  alors il existe une sous-variété abélienne \( \vai \subset \va \) dont un
  translaté est contenue dans \( \varapx \) et deux indices \( i \) et \( j \)
  tels que \( x_i - x_j \in \vai \).
\end{thm}

Avant d'attaquer les preuves proprement dites, on démontre une inégalité de
\bsc{Liouville}, qui est intéressante en elle-même et sera un ingrédient
crucial dans les inégalités de \bsc{Mumford}, puis on explicite quelques
aspects métriques des opérations de \( \va \) : cette étape sera utile non
seulement pour la preuve des inégalités de \bsc{Mumford}, mais aussi pour
expliciter l'obstruction à la version naïve de cette inégalité.

Une fois réunis ces ingrédients techniques, on peut conclure directement dans
le cas des translatés de sous-variété abéliennes ; pour le cas général, on
étudie préalablement un morphisme « des différences » qui joue un rôle central
dans la preuve.


\section{Inégalité de \bsc{Liouville}} \label{sec:liouville}

On établit ici l'analogue suivant de la classique inégalité de
\bsc{Liouville}.

\begin{prop} \label{p:liouville}
  Soit \( V \) une sous-variété de \( \projd \), de dimension \( u \) et de
  degré \( d \). Pour tout point \( x \in \projd(\Qbar) \), on a soit \( x \in
  V(\Qbar) \) soit
  \begin{equation}
    \prod_{v \in \placesapx} \distv x V ^\degv
    \ge
    \frac1{
      (n+1)^{3/2}
      (3d)^{d (u+1)}
      \, \hautm[1]{ \chow V }
      \, \hautm[2] x ^d
    }
    \pmm.
  \end{equation}
\end{prop}

Ce type d'inégalité se prouve de façon naturelle avec la distance algébrique
(section~\vref{sec:distv-cmp}). Cependant, si l'on traite directement le cas
général en utilisant la définition, on obtient \( d (u+1) \) comme exposant de
\( \hautm[2] x \). On commence donc par le cas d'une hypersurface, où la
distance algébrique admet une expression plus simple, puis on remarque que
tout variété est une intersection d'hypersurfaces.

On pourrait se ramener au cas d'un hyperplan, où la distance admet une
expression simple, par un plongement de \bsc{Veronese} (remodelé) comme dans
le lemme~\vref{l:hs-vero}, mais on obtient en fait de meilleurs constantes en
utilisant la propriété~\vref{p:dv-p2alg-hs}.

\begin{lem} \label{l:liou-hs}
  Soit \( V \) une hypersurface de \( \projd \), d'équation \( F \) et de
  degré \( d \). Pour tout point \( x \in \projd(\Qbar) \), on a soit \( x \in
  V(\Qbar) \) soit
  \begin{equation}
    \prod_{v \in \placesapx} \distv x V ^\degv
    \ge
    \frac1{
      (17/8)^d (n+1)^{3/2}
      \, \hautm[2] G
      \, \hautm[2] x ^d
    }
    \pmm.
  \end{equation}
\end{lem}

\begin{proof}
  Plutôt que la proposition~\vref{p:dv-p2alg-hs}, on utilise en fait
  l'inégalité~\eqref{e:dv-p2alg-hs} établie au cours de sa preuve afin
  d'éviter une comparaison norme-mesure superflue. On remarque également qu'on
  peut supposer \( m = 1 \) dans cette comparaison. On a alors :
  \begin{align}
    \prod_{v \in \placesapx}
    \distv x V ^\degv
    & \ge
    \prod_{v \in M(\cdn)}
    \distv x V ^\degv
    \\ & \ge
    \prod_{v \in M(\cdn)}
    \left(
      \left(
        \frac{27}{26} (n+1)^{3/2}
        \left( \frac{164}{81} \right)^{d}
      \right)^{-\dv}
      \frac{ \av{G(x)} }{ \nv2 x^{d} \nv2 G }
    \right)^\degv
  \end{align}
  qui implique le résultat annoncé, \lat{via} une simple comparaison
  numérique pour la constante.
\end{proof}

\begin{proof}[\proofname{} de la proposition~\vref{p:liouville}]
  Si \( x \not\in V(\Qbar) \), la proposition~6.1 de~\cite{remdcl}
  donne une forme \( G \) homogène de degré \( d \), contenant \( V \) mais
  pas \( x \), telle que
  \begin{equation}
    \hautm[1] G
    \le
    \hautm[1] V \cdot (d + 1)^{ d (u + 1) }
  \end{equation}
  en tenant compte des différences de hauteurs utilisées, comme observé dans
  la preuve du lemme~\vref{l:hs-choice}. Il suffit alors d'appliquer le
  lemme précédent à \( \zeros G \) et de remarquer que \( 17/8 \cdot (d+1)^2
    \le (3d)^2 \) pour conclure.
\end{proof}



\section{Comportement métrique des opérations}

La situation d'approximation considérée met en jeu plusieurs types de
géométrie sur \( \va \) : la géométrie euclidienne de son espace de
\bsc{Mordell-Weil}, la géométrie algébrique et projective de la variété
plongée, une géométrie métrique, locale, en chaque place de \( \placesapx \). On
s'attend à ce que les métriques locales sur \( \va \) se comportent de façon
agréable vis-à-vis des structures géométrique et arithmétique, par exemple que
les opérations soient uniformément continues, voire lipschitziennes. On montre
ici que c'est bien le cas et on explicite une valeur admissible pour la
constante de \bsc{Lipschitz}.

\begin{prop} \label{p:addsub-dv}
  Pour tous \( x \), \( x' \), \( y \), \( y' \) dans \( \va(\C_v) \), on a :
  \begin{equation}
  \Distv(x \pm y, x' \pm y')
  \le
  \max \bigl( \Distv(x, x'), \Distv(y, y') \bigr)
  \cdot \hmclab* \bigl( 5 \sqrt2 (n + 1)^3 \bigr)^\dv
  \pmm,
  \end{equation}
  où dans le membre de gauche il faut prendre le même signe des deux côtés.
\end{prop}

\begin{proof}
  On considère le plongement de \bsc{Segre} \( s \colon (\projd)^2 \to
    \proj{n'} \) avec \( n' = n^2 + 2n \). D'après le lemme~4.3 (p.~121) de
  \cite{remgdmp}, appliqué avec \( q = 2 \) et \( d = (1,1) \), ainsi que le
  paragraphe~2.3 (p.~103 notamment) de cette référence donnant le lien entre
  indices (d'une forme de \bsc{Chow} ou, par extension, d'une distance) et
  plongements de \bsc{Segre} (et de \bsc{Veronese} remodelé), on a
  \begin{equation} \label{e:dv-segre}
    1
    \le
    \frac{
      \Distv(s(x, x'), s(y, y'))
    }{
      \max \bigl( \Distv(x, x'), \Distv(y, y') \bigr)
    }
    \le
    2^{\dv/2}
  \end{equation}
  dans le cas où le dénominateur est différent de zéro (dans le cas contraire,
  on a \( x = x' \) et \( y = y' \) et le résultat est immédiat).

  On considère alors le morphisme
  \( \xi : \va^2 \to \va^2,\ (x, y) \mapsto (x + y, x - y) \), qui
  présente l'avantage de pouvoir être représenté globalement
  par une famille \( F \) de formes de degré \( 2 \) dans le plongement de
  \bsc{Segre} (voir section~\vref{sec:vaemb}). Si l'on note \( z = s(x, y) \)
  et \( z' = s(x', y') \) les images dans ce plongement, on est ainsi ramené à
  montrer que
  \begin{equation} \label{e:addsub-dv}
    \Distv(\xi z, \xi z')
    \le
    \Distv(z, z')
    \cdot \hmclab* \bigl( 5 (n + 1)^3 \bigr)^\dv
  \end{equation}
  en appliquant deux fois~\eqref{e:dv-segre}. On peut donc supposer
  \( \Distv( z, z') \le \bigl( 5 (n + 1)^3 \bigr)^{-\dv} \), car sinon
  l'inégalité précédente est banale.

  On va utiliser un développement de \bsc{Taylor} au voisinage de \( z \) des
  formes représentant \( \xi \) pour montrer que \( \nv2{F(z')} \) n'est pas
  trop petit devant \( \nv2 {F(z)} \) puis que \( \nv2{F(z) \wedge F(z')} \)
  n'est pas trop grand devant \( \nv2 {F(z)}^2 \). L'hypothèse faite sur la
  distance permet d'appliquer le lemme~\vref{l:dv-common-i} à \( z \) et \( z'
  \) (en remplaçant \( n \) par \( n' = n^2 + 2n \)) et donc de supposer,
  quitte à renuméroter les coordonnées, que \( z_0 = z'_0 = 1 \) et \( \nv2{
      z' } \le (n')^{\dv/2} \).  Le développement de
  \bsc{Taylor} de \( F \) s'écrit alors :
  \begin{equation}
    F(z')
    =
     F(z)
    + \sum_{1 \le i \le n'}
    \underbrace{
      \frac{\partial F}{\partial Z_i}(z)
      \cdot ( z'_i - z_i )
    }_{\textstyle R_i}
    + \sum_{1 \le i, j \le n'}
    \underbrace{
      \frac12 \frac{ \partial^2 F }{ \partial Z_i \partial Z_j }(z)
      \cdot ( z'_i - z_i ) ( z'_j - z_j )
    }_{\textstyle R_{ij}}
    \pmm.
  \end{equation}
  On remarque alors que
  \(
    \av{z'_i - z_i}
    \le
    \nv2{z \wedge z'}
    \le
    \nv2 z \nv2{z'}
    \distv{ z }{ z' }
  \)
  pour estimer \( R_i \) en utilisant de plus~\eqref{e:addsub-loc} :
  \begin{align}
    \nv2{R_i}
    & \le
    2^\dv \nnv2 F \nv2 z
    \cdot \nv2 z \nv2{ z' }
    \distv{ z }{ z' }
    \\ & \le
    \hmclab* \nv2{ F(z) }
    (2 \sqrt{n'})^\dv
    \distv{ z }{ z' }
    \pmm.
  \end{align}
  On obtient de même
  \(
    \nv2{R_{ij}}
    \le
    \hmclab* \nv2{ F(z) }
    (n')^\dv
    \distv{ z }{ z' }^2
    \pmm.
  \)
  En prenant la somme, on obtient un facteur \( (n')^\dv \) supplémentaire
  pour les \( R_i \), et son carré pour les \( R_{ij} \). Au final on a :
  \begin{align}
    \nv*2{\sum R_i + \sum R_{ij}}
     & \le
    \nv2{F(z)}
    \cdot \hmclab* \left( \frac52 \cdot (n')^{3/2} \right)^\dv
    \distv{ z }{ z' }
     <
    \left( \frac12 \right)^\dv
    \nv2{F(z)}
  \end{align}
  en remarquant aux places archimédiennes que \( 2t + t^2 \le 5t/2 \) pour \(
    0 \le t \le 1/2 \), ce qui est le cas de \( t = (n')^{3/2}
    \distv{ z }{ z' } \) vu l'hypothèse sur la distance et le fait que \( n'
    < (n+1)^2 \).

  Les inégalités triangulaire aux places infinies et ultramétrique aux places
  finies donnent alors \( \nv2{F(z')} \ge (1/2)^\dv \nv2{F(z)} \).
  Afin d'estimer la distance, il reste à majorer \( \nv2{F(z) \wedge F(z')}
  \). Pour ce faire, on développe le second facteur, on remarque que le
  premier terme du produit est nul et on majore brutalement le terme restant
  par le produit des normes ; il vient ainsi :
  \begin{align}
    \Distv(\xi(z), \xi(z'))
    & =
    \frac{ \nv2{F(z) \wedge F(z')} }{ \nv2{F(z)} \nv2{F(z')} }
    \\ & \le
    \frac{
      \nv2{F(z)} \nv2{\sum R_i + \sum R_{ij}}
    }{
      \nv2{F(z)} \nv2{F(z)} \cdot 2^{-\dv}
    }
    \\ & \le
    \hmclab* \bigl( 5 (n')^{3/2} \bigr)^\dv
    \distv{ z }{ z' }
  \end{align}
  qui donne bien~\eqref{e:addsub-dv} et achève la preuve.
\end{proof}



\section{Inégalité de \bsc{Mumford} pour les translatés de sous-groupes}
\label{sec:mumford-grp}

La stratégie de la preuve est la suivante : si on a deux approximations
exceptionnelles, suffisamment proches dans l'espace de \bsc{Mordell-Weil} (au
sens des hypothèses \eqref{e:Mcos} et \eqref{e:Mclose} du
théorème~\vref{t:mumford-grp}) et de hauteur assez grande, on fabrique, (en prenant
leur différence, sous l'hypothèse que \( \vai \) est un sous-groupe) une
approximation de qualité telle qu'elle appartiendra nécessairement à \( \vai
\), d'après l'inégalité de \bsc{Liouville}.

\begin{lem} \label{l:diff-apx}
  Soient \( x \) et \( y \) dans \( \proj(\Qbar) \), on pose \( z = x -
    y \) et \( H = \min\bigl( H(x), H(y) \bigr) \).
  Si \( x \) et \( y \) satisfont à~\eqref{e:Mapx}, on a
  \begin{equation}
    \prod\placerange
    \Distv(z, \vai)^\degv
    \le
    H^{-\expapx}
    \cdot 5 \sqrt2 (n + 1)^3 \hmclab
    \pmm.
  \end{equation}
\end{lem}

\begin{proof}
  En chaque place \( v \), il existe des points \( x'_v \) et \( y'_v \) dans
  \( \varapx(\Cv) \) tels que \( \distv x \varapx = \distv{ x }{ x'_v } \) et
  \( \distv y \varapx = \distv{ y }{ y'_v } \). L'hypothèse~\eqref{e:Mapx}
  donne alors
  \begin{equation}
    \max \bigl( \Distv(x, x'_v), \Distv(y, y'_v) \bigr)
    \le
    H^{-\wtapx \expapx}
    \pmm.
  \end{equation}
  Comme \( \varapx \) est translaté d'un sous-groupe \( \vai \), on a
  \( x'_v - y'_v \in \vai(\Cv) \), donc la proposition~\vref{p:addsub-dv} donne
  \begin{equation}
    \distv z \vai
    \le
    \distv z {x'_v - y'_v}
    \le
    H^{-\wtapx \expapx}
    \cdot \hmclab* \bigl( 5 \sqrt2 (n + 1)^3 \bigr)^\dv
    \pmm.
  \end{equation}
  On prend alors le produit sur \( v \in \placesapx \), en supposant (c'est le
  cas défavorable) que \( \placesapx \) contient toutes les places
  archimédiennes et toutes celles où \( \hmclab* > 1 \).  La normalisation \(
    \sum\placerange \wtapx \degv = 1 \) permet alors de conclure.
\end{proof}

\begin{lem} \label{l:diff-small}
  Soient \( x \) et \( y \) satisfaisant aux hypothèses \eqref{e:Mcos}
  et \eqref{e:Mclose} du théorème~\vref{t:mumford-grp}, notons \( z \) leur
  différence. On a alors \( \hautn z \le (\rho^2/4 + 2\phi + \rho\phi)
    \hautn x \).
\end{lem}

\begin{proof}
  On note \( \scalnt \truc\truc \) le produit scalaire et \(
    \nnt\truc = \sqrt{\hautn\truc} \) la norme dans l'espace de
  \bsc{Mordell-Weil}. On remarque de plus que l'hypothèse \eqref{e:Mclose}
  implique
  \( \nnt x \le \nnt y \le (1+\rho)^{1/2}\nnt x \le (1+\rho/2)
    \nnt x \). Il vient alors :
  \begin{align}
    \Hautn(z)
    & =
    \nnt{x-y}^2
    \\ & =
    \nnt x ^2 + \nnt y ^2 - 2 \scalnt x y
    \\ & =
    \left( \nnt y  - \nnt x  \right)^2
    + 2 \nnt x  \nnt y  \left( 1- \cos(x, y) \right)
    \\ & \le
    \left( \frac{\rho}{2}\nnt x  \right)^2
    + 2\left( 1+\frac{\rho}{2} \right)
    \nnt x ^2 \cdot \phi
    \\ & \le
    \bigl( (\rho^2/4) + 2\phi + \rho\phi \bigr)
    \Hautn(x)
    \pmm.
    \qedhere
  \end{align}
\end{proof}

\begin{proof}[Démonstration (du théorème~\vref{t:mumford-grp}).]
  \label{page:demo-mumgrp}
  Compte tenu de l'inégalité de gauche de~\eqref{e:Mclose} et de la
  comparaison entre hauteurs projective et normalisée, le
  lemme~\vref{l:diff-apx} donne, en prenant le logarithme :
  \begin{equation} \label{e:close-log}
    \sum\placerange
    \degv \ln\Distv(z, \vai)
    \le
    - \expapx \hautn x
    + \expapx \htcmp
    + \ln(5\sqrt2) + 3 \ln(n + 1) + \hlclab
    \pmm.
  \end{equation}
  Par ailleurs, le lemme~\vref{l:diff-small} et
  l'hypothèse~\eqref{e:rho-phi-grp} donnent
  \begin{equation} \label{e:h2-diff}
    \hautl[2] z
    \le
    \hautn z + \htcmp
    \le
    (\rho^2/4 + 2\phi + \rho\phi)
    \hautn x
    + \htcmp
    \le
    \frac{ \expapx }{ 2d }
    \hautn x
    + \htcmp
    \pmm.
  \end{equation}

  Supposons maintenant que \( z \not\in \vai(\Qbar) \) et montrons qu'on
  contredit~\eqref{e:Mbig}. En effet, l'inégalité de \bsc{Liouville}
  (proposition~\vref{p:liouville}) appliquée à \( z \) et \( \vai \) donne
  \begin{align}
    \sum\placerange
    \degv \ln\Distv(z, \vai)
    & \ge
    - d \hautl[2] z
    - \hautl[1]{ \chow\vai }
    - \frac32 \ln(n + 1)
    - ld \ln(3d)
    \\ & \ge
    - \frac{ \expapx }{ 2 } \hautn x
    - d \htcmp
    - \hautl[1]{ \chow\vai }
    - \frac32 \ln(n + 1)
    - ld \ln(3d)
  \end{align}
  où la deuxième ligne vient en substituant~\eqref{e:h2-diff}.
  En comparant avec~\eqref{e:close-log}, il vient :
  \begin{equation}
    \frac\expapx2 \hautn x
    \le
    \hautl[1]{ \chow\vai }
    + ld \ln(3d)
    + (d + \expapx) \htcmp
    + \hlclab
    + \frac92 \ln(n + 1)
    + \ln(5\sqrt2)
  \end{equation}
  qui contredit bien~\eqref{e:Mbig}, achevant ainsi la preuve.
\end{proof}


\clearpage % XXX provisoire

\section{Étude du morphisme des différences}

Soit \( X \) une sous-variété de \( \va \) ; on note \( D = \deg X \) et \( u
  = \dim X \). Remarquons de suite que le degré de \( X \) est nécessairement
au moins \( 2 \), car si \( X \) était linéaire, elle contiendrait une droite
projective, c'est-à-dire une courbe de genre \( 0 \), or il est impossible de
plonger une telle courbe dans une variété abélienne.

Pour tout entier \( m \ge 2 \) on définit un morphisme
\begin{align}
  s_m \colon \va^m & \to \va^{m-1} \\
  (x_1, \dots, x_m) & \mapsto (x_1 - x_m, \dots, x_{m-1} - x_m)
\end{align}
et on se propose d'étudier (dimension, degré, hauteur) l'image \( s_m(X^m) \),
qu'on notera \( Z \). Le degré et la hauteur seront relatifs au
plongement projectif suivant :
\begin{equation}
  s_m(X^m) \subset \va^{m-1} \embedin (\projd)^{m-1} \to \proj N
\end{equation}
où la deuxième flèche est un plongement de \bsc{Segre}, donc \( N =
  (n+1)^{m-1} - 1 \).

On commence par étudier la forme des fibres du morphisme restreint à \( X \).
Dans l'énoncé suivant, \( S_X \) désigne le stabilisateur de \( X \).

\begin{lem}
  Pour tous \( m \ge 2 \) et \( x \in X^m \), on a
  \begin{equation}
    x + \delta(S_X)
    \subset
    s_m^{-1} ( s_m(x) ) \cap X^m
    \subset
    x + \delta(\bigcap_{i=1}^m X - x_i)
  \end{equation}
  où \( \delta \colon X \to X^m \) est le plongement diagonal. En particulier,
  si \( S_X = \bigcap_{i=1}^m X - x_i \), ces inclusions sont des égalités.
\end{lem}

\begin{proof}
  Par définition de \( s_m \), il est clair que \( x + \delta(A) \subset
    s_m^{-1} ( s_m(x) ) \). Par ailleurs, la définition de \( S_X \) donne
  clairement \( x + \delta(S_X) \subset X \), ce qui établit l'inclusion
  annoncé.

  Dans l'autre sens, pour tout \( y \in s_m^{-1} ( s_m(x) ) \), on a
  \( y_m - x_m = y_i - x_i \) pour tout \( i \) donc \( y = x +
    \delta(y_m - x_m) \). Si de plus \( y \in X^m \) on a \( y_m - x_m \in
    \bigcap_{i=1}^m X - x_i \).
\end{proof}

Étudions dans quel cas l'égalité peut être obtenue. Par définition du
stabilisateur, on a (ensemblistement) :
\begin{equation}
  S_X = \bigcap_{x \in X(\Qbar)} X - x
  \pmm.
\end{equation}
Par noethérieneté, il est clair qu'on peut en fait prendre
une intersection finie ; le lemme suivant précise ce résultat.

\begin{lem}
  Posons \( M = 2 D^{u+1} - 2 \) ; pour tout \( m \ge M \) il existe
  \( (x_1, \dots, x_m) \in X^m(\Qbar) \) tel que
  \( S_X = \bigcap_{i=1}^m X - x_i \).
\end{lem}

\begin{proof}
  En partant d'un \( X - x_1 \) pour \( x_1 \) quelconque, on va couper
  successivement par des \( X - x_i \) choisis de sorte à faire chuter la
  dimension d'au moins une des composantes de dimension maximale de
  l'intersection partielle qui n'est pas une composante de \( S_X \).
  Clairement, ce processus se termine et l'intersection obtenue est \( S_X \).

  Plus précisément, on note \( Y_j = \bigcap_{i=1}^{j+1} X - x_i \) le fermé de
  \bsc{Zarisiki} obtenu après \( j \) intersections.  Pour nous aider à
  compter les composantes à chaque étape, on considère la quantité :
  \begin{equation}
    SH(Y, d)
    =
    \sum_Z \deg Z d^{\dim Z}
  \end{equation}
  où la somme est prise sur l'ensemble des composantes de \( Y \). Cette
  définition coïncide avec celle donnée p.~364 de \cite{philz} (voir aussi (*)
  p.~362) ; on est ici dans le cas homogène c'est-à-dire \( p = 1 \) dans les
  notations de cette référence.

  Il est clair que \( SH(Y_1, D) = SH(X, D) = D^{u+1} \). Par ailleurs, comme
  chaque \( Y_j \) peut être défini (ensemblistement) par des équations de
  degré au plus \( D \), on peut appliquer à ces équations la proposition~3.3
  de la référence citée, ce qui garantit que \( SH(Y_{j+1}, D) \le SH(Y_j, D)
  \) et donne immédiatement par récurrence \( SH(Y_j, D) \le D^{u+1} \).

  Par définition de \( SH \), on voit donc que pour tout \( k \), le nombre de
  composantes de \( Y_j \) de dimension \( u - k \) est au plus \( D^{k+1} \).
  On peut être un peu plus précis pour \( Y_1 \) et voir que toutes ses
  composantes sont de dimension au plus \( u - 1 \) car \( X - x_1 \) n'a
  qu'une composante. Ainsi, en choisissant successivement les \( x_i \) de
  façon à ce que \( X - x_{j+2} \) coupe strictement une des composantes de \(
    Y_j \) de dimension maximale parmi celles qui ne sont pas composantes de \(
    S_X \), on voit qu'il existe un \( j_1 \le 1 + D^2 \) tel que \( Y_j \) n'ai
  plus de composante de dimension \( u - 1 \) à part celles de \( S_X \), et
  en continuant jusqu'à avoir éliminé les composantes indésirables de
  dimension \( 0 \), qu'il existe un \( j_u \le 1 + D^2 + D^3 + \dots +
    D^{u+1} \) tel que \( Y_{j_u} \) n'ait plus aucune composante autre que
  celles de \( S_X \), c'est-à-dire tel que \( Y_{j_u} = S_X \).

  Le nombre de points utilisés est alors
  \begin{equation}
    j_u + 1
    \le
    2 + D^2 \sum_{l=0}^{u-1} D^l
    =
    2 + \frac{D}{D+1} (D^{u+1} - D)
    \le
    2 + 2 (D^{u+1} - D)
    \le M
    \pmm,
  \end{equation}
  en utilisant le fait que \( D \ge 2 \).

  Par ailleurs, on constate qu'une fois qu'on a choisi \( x_1, \dots,
    x_{j_u+1} \) tels que \( Y_{j_u} = S_X \), on peut rajouter des \( x_i \)
  arbitraires pour \( i > j_u + 1 \) et on aura toujours \( S_X = \bigcap_i
    X - x_i \). Ainsi, pour tout \( m \ge M \) il existe une famille \(
    (x_i)_i \in X^m \) possédant la propriété voulue.
\end{proof}

On note désormais \( M \) l'entier introduit par le lemme précédent et on
suppose \( m \ge M \) jusqu'à la fin de cette section. Sous cette hypothèse,
on peut calculer la dimension de \( Z = s_m(X^m) \) puis estimer son degré et
sa hauteur.

\begin{coro} \label{c:gen-fiber}
  Il existe un ouvert dense de \( Z \) où les fibres de
  \( {s_m}_{|X} \) sont de dimension \( \dim S_X \). En particulier,
  \( \dim s_m(X^m) = m \dim X - \dim S_X \)
\end{coro}

\begin{proof}
  On choisit un point \( (x_1, \dots, x_m) \in X^m \) tel que
  \( S_X = \bigcap_{i=1}^m X - x_i \), ce qui est possible d'après le choix de
  \( m \). La fibre de \( {s_m}_{|X} \) en \( s_m(x) \) est alors isomorphe à
  \( S_X \) et en particulier, de même dimension.

  Par ailleurs, toutes les fibres sont de dimension au moins \( \dim S_X \).
  Par semi-continuité supérieure de la dimension des fibres, l'ensemble des
  points où elle est égale à \( \dim S_X \) est donc un ouvert. D'après le
  paragraphe précédent, cet ouvert n'est pas vide, il est donc dense.
\end{proof}

On rappelle que dans l'énoncé suivant, le degré est relatif au plongement
\( Z \embedin \proj N \) évoqué précédemment.

\begin{lem}
  On a \( \deg Z \le D^m (2m)^{ (m+1)u } \).
\end{lem}

\begin{proof}
  On note \( u' = \dim Z = mu - \dim S_X \) et on choisit des hyperplans de \(
    \proj N \) qu'on notera \( H_1, \dots, H_{u'} \) et qu'on supposera en
  position générale, de sorte que \( F = \bigcap_i H_i \) est un ensemble fini
  de points, de cardinal \( \deg Z \). On peut même supposer que \( F \) est
  inclus dans l'ouvert donné par le corollaire~\vref{c:gen-fiber} car ce
  dernier est dense.

  On note donc \( F = \set{y_1, \dots, y_p} \) ; pour chaque \( y_j \) on
  choisit un point \( x_j \in s_m^{-1}(y_j) \cap X^m \). Pour tout \( j \in
    \set{1, \dots, p} \) et tout \( k \in \set{1, \dots, m} \), le morphisme
  \( \pi_k s_m \), où \( \pi_k \) est la projection sur le \( k \)-ième
  facteur, qui est en fait la différence de \( \va \), est représenté sur un
  voisinage de \( x_j \) par une famille de \( n + 1 \) formes multihomogènes
  de multidegré \( (0, \dots, 0, 2, 0, \dots, 0, 2) \) où le premier \( 2 \)
  est en \( k \)-ième position. On en déduit aisément que \( s_m \) est
  représenté sur un voisinage de \( x_j \) par une famille de \( N + 1 \)
  formes multihomogènes de multidegré \( (2, \dots, 2, 2(m-1)) \). On a \lat{a
    priori} une telle famille par point \( x_j \), mais quitte à prendre une
  combinaison linéaire de toutes ces familles, on voit qu'il existe une
  famille de formes de multidegré \( (2, \dots, 2, 2(m-1)) \) représentant \(
    s_m \) sur un ouvert contenant tous les \( x_j \).

  Maintenant, pour chaque \( i \in \set{1, \dots, u'} \), on choisit une
  équation \( E_i \) de \( H_i \). Dans cette forme linéaire, on substitue aux
  variables les éléments de la famille du paragraphe précédent et on note \(
    \tilde E_i \) la forme multihomogène obtenue, qui est de multidegré \( (2,
    \dots, 2, 2(m-1)) \), puis on note \( \tilde H_i \) l'hypersurface de \(
    (\projd)^m \) définie par \( \tilde E_i \).

  Par construction, il est clair que pour tout \( j \in \set{1, \dots, p} \),
  la fibre \( s_m^{-1}(-y_j) \cap X^m \) est un translaté de \( \delta(S_X) \)
  et est une composante isolée de l'intersection \( X^m \cap \bigcap_i \tilde
    H_i \) : en effet, les fibres sont deux à deux d'intersection vide et les
  autres composantes éventuelles de l'intersection sont celles provenant du
  lieu des zéros commun des formes choisies pour représenter \( s_m \), qui ne
  contient aucune des fibres en question grâce au choix de cette famille.

  On invoque alors la proposition~3.3, p.~365 de \cite{philz} dont on reprend
  les notations (voir p.~364 et p.~362) :
  \begin{equation}
    SH( X^m \cap \bigcap_i \tilde H_i \, ; 2, \dots, 2, 2(m-1) )
    \le
    SH( X^m ; 2, \dots, 2, 2(m-1) )
    \pmm.
  \end{equation}
  On remarque que le seul multidegré non nul de \( X^m \) est celui
  d'indice \( (u, \dots, u) \) et qu'il vaut \( D^m \),  de sorte que l'on a
  \begin{equation}
    SH( X^m ; 2, \dots, 2, 2(m-1) )
    =
    D^m
    \frac{ (mu)! }{ (u!)^m }
    \, 2^{mu} (m-1)^u
    \le
    D^m
    (2m)^{mu} (m-1)^{u-1}
  \end{equation}
  en majorant le coefficient multinomial qui apparaît dans cette expression
  par \( m^{mu-1} \). Par ailleurs, on a
  \begin{align}
    SH( X^m \cap \bigcap_i \tilde H_i \, ; 2, \dots, 2, 2(m-1) )
    \ge
    p \cdot H( \delta(S_X); 2, \dots, 2, 2(m-1) )
  \end{align}
  d'après le paragraphe précédent. De plus, on voit assez facilement que tous
  les multidegrés de \( \delta(S_X) \) sont égaux à \( \deg S_X \), ce qui
  donne
  \begin{align}
    H( \delta(S_X); 2, \dots, 2, 2(m-1) )
    & =
    \ \sum_{ \mathclap{\alpha_1 + \dots + \alpha_m = \dim S_X} } \
    \deg S_X
    \, \frac{ (\dim S_X)! }{ \alpha_1! \cdots \alpha_m! }
    \, 2^{m\dim S_X} (m-1)^{\dim S_X}
    \\ & =
    \deg S_X
    (2m)^{m\dim S_X} (m-1)^{\dim S_X}
  \end{align}
  d'après la formule multinomiale. Ainsi, on a
  \begin{equation} \label{e:deg-img-good}
    \deg Z = p
    \le
    \frac{ D^m }{ \deg S_X }
    (2m)^{m(u-\dim S_X)} (m-1)^{u - \dim S_X - 1}
  \end{equation}
  qui est un peu plus précis que la majoration annoncée.
\end{proof}

Dans le lemme suivant, on rappelle que \( \hautl[2]{ s_m(x) } \) est la
hauteur donnée par le plongement dans \( \proj N \).

\begin{lem}
  Pour tout \( x \in X^m \), on a
  \begin{equation}
    \hautl[2]{ s_m(x) }
    \le
    4 (m-1) \max_{1 \le i \le m} \bigl( \hautl[2]{ x_i } \bigr)
    + (m-1) \hlclab
    \pmm.
  \end{equation}
\end{lem}

\begin{proof}
  On reprend la construction faite dans la démonstration du lemme précédent
  pour exhiber une famille de formes représentant \( s_m \) au voisinage de \(
    x \), en détaillant la construction pour bien contrôler la hauteur de cette
  famille.

  La section~\vref{sec:vaemb} fournit, pour chaque \( i \in \set{1, \dots,
      m-1} \) une famille de formes bihomogènes de bidegré \( (2, 2) \)
  représentant la soustraction de \( \va^2 \) dans \( \va \) au voisinage de
  \( (x_i, x_m) \), que l'on notera \( (L\pexp i[k])_{k \in \set{0,
        \dots, n}} \), telle que \( \nnv1{ L\pexp i } \le \hmclab* \).
  Pour chaque \( l \in \set{0, \dots, n}^{m-1} \) on pose
  \begin{equation}
    S_l(\vmp[1], \dots, \vmp[m-1])
    =
    \prod_{i=1}^m L\pexp i[l_i]( \vmp[i], \vmp[m] )
  \end{equation}
  de sorte que la famille \( (S_l)_l \) représente \( s_m \) au voisinnage de
  \( x \). On a alors \( \nnv1{ S } \le \hmclab* ^{m-1} \) par les propriétés
  de la norme \( L_1 \). Par ailleurs, les propriétés de la norme euclidienne
  donnent \( \nv2{ S(x) } \le \nnv2{ S } \nv2{ x_1 }^2 \dots \nv2{ x_{m-1} }^2
    \nv2{ x_m }^{2(m-1)} \) compte tenu du multidegré des formes \( S_l \).

  Le résultat annoncé vient en prenant le logarithme puis la somme sur toutes
  les places.
\end{proof}

\begin{coro}
  On a \(
    \hautl[\htpph] Z
    \le
    \bigl( 8 D^{m-1}\hautl[\htpph] X + D^m \hlclab)
    (2m)^{ (m+2)u }
  \).
\end{coro}

\begin{proof}
  Il suffit de savoir majorer le minimum essentiel de \( Z \) en fonction de
  celui de \( X \), en vertu des comparaison suivantes, données par le
  théorème 3.1 de \cite{daphimhva1} :
  \worknote{Recopié depuis \cite{daphimht}, à vérifier directement dans la
    réf. citée quand même.}
  \begin{equation}
    \miness(V) \deg V
    \le
    \hautl[\htpph] V
    \le
    \miness(V) \deg V (\dim V + 1)
  \end{equation}
  où \( \miness \) désigne le minimum essentiel, défini en utilisant la
  hauteur \( \Hautl[2] \) pour les points. (On rappelle que \( \Hautl[\htpph]
  \) est la hauteur obtenue en utilisant la mesure de \bsc{Philippon} aux
  places archimédiennes.)

  Notons \( F \) l'ensemble des points de \( x \in X \) tels que \( \hautl[2]
    x \le 2 \miness(X) \) ; par définition \( F \) est dense dans \( X \),
  donc \( F^m \) est dense dans \( X^m \) et \( s_m(F^m) \) est dense dans \(
    Z \). On applique alors le lemme précédent pour majorer la hauteur des
  points de \( s_m(F^m) \) et obtenir
  \begin{equation}
    \miness(Z)
    \le
    8 (m-1) \miness(X) + (m-1) \hlclab
    \pmm.
  \end{equation}

  Il est alors aisé de passer aux hauteurs grâce à l'encadrement rappelé
  ci-dessus, en utilisant les informations déjà obtenues sous la forme suivant
  : \( \dim Z \le mu \) et \( \deg Z \le D^m (2m)^{ mu } (m-1)^{u-1} \)
  par~\eqref{e:deg-img-good} :
  \begin{align}
    \hautl[\htpph] Z
    & \le
    \miness(Z) \deg Z (\dim Z + 1)
    \\ & \le
    (8 (m-1) \miness(X) + (m-1) \hlclab)
    D^m (2m)^{ mu } (m-1)^{u-1}
    mu
    \\ & \le
    (8  \hautl[\htpph] X / D +  \hlclab)
    D^m (2m)^{ (m+2)u }
  \end{align}
  qui donne bien la majoration annoncée.
\end{proof}



\section{Inégalité de \bsc{Mumford} dans le cas général}

Dans toute cette section, on continue de supposer que \( m \ge M \) et on fixe
une famille \( x = (x_1, \dots, x_m) \in \va(\Qbar)^m \) (et non plus \( X^m
\)) et on note \( z = s_m(x) \). Les trois premiers énoncés sont des
conséquences ou analogues directs de ceux de la
section~\vref{sec:mumford-grp}.

\begin{lem} \label{l:img-small}
  Sous les hypothèses~\eqref{e:Mcos-gen} et~\eqref{e:Mclose-gen} du
  théorème~\eqref{t:mumford-gen}, on a
  \begin{equation}
    \hautl z
    \le
    (\rho^2/4 + 2\phi + \rho\phi) \hautn{ x_m }
    + (m-1) \htcmp
    \pmm.
  \end{equation}
\end{lem}

\begin{proof}
  Pour chaque \( i \in \set{1, \dots, m-1} \), on applique le lemme
  \vref{l:diff-small} avec \( x = x_m \) et \( y = x_i \), ce qui montre que
  \( \hautn{z_i} \le (\rho^2/4 + 2\phi + \rho\phi) \hautn{ x_m } \).
  On écrit ensuite
  \begin{equation}
    \hautl[2]{z}
    =
    \sum_{i=1}^{m-1} \hautl[2]{z_i}
    \le
    \sum_{i=1}^{m-1} (\hautn{z_i} + \htcmp)
  \end{equation}
  pour aboutir au résultat annoncé.
\end{proof}

\begin{lem} \label{l:img-apx}
  Si \( x \) satisfait l'hypothèse~\eqref{e:Mapx-gen} et la première inégalité
  de l'hypothèse~\eqref{e:Mclose-gen} du théorème~\vref{t:mumford-gen}, on a
  \begin{equation}
    \prod_\placerange \distv z Z ^\degv
    \le
    \expb^{-\expapx \hautn{ x_m }}
    \cdot \hmclab \expb^\htcmp
    \cdot 5 \sqrt{2m-2} \, (n + 1)^3
  \end{equation}
\end{lem}

\begin{proof}
  C'est une variante du lemme~\vref{l:diff-apx} ; la démonstration est
  identique sauf pour l'étape finale et quelques comparaisons de hauteurs.
  Pour chaque \( i \in \set{1, \dots, m-1} \) et chaque \( v \in \placesapx
  \), la définition de la distance permet de choisir un point \( y_{i, v} \in
    \varapx(\Cv) \) tel que
  \begin{equation}
    \distv{ x_i }{ y_{i, v} }
    =
    \distv{ x_i }\varapx
    \le
    \exp(-\wtapx \expapx \hautn{ x_m } + \htcmp)
  \end{equation}
  où l'inégalité découle directement des hypothèses et d'une comparaison entre
  hauteurs.
  La proposition~\vref{p:addsub-dv} donne alors
  \begin{equation}
    \max_i \distv{ x_i - x_m }{ y_{i, v} - y_{m, v} }
    \le
    \exp(-\wtapx \expapx \hautn{ x_m } + \htcmp)
    \cdot \hmclab* \bigl( 5 \sqrt2 (n + 1)^3 \bigr)^\dv
    \pmm.
  \end{equation}
  Or, d'après le lemme~4.3, p.~121 de \cite{remgdmp} (interprété à la lumière
  des remarques p.~102 de la même référence concernant le lien entre indices
  d'une distance et plongement de \bsc{Segre}), en notant \( y_v = (y_{1, v},
    \dots, y_{m, v}) \in X^m(\Cv) \), on a
  \begin{equation}
    \distv{ s_m(x) }{ s_m(y_v) }
    \le
    (m-1)^{\dv/2}
    \, \max_i \distv{ x_i - x_m }{ y_{i, v} - y_{m, v} }
    \pmm.
  \end{equation}
  Or par définition le membre de gauche majore \( \distv{ z, Z } \). Il suffit
  alors de prendre le produit sur \( v \) (en supposant que \( S \) contient
  toutes les places archimédiennes que c'est le cas défavorable) pour
  conclure.
\end{proof}

\begin{lem}
  Sous les hypothèses du théorème~\vref{t:mumford-gen} (sauf \( x_i \in \Gamma
  \) et la définition de \( m \)), on a \( z \in Z \).
\end{lem}

\begin{proof}
  On procède comme en page~\pageref{page:demo-mumgrp} pour la preuve du
  théorème~\ref{t:mumford-grp}. Si \( z \not\in Z \), l'inégalité de
  \bsc{Liouville} (proposition~\vref{p:liouville}) appliquée à \( z \) et \( Z
  \) donne, en prenant les logarithmes :
  \begin{align} \label{e:liou}
    \sum\placerange \degv \ln\distv z Z
    & \ge
    - (\deg Z) \hautl[2] z
    - \hautl[1]{ Z }
    \\ & \qquad
    - \frac32 \ln(N + 1)
    - (\dim Z + 1) (\deg Z) \ln(3 \deg Z)
    \pmm.
  \end{align}
  On va maintenant estimer une par une les quantités apparaissant dans le
  membre de droite.

  Dans la démonstration du lemme~3.3, p. 111 de \cite{remgdmp}, il est établit
  que si \( P \) est une forme multihomogène de multidegré \( \delta \) en
  plusieurs groupes de \( n + 1 \) variables, dont on note \( p_\alpha \)
  les coefficients, on a aux places archimédiennes
  \begin{equation}
    \abs{ p_\alpha }
    \le
    \binom \delta \alpha
    \mahler P
    \le
    \binom \delta \alpha
    \mespph P
  \end{equation}
  ce qui en sommant sur les multimultiindices \( \alpha \) tels que \(
    \vlg\alpha = \delta \), donne immédiatement, par la formule multinomiale :
  \begin{equation}
    \normlun P
    \le
    (n+1)^{\lgr\delta}
    \mespph P
    \pmm.
  \end{equation}
  En appliquant ceci à la forme de \bsc{Chow} de \( Z \), qui est
  multihomogène de degré \( \deg Z \) en \( \dim Z + 1 \) groupes de \( N + 1
  \) variables, puis en utilisant les estimations de la section précédente, il
  vient :
  \begin{align} \label{e:liou-h1-img}
    \hautl[1] Z
    & \le
    \hautl[\htpph] Z
    + (\deg Z) (\dim Z + 1) \ln(N+1)
    \\ & \le
    \bigl( 8 D^{m-1}\hautl[\htpph] X + D^m \hlclab)
    (2m)^{ (m+2)u }
    \\ & \qquad
    + D^m (2m)^{mu} (m-1)^{u-1} (um + 1) (m-1)\ln(n+1)
    \\ & \le
    (2m)^{ (m+2)u } \Bigl(
    D^{m-1} \hautl[\htpph] X
    + D^m \bigl( \hlclab + \ln(n+1) \bigr)
    \Bigr)
    \pmm.
  \end{align}

  Par ailleurs, ces mêmes estimations donnent :
  \begin{align}
    (\dim Z + 1) \ln(3 \deg Z)
    & \le
    (um + 1) \bigl( (m+1) u \ln(2m) + \ln 3 \bigr)
    \\ & \le
    u^2 (m+1) \bigl( (m+1) \ln(2m) + \ln 3 \bigr)
    \\ & \le
    2^{u+2} m^3
  \end{align}
  où, la dernière ligne utilise le fait que \( u^2 \le 2^{u+1} \) et,
  pour majorer le facteur en \( m \), une vérification numérique pour les
  petites valeurs. Au final on a
  \begin{equation} \label{e:liou-cst2}
    (\dim Z + 1) (\deg Z) \ln(3 \deg Z)
    \le
    2^{mu} m^{(m+1)u - 1}
    \cdot 2^{u+2} m^3
    \le
    D^m (2m)^{(m+1)u + 2}
    \pmm.
  \end{equation}

  Enfin, le lemme~\vref{l:img-small} et l'hypothèse~\eqref{e:rho-phi-gen}
  donnent
  \begin{align}
    (\deg Z) \hautl[2] z
    & \le
    \frac\expapx2 \hautn{ x_m }
    + D^m (2m)^{(m+1)u} \htcmp
    \pmm.
  \end{align}
  On peut maintenant substituer cette estimation ainsi
  que~\eqref{e:liou-h1-img} et~\eqref{e:liou-cst2} dans l'inégalité de
  \bsc{Liouville} écrite en~\eqref{e:liou} :
  \begin{align}
    \sum\placerange \degv \ln\distv z Z
    & \ge
    - \frac\expapx2 \hautn{ x_m }
    - D^m (2m)^{(m+1)u} \htcmp
    \\ & \qquad
    - (2m)^{ (m+2)u } \Bigl(
    D^{m-1} \hautl[\htpph] X
    + D^m \bigl( \hlclab + \ln(n+1) \bigr)
    \Bigr)
    \\ & \qquad
    - 2(m-1) \ln(n+1)
    - D^m (2m)^{(m+1)u + 2}
    \\ & \ge
    - \frac\expapx2 \hautn{ x_m }
    - D^{m-1} (2m)^{ (m+2)u } \, \hautl[\htpph] X
    \\ & \qquad
    - D^m (2m)^{ (m+2)u + 1}
    \bigl( \hlclab + \htcmp + \ln(n+1) \bigr)
    \pmm.
  \end{align}
















\end{proof}

\cleardoublepage
\endinput

% vim: spell spelllang=fr
