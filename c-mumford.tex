% !TEX root = main.tex

\chapter{Inégalité de \bsc{Mumford}} \label{chap:mumford}

\section{Énoncé principal}

On prouve ci-dessous une inégalité de \bsc{Mumford} explicite dans le cas
particulier où \( \varapx \) est une sous-variété abélienne de \( \va \),
qu'on notera donc désormais \( \vai \). Le résultat principal de ce chapitre
est le suivant.

\begin{thm} \label{t:mumford}
  Soient \( \vai \) une sous-variété abélienne de \( \va \) et \( \varapx \)
  un translaté de \( \vai \) par un point algébrique ;
  on note \( d = \deg \vai \) et \( l = \dim\vai + 1 \).
  On choisit \( \expapx > 0 \) puis \( \phi > 0 \) et \( \rho > 0 \) tels que
  \begin{equation} \label{e:rho-phi}
    \frac{ \rho^2 }{ 4 } + \rho\phi + 2\phi
    \le
    \frac{ \expapx }{ 2d }
    \pmm.
  \end{equation}
  Si \( x \) et \( y \) sont deux points de \( \va(\Qbar) \) tels que
  \begin{align}
    0 < \distv z \varapx
    & \le
    \hautm[2] z ^{-\wtapx\expapx}
    \quad \forall \place \in \placesapx
    \quad \text{où \( z \) est \( x \) ou \( y \)}
    \label{e:Mapx}
    \\
    \hautn x
    & \ge
    \frac2\expapx \bigl(
      \hautl[1]{ \chow\vai }
      + ld \ln(3d)
      + (d + 1) \htcmp
      + \hlclab
      + 6 \ln(n + 1)
    \bigr)
    \label{e:Mbig}
    \\
    \cos(x, y)
    & \ge
    1 - \phi
    \label{e:Mcos}
    \\
    \hautn x
    & \le
    \hautn y \le (1+\rho) \hautn x
    \label{e:Mclose}
  \end{align}
  alors \( x - y \in \vai(\Qbar) \).
\end{thm}

La preuve consiste à montrer que la différence \( x - y \) est encore très
proche de \( \vai \) tout en étant de hauteur relativement petite, de sorte
qu'une inégalité de \bsc{Liouville} la contraint à être sur \( \vai \). On
commence donc par établir une inégalité de \bsc{Liouville}, puis on étudie
l'action des opérations sur la distance, avant d'achever la preuve du
théorème.



\section{Inégalité de \bsc{Liouville}} \label{sec:liouville}

On établit ici l'analogue suivant de la classique inégalité de
\bsc{Liouville}.

\begin{prop} \label{p:liouville}
  Soit \( V \) une sous-variété de \( \projd \), de dimension \( u \) et de
  degré \( d \). Pour tout point \( x \in \projd(\Qbar) \), on a soit \( x \in
  V(\Qbar) \) soit
  \begin{equation}
    \prod_{v \in \placesapx} \distv x V ^\degv
    \ge
    \frac1{
      (n+1)^{3/2}
      (3d)^{d (u+1)}
      \, \hautm[1]{ \chow V }
      \, \hautm[2] x ^d
    }
    \pmm.
  \end{equation}
\end{prop}

Ce type d'inégalité se prouve de façon naturelle avec la distance algébrique
(section~\vref{sec:distv-cmp}). Cependant, si l'on traite directement le cas
général en utilisant la définition, on obtient \( d (u+1) \) comme exposant de
\( \hautm[2] x \). On commence donc par le cas d'une hypersurface, où la
distance algébrique admet une expression plus simple, puis on remarque que
tout variété est une intersection d'hypersurfaces.

On pourrait se ramener au cas d'un hyperplan, où la distance admet une
expression simple, par un plongement de \bsc{Veronese} (remodelé) comme dans
le lemme~\vref{l:hs-vero}, mais on obtient en fait de meilleurs constantes en
utilisant la propriété~\vref{p:dv-p2alg-hs}.

\begin{lem} \label{l:liou-hs}
  Soit \( V \) une hypersurface de \( \projd \), d'équation \( F \) et de
  degré \( d \). Pour tout point \( x \in \projd(\Qbar) \), on a soit \( x \in
  V(\Qbar) \) soit
  \begin{equation}
    \prod_{v \in \placesapx} \distv x V ^\degv
    \ge
    \frac1{
      (17/8)^d (n+1)^{3/2}
      \, \hautm[2] G
      \, \hautm[2] x ^d
    }
    \pmm.
  \end{equation}
\end{lem}

\begin{proof}
  Plutôt que la proposition~\vref{p:dv-p2alg-hs}, on utilise en fait
  l'inégalité~\eqref{e:dv-p2alg-hs} établie au cours de sa preuve afin
  d'éviter une comparaison norme-mesure superflue. On remarque également qu'on
  peut supposer \( m = 1 \) dans cette comparaison. On a alors :
  \begin{align}
    \prod_{v \in \placesapx}
    \distv x V ^\degv
    & \ge
    \prod_{v \in M(\cdn)}
    \distv x V ^\degv
    \\ & \ge
    \prod_{v \in M(\cdn)}
    \left(
      \left(
        \frac{27}{26} (n+1)^{3/2}
        \left( \frac{164}{81} \right)^{d}
      \right)^{-\dv}
      \frac{ \av{G(x)} }{ \nv2 x^{d} \nv2 G }
    \right)^\degv
  \end{align}
  qui implique le résultat annoncé, \lat{via} une simple comparaison
  numérique pour la constante.
\end{proof}

\begin{proof}[\proofname{} de la proposition~\vref{p:liouville}]
  Si \( x \not\in V(\Qbar) \), la proposition~6.1 de~\cite{remdcl}
  donne une forme \( G \) homogène de degré \( d \), contenant \( V \) mais
  pas \( x \), telle que
  \begin{equation}
    \hautm[1] G
    \le
    \hautm[1] V \cdot (d + 1)^{ d (u + 1) }
  \end{equation}
  en tenant compte des différences de hauteurs utilisées, comme observé dans
  la preuve du lemme~\vref{l:hs-choice}. Il suffit alors d'appliquer le
  lemme précédent à \( \zeros G \) et de remarquer que \( 17/8 \cdot (d+1)^2
    \le (3d)^2 \) pour conclure.
\end{proof}



\section{Comportement métrique des opérations}

La situation d'approximation considérée met en jeu plusieurs types de
géométrie sur \( \va \) : la géométrie euclidienne de son espace de
\bsc{Mordell-Weil}, la géométrie algébrique et projective de la variété
plongée, une géométrie métrique, locale, en chaque place de \( \placesapx \). On
s'attend à ce que les métriques locales sur \( \va \) se comportent de façon
agréable vis-à-vis des structures géométrique et arithmétique, par exemple que
les opérations soient uniformément continues, voire lipschitziennes. On montre
ici que c'est bien le cas et on explicite une valeur admissible pour la
constante de \bsc{Lipschitz}.

\begin{prop} \label{p:addsub-dv}
  Pour tous \( x \), \( x' \), \( y \), \( y' \) dans \( \va(\C_v) \), on a :
  \begin{equation}
  \Distv(x \pm y, x' \pm y')
  \le
  \max \bigl( \Distv(x, x'), \Distv(y, y') \bigr)
  \cdot \hmclab* \bigl( 5 \sqrt2 (n + 1)^3 \bigr)^\dv
  \pmm,
  \end{equation}
  où dans le membre de gauche il faut prendre le même signe des deux côtés.
\end{prop}

\begin{proof}
  On considère le plongement de \bsc{Segre} \( s \colon (\projd)^2 \to
    \proj{n'} \) avec \( n' = n^2 + 2n \). D'après le lemme~4.3 (p.~121) de
  \cite{remgdmp}, appliqué avec \( q = 2 \) et \( d = (1,1) \), ainsi que le
  paragraphe~2.3 (p.~103 notamment) de cette référence donnant le lien entre
  indices (d'une forme de \bsc{Chow} ou, par extension, d'une distance) et
  plongements de \bsc{Segre} (et de \bsc{Veronese} remodelé), on a
  \begin{equation} \label{e:dv-segre}
    1
    \le
    \frac{
      \Distv(s(x, x'), s(y, y'))
    }{
      \max \bigl( \Distv(x, x'), \Distv(y, y') \bigr)
    }
    \le
    2^{\dv/2}
  \end{equation}
  dans le cas où le dénominateur est différent de zéro (dans le cas contraire,
  on a \( x = x' \) et \( y = y' \) et le résultat est immédiat).

  On considère alors le morphisme
  \( \xi : \va^2 \to \va^2,\ (x, y) \mapsto (x + y, x - y) \), qui
  présente l'avantage de pouvoir être représenté globalement
  par une famille \( F \) de formes de degré \( 2 \) dans le plongement de
  \bsc{Segre} (voir section~\vref{sec:vaemb}). Si l'on note \( z = s(x, y) \)
  et \( z' = s(x', y') \) les images dans ce plongement, on est ainsi ramené à
  montrer que
  \begin{equation} \label{e:addsub-dv}
    \Distv(\xi z, \xi z')
    \le
    \Distv(z, z')
    \cdot \hmclab* \bigl( 5 (n + 1)^3 \bigr)^\dv
  \end{equation}
  en appliquant deux fois~\eqref{e:dv-segre}. On peut donc supposer
  \( \Distv( z, z') \le \bigl( 5 (n + 1)^3 \bigr)^{-\dv} \), car sinon
  l'inégalité précédente est banale.

  On va utiliser un développement de \bsc{Taylor} au voisinage de \( z \) des
  formes représentant \( \xi \) pour montrer que \( \nv2{F(z')} \) n'est pas
  trop petit devant \( \nv2 {F(z)} \) puis que \( \nv2{F(z) \wedge F(z')} \)
  n'est pas trop grand devant \( \nv2 {F(z)}^2 \). L'hypothèse faite sur la
  distance permet d'appliquer le lemme~\vref{l:dv-common-i} à \( z \) et \( z'
  \) (en remplaçant \( n \) par \( n' = n^2 + 2n \)) et donc de supposer,
  quitte à renuméroter les coordonnées, que \( z_0 = z'_0 = 1 \) et \( \nv2{
      z' } \le (n')^{\dv/2} \).  Le développement de
  \bsc{Taylor} de \( F \) s'écrit alors :
  \begin{equation}
    F(z')
    =
     F(z)
    + \sum_{1 \le i \le n'}
    \underbrace{
      \frac{\partial F}{\partial Z_i}(z)
      \cdot ( z'_i - z_i )
    }_{\textstyle R_i}
    + \sum_{1 \le i, j \le n'}
    \underbrace{
      \frac12 \frac{ \partial^2 F }{ \partial Z_i \partial Z_j }(z)
      \cdot ( z'_i - z_i ) ( z'_j - z_j )
    }_{\textstyle R_{ij}}
    \pmm.
  \end{equation}
  On remarque alors que
  \(
    \av{z'_i - z_i}
    \le
    \nv2{z \wedge z'}
    \le
    \nv2 z \nv2{z'}
    \distv{ z }{ z' }
  \)
  pour estimer \( R_i \) en utilisant de plus~\eqref{e:addsub-loc} :
  \begin{align}
    \nv2{R_i}
    & \le
    2^\dv \nnv2 F \nv2 z
    \cdot \nv2 z \nv2{ z' }
    \distv{ z }{ z' }
    \\ & \le
    \hmclab* \nv2{ F(z) }
    (2 \sqrt{n'})^\dv
    \distv{ z }{ z' }
    \pmm.
  \end{align}
  On obtient de même
  \(
    \nv2{R_{ij}}
    \le
    \hmclab* \nv2{ F(z) }
    (n')^\dv
    \distv{ z }{ z' }^2
    \pmm.
  \)
  En prenant la somme, on obtient un facteur \( (n')^\dv \) supplémentaire
  pour les \( R_i \), et son carré pour les \( R_{ij} \). Au final on a :
  \begin{align}
    \nv*2{\sum R_i + \sum R_{ij}}
     & \le
    \nv2{F(z)}
    \cdot \hmclab* \left( \frac52 \cdot (n')^{3/2} \right)^\dv
    \distv{ z }{ z' }
     <
    \left( \frac12 \right)^\dv
    \nv2{F(z)}
  \end{align}
  en remarquant aux places archimédiennes que \( 2t + t^2 \le 5t/2 \) pour \(
    0 \le t \le 1/2 \), ce qui est le cas de \( t = (n')^{3/2}
    \distv{ z }{ z' } \) vu l'hypothèse sur la distance et le fait que \( n'
    < (n+1)^2 \).

  Les inégalités triangulaire aux places infinies et ultramétrique aux places
  finies donnent alors \( \nv2{F(z')} \ge (1/2)^\dv \nv2{F(z)} \).
  Afin d'estimer la distance, il reste à majorer \( \nv2{F(z) \wedge F(z')}
  \). Pour ce faire, on développe le second facteur, on remarque que le
  premier terme du produit est nul et on majore brutalement le terme restant
  par le produit des normes ; il vient ainsi :
  \begin{align}
    \Distv(\xi(z), \xi(z'))
    & =
    \frac{ \nv2{F(z) \wedge F(z')} }{ \nv2{F(z)} \nv2{F(z')} }
    \\ & \le
    \frac{
      \nv2{F(z)} \nv2{\sum R_i + \sum R_{ij}}
    }{
      \nv2{F(z)} \nv2{F(z)} \cdot 2^{-\dv}
    }
    \\ & \le
    \hmclab* \bigl( 5 (n')^{3/2} \bigr)^\dv
    \distv{ z }{ z' }
  \end{align}
  qui donne bien~\eqref{e:addsub-dv} et achève la preuve.
\end{proof}



\section{Inégalité de \bsc{Mumford}}

La stratégie de la preuve est la suivante : si on a deux approximations
exceptionnelles, suffisamment proches dans l'espace de \bsc{Mordell-Weil} (au
sens des hypothèses \eqref{e:Mcos} et \eqref{e:Mclose} du
théorème~\vref{t:mumford}) et de hauteur assez grande, on fabrique, (en prenant
leur différence, sous l'hypothèse que \( \vai \) est un sous-groupe) une
approximation de qualité telle qu'elle appartiendra nécessairement à \( \vai
\), d'après l'inégalité de \bsc{Liouville}.

\begin{lem} \label{l:diff-apx}
  Soient \( x \) et \( y \) dans \( \proj(\Qbar) \), on pose \( z = x -
    y \) et \( H = \min\bigl( H(x), H(y) \bigr) \).
  Si \( x \) et \( y \) satisfont à~\eqref{e:Mapx}, on a
  \begin{equation}
    \prod\placerange
    \Distv(z, \vai)^\degv
    \le
    H^{-\expapx}
    \cdot 5 \sqrt2 (n + 1)^3 \hmclab
    \pmm.
  \end{equation}
\end{lem}

\begin{proof}
  En chaque place \( v \), il existe des points \( x'_v \) et \( y'_v \) dans
  \( \varapx(\Cv) \) tels que \( \distv x \varapx = \distv{ x }{ x'_v } \) et
  \( \distv y \varapx = \distv{ y }{ y'_v } \). L'hypothèse~\eqref{e:Mapx}
  donne alors
  \begin{equation}
    \max \bigl( \Distv(x, x'_v), \Distv(y, y'_v) \bigr)
    \le
    H^{-\wtapx \expapx}
    \pmm.
  \end{equation}
  Comme \( \varapx \) est translaté d'un sous-groupe \( \vai \), on a
  \( x'_v - y'_v \in \vai(\Cv) \), donc la proposition~\vref{p:addsub-dv} donne
  \begin{equation}
    \distv z \vai
    \le
    \distv z {x'_v - y'_v}
    \le
    H^{-\wtapx \expapx}
    \cdot \hmclab* \bigl( 5 \sqrt2 (n + 1)^3 \bigr)^\dv
    \pmm.
  \end{equation}
  On prend alors le produit sur \( v \in \placesapx \), en supposant (c'est le
  cas défavorable) que \( \placesapx \) contient toutes les places
  archimédiennes et toutes celles où \( \hmclab* > 1 \).  La normalisation \(
    \sum\placerange \wtapx \degv = 1 \) permet alors de conclure.
\end{proof}

\begin{lem} \label{l:diff-small}
  Soient \( x \) et \( y \) satisfaisant aux hypothèses \eqref{e:Mcos}
  et \eqref{e:Mclose} du théorème~\vref{t:mumford}, notons \( z \) leur
  différence. On a alors \( \hautn z \le (\rho^2/4 + 2\phi + \rho\phi)
    \hautn x \).
\end{lem}

\begin{proof}
  On note \( \scalnt \truc\truc \) le produit scalaire et \(
    \nnt\truc = \sqrt{\hautn\truc} \) la norme dans l'espace de
  \bsc{Mordell-Weil}. On remarque de plus que l'hypothèse \eqref{e:Mclose}
  implique
  \( \nnt x \le \nnt y \le (1+\rho)^{1/2}\nnt x \le (1+\rho/2)
    \nnt x \). Il vient alors :
  \begin{align}
    \Hautn(z)
    & =
    \nnt{x-y}^2
    \\ & =
    \nnt x ^2 + \nnt y ^2 - 2 \scalnt x y
    \\ & =
    \left( \nnt y  - \nnt x  \right)^2
    + 2 \nnt x  \nnt y  \left( 1- \cos(x, y) \right)
    \\ & \le
    \left( \frac{\rho}{2}\nnt x  \right)^2
    + 2\left( 1+\frac{\rho}{2} \right)
    \nnt x ^2 \cdot \phi
    \\ & \le
    \bigl( (\rho^2/4) + 2\phi + \rho\phi \bigr)
    \Hautn(x)
    \pmm.
    \qedhere
  \end{align}
\end{proof}

\begin{proof}[Démonstration (du théorème~\vref{t:mumford}).]
  Compte tenu de l'inégalité de gauche de~\eqref{e:Mclose} et de la
  comparaison entre hauteurs projective et normalisée, le
  lemme~\vref{l:diff-apx} donne, en prenant le logarithme :
  \begin{equation} \label{e:close-log}
    \sum\placerange
    \degv \ln\Distv(z, \vai)
    \le
    - \expapx \hautn x
    + \expapx \htcmp
    + \ln(5\sqrt2) + 3 \ln(n + 1) + \hlclab
    \pmm.
  \end{equation}
  Par ailleurs, le lemme~\vref{l:diff-small} et l'hypothèse~\eqref{e:rho-phi}
  donnent
  \begin{equation} \label{e:h2-diff}
    \hautl[2] z
    \le
    \hautn z + \htcmp
    \le
    (\rho^2/4 + 2\phi + \rho\phi)
    \hautn x
    + \htcmp
    \le
    \frac{ \expapx }{ 2d }
    \hautn x
    + \htcmp
    \pmm.
  \end{equation}

  Supposons maintenant que \( z \not\in \vai(\Qbar) \) et montrons qu'on
  contredit~\eqref{e:Mbig}. En effet, l'inégalité de \bsc{Liouville}
  (proposition~\vref{p:liouville}) appliquée à \( z \) et \( \vai \) donne
  \begin{align}
    \sum\placerange
    \degv \ln\Distv(z, \vai)
    & \ge
    - d \hautl[2] z
    - \hautl[1]{ \chow\vai }
    - \frac32 \ln(n + 1)
    - ld \ln(3d)
    \\ & \ge
    - \frac{ \expapx }{ 2 } \hautn x
    - d \htcmp
    - \hautl[1]{ \chow\vai }
    - \frac32 \ln(n + 1)
    - ld \ln(3d)
  \end{align}
  où la deuxième ligne vient en substituant~\eqref{e:h2-diff}.
  En comparant avec~\eqref{e:close-log}, il vient :
  \begin{equation}
    \frac\expapx2 \hautn x
    \le
    \hautl[1]{ \chow\vai }
    + ld \ln(3d)
    + (d + \expapx) \htcmp
    + \hlclab
    + \frac92 \ln(n + 1)
    + \ln(5\sqrt2)
  \end{equation}
  qui contredit bien~\eqref{e:Mbig}, achevant ainsi la preuve.
\end{proof}


\clearpage % XXX provisoire

\section{Étude du morphisme des différences}

Soit \( X \) une sous-variété de \( \va \) ; on note \( D = \deg X \) et \( u
  = \dim X \). Remarquons de suite que le degré de \( X \) est nécessairement
au moins \( 2 \), car si \( X \) était linéaire, elle contiendrait une droite
projective, c'est-à-dire une courbe de genre \( 0 \), or il est impossible de
plonger une telle courbe dans une variété abélienne.

Pour tout entier \( m \ge 2 \) on définit un morphisme
\begin{align}
  s_m \colon \va^m & \to \va^{m-1} \\
  (x_1, \dots, x_m) & \mapsto (x_1 - x_m, \dots, x_{m-1} - x_m)
\end{align}
et on se propose d'étudier (dimension, degré, hauteur) l'image \( s_m(X^m) \),
qu'on notera \( Z \). Le degré et la hauteur seront relatifs au
plongement projectif suivant :
\begin{equation}
  s_m(X^m) \subset \va^{m-1} \embedin (\projd)^{m-1} \to \proj N
\end{equation}
où la deuxième flèche est un plongement de \bsc{Segre}, donc \( N =
  (n+1)^{m-1} - 1 \).

On commence par étudier la forme des fibres du morphisme restreint à \( X \).
Dans l'énoncé suivant, \( S_X \) désigne le stabilisateur de \( X \).

\begin{lem}
  Pour tous \( m \ge 2 \) et \( x \in X^m \), on a
  \begin{equation}
    x + \delta(S_X)
    \subset
    s_m^{-1} ( s_m(x) ) \cap X^m
    \subset
    x + \delta(\bigcap_{i=1}^m X - x_i)
  \end{equation}
  où \( \delta \colon X \to X^m \) est le plongement diagonal. En particulier,
  si \( S_X = \bigcap_{i=1}^m X - x_i \), ces inclusions sont des égalités.
\end{lem}

\begin{proof}
  Par définition de \( s_m \), il est clair que \( x + \delta(A) \subset
    s_m^{-1} ( s_m(x) ) \). Par ailleurs, la définition de \( S_X \) donne
  clairement \( x + \delta(S_X) \subset X \), ce qui établit l'inclusion
  annoncé.

  Dans l'autre sens, pour tout \( y \in s_m^{-1} ( s_m(x) ) \), on a
  \( y_m - x_m = y_i - x_i \) pour tout \( i \) donc \( y = x +
    \delta(y_m - x_m) \). Si de plus \( y \in X^m \) on a \( y_m - x_m \in
    \bigcap_{i=1}^m X - x_i \).
\end{proof}

Étudions dans quel cas l'égalité peut être obtenue. Par définition du
stabilisateur, on a (ensemblistement) :
\begin{equation}
  S_X = \bigcap_{x \in X(\Qbar)} X - x
  \pmm.
\end{equation}
Par noethérieneté, il est clair qu'on peut en fait prendre
une intersection finie ; le lemme suivant précise ce résultat.

\begin{lem}
  Posons \( M = 2 D^{u+1} - 2 \) ; pour tout \( m \ge M \) il existe
  \( (x_1, \dots, x_m) \in X^m(\Qbar) \) tel que
  \( S_X = \bigcap_{i=1}^m X - x_i \).
\end{lem}

\begin{proof}
  En partant d'un \( X - x_1 \) pour \( x_1 \) quelconque, on va couper
  successivement par des \( X - x_i \) choisis de sorte à faire chuter la
  dimension d'au moins une des composantes de dimension maximale de
  l'intersection partielle qui n'est pas une composante de \( S_X \).
  Clairement, ce processus se termine et l'intersection obtenue est \( S_X \).

  Plus précisément, on note \( Y_j = \bigcap_{i=1}^{j+1} X - x_i \) le fermé de
  \bsc{Zarisiki} obtenu après \( j \) intersections.  Pour nous aider à
  compter les composantes à chaque étape, on considère la quantité :
  \begin{equation}
    SH(Y, d)
    =
    \sum_Z \deg Z d^{\dim Z}
  \end{equation}
  où la somme est prise sur l'ensemble des composantes de \( Y \). Cette
  définition coïncide avec celle donnée p.~364 de \cite{philz} (voir aussi (*)
  p.~362) ; on est ici dans le cas homogène c'est-à-dire \( p = 1 \) dans les
  notations de cette référence.

  Il est clair que \( SH(Y_1, D) = SH(X, D) = D^{u+1} \). Par ailleurs, comme
  chaque \( Y_j \) peut être défini (ensemblistement) par des équations de
  degré au plus \( D \), on peut appliquer à ces équations la proposition~3.3
  de la référence citée, ce qui garantit que \( SH(Y_{j+1}, D) \le SH(Y_j, D)
  \) et donne immédiatement par récurrence \( SH(Y_j, D) \le D^{u+1} \).

  Par définition de \( SH \), on voit donc que pour tout \( k \), le nombre de
  composantes de \( Y_j \) de dimension \( u - k \) est au plus \( D^{k+1} \).
  On peut être un peu plus précis pour \( Y_1 \) et voir que toutes ses
  composantes sont de dimension au plus \( u - 1 \) car \( X - x_1 \) n'a
  qu'une composante. Ainsi, en choisissant successivement les \( x_i \) de
  façon à ce que \( X - x_{j+2} \) coupe strictement une des composantes de \(
    Y_j \) de dimension maximale parmi celles qui ne sont pas composantes de \(
    S_X \), on voit qu'il existe un \( j_1 \le 1 + D^2 \) tel que \( Y_j \) n'ai
  plus de composante de dimension \( u - 1 \) à part celles de \( S_X \), et
  en continuant jusqu'à avoir éliminé les composantes indésirables de
  dimension \( 0 \), qu'il existe un \( j_u \le 1 + D^2 + D^3 + \dots +
    D^{u+1} \) tel que \( Y_{j_u} \) n'ait plus aucune composante autre que
  celles de \( S_X \), c'est-à-dire tel que \( Y_{j_u} = S_X \).

  Le nombre de points utilisés est alors
  \begin{equation}
    j_u + 1
    \le
    2 + D^2 \sum_{l=0}^{u-1} D^l
    =
    2 + \frac{D}{D+1} (D^{u+1} - D)
    \le
    2 + 2 (D^{u+1} - D)
    \le M
    \pmm,
  \end{equation}
  en utilisant le fait que \( D \ge 2 \).

  Par ailleurs, on constate qu'une fois qu'on a choisi \( x_1, \dots,
    x_{j_u+1} \) tels que \( Y_{j_u} = S_X \), on peut rajouter des \( x_i \)
  arbitraires pour \( i > j_u + 1 \) et on aura toujours \( S_X = \bigcap_i
    X - x_i \). Ainsi, pour tout \( m \ge M \) il existe une famille \(
    (x_i)_i \in X^m \) possédant la propriété voulue.
\end{proof}

On note désormais \( M \) l'entier introduit par le lemme précédent.

\begin{coro} \label{c:gen-fiber}
  Pour \( m \ge M \), il existe un ouvert dense de \( s_m(X^m) \) où les
  fibres de \( {s_m}_{|X} \) sont de dimension \( \dim S_X \). En particulier,
  \( \dim s_m(X^m) = m \dim X - \dim S_X \)
\end{coro}

\begin{proof}
  On choisit un point \( (x_1, \dots, x_m) \in X^m \) tel que
  \( S_X = \bigcap_{i=1}^m X - x_i \), ce qui est possible d'après le choix de
  \( m \). La fibre de \( {s_m}_{|X} \) en \( s_m(x) \) est alors isomorphe à
  \( S_X \) et en particulier, de même dimension.

  Par ailleurs, toutes les fibres sont de dimension au moins \( \dim S_X \).
  Par semi-continuité supérieure de la dimension des fibres, l'ensemble des
  points où elle est égale à \( \dim S_X \) est donc un ouvert. D'après le
  paragraphe précédent, cet ouvert n'est pas vide, il est donc dense.
\end{proof}

On rappelle que dans l'énoncé suivant, \( Z = s_m(X^m) \) et que le degré est
relatif au plongement \( Z \embedin \proj N \) évoqué précédemment.

\begin{lem}
  On a \( \deg Z \le D^m (2m)^{ (m+1)(u - \dim S_X) } / \deg S_X \).
\end{lem}

\begin{proof}
  On note \( u' = \dim Z = mu - \dim S_X \) et on choisit des hyperplans de \(
    \proj N \) qu'on notera \( H_1, \dots, H_{u'} \) et qu'on supposera en
  position générale, de sorte que \( F = \bigcap_i H_i \) est un ensemble fini
  de points, de cardinal \( \deg Z \). On peut même supposer que \( F \) est
  inclus dans l'ouvert donné par le corollaire~\vref{c:gen-fiber} car ce
  dernier est dense.

  On note donc \( F = \set{y_1, \dots, y_p} \) ; pour chaque \( y_j \) on
  choisit un point \( x_j \in s_m^{-1}(y_j) \cap X^m \). Pour tout \( j \in
    \set{1, \dots, p} \) et tout \( k \in \set{1, \dots, m} \), le morphisme
  \( \pi_k s_m \), où \( \pi_k \) est la projection sur le \( k \)-ième
  facteur, qui est en fait la différence de \( \va \), est représenté sur un
  voisinage de \( x_j \) par une famille de \( n + 1 \) formes multihomogènes
  de multidegré \( (0, \dots, 0, 2, 0, \dots, 0, 2) \) où le premier \( 2 \)
  est en \( k \)-ième position. On en déduit aisément que \( s_m \) est
  représenté sur un voisinage de \( x_j \) par une famille de \( N + 1 \)
  formes multihomogènes de multidegré \( (2, \dots, 2, 2(m-1)) \). On a \lat{a
    priori} une telle famille par point \( x_j \), mais quitte à prendre une
  combinaison linéaire de toutes ces familles, on voit qu'il existe une
  famille de formes de multidegré \( (2, \dots, 2, 2(m-1)) \) représentant \(
    s_m \) sur un ouvert contenant tous les \( x_j \).

  Maintenant, pour chaque \( i \in \set{1, \dots, u'} \), on choisit une
  équation \( E_i \) de \( H_i \). Dans cette forme linéaire, on substitue aux
  variables les éléments de la famille du paragraphe précédent et on note \(
    \tilde E_i \) la forme multihomogène obtenue, qui est de multidegré \( (2,
    \dots, 2, 2(m-1)) \), puis on note \( \tilde H_i \) l'hypersurface de \(
    (\projd)^m \) définie par \( \tilde E_i \).

  Par construction, il est clair que pour tout \( j \in \set{1, \dots, p} \),
  la fibre \( s_m^{-1}(-y_j) \cap X^m \) est un translaté de \( \delta(S_X) \)
  et est une composante isolée de l'intersection \( X^m \cap \bigcap_i \tilde
    H_i \) : en effet, les fibres sont deux à deux d'intersection vide et les
  autres composantes éventuelles de l'intersection sont celles provenant du
  lieu des zéros commun des formes choisies pour représenter \( s_m \), qui ne
  contient aucune des fibres en question grâce au choix de cette famille.

  On invoque alors la proposition~3.3, p.~365 de \cite{philz} dont on reprend
  les notations (voir p.~364 et p.~362) :
  \begin{equation}
    SH( X^m \cap \bigcap_i \tilde H_i \, ; 2, \dots, 2, 2(m-1) )
    \le
    SH( X^m ; 2, \dots, 2, 2(m-1) )
    \pmm.
  \end{equation}
  On remarque que le seul multidegré non nul de \( X^m \) est celui
  d'indice \( (u, \dots, u) \) et qu'il vaut \( D^m \),  de sorte que l'on a
  \begin{equation}
    SH( X^m ; 2, \dots, 2, 2(m-1) )
    =
    D^m
    \frac{ (mu)! }{ (u!)^m }
    \, 2^{mu} (m-1)^u
    \le
    D^m
    (2m)^{mu} (m-1)^u
  \end{equation}
  en majorant le coefficient multinomial qui apparaît dans cette expression
  par \( m^{mu} \). Par ailleurs, on a
  \begin{align}
    SH( X^m \cap \bigcap_i \tilde H_i \, ; 2, \dots, 2, 2(m-1) )
    \ge
    p \cdot H( \delta(S_X); 2, \dots, 2, 2(m-1) )
  \end{align}
  d'après le paragraphe précédent. De plus, on voit assez facilement que tous
  les multidegrés de \( \delta(S_X) \) sont égaux à \( \deg S_X \), ce qui
  donne
  \begin{align}
    H( \delta(S_X); 2, \dots, 2, 2(m-1) )
    & =
    \ \sum_{ \mathclap{\alpha_1 + \dots + \alpha_m = \dim S_X} } \
    \deg S_X
    \, \frac{ (\dim S_X)! }{ \alpha_1! \cdots \alpha_m! }
    \, 2^{m\dim S_X} (m-1)^{\dim S_X}
    \\ & =
    \deg S_X
    (2m)^{m\dim S_X} (m-1)^{\dim S_X}
  \end{align}
  d'après la formule multinomiale. Ainsi, on a
  \begin{equation}
    p
    \le
    \frac{ D^m }{ \deg S_X }
    (2m)^{m(u-\dim S_X)} (m-1)^{u - \dim S_X}
    \le
    \frac{ D^m }{ \deg S_X }
    (2m)^{(m+1)(u-\dim S_X)}
  \end{equation}
  comme annoncé.
\end{proof}

Dans le lemme suivant, on rappelle que \( \hautl[2]{ s_m(x) } \) est la
hauteur donnée par le plongement dans \( \proj N \).

\begin{lem}
  Pour tout \( x \in X^m \), on a 
  \begin{equation}
    \hautl[2]{ s_m(x) }
    \le
    4 (m-1) \max_{1 \le i \le m} \bigl( \hautl[2]{ x_i } \bigr)
    + (m-1) \hlclab
    \pmm.
  \end{equation}
\end{lem}

\begin{proof}
  On reprend la construction faite dans la démonstration du lemme précédent
  pour exhiber une famille de formes représentant \( s_m \) au voisinage de \(
    x \), en détaillant la construction pour bien contrôler la hauteur de cette
  famille.

  La section~\vref{sec:vaemb} fournit, pour chaque \( i \in \set{1, \dots,
      m-1} \) une famille de formes bihomogènes de bidegré \( (2, 2) \)
  représentant la soustraction de \( \va^2 \) dans \( \va \) au voisinage de
  \( (x_i, x_m) \), que l'on notera \( (L\pexp i[k])_{k \in \set{0,
        \dots, n}} \), telle que \( \nnv1{ L\pexp i } \le \hmclab* \).
  Pour chaque \( l \in \set{0, \dots, n}^{m-1} \) on pose
  \begin{equation}
    S_l(\vmp[1], \dots, \vmp[m-1])
    =
    \prod_{i=1}^m L\pexp i[l_i]( \vmp[i], \vmp[m] )
  \end{equation}
  de sorte que la famille \( (S_l)_l \) représente \( s_m \) au voisinnage de
  \( x \). On a alors \( \nnv1{ S } \le \hmclab* ^{m-1} \) par les propriétés
  de la norme \( L_1 \). Par ailleurs, les propriétés de la norme euclidienne
  donnent \( \nv2{ S(x) } \le \nnv2{ S } \nv2{ x_1 }^2 \dots \nv2{ x_{m-1} }^2
    \nv2{ x_m }^{2(m-1)} \) compte tenu du multidegré des formes \( S_l \).

  Le résultat annoncé vient en prenant le logarithme puis la somme sur toutes
  les places.
\end{proof}

\begin{coro}
  On a \( \hautl Z \le \dots \).
\end{coro}

\begin{proof}
  On passe par le minimum essentiel.
\end{proof}



\section{Cas général}

On fixe désormais \( m = \) la borne de Rémond plus 2. On fixe de plus \(
  (x_1, \dots x_m) \in \va(\Qbar)^m \) satisfaisant les hypothèse de Mumford
(en particulier \( x_m \) a la plus petite hauteur). On note \( z = s_m(x) \).

\begin{lem}
  On a 
  \begin{equation}
    \hautl z 
    \le
    (\rho^2/4 + 2\phi + \rho\phi) (\dots) \hautn{ x_m }
  \end{equation}
\end{lem}

\begin{ideas}
  On utilise \vref{l:diff-small} pour contrôler chaque \( \hautn{ x_i - x_m }
  \). On compare ensuite à la hauteur projective. On conclut en utilisant
  la formule pour les hauteurs dans un Segre. (référence ?)\todo
\end{ideas}

\begin{lem}
  \begin{equation}
    \prod_\placerange \distv z Z ^\degv
    \le
    cte(\va) \hautm[2]{ x_m }^{-\expapx}
  \end{equation}
\end{lem}

\begin{ideas}
  On utilise \vref{l:diff-apx} pour chaque composante puis la formule pour les
  distances dans un Segre. (référence ?)
\end{ideas}

\begin{lem}
  On a \( z \in Z \).
\end{lem}

\begin{ideas}
  On procède comme dans la démo du thm 3.1.1.
\end{ideas}

La suite : cf (5) sur papier.


\cleardoublepage
\endinput

% vim: spell spelllang=fr
