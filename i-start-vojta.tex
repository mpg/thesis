% !TEX root = main.tex

\chapter{Inégalité de Vojta} \label{chap:vojta}

\section{Stratégie générale et notations additionnelles}
\label{sec:vojta-intro}

Le but de cette section de des suivantes est de prouver une inégalité de
\bsc{Vojta} dans le cas particulier où \( \varapx \) est découpé par un
hyperplan. Nous en déduirons ensuite le cas général à la
section~\ref{sec:vojta-gene}.

On suppose que $\va$ n'est pas contenue dans l'hyperplan
d'équation $\ch0 = 0$, et on considère le diviseur $\divi$ découpé par
$\ch0$ sur $\va$. On note enfin $\ideal\va$ l'idéal homogène saturé
définissant $\va$ dans $\cdn [\ch0, \ldots, \ch\dimp]$.

On considère $\puiss$ un entier positif, et $(\proj\dimp)^\puiss$ l'espace
multiprojectif associé à l'anneau multigradué
\begin{math}
  \cdn [
    \cmh1[0], \dots, \cmh1[\dimp];
    \dots;
    \cmh\puiss[0], \dots \cmh\puiss[\dimp]
  ]
  =
  \cdn [ \cmh1, \dots, \cmh\puiss ]
  =
  \cdn [ \cmmh ]
\end{math}.
On s'intéresse à une sous-variété produit de $\va^\puiss$ plongée dans
$(\proj\dimp)^\puiss$, notée
\begin{math}
  \var = \var[1] \times \dots \times \var[\puiss]
\end{math}
et dont aucun facteur n'est contenu dans un hyperplan d'équation $\cmh*0 = 0$.
On notera en outre $\vdim* = \dim \var*$ et $\vdim = (\vdim*)_\fct$ puis
$\vdeg* = \deg \var*$, et $\varfc*$ une forme de \bsc{Chow} de $\var*$. On
note enfin $\varid$ l'idéal multihomogène saturé de $\var$ et $\varid*$ ceux
de ses facteurs.

De façon générale, si $A$ est une algèbre graduée et $\Ideal$ un idéal
homogène, on notera $A_d$ et $\Ideal_d$ leur partie homogène de degré $d$ ; on
utilisera la même notation pour les algèbres et idéaux multigradués, où $d$
désignera une famille d'entiers.

\subsection{Plongements projectifs adaptés} \label{sub:plong-adapt}

\begin{tdef} \label{d:plong-adapt}
  Suivant \cite{remivg}, on dit qu'un plongement
  \begin{math}
    \iota\colon V \embedin \proj\dimp
  \end{math}
  d'une variété de dimension $t$ dans un espace projectif muni de coordonnées
  homogènes $X_0, \dots, X_\dimp$ est \emph{adapté} si les deux conditions
  suivantes sont satisfaites :
  \begin{enumthm}
    \item $V \cap \zeros{X_0, \dots, X_t} = \emptyset$ ;
    \item $\korper{V}$ est engendré par
      $\frac{X_1}{X_0}, \dots, \frac{X_{t+1}}{X_0}$.
  \end{enumthm}
\end{tdef}

\begin{lem}
  Soit $\iota \colon V \embedin \proj\dimp$ un sous-schéma fermé intègre de
  degré $\Delta$, non contenu dans l'hyperplan d'équation $X_0 = 0$. Il existe
  une matrice $M \in \GL_{n+1}(\Q)$, à coefficients entiers de valeur absolue
  (archimédienne) majorée par $\max(\frac\Delta2, 1)$, telle que, si $e_M$ est
  l'automorphisme linéaire de $\proj\dimp$ associé à $M$, alors $e_M \circ
  \iota$ est un plongement adapté et que $X_0$ soit invariant par ce
  changement de coordonnées.
\end{lem}

\begin{proof} \later
  On reprend la preuve de la proposition 2.2 de \cite{remivg} (p.~469). Au
  moment de choisir des formes linéaires $W_0, \dots, W_n$ telles que
  \begin{equation*}
    \chow V (W_0, \dots, W_\dimp) \neq 0
    \pmm,
  \end{equation*}
  on commence en fait par fixer $W_0 = \ch0$. Le polynôme $\varfc(W_0, \truc,
  \dots, \truc)$ est multihomogène de degré $\Delta$ en chaque variable ; vu
  l'hypothèse sur $V$, il est non nul grâce au théorème fondamental de
  l'élimination. On peut donc choisir $W_1, \dots W_n$ comme dans
  \cite{remivg} et continuer la preuve sans autre modification.
\end{proof}

\begin{lem}
  Soit $V \colon \embedin \proj\dimp$ un sous-schéma fermé intègre, et $M \in
  \GL_{\dimp+1}(\Qbar)$ à coefficients entiers dans l'intervalle $[-B, B]$, où
  $B$ est un réel positif fixé, et $e_M$ l'automorphisme linéaire de
  $\proj\dimp$ associé à $M$. Alors
  \[
    \hautm[1]{e_M(V)}
    \le
    \hautm[1]{V}
    \cdot \bigl( (n+1)B \bigr)^{\Delta(n+1)}
    \pmm.
  \]
\end{lem}

\begin{proof} \later
  Inégalité classique sur la norme $L_1$ et la spécialisation des polynômes.
\end{proof}

\begin{scho} \label{s:plong-adapt}
  Vu les hypothèses sur $V$, on peut sans perte de généralité supposer de plus
  que $V \embedin \proj\dimp$ est adapté, quitte à multiplier sa hauteur
  ($L_1$) par
  $\bigl ((\dimp + 1) \max (\frac\Delta2, 1) \bigr)^{\Delta(\dimp+1)}$.
  \worknote[later]{Il faut aussi estimer le coût sur les formules de
    multiplication et soustraction.}

  On supposera donc désormais que chaque $\var*$ est plongé de façon adaptée
  dans son facteur $\proj\dimp$. Ceci implique que
  $\varfc*(\cmh*0, \dots,\cmh*{\vdim*})$
  est non nul pour tout $\fct$ ; on normalisera donc notre choix de $\varfc*$
  (qui est unique à une constante multiplicative près) en imposant que cette
  quantité vaille $1$.
\end{scho}

\subsection{Plongement abélien pondéré} \label{sub:wemb}

Par la suite, on se donnera un point
$(\excep = \excep[1], \dots, \excep[\puiss])$
de $\var(\Qbar)$, où les $\excep*$ sont de hauteur grandes et étagées entre
elles, contenus dans un secteur de cône de l'espace de \bsc{Mordell-Weil}. On
définira alors les poids $\wt*$ par
\begin{equation}
  \wt* =
  \left\lfloor
  \frac {\norm{\excep*}} {\norm{\excep1}}
  + \frac12
  \right\rfloor
  \pmm,
\end{equation}
où $\norm\truc$ représente la norme associée à la hauteur normalisée dans
l'espace de \bsc{Mordell-Weil} de $\va$ (on commet l'abus consistant à noter
encore $\excep*$ l'image de $\excep*$ dans cet espace).

On introduit un réel $\epsiii$ satisfaisant à
\begin{equation} \label{e:def-epsiii}
  \sum_{\fct=1}^\puiss (\wts* + 1) < \wts[1] (1 + \epsiii)
  \pmm.
\end{equation}
\todo La définition de \( \wt \), les hypothèses faites sur \( \excep \) et la
géométrie euclidienne montrent alors qu'on peut prendre \( \epsiii \) plutôt
petit\dots

La géométrie euclidienne montre\footnote{
  Les détails sont présents dans la preuve de l'inégalité à la \bsc{Mumford}.
  Cette information est en fait inutile ici ; on souhaite seulement justifier
  vaguement le rôle du plongement éclatant.}
aussi que $\wt* \excep* - \excep[\puiss]$ est de hauteur petite devant les
hauteurs des différents $\excep*$. Pour nous aider à exploiter ultérieurement
cette information, on introduit un plongement, dit \emph{éclatant}, associé à
$\wt$, défini par
\begin{align} \label{e:def-wemb}
  \wemb \colon \var
  & \longto \va^\puiss \times \va^{\puiss-1}
  = \va^{2\puiss-1} \subset (\projd)^{2\puiss-1}
  \\
  (x_1, \dots, x_\puiss)
  & \longmapsto
  (x_1, \dots, x_\puiss;
  \wt[1] x_1 - x_\puiss, \wt[\puiss-1] x_{\puiss-1} - x_\puiss)
  \pmm.
\end{align}

On munit l'espace d'arrivée $(\projd)^{2\puiss-1}$ des coordonnées
multihomogènes $\cmmh, \cmmhi = \cmh1, \dots, \cmh\puiss, \cmhi1,
\cmhi{\puiss-1}$ ; dans ce contexte quand $i$ et $j$ sont deux indices non
précisés, on supposera implicitement $1 \le i \le \puiss$ et $1 \le j \le
\puiss-1$.

On va maintenant s'attacher à représenter localement ce morphisme par une
famille de formes multihomogènes. En appliquant ce formulaire de la
section~\ref{sec:form-ab} à notre plongement éclatant, on voit que tout
point $x \in \var(\Qbar)$ est contenu dans un ouvert $\opdef \subset \var$ sur
lequel il existe des polynômes représentant
$\wemb\vert_\opdef \colon \opdef \to (\projd)^{2\puiss-1}$.
Plus précisément, il existe un morphisme d'algèbres, dépendant de $\opdef$,
\begin{align}
  \wemba \colon \cdn [\cmmh, \cmmhi] & \to \cdn[\cmmh] \\
  \cmh{i} & \mapsto \cmh{i} \\
  \cmhi{j} & \mapsto S^{-}(Q_{\wt[j]}(\cmh{j}, \cmh{\puiss}))
\end{align}
tel que le diagramme suivant, dont les flèches verticales sont les projections
canoniques, commute :
\begin{equation}
  \xymatrix{
    \cdn [\cmmh, \cmmhi]
    \ar[r] ^{\wemba}
    \ar[d]
    & \cdn [\cmmh]
    \ar[d]
    \\ \cdn [\cmmh, \cmmhi] / \ideal{\wemb(\var)}
    \ar[r] ^{\wemb*}
    & ( \cdn [\cmmh] / \varid )_\opdef
  }
\end{equation}
On a choisi $S^{-}$ et $Q_{\wt[j]}$ parmi les familles de formes données
respectivement part le corollaire~\ref{c:addsub-form} et le
fait~\ref{f:mult-form} de sorte que les formules obtenues soient valables au
voisinages de $x$.

Les informations connues sur les degrés de $S^{-}$ et des $Q_{\wt[j]}$ donnent
immédiatement le résultat suivant.

\begin{lem} \label{l:deg-wemba}
  Soient $\wemba$ un morphisme d'algèbre comme ci-dessus et $F \in
  \Qbar[\cmmh, \cmmhi]$ une forme multihomogène de multidegré $(\alpha,
  \beta)$ où $\alpha \in \N^\puiss$ et $\beta \in \N^{\puiss-1}$. On a alors
  \begin{equation}
    \deg \wemba(F)
    \le
    \bigr(
    \alpha_1 + \frac94 \beta_1 \wts[1],
    \dots,
    \alpha_{\puiss-1} + \frac94 \beta_{\puiss-1} \wts[\puiss-1],
    \alpha_\puiss + 2 \lgr\beta
    \bigl)
    \pmm.
  \end{equation}
\end{lem}



\endinput

% vim: spell spelllang=fr
