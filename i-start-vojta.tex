% !TEX root = main.tex

\chapter{Inégalité de \bsc{Vojta}} \label{chap:vojta}


\section{Stratégie de la preuve}

Nous établissons ici une inégalité de \bsc{Vojta} effective dans le cas
particulier où la variété à approcher est la diviseur \( \divi \) découpé par
l'hyperplan d'équation \( X_0 = 0 \), supposé ne pas contenir \( \va \). Nous
verrons plus tard que ce cas particulier implique le cas général.

Le but du chapitre est de prouver le théorème suivant.

\begin{thm} \label{t:vojta-div}
  Dans \( \va(\Qbar) \), il n'existe pas de famille de points \( \excep[1],
    \dots, \excep[\puiss] \) satisfaisant simultanément aux conditions
  suivantes :
  \begin{align}
    0 < \distv{\excep*}{\divi}
    & < \hautm{\excep*}^{-\wtapx \expapx}
    \quad \forall \place \in \placesapx
    \label{e:Vapx}
    \\
    \hautn{\excep[1]} & > \Vbig
    \label{e:Vbig}
    \\
    \hautn{\excep*} & > \Vfar \hautn{\excep[\fct-1]}
    \label{e:Vfar}
    \\
    \cos(\excep*, \excep[\fcti]) & > 1 - \Vcos
    \label{e:Vcos}
  \end{align}
  avec
  \nomg{alpha}{\Vbig}{Une constante du théorème~\ref{t:vojta-div}}
  \nomg{beta} {\Vfar}{Une constante du théorème~\ref{t:vojta-div}}
  \nomg{gamma}{\Vcos}{Une constante du théorème~\ref{t:vojta-div}}
  \begin{align}
    \Vbig & = \dots (\puiss)
    \\
    \Vfar & = \dots (\puiss)
    \\
    \Vcos & = \dots (\puiss)
  \end{align}
\end{thm}

Nous commençons par réduire ce théorème à un énoncé d'aspect technique se
prêtant bien à une démonstration par la méthode de \TS, que nous appliquons
alors.

\noma*{e}{(\excep*)}{Une famille de points dont l'existence contredirait le
  théorème~\ref{t:vojta-div}}
La démonstration procède par l'absurde : si le théorème est faux, fixons une
famille \( \excep[1], \dots, \excep[\puiss] \) qui le contredit. Nous
regarderons cette famille comme un point de \( \va^\puiss \) plongée dans
l'espace multiprojectif \( (\proj\dimp)^\puiss \) associé à l'anneau
multigradué
\begin{math}
  \cdn [
    \vmp[1][0], \dots, \vmp[1][\dimp];
    \dots;
    \vmp[\puiss][0], \dots \vmp[\puiss][\dimp]
  ]
  =
  \cdn [ \vmp[1], \dots, \vmp[\puiss] ]
  =
  \cdn [ \vmp ]
\end{math}.
La famille \( \excep \) est toujours supposée satisfaire aux
conditions~\eqref{e:Vapx} à~\eqref{e:Vcos} ; toutefois par souci de clarté on
rappellera dans la plupart des énoncés lesquelles de ces conditions sont
utilisées.

La preuve utilisera des combinaisons linéaires de \( \excep* \) de petit
hauteur. Les lemmes suivants nous donneront des bons coefficients pour ces
combinaisons ; nous les choisirons entiers et n'ayant que $2$ et $3$ pour
diviseurs premiers, de sorte à disposer de représentations polynomiales
convenable, voir section~\ref{sec:form-ab-alt}.

\begin{lem} \label{l:wt-choose-gen}
  Soit \( \zeta \) un réel positif. Il existe des entiers \( \wt* \in 2^\N
    3^\N \) tels que, pour tout \( \fct \in \set{1, \dots, \puiss} \) :
  \begin{equation} \label{e:wt-ratio}
    \frac1{1 + \zeta}
    \le
    \frac{ \wts* \hautn{\excep*} }{ \wts[1] \hautn{\excep[1]} }
    \le
    1 + \zeta
    \pmm.
  \end{equation}
\end{lem}

\begin{proof}
  On commence par choisir des rationnels \( b_\fct = 2^{b_{\fct2}}
    3^{b_{\fct3}} \) tels que
  \begin{equation}
    \frac1{1 + \zeta}
    \le
    b_\fct^2 \frac{ \hautn{\excep*} }{ \hautn{\excep[1]} }
    \le
    1 + \zeta
    \pmm,
  \end{equation}
  soit en prenant les logarithmes et en divisant par deux :
  \begin{equation}
    \abs{
      b_{\fct2} \log 2 + b_{\fct3} \log 3
      - \frac12 \log \frac{ \hautn{\excep*} }{ \hautn{\excep[1]} }
    }
    \le
    \log (1 + \zeta)
    \pmm.
  \end{equation}
  Comme \( \Z \log2 + \Z \log 3 \) est dense dans \( \R \), il est
  certainement possible de choisir indépendamment pour chaque \( \fct \) deux
  entiers \( b_{\fct2} \) et \( b_{\fct3} \) tels que cette dernière condition
  soit satisfaite.

  Il reste plus qu'à définir \( \wt[1] \) comme le plus petit dénominateur
  commun des \( b_\fct \) puis à poser \( \wt* = \wt[1] b_\fct \) pour tout \(
    \fct > 1 \).
\end{proof}

\begin{scho}
  Pour tout famille \( (\wt*) \) satisfaisant à~\eqref{e:wt-ratio}, on a
  \begin{align} \label{wt-diff-hautn}
    \abs[\big]{ \wts* \hautn{\excep*} - \wts[1] \hautn{\excep[1]} }
    & \le \zeta \wts[1] \hautn{\excep[1]}
    \\ \label{e:wt-diff-normn}
    \abs[\big]{ \wt* \normn{\excep*} - \wt[1] \normn{\excep[1]} }
    & \le \frac\zeta2 \wt[1] \normn{\excep[1]}
  \end{align}
\end{scho}

Ces inégalités découlent immédiatement de l'encadrement donné en utilisant les
comparaisons classiques \( (1 + \zeta)^{-1} \ge 1 - \zeta \) et \( (1 +
  \zeta)^{1/2} \le 1 + \zeta/2 \) et le fait que \( \hautn\truc =
  \normn\truc^2 \) par définition de la norme de \NT.

\begin{lem}
  Soit \( \zeta > 0 \) et \( (\wt*) \) une famille d'entiers satisfaisant
  à~\eqref{e:wt-ratio}. Si de plus la famille \( (\excep*) \) satisfait
  à~\eqref{e:Vcos}, alors pour tout \( \fct \in \set{1, \dots, \puiss} \) on a
  \begin{equation} \label{e:hautn-wt-diff-gen}
    \hautn{\wt* \excep* - \wt** \excep**}
    \le
    \wt[1] \hautn{\excep[1]} \left(
      \frac{\zeta^2}4 + \Vcos (1 + \zeta)
    \right)
    \pmm.
  \end{equation}
\end{lem}

\begin{proof}
  En développant le membre de gauche, il vient successivement
  \begin{alignat}{2}
    \normn{\wt* \excep* - \wt** \excep**}^2
    & =
    \normn{\wt* \excep*}^2 + \normn{\wt** \excep**}^2
    - 2 \scalaire{\wt* \excep*}{\wt** \excep**}
    \\
    & = \wts* \normn{\excep*}^2 + \wts** \normn{\excep**}^2
    - 2 \wt* \wt** \scalaire{\excep*}{\excep**}
    \\
    & = \left( \wt* \normn{\excep*} - \wt** \normn{\excep**} \right)^2
    + 2 \wt* \wt** \left(
      \normn{\excep*} \normn{\excep**} - \scalaire{\excep*}{\excep**}
    \right)
    \\
    & \le \left( \wt* \normn{\excep*} - \wt** \normn{\excep**} \right)^2
    + 2 \wt* \wt** \normn{\excep*} \normn{\excep**} \Vcos
    && \text{d'après~\eqref{e:Vcos}}
    \\
    & \le \left( \frac\zeta2 \wt[1] \normn{\excep[1]} \right)^2
    && \text{d'après~\eqref{e:wt-diff-normn}}
    \\
    & \qquad + \left( \wt[1] \normn{\excep[1]} \sqrt{1 + \zeta} \right)^2 \Vcos
    && \text{d'après~\eqref{e:wt-ratio}}
    \\
    & \le \wts[1] \normn{\excep[1]}^2 \left(
      \frac{\zeta^2}4 + \Vcos (1 + \zeta)
    \right)
    && \qedhere
  \end{alignat}
\end{proof}

\begin{scho} \label{s:wt-choose}
  \noma*{a}{(\wt*)}{Poids associés à \( \excep \) et fixés par la
    scholie~\ref{s:wt-choose}}
  On choisit désormais une fois pour toute une famille \( (\wt*) \) donnée par
  l'application du lemme~\ref{l:wt-choose-gen} avec \( \zeta = \sqrt{2\Vcos}
  \) de sorte qu'on a
  \begin{equation} \label{e:hautn-wt-diff}
    \hautn{\wt* \excep* - \wt** \excep**}
    \le
    \wt[1] \hautn{\excep[1]} 2 \Vcos
    \pmm.
  \end{equation}
\end{scho}

Cette dernière égalité découle directement de~\eqref{e:hautn-wt-diff-gen} en
tenant compte du fait que \( 1 + \zeta \le 3/2 \) car \( \Vcos \le 1/4 \).
\label{ct:Vcos<1/4}

% TODO: énoncer la prop avec la var produit

% TODO: prouver qu'elle implique vojta

% TODO: énoncer la prop avec la forme obstructrice

% TODO: prouver qu'elle implique la prop prec

% TODO: introduire le plongement éclatant, lemme de degré et stratégie Thue

% TODO: plongements adaptés : def, prop, supposition et hauteur


\section{Provisoire, à ventiler}

On s'intéresse à une sous-variété produit de $\va^\puiss$ plongée dans
$(\proj\dimp)^\puiss$, notée
\begin{math}
  \var = \var[1] \times \dots \times \var[\puiss]
\end{math}
et qui n'est contenue dans aucun hyperplan d'équation $\vmp*[0] = 0$.
On notera en outre $\vdim* = \dim \var*$ et $\vdim = (\vdim*)_\fct$ puis
$\vdeg* = \deg \var*$, et $\varfc*$ une forme de \bsc{Chow} de $\var*$. On
note enfin $\varid$ l'idéal multihomogène saturé de $\var$ et $\varid*$ ceux
de ses facteurs.

De façon générale, si $A$ est une algèbre graduée et $\Ideal$ un idéal
homogène, on notera $A_d$ et $\Ideal_d$ leur partie homogène de degré $d$ ; on
utilisera la même notation pour les algèbres et idéaux multigradués, où $d$
désignera une famille d'entiers.


\subsection{Plongements projectifs adaptés} \label{sec:plong-adapt}

\begin{tdef} \label{d:plong-adapt}
  Suivant \cite{remivg}, on dit qu'un plongement
  \(
    \iota\colon \anyvar \embedin \proj\dimp
  \)
  d'une variété de dimension \( \anydim \) est adapté si
  \begin{enumthm}
    \item \( \anyvar \cap \zeros{\anyvp[0], \dots, \anyvp[\anydim]}
        = \emptyset \) ;
    \item \( \korper{\anyvar} \) est engendré par
      \( \frac{\anyvp[1]}{\anyvp[0]}, \dots,
        \frac{\anyvp[\anydim+1]}{\anyvp[0]} \) ;
    \item \( \frac{\anyvp[\anydim+1]}{\anyvp[0]} \neq 0 \) dans \(
        \korper\anyvar \).
  \end{enumthm}
\end{tdef}

Le fait suivant, qui ne fait que rappeler \cite[partie~4.1, p.~114]{remivds},
explicite les principales propriétés d'une plongement adapté, qui est en fait
une version plus précise de la mise en position de \bsc{Noether}.

\begin{fact} \label{f:plong-adapt-gen}
  Si \( \anyvar \) de dimension \( \anydim \)  est plongée dans \( \projd \)
  de façon adaptée, alors les fonctions rationnelles
  \( \frac{\anyvp[1]}{\anyvp[0]}, \dots, \frac{\anyvp[\anydim]}{\anyvp[0]} \)
  forment une base de transcendance de \( \korper\anyvar \) sur \( \cdn \). De
  plus, \( \frac{\anyvp[\anydim+1]}{\anyvp[0]} \) est un élément primitif de
  \( \korper\anyvar \) sur \( \cdn( \frac{\anyvp[1]}{\anyvp[0]}, \dots,
    \frac{\anyvp[\anydim]}{\anyvp[0]} ) \).

  La projection linéaire \( \anyvar \to \proj\anydim \) obtenue en ne gardant
  que les \( \anydim + 1 \) variables est un revêtement ramifié.
\end{fact}

On contrôle en fait des relations de dépendance intégrale des dernières
variables sur la base de transcendance choisie.

\begin{fact} \label{f:plong-adapt-dep}
  Si le plongement \( \anyvar \embedin \projd \) est adapté, il existe des
  formes homogènes \( \poldep[][\ind] \) pour \( \ind \in \set{\anydim+1,
      \dots, \dimp} \) telles que :
  \begin{enumthm}
    \item \(
        \poldep[][\ind]
        \in
        \cdn [ \anyvp[0], \dots, \anyvp[\anydim], \anyvp[\ind] ]
        \cap \ideal\anyvar \) ;
    \item \( \poldep[][\ind] \) est unitaire de degré \( \anydeg \) en \(
        \anyvp[\ind] \) ;
    \item \( \deg \poldep[][\ind] = \anydeg \) ;
    \item \( \nv1{ \poldep[][\ind] } \le \nv1{ \chow\anyvar } \) ;
  \end{enumthm}
  où \( \anydeg \) est le degré de \( \anyvar \) dans ce plongement.
\end{fact}

\begin{proof}
  Le lemme~4.1 de \cite{remivds} donne explicitement des formes satisfaisant
  les trois premières conditions.

  Seule l'assertion sur la norme n'y est pas énoncée sous cette forme mais
  elle vient en remarquant que \( \poldep_\indi \) est une spécialisation de
  \( \varfc \) qui annule certaines variables et remplace les autres par des
  monômes unitaires, argument qui est utilisé la proposition~4.2 de la
  référence citée pour obtenir une estimation similaire dans un contexte
  légèrement différent.
\end{proof}

Nous montrons maintenant qu'il est possible de rendre adapté un plongement
donné tout en gardant fixe le diviseur \( \divi \), à peu de frais.

\begin{lem}
  Soit $\iota \colon \anyvar \embedin \projd$ un sous-schéma fermé intègre de
  degré $\anydeg$, non contenu dans l'hyperplan d'équation $\anyvp[0] = 0$.
  Il existe une matrice $M \in \GL_{\dimp*}(\Q)$, à coefficients entiers de
  valeur absolue (archimédienne) majorée par $\max(\frac\anydeg2, 1)$, telle
  que, si $e_M$ est l'automorphisme linéaire de $\proj\dimp$ associé à $M$,
  alors $e_M \circ \iota$ est un plongement adapté et que $\anyvp[0]$ soit
  invariant par ce changement de coordonnées.
\end{lem}

\begin{proof} \later
  On reprend la preuve de la proposition 2.2 de \cite{remivg} (p.~469). Au
  moment de choisir des formes linéaires $M_0, \dots, M_n$ telles que
  \begin{equation*}
    \chow \anyvar (M_0, \dots, M_\dimp) \neq 0
    \pmm,
  \end{equation*}
  on commence en fait par fixer $M_0 = \anyvp[0]$. Le polynôme $\varfc(M_0,
  \truc, \dots, \truc)$ est multihomogène de degré $\anydeg$ en chaque
  variable ; vu l'hypothèse sur $\anyvar$, il est non nul grâce au théorème
  fondamental de l'élimination. On peut donc choisir $M_1, \dots M_n$ comme
  dans \cite{remivg} et continuer la preuve sans autre modification.
\end{proof}

\begin{lem}
  Soit $\anyvar \colon \embedin \proj\dimp$ un sous-schéma fermé intègre, et
  $M \in \GL_{\dimp*}(\Qbar)$ à coefficients entiers dans l'intervalle $[-B,
  B]$, où $B$ est un réel positif fixé, et $e_M$ l'automorphisme linéaire de
  $\proj\dimp$ associé à $M$. Alors
  \begin{equation}
    \hautm[1]{e_M(\anyvar)}
    \le
    \hautm[1]{\anyvar}
    \cdot \bigl( \dimp** B \bigr)^{\anydeg\dimp**}
    \pmm.
  \end{equation}
\end{lem}

\begin{proof} \later
  Inégalité classique sur la norme $L_1$ et la spécialisation des polynômes.
\end{proof}

\begin{scho} \label{s:plong-adapt}
  Vu les hypothèses sur \( \var \) on peut sans perte de généralité supposer
  de plus que chaque plongement $\var* \embedin \projd$ est adapté, quitte à
  multiplier la hauteur de \( \var* \) par
  \( \bigl (\dimp** \max (\frac{\vdeg*}2, 1) \bigr)^{\vdeg*\dimp**} \).
  \worknote[later]{Il faut aussi estimer le coût sur les formules de
    multiplication et soustraction.}

  On supposera donc désormais que chaque $\var*$ est plongé de façon adaptée
  dans son facteur $\proj\dimp$. Ceci implique que
  $\varfc*(\vmp*[0], \dots,\vmp*[\vdim*])$
  est non nul pour tout $\fct$ ; on normalisera donc notre choix de $\varfc*$
  (qui est unique à une constante multiplicative près) en imposant que cette
  quantité vaille $1$.

  On notera par ailleurs \( \poldep** \) une relation de dépendance
  de \( \vmp*[\indi] \) sur \( \vmp*[0], \dots, \vmp*[\vdim*] \)
  telle que donnée par l'application du fait~\ref{f:plong-adapt-dep} au
  facteur \( \var* \).
\end{scho}


\subsection{Plongement abélien pondéré} \label{sec:wemb}

Par la suite, on se donnera un point
$(\excep = \excep[1], \dots, \excep[\puiss])$
de $\var(\Qbar)$, où les $\excep*$ sont de hauteur grandes et étagées entre
elles, contenus dans un secteur de cône de l'espace de \bsc{Mordell-Weil}. On
définira alors les poids $\wt*$ par
\begin{equation}
  \wt* =
  \left\lfloor
  \frac {\norm{\excep*}} {\norm{\excep[1]}}
  + \frac12
  \right\rfloor
  \pmm,
\end{equation}
où $\norm\truc$ représente la norme associée à la hauteur normalisée dans
l'espace de \bsc{Mordell-Weil} de $\va$ (on commet l'abus consistant à noter
encore $\excep*$ l'image de $\excep*$ dans cet espace).

On introduit un réel $\epsiii$ satisfaisant à
\begin{equation} \label{e:def-epsiii}
  \sum_{\fct=1}^\puiss (\wts* + 1) < \wts[1] (1 + \epsiii)
  \pmm.
\end{equation}
\todo La définition de \( \wt \), les hypothèses faites sur \( \excep \) et la
géométrie euclidienne montrent alors qu'on peut prendre \( \epsiii \) plutôt
petit\dots

La géométrie euclidienne montre\footnote{
  Les détails sont présents dans la preuve de l'inégalité à la \bsc{Mumford}.
  Cette information est en fait inutile ici ; on souhaite seulement justifier
  vaguement le rôle du plongement éclatant.}
aussi que $\wt* \excep* - \excep[\puiss]$ est de hauteur petite devant les
hauteurs des différents $\excep*$. Pour nous aider à exploiter ultérieurement
cette information, on introduit un plongement, dit \emph{éclatant}, associé à
$\wt$, défini par
\begin{align} \label{e:def-wemb}
  \wemb \colon \var
  & \longto \va^\puiss \times \va^{\puiss-1}
  = \va^{2\puiss-1} \subset (\projd)^{2\puiss-1}
  \\
  (x_1, \dots, x_\puiss)
  & \longmapsto
  (x_1, \dots, x_\puiss;
  \wt[1] x_1 - x_\puiss, \wt[\puiss-1] x_{\puiss-1} - x_\puiss)
  \pmm.
\end{align}

On munit l'espace d'arrivée $(\projd)^{2\puiss-1}$ des coordonnées
multihomogènes $\vmp, \vmpi = \vmp[1], \dots, \vmp[\puiss], \vmpi[1],
\vmpi[\puiss*]$ ; dans ce contexte quand $i$ et $j$ sont deux indices non
précisés, on supposera implicitement $1 \le i \le \puiss$ et $1 \le j \le
\puiss-1$.

On va maintenant s'attacher à représenter localement ce morphisme par une
famille de formes multihomogènes. En appliquant ce formulaire de la
section~\ref{sec:form-ab} à notre plongement éclatant, on voit que tout
point $x \in \var(\Qbar)$ est contenu dans un ouvert $\opdef \subset \var$ sur
lequel il existe des polynômes représentant
$\wemb\vert_\opdef \colon \opdef \to (\projd)^{2\puiss-1}$.
Plus précisément, il existe un morphisme d'algèbres, dépendant de $\opdef$,
\begin{align}
     \wemba \colon \cdn [\vmp, \vmpi]
  &  \to \cdn[\vmp]
  \\ \vmp*
  &  \mapsto \vmp*
  \\ \vmpi*
  &  \mapsto S^{-}(Q_{\wti*}(\vmp[\fcti]), \vmp[\puiss])
\end{align}
tel que le diagramme suivant, dont les flèches verticales sont les projections
canoniques, commute :
\begin{equation}
  \xymatrix{
    \cdn [\vmp, \vmpi]                          \ar [r] ^{\wemba}   \ar [d]
    & \cdn [\vmp]                                                   \ar [d]
    \\ \cdn [\vmp, \vmpi] / \ideal{\wemb(\var)} \ar [r] ^{\wemb^*}
    & ( \cdn [\vmp] / \varid )_\opdef
  }
\end{equation}
On a choisi $S^{-}$ et $Q_{\wt[j]}$ parmi les familles de formes données
respectivement part le corollaire~\ref{c:addsub-form} et le
fait~\ref{f:mult-form} de sorte que les formules obtenues soient valables au
voisinages de $x$.

Les informations connues sur les degrés de $S^{-}$ et des $Q_{\wt[j]}$ donnent
immédiatement le résultat suivant.

\begin{lem} \label{l:deg-wemba}
  Soient $\wemba$ un morphisme d'algèbre comme ci-dessus et $F \in
  \Qbar[\vmp, \vmpi]$ une forme multihomogène de multidegré $(\alpha,
  \beta)$ où $\alpha \in \N^\puiss$ et $\beta \in \N^{\puiss-1}$. On a alors
  \begin{equation}
    \deg \wemba(F)
    =
    \bigr(
    \alpha_1 + \frac94 \beta_1 \wts[1],
    \dots,
    \alpha_{\puiss-1} + \frac94 \beta_{\puiss-1} \wts[\puiss-1],
    \alpha_\puiss + 2 \lgr\beta
    \bigl)
    \pmm.
  \end{equation}
\end{lem}


\endinput

% vim: spell spelllang=fr
