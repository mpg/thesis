% !TEX root = main.tex

\chapter{Inégalité de \bsc{Vojta}} \label{chap:vojta}


\section{Stratégie de la preuve}

Nous établissons ici une inégalité de \bsc{Vojta} effective dans le cas
particulier où la variété à approcher est le diviseur \( \divi \) découpé par
l'hyperplan d'équation \( X_0 = 0 \), supposé ne pas contenir \( \va \). Nous
verrons plus tard que ce cas particulier implique le cas général.
Le but du chapitre est donc de prouver le théorème suivant.

\begin{thm} \label{t:vojta-div}
  Dans \( \va(\Qbar) \), il n'existe pas de famille de points \( \ex[1],
    \dots, \ex[\puiss] \) satisfaisant simultanément aux conditions
  suivantes :
  \begin{align}
    0 < \distv{\ex*}{\divi}
    & < \hautm{\ex*}^{-\wtapx \expapx}
    \quad \forall \place \in \placesapx
    \label{e:Vapx}
    \\
    \hautn{\ex[1]} & > \Vbig
    \label{e:Vbig}
    \\
    \hautn{\ex*} & > \Vfar \hautn{\ex[\fct-1]}
    \label{e:Vfar}
    \\
    \cos(\ex*, \ex[\fcti]) & > 1 - \Vcos
    \label{e:Vcos}
  \end{align}
  avec
  \nomg{alpha}{\Vbig}{Une constante du théorème~\ref{t:vojta-div}}
  \nomg{beta} {\Vfar}{Une constante du théorème~\ref{t:vojta-div}}
  \nomg{gamma}{\Vcos}{Une constante du théorème~\ref{t:vojta-div}}
  \begin{align}
    \Vbig & = \dots (\puiss)
    \\
    \Vfar & = \dots (\puiss)
    \\
    \Vcos & = \dots (\puiss)
  \end{align}
\end{thm}

\noma*{e}{(\ex*)}{Points dont l'existence contredirait le
  théorème~\ref{t:vojta-div}}
La démonstration procède par l'absurde : si le théorème est faux, fixons une
famille \( \ex[1], \dots, \ex[\puiss] \) qui le contredit. Bien que cette
famille soit toujours supposée satisfaire à toutes les conditions du théorème,
nous préciserons dans les hypothèses de la plupart des énoncés la ou
lesquelles de ces conditions nous utilisons, par souci de clarté.

Nous utiliserons des combinaisons linéaires des \( \ex* \) de petite hauteur.
Les lemmes suivants permettent de choisir les coefficients pour ces
combinaisons ; nous les prendrons entiers et n'ayant que $2$ et $3$ pour
diviseurs premiers, de sorte à disposer de représentations polynomiales
convenables des formes linéaires abéliennes associées données par
l'annexe~\ref{sec:form-ab-alt}.

\begin{lem} \label{l:wt-choose-gen}
  Soit \( \zeta \) un réel positif. Il existe des entiers \( \wt* \in 2^\N
    3^\N \) tels que, pour tout \( \fct \in \set{1, \dots, \puiss} \) :
  \begin{equation} \label{e:wt-ratio}
    \frac1{1 + \zeta}
    \le
    \frac{ \wts* \hautn{\ex*} }{ \wts[1] \hautn{\ex[1]} }
    \le
    1 + \zeta
    \pmm.
  \end{equation}
\end{lem}

\begin{proof}
  On commence par choisir des rationnels \( b_\fct = 2^{b_{\fct2}}
    3^{b_{\fct3}} \) tels que
  \begin{equation}
    \frac1{1 + \zeta}
    \le
    b_\fct^2 \frac{ \hautn{\ex*} }{ \hautn{\ex[1]} }
    \le
    1 + \zeta
    \pmm,
  \end{equation}
  soit en prenant les logarithmes et en divisant par deux :
  \begin{equation}
    \abs*{
      b_{\fct2} \log 2 + b_{\fct3} \log 3
      - \frac12 \log \frac{ \hautn{\ex*} }{ \hautn{\ex[1]} }
    }
    \le
    \frac12 \log (1 + \zeta)
    \pmm.
  \end{equation}
  Comme \( \Z \log2 + \Z \log 3 \) est dense dans \( \R \), il est
  certainement possible de choisir indépendamment pour chaque \( \fct \) deux
  entiers \( b_{\fct2} \) et \( b_{\fct3} \) tels que cette dernière condition
  soit satisfaite.

  Il reste plus qu'à définir \( \wt[1] \) comme le plus petit dénominateur
  commun des \( b_\fct \) puis à poser \( \wt* = \wt[1] b_\fct \) pour tout \(
    \fct > 1 \).
\end{proof}

\begin{scho}
  Pour toute famille \( (\wt*) \) satisfaisant à~\eqref{e:wt-ratio}, on a
  \begin{align} \label{e:wt-diff-hautn}
    \abs[\big]{ \wts* \hautn{\ex*} - \wts[1] \hautn{\ex[1]} }
    & \le \zeta \wts[1] \hautn{\ex[1]}
    \\ \label{e:wt-diff-normn}
    \abs[\big]{ \wt* \normn{\ex*} - \wt[1] \normn{\ex[1]} }
    & \le \frac\zeta2 \wt[1] \normn{\ex[1]}
  \end{align}
\end{scho}

Ces inégalités découlent immédiatement de l'encadrement donné en utilisant les
comparaisons classiques \( (1 + \zeta)^{-1} \ge 1 - \zeta \) et \( (1 +
  \zeta)^{1/2} \le 1 + \zeta/2 \) et le fait que \( \hautn\truc =
  \normn\truc^2 \) par définition de la norme de \NT.

\begin{lem}
  Soit \( \zeta > 0 \) et \( (\wt*) \) une famille d'entiers satisfaisant
  à~\eqref{e:wt-ratio}. Si de plus la famille \( (\ex*) \) satisfait
  à~\eqref{e:Vcos}, alors pour tout \( \fct \in \set{1, \dots, \puiss} \) on a
  \begin{equation} \label{e:hautn-wt-diff-gen}
    \hautn{\wt* \ex* - \wt** \ex**}
    \le
    \wt[1] \hautn{\ex[1]} \left(
      \frac{\zeta^2}4 + \Vcos (1 + \zeta)
    \right)
    \pmm.
  \end{equation}
\end{lem}

\begin{proof}
  En développant le membre de gauche, il vient successivement
  \begin{alignat}{2}
    \normn{\wt* \ex* - \wt** \ex**}^2
    & =
    \normn{\wt* \ex*}^2 + \normn{\wt** \ex**}^2
    - 2 \scalarn{\wt* \ex*}{\wt** \ex**}
    \\
    & = \wts* \normn{\ex*}^2 + \wts** \normn{\ex**}^2
    - 2 \wt* \wt** \scalarn{\ex*}{\ex**}
    \\
    & = \left( \wt* \normn{\ex*} - \wt** \normn{\ex**} \right)^2
    + 2 \wt* \wt** \left(
      \normn{\ex*} \normn{\ex**} - \scalarn{\ex*}{\ex**}
    \right)
    \\
    & \le \left( \wt* \normn{\ex*} - \wt** \normn{\ex**} \right)^2
    + 2 \wt* \wt** \normn{\ex*} \normn{\ex**} \Vcos
    && \text{d'après~\eqref{e:Vcos}}
    \\
    & \le \left( \frac\zeta2 \wt[1] \normn{\ex[1]} \right)^2
    && \text{d'après~\eqref{e:wt-diff-normn}}
    \\
    & \qquad + \left( \wt[1] \normn{\ex[1]} \sqrt{1 + \zeta} \right)^2 \Vcos
    && \text{d'après~\eqref{e:wt-ratio}}
    \\
    & \le \wts[1] \normn{\ex[1]}^2 \left(
      \frac{\zeta^2}4 + \Vcos (1 + \zeta)
    \right)
    && \qedhere
  \end{alignat}
\end{proof}

\begin{scho} \label{s:wt-choose}
  \noma*{a}{(\wt*)}{Poids associés à \( (\ex*) \) et fixés par la
    scholie~\ref{s:wt-choose}}
  \nomg*{zeta}{\zeta}{Constante proche de zéro fixée par la
    scholie~\ref{s:wt-choose}}
  On choisit désormais et jusqu'à la fin du chapitre une famille \( (\wt*) \)
  donnée par l'application du lemme~\ref{l:wt-choose-gen} avec \( \zeta =
    \sqrt{2\Vcos} \) de sorte qu'on a
  \begin{equation} \label{e:hautn-wt-diff}
    \hautn{\wt* \ex* - \wt** \ex**}
    \le
    \wt[1] \hautn{\ex[1]} 2 \Vcos
    \pmm.
  \end{equation}
\end{scho}

Cette dernière égalité découle directement de~\eqref{e:hautn-wt-diff-gen} en
tenant compte du fait que \( 1 + \zeta \le 3/2 \) car \( \Vcos \le 1/4 \).
\label{ct:Vcos<1/4}

Étudions maintenant quelques propriétés de ces poids, en sus de celles déjà
énoncées, en exploitant le fait que les \( \ex* \) satisfont à
\eqref{e:Vbig}. Pour commencer, la suite \( \wts[1], \dots, \wts** \) décroît
au moins comme une suite géométrique de raison inférieure à $1$. Plus
précisément, on a
\begin{equation} \label{e:wt-geom}
  \frac1{1+\zeta} \cdot \frac{\wts[1]}{\Vfar^{\fct-1}}
  \le
  \wts*
  \le
  (1+\zeta) \cdot \frac{\wts[1]}{\Vfar^{\fct-1}}
\end{equation}
en appliquant directement la définition de \( \wt* \). On peut ainsi majorer
des sommes faisant intervenir les \( \wts* \) essentiellement par \( \wts[1]
\) ; par exemple, l'énoncé suivant nous sera utile par la suite.

\begin{lem}
  Il existe \( \epsiii \le 5/\Vfar \) tel que
  \begin{equation} \label{e:def-epsiii}
    \sum_{\fct=1}^\puiss (\wts* + \wts**) \le \wts[1] (1 + \epsiii)
    \pmm.
  \end{equation}
\end{lem}

\begin{proof}
  En effet, on a
  \begin{align}
    \frac1{\wts[1]} \sum_{\fct=1}^\puiss (\wts* + \wts**)
    & = 1
    + \frac{\puiss \wts**}{\wts[1]}
    + \sum_{\fct=2}^\puiss \frac{\wts*}{\wts[1]}
    \\
    & \le 1
    + \frac{\puiss(1+\zeta)}{\Vfar^{\puiss*}}
    + (1+\zeta) \sum_{\fct=2}^\puiss \Vfar^{-\fct+1}
    \\
    & \le 1 + (1+\zeta) \left(
      \Bigl( \frac2\Vfar \Bigr)^{\puiss-1}
      + \frac1\Vfar
    \right)
    \pmm.
  \end{align}
  Pour conclure, il suffit d'observer que \( \frac2\Vfar < 1 \) et que \(
    \zeta < 1/2 \).\label{ct:Vfar>2}
\end{proof}

\subsection{Première réduction}

Nous regardons \( \ex = (\ex*) \) comme un point de \( \va^\puiss \) plongée
dans \( (\projd)^\puiss \), et introduisons l'ensemble \( \varset \) des
sous-variétés produit \( \var \) de \( \va^\puiss \) qui contiennent \( \ex \)
et satisfont aux majorations suivantes :
\begin{align}
  \label{e:varset-deg}
  \vdeg* & \le \Lambda^{f_1(u)} \quad \forall \fct
  \\ \label{e:varset-deg-prod}
  \prod\fctrange \vdeg* & \le (?) \Lambda^{f_1(u)}
  \\ \label{e:varset-ht}
  \sum\fctrange \wts* \hautl{\var*}
  & \le (\dots) \Lambda^{f_2(u)} \wts[1]
  \pmm,
\end{align}
où l'on a noté \( \vdeg* = \deg \var* \) et \( \vdim = \dim \var \), ainsi que
\begin{align}
  \Lambda & = \text{ne dépend que des degrés et dimensions ambiantes}
  \\
  f_1 & = \text{est une fonction décroissante}
  \\
  f_2 & = \text{est une fonction décroissante}
\end{align}

Si \( \ex \) satisfait \eqref{e:Vbig}, cet ensemble ne contient que des
variétés dont tous les facteurs sont de dimension au moins \( 1 \). En effet,
dans le cas contraire, on aurait \( \var* = \ex* \) pour un certain \( \fct \)
dont on déduirait successivement
positifs,
\begin{alignat}{2}
  \wts* \hautl{\ex*}
  & \le (\dots) \Lambda^{f_2(u)} \wts[1]
  &\quad& \text{par \eqref{e:varset-ht},}
  \\
  \wt[1] \hautl{\ex[1]}
  & \le (1+\zeta) (\dots) \Lambda^{f_2(u)} \wts[1]
  && \text{par \eqref{e:wt-ratio},}
\end{alignat}
qui contredirait précisément \eqref{e:Vbig}.

Or cet ensemble \( \varset \) n'est certainement pas vide, puisqu'il contient
au moins \( \va^\puiss \) ; il possède donc au moins un élément minimal. La
proposition suivante implique que cet élément a au moins un facteur réduit à
un point, ce qui, nous venons de le voir, fournit la contradiction prouvant le
théorème \ref{t:vojta-div}.

\begin{prop} \label{p:varset-notmin}
  Soit \( \var \in \varset \) n'ayant aucun facteur de dimension nulle. Alors
  il existe un \( \fct \) et une forme \( T \) ne dépendant que de \( \vmp*
  \), s'annulant en \( \ex \) mais pas identiquement sur \( \var \), telle que
  \begin{align}
    \deg T
    & \le \Lambda^{f_1(u)}
    \\
    \wts* \hautl(T)
    & \le (\dots) \Lambda{f_3(u)} \wts[1]
    \pmm.
  \end{align}
\end{prop}

En effet, on note alors \( \var' \) la variété égale à \( \var \) sur tous les
facteurs sauf le \( \fct \)-ème où on choisit une composante irréductible de
\( \var* \cap \zeros T \) qui contient \( \ex* \), de sorte que \( \var' \)
est une variété strictement contenue dans \( \var \) et contenant \( \ex \).
Le théorème de \bsc{Bézout} donne alors
\begin{equation}
  \deg \var*' \le \vdeg* \deg T
\end{equation}
et sa version arithmétique indique que
\begin{equation}
  \hautl{\var*'} \le \vdeg* \hautl{T} + \hautl{\var*} \deg T
  \pmm,
\end{equation}
ce qui montre que \( \var' \in \varset \) (au vu des constantes qui seront
déterminées par ce point) et que \( \var \) n'était pas minimale.

\medskip

Il suffit donc d'établir la proposition \ref{p:varset-notmin} pour prouver le
théorème \ref{t:vojta-div} ; la démonstration de cette proposition nous
occupera le reste du chapitre. L'intérêt de l'énoncer immédiatement plutôt
qu'au moment où nous serons en mesure de l'établir et de pouvoir écarter
d'emblée quelques cas particuliers pour lesquels la méthode qui suit ne
s'applique pas (mais qui sont heureusement immédiats). Auparavant nous
introduirons quelques hypothèses commodes sur les coordonnées projectives
utilisées.

Désormais et jusqu'à la fin du chapitre, nous fixons une variété
\begin{math}
  \var = \var[1] \times \dots \times \var[\puiss]
\end{math}
satisfaisant aux hypothèses de la proposition ; en particulier elle n'est
contenue dans aucun hyperplan d'équation $\vmp*[0] = 0$ puisqu'elle contient
\( \ex \) qui n'est d'après \eqref{e:Vapx} sur aucun de ces hyperplans.  On
notera en outre $\vdim* = \dim \var*$ et $\vdim = (\vdim*)_\fct$ puis $\vdeg*
= \deg \var*$, et $\varfc*$ une forme de \bsc{Chow} de $\var*$. On note enfin
$\varid$ l'idéal multihomogène saturé de $\var$ et $\varid*$ ceux de ses
facteurs.

De façon générale, si $A$ est une algèbre graduée et $\Ideal$ un idéal
homogène, on notera $A_d$ et $\Ideal_d$ leur partie homogène de degré $d$ ; on
utilisera la même notation pour les algèbres et idéaux multigradués, où $d$
désignera une famille d'entiers.


\subsection{Plongements projectifs adaptés} \label{sec:plong-adapt}

\begin{tdef} \label{d:plong-adapt}
  Suivant \cite{remivg}, on dit qu'un plongement
  \(
    \iota\colon \anyvar \embedin \proj\dimp
  \)
  d'une variété de dimension \( \anydim \) est adapté si
  \begin{enumthm}
    \item \( \anyvar \cap \zeros{\anyvp[0], \dots, \anyvp[\anydim]}
        = \emptyset \) ;
    \item \( \korper{\anyvar} \) est engendré par
      \( \frac{\anyvp[1]}{\anyvp[0]}, \dots,
        \frac{\anyvp[\anydim+1]}{\anyvp[0]} \) ;
    \item \( \frac{\anyvp[\anydim+1]}{\anyvp[0]} \neq 0 \) dans \(
        \korper\anyvar \).
  \end{enumthm}
\end{tdef}

Le fait suivant, qui ne fait que rappeler \cite[partie~4.1, p.~114]{remivds},
explicite les principales propriétés d'une plongement adapté, qui est en fait
une version plus précise de la mise en position de \bsc{Noether}.

\begin{fact} \label{f:plong-adapt-gen}
  Si \( \anyvar \) de dimension \( \anydim \)  est plongée dans \( \projd \)
  de façon adaptée, alors les fonctions rationnelles
  \( \frac{\anyvp[1]}{\anyvp[0]}, \dots, \frac{\anyvp[\anydim]}{\anyvp[0]} \)
  forment une base de transcendance de \( \korper\anyvar \) sur \( \cdn \). De
  plus, \( \frac{\anyvp[\anydim+1]}{\anyvp[0]} \) est un élément primitif de
  \( \korper\anyvar \) sur \( \cdn( \frac{\anyvp[1]}{\anyvp[0]}, \dots,
    \frac{\anyvp[\anydim]}{\anyvp[0]} ) \).

  La projection linéaire \( \anyvar \to \proj\anydim \) obtenue en ne gardant
  que les \( \anydim + 1 \) variables est un revêtement ramifié.
\end{fact}

On contrôle en fait des relations de dépendance intégrale des dernières
variables sur la base de transcendance choisie.

\begin{fact} \label{f:plong-adapt-dep}
  Si le plongement \( \anyvar \embedin \projd \) est adapté, il existe des
  formes homogènes \( \poldep[][\ind] \) pour \( \ind \in \set{\anydim+1,
      \dots, \dimp} \) telles que :
  \begin{enumthm}
    \item \(
        \poldep[][\ind]
        \in
        \cdn [ \anyvp[0], \dots, \anyvp[\anydim], \anyvp[\ind] ]
        \cap \ideal\anyvar \) ;
    \item \( \poldep[][\ind] \) est unitaire de degré \( \anydeg \) en \(
        \anyvp[\ind] \) ;
    \item \( \deg \poldep[][\ind] = \anydeg \) ;
    \item \( \nv1{ \poldep[][\ind] } \le \nv1{ \chow\anyvar } \) ;
  \end{enumthm}
  où \( \anydeg \) est le degré de \( \anyvar \) dans ce plongement.
\end{fact}

\begin{proof}
  Le lemme~4.1 de \cite{remivds} donne explicitement des formes satisfaisant
  les trois premières conditions.

  Seule l'assertion sur la norme n'y est pas énoncée sous cette forme mais
  elle vient en remarquant que \( \poldep_\indi \) est une spécialisation de
  \( \varfc \) qui annule certaines variables et remplace les autres par des
  monômes unitaires, argument qui est utilisé la proposition~4.2 de la
  référence citée pour obtenir une estimation similaire dans un contexte
  légèrement différent.
\end{proof}

Nous montrons maintenant qu'il est possible de rendre adapté un plongement
donné tout en gardant fixe le diviseur \( \divi \), à peu de frais.

\begin{lem}
  Soit $\iota \colon \anyvar \embedin \projd$ un sous-schéma fermé intègre de
  degré $\anydeg$, non contenu dans l'hyperplan d'équation $\anyvp[0] = 0$.
  Il existe une matrice $M \in \GL_{\dimp*}(\Q)$, à coefficients entiers de
  valeur absolue (archimédienne) majorée par $\max(\frac\anydeg2, 1)$, telle
  que, si $e_M$ est l'automorphisme linéaire de $\proj\dimp$ associé à $M$,
  alors $e_M \circ \iota$ est un plongement adapté et que $\anyvp[0]$ soit
  invariant par ce changement de coordonnées.
\end{lem}

\begin{proof} \later
  On reprend la preuve de la proposition 2.2 de \cite{remivg} (p.~469). Au
  moment de choisir des formes linéaires $M_0, \dots, M_n$ telles que
  \begin{equation*}
    \chow \anyvar (M_0, \dots, M_\dimp) \neq 0
    \pmm,
  \end{equation*}
  on commence en fait par fixer $M_0 = \anyvp[0]$. Le polynôme $\varfc(M_0,
  \truc, \dots, \truc)$ est multihomogène de degré $\anydeg$ en chaque
  variable ; vu l'hypothèse sur $\anyvar$, il est non nul grâce au théorème
  fondamental de l'élimination. On peut donc choisir $M_1, \dots M_n$ comme
  dans \cite{remivg} et continuer la preuve sans autre modification.
\end{proof}

\begin{lem}
  Soit $\anyvar \colon \embedin \proj\dimp$ un sous-schéma fermé intègre, et
  $M \in \GL_{\dimp*}(\Qbar)$ à coefficients entiers dans l'intervalle $[-B,
  B]$, où $B$ est un réel positif fixé, et $e_M$ l'automorphisme linéaire de
  $\proj\dimp$ associé à $M$. Alors
  \begin{equation}
    \hautm[1]{e_M(\anyvar)}
    \le
    \hautm[1]{\anyvar}
    \cdot \bigl( \dimp** B \bigr)^{\anydeg\dimp**}
    \pmm.
  \end{equation}
\end{lem}

\begin{proof} \later
  Inégalité classique sur la norme $L_1$ et la spécialisation des polynômes.
\end{proof}

\begin{scho} \label{s:plong-adapt}
  Vu les hypothèses sur \( \var \) on peut sans perte de généralité supposer
  de plus que chaque plongement $\var* \embedin \projd$ est adapté, quitte à
  multiplier la hauteur de \( \var* \) par
  \( \bigl (\dimp** \max (\frac{\vdeg*}2, 1) \bigr)^{\vdeg*\dimp**} \).
  \worknote[later]{Il faut aussi estimer le coût sur les formules de
    multiplication et soustraction.}

  On supposera donc désormais que chaque $\var*$ est plongé de façon adaptée
  dans son facteur $\proj\dimp$. Ceci implique que
  $\varfc*(\vmp*[0], \dots,\vmp*[\vdim*])$
  est non nul pour tout $\fct$ ; on normalisera donc notre choix de $\varfc*$
  (qui est unique à une constante multiplicative près) en imposant que cette
  quantité vaille $1$.

  On notera par ailleurs \( \poldep** \) une relation de dépendance
  de \( \vmp*[\indi] \) sur \( \vmp*[0], \dots, \vmp*[\vdim*] \)
  telle que donnée par l'application du fait~\ref{f:plong-adapt-dep} au
  facteur \( \var* \).
\end{scho}

Nous pouvons maintenant énoncer les cas particuliers à exclure dans la
démonstration de la proposition~\ref{p:varset-notmin}.

\begin{scho} \label{s:part-cases}
  On peut supposer que :
  \begin{enumthm}
    \item \( \cex** \neq 0 \) pour tous \( \fct \) et \( \ind \) ;
    \item \( \pden_\fct(\cex*) \neq 0 \) pour tout \( \fct \).
  \end{enumthm}
\end{scho}

En effet, dans ces cas la conclusion de la proposition est immédiate.


\subsection{Plongement abélien pondéré et stratégie} \label{sec:wemb}

Nous utiliserons un plongement, dit \emph{éclatant} ou pondéré par \( \wt \)
et défini par
\begin{align} \label{e:def-wemb}
  \wemb \colon \var
  & \longto \va^\puiss \times \va^{\puiss-1}
  = \va^{2\puiss-1} \subset (\projd)^{2\puiss-1}
  \\
  (x_1, \dots, x_\puiss)
  & \longmapsto
  (x_1, \dots, x_\puiss;
  \wt[1] x_1 - \wt** x_\puiss, \wt[\puiss-1] x_{\puiss-1} - \wt** x_\puiss)
  \pmm.
\end{align}
% TODO macros ci-dessus pour x
% TODO répercuter les \wt** manquants dans toute la suite

On munit l'espace d'arrivée $(\projd)^{2\puiss-1}$ des coordonnées
multihomogènes $\vmp, \vmpi = \vmp[1], \dots, \vmp[\puiss], \vmpi[1],
\vmpi[\puiss*]$ ; dans ce contexte quand $i$ et $j$ sont deux indices non
précisés, on supposera implicitement $1 \le i \le \puiss$ et $1 \le j \le
\puiss-1$.

On va maintenant s'attacher à représenter localement ce morphisme par une
famille de formes multihomogènes. En appliquant ce formulaire de la
section~\ref{sec:form-ab} à notre plongement éclatant, on voit que tout
point $x \in \var(\Qbar)$ est contenu dans un ouvert $\opdef \subset \var$ sur
lequel il existe des polynômes représentant
$\wemb\vert_\opdef \colon \opdef \to (\projd)^{2\puiss-1}$.
Plus précisément, il existe un morphisme d'algèbres, dépendant de $\opdef$,
\begin{align}
     \wemba \colon \cdn [\vmp, \vmpi]
  &  \to \cdn[\vmp]
  \\ \vmp*
  &  \mapsto \vmp*
  \\ \vmpi*
  &  \mapsto S^{-}(Q_{\wti*}(\vmp[\fcti]), \vmp[\puiss])
\end{align}
tel que le diagramme suivant, dont les flèches verticales sont les projections
canoniques, commute :
\begin{equation}
  \xymatrix{
    \cdn [\vmp, \vmpi]                          \ar [r] ^{\wemba}   \ar [d]
    & \cdn [\vmp]                                                   \ar [d]
    \\ \cdn [\vmp, \vmpi] / \ideal{\wemb(\var)} \ar [r] ^{\wemb^*}
    & ( \cdn [\vmp] / \varid )_\opdef
  }
\end{equation}
On a choisi $S^{-}$ et $Q_{\wt[j]}$ parmi les familles de formes données
respectivement part le corollaire~\ref{c:addsub-form} et le
fait~\ref{f:mult-form} de sorte que les formules obtenues soient valables au
voisinages de $x$.

Les informations connues sur les degrés de $S^{-}$ et des $Q_{\wt[j]}$ donnent
immédiatement le résultat suivant.

\begin{lem} \label{l:deg-wemba}
  Soient $\wemba$ un morphisme d'algèbre comme ci-dessus et $F \in
  \Qbar[\vmp, \vmpi]$ une forme multihomogène de multidegré $(\alpha,
  \beta)$ où $\alpha \in \N^\puiss$ et $\beta \in \N^{\puiss-1}$. On a alors
  \begin{equation}
    \deg \wemba(F)
    =
    \bigr(
    \alpha_1 + \frac94 \beta_1 \wts[1],
    \dots,
    \alpha_{\puiss-1} + \frac94 \beta_{\puiss-1} \wts[\puiss-1],
    \alpha_\puiss + 2 \lgr\beta
    \bigl)
    \pmm.
  \end{equation}
\end{lem}


\endinput

% vim: spell spelllang=fr
