\documentclass{mpg-thslides}

\author[MPG]{Manuel Pégourié-Gonnard}
\institute[UPMC]{\normalsize Université Pierre et Marie Curie}
\title{Approximation diophantienne \\ dans les variétés abéliennes}
\date{Soutenance de thèse \\ 22 octobre 2012
 \\ \normalsize (version corrigée)}

\begin{document}

\maketitle

\toc


\section[Intro]{Introduction : rappels historiques}
\tocsect

\begin{frame}{Théorème de Roth}
  \begin{thm}
    Soient \( \xi \in \R \) un nombre algébrique et \( \eps > 0 \) un réel.
    Il n'existe qu'un nombre fini d'entiers \( p \) et \( q \) tels que
    \begin{equation}
      \abs[\Big]{\frac pq - \xi}
      <
      \abs{q}^{-2-\eps}
      \pmm.
    \end{equation}
  \end{thm}
  \begin{itemize}
    \item Version quantitative : connue.
    \item Version effective : problème ouvert.
  \end{itemize}
\end{frame}

\begin{frame}{Théorème de Roth étendu}
  \begin{thm}[Ridout]
    Soit \( \xi \in \Qbar \) un nombre algébrique. Soient par ailleurs \( \cdn
    \) un corps de nombres et \( \placesapx \) un ensemble fini de places de
    \( \cdn \), étendues de façon arbitraire à \( \cdn(\xi) \). Pour tout \(
      \eps > 0 \), il n'existe qu'un nombre fini de points \( x \in \cdn \)
    tels que
    \begin{equation}
      \prod\placerange \av{x - \xi}^\degv
      <
      \hautm2 x ^{-2-\eps}
    \end{equation}
    où \( \degv = [\cdn(\xi)_v : \Q_v] / [\cdn(\xi) : \Q] \) et chaque valeur
    absolue est normalisée de façon à prolonger une des valeurs absolues
    usuelles de \( \Q \).
  \end{thm}
  % Version quantitative ???
\end{frame}

\begin{frame}{Théorème du sous-espace}
  \begin{thm}[Schmidt, Schlickewei]
    Soient \( n \) un entier et \( L_0, \dots, L_n \in \Qbar[X_0, \dots, X_n]
    \) des formes linéaires indépendantes.  Soient par ailleurs \( \cdn \) un
    corps de nombres et \( \placesapx \) un ensemble fini de places de \( \cdn
    \), étendues de façon arbitraire à \( \cdn(L_0, \dots, L_n) \). Pour tout
    \( \eps > 0 \), l'ensemble des points \( x \in \projd(\cdn) \) tels que
    \begin{equation}
      \prod\placerange
      \prod_{i=0}^n
      \left( \frac{ \av{L_i(x)} }{ \nv{}{x} } \right)^\degv
      <
      \hautm2 x ^{-n-1-\eps}
    \end{equation}
    est contenue dans une union finie de sous-variété linéaires strictes de \(
      \projd \).
  \end{thm}
  % Version quantitative : Sclickewei ???
\end{frame}

\begin{frame}{Ex-conjecture de Mordell}
  \begin{thm}[Faltings 1983]
    Soient \( C \) une courbe projective de genre \( g \ge 2 \) et \( \cdn \)
    un corps de nombres. Alors \( C(\cdn) \) est fini.
  \end{thm}
  \begin{itemize}
    \item Autre démonstration indépendante : Vojta 1991.
    \item Versions quantitatives : Rémond 2002, Farhi 2003.
  \end{itemize}
\end{frame}

\begin{frame}{Ex-conjecture de Mordell-Lang}
  \begin{thm}[Faltings 1991, théorème I]
    Soit \( \avar \) une sous-variété d'une variété abélienne \( \va \)
    définie sur un corps de nombres \( \cdn \).
    Si \( \avar \) ne contient pas de translaté de sous-variété abélienne non
    nulle, alors \( \avar(\cdn) \) est fini.
  \end{thm}
  \begin{itemize}
    \item Version quantitative : Rémond 2002.
  \end{itemize}
\end{frame}

\begin{frame}{Approximation sur les variété abéliennes}
  \begin{thm}[Faltings 1991, théorème II]
    Soit \( \avar \) une sous-variété quelconque de \( \va \), \( v \)
    une place de \( \cdn \), et \( \eps > 0 \). Il n'existe qu'un nombre fini de
    points \( x \) dans \( \va(\cdn) \) tels que
    \begin{equation} \label{e:has}
      0
      <
      \distv x \avar
      \le
      \hautm{} x ^{-\eps}
      \pmm.
    \end{equation}
  \end{thm}

  \begin{coro}[Ex-conjecture de Lang]
    Soit \( \va \) une variété abélienne plongée dans \( \projd \) et \( \divi
    \) un hyperplan de \( \projd \). Alors \( \va \setminus \divi \) ne possède
    qu'un nombre fini de points entiers.
  \end{coro}

  \begin{block}{Versions quantitatives}
    \begin{itemize}
      \item Cas des courbes elliptiques : Gross-Silverman 1995, Farhi 2003,
        Wagener 2012.
      \item Cas général (avec restrictions) : MPG 2012.
    \end{itemize}
  \end{block}
\end{frame}



\section[Grappes]{Grappes et conditions de décompte}
\tocsect

\begin{frame}{Multiplication des approximations exceptionnelles}
  \begin{prop}
    Soit \( \vai \) une sous-variété abélienne et \( \avar = Z + \vai \).
    On considère \( \eps > 0 \) et \( x \in \apx(\avar, \eps) \cap \grp \).
    De plus, on note \( \hinf(\vai \cap \grp) \) l'infemum de \( \hautn y \)
    lorsque \( y \) parcourt l'ensemble des points d'ordre infini de
    \( \vai(\Qbar) \cap \grp \), ou \( +\infty \) si cet ensemble est vide.

    Pour tout \( 0 < \tau \le 1 \), il existe \( \eps' > 0 \) explicite tel que
    l'ensemble \( \apx(\avar, \eps') \cap \grp \) contient au moins
    \begin{equation}
      \card \bigl( \vai(\Qbar)_\torsion \cap \grp \bigr)
      \left(
        2 \floor*{ \sqrt{ \tau \hautn x / \hinf(\vai \cap \grp) } } + 1
      \right)
    \end{equation}
    points de hauteur normalisée supérieure ou égale à
    \( (1-\sqrt\tau)^{2} \, \hautn x \) (avec la convention \( (+\infty)^{-1} =
      0 \)).
  \end{prop}
\end{frame}

\begin{frame}{Obstruction au décompte explicite}
  \begin{coro}
    Soit \( \vai \) une sous-variété abélienne et \( \avar = Z + \vai \).
    Supposons que pour un certain \( \eps_0 > 0 \) et un certain \( \grp \) de
    type fini, il existe \( R \) et \( N \) tels que
    \begin{equation}
      \card
      \set{
        x \in \apx(\avar, \eps_0) \cap \grp
        \text{ tel que }
        \hautn x \ge R
      }
      \le
      N
      \pmm.
    \end{equation}
    Alors, pour tout \( \eps > \eps_0 \), il existe \( B = B\bigl(\eps_0, R,
      N, \hmin(\vai \cap \grp)\bigr) \) explicite tel que les points de \(
      \apx(\avar, \eps) \cap \grp \) sont tous de hauteur normalisée
    inférieure à \( B \).
  \end{coro}
\end{frame}

\begin{frame}{Conditions pour le décompte}
  Soient \( F \subset \va(\Qbar) \) un sous-ensemble et \( \tau > 0 \) un
  réel.
  \begin{tdef}
    On dit que \( F \) satisfait \( \condi*\tau\eps \) s'il existe deux points
    distincts \( x \) et \( y \) dans \( F \) et \( \vai \) une sous-variété
    abélienne telle que \( x \in \apx(\utrans\avar\vai, \eps) \) avec \( x - y
      \in \vai \) et \( \hautn{ x-y } \le \tau \hautn x \).

    On dit que \( F \) satisfait \( \condi\tau\eps \) si \( F \)
    ne satisfait pas \( \condi*\tau\eps \).
  \end{tdef}

  \begin{tdef}
    On dit que \( F \) satisfait \( \cond*\tau \) s'il existe une
    sous-variété abélienne \( \vai \) de \( \va \) dont un translaté est contenu
    dans \( \avar \) et deux points distincts \( x \) et \( y \) dans \( F \),
    tels que \( x - y \in \vai \) et \( \hautn{ x-y } \le \tau \hautn x \).

    On dit que \( F \) satisfait la condition \( \cond\tau \) si \( F \) ne
    satisfait pas \( \cond*\tau \).
  \end{tdef}
\end{frame}



\section[Résultats]{Énoncés des résultats principaux}
\tocsect

\begin{frame}{Cas d'un translaté de sous-variété abélienne}
  \begin{thm}
  Soient \( \avar = z + \vai \) et une famille \( x_1, \dots, x_p \) de points
  de \( \va(\cdn) \) satisfaisant à \( \cond{\eps/2d} \) et telle que pour
  tout \( i \),
  \begin{overprint}
  \begin{align}
    0 <
    \only<2->{\prod\placerange}
    \distv{x_i} \avar
    \only<2->{^\degv}
    & <
    \only<1>{
      \omega_v
    }
    \hautm2{x_i}^{- \only<1>{\wtapx} \eps}
    \only<1>{\ \forall v}
    \only<-2>{
      \quad\text{et }
      \hautn{x_i} > \omega \Lambda^{(4g)^{4g^2}}
    }
    \only<3>{
      \exp(- \adeg \omega \Lambda^{(4g)^{4g^2}})
    }
  \end{align}
  \end{overprint}
  \begin{overprint}
    \par\vspace*{-1em}
    avec
    \only<1>{%
      \( \sum \wtapx \degv = 1 \) et certains \( \omega_v \) tels que
      \( \prod \omega_v \le \omega \), ainsi que
    }
    \par\vspace*{-3em}
  \end{overprint}
  \begin{align}
    \Lambda
    & =
    34 \eps^{-2} \bigl( 5 (\deg \va) (3 g^2 \adeg)^g \bigr)^{2g+2}
    \\
    \omega
    & =
    \adeg^{g+1} \vacst
    + (g + 1) \deg \va \Bigl(
      \adeg^g \hautl1 \avar
      + 2 \bigl( 2 (\adeg+1) \bigr)^{n+1}
    \Bigr)
  \end{align}
  Alors on a
  \begin{overprint}
    \begin{equation}
      p
      \le
      2 \only<2->{\cdot 5^{\card \placesapx}} \cdot
      \sqrt{ \frac{\adeg}\eps }
      (4g)^{4g^2+1}
      (\log \Lambda)
      \left(
        120760 (\deg \va) \adeg \, \eps^{-1}
      \right)^r
      \pmm.
    \end{equation}
  \end{overprint}
\end{thm}
\end{frame}

\begin{frame}{Cas général}
  \begin{thm}
    Soit une famille \( x_1, \dots, x_p \) de points de \(
      \va(\cdn) \) satisfaisant à \( \cond{
      \frac{ \eps }{ \adeg^M (2M)^{(M+1)\adim} }
    } \) et telle que
    \begin{align}
      0
      & < \prod\placerange \distv{x_i} \avar^\degv
      < \hautm2{x_i}^{-\eps}
      \\
      \hautn{x_i}
      & >
      ( \hautl1 \avar + \vacst )
      \, \eps^{-2(4g)^{4g^2}}
      \adeg^M (3M)^{ (M+1)\adim + 3}
    \end{align}
    pour tout \( i \), avec
    \(
      M
      =
      \bigl(
      2^{34} \, \vacst' \, \adeg
      \bigr)^{ (r+1) g^{ 5(\adim + 1)^2 } }
      + 1
    \).
    Alors on a
    \begin{equation}
      p
      \le
      5^{\card\placesapx}
      M^2 \Bigl( \adeg^{M} (3M)^{(M+1)\adim} \Bigr)^{(r+1)/2}
      \, \eps^{-r - 1/2} \log(\expb/\eps)
      \pmm.
    \end{equation}
  \end{thm}
\end{frame}



\section[Stratégie]{Stratégie générale}
\tocsect

\begin{frame}{Idée rayonnante de Vojta}
  \begin{itemize}
    \item Séparation en petits points et grands points.
    \item Séparation des grands points en cônes.
    \item Inégalité de Vojta pour la finitude.
    \item Inégalités de Vojta et Mumford explicites pour le décompte.
  \end{itemize}
  \begin{block}{Inégalité de \dots}
    Il n'existe pas de famille d'approximations exceptionnelles \( (x_1,
      \dots, x_m) \), pour \( m \) assez grand, telle que
    \begin{columns}
      \column{.4\linewidth}
      \begin{block}{Vojta}
        \begin{itemize}
          \item \( \hautn{x_1} \ge \Vbig \)
          \item \( \cos(x_i, x_j) \ge 1 - \Vcos \)
          \item \( \hautn{x_i} \ge \Vfar \hautn{x_{i-1}} \)
        \end{itemize}
      \end{block}
      \column{.5\linewidth}
      \begin{block}{Mumford}
        \begin{itemize}
          \item \( \hautn{x_1} \ge \Mbig \)
          \item \( \cos(x_i, x_j) \ge 1 - \Mcos \)
          \item \( \hautn{x_1} \le \hautn{x_i} \le \Mfar \hautn{x_1} \)
        \end{itemize}
      \end{block}
    \end{columns}
  \end{block}
\end{frame}



\section[Inégalités]{Inégalités de Vojta et Mumford}
\tocsect

\begin{frame}{Inégalité de Vojta}
  \begin{thm}
    Il n'existe dans \( \va(\Qbar) \) aucune famille de points \( x_1, \dots,
      x_m \) avec \( m \ge g + 1 \) satisfaisant simultanément aux conditions
    suivantes :
    \begin{align}
      0 < \distv{x_i} \avar
      & <
      \omega_v^{-1}
      \hautm2{x_i}^{-\wtapx \eps}
      \quad \forall v \in \placesapx
      \\
      \hautn{x_1}
      & > \omega \Lambda_4^{(2mg)^{mg}}
      \\
      \hautn{x_i} & > \hautn{x_{i-1}}
      \cdot \Lambda_4^{(2mg)^{mg}}
      \\
      \cos(x_i, x_j) & > 1 - ( m \, \eta )^{-1}
    \end{align}
    avec certains \( \omega_v \) tels que \( \prod \omega_v \le \omega \) et
    \begin{align*}
      \eta
      & =
      \bigl( 86 \nclmaps \cdot 5^g \adeg \eps^{-1} \bigr)^{ \frac{m}{m-g} }
      \hspace*{4em}
      \Lambda_4
      =
      \eta
      \bigl( (\sqrt2 m g\adeg)^g \deg \va \bigr)^m
      \\
      \omega
      & =
      \adeg \max \bigl(
      \adeg^g \hautl1{\va}, \hlclab, \htcmp
      \bigr)
    + (g + 1) \deg \va \Bigl(
      \adeg^g \hautl1 \avar
      + \bigl( 4 \adeg \bigr)^{n+2}
    \Bigr)
    \end{align*}
  \end{thm}
\end{frame}

\begin{frame}{Inégalité de Mumford \\
    Cas des translatés de sous-variété abélienne}
  \begin{thm}
    Soient \( \phi > 0 \) et \( \rho > 0 \) tels que
    \begin{equation}
      \frac{ \rho^2 }{ 4 } + \rho\phi + 2\phi
      \le
      \frac{ \eps }{ 2\adeg }
      \pmm.
    \end{equation}
    Si \( x \) et \( y \) sont deux points de \( \va(\Qbar) \) tels que
    \begin{align*}
      0
      & <
      \distv z \avar
      \le
      \hautm2 z ^{-\wtapx\eps}
      \quad \forall v \in \placesapx
      \quad \text{où \( z \) est \( x \) ou \( y \)}
      \\
      \hautn x
      & >
      \frac2\eps
      \adeg (\adim + 1)
      \Bigl(
      \log(\adeg)
      + (2 + \frac\eps\adeg) \htcmp
      + 2 \hlpm
      + 11 \log(n+1)
      \Bigr)
      \quad
      \\
      \cos(x, y)
      & \ge
      1 - \phi
      \\
      \hautn x
      & \le
      \hautn y \le (1+\rho) \hautn x
    \end{align*}
    alors \( x - y \in \vai(\Qbar) \).
  \end{thm}
\end{frame}

\begin{frame}{Inégalité de Mumford --- Cas général}
  \begin{thm}
    Soient \( \phi > 0 \) et \( \rho > 0 \), on note \( \tau =
      \rho^2 / 4 + \rho\phi + 2\phi \) et on suppose
    \vspace{-1em}
    \begin{equation}
      \tau
      \le
      \frac{ \eps }{ \adeg^m (2m)^{(m+1)\adim} }
      \quad\text{avec }
      m
      =
      \bigl(
      2^{34} \, \vahtr \, \adeg
      \bigr)^{ (r+1) g^{ 5(\adim + 1)^2 } }
      + 1
      \pmm.
    \end{equation}
    Si \( x_1, \dots, x_m \) est une famille de points de \( \grp \)
    telle que
    \begin{align*}
      0
      & <
      \distv{ x_i }\avar
      \le
      \hautm2{ x_i }^{-\wtapx\eps}
      \quad \forall v \in \placesapx
      \\
      \hautn{ x_m }
      & \ge
      \frac4\eps
      \adeg^{m-1} (2m)^{ (m+1)\adim + 1}
      \Bigl(
      \hautl1 \avar
      + 4 \adeg m \bigl( \log \adeg + \hlpm + \htcmp \bigr)
      \Bigr)
      \\
      \cos(x_i, x_j)
      & \ge
      1 - \phi
      \\
      \hautn{ x_m }
      & \le
      \hautn{ x_i }
      \le
      (1+\rho) \hautn{ x_m }
    \end{align*}
    alors \( \set{x_1, \dots, x_m} \) satisfait \( \cond*\tau \).
  \end{thm}
\end{frame}



\section[Décomptes]{Résultats de décompte}
\tocsect

\begin{frame}{Par cône tronqué}
  \begin{lem}
    Soit \( x_1, \dots, x_p \) une famille de points de \( \grp \)
    satisfaisant \( \cond\tau \) et
    \begin{align*}
      0
      <
      \distv{ x_i }\avar
      & \le
      \hautm2{ x_i }^{-\wtapx\eps}
      \quad \forall v \in \placesapx
      \\
      \hautn{ x_i }
      & \ge
      \Bbig
      \\
      \cos(x_i, x_j)
      & \ge
      1 - \Bcos
    \end{align*}
    avec \( \Bbig = \max(\Vbig, \Mbig) \) et \( \Bcos = \min(\Vcos, \Bcos) \).
    Alors on a
    \begin{equation}
      p
      \le
      (m-1) \ceil*{ \frac{ \ln \Vfar }{ \ln \Mfar } }
      \pmm.
    \end{equation}
  \end{lem}
\end{frame}

\begin{frame}{Tous les grands points}
  \begin{fact} \label{f:nb-cones}
    Soient \( r \) un entier et \( \Bcos > 0 \) un réel. On peut recouvrir \(
      \R^r \) par \( \floor{(1 + \sqrt{8/\Bcos})^r} \) ensembles dans chacun
    desquels deux points quelconques satisfont \( \cos(x, y) \ge 1 - \Bcos \).
  \end{fact}
  \begin{lem}
    Soit \( x_1, \dots, x_p \) une famille de points de \( \grp \)
    satisfaisant \( \cond\tau \) et
    \begin{align*}
      0
      <
      \distv{ x_i }\avar
      & \le
      \hautm2{ x_i }^{-\wtapx\eps}
      \quad \forall v \in \placesapx
      \\
      \hautn{ x_i }
      & \ge
      \Bbig
    \end{align*}
    Alors on a
    \begin{equation}
      p
      \le
      (m-1) \ceil*{ \frac{ \ln \Vfar }{ \ln \Mfar } }
      (1 + \sqrt{8/\Bcos})^r
      \pmm.
    \end{equation}
  \end{lem}
\end{frame}

\begin{frame}{Décompte trivial des petits points}
  \begin{fact}
    Soient \( E \) un espace euclidien de dimension \( r \) et deux réels \(
      \rho \) et \( \mu \). On peut recouvrir toute boule (fermée) de rayon \(
      \rho \) par des boules (ouvertes) de rayon \( \mu \) en nombre inférieur
    à \( ( 2 \, \frac\rho\mu + 1 )^r \).
  \end{fact}
  \begin{coro}
    Soit \( \grp \) un sous-groupe de type fini de \( \va(\Qbar) \) ; on note
    \( r \) le rang de \( \grp \) et \( \hmin(\grp) \) le minimum de \(
      \hautn x \) quand \( x \) parcourt l'ensemble des points d'ordre infini de
    \( \grp \).  Pour tout réel positif \( R \), on a
    \begin{equation}
      \card \set*{
        x \in \grp
        \text{ tel que }
        \hautn x \le R
      }
      \le
      \card \grp_\torsion
      \left( 1 + 2\sqrt{R / \hmin(\grp)} \right)^r
    \end{equation}
    où \( \grp_\torsion \) désigne l'ensemble des points de torsion de \( \grp
    \).
  \end{coro}
\end{frame}

\begin{frame}{Inégalité de Liouville}
  \begin{prop} \label{p:liouville}
    Pour tout point \( x \in \projd(\Qbar) \), on a soit \( x \in \avar(\Qbar)
    \) soit
    \begin{equation}
      \prod_{v \in \placesapx} \distv x \avar ^\degv
      \ge
      \frac1{
        (n+1)^{3/2}
        (3\adeg)^{\adeg (\adim+1)}
        \, \hautm1 \avar
        \, \hautm2 x ^\adeg
      }
      \pmm.
    \end{equation}
  \end{prop}
  \begin{coro} \label{c:kill-small}
    Si \( R \) est un réel supérieur ou égal à
    \begin{equation}
      \frac1\eps \left(
        \hautl1{ \chow \avar }
        + \adeg (\adim+1) \log(3\adeg)
        + \frac32 \log(n+1)
        + \adeg \htcmp
      \right)
      \pmm,
    \end{equation}
    il n'existe aucun point \( x \in \va(\Qbar) \) tel que
    \begin{equation}
      0
      <
      \prod\placerange
      \distv x \avar ^\degv
      \le
      \expb^{- \adeg R}
      \hautm2{x}^{-\eps}
      \quad\text{et}\quad
      \hautn x \le R
      \pmm.
    \end{equation}
  \end{coro}
\end{frame}

\end{document}
