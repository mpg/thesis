\documentclass{mpg-thslides}

\author[MPG]{Manuel Pégourié-Gonnard}
\institute[UPMC]{Université Pierre et Marie Curie}
\title{Approximation diophantienne \\ dans les variétés abéliennes}
\date{13 septembre 2012 \\ GÉPBD}

\begin{document}

\maketitle

\frame{\tableofcontents}


\section[Intro]{Introduction : rappels historiques}
\frame{\tableofcontents[currentsection]}

\begin{frame}{Théorème de Roth}
  \begin{thm}
    Soient \( \xi \in \R \) un nombre algébrique et \( \eps > 0 \) un réel.
    Il n'existe qu'un nombre fini d'entiers \( p \) et \( q \) tels que
    \begin{equation}
      \abs[\Big]{\frac pq - \xi}
      <
      \abs{q}^{-2-\eps}
      \pmm.
    \end{equation}
  \end{thm}
\end{frame}

\begin{frame}{Théorème de Roth étendu}
  \begin{thm}[Ridout]
    Soit \( \xi \in \Qbar \) un nombre algébrique. Soient par ailleurs \( \cdn
    \) un corps de nombres et \( \placesapx \) un ensemble fini de places de
    \( \cdn \), étendues de façon arbitraire à \( \cdn(\xi) \). Pour tout \(
      \eps > 0 \), il n'existe qu'un nombre fini de points \( x \in \cdn \)
    tels que
    \begin{equation}
      \prod\placerange \av{x - \xi}^\degv
      <
      \hautm2 x ^{-2-\eps}
    \end{equation}
    où \( \degv = [\cdn(\xi)_v : \Q_v] / [\cdn(\xi) : \Q] \) et chaque valeur
    absolue est normalisée de façon à prolonger une des valeurs absolues
    usuelles de \( \Q \).
  \end{thm}
  % Version quantitative ???
\end{frame}

\begin{frame}{Théorème du sous-espace}
  \begin{thm}[Schmidt, Schlickewei]
    Soient \( n \) un entier et \( L_0, \dots, L_n \in \Qbar[X_0, \dots, X_n]
    \) des formes linéaires indépendantes.  Soient par ailleurs \( \cdn \) un
    corps de nombres et \( \placesapx \) un ensemble fini de places de \( \cdn
    \), étendues de façon arbitraire à \( \cdn(L_0, \dots, L_n) \). Pour tout
    \( \eps > 0 \), l'ensemble des points \( x \in \projd(\cdn) \) tels que
    \begin{equation}
      \prod\placerange
      \prod_{i=0}^n
      \left( \frac{ \av{L_i(x)} }{ \nv{}{x} } \right)^\degv
      <
      \hautm2 x ^{-n-1-\eps}
    \end{equation}
    est contenue dans une union finie de sous-variété linéaires strictes de \(
      \projd \).
  \end{thm}
  % Version quantitative : Sclickewei ???
\end{frame}

\begin{frame}{Ex-conjecture de Mordell}
  \begin{thm}[Faltings 1983]
    Soient \( C \) une courbe projective de genre \( g \le 2 \) et \( \cdn \)
    un corps de nombres. Alors \( C(\cdn) \) est fini.
  \end{thm}
  \begin{itemize}
    \item Autre démonstration indépendante : Vojta 1991.
    \item Versions quantitatives : Rémond 2002, Farhi 2003.
  \end{itemize}
\end{frame}

\begin{frame}{Ex-conjecture de Mordell-Lang}
  \begin{thm}[Faltings 1991, théorème I]
    Soit \( \avar \) une sous-variété d'une variété abélienne \( \va \)
    définie sur un corps de nombres \( \cdn \).
    Si \( \avar \) ne contient pas de translaté de sous-variété abélienne non
    nulle, alors \( \avar(\cdn) \) est fini.
  \end{thm}
  \begin{itemize}
    \item Version quantitative : Rémond 2002.
  \end{itemize}
\end{frame}

\begin{frame}{Approximation sur les variété abéliennes}
  \begin{thm}[Faltings 1991, théorème I]
    Soit \( \avar \) une sous-variété quelconque de \( \va \), \( v \)
    une place de \( \cdn \), et \( \eps > 0 \). Il n'existe qu'un nombre fini de
    points \( x \) dans \( \va(\cdn) \) tels que
    \begin{equation} \label{e:has}
      0
      <
      \distv x \avar ^\degv
      \le
      \hautm{} x ^{-\eps}
      \pmm.
    \end{equation}
  \end{thm}
  \begin{itemize}
    \item Version quantitative (partielle) : MPG 2012
  \end{itemize}
\end{frame}


\section[Grappes]{Grappes et conditions de décompte}
\frame{\tableofcontents[currentsection]}


\section[Résultats]{Énoncé des résultats principaux}
\frame{\tableofcontents[currentsection]}


\section[Stratégie]{Stratégie générale}
\frame{\tableofcontents[currentsection]}


\section[Inégalités]{Inégalités de Vojta et Mumford}
\frame{\tableofcontents[currentsection]}


\section[Décomptes]{Résultats de décompte}
\frame{\tableofcontents[currentsection]}

\end{document}
