% !TEX root = main.tex

\chapter{Normes et mesures archimédiennes}


\section{Notations}

On note $X = (X_1, \dots, X_n)$ un groupe de variables. Pour $i \in \N^n$ un
multiindice, on notera $X^i = \prod_i X_j^{i_j}$. De même, notera
$\frac{\partial^i}{\partial X^i} = \frac{\partial^{i_1}}{\partial
  X_1^{i_1}}\cdots\frac{\partial^{i_n}}{\partial X_n^{i_n}}$. On portera notre
attention sur les dérivations $\der{i} = \frac1{i!}\frac{\partial^i}{\partial
  X^i}$, où l'on a noté $i! = i_1! \cdots i_n!$. On introduit par ailleurs les
notations $\lgr i = i_1 + \dots + i_n$ et, pour $d = \lgr i$, on note
$\binom{d}{i} = \frac{d!}{i!}$ le symbole multinomial. On notera à l'occasion
$d_i$ le degré partiel de $P$ en la variable $X_i$.

Introduisons maintenant les notations utiles dans le cas multihomogène. On
regardera des polynômes homogènes de multidegré $\delta = (\delta_1, \dots,
\delta_l)$ en $l$ groupes de $n_k$ variables $U_0^{(k)}, \dots,
U_{n_l}^{(k)}$, pour $1\le k \le l$. À un multiindice $\alpha =
(\alpha^{(1)}, \dots, \alpha^{(l)}) \in \N^{n_1} \times \dots \times
\N^{n_l}$, on associe son vecteur longueur $\vlg\alpha =
(\lgr{\alpha_1},\dots, \lgr{\alpha_l})$, et on notera ainsi $F =
\sum_{\vlg\alpha = \delta} f_\alpha X^\alpha$, avec la notation évidente pour
$X^\alpha$. On définit également $\der\alpha$, $\alpha!$ et
$\binom{\delta}{\alpha}$ comme dans le cas précédent, dont le cas
multihomogène est bien sûr un cas particulier (avec $n = \sum_k (n_k + 1)$,
$d = \lgr\delta$, et $\alpha$ correspondant naturellement à un $i \in \N^n$).
Toutefois, il faudra prêter attention au fait que $\delta = \vlg\alpha$ est
maintenant une égalité de vecteurs.

On rappelle les majorations commodes $\binom{d + n}{n} \le (d + 1)^{n+1}$ et
symétriquement $\binom{d + n}{n} \le (n + 1)^{d+1}$. On notera $m(\delta) =
\max_{\lgr\alpha = \delta} {\textstyle \binom\delta\alpha} $, qui est majorée
par $\prod\nolimits_k (n_k + 1)^{\delta_k}$. On introduit également la
constante (liée à la hauteur arakelovienne de l'espace projectif) $\gamma_n =
\frac12 \sum_{q=1}^{n} \frac1q$, dont on rappelle la majoration évidente
$\gamma_n \le \frac12 (\log(n) + 1)$.


\cleardoublepage
\endinput

% vim: spell spelllang=fr
