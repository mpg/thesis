% !TEX root = main

\section{Estimation de dérivées} \label{sec:vojta-param}

Le but de cette section est de fournir des représentations des dérivées de
fonctions rationnelles sur \( \var \) par des fraction rationnelles de
dénominateur explicite dont on contrôle le degré et la hauteur du numérateur.
Pour commencer, nous effectuons ce travail sur une variété projective
quelconque (plongée de façon adaptée) et certaines fonctions rationnelles
seulement, avant de passer à une variété produit et de généraliser à toutes
les fonctions rationnelles.


\subsection{Sur une variété projective}

Soit \( \anyvar \) une variété projective de dimension \( \anydim \), plongée
de façon adaptée dans un espace projectif \( \projd \), de degré \( \anydeg \)
dans ce plongement. Alors \( \cdn(\anyvar) \) est une extension finie de
\begin{equation}
  \cdn\Big(
    \frac{ \vp[0]           }{ \vp[\anydim] }, \dots,
    \frac{ \vp[\anydim-1]   }{ \vp[\anydim] }
  \Big)
\end{equation}
dont \( \frac{ \vp[\anydim+1] }{ \vp[\anydim] } \) est un élément primitif.
Sur ce dernier corps, on dispose des dérivations standard définies par
\(
  \diff_\ind \frac{ \vp[\indi] }{ \vp[\anydim] } = \delta_\ind^\indi
\)
qui forment une base de l'espace des dérivations, et s'étendent de façon
unique à \( \cdn(\anyvar) \) pour former une base de son espace de
dérivations. Les propriétés classiques des dérivations montrent alors que
l'application
\begin{equation}
  \begin{aligned}
    \pmor \colon \cdn(Z)
    & \to \cdn(Z)\series\psp
    \\
    f
    & \mapsto
    \sum_{\derp \in \N^\anydim} \der[\derp] \!f \, \psp^\derp
  \end{aligned}
  \qquad \text{où }
  \der[\derp]
  =
  \frac1{\derp!}
  \prod_{\ind = 0}^\anydim \diff_\ind^{\derp*}
  \pmm.
\end{equation}
est un morphisme de \( \cdn \)-algèbres.

Il est intéressant de pouvoir le représenter par un morphisme \( \pmor* \)
faisant commuter le diagramme
\begin{equation} \label{pmor*}
  \xymatrix{
    \cdn[X]_{(\ideal\anyvar)}                 \ar[d]^\pi \ar@{.>}[r]^{\pmor*}
    & \cdn[X]_{(\ideal\anyvar)} \series\psp   \ar[d]^\pi
    \\ \cdn(\anyvar)                                           \ar[r]^{\pmor}
    & \cdn(\anyvar)\series\psp
  }
\end{equation}
où les flèches verticales sont les projections canoniques. Posons
\( \pmor* = \sum_\derp \pdiff*^\derp \psp^\derp \) où les \( \pdiff*^\derp \)
sont des applications linéaires sur \( \cdn[X]_{(\ideal\anyvar)} \) vérifiant
la règle de \bsc{Leibniz} pour les dérivées divisées. Pour chaque fraction
rationnelle \( f = F/G \), nous exhiberons (au moins) un polynôme \( R_\derp
\), dépendant éventuellement de la représentation \( F/G \) choisie, tel que
\( R \cdot \pdiff*^\derp(f) \) soit un polynôme de degré et normes locales
contrôlées.

En pratique, nous construirons en fait deux morphismes, \( \pmor*_0 \) et \(
  \pmor*_1 \), et deux familles d'applications \( \pdiff^\derp \) telles que
l'on contrôlera la norme \( \place \)-adique de \( R \cdot \pdiff^\derp \), où
l'on rappelle que \( \dv \) vaut \( 1 \) si \( \place \) est archimédienne et
\( 0 \) sinon.

Pour définir \( \pmor*_\dv \), il suffit de définir les images des différents
\( \vp* / \vp[\anydim] \). En effet, ceci donne un morphisme de \( \cdn[ \vp /
  \vp[\anydim] ] \) dans \( \cdn[X]_{(\ideal\anyvar)} \series\psp \). Si l'on
impose de plus que \( \pdiff^0 \) soit l'identité, on constate que l'image du
complémentaire de \( \ideal\anyvar \) ne contient que des séries inversibles
(car leur terme constant l'est), ce qui permet d'étendre le morphisme à \(
  \cdn[X]_{(\ideal\anyvar)} \).

Pour tout \( \ind \in \set{0, \dots, \anydim-1} \), on peut poser \(
  \pmor*_\dv(\vp*/\vp[\anydim]) = \vp*/\vp[\anydim] + \psp* \). Pour \( \ind >
  \anydim \), nous utiliserons le lemme suivant, que l'on énonce dans un cadre
affine.

\begin{lem} \label{l:param-aff}
  Soient \( \anyvp[1], \dots, \anyvp[\anydim], Y \) des variables et \( L \)
  une algébrique extension finie de \( \cdn(\anyvp[1], \dots, \anyvp[\anydim])
  \). On fixe \( y \) un élément de \( L \) ; on note \( \pi \) le
  morphisme de \( \cdn[\anyvp[1], \dots, \anyvp[\anydim], Y ] \) dans \( L \)
  qui laisse stable les \( \anyvp* \) et envoie \( Y \) sur \( y \), et \( \Pi
  \) un générateur de son noyau, c'est-à-dire un polynôme minimal de \( y \).

  On considère les dérivations standard \( \diff_\ind \) sur \( \cdn[
    \anyvp[1], \dots, \anyvp[\anydim] ] \) ainsi que leurs extensions à \( L
  \), et \( \der[\derp] = \frac1{\derp!} \prod_{\ind = 0}^\anydim
    \diff_\ind^{\derp*} \).  Il existe des polynômes \( P_\dv^\derp \), pour
  \( \derp \in \N^{\anydim} \minusset0 \), tels que :
  \begin{enumthm}
    \item \( \der[\derp] y
        = \pi\left(
          \frac{ P_\dv^\derp }{ \pden^{2\lgr\kappa - 1} }
        \right)
      \) ;
    \item \( \deg P_\dv^\derp \le (\deg \Pi - 1) (2\lgr\derp - 1) \) ;
    \item \( \nv1{ P_\dv^\derp }
        \le \nv1 \poldep ^{2\lgr\derp - 1}
        \cdot \left(
          (4\anydim)^{\lgr\derp -1} (\deg\Pi)^{3\lgr\derp -2}
        \right)^\dv \) ;
  \end{enumthm}
  où \( \pden = \diff_Y \Pi\) est la dérivée de \( \Pi \) par rapport à la
  dernière variable.
\end{lem}

\begin{proof}
  Il s'agit en fait de compléter\footnote{On contrôle en fait le développement
    autour d'un point générique, alors que \bsc{Rémond} l'étudie en un point
    fixé.} la preuve du lemme~6.1 de \cite{remivds}, en
  utilisant aux places archimédiennes une généralisation de
  \cite[relation~2.3.1, p.~63]{farhith}.

  On va construire $P_0^\derp$ et $P_1^\derp$ indépendamment par récurrence
  sur la longueur de $\derp$, en partant à chaque fois de $P_\dv^\derp = -
  \diff_{\ind_0} \Pi$ quand $\derp_{\ind_0} = 1$ et $\derp_\ind = 0$ sinon
  (cas $\lgr\derp = 1$), car ce choix convient. Pour la suite, on fixe un
  $\dv$, un $\derp$ de longueur au moins $2$, et on suppose qu'on a choisi un
  $P_\dv^{\derp'}$ convenable pour chaque $\derp'$ de longueur strictement
  inférieure à celle de $\derp$.

  On commence par le cas ultramétrique et on note donc provisoirement $P^\derp
  = P_0^\derp$ pour alléger. Les polynômes recherchés sont caractérisés par la
  relation
  \begin{equation}
    \Pi \left(
      \anyvp[1] + \psp[1], \dots, \anyvp[\anydim] + \psp[\anydim],
      \anyvp[\anydim+1] + \sum_{ \derp \in \N^\anydim \minusset 0 }
      \frac {P^\derp} {\pden^{2\lgr\derp -1}} \psp^\derp
    \right)
    = 0 \mod (\Pi)
    \pmm.
  \end{equation}
  On remplace alors \( \Pi \) par son développement de \bsc{Taylor}, pour
  obtenir les égalités suivantes modulo \( \Pi \) :
  \begin{align}
    0
    & =
    \sum_{(\ip, \mu) \in \N^{\anydim+1}}
    \der[\ip, \mu] \Pi
    \cdot \psp^\ip
    \cdot \left(
      \sum_{ \derp \in \N^\anydim \minusset 0 }
      \frac {P^\derp} {\pden^{2\lgr\derp - 1}} \psp^\derp
    \right)^\mu
    \\
    & =
    \sum_{\substack{ (\ip, \mu) \in \N^{\anydim+1} \minusset{(0, 0)}
        \\ \gmp\nu* \in \N^\anydim \minusset 0 }}
    \left(
      \der[\ip, \mu] \Pi
      \cdot \prod_{\fct = 1}^\mu
      \frac {P^{\gmp\nu*}} {\pden^{2\lgr{\gmp\nu*} - 1}}
    \right)
    \psp^{\sum_\fct \gmp\nu* + \ip}
    \\
    & =
    \sum_{\derp \in \N^\anydim \minusset 0}
    \Biggl(
    \frac {P^\derp} {\pden^{2\lgr\derp - 2}}
    + \sum_{\substack{
        (\ip, \mu) \in \N^{\anydim+1} \minusset{(0, 0), (0, 1)}
        \\ \gmp\nu* \in \N^\anydim \minusset 0
        \\ \sum_\fct \gmp\nu* + \ip = \derp }}
    \der[\ip, \mu] \Pi
    \cdot \prod_{\fct = 1}^\mu
    \frac {P^{\gmp\nu*}} {\pden^{2\lgr{\gmp\nu*} - 1}}
    \Biggr)
    \psp^\derp
    \pmm,
  \end{align}
  où l'on a noté \( (\ip, \mu) = (\ip[1], \dots, \ip[\anydim], \mu) \).
  Il suffit donc de définir \( P^\derp \) par la relation de récurrence
  \begin{equation}
    - P^\derp
    =
    \sum_{\substack{
        (\ip, \mu) \in \N^{\anydim+1} \minusset{(0, 0), (0, 1)}
        \\ \gmp\nu* \in \N^\anydim \minusset 0
        \\ \sum_\fct \gmp\nu* + \ip = \derp }}
    \der[\ip, \mu] \Pi
    \cdot \pden^{2\lgr\derp - 2}
    \cdot \prod_{\fct = 1}^\mu
    \frac {P^{\gmp\nu*}} {\pden^{2\lgr{\gmp\nu*} - 1}}
  \end{equation}
  qui consiste à imposer que chaque terme de la série précédente soit nul, ce
  qui assure bien sa nullité modulo \( \Pi \).

  On majore alors le degré de \( P^\derp \) par récurrence :
  \begin{align}
    \deg P^\derp
    & \le
    \deg\Pi - \lgr\ip - \mu + (\deg\Pi - 1) (2\lgr\derp - 2)
    \\ & \le
    1 - \lgr\ip - \mu + (\deg\Pi - 1) (2\lgr\derp - 1)
    \\ & \le
    (\deg\Pi - 1) (2\lgr\derp - 1)
    \pmm,
  \end{align}
  car \( \ip \) et \( \mu \) ne sont pas simultanément nuls.

  La majoration de norme locale est par ailleurs immédiate par analogie avec
  le degré vu les propriétés de la norme aux places ultramétriques.

  Traitons maintenant le cas archimédien (désormais \( P^\derp = P^\derp_1 \)
  pour alléger). On utilise la relation de récurrence suivante, établie dans
  la démonstration de \cite[lemme~6.1]{remivds}, avec \( Q_\derp =
    P^\derp \cdot \derp! \) et où, rappelons-le, \( \derp' \) est tel que
  \( \derp[\ind_0] = \derp[\ind_0]' + 1 \) et \( \derp* = \derp*' \) sinon :
  \begin{equation}
    Q_\derp
    =
    \pden^2 \, \diff_{\ind_0} Q_{\derp'}
    - \pden \, \diff_{\ind_0} P \, \diff_{\anydim+1} Q_{\derp'}
    + (2\lgr{\derp'} - 1)
    (\diff_{\ind_0} P \, \diff_Y \pden - \pden \diff_{\ind_0} \pden)
    Q_{\derp'}
    \pmm.
  \end{equation}
  On en déduit immédiatement l'estimation de degré suivante :
  \begin{equation}
    \deg P^\derp
    \le 2 (\deg\Pi - 1) + \deg P^{\derp'}
    \le (\deg\Pi - 1) (2\lgr\derp - 1)
    \pmm.
  \end{equation}
  Pour la norme, on prouve que \(
    \nv1{Q_\derp}
    \le
    \nv1\Pi^{2\lgr\derp-1} 4^{\lgr\derp-1} (\deg\Pi)^{3\lgr\derp-2}
    (\lgr\derp - 1) !
  \)
  (ce qui implique le résultat annoncé vu que \( \binom{\lgr\derp}{\derp}
    \le \anydim^{\lgr\derp-1} \)) en exploitant la majoration de degré
  sous la forme \( \deg P^\derp \le 2\lgr\derp \deg\Pi \).
  \begin{align}
    \nv1{ Q_\derp }
    & \le
    2(\deg\Pi)^2 \cdot \deg Q_{\derp'} \cdot \nv1\Pi^2 \nv1{ Q_{\derp'} }
    + 2 (2\lgr{\derp'} - 1) (\deg\Pi)^3 \nv1\Pi^2
    \\ & \le
    \nv1\Pi^2 \nv1{ Q_{\derp'} } \cdot 4 (\deg\Pi)^{3\lgr{\derp'}}
    \\ & \le
    \nv1\Pi^{2\lgr\derp-1} \cdot 4^{\lgr\derp-1} (\deg\Pi)^{3\lgr\derp-2}
    (\lgr\derp - 1) !
    \qedhere
  \end{align}
\end{proof}

Nous sommes maintenant en mesure de définir les morphismes \( \pmor*_\dv \).

\begin{lem}
  Dans les notations précédentes, on peut définir \( \pmor*_\dv \) faisant
  commuter le diagramme~\ref{pmor*} par :
  \begin{enumthm}
    \item \( \pmor*(\frac{ \vp* }{ \vp[\anydim] })
        = \frac{ \vp* }{ \vp[\anydim] } + \psp* \)
      pour tout \( \ind \in \set{0, \dots, \anydim-1} \) ;
    \item \( \pmor*(\frac{ \vp* }{ \vp[\anydim] })
        = \frac{ \vp* }{ \vp[\anydim] }
        + \sum_{\derp \neq 0}
        \frac{ P^\derp_{\ind, \dv} }{ U_\ind^{\lgr\derp - 1} }
      \)
  \end{enumthm}
  où les polynômes \( P^\derp_{\ind, \dv} \) et \(  U_\ind^{\lgr\derp - 1} \)
  satisfont
  \begin{enumthm}
    \item \( \deg U_\ind^{\lgr\derp - 1} = \anydeg - 1 \) ;
    \item \( \nv1{ U_\ind^{\lgr\derp - 1} }
        \le \nv1{ \chow\anyvar } \bigl( 2^{\anydeg-1} \bigr)^\dv
      \)
    \item \( \deg P^\derp_{\ind, \dv} = (\anydeg - 1) (2\lgr\derp - 1) \) ;
    \item \( \nv1{ P^\derp_{\ind, \dv} }
      \)
    \item \( \nv1{ \pmnum^\derp_{1, \dv} (\anyvp[\indi]) }
        \le \left(
          \nv1{ \chow\anyvar }
          \bigl( 2^\anydeg \sqrt{\anydim \anydeg^{3}} \bigr)^\dv
        \right)^{2\lgr\derp - 1} \).
  \end{enumthm}
\end{lem}

\begin{proof}
  On aimerait appliquer le lemme précédent à chaque \( \vp* / \vp[\anydim] \)
  pour \( \ind \in \set{\anydim + 1, \dots, \dimp} \) avec les polynômes \(
    P_\ind \) donnés par le fait~\ref{f:plong-adapt-dep}, mais il n'est
  pas clair que ces polynômes satisfassent la dernière condition du lemme
  précédent. Cependant, une de leur dérivées divisées successives par rapport
  à la dernière variable convient nécessairement. On applique alors le lemme
  précédent à cette dérivée, ce qui permet de définir \( \pmor*_\dv \) et
  donne immédiatement les estimations de normes ci-dessus, mais ne donne
  qu'une majoration pour les degrés. On multiplie alors les \( U_\ind \) et \(
    P^\derp_{\ind, \dv} \) obtenus par une puissance convenable de \(
    \vp[\anydim] \), ce qui permet d'obtenir les degrés exacts annoncés sans
  modifier les normes.
\end{proof}

Nous allons maintenant étudier les images de certaines fraction rationnelles.

\begin{lem} \label{l:par-anyvar-mono}
  Soit \( G \) une forme homogène de degré \( \alpha \). Posons \( \pden =
    \prod_{\ind = \anydim + 1}^{\dimp} \pden_\ind \) où les \( \pden_\ind \)
  sont donnés par le lemme précédent. Alors :
  \begin{enumthm}
    \item \( P^\derp_{G, \dv}
        = \pdiff^\derp(G / \vp[\anydim]^\alpha)
        \cdot \vp[\anydim]^\alpha \pden^{2\lgr\kappa} \) est un polynôme ;
    \item \( \deg P^\derp_{G, \dv}
        = \alpha + 2 (\anydeg - 1) (\dimp - \anydim) \lgr\derp \) ;
    \item \( \nv1{ P^\derp_{G, \dv} }
        \le
        \nv1{ G }
        \nv1{ \chow\anyvar }^{(\dimp - \anydim) \lgr\derp}
        \left(
          \bigl( 8^\Delta \anydim \anydeg^{3} \bigr) ^{
            (\dimp - \anydim) \lgr\derp }
          \cdot 2^{ \anydim \alpha }
        \right)^\dv
      \) ;
  \end{enumthm}
  De plus, \( \ord_{\vp[0]} \bigl( P^\derp_{G, \dv} \bigr)
    \ge \ord_{\vp[0]} (G)  - \derp[0] \).
\end{lem}

% XXX

\begin{proof}
  On utilise le résultat du lemme précédent et on se souvient que \( \pmor* \)
  est un morphisme d'anneaux :
  \newcommand \indl {{ \gmp\nu[\ind, \indi_\ind] }}
  \begin{align}
    \pmor*( \anyvp^\ip )
    & =
    \prod\indrange \pmor*( \anyvp* )^\ip*
    =
    \prod\indrange \left(
      \sum_{\derp \in \N^\anydim}
      \frac {
        \pmnum^\derp_{1, \dv}( \anyvp* )
      }{
        \pmden_1^\derp    ( \anyvp* )
      }
      \psp^\derp
    \right) ^\ip*
    \\ & =
    \prod\indrange
    \prod_{ \indi_\ind = 1 }^{ \ip* }
    \sum _{ \indl \in \N^\anydim }
    \frac {
      \pmnum^\indl_{1, \dv}( \anyvp* )
    }{
      \pmden_1^\indl    ( \anyvp* )
    }
    \psp^\indl
    \\ & =
    \sum_{\derp \in \N^\anydim}
    \left(
      \sum_{\gmp\nu \in N}
      \prod\indrange
      \prod_{ \indi_\ind = 1 }^{ \ip* }
      \frac {
        \pmnum^\indl_{1, \dv}( \anyvp* )
      }{
        \pmden_1^\indl    ( \anyvp* )
      }
    \right)
    \psp^\derp
    \pmm,
    \label{e:par-mono-dev}
  \end{align}
  où la somme est prise sur l'ensemble
  \begin{equation}
    N = \left\{
      \gmp\nu \in (\N^\anydim)^{\lgr\ip}
      \text{ tels que }
      \sum\indrange \sum_{\indi_\ind = 1}^{\ip*} \indl = \derp
    \right\}
    \pmm.
  \end{equation}

  On peut alors poser
  \begin{align}
    \pmden^\derp    ( \anyvp^\ip )
    & =
    \anyvp[\anydim]^{\lgr\ip}
    \prod_{\ind = \anydim + 1}^{\dimp} \pden_\ind^{2\lgr\derp}
    \\
    \pmnum^\derp_\dv( \anyvp^\ip )
    & =
    \sum_{\gmp\nu \in N}
    \Biggl(
    \prod\indrange
    \prod_{ \indi_\ind = 1 }^{ \ip* }
    \pmnum^\indl_{1, \dv}( \anyvp* )
    \cdot
    \anyvp[\anydim]^{\lgr\ip - \star}
    \cdot
    \prod_{\ind = \anydim + 1}^{\dimp}
    \pden_\ind^{ 2\lgr\derp
      - \smash{ \sum\limits_{\indi_\ind = 1}^{\ip*} } f(\indl) }
    \Biggr)
    \pmm,
  \end{align}
  où \( \star \) désigne le nombre de \( \indl \) nuls. On vérifie en effet
  immédiatement que \( \pmnum^\derp_\dv( \anyvp^\ip ) \) est bien un polynôme
  (\( f \) est croissante et sous-additive et \( f(\derp) \le 2\lgr\derp \))
  et que \( \pmnum^\derp_\dv( \anyvp^\ip ) / \pmden^\derp ( \anyvp^\ip ) \)
  est bien le terme général de la série~\ref{e:par-mono-dev}.

  Les deux premiers points annoncés découlent directement de ces définitions.
  Pour le numérateur, le calcul du degré est direct et on ne détaille donc que
  l'estimation de norme : chaque terme de la somme est majoré en norme par
  \begin{align}
    \prod\indrange
    \prod_{ \indi_\ind = 1 }^{ \ip* }
    &
    \nv1{ \pmnum^\indl_{1, \dv}( \anyvp* ) }
    \cdot
    \prod_{\ind = \anydim + 1}^{\dimp}
    \nv1{ \pden_\ind }^{ 2\lgr\derp
      - \smash{ \sum\limits_{\indi_\ind = 1}^{\ip*} } f(\indl) }
    \\ & \le
    \prod_{\ind = \anydim + 1}^{\dimp}
    \left(
      \nv1{ \chow\anyvar }
      \bigl( 2 \sqrt{\anydim \anydeg^{3}} \bigr)^\dv
    \right)^{ \smash{ \sum\limits_{\indi_\ind = 1}^{\ip*} } f(\indl) }
    \cdot \left(
      \nv1{ \chow\anyvar } \anydeg^\dv
    \right)^{ 2\lgr\derp
      - \smash{ \sum\limits_{\indi_\ind = 1}^{\ip*} } f(\indl) }
    \\ & \le
    \nv1{ \chow\anyvar }^{2 (\dimp - \anydim) \lgr\derp}
    \left(
    \bigl( 4 \anydim \anydeg^{3} \bigr)^{(\dimp - \anydim) \lgr\derp}
    \right)^\dv
  \end{align}
  On remarque alors que l'ensemble de sommation \( N \) s'écrit aussi
  \begin{equation}
    \prod_{p = 0}^{\anydim - 1} \left\{
      \gmp\nu[][p] \in \N^{\lgr\ip}
      \text{ tels que }
      \sum\indrange \sum_{\indi_\ind = 1}^{\ip*}
      \gmp\nu[\ind, \indi_\ind][p]
      = \derp[p]
    \right\}
    \pmm.
  \end{equation}
  Chacun des facteurs de ce produit est de cardinal
  \(
    \binom{ \derp[p] + \lgr\ip - 1 }{ \lgr\ip - 1 }
    \le
    2^{ \derp[p] + \lgr\ip - 2 }
  \).
  L'estimation annoncée suit en prenant le produit en majorant assez largement
  certains facteurs.

  Enfin, on constate que chaque terme de la somme définissant
  \( \pmnum^\derp_\dv ( \anyvp^\ip ) \) contient un facteur de la forme
  \(
    \prod_{\indi_0 = 1}^{\ip[0]}
    \pmnum_{1, \dv}^{\gmp\nu[0, \indi_0]}( \anyvp[0] )
  \). On peut supposer que \( \gmp\nu[0, \indi_0] \in \N \times
    \set{0}^{\anydim-1} \) car sinon ce facteur, donc le terme correspondant,
  est nul. De plus, on a \( \sum_{\indi_0 = 1}^{\ip[0]} \gmp\nu[0, \indi_0][0]
    \le \derp[0] \). Ainsi, il y a au moins \( \ip[0] - \derp[0] \) termes
  nuls dans cette somme, donc \( \anyvp[0]^{\!\!\ip[0] - \derp[0]} \) est en
  facteur de chaque terme non nul de la somme définissant \( \pmnum^\derp_\dv
    ( \anyvp^\ip ) \), ce qui prouve l'assertion sur l'indice.
  % FIXME: \!\! above is a bad hack working around bad macros
\end{proof}

\begin{rem} \label{r:leibniz}
  Les applications \( \pmnum^\derp_\dv \) construites au lemme précédent
  satisfont la formule de \bsc{Leibniz} pour les dérivées divisées :
  \begin{equation}
    \pmnum^\derp_\dv (H J)
    =
    \sum_{\gp\nu \le \derp}
    \pmnum^{\gp\nu}_\dv (H)
    \cdot
    \pmnum^{\derp - \gp\nu}_\dv (J)
    \pmm,
  \end{equation}
  où \( H \) et \( J \) sont des formes homogènes.
\end{rem}

\begin{proof}
  Par construction, les applications \( \pmnum^\derp_\dv / \pmden^\derp \)
  satisfont cette formule, ce qui est une autre façon de dire que \( \pmor* \)
  est un morphisme d'anneaux. Le choix de dénominateurs effectué implique que
  \(
    \pmden^\derp(H J) =
    \pmden^{\gp\nu} (H) \cdot \pmden^{\derp - \gp\nu} (J)
  \)
  dès que \( H \) et \( J \) sont homogènes, ce qui donne le résultat annoncé.
\end{proof}


\subsection{Paramétrisation de \( \protect\var \)
  et image par \( \protect\wemb \)} \label{sub:param-var-img}

\begin{lem} \label{l:par-var}
  Soient \( \poldep** \) les polynômes donnés par le
  fait~\ref{f:plong-adapt-dep} et \( \pden\mexp*[\indi] \) leurs dérivées par
  rapport à \( \vmp*[\indi] \).  Il existe des applications \(
    \pmnum^\dermp_\dv \) et \( \pmden^\dermp \) représentant le morphisme de
  paramétrisation de \( \var \) sur l'ouvert défini par \( \pden\mexp*[\indi]
    \neq 0 \) et \( \vmp*[\vdim*] \neq 0 \) pour \( \fct \in \set{1, \dots,
      \puiss} \) et \( \indi \in \set{\vdim*+1, \dots, \dimp} \) tels que,
  pour toute forme multihomogène \( H \) de multidegré \( \alpha = \alpha_1,
    \dots, \alpha_\puiss \),
  \begin{enumthm}
    \item \( \pmnum^{\gmp 0}_\dv (H) = H \) ;
    \item \(
        \pmden^\dermp(H)
        =
        \prod\fctrange \Bigl(
          (\vmp*[\vdim*])^{\alpha_\fct}
          \prod_{\ind=\vdim*+1}^{\dimp} (\pden\mexp**)^{2\lgr{\dermp*}}
        \Bigr)
      \) ;
    \item \(
        \deg_\fct \pmnum^\dermp_\dv (H)
        =
        \alpha_\fct
        + 2 (\vdeg* - 1) (\dimp - \vdim*) \lgr{\dermp*}
      \) sauf si \( \pmnum^\dermp_\dv (H) = 0 \) ;
    \item \(
        \nv1{ \pmnum^\dermp_\dv (H) }
        \le
        \nv1 H
        \prod\limits\fctrange
        \nv1{ \varfc* }^{ 2 (\dimp - \vdim*) \lgr{\dermp*} }
        \left(
          \bigl( 8 \vdim* \vdeg*^3 \bigr)^{(\dimp - \vdim*) \lgr{\dermp*}}
          \cdot 2^{ \vdim* \alpha_\fct }
        \right)^\dv
      \) ;
    \item \( \inda* \bigl( \pmnum^\dermp_\dv (H) \bigr)
        \ge \inda*(H) - \wtsum*(\dermp) \).
  \end{enumthm}
\end{lem}

\begin{proof}
  On défini les applications \( \pmnum^\dermp_\dv \) et \( \pmden^\dermp \)
  sur les monômes par
  \begin{align}
    \pmnum^\dermp_\dv (\vmp^\imp)
    & =
    \prod\fctrange \pmnum^{\fct, \dermp*}_\dv \bigl( (\vmp*)^{\imp*} \bigr)
    &
    \pmden^\dermp (\vmp^\imp)
    & =
    \prod\fctrange \pmden^{\fct, \dermp*} \bigl( (\vmp*)^{\imp*} \bigr)
  \end{align}
  où les différentes applications \( \pmnum \) et \( \pmden \) des membres de
  droite sont obtenues en appliquant le lemme~\ref{l:par-anyvar-mono} à
  chacun des facteurs \( \var* \). On remarque alors que \( H \) est une
  combinaison linéaire de monômes de même degré.

  Seul le point sur l'indice reste à vérifier. Pour cela, considérons \(
    \vmp^\imp \) un monôme apparaissant dans l'écriture \( H \), puis \(
    \vmp^{\gmp\nu} \) un monôme apparaissant dans \( \pmnum^\dermp_\dv(
    \vmp^\imp ) \). D'après le lemme~\ref{l:par-anyvar-mono}, on alors \(
    \gmp\nu*[0] \ge \imp*[0] - \dermp*[0] \) d'où, en sommant, \(
    \wtsum*(\gmp\nu) \ge \wtsum*(\imp) - \wtsum*(\dermp) \) qui est équivalent
  à l'estimation annoncée vu la définition de l'indice.
\end{proof}


\endinput

% vim: spell spelllang=fr
