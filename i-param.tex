% !TEX root = main

\section{Paramétrisations} \label{sec:vojta-param}

Le but de cette section est d'expliciter une paramétrisation de \( \var \) en
tout point d'un ouvert dense \( \opdef \) que nous préciserons.
Géométriquement, une telle paramétrisation peut se voir comme un morphisme
\(
  \bigl( \aff1_{\hat{0}} \bigr)^{\lgr\vdim} \times \opdef
  \to
  \var
\),
où \( \aff1_{\hat{0}} \) désigne le complété formel de la droite affine en
l'origine. Algébriquement, il s'agit d'un morphisme (continu) d'anneaux
\(
  \pmor
  \colon
  \ring\var
  \to
  \ring\opdef \series\psmp
\),
où \( \ring\truc \) désigne l'anneau des coordonnées homogènes. Notre objectif
est de représenter ce morphisme par des polynômes. Plus précisément, nous
allons expliciter un morphisme \( \pmor* \) faisant commuter le diagramme
\begin{equation}
  \xymatrix{%
    \cdn [\vmp]
    % <>(.42) adjusts the location of the label between nodes
    \ar[r] ^<>(.42){\pmor*}
    \ar[d]
    & \cdn [\vmp] [\pden^{-1}] \series\psmp
    \ar[d]
    \\ \cdn [\vmp] / \varid
    \ar[r] ^<>(.42){\pmor}
    & \left( \cdn [\vmp] / \varid \right) [\pden^{-1}] \series\psmp
  }
\end{equation}
et, pour tout \( \dermp \in \N^{\lgr\vdim} \), des applications
\( \pmnum^\dermp \) et \( \pmden^\dermp \) de \( \cdn [\vmp] \) dans lui-même,
telles que
\begin{equation}
  \pmor*(\forme)
  =
  \sum_{ \dermp \in \N^{\lgr\vdim} }
  \frac{ \pmnum^\dermp(\forme) }{ \pmden^\dermp(\forme) }
  \psmp^\dermp
  \pmm.
\end{equation}
Notons que \( \pmnum^\dermp / \pmden^\dermp \) est donc une dérivation d'ordre
\( \dermp \).

Concrètement, soit \( \forme \) une forme multihomogène sur \( \var \) et
\( \tilde\forme \) la fonction rationnelle obtenue en la déshomogénéisant
d'une façon que l'on précisera en même temps que \( \opdef \), et \( \pmp
  \in \opdef \) de coordonnées homogènes \( \cmp \). Alors
\( \pmor*(\forme)(\cmp) \), c'est-à-dire la série de terme général
\( \pmnum^\dermp(\forme)(\cmp) / \pmden^\dermp(\forme)(\cmp) \),
est bien définie (le terme général est une expression homogène de degré \( 0 \)
en \( \cmp \)) et est le développement en série de \( \tilde\forme \) au
point \( \pmp \). En particulier, le terme \og constant \fg de \(
  \pmor(\forme) \) est \( \tilde\forme \).

\medskip

Plus précisément, comme nous ne savons pas construire des applications
\( \pmnum^\dermp \) dont nous contrôlons les normes à toutes les places, nous
construirons en fait deux familles d'applications \( \pmnum^\dermp_\dv \)
ayant les mêmes propriétés et dont nous contrôlerons les normes aux places
ultramétriques et archimédiennes respectivement.

Dans un premier temps, nous établissons un lemme général de paramétrisation
des variétés projectives, puis nous appliquons ce résultats à notre variété
\( \var \) et à la forme auxiliaire \( F \).


\subsection{Lemmes généraux de paramétrisation} \label{sub:param-gene}

Pour fournir une paramétrisation bien contrôlée
d'une variété projective plongée de façon adaptée. Les applications
\( \pmnum_\dv^\derp \) et \( \pmden^\derp \) étant linéaires, il nous suffit
d'étudier leur comportement sur une base ; nous choisirons la base monomiale
évidente.

Dans un premier temps, nous construirons des applications partielles,
définies seulement sur la partie homogène de degré \( 1 \) de l'espace de
départ (dont une base est tout simplement l'ensemble des variables), que nous
noterons
\( \pmnum_{1, \dv} \) et \( \pmden_1 \).  De plus, avant de nous intéresser à
une variété projective générale, nous étudions d'abord le cas plus simple
d'une hypersurface affine.

\begin{lem} \label{l:param-aff}
  Soit \( \anyvar \) une variété affine de dimension \( \anydim \), plongée
  dans un espace affine \( \aff{\anydim+1} \) de sorte que les coordonnées \(
    \anyvp[1], \dots, \anyvp[\anydim] \) sont indépendantes, et \( \poldep \in
    \cdn[\anyvp] \) l'équation de \( \anyvar \), de degré \( \anydeg \).
  Notons de plus \( \pden \) la dérivée de ce polynôme par rapport à la
  dernière variable.

  Alors on peut prendre choisir \( \pmnum_{1, 0} \), \( \pmnum_{1, 1} \) et \(
    \pmden_1 \) tels que, pour tout \( \place \) :
  \begin{enumthm}
    \item \( \pmor*(\anyvp*) = \anyvp* + \psp* \) pour tout
      \( \ind \in \set{1, \dots, \anydim} \) ;
    \item \( \pmnum^0_{1, \dv} (\anyvp[\anydim+1]) = \anyvp[\anydim+1] \) et
      \( \pmden_1^0 (\anyvp[\anydim+1]) = 1 \) ;
    \item \( \pmden_1^\derp (\anyvp[\anydim+1]) = \pden^{2\lgr\derp - 1} \)
      pour tout \( \derp \neq 0 \) ;
    \item \( \deg \pmnum^\derp_{1, \dv} (\anyvp[\anydim+1])
        \le (\anydeg - 1) (2\lgr\derp - 1) \)
      pour tout \( \derp \neq 0 \) ;
    \item \( \nv1{ \pmnum^\derp_{1, \dv} (\anyvp[\anydim+1]) }
        \le \nv1 \poldep ^{2\lgr\derp - 1}
        \cdot \left(
          (4\anydim)^{\lgr\derp -1} \anydeg^{3\lgr\derp -2}
        \right)^\dv \)
      pour tout \( \derp \) ;
  \end{enumthm}
  et \( \pmor* \) représente bien le morphisme de paramétrisation sur
  \( \anyvar \setminus \zeros{\pden} \).
\end{lem}

\begin{proof}
  Il s'agit en fait de compléter\footnote{On contrôle en fait le développement
    autour d'un point générique, alors que \bsc{Rémond} l'étudie en un point
    fixé.} la preuve du lemme~6.1 de \cite{remivds}, en
  utilisant aux places archimédiennes une généralisation de
  \cite[relation~2.3.1, p.~63]{farhith}.

  Jusqu'à la fin de la preuve, on note \( P_\dv^\derp = \pmnum^\derp_{1, \dv}
    (\anyvp[\anydim+1]) \).  On va construire $P_0^\derp$ et $P_1^\derp$
  indépendamment par récurrence
  sur la longueur de $\derp$, en partant à chaque fois de $P_\dv^\derp = -
  \diff[\ind_0] \poldep$ quand $\derp_{\ind_0} = 1$ et $\derp_\ind = 0$ sinon
  (cas $\lgr\derp = 1$), car ce choix convient. Pour la suite, on fixe un
  $\dv$, un $\derp$ de longueur au moins $2$, et on suppose qu'on a choisi un
  $P_\dv^{\derp'}$ convenable pour chaque $\derp'$ de longueur strictement
  inférieure à celle de $\derp$.

  On commence par le cas ultramétrique et on note donc provisoirement $P^\derp
  = P_0^\derp$ pour alléger. Si de tels polynômes existent, ils doivent
  satisfaire à la relation
  \begin{equation}
    \poldep \left(
      \anyvp[1] + \psp[1], \dots, \anyvp[\anydim] + \psp[\anydim],
      \anyvp[\anydim+1] + \sum_{ \derp \in \N^\anydim \minusset 0 }
      \frac {P^\derp} {\pden^{2\lgr\derp -1}} \psp^\derp
    \right)
    = 0 \mod \ideal\anyvar
    \pmm.
  \end{equation}
  On remplace alors \( \poldep \) par son développement de \bsc{Taylor}, on a
  les égalités suivantes modulo \( \ideal\anyvar \) :
  \begin{align}
    0
    & =
    \sum_{(\ip, \mu) \in \N^{\anydim+1}}
    \der[\ip, \mu] \poldep
    \cdot \psp^\ip
    \cdot \left(
      \sum_{ \derp \in \N^\anydim \minusset 0 }
      \frac {P^\derp} {\pden^{2\lgr\derp - 1}} \psp^\derp
    \right)^\mu
    \\
    & =
    \sum_{\substack{ (\ip, \mu) \in \N^{\anydim+1} \minusset{(0, 0)}
        \\ \gmp\nu* \in \N^\anydim \minusset 0 }}
    \left(
      \der[\ip, \mu] \poldep
      \cdot \prod_{\fct = 1}^\mu
      \frac {P^{\gmp\nu*}} {\pden^{2\lgr{\gmp\nu*} - 1}}
    \right)
    \psp^{\sum_\fct \gmp\nu* + \ip}
    \\
    & =
    \sum_{\derp \in \N^\anydim \minusset 0}
    \Biggl(
    \frac {P^\derp} {\pden^{2\lgr\derp - 2}}
    + \sum_{\substack{
        (\ip, \mu) \in \N^{\anydim+1} \minusset{(0, 0), (0, 1)}
        \\ \gmp\nu* \in \N^\anydim \minusset 0
        \\ \sum_\fct \gmp\nu* + \ip = \derp }}
    \der[\ip, \mu] \poldep
    \cdot \prod_{\fct = 1}^\mu
    \frac {P^{\gmp\nu*}} {\pden^{2\lgr{\gmp\nu*} - 1}}
    \Biggr)
    \psp^\derp
    \pmm,
  \end{align}
  où l'on a noté \( (\ip, \mu) = (\ip[1], \dots, \ip[\anydim], \mu) \).
  Réciproquement, pour que \( \pmor* \) représente bien \( \pmor \), il suffit
  de définir \( P^\derp \) par la relation de récurrence
  \begin{equation}
    - P^\derp
    =
    \sum_{\substack{
        (\ip, \mu) \in \N^{\anydim+1} \minusset{(0, 0), (0, 1)}
        \\ \gmp\nu* \in \N^\anydim \minusset 0
        \\ \sum_\fct \gmp\nu* + \ip = \derp }}
    \der[\ip, \mu] \poldep
    \cdot \pden^{2\lgr\derp - 2}
    \cdot \prod_{\fct = 1}^\mu
    \frac {P^{\gmp\nu*}} {\pden^{2\lgr{\gmp\nu*} - 1}}
  \end{equation}

  On majore alors le degré de \( P^\derp \) par récurrence :
  \begin{align}
    \deg P^\derp
    & \le
    \anydeg - \lgr\ip - \mu + (\anydeg - 1) (2\lgr\derp - 2)
    \\ & \le
    1 - \lgr\ip - \mu + (\anydeg - 1) (2\lgr\derp - 1)
    \\ & \le
    (\anydeg - 1) (2\lgr\derp - 1)
    \pmm,
  \end{align}
  car \( \ip \) et \( \mu \) ne sont pas simultanément nuls.

  La majoration de norme locale est par ailleurs immédiate vu les propriétés
  de la norme aux places ultramétriques.

  Traitons maintenant le cas archimédien (désormais \( P^\derp = P^\derp_1 \)
  pour alléger). On utilise la relation de récurrence suivante, établie dans
  la démonstration de \cite[lemme~6.1]{remivds}, avec \( Q_\derp =
    P^\derp \cdot \derp! \) et où, rappelons-le, \( \derp' \) est tel que
  \( \derp[\ind_0] = \derp[\ind_0]' + 1 \) et \( \derp* = \derp*' \) sinon :
  \begin{equation}
    Q_\derp
    =
    \pden^2 \, \diff[\ind_0] Q_{\derp'}
    - \pden \, \diff[\ind_0] P \, \diff[\anydim+1] Q_{\derp'}
    + (2\lgr{\derp'} - 1)
    (\diff[\ind_0] P \, \diff[\anydim+1] R - R \diff[\ind_0] R)
    Q_{\derp'}
    \pmm.
  \end{equation}
  On en déduit immédiatement l'estimation de degré suivante :
  \begin{equation}
    \deg P^\derp
    \le 2 (\anydeg - 1) + \deg P^{\derp'}
    \le (\anydeg - 1) (2\lgr\derp - 1)
    \pmm.
  \end{equation}
  Pour la norme, on prouve que \(
    \nv1{Q_\derp}
    \le
    \nv1\poldep^{2\lgr\derp-1} 4^{\lgr\derp-1} \anydeg^{3\lgr\derp-2}
    (\lgr\derp - 1) !
  \)
  (ce qui implique le résultat annoncé vu que \( \binom{\lgr\derp}{\derp}
    \le \anydim^{\lgr\derp-1} \)) en exploitant la majoration de degré
  sous la forme \( \deg P^\derp \le 2\lgr\derp \anydeg \).
  \begin{align}
    \nv1{ Q_\derp }
    & \le
    2\anydeg^2 \cdot \deg Q_{\derp'} \cdot \nv1\poldep^2 \nv1{ Q_{\derp'} }
    + 2 (2\lgr{\derp'} - 1) \anydeg^3 \nv1\poldep^2
    \\ & \le
    \nv1\poldep^2 \nv1{ Q_{\derp'} } \cdot 4 \anydeg^{3\lgr{\derp'}}
    \\ & \le
    \nv1\poldep^{2\lgr\derp-1} \cdot 4^{\lgr\derp-1} \anydeg^{3\lgr\derp-2}
    (\lgr\derp - 1) !
    \qedhere
  \end{align}
\end{proof}

Notons désormais
\begin{equation} \label{e:def-f-g}
  f(\derp) =
  \begin{cases*}
    0                               & si \( \derp = 0 \), \\
    2\lgr\derp - 1                  & sinon ;
  \end{cases*}
  \qquad
  g(\derp) =
  \begin{cases*}
    1                               & si \( \derp = 0 \), \\
    2 (\anydeg - 1) (\lgr\derp - 1) & sinon ;
  \end{cases*}
\end{equation}
de sorte que pour tout \( \derp \) et \( \indi \ge \anydim + 1 \) on a
\( \pmden_1(\anyvp[\indi]) = \pden^{f(\derp)} \) et
\( \deg \pmnum(\anyvp[\indi]) \le g(\derp) \) dans le lemme précédent.
Remarquons également que \( f \) est croissante en \( \lgr\derp \) et
sous-additive en \( \derp \), propriétés que nous utiliserons par la suite.

Passons maintenant au cas où \( \anyvar \) est une variété quelconque, plongée
de façon adaptée (voir section~\ref{sub:plong-adapt}) dans un espace
projectif.

\begin{lem}
  Soit \( \anyvar \) une variété projective de dimension \( \anydim \),
  plongée de façon adaptée dans \( \projd \), de degré \( \anydeg \). Pour \(
    \indi > \anydim \), notons \( \poldep[][\indi] \) une relation de
  dépendance intégrale de \( \anyvp[\indi] \) sur \( \anyvp[1], \dots,
    \anyvp[\anydim] \) telle que donnée par le fait~\ref{f:plong-adapt-dep},
  et \( \pden_\indi = \diff[\indi] \poldep[][\indi] \).

  Alors on peut prendre choisir \( \pmnum_{1, 0} \), \( \pmnum_{1, 1} \) et \(
    \pmden_1 \) tels que, pour tout \( \place \) :
  \begin{enumthm}
    \item \( \pmnum^0_{1, \dv} (\anyvp[\ind]) = \anyvp[\ind] \) et
      \( \pmden_1^0 (\anyvp[\ind]) = \anyvp[\anydim] \) pour tout \( \ind
        \in \set{1, \dots, \dimp} \) ;
    \item \( \pmor*(\anyvp*) = \frac{ \anyvp* }{ \anyvp[\anydim] } + \psp* \)
      pour tout \( \ind \in \set{0, \dots, \anydim-1} \), et
      \( \pmor*(\anyvp[\anydim]) = 1 \) ;
    \item \( \pmden_1^\derp (\anyvp[\indi]) = \pden_\indi^{f(\derp)} \) pour
      tout \( \indi \in \set{\anydim + 1, \dots, \dimp} \) et \( \derp \neq 0
      \) ;
    \item \( \deg \pmnum^\derp_{1, \dv} (\anyvp[\indi]) = g(\derp) \) pour tout \(
        \indi \in \set{\anydim + 1, \dots, \dimp} \) et tout \( \derp \) ;
    \item \( \nv1{ \pmnum^\derp_{1, \dv} (\anyvp[\indi]) }
        \le \left(
          \nv1{ \chow\anyvar }
          \bigl( 2 \sqrt{\anydim \anydeg^{3}} \bigr)^\dv
        \right)^{f(\derp)} \)
      pour tout \( \indi \in \set{\anydim + 1, \dots, \dimp} \) et tout \( \derp
      \) ;
  \end{enumthm}
  et \( \pmor* \) représente bien le morphisme de paramétrisation sur
  \( \anyvar \setminus \zeros{ \anyvp[\anydim], \pden_{\anydim+1},
      \dots, \pden_{\dimp} } \).
\end{lem}

\begin{proof}
  On sait que les différents \( \poldep[][\indi] \) sont de degré \( \anydeg
  \) et majorés en norme par \( \chow\anyvar \). On applique alors le
  lemme précédent à chaque \( \anyvp[\indi] \) après avoir déshomogénéisé par
  rapport à \( \anyvp[\anydim] \) et projeté sur  \( \aff{\anydim+1} \)
  de façon convenable.
\end{proof}

Nous allons maintenant construire des applications \( \pmnum_\dv \) et \(
  \pmden \) non restreintes aux formes linéaires. Pour simplifier leur
expression et obtenir des dénominateurs uniformes, nous choisissons des
applications qui ne prolongent pas celles construites en degré \( 1 \). La
perte de précision sur le degré des images de monômes particuliers n'est pas
importante en pratique, car ce sont surtout les images de formes de grand
degré qui nous intéressent. Au final, le dénominateur obtenu ne dépend que du
degré du monôme de départ.

\begin{lem} \label{l:par-anyvar-mono}
  Dans les hypothèses et notations du lemme précédent, pour tout \( \derp \),
  toute place \( \place \) de \( \cdn \) et tout exposant \( \ip \in \N^\dimp*
  \), on a :
  \begin{enumthm}
    \item \( \pmden^\derp( \anyvp^\ip )
        =
        \anyvp[\anydim]^{\lgr\ip}
        \prod_{\ind = \anydim + 1}^{\dimp} \pden_\ind^{2\lgr\derp}
      \) ;
    \item \( \deg \pmnum^\derp_\dv (\anyvp^\ip)
        = 2 (\anydeg - 1) (\dimp - \anydim) \lgr\derp + \lgr\ip \) sauf quand
      \( \pmnum^\derp_\dv (\anyvp^\ip) = 0 \) ;
    \item \(
        \nv1{ \pmnum^\derp_\dv (\anyvp^\ip) }
        \le
        2^{ (\lgr\derp + \anydim (\lgr\ip - 2))\dv }
        \left(
          \nv1{ \chow\anyvar }
          \bigl( 2 \sqrt{\anydim \anydeg^{3}} \bigr)^\dv
        \right)^{2 (\dimp - \anydim) \lgr\derp}
      \) ;
    \item \( \ord_{\anyvp[0]} \bigl( \pmnum^\derp_\dv (\anyvp^\ip) \bigr)
        \ge \ip[0] - \derp[0] \).
  \end{enumthm}
\end{lem}

\begin{proof}
  On utilise le résultat du lemme précédent et on se souvient que \( \pmor* \)
  est un morphisme d'anneaux :
  \newcommand \indl {{ \gmp\nu[\ind, \indi_\ind] }}
  \begin{align}
    \pmor*( \anyvp^\ip )
    & =
    \prod\indrange \pmor*( \anyvp* )^\ip*
    =
    \prod\indrange \left(
      \sum_{\derp \in \N^\anydim}
      \frac {
        \pmnum^\derp_{1, \dv}( \anyvp* )
      }{
        \pmden_1^\derp    ( \anyvp* )
      }
      \psp^\derp
    \right) ^\ip*
    \\ & =
    \prod\indrange
    \prod_{ \indi_\ind = 1 }^{ \ip* }
    \sum _{ \indl \in \N^\anydim }
    \frac {
      \pmnum^\indl_{1, \dv}( \anyvp* )
    }{
      \pmden_1^\indl    ( \anyvp* )
    }
    \psp^\indl
    \\ & =
    \sum_{\derp \in \N^\anydim}
    \left(
      \sum_{\gmp\nu \in N}
      \prod\indrange
      \prod_{ \indi_\ind = 1 }^{ \ip* }
      \frac {
        \pmnum^\indl_{1, \dv}( \anyvp* )
      }{
        \pmden_1^\indl    ( \anyvp* )
      }
    \right)
    \psp^\derp
    \pmm,
    \label{e:par-mono-dev}
  \end{align}
  où la somme est prise sur l'ensemble
  \begin{equation}
    N = \left\{
      \gmp\nu \in (\N^\anydim)^{\lgr\ip}
      \text{ tels que }
      \sum\indrange \sum_{\indi_\ind = 1}^{\ip*} \indl = \derp
    \right\}
    \pmm.
  \end{equation}

  On peut alors poser
  \begin{align}
    \pmden^\derp    ( \anyvp^\ip )
    & =
    \anyvp[\anydim]^{\lgr\ip}
    \prod_{\ind = \anydim + 1}^{\dimp} \pden_\ind^{2\lgr\derp}
    \\
    \pmnum^\derp_\dv( \anyvp^\ip )
    & =
    \sum_{\gmp\nu \in N}
    \Biggl(
    \prod\indrange
    \prod_{ \indi_\ind = 1 }^{ \ip* }
    \pmnum^\indl_{1, \dv}( \anyvp* )
    \cdot
    \anyvp[\anydim]^{\lgr\ip - \star}
    \cdot
    \prod_{\ind = \anydim + 1}^{\dimp}
    \pden_\ind^{ 2\lgr\derp
      - \smash{ \sum\limits_{\indi_\ind = 1}^{\ip*} } f(\indl) }
    \Biggr)
    \pmm,
  \end{align}
  où \( \star \) désigne le nombre de \( \indl \) nuls. On vérifie en effet
  immédiatement que \( \pmnum^\derp_\dv( \anyvp^\ip ) \) est bien un polynôme
  (\( f \) est croissante et sous-additive et \( f(\derp) \le 2\lgr\derp \))
  et que \( \pmnum^\derp_\dv( \anyvp^\ip ) / \pmden^\derp ( \anyvp^\ip ) \)
  est bien le terme général de la série~\ref{e:par-mono-dev}.

  Le calcul des degrés ainsi que l'estimation de norme du dénominateur sont
  immédiats. Pour le numérateur, chaque terme de la somme est majoré en norme
  par
  \begin{align}
    \prod\indrange
    \prod_{ \indi_\ind = 1 }^{ \ip* }
    &
    \nv1{ \pmnum^\indl_{1, \dv}( \anyvp* ) }
    \cdot
    \prod_{\ind = \anydim + 1}^{\dimp}
    \nv1{ \pden_\ind }^{ f(\derp)
      - \smash{ \sum\limits_{\indi_\ind = 1}^{\ip*} } f(\indl) }
    \\ & \le
    \prod_{\ind = \anydim + 1}^{\dimp}
    \left(
      \nv1{ \chow\anyvar }
      \bigl( 2 \sqrt{\anydim \anydeg^{3}} \bigr)^\dv
    \right)^{ \smash{ \sum\limits_{\indi_\ind = 1}^{\ip*} } f(\indl) }
    \cdot \left(
      \nv1{ \chow\anyvar } \anydeg^\dv
    \right)^{ f(\derp)
      - \smash{ \sum\limits_{\indi_\ind = 1}^{\ip*} } f(\indl) }
    \\ & \le
    \Biggl(
    \nv1{ \chow\anyvar }
    \bigl( 2 \sqrt{\anydim \anydeg^{3}} \bigr)^\dv
    \Biggr)^{(\dimp - \anydim) f(\derp)}
  \end{align}
  On remarque alors que l'ensemble de sommation \( N \) s'écrit aussi
  \begin{equation}
    \prod_{p = 0}^{\anydim - 1} \left\{
      \gmp\nu[][p] \in \N^{\lgr\ip}
      \text{ tels que }
      \sum\indrange \sum_{\indi_\ind = 1}^{\ip*}
      \gmp\nu[\ind, \indi_\ind][p]
      = \derp[p]
    \right\}
    \pmm.
  \end{equation}
  Chacun des facteurs de ce produit est de cardinal
  \(
    \binom{ \derp[p] + \lgr\ip - 1 }{ \lgr\ip - 1 }
    \le
    2^{ \derp[p] + \lgr\ip - 2 }
  \).
  L'estimation annoncée suit en prenant le produit.

  Enfin, on constate que chaque terme de la somme définissant
  \( \pmnum^\derp_\dv ( \anyvp^\ip ) \) contient un facteur de la forme
  \(
    \prod_{\indi_0 = 1}^{\ip[0]}
    \pmnum_{1, \dv}^{\gmp\nu[0, \indi_0]}( \anyvp[0] )
  \). On peut supposer que \( \gmp\nu[0, \indi_0] \in \N \times
    \set{0}^{\anydim-1} \) car sinon ce facteur, donc le terme correspondant,
  est nul. De plus, on a \( \sum_{\indi_0 = 1}^{\ip[0]} \gmp\nu[0, \indi_0][0]
    \le \derp[0] \). Ainsi, il y a au moins \( \ip[0] - \derp[0] \) termes
  nuls dans cette somme, donc \( \anyvp[0]^{\!\!\ip[0] - \derp[0]} \) est en
  facteur de chaque terme non nul de la somme définissant \( \pmnum^\derp_\dv
    ( \anyvp^\ip ) \), ce qui prouve l'assertion sur l'indice.
  % FIXME: \!\! above is a bad hack working around bad macros
\end{proof}

\begin{rem}
  Les applications \( \pmnum^\derp_\dv \) construites au lemme précédent
  satisfont la formule de \bsc{Leibniz} simplifiée :
  \begin{equation}
    \pmnum^\derp_\dv (H J)
    =
    \sum_{\gp\nu \le \derp}
    \pmnum^{\gp\nu}_\dv (H)
    \cdot
    \pmnum^{\derp - \gp\nu}_\dv (H J)
    \pmm,
  \end{equation}
  où \( H \) et \( J \) sont des formes homogènes.
\end{rem}

\begin{proof}
  Par construction, les applications \( \pmnum^\derp_\dv / \pmden^\derp \)
  satisfont cette formule, ce qui est une autre façon de dire que \( \pmor* \)
  est un morphisme d'anneaux. Le choix de dénominateurs effectué implique que
  \(
    \pmden^\derp(H J) =
    \pmden^{\gp\nu} (H) \cdot \pmden^{\derp - \gp\nu}_\dv (H J)
  \)
  dès que \( H \) et \( J \) sont homogènes, ce qui donne le résultat annoncé.
\end{proof}


\subsection{Paramétrisation de \( \protect\var \)
  et image par \( \protect\wemb \)} \label{sub:param-var-img}

\begin{lem} \label{l:par-var}
  Soient \( \poldep** \) les polynômes donnés par le
  fait~\ref{f:plong-adapt-dep} et \( \pden\mexp*[\indi] \) leurs dérivées par
  rapport à \( \vmp*[\indi] \).  Il existe des applications \(
    \pmnum^\dermp_\dv \) et \( \pmden^\dermp \) représentant le morphisme de
  paramétrisation de \( \var \) sur l'ouvert défini par \( \pden\mexp*[\indi]
    \neq 0 \) et \( \vmp*[\vdim*] \neq 0 \) pour \( \fct \in \set{1, \dots,
      \puiss} \) et \( \indi \in \set{\vdim*+1, \dots, \dimp} \) tels que,
  pour toute forme multihomogène \( H \) de multidegré \( \alpha = \alpha_1,
    \dots, \alpha_\puiss \),
  \begin{enumthm}
    \item \(
        \pmden^\dermp(H)
        =
        \prod\fctrange \Bigl(
          (\vmp*[\vdim*])^{\alpha_\fct}
          \prod_{\ind=\vdim*+1}^{\dimp} (\pden\mexp**)^{2\lgr{\dermp*}}
        \Bigr)
      \) ;
    \item \(
        \deg_\fct \pmnum^\dermp_\dv (H)
        =
        2 (\vdeg* - 1) (\dimp - \vdim*) \lgr{\dermp*} + \alpha_\fct
      \) sauf si \( \pmnum^\dermp_\dv (H) = 0 \) ;
    \item \(
        \nv1{ \pmnum^\dermp_\dv (H) }
        \le
        \nv1 H \cdot
        \prod\limits\fctrange
        2^{ (\lgr{\dermp*} + \vdim* (\lgr{\imp*} - 2))\dv }
        \left(
          \nv1{ \varfc* }
          \bigl( 2 \sqrt{\vdim* \vdeg*^{3}} \bigr)^\dv
        \right)^{2 (\dimp - \vdim*) \lgr{\dermp*}}
      \) ;
    \item \( \inda* \bigl( \pmnum^\dermp_\dv (H) \bigr)
        \ge \inda*(H) - \wtsum*(\dermp) \).
  \end{enumthm}
\end{lem}

\begin{proof}
  On défini les applications \( \pmnum^\dermp_\dv \) et \( \pmden^\dermp \)
  par
  \begin{align}
    \pmnum^\dermp_\dv (\cmp)
    & =
    \prod\fctrange \pmnum^{\fct, \dermp*}_\dv (\cmp*)
    &
    \pmden^\dermp (\cmp)
    & =
    \prod\fctrange \pmden^{\fct, \dermp*} (\cmp*)
  \end{align}
  où les différentes applications \( \pmnum \) et \( \pmden \) des membres de
  droite sont obtenues en appliquant le lemme~\ref{l:par-anyvar-mono} à
  chacun des facteurs \( \var* \). On remarque alors que \( H \) est une
  combinaison linéaire de monômes de même degrés.

  Seul le point sur l'indice reste à vérifier. Pour cela, considérons \(
    \vmp^\imp \) un monôme apparaissant dans l'écriture \( H \), puis \(
    \vmp^{\gmp\nu} \) un monôme apparaissant dans \( \pmnum^\dermp_\dv(
    \vmp^\imp ) \). D'après le lemme~\ref{l:par-anyvar-mono}, on alors \(
    \gmp\nu*[0] \ge \imp*[0] - \dermp*[0] \) d'où, en sommant, \(
    \wtsum*(\gmp\nu) \ge \wtsum*(\imp) - \wtsum*(\dermp) \) qui est équivalent
  à l'estimation annoncée vu la définition de l'indice.
\end{proof}


\endinput

% vim: spell spelllang=fr
