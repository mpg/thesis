% !TEX root = main

\section{Paramétrisations} \label{sec:vojta-param}

Le but de cette section est d'expliciter une paramétrisation de \( \var \) en
tout point d'un ouvert dense \( \opdef \) que nous préciserons.
Géométriquement, une telle paramétrisation peut se voir comme un morphisme
\(
  \bigl( \aff1_{\hat{0}} \bigr)^{\lgr\vdim} \times \opdef
  \to
  \var
\),
où \( \aff1_{\hat{0}} \) désigne le complété formel de la droite affine en
l'origine. Algébriquement, il s'agit d'un morphisme (continu) d'anneaux
\(
  \tau
  \colon
  \ring\var
  \to
  \ring\opdef \llbrack \param \rrbrack
\),
où \( \ring\truc \) désigne l'anneau des coordonnées homogènes. Notre objectif
est de représenter ce morphisme par des polynômes. Plus précisément, nous
allons expliciter un morphisme \( \tilde\tau \) faisant commuter le diagramme
\begin{equation}
  \xymatrix{%
    \cdn [\cmmh]
    % <>(.42) adjusts the location of the label between nodes
    \ar[r] ^<>(.42){\tau}
    \ar[d]
    & \cdn [\cmmh] [\pden^{-1}] \llbrack\param\rrbrack
    \ar[d]
    \\ \cdn [\cmmh] / \varid
    \ar[r] ^<>(.42){\tilde\tau}
    & \left( \cdn [\cmmh] / \varid \right) [\pden^{-1}] \llbrack\param\rrbrack
  }
\end{equation}
et, pour tout \( \expo \in \N^{\puiss\dimp**} \), des applications linéaires
\( \Delta^\expo \) et \( \Gamma^\expo \) de \( \cdn [\cmmh] \) dans lui-même
telles que
\begin{equation}
  \tilde\tau(\forme)
  =
  \sum_{ \expo \in \N^{\puiss\dimp**} }
  \frac{ \Delta^\expo(\forme) }{ \Gamma^\expo(\forme) }
  \param^\expo
  \pmm.
\end{equation}
Notons que \( \Delta^\expo / \Gamma^\expo \) est donc une dérivation d'ordre
\( \expo \).

Concrètement, soit \( \forme \) une forme multihomogène sur \( \var \) et
\( \tilde\forme \) la fonction rationnelle obtenue en la déshomogénéisant
d'une façon que l'on précisera en même temps que \( \opdef \), et \( \pmp
  \in \opdef \) de coordonnées homogènes \( \cmp \). Alors
\( \tilde\tau(\forme)(\cmp) \), c'est-à-dire la série de terme général
\( \Delta^\expo(\forme)(\cmp) / \Gamma^\expo(\forme)(\cmp) \),
est bien définie (le terme général est une expression homogène de degré \( 0 \)
en \( \cmp \)) et est le développement en série de \( \tilde\forme \) au
point \( \pmp \).

\endinput

% vim: spell spelllang=fr
