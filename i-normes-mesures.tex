% !TEX root = main.tex

\chapter{Normes et mesures archimédiennes}

Sur un corps $\cdn$ muni d'une valeur absolue $\abs\truc$ ultramétrique, on
définit la norme d'un polynôme $P = \sum p_i X^i$ par $\norm P = \sup_i
\abs{p_i}$. (Ici, $X$ désigne un groupe de variables ; les notations seront
précisées plus bas.) Cette norme définie naïvement jouit de toutes les
propriétés agréables auxquelles on peut s'attendre : elle vérifie l'inégalité
triangulaire (même ultramétrique), est multiplicative, se comporte bien
vis-à-vis de la composition des polynômes et en particulier de l'évaluation
en un point, ainsi que sous l'action de dérivations. Elle est par ailleurs
invariante par ajout de variables muettes. Plus précisément, on a les
relations évidentes suivantes :
\begin{align*}
  \norm{P'(X, Y)}
  & = \norm P
  &
  \norm{P+Q}
  & \le \max (\norm P, \norm Q)
  &
  \norm{P\cdot Q}
  & = \norm P \cdot \norm Q
  \\
  \norm { \frac{1}{i!} \frac {\partial^i P} {\partial X^i} }
  & \le \norm P
  &
  \abs{P(X)}
  & \le \norm P \cdot \norm X^d
  &
  \norm{P(Q)}
  & \le \norm P \cdot \norm Q^d
  \pmm,
\end{align*}
où $d$ est le degré de $P$, $Y$ un groupe de variables indépendantes de $X$,
et $P'$ l'image de $P$ par l'inclusion évidente $\cdn[X] \hookrightarrow
\cdn[X, Y]$. De plus, pour que les deux dernières inégalités soient vraies, il
convient de supposer $P$ homogène en un ou plusieurs groupes de variables,
avec les notations qui seront détaillées ci-dessous.

Sur les corps archimédiens par contre, aucune norme ou mesure usuelle n'est
vraiment \og la\fg bonne notion comme c'est le cas de la norme ci-dessus dans
le cas ultramétrique. On est donc amené suivant les circonstances à considérer
diverses grandeurs, qui présenteront chacune certaines des qualités
recherchées ; on aura alors besoin de les comparer entre elles. L'objet de ce
mémento est de réunir et de présenter de façon cohérente les diverses notions
utilisées, leurs propriétés, et les constantes de comparaison, qui sont
souvent bien connues mais un peu dispersées dans la littérature.

\section{Notations}

On note $X = (X_1, \dots, X_n)$ un groupe de variables. Pour $i \in \N^n$ un
multiindice, on notera $X^i = \prod_i X_j^{i_j}$. De même, notera
$\frac{\partial^i}{\partial X^i} = \frac{\partial^{i_1}}{\partial
  X_1^{i_1}}\cdots\frac{\partial^{i_n}}{\partial X_n^{i_n}}$. On portera notre
attention sur les dérivations $\der{i} = \frac1{i!}\frac{\partial^i}{\partial
  X^i}$, où l'on a noté $i! = i_1! \cdots i_n!$. On introduit par ailleurs les
notations $\lgr i = i_1 + \dots + i_n$ et, pour $d = \lgr i$, on note
$\binom{d}{i} = \frac{d!}{i!}$ le symbole multinomial. On notera à l'occasion
$d_i$ le degré partiel de $P$ en la variable $X_i$.

Introduisons maintenant les notations utiles dans le cas multihomogène. On
regardera des polynômes homogènes de multidegré $\delta = (\delta_1, \dots,
\delta_l)$ en $l$ groupes de $n_k$ variables $U_0^{(k)}, \dots,
U_{n_l}^{(k)}$, pour $1\le k \le l$. À un multiindice $\alpha =
(\alpha^{(1)}, \dots, \alpha^{(l)}) \in \N^{n_1} \times \dots \times
\N^{n_l}$, on associe son vecteur longueur $\vlg\alpha =
(\lgr{\alpha_1},\dots, \lgr{\alpha_l})$, et on notera ainsi $F =
\sum_{\vlg\alpha = \delta} f_\alpha X^\alpha$, avec la notation évidente pour
$X^\alpha$. On définit également $\der\alpha$, $\alpha!$ et
$\binom{\delta}{\alpha}$ comme dans le cas précédent, dont le cas
multihomogène est bien sûr un cas particulier (avec $n = \sum_k (n_k + 1)$,
$d = \lgr\delta$, et $\alpha$ correspondant naturellement à un $i \in \N^n$).
Toutefois, il faudra prêter attention au fait que $\delta = \vlg\alpha$ est
maintenant une égalité de vecteurs.

Par ailleurs, on est amené à considérer des familles $\mathcal F = (F_1,
\dots, F_p)$ de polynômes. À chaque norme que l'on introduira ci-dessous,
on associera une norme naturelle pour ces familles. On notera
ainsi $\norm{\mathcal F} = (\norm {F_1}, \ldots, \norm {F_p})$ le vecteur des
normes de la famille, $\nnorm F$ la norme de ce vecteur. Si $\mathcal Q =
\big(Q_j^{(k)}\big)_{0 \le j \le n_k}^{1 \le k \le l}$ est une famille de
polynômes quelconques, on notera enfin $\norm{\mathcal Q} =
(\nnorm{\smash{Q^{(1)}}}, \ldots, \nnorm{\smash{Q^{(l)}}})$ son vecteur norme,
de sorte que $\norm{\mathcal Q} ^\delta$ aura un sens naturel.

Pour les quatre normes, à chaque fois que l'on aura une majoration du type
$\norm{F(\mathcal Q)}_\truc \le C \norm F _\truc \norm{\mathcal
  Q}_\truc^\delta$, on aura aussi la majoration $\norm{\mathcal F(\mathcal
  Q)}_\truc \le C \nnorm{\mathcal F}_\truc \norm{\mathcal Q}^\delta$ si tous
les éléments de le famille $\mathcal F$ sont homogènes de même degré $\delta$.
Dans chaque cas, on ne notera donc que la majoration $\norm{F(\mathcal Q)} \le
C \norm F \norm{\mathcal Q}^\delta$, et on laissera le soin au lecteur d'en
déduire les autres majorations. En particulier, le cas où les $Q_j^{(k)}$ sont
des constantes correspond à l'évaluation en un point.

Enfin, la notation $\ncoef P$ désignera le nombre de coefficients non nuls
dans l'écriture de $P$ comme somme réduite de monômes. Ainsi, on a en général
$\ncoef P \le (d+1)^n$, et plus précisément dans le cas homogène, $\ncoef F
\le \prod\nolimits_k \binom{\delta_k + n_k}{n_k}$. On rappelle les majorations
commodes $\binom{d + n}{n} \le (d + 1)^{n+1}$ et symétriquement $\binom{d +
  n}{n} \le (n + 1)^{d+1}$. On notera $m(\delta) = \max_{\lgr\alpha = \delta}
{\textstyle \binom\delta\alpha} $, qui est majorée par $\prod\nolimits_k (n_k
+ 1)^{\delta_k}$. On introduit également la constante (liée à la hauteur
arakelovienne de l'espace projectif) $\gamma_n = \frac12 \sum_{q=1}^{n}
\frac1q$, dont on rappelle la majoration évidente $\gamma_n \le \frac12
(\log(n) + 1)$.

Certaines normes ou mesures seront définies uniquement pour les polynômes
homogènes. Les trois autres normes sont invariantes par homogénéisation. Ce
n'est par contre \lat{a priori} pas le cas de la mesure de \bsc{Mahler}.
Certaines opérations (évaluation) n'ont un comportement satisfaisant que dans
le cas homogène. On peut en déduire facilement des résultats dans le cas
général par homogénéisation. Dans d'autres cas, on supposera les polynômes
homogènes par commodité de notation (degrés partiels et coefficients
multinomiaux). Parfois même on notera $F$ n'importe quel multihomogénéisé de
$P$ de multidegré $\delta$. Ce sera implicitement le cas chaque fois que $F$
et $P$ apparaîtront dans une même inégalité.

Dans toute la suite, on conserve les notations introduites ci-dessus ; en
particulier $F$ et $G$ seront toujours des formes multihomogènes tandis que
$P$ et $Q$ incarneront des polynômes parfaitement généraux. De plus, $f_e$
sera une famille de polynômes dont le produit vaut $F$. On supposera par
ailleurs les polynômes non nuls chaque fois que ce sera utile. Enfin, comme on
s'intéresse ici au cas archimédien, on supposera que tous les polynômes
prennent leurs coefficients dans $\C$, quitte à fixer un plongement $\cdn
\hookrightarrow \C$ s'ils provenaient initialement d'un corps de nombres
$\cdn$.

\section{Norme du \texorpdfstring{$\sup$}{sup}}

Elle est définie pour tout $P$ par $\normsup P = \sup_i \abs{P_i}$ et
naturellement associée à la norme similaire pour les vecteurs.  C'est une
norme, évidemment invariante par ajout de variables muettes, qui jouit par
ailleurs des propriétés suivantes :
\begin{gather}
  \normsup P \normsup Q
  \le \normsup {PQ}
  \le \normsup P \normsup Q \todom
  \\
  \normsup {F(\mathcal Q)}
  \le \normsup F \normsup {\mathcal Q}^\delta \todom
  \\
  \normsup {\der\alpha F}
  \le \tbinom\delta\alpha \normsup F
\end{gather}

\section{Norme \texorpdfstring{$L_1$}{1}, ou longueur}

Elle est définie pour tout $P$ par $\normlun P = \sum_i \abs{P_i}$ et
naturellement associée à la norme similaire pour les vecteurs.

C'est une norme, évidemment invariante par ajout de variables muettes, qui
jouit par ailleurs des propriétés suivantes :
\begin{gather}
  \prod\nolimits_e \normlun{f_e} / \prod\nolimits_k (n_k +1)^{\delta_k}
  \le
  \normlun {F}
  \le
  \prod\nolimits_e \normlun{f_e} \\
  \normlun {F(\mathcal Q)}
  \le
  \normlun F \normlun {\mathcal Q}^\delta \\
  \normlun L(Q_1, \dots, Q_l)
  \le
  \normlun L \max_k\normlun{Q_k} \\
  \normlun{\der\alpha F}
  \le
  {\textstyle \binom\delta\alpha} \normlun P
\end{gather}

Elle se compare aux autres normes et mesures de la façon suivante.
\begin{align}
  \normlun P / \ncoef P
  & \le \normsup P
  \le \normlun P
  \\
  \normhom F
  & \le \normlun F
  \le \normhom F \prod\nolimits_k (n_k +1)^{\delta_k/2}
  \\
  \mespph F / \exp\big(
  {\textstyle \sum_k} \delta_k\gamma_{n_k}
  \big)
  & \le \normlun F
  \le \mespph F \prod\nolimits_k (n_k +1)^{\delta_k/2}
  \\
  \mahler P
  & \le \normlun F
  \le \mahler P \prod\nolimits_k (n_k +1)^{\delta_k/2}
\end{align}

\section{Norme euclidienne}

Elle est définie pour tout $P$ par
$\normeuc P = \big( \sum_i \abs{P_i}^2 \big)^{1/2}$.
C'est une norme, évidemment invariante par ajout de variables muettes, qui
jouit par ailleurs des propriétés suivantes :
\begin{gather}
  \normeuc P \cdot \normeuc Q
  \le \normeuc {PQ}
  \le \normeuc P \cdot \normeuc Q \todom
  \\
  \normeuc {F(\mathcal Q)}
  \le
  \normeuc F \normeuc {\mathcal Q}^\delta \todom
  \\
  \abs{F (u)}
  \le \normeuc{F}  \normeuc u ^\delta
  \\
  \normeuc{\der\alpha F}
  \le {\textstyle \binom\delta\alpha}  \normeuc F
\end{gather}

Elle se compare aux autres normes et mesures de la façon suivante :
\begin{align}
  \normeuc P / \sqrt{\ncoef P}
  & \le \normsup P
  \le \normeuc P
  \\
  \normeuc P
  & \le \normlun P
  \le \normeuc P \sqrt{\ncoef P}
  \\
  \normhom F
  & \le \normeuc P
  \le \sqrt{m(\delta)} \normhom F
  \\
  \mespph F / \exp\big( {\textstyle \sum_k}\delta_k\gamma_{n_k} \big)
  & \le \normeuc F
  \le \prod\nolimits_k (n_k +1)^{\delta_k} \mespph F
  \\
  \mahler P
  & \le \normeuc F
  \le \prod\nolimits_k (n_k +1)^{\delta_k} \mahler P
\end{align}
De plus si $L$ est une forme linéaire on a
$\normeuc L = \normhom L = \mespph L$.

\section{Norme euclidienne remodelée}

Elle est définie pour $F$ homogène par $\normhom F = \big(\sum_\alpha
\abs{P_\alpha}^2 \cdot\binom{\delta}{\alpha}^{-1} \big)^{1/2}$ et
naturellement associée à la norme euclidienne pour les vecteurs.

C'est une norme, évidemment invariante par ajout de variables muettes, qui
jouit par ailleurs des propriétés suivantes :
\begin{gather}
  \normhom F \normhom G
  \le \normhom {FG}
  \le \normhom F \normhom G  \todom
  \\
  \normhom {F(\mathcal Q)}
  \le  \normhom F \normhom {\mathcal Q}^\delta \todom
  \\
  \abs {F (u)}
  \le \normhom F  \normeuc u ^\delta
  \\
  \normhom {\der{i} P}
  \le \normhom P \todom
\end{gather}

Elle se compare aux autres normes et mesures de la façon suivante :
\begin{align}
  \normhom F / \exp\big( {\textstyle\sum_k} \sqrt{n_k} \,\big)
  & \le \normsup F
  \le \normhom F \sqrt{m(\delta)}
  \\
  \mespph F / \exp(\sum\delta_k\gamma_{n_k})
  & \le \normhom F
  \le \prod\nolimits_k (n_k +1)^{\delta_k/2} \mespph F
  \\
  \mahler F / \sqrt{m(\delta)}
  \le \normhom F
  & \le \prod\nolimits_k (n_k +1)^{\delta_k/2} \mahler F
\end{align}

\section{Mesure homogène à la \texorpdfstring{\bsc{Philippon}}{Philippon}}

Elle est définie pour $F$ homogène par
\[
  \log \mespph F
  = \big(
  \int_{S_{1}\times\cdots\times S_{l}}
  \log\abs F\ \mu_1 \wedge \cdots \wedge \mu_l
  \big) + \sum_{k=1}^l \delta_k \cdot \gamma_{n_k}
  \pmm,
\]
où $S_k = \{ u \in \C^{n_k},\ \normeuc u = 1 \}$ est la sphère de dimension
$2n_k-1$ plongée dans $\C^{n_k}$ muni de sa structure hermitienne usuelle,
$\mu_k$ désigne la mesure sur $S_k$ invariante par l'action du groupe unitaire
et normalisée par $\mu_k(S_k) = 1$, et enfin $\gamma_n = \frac12
\sum_{q=1}^{n} \frac1q$.

C'est une mesure multiplicative ($\mespph{FG} = \mespph F \cdot \mespph G$),
invariante par ajout de variables muettes grâce à la normalisation. \todo[qui
jouit par ailleurs des propriétés suivantes ?] Elle se compare aux autres
notions de la façon suivante.
\begin{align}
  \mespph F / \exp\big(
    {\textstyle\sum_k} \delta_k \gamma_{n_k} \sqrt{n_k}
  \,\big)
  & \le \normsup F
  \le \mespph F m(\delta)
  \\
  \mahler F
  \le \mespph F
  \le \mahler F \exp(\sum\delta_k\gamma{n_k})
\end{align}

\section{Mesure de \texorpdfstring{\bsc{Mahler}}{Mahler}}

Elle est définie pour tout $P$ par
\[
  \log\mahler P
  = \int_S \log\abs P\, \mu_S \pmm,
\]
où $S$ est le produit des cercles unités de $\C$, muni de la mesure $\mu$
produit des mesures de \bsc{Haar} normalisées de masse $1$.

C'est une mesure multiplicative ($\mahler {PQ} = \mahler P \cdot \mahler Q$),
invariante par ajout de variables muettes grâce à la normalisation. \todo[qui
jouit par ailleurs des propriétés suivantes ?]

Elle se compare aux autres normes et mesures de la façon suivante.
\begin{align}
  \mahler P / \sqrt{\ncoef P}
  & \le \normsup F
  \le \mahler P m(\delta)
\end{align}

\section{Preuves et questions d'optimalité}

\nocite{phicia, remgdmp, remstp}
% TODO: mahler, lelong

\endinput

% vim: spell spelllang=fr
