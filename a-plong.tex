% !TEX root = main

\chapter{Plongement et formulaires abéliens}
\label{chap:plong-mm}


\section{Notations et problématique}
\label{sec:plong-mm-def}

Dans toute cette annexe, on supposera que \( \va \) est plongée dans un espace
projectif par un plongement \( \Theta \) modifié tel que décrit dans
\cite{daphimhva2}, associé à la puissance seizième d'un fibré ample et
symétrique \( \fibrei \) ou, de façon équivalente, à la puissance quatrième du
fibré très ample et totalement symétrique \( \fibre = \fibrei^{\otimes 4} \).

On utilisera sans trop les rappeler les définitions et notations de
\cite{daphimhva2} (section~3.1.1) sauf sur les points suivants : on omettra
systématiquement les indices se rapportant à la puissance de
\( \fibre \) utilisée et on notera plus volontiers en indice les coordonnées.
On écrira ainsi $\coa_{(a, l)} = \Delta_{a, l}(\coa)  = \Delta_{(a,
  l)}^{(2)}(\coa) = \coa_{\fibre^{\otimes 4}}(a, l)$. Par ailleurs, bien que
les coordonnées soient naturellement indexées (\lat{ibid.} p. 651) par
$\mathcal{Z}_2 \times \widehat{K_2(4)}$, on les indexera souvent par $\{0,
  \dots, \dimp\}$ pour alléger l'écriture, où \( \dimp = 16^\genre \deg
  \fibrei - 1 \) ; ainsi on écrira $\coa = (\coa_0, \dots \coa_\dimp)$ des
coordonnées de l'origine. On rappelle que le degré de \( \Theta(\va) \) en
tant que variété projective est égal à \( \dimp + 1 \).
\nomuse \coa {Coordonnées de l'origine de \( \va \) dans le plongement par
  défaut}

\nomuse \vaht {Hauteur de l'origine de \( \va \) dans le plongement par
  défaut}
On note par ailleurs \( \vaht = \hautl[2]{\coa} \) la quantité désignée par \(
  h(\va) \) dans le texte cité (notation~3.2 page~652), qui représente la
hauteur de la variété. Plus précisément, le corollaire~6.9 page~698 du même
article donne une comparaison explicite entre \( \vaht \) et la hauteur de
\bsc{Faltings} de \( \va \).

Par ailleurs, le lemme~3.5 de l'article en question garantit qu'il existe dans
chaque classe de \( K_2 \times \widehat{K_2(4)} \mod 2K_2 \times
  \widehat{K_2(4)}^2 \) au moins un indice \( (a, l) \) tel que \(
  \coa_{(a,l)} \neq 0 \).  On note \( \mathcal{Z}_\coi \) un système complet
de représentants vérifiant cette condition, et on pose \( \coi = (\coa_{(a,
    l)}^{-1})_{(a, l) \in \mathcal{Z}_\coi} \). La constante \( c_v(\va) =
  \nv2{\coa} \nv2{\coi} \ge 1 \) est alors bien définie (c'est-à-dire homogène
de degré \( 0 \) en \( \coa \)) ; c'est l'équivalent local de \( h(\coa) \).

\medskip

Un premier avantage du plongement utilisé est qu'on y connaît une constante de
comparaison explicite entre hauteurs normalisée et projective.  Plus
précisément, le fait suivant est une reformulation de la proposition~3.9 de
\emph{loc. cit} et de la remarque subséquente.

\begin{fact} \label{f:comp-h-hn}
  Pour tout point \( \pp \in \va(\Qbar) \), on a \(
    \abs{\hautn\pp - \hautl[2]\pp} \le \htcmp \) avec
  \begin{equation}
    \htcmp = (4^\genre - 1) \vaht + 3\genre \ln2
    \pmm.
  \end{equation}
\end{fact}

Un second avantage de ce plongement \( \Theta \) est qu'on dispose de formules
explicites représentant globalement le morphisme d'addition-soustraction, dont
on peut déduire des formules explicites représentant respectivement
l'addition, la soustraction, la multiplication par un entier arbitraire et
finalement des combinaisons linéaires arbitraires. Ces représentations ne sont
plus globales, mais les systèmes de cartes recouvrant \( \va \) ou \( \va
  \times \va \) sur lesquelles les différentes formules sont valables sont eux
aussi explicites.

Par ailleurs, \worknote{Référence ?}[on sait que] la multiplication par un
entier \( b \) peut être représentée globalement par une famille de formes
homogènes de degré \( b^2 \). Il serait intéressant de savoir expliciter une
telle famille dans un plongement adéquat. Dans le cas des courbes elliptiques
plongées à la \bsc{Weierstrass}, le théorème~2.13.2, page~128 de \cite{farhith}
donne des formules de multiplication globales totalement explicites ;
malheureusement, aucun résultat analogue n'existe à ce jour pour les variétés
abéliennes générales.

Dans le cas particulier de la multiplication par \( 2 \) (et par voie de
conséquence, par une puissance de \( 2 \)), la proposition~3.8 de
\cite{daphimhva2} donne une représentation globale explicite, qui est en fait
à la base de la comparaison entre hauteurs projective et normalisée rappelée
ci-dessus.

\medskip

Cependant, dans certaines applications
(sous-section~\ref{sec:vojta-extrap-core}) on aura besoin non seulement de
savoir que chaque point appartient à une carte où des formules représentant
les combinaisons linéaires existent, mais de préciser ceci par des minoration
de valeur absolues en différentes places. On ne peut expliciter cette
minoration par rapport aux coefficients de la combinaison linéaire que si l'on
dispose de représentation globales de la multiplication pour (les facteurs
premiers de) ces coefficients. En pratique, on ne peut se contenter de prendre
des puissances de \( 2 \) comme coefficients. Ceci nous amènera à considérer
un formulaire non explicite dans lequel on peut établir de telles minorations
de valeurs absolues relativement à une constante non explicite mais ne
dépendant que de \( \va \) plongée.

Il serait évidemment souhaitable de pouvoir contrôler cette constante de façon
explicite. Pour l'application considérée, il suffirait de contrôler des
formules globales de multiplication par un entier autre qu'une puissance de \(
  2 \), ce qui n'est pas le cas actuellement. La suite de cette annexe se
divise donc en deux sections : la première rappelle les formules explicites
connues dans le plongement \( \Theta \), la seconde présente des formules non
explicites mais possédant les propriétés métriques voulues.

Auparavant, introduisons quelques notations communes aux deux sections
suivantes. Nous noterons \( \multab b \) le morphisme de multiplication par un
entier \( b \) et
\begin{align}
  \clab ab \colon \va \times \va & \to \va \times \va \\
  (x, y) & \mapsto (ax + by, ax - by)
\end{align}
puis \( \clab[+] ab \) (resp. \( \clab[-] ab \)) la composition du morphisme
précédent avec la projection sur le premier (resp. le deuxième) facteur.
Par exemple, \( \clab[+] 11 \) est le morphisme d'addition.



\section{Formulaire explicite}
\label{sec:form-ab}

\subsection{Combinaisons linéaires générales}
\label{sec:form-ab-cl}

On regarde \( \va \times \va \) plongée dans \( \projd \times \projd \) par \(
  \Theta^2 \) et on note \( V_0, \dots, V_\dimp \) les coordonnées sur le
premier facteur \( \projd \) et \( W_0, \dots, W_\dimp \) celles sur le
second. On peut ensuite plonger \( (\projd)^2 \) dans \( \proj\dimpi \) par un
plongement de \bsc{Segre}, c'est-à-dire qu'un système de coordonnées sur cet
espace est donné par la famille \( V_l W_{l'} \) où \( (l, l') \) parcourt \(
  \set{0, \dots, \dimp}^2 \).

Le résultat fondamental est le fait suivant, qui est une simple reformulation
de la proposition~3.6 de \cite{daphimhva2} en utilisant les notations
ci-dessus pour simplifier l'énoncé et le rendre plus accessible indépendamment
du reste de l'article cité.

\begin{fact} \label{f:addsub}
  Il existe une famille \( S_{l, l'} \) de formes bihomogènes de degré \( 2 \)
  en chacun des groupes de variables \( V \) et \( W \), c'est-à-dire
  homogènes de degré \( 2 \) en \( (V_l W_{l'})_{l, l'} \), représentant
  globalement le morphisme d'addition-soustraction \( \clab 11 \) dans le
  plongement \( \Theta^2 \) suivi d'un plongement de \bsc{Segre}. De plus
  cette famille peut être choisie de la façon suivante :
  \begin{equation}
    S_{l, l'} (V, W)
    =
    \coi_{\xi(l)} \coi_{\xi(l')}
    \sum_{p, p', q, q' \in E}
    \zeta_{p, p', q, q'} V_p V_{p'} W_q W_{q'}
    \pmm,
  \end{equation}
  où \( E \) est un sous-ensemble à \( 4^\genre \) éléments de \( \{0, \dots,
      \dimp\} \), les \( \zeta_{p, p', q, q'} \) sont des racines quatrièmes
  de l'unité, et \( \xi \) est une certaine fonction \( \{0, \dots, \dimp\}
    \to \{0, \dots, \dimp\} \) telle que pour tout \( p \), \(
    \coi^{-1}_{\xi(p)} \) est une coordonnée (non nulle) de \( \oa \) (voir le
  paragraphe précédent cette sous-section).

  En particulier, on peut estimer les normes locales :
  \( \nv1{S_{l, l'}} \le 4^{\genre\dv} \nv1{\coi}^2 \)
  pour tout \( (l, l') \) et
  \(
    \nnv1{S}
    \le
    \bigl( 4^\genre (\dimp+1)^2 \bigr)^\dv \nv1{\coi}^2
  \)
  pour la famille.
\end{fact}

Cette représentation suppose \( \va^2 \) plongée dans un espace projectif
\( \proj{\dimp^2 + 2\dimp} \). Si l'on veut en déduire simplement, par
projection sur un facteur, des représentations de l'addition et de la
soustraction, on doit d'abord « redescendre » \( \va^2 \) dans \( (\projd)^2
\) : ceci se fait simplement par des projections linéaires. Or celles-ci ont
un centre en lequel elles ne sont pas définies, et qui rencontre \( \va \). On
n'obtient donc pas une représentation globale de l'addition et de la
soustraction, mais une famille de représentations locales.

Plus précisément, pour chaque couple de points \( x, y \) de \( \va \) (dont
on note également \( x \) et \( y \) des coordonnées homogènes) , la famille
\( (S_{l, l'}(x, y))_{l} \) est un système de coordonnées homogènes de \( x +
  y \) à condition qu'elle ne soit pas identiquement nulle et \( (S_{l, l'}(x,
  y))_{l'} \) est un système de coordonnées homogènes de \( x - y \) à la même
condition. Autrement dit, si l'on note \( \opdefi_{l} \) (resp. \(
  \opdefi'_{l'} \)) l'ouvert \( V_{l} \neq 0 \) (resp. \( W_{l'} \neq 0 \)) et,
que par abus, on note de même leurs images dans \( \proj\dimpi \), on voit que
la famille \( (S_{l, l'})_l \) fournit des formules d'addition valables sur \(
  \clab[-]11^{-1}(\opdefi'_{l'}) \), et de même pour la soustraction.

En remarquant de plus qu'on peut dans chacune de ces familles simplifier par
\( \coi_{\xi(l')} \) puisqu'on travaille à \( l' \) fixé, on a donc le résultat
suivant.

\begin{coro} \label{c:addsub-form}
  Il existe des ensembles de \( \dimp + 1 \) ouverts recouvrant \( \va^2 \),
  notés \( \clmaps_+ \) et \( \clmaps_- \), tels que pour tout ouvert \(
    \clmap \) dans \( \clmaps_+ \) (resp. dans \( \clmaps_- \)) il existe une
  famille de formes biquadratiques \( (S\pexp{+,\clmap}[l])_l \) (resp. \(
    S\pexp{-,\clmap}[l])_l \)) représentant l'addition (resp. la soustraction)
  sur \( \clmap \) et telle que
  \begin{equation}
    \norm{S^{\bullet, \clmap}_l}_{v,1}
    \le
    4^{\genre\dv} \norm{\coi}_{v, 1}
    \quad \forall l \in \{0, \dots, \dimp\}
    \pmm,
  \end{equation}
  où \( \bullet \in \set{ +, - } \).
\end{coro}

À partir de cette représentation de l'addition, on peut fabriquer des formes
représentant localement la multiplication, puis les combiner à nouveau avec
les formes représentant l'addition et la soustraction pour représenter une
combinaison linéaire générale. Ce travail est fait par la proposition~5.2
page~126 de \cite{remivds}, dont on reformule l'énoncé dans le fait suivant.

\begin{fact} \label{f:clab}
  Posons \( f(a) = \lfloor (a^2 - 1)/8 \rfloor \) pour tout entier \( a \).
  Il existe un ensemble de \( (\dimp + 1)^2 \) ouverts recouvrant \( \va^2 \),
  noté \( \clmaps' \) et, pour tout \( \clmap \in \clmaps' \) et tout couple
  d'entiers naturels \( a \) et \( b \), une famille de formes \(
    (L\pexp{\clmap, a, b}[l,l'])_{l,l'} \) représentant \( \clab ab \) sur \(
    \clmap \) dans le plongement \( \Theta^2 \) suivi d'un plongement de
  \bsc{Segre} à l'arrivée, telle que
  \begin{enumthm}
    \item \(
        \deg L\pexp{\clmap, a, b}[l,l']
        =
        ( 2(a^2 + f(a)), 2(b^2 + f(b)) )
      \);
    \item \(
        \hautl[\infty]{ (L\pexp{\clmap, a, b}[l,l'])_{l,l'} }
        \le
        (a^2 + b^2) (2 (4^\genre-1) \vaht + 6\dimp)
      \).
  \end{enumthm}
\end{fact}

\begin{proof}
  On remarque tout d'abord qu'on peut prendre
  \( P_{1, 1, 0, 0, l, l'} = S_{l, l'} \) où le membre de gauche reprend les
  notations de la proposition de \bsc{Rémond}, alors que le membre de droite
  est la famille donnée par le fait~\ref{f:addsub}, qui n'était pas disponible
  au moment de la rédaction de \cite{remivds}. On constate par ailleurs que \(
    \hautl[2]{(S_{l, l'})_{l, l'}} \le 2(4^\genre-1) \vaht \) d'après la
  proposition~3.7 de~\cite{daphimhva2} (la hauteur considérée dans cette
  référence étant celle que nous notons \( \Hautl[2] \), voir en bas de la
  page~644 de l'article) et que la hauteur considérée par \bsc{Rémond} est
  celle que nous notons \( \Hautl[\infty] \) (page~104 de \cite{remivds}), qui
  est majorée par la précédente.

  Par ailleurs l'ensemble de cartes choisi est la famille des \( \clmap_{i,
      j} = (V_i W_j \neq 0) \) (avec les notations ci-dessus pour les
  coordonnées). Sur chacune de ces cartes, on pose \( L\pexp{\clmap_{i, j}, a,
      b}[l,l'] = P_{a, b, i, j, l, l'} \) (où le membre de droite est donné
  par la proposition citée dans ses notations). C'est bien une représentation
  de \( \clab ab \) valable sur cette carte, car la quantité notée \(
    p_1^* \xi_i^{\otimes f(a)} \otimes p_2^* \xi_j^{\otimes f(b)} \) chez
  \bsc{Rémond} ne s'annulle jamais sur cet ouvert, par définition.
\end{proof}

On peut, en effectuant des projections linéaires à l'arrivée, en déduire des
représentations de \( \clab[+] ab \) et \( \clab[-] ab \) de la même façon
qu'on a obtenu le corollaire~\ref{c:addsub-form}. Cependant, il n'est pas clair
que ceci soit utilisé dans le reste du texte, nous nous arrêtons donc ici (pour
l'instant).


\subsection{Multiplication par deux}
\label{sec:form-ab2}

\startout[Section utilisée seulement dans \bsc{Mumford} et assez mal extraite
de ce dernier. À revoir.]

Il convient de commencer par rappeler quelques faits concernant les
plongements utilisés. On a dit que les coordonnées étaient naturellement
indexées par $\mathcal{Z}_2 \times \widehat{K_2(4)}$, où $\mathcal{Z}_2$ est
un système complet de représentants des classes de $K_2$ modulo $K_2(4)$. Il
est parfois plus commode d'utiliser un système étendu de coordonnées, indexées
par tout $K_2 \times \widehat{K_2(4)}$, comme définies dans
\cite[p.~651]{daphimhva2}, que nous noterons $\Delta^{(\text{ét})}$ au
besoin. Le changement de système de coordonnées de $\Delta^{(\text{ét})}$ vers
$\Delta$ consiste à prendre une sous-famille ; le sens contraire est donné par
le point (iv) du fait~3.3 de \lat{loc. cit.} : il nous suffira de savoir que
les coordonnées manquantes s'obtiennent en multipliant les anciennes par des
racines quatrièmes de l'unité, ce qui préserve les estimations de normes
locales.

Par ailleurs, on note $\tilde\Theta$ le plongement composé $\va^2 \to
(\Proj^n)^2 \to \Proj^N$ et $\tilde\Delta_{ij} = \Delta_i\Delta_j$ les
coordonnées correspondantes. On définit de façon similaire un plongement
$\tilde\Theta^{(\text{ét})}$ associé au système étendu de coordonnées. La
proposition~3.7 de \cite{daphimhva2} décrit alors explicitement l'action du
morphisme $\xi$ sur le système de coordonnées $\tilde\Delta^{(\text{ét})}$. En
ramenant cette description à notre système restreint de coordonnées, cette
proposition se lit :

\begin{fact}
  Il existe une famille de formes $F$ sur $\Proj^N$ satisfaisant\footnote{Dans
    cette égalité, on a identifié sections globales de $\xi^*\mathcal N$ et de
    $\mathcal N ^{\otimes 2}$ \lat{via} l'isomorphisme naturel, où $\mathcal N
    = \tilde\Theta^*(O(1))$. Par la suite, on ne signalera plus des
    identifications similaires.} $F(\tilde\Delta) = \xi^*(\tilde\Delta)$ et de
  la forme :
  \begin{equation}
    F_{ij}(Z) = \frac 1{2^g \theta_{i'}\theta_{j'}}\sum_{(k,l)\in I^2}
    \zeta_{ij}(k, l) Z_{\alpha(k, l)} Z_{\beta(k, l)} \pmm,
  \end{equation}
  où $\zeta_{ij}(k, l)$ est une racine quatrième de l'unité, $I$ un certain
  ensemble à $4^g$ éléments, $\alpha$ et $\beta$ des applications de $I^2$
  dans $\{0,\dots,N\}^2$ et $i'$, $j'$ sont dans $\mathcal Z_{\mathcal B}$.
\end{fact}

En particulier cette famille $F$ représente le morphisme $\xi$ dans le
plongement $\tilde\Theta$, c'est-à-dire que $F(\tilde\Delta(z))$ et
$\tilde\Delta(\xi(z))$ sont colinéaires. La proposition~affirme qu'il sont
même égaux ; ce supplément d'information joue un rôle crucial dans la preuve
du lemme. Par ailleurs, on tire de même des propositions 3.8 et 3.11 de
\lat{loc.  cit.} l'existence d'une famille $G$ de formes sur $\Proj^n$ telles
que $G(\Delta) = [2]^*\Delta$ (représentant donc le morphisme de
multiplication par $2$) et satisfaisant à l'inégalité suivante :
\begin{equation} \label{NormDupl}
  \frac {\Onv {G(x)}}  {\Onv x ^4} \ge \av 2^g 4^{-g\dv} \Onv\coa^{-3} \pmm.
\end{equation}

\stopout



\section{Formulaire abélien alternatif}
\label{sec:form-ab-alt}

Le formulaire présenté dans cette section n'est valable que pour des
coefficients de la forme \( 2^\alpha 3^\beta \) et ne permet pas un contrôle
totalement explicite de la hauteur des formules, mais présente en revanche
l'avantage que la multiplication est représentée globalement, ce qui permet
d'établir des minorations de normes locales qui se révèlent essentielles pour
notre preuve de l'inégalité de \bsc{Vojta}.

\begin{fact}
  Supposons fixé \worknote{Quelles propriétés pour ce plongement ?}[un
  plongement] de \( \va \) dans \( \projd \) ; il existe un
  réel \( C_{2, 3} \ge 1\) et un entier \( r_{2, 3} \in \cdn^* \), ne
  dépendant que de ce plongement, et, pour tout \( \gamma = 2^\alpha 3^\beta
  \), une famille \( F_\gamma \) de polynômes à coefficients entiers dans \(
    \cdn \), représentant globalement la multiplication par \( \gamma \) sur
  \( \va \), de degré \( \gamma^2 \) et de hauteur multiplicative majorée par
  \( C_{2, 3}^{\gamma^2} \), telle que pour tout \( x \in \va \) et toutes
  place \( \place \) de \( \cdn \), on a :
  \begin{equation}
    \av{r_{2, 3}}^{\gamma^2} C_{2, 3}^{-\dv \gamma^2}
    \le
    \frac{ \nv1{ F_\gamma(x) } }{ \nnv1 {F_\gamma} \nv1 x ^{\gamma^2} }
    \le
    C_{2, 3}^{\dv \gamma^2}
    \pmm.
  \end{equation}
\end{fact}

\begin{proof}
  \worknote{Argument insuffisant, il faut préciser la dépendance de \(
      \nnv1{F_\gamma} \) en \( \gamma \).}
  C'est le fait~3, page~276 de~\cite{phiha1}, restreint au cas particulier qui
  nous intéresse. On a par ailleurs rajouté au dénominateur un facteur \(
    \nnv1 {F_\gamma} \), qui vaut \( 1 \) aux places finies, et qui n'invalide
  pas la comparaison aux places infinies quitte à modifier un peu la constante
  \( C_{2, 3} \) choisie, qui n'est de toute façon pas explicite.
\end{proof}

\begin{lem}
  Supposons fixé un plongement de \( \va \) dans \( \projd \) et un ensemble
  \( \clmap \) d'ouverts recouvrant \( \va^2 \) ainsi que pour tout \( \clmap
    \in \clmaps \) un système de formes \( (S\pexp{\clmap}[l])_l \)
  biquadratiques représentant la soustraction sur \( \clmap \).  Il existe un
  réel \( C_s \ge 1 \) et un entier \( r_s \in \cdn^* \), ne dépendant que de
  ces données ce plongement, et, pour tout couple de points \( (x, y) \in
    \va^2 \) et toute place \( \place \) de \( \cdn \), un ouvert \(
    \clmap_{x, y, \place} \in \clmaps \) tel que
  \begin{equation}
    \av{r_s} C_s^{-\dv}
    \le
    \frac{
      \nv1{ S\pexp{\clmap_{x, y, \place}}(x, y) }
    }{
      \nnv1{ S\pexp{\clmap_{x, y, \place}} } \nv1 x ^2 \nv1 y ^2
    }
    \le
    C_s^{\dv}
    \pmm.
  \end{equation}
\end{lem}

\begin{ideas}
  La majoration de droite est triviale, à condition de ne pas prendre \( C_s
  \) trop petit. Pour celle de gauche, on considère la famille \(
    S\pexp{\clmap}_l \) quand \( \clmap \) et \( l \) varient et on invoque le
  théorème des zéros multihomogène sur l'idéal engendré cette famille et
  l'idéal de \( \va \). Il reste à comparer un peu les normes.
\end{ideas}

\begin{lem} \label{l:hclab}
  Il existe une famille (ne dépendant que de \( \va \) et \( \Theta \)) de
  réels \( \hmclab* \) presque tous égaux à \( 1 \) (de sorte que \(
    \hmclab = \prod_\place \hmclab*^{\degv} \) est bien défini) possédant la
  propriété suivante.

  Dans le plongement \( \Theta \), il existe un ensemble \( \clmaps \) de \(
    \dimp+1 \) ouverts recouvrant \( \va^2 \) et, pour tout \( \clmap \in
    \clmaps \) et tout couple d'entiers \( a, b \) n'admettant que \( 2 \) et
  \( 3 \) comme diviseurs premiers, une famille \( (L\pexp{a, b, \clmap}[l])_l
  \) de formes bihomogènes de degré \( (2a^2, 2b^2) \) et de hauteur majorée
  par \( \hmclab^{a^2 + b^2} \), représentant \(
    \clab[-] ab \) sur \( \clmap \). De plus, pour
  tout couple de points \( (x, y) \in \va^2 \), toute place \( \place
  \) de \( \cdn \) et tous entiers \( a \) et \( b \) comme ci-dessus, il
  existe une carte \( \clmap_{a, b, x, y, \place} \in \clmaps \) telle que
  \begin{equation}
    \hmclab*^{-(a^2 + b^2)}
    \le
    \frac{
      \nv1{ L\pexp{ a, b, \clmap_{a, b, x, y, \place} }(x, y) }
    }{
      \nnv1{ L\pexp{ a, b, \clmap_{a, b, x, y, \place} } }
      \nv1 x ^{2a^2} \nv1 y ^{2b^2}
    }
    \le
    \hmclab*^{a^2 + b^2}
    \pmm.
  \end{equation}
\end{lem}

\begin{proof} \later
  On combine les résultats des deux lemmes précédents ainsi que du
  corollaire~\ref{c:addsub-form}.
\end{proof}

\begin{tdef} \label{d:hclab}
  On note désormais \( \hmclab* \) et \( \hmclab \) les constantes données par
  l'application du lemme précédent et \( \hlclab = \ln \hmclab \).
\end{tdef}

\cleardoublepage
\endinput

% vim: spell spelllang=fr
