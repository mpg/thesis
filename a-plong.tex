% !TEX root = main

\chapter{Plongement projectif et formulaire} \label{chap:plong-mm}

\section{Rappel du plongement et notations} \label{sec:plong-mm-def}

Sauf précision contraire, on supposera toujours implicitement que \( \va \)
est plongée dans un espace projectif par un plongement \( \Theta \) modifié
tel que décrit dans \cite{daphimhva2}, associé à la puissance quatrième d'un
fibré très ample et totalement symétrique \( \fibre \).

\later
Pour l'instant, on utilise sans les rappeler les définitions et notations de
\cite{daphimhva2} (section~3.1.1) sauf sur les points suivants : on omettra
systématiquement les indices se rapportant à la puissance de
\( \fibre \) utilisée et on notera plus volontiers en indice les coordonnées.
On écrira ainsi $\coa_{(a, l)} = \Delta_{a, l}(\coa)  = \Delta_{(a,
  l)}^{(2)}(\coa) = \coa_{\fibre^{\otimes 4}}(a, l)$. Par ailleurs, bien que
les coordonnées soient naturellement indexées (\lat{ibid.} p. 651) par
$\mathcal{Z}_2 \times \widehat{K_2(4)}$, on les indexera souvent par $\{0,
  \dots, \dimp\}$ pour alléger l'écriture  ; ainsi on écrira $\coa = (\coa_0,
\dots \coa_n)$ des coordonnées de l'origine.  On note enfin \( \vaht =
  \norm2\coa \) la quantité désignée par \( h(\va) \) dans le texte cite
(notation~3.2).
\nomuse \coa {Coordonnées de l'origine de \( \va \) dans le plongement par
  défaut}
\nomuse \vaht {Hauteur de l'origine de \( \va \) dans le plongement par
  défaut}

Un premier avantage du plongement utilisé est qu'on y connaît une constante de
comparaison explicite entre hauteurs normalisée et projective.  Plus
précisément, le fait suivant est une reformulation de la proposition~3.9 de
\emph{loc. cit} et de la remarque subséquente.

\begin{fact} \label{f:comp-h-hn}
  Pour tout point \( \pp \in \va(\Qbar) \), on a
  \begin{equation}
    \abs{\hautn\pp - \hautl[2]\pp}
    \le
    (4^\genre - 1) \vaht + 3\genre \ln2
    \pmm.
  \end{equation}
\end{fact}

Un second avantage est qu'on dispose de formules explicites pour
l'addition-soustraction, ainsi que de formules explicites globales pour la
multiplication par deux. On peut en déduire des formules explicites locales
représentant des combinaisons linéaires, cependant si les coefficients ne sont
pas des puissances de $2$, on ne disposera pas des minorations de normes
locales qui nous seraient utiles. On peut au contraire montrer l'existence de
formules donnant ces minorations, mais dont on ne contrôle pas explicitement
la hauteur. Les deux sections suivantes établissent respectivement des
formulaires de ces deux types.


\section{Formulaire abélien alternatif} \label{sec:form-ab-alt}

Le formulaire présenté dans cette section n'est valable que pour des
coefficients de la forme \( 2^\alpha 3^\beta \) et ne permet pas un contrôle
totalement explicite de la hauteur des formules, mais présente en revanche
l'avantage que la multiplication est représentée globalement, ce qui permet
d'établir des minorations de normes locales qui se révèlent essentielles pour
notre preuve de l'inégalité de \bsc{Vojta}.

\begin{fact}
  Supposons fixé un plongement de \( \va \) dans \( \projd \) ; il existe un
  réel \( c_{2, 3} \ge 1\) et un entier \( r_{2, 3} \in \cdn^* \), ne
  dépendant que de ce plongement, et, pour tout \( \gamma = 2^\alpha 3^\beta
  \), une famille \( F_\gamma \) de polynômes à coefficients entiers dans \(
    \cdn \), représentant globalement la multiplication par \( \gamma \) sur
  \( \va \), de degré \( \gamma^2 \) et de hauteur multiplicative majorée par
  \( c_{2, 3}^{\gamma^2} \), telle que pour tout \( x \in \va \) et toutes
  place \( \place \) de \( \cdn \), on a :
  \begin{equation}
    \av{r_{2, 3}}^{\gamma^2} c_{2, 3}^{-\dv \gamma^2}
    \le
    \frac{ \nv2{ F_\gamma(x) } }{ \nnv2 {F_\gamma} \nv2 x ^{\gamma^2} }
    \le
    c_{2, 3}^{\dv \gamma^2}
    \pmm.
  \end{equation}
\end{fact}

\begin{proof}
  C'est le fait~3, page~276 de~\cite{phiha1}, restreint au cas particulier qui
  nous intéresse. On a par ailleurs rajouté au dénominateur un facteur \(
    \nnv2 {F_\gamma} \), qui vaut \( 1 \) aux places finies, et qui n'invalide
  pas la comparaison aux places infinies quitte à modifier un peu la constante
  \( c_{2, 3} \) choisie, qui n'est de toute façon pas explicite.
\end{proof}

\begin{lem}
  Supposons fixé un plongement de \( \va \) dans \( \projd \) ; il existe un
  réel \( c_s \ge 1 \) et un entier \( r_s \in \cdn^* \), ne dépendant que de
  ce plongement, et, pour tout couple de points \( (x, y) \in \va^2 \) et
  toute place \( \place \) de \( \cdn \), une famille \( S_{x, y, \place} \)
  de polynômes à coefficients entiers dans \( \cdn \) représentant la
  soustraction sur un voisinage de \( (x, y) \), de bidegré \( (2, 2) \) et de
  hauteur multiplicative majorée par \( c_s \), telle que
  \begin{equation}
    \av{r_s} c_s^{-\dv}
    \le
    \frac{
      \nv2{ S_{x, y, \place}(x, y) }
    }{
      \nnv2 {S_{x, y, \place}} \nv2 x ^2 \nv2 y ^2
    }
    \le
    c_s^{\dv}
    \pmm.
  \end{equation}
\end{lem}

\begin{ideas}
  La majoration de droite est triviale, à condition de ne pas prendre \( c_s
  \) trop petit. Pour celle de gauche, on considère des familles représentant
  la soustraction sur une carte de \( \va \), on invoque le théorème des zéros
  multihomogène sur l'idéal engendré par l'idéal de \( \va \) et des familles.
\end{ideas}

On note désormais \( [ \alpha, \beta ] \) le morphisme envoyant \( (x, y) \)
sur \( \alpha x - \beta y \) et on s'intéresse à sa représentation par des
formes bihomogènes.

\begin{lem} \label{l:hclab}
  \renewcommand\Theta{\iota}% cradobeurk mais la flemme
  Pour tout plongement \( \iota \colon \va \embedin \projd \), il existe une
  famille (ne dépendant que de \( \va \) et de ce plongement) de réels \(
    \hclab* \) presque tous égaux à \( 1 \) (de sorte que \( \hclab =
    \prod_\place \hclab*^{\degv} \) est bien défini) et, pour tout couple de
  points \( (x, y) \in \va^2 \), toute place \( \place \) de \( \cdn \) et
  tous entiers \( \alpha \) et \( \beta \) n'ayant que \( 2 \) et \( 3 \)
  comme diviseurs premiers, une famille \( L^{(\alpha, \beta)}_{x, y, v} \) de
  polynômes bihomogènes à coefficients entiers dans \( \cdn \) représentant \(
    [ \alpha, \beta ] \) au voisinage de \( (x, y) \), de bidegré \(
    (2\alpha^2, 2\beta^2) \) et de hauteur multiplicative majorée par \(
    \hclab^{\alpha^2+\beta^2} \), telle que
  \begin{equation}
    \hclab*^{-(\alpha^2 + \beta^2)}
    \le
    \frac{
      \nv2{ L^{(\alpha, \beta)}_{x, y, v}(x, y) }
    }{
      \nnv2{ L^{(\alpha, \beta)}_{x, y, v} }
      \nv2 x ^{2\alpha^2} \nv2 y ^{2\beta^2}
    }
    \le
    \hclab*^{\alpha^2 + \beta^2}
    \pmm.
  \end{equation}
\end{lem}


\section{Formulaire abélien explicite} \label{sec:form-ab}
\startout[Section non utilisée dans \bsc{Vojta}, probablement pour
  \bsc{Mumford} mais notations à revoir.]

Le résultat fondamental est le fait suivant, qui est une simple reformulation
de la proposition~3.6 de \cite{daphimhva2}.

\begin{fact} \label{f:addsub}
  Il existe une famille $S_{l, l'}$ de formes bihomogènes de bidegré $(2,
  2)$ représentant globalement de morphisme d'addition-soustraction
  \begin{align}
    \va \times \va
    & \to
    \va \times \va
    \\
    (x, y)
    & \mapsto
    (x + y, x - y)
  \end{align}
  dans le plongement de \bsc{Mumford} modifié suivi d'un plongement de
  \bsc{Segre}. On peut les prendre de la forme
  \begin{equation}
    S_{l, l'} (V, W)
    =
    \coi_{\xi(l)} \coi_{\xi(l')}
    \sum_{p, p', q, q' \in E}
    \zeta_{p, p', q, q'} V_p V_{p'} W_q W_{q'}
    \pmm,
  \end{equation}
  où $E$ est un sous-ensemble à $4^\genre$ éléments de $\{0, \dots, \dimp\}$,
  les $\zeta_{p, p', q, q'}$ sont des racines quatrièmes de l'unité, et $\xi$
  est une certaine fonction $\{0, \dots, \dimp\} \to \{0, \dots, \dimp\}$
  telle que pour tout $p$, $\coi^{-1}_{\xi(p)}$ est une coordonnée (non nulle)
  de $\oa$.

  En particulier, on peut estimer les normes locales :
  $\norm{S_{l, l'}}_{v, 1} \le 4^{\genre\dv} \norm{\coi}^2_{v, 1}$
  pour tout $(l, l')$ et
  $\nnorm{S}_{v, 1} \le
  \bigl( 4^\genre (\dimp+1)^2 \bigr)^\dv \norm{\coi}^2_{v, 1}$
  pour la famille.
\end{fact}

Cette représentation suppose $\va^2$ plongée dans un espace projectif
$\proj{\dimp^2 + 2\dimp}$. Si l'on veut en déduire simplement, par projection
sur un facteur, des représentations de l'addition et de la soustraction, on
doit d'abord « redescendre » $\va^2$ dans $(\projd)^2$ : ceci se fait
simplement par des projections linéaires. Or celles-ci ont un centre en lequel
elles ne sont pas définies, et qui rencontre $\va$. On n'obtient donc pas une
représentation globale de l'addition et de la soustraction, mais une famille
de représentations locales. En remarquant de plus qu'on peut dans chacune de
ces formes simplifier par $\coi_{k(l')}$ puisqu'on projette à $l'$ fixé, on a
donc le résultat suivant.

\begin{coro}\label{c:addsub-form} \worknote{L'ouvert dépend-il du signe ? Si
    oui, reformuler.}
  Tout couple de points de $\va$ est contenu dans un ouvert $\opdef$ de
  $\va^2$ sur lequel il existe deux familles de formes biquadratiques
  $(S_l^{+})$ et $(S_l^{-})$ pour $0 \le l \le \dimp$ représentant sur
  $\opdef$ l'addition (resp. la soustraction) de $\va$, telles que
  \begin{equation}
    \norm{S^{\bullet}_l}_{v,1}
    \le
    4^{\genre\dv} \norm{\coi}_{v, 1}
    \quad \forall l \in \{0, \dots, \dimp\}
    \pmm,
  \end{equation}
  où $\bullet \in \{ +, - \}$.
\end{coro}

Par ailleurs, \worknote{Référence ?}[on sait que] la multiplication par un
entier $b$ peut être
représentée globalement par une famille de formes homogènes de degré $b^2$. On
connaît même \cite[prop. 3.8]{daphimhva2} une telle famille de façon
totalement explicite pour $b = 2$, donc à chaque fois que $b$ est une
puissance de $2$, en itérant.

On ne sait \lat{a priori} pas expliciter une telle famille dans le cas
général\footnote{
  Pour les courbes elliptiques, c'est fait dans \cite[th. 2.13.2]{farhith}
  pour tout $b$.},
mais on peut néanmoins obtenir des formules locales de multiplication, en
combinant les formules locales d'addition et de multiplication par deux.

\worknote{C'est idiot de faire la multiplication pour ensuite refaire des
  combinaisons linéaires à la main, autant faire directement les combinaisons
  linéaires comme dans le résultat de \bsc{Rémond} cité.}
\begin{fact}\label{f:mult-form}
  Pour tout $j \in \{ 0, \dots, \dimp \}$ et pour tout entier positif, il
  existe une famille de formes homogènes $Q_{b, i, j}$ pour $i \in \{ 0,
    \dots, \dimp \}$ représentant la multiplication par $b$ sur l'ouvert $V_j
  \neq 0$, telle que, pour tout $i$ :
  \begin{gather}
    \deg Q_{b, i, j} = \left\lceil \frac98 b^2 \right\rceil
    \\
    \norm {Q_{b, i, j}}_{v, 1}
    \le
    \norm{\coi}_{v, 1}^{b^2 -1}
    \cdot \bigl( 2^\genre \sqrt{\dimp+1} \bigr)^{\dv(b^2-1)}
    \pmm.
  \end{gather}
\end{fact}

\begin{proof} \later
  Tout est contenu dans la démonstration de la proposition~5.2 de \cite[pp.
  126-128]{remivds} une fois remarqué qu'on peut choisir
  \begin{equation}
    P_{1, 1, 0, 0, l, l'} = S_{l, l'}
    \pmm,
  \end{equation}
  où le membre de gauche reprend les notations de la preuve de \bsc{Rémond},
  alors que le membre de droite est la famille donnée par le
  fait~\ref{f:addsub}, qui n'était pas disponible au moment de la
  rédaction de \cite{remivds}.

  De plus, \bsc{Rémond} donne seulement la majoration du degré. Il est en
  fait plus commode de fixer une valeur exacte du degré, ce que l'on peut
  faire en multipliant par $V_j$ sans changer l'ouvert sur lequel les
  formules sont valables, ni leur hauteur.

  On a par ailleurs utilisé la norme $L_1$ aux places archimédiennes, ce qui
  réduit légèrement la partie archimédienne de la constante par rapport à
  celle de \bsc{Rémond}.
\end{proof}

\stopout


\section{Compléments au formulaire abélien explicite} \label{sec:form-ab2}
\startout[Section utilisée seulement dans \bsc{Mumford} et assez mal extraite
de ce dernier. À revoir et peut-être intégrer dans la section précédente.]

Il convient de commencer par rappeler quelques faits concernant les
plongements utilisés. On a dit que les coordonnées étaient naturellement
indexées par $\mathcal{Z}_2 \times \widehat{K_2(4)}$, où $\mathcal{Z}_2$ est
un système complet de représentants des classes de $K_2$ modulo $K_2(4)$. Il
est parfois plus commode d'utiliser un système étendu de coordonnées, indexées
par tout $K_2 \times \widehat{K_2(4)}$, comme définies dans
\cite[p.~651]{daphimhva2}, que nous noterons $\Delta^{(\text{ét})}$ au
besoin. Le changement de système de coordonnées de $\Delta^{(\text{ét})}$ vers
$\Delta$ consiste à prendre une sous-famille ; le sens contraire est donné par
le point (iv) du fait~3.3 de \lat{loc. cit.} : il nous suffira de savoir que
les coordonnées manquantes s'obtiennent en multipliant les anciennes par des
racines quatrièmes de l'unité, ce qui préserve les estimations de normes
locales.

Par ailleurs, on note $\tilde\Theta$ le plongement composé $\va^2 \to
(\Proj^n)^2 \to \Proj^N$ et $\tilde\Delta_{ij} = \Delta_i\Delta_j$ les
coordonnées correspondantes. On définit de façon similaire un plongement
$\tilde\Theta^{(\text{ét})}$ associé au système étendu de coordonnées. La
proposition~3.7 de \cite{daphimhva2} décrit alors explicitement l'action du
morphisme $\xi$ sur le système de coordonnées $\tilde\Delta^{(\text{ét})}$. En
ramenant cette description à notre système restreint de coordonnées, cette
proposition se lit :

\begin{fact}
  Il existe une famille de formes $F$ sur $\Proj^N$ satisfaisant\footnote{Dans
    cette égalité, on a identifié sections globales de $\xi^*\mathcal N$ et de
    $\mathcal N ^{\otimes 2}$ \lat{via} l'isomorphisme naturel, où $\mathcal N
    = \tilde\Theta^*(O(1))$. Par la suite, on ne signalera plus des
    identifications similaires.} $F(\tilde\Delta) = \xi^*(\tilde\Delta)$ et de
  la forme :
  \begin{equation}
    F_{ij}(Z) = \frac 1{2^g \theta_{i'}\theta_{j'}}\sum_{(k,l)\in I^2}
    \zeta_{ij}(k, l) Z_{\alpha(k, l)} Z_{\beta(k, l)} \pmm,
  \end{equation}
  où $\zeta_{ij}(k, l)$ est une racine quatrième de l'unité, $I$ un certain
  ensemble à $4^g$ éléments, $\alpha$ et $\beta$ des applications de $I^2$
  dans $\{0,\dots,N\}^2$ et $i'$, $j'$ sont dans $\mathcal Z_{\mathcal B}$.
\end{fact}

En particulier cette famille $F$ représente le morphisme $\xi$ dans le
plongement $\tilde\Theta$, c'est-à-dire que $F(\tilde\Delta(z))$ et
$\tilde\Delta(\xi(z))$ sont colinéaires. La proposition~affirme qu'il sont
même égaux ; ce supplément d'information joue un rôle crucial dans la preuve
du lemme. Par ailleurs, on tire de même des propositions 3.8 et 3.11 de
\lat{loc.  cit.} l'existence d'une famille $G$ de formes sur $\Proj^n$ telles
que $G(\Delta) = [2]^*\Delta$ (représentant donc le morphisme de
multiplication par $2$) et satisfaisant à l'inégalité suivante :
\begin{equation} \label{NormDupl}
  \frac {\Onv {G(x)}}  {\Onv x ^4} \ge \av 2^g 4^{-g\dv} \Onv\coa^{-3} \pmm.
\end{equation}

\stopout

\endinput

% vim: spell spelllang=fr
