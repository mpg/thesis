% !TEX root = main.tex

\let\chapter\chapter % pour satisfaire mon éditeur (syntaxe => dictionnaire)
\addchap{Remerciements}

Cette thèse a bien sûr été un passionnant parcours intellectuel, l'occasion
d'approfondir ma compréhension de problèmes mathématiques fascinants et de
goûter aux plaisirs et aux exigences de la recherche en apportant ma petite
pierre à l'édifice. Elle a aussi et avant tout été une formidable
aventure personnelle, souvent difficile, mais de ces difficultés dont on sort
grandi.

Si l'aventure se termine de façon heureuse, c'est en grande partie grâce aux
personnes que j'ai eu le privilège de rencontrer en chemin, qui m'ont beaucoup
apporté que ce soit sur le plan mathématique ou humain (ce « ou » étant bien
sûr inclusif) et que je souhaite ici remercier.

\medskip

Ma gratitude va tout d'abord à mon directeur de thèse Patrice \bsc{Philippon},
qui m'a proposé un projet de recherche à la fois réaliste et stimulant, puis
m'a accompagné avec une disponibilité et une patience sans faille, me faisant
bénéficier de sa grande expérience et de sa profonde connaissance du domaine.
C'est d'ailleurs lui et Sinnou \bsc{David} qui, en cours puis mémoire de
Master, m'ont initié aux problèmes et outils diophantiens, et donné l'envie de
poursuivre une thèse sur ce sujet.

Je suis également très reconnaissant à Gaël \bsc{Rémond} qui a accepté de
rapporter mes travaux, mais les a surtout rendus possibles en ouvrant la voie
par deux de ses articles, dont il a très gentiment pris la peine de
m'expliquer quelques points lorsque nous nous sommes croisés pendant ma
thèse. Sa relecture rigoureuse et ses remarques pertinentes m'ont permis
d'améliorer le manuscrit final en plusieurs endroits.
Je remercie également Philipp \bsc{Habegger} d'avoir bien voulu
rapporter cette thèse et me faire part de ses remarques, qui ont aussi
contribué à la qualité du manuscrit. Les erreurs restantes sont miennes, il
va sans dire.

J'ai déjà remercié Sinnou \bsc{David} pour son cours inspirant, je le remercie
d'autant plus d'avoir accepté de faire partie de mon jury, ainsi que Francesco
\bsc{Amoroso}, Daniel \bsc{Bertrand}, Guillaume \bsc{Maurin} et Fabien
\bsc{Pazuki}.

\medskip

J'ai effectué cette thèse au sein de l'institut de mathématiques de Jussieu
qui (surtout à l'époque de Chevaleret) a été un environnement à la fois très
riche mathématiquement et agréable à vivre. Avant de passer aux personnes que
j'y ai rencontré plus personnellement, j'aimerais remercier toutes celles qui
font ou ont fait vivre cet endroit, notamment les administrateurs et
gestionnaires, dont le travail indispensable est trop souvent oublié. Je pense
en particulier à Marceline \bsc{Prosper} qui s'occupait des doctorants, à
Zouber \bsc{Zadvat} avec qui j'ai plus particulièrement été en contact en tant
que membre du bureau des doctorants, qui étaient toujours efficaces et à
l'écoute des doctorants.

Parmi les chercheurs qui font de l'IMJ un endroit si spécial, je remercie
particulièrement Marc \bsc{Hindry}, toujours et prêt à nous faire profiter
avec une grande gentillesse de ses impressionnantes connaissances
mathématiques. Je remercie également les professeurs avec lesquels j'ai eu le
plaisir d'enseigner, en particulier Jan \bsc{Nekovář} et Alain \bsc{Kraus}.

Durant ces années de thèse, j'ai enseigné avec un plaisir toujours
renouvelé, qui a souvent été une ressource précieuse quand la recherche se
faisait difficile. Je remercie chaleureusement mes étudiants de l'UPMC, de
l'université d'Évry, de l'ENSIIE et du CIES Jussieu, de m'avoir
supporté et souvent surpris heureusement.

\medskip

Pendant ma première année à l'IMJ, j'ai eu le plaisir non seulement de
partager un bureau, mais aussi de découvrir la capitale avec Antonella et
Philipp. Vint ensuite la période bénie du 7C04, dont j'ai plaisir à remercier
les occupants successifs : François dont j'ai pris la place, Joël pour ses
réponses toujours parfaites à mes questions de géométrie, Olivier et le culte
de l'ours, Fabien le collègue diophantien, Jérôme aux conseils toujours
avisés, Benjamin W. et Élodie croisés plus furtivement, et \eng{last but not
  least}, Cécile, pour qui la page est trop courte.

Ce fabuleux bureau ne vivant pas en autarcie, je remercie aussi avec plaisir
les autres doctorants que j'ai rencontrés : Nicolas B. avec qui j'ai partagé
mes trois premières années d'enseignement, Benoît, Ismaël, Luc, Banafsheh,
Julien G., Clémence, Mirjam, toutes celles et ceux que j'oublie (qu'ils
veuillent bien m'en excuser), et enfin Benjamin C., Julien C. et Sarah F.
que je suis heureux de pouvoir appeler des amis.

\medskip

Merci à la Ludi et aux Ours pour tous ces moments d'improvisation, de rire et
de détente. Merci à la FNF, à la RCL et au Mans pour les moments d'effort et
de plaisir. Merci à mes tous amis non-mathématiques, en particulier Émilie,
Chloé F. et Nicolas P. pour leur soutien précieux et leur présence
réconfortante au cours de cette thèse.

Enfin, merci à mon frère Joris dont les exploits sportifs sont pour moi une
source d'inspiration, et à mère sans qui rien de tout cela ne serait arrivé.

\cleardoublepage
\endinput

% vim: spell spelllang=fr
