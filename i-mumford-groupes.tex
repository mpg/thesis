% !TEX root = main.tex

\chapter{Inégalité de \bsc{Mumford}} \label{chap:mumford}

\section{Énoncé principal}

On prouve ci-dessous une inégalité de \bsc{Mumford} explicite dans le cas
particulier où $\varapx$ est une sous-variété abélienne de \( \va \), qu'on
notera donc désormais $\vai$. Le résultat principal de ce chapitre est le
suivant.

\begin{thm} \label{Mumford}
  Soient $\vai$ une sous-variété abélienne de $\va$, $\p x$ et $\p y$ deux
  points de $\va(\Qbar)$ satisfaisant la condition d'approximation
  (\eqref{e:HA}) ci-dessus. On note $d = \deg \vai$, $l = \dim\vai + 1$ et on
  choisit $\phi > 0$ et $\rho > 0$ tels que $ld^2((\rho^2/4) + 2\phi +
  \rho\phi)< \eps$. On pose $B = (4^g-1)h(\va) + 3g\log(2)$ et $C =
  3\cdot4^gh(\va) + \log\Big( 10\sqrt2\cdot 2^{7g/2}
    (n+1)^6\big(\frac{16e\sqrt 2}{30}(n+1)^{9/2}\big)^{ld}\Big)$. On suppose
  en outre :
  \begin{enumerate}
    \item $\cos(\p{x},\p{y}) \ge 1-\phi$, \label{cone}
    \item $\Hautn(\p{x}) \le \Hautn(\p{y}) \le (1+\rho)\Hautn(\p{x})$,
      \label{sect}
    \item $\Hautn(\p{x}) > [h(\vai) + C + B(ld + \eps/d)] \cdot [\eps/d -
      ld((\rho^2/4) + 2\phi + \rho\phi)]^{-1}$\label{Loin}
  \end{enumerate}
  On a alors $\p x - \p y \in \vai(\Qbar)$.
\end{thm}

Les conditions sur $\p{x}$ et $\p{y}$ ont une interprétation géométrique très
simple dans l'espace de \bsc{Mordell-Weil}. En effet (\ref{cone}) demande que
les points se trouvent dans un même demi-cône d'angle petit, (\ref{sect})
précise qu'ils doivent se trouver dans un même secteur de petite largeur
exponentielle et (\ref{Loin}) que ce secteur est suffisamment loin de
l'origine. Le théorème~affirme alors qu'il y a, modulo $\vai(\Qbar)$, au plus
une approximation exceptionnelle dans un secteur de cône d'angle
$\arccos(1-\phi)$, de largeur exponentielle $1+\rho$, extérieur à une sphère
de rayon adéquat centrée en l'origine.


\section{Comportement métrique des opérations}

La situation d'approximation considérée met en jeu plusieurs types de
géométrie sur $\va$ : la géométrie euclidienne de son espace de
\bsc{Mordell-Weil}, la géométrie algébrique et projective de la variété
plongée, une géométrie métrique, locale, en chaque place de $S$. On s'attend à
ce que les métriques locales sur $\va$ se comportent de façon agréable
vis-à-vis des structures géométrique et arithmétique, par exemple que les
opérations soient uniformément continues, voire lipschitziennes. On montre ici
que c'est bien le cas et on explicite une valeur admissible pour la constante
de \bsc{Lipschitz}.

\begin{prop} \label{MetricOp}
  Il existe une constante $\cst{CMetricOp}$ telle que, pour tous $\p x$,
  $\p x'$, $\p y$, $\p y'$ dans $\va(\C_v)$, on a :
  \begin{equation}
  \max\big(\Distv(\p x \pm \p y, \p x' \pm \p y')\big) \le \cst{CMetricOp}
  \cdot \max \big(\Distv(\p x, \p x'), \Distv(\p y, \p y')\big) \pmm,
  \end{equation}
  où dans le membre de gauche il faut prendre le même signe sur les deux
  termes.  De plus, on peut choisir $\newcst{CMetricOp} = c_v(\va)^3 \av
  2^{-5g/2}[5\sqrt2\cdot2^{7g/2}(n+1)^{9/2}]^\dv$.
\end{prop}

Pour établir ce résultat, on va en fait étudier les propriétés métriques du
morphisme $\xi : \va^2 \to \va^2,\ (\p x, \p y) \mapsto (\p x + \p y, \p x -
\p y)$. Ce dernier présente en effet l'avantage de pouvoir être représenté
globalement par une famille $F$ de polynômes totalement explicite (en
particulier, de degré et hauteur contrôlés) à condition toutefois de regarder
$\va^2$ comme plongée dans $\Proj^N$ (où $N = n^2+2n$) par un plongement de
\bsc{Segre}. Si l'on note $\p z$ (resp. $\p z'$) l'image de $(\p x, \p y)$
(resp. $(\p x', \p y')$) par ce plongement, on est ainsi ramené à montrer que
$\Distv(\xi \p z, \xi \p z') \le \cst{CMetricOp} \sqrt2^{\,-\dv} \Distv(\p z,
\p z')$. Par ailleurs on pourra supposer $\Distv( \p z, \p z') \le \sqrt2\,
\cst{CMetricOp}^{-1}$, car sinon la conclusion du lemme~est vide, compte tenu
du fait que la distance est bornée par $1$. On utilisera alors un
développement de \bsc{Taylor} des formes $F$ représentant $\xi$ au voisinage
de $\p z$.

En comparant les termes d'ordres $1$ et $2$ au terme principal dans ce
développement, va apparaître une quantité $\Onv{F(z)} / (\Onnv{F}\Onv{z}^2)
\le 1$, qui détermine en grande partie le rayon sur lequel le développement de
\bsc{Taylor} donne une majoration utile. Il est donc indispensable de pouvoir
minorer cette quantité uniformément en $\p z$. Le lemme~suivant donne une
telle minoration en fonction de la constante locale $c_v(\va)$. Notons que
cette minoration est la seule étape de la preuve où intervient le fait que le
morphisme $\xi$ est donné par les opérations de $\va$, le reste s'appliquant
en fait à n'importe quel morphisme d'espaces projectifs.

\begin{lem} \label{TvIndepF}
  Il existe un choix de $F$ tel que pour tout $\p z \in \va(\C_v)$ on ait
  \begin{equation}
  \frac{\Onv{F(z)}}{\Onnv{F}\Onv{z}^2} \ge \cst{Tv} \pmm,
  \end{equation}
  avec $\newcst{Tv} = \av 2^{5g/2} 2^{-7g\dv/2} (n+1)^{-3\dv/2}
  c_v(\va)^{-3}$.
\end{lem}

\begin{proof}[Démonstration (du lemme~\ref{TvIndepF}).]
  On utilise le formulaire de la section~\ref{sec:form-ab2}.

  On commence par estimer $\Onnv{F}$. Chaque $F_{ij}$ s'écrit comme une somme
  de $4^g$ termes de norme $\Onv[v, \infty]{\truc}$ au plus $\av 2^{-g} \Onv
  \vai^2$ ; on a donc $\Onv {F_{ij}} \le 2^{g\dv} \av 2 ^{-g} \Onv \vai ^2$ et
  $\Onnv F \le  \sqrt{N+1}^\dv 2^{g\dv} \av 2 ^{-g} \Onv \vai ^2$. Pour la
  suite, on se base sur la minoration (\ref{NormDupl}) et le fait que $\xi^2$
  est le morphisme de multiplication par $2$ sur les deux facteurs. On a ainsi
  \begin{equation}
    F^2(\tilde\Delta)
    = (\xi^2)^*\tilde\Delta
    = [2]^* \tilde\Delta = G(\Delta) \otimes G(\Delta)
    \pmm.
  \end{equation}
  En appliquant ceci en un point $\p z = (\p x, \p y)$, on peut, par
  homogénéité remplacer $\Delta(\p x)$ par n'importe quel système de
  coordonnées $x$ de $\p x$ et faire de même pour $\p y$, puis choisir $z = x
  \otimes y$ comme représentant de $\p z$. Il vient alors :
  \begin{equation}
    \frac {\Onv{F^2(z)}} {\Onv z ^4}
    = \frac {\Onv{G(x)} \Onv{G(y)}} {\Onv x ^4 \Onv y ^4}
    \ge \av 2 ^{2g} 4^{-2g\dv} \Onv\coa ^{-6}
    \pmm.
  \end{equation}
  Or,
  \begin{equation}
    \frac {\Onv{F^2(z)}} {\Onv z ^4}
    \le \frac {\Onnv F \Onv{F(z)}^2} {\Onv z ^4}
    = \Onnv F ^3 \left( \frac {\Onv{F(z)}} {\Onnv F \Onv z ^2} \right) ^2
    \pmm,
  \end{equation}
  d'où (en se rappelant que $N+1 = (n+1)^2$)
  \begin{align*}
    \left(\frac {\Onv{F(z)}} {\Onnv F \Onv z ^2} \right)^2
    & \ge \av2^{2g} 4^{-2g\dv} \Onv\coa^{-6} \Onnv F ^{-3} \\
    & \ge \av2^{5g} 2^{-7d\dv} (n+1)^{-3\dv} c_v(\va)
  \end{align*}
  et le lemme.
\end{proof}

\begin{proof}[Démonstration (de la proposition~\ref{MetricOp}).]
  Il s'agit de montrer que $\Onv{F(z')}$ n'est pas trop petit devant $\Onv
  {F(z)}$ et que $\Onv{F(z) \wedge F(z')}$ n'est pas trop grand devant $\Onv
  {F(z)}^2$. Comme annoncé, on supposera que $\Distv (\p z, \p z') <
  \sqrt2^\dv\, \cst{CMetricOp}^{-1}$. Vu que $\cst{CMetricOp} = (5\sqrt2\cdot
  N\sqrt{N+1})^\dv / \cst{Tv}$, cela revient à supposer $\Distv (\p z, \p z')
  = \cst{Tv}\eps_1/(5 N \sqrt{N+1})^\dv$ pour un certain $\eps_1 < 1$. On peut
  en particulier appliquer le lemme~\ref{ChoixI} à $z$ et $z'$ (en remplaçant
  $n$ par $N$) et supposer, quitte à renuméroter les coordonnées, que
  $\av{z'_0} \ge (N+1)^{-\dv/2}\Onv{z'}$ et $z_0 \neq 0$. On écrit alors le
  développement de \bsc{Taylor} homogène de $F$ comme suit :
  \begin{equation}
    F(z') = \frac{(z'_0)^2}{z_0^2}F(z) + \sum_{1 \le i \le N} R_i + \sum_{1
      \le i, j \le N} R_{ij} \pmm,
  \end{equation}
  où
  \begin{equation}
    R_i = \frac{z'_0}{z_0} \frac{\partial F}{\partial Z_i}(z) \left( z'_i -
      \frac{z'_0}{z_0} z_i \right)
  \end{equation}
  et
  \begin{equation}
    R_{ij} = \frac 1 2 \frac{\partial^2 F}{\partial Z_i \partial Z_j}(z)
    \left( z'_i - \frac{z'_0}{z_0} z_i \right)\left( z'_j - \frac{z'_0}{z_0}
      z_j \right) \pmm.
  \end{equation}
  On utilise alors l'hypothèse sous la forme
  \begin{equation}
    \av{z'_i - \frac{z'_0}{z_0} z_i}
    \le \Onv{z \wedge z'}/\av{z_0}
    \le \frac{\cst{Tv}\,\eps_1}{(5N\sqrt{N+1})^\dv} \frac{\Onv z
      \Onv{z'}}{\av{z_0}} \pmm.
  \end{equation}
  Il vient alors, grâce au lemme~\ref{TvIndepF},
  \begin{align*}
    \frac{\Onv{R_i}}{\Onv{F(z) \cdot {z_0'}^2/z_0^2}}
    &\le \frac{\Onv{z'}}{\av{z'_0}}\frac{\Onnv{F} \Onv z^2}{\Onv{F(z)}}
    \frac{2^\dv \cst{Tv}\,\eps_1}{(5N\sqrt{N+1})^\dv} \\
    &\le \left(\frac2{5N\sqrt{N+1}}\right)^\dv \frac{\Onv{z'}}{\av{z'_0}}
    \eps_1 \le \left(\frac2{5N}\right)^\dv \eps_1 \pmm.
  \end{align*}
  On obtient de même
  \begin{equation}
    \frac{\Onv{R_{ij}}}{\Onv{F(z) \cdot {z_0'}^2/z_0^2}} \le
    \left(\frac1{25N^2}\right)^\dv \eps_1 \pmm,
  \end{equation}
  puis en sommant : $\Onv{\sum R_i + \sum R_{ij}} \le (1/2)^\dv \Onv{F(z)
    \cdot {z_0'}^2/z_0^2} \eps_1$. Enfin, les inégalités triangulaire aux
  places infinies et ultramétrique aux places finies donnent $\Onv{F(z')} \ge
  (1/2)^\dv \Onv{F(z) \cdot {z_0'}^2/z_0^2}$. Afin d'estimer la distance, il
  reste à majorer $\Onv{F(z) \wedge F(z')}$. Pour ce faire, on développe le
  second facteur, on remarque que le premier terme du produit est nul et on
  majore brutalement le terme restant par le produit des normes ; il vient
  ainsi :
  \begin{align*}
    \Distv(\xi(\p z), \xi(\p z'))
    & = \frac{\Onv{F(z) \wedge F(z')}}{\Onv{F(z)}\Onv{F(z')}} \\
    & \le \frac{2^\dv \Onv{\sum R_i + \sum R_{ij}} \Onv{F(z)}}{\Onv{F(z)}
      \Onv{F(z) \cdot {z_0'}^2/z_0^2}} \\
    & \le \eps_1 \pmm,
  \end{align*}
  qui, vu la définition de $\eps_1$, équivaut à la conclusion de la
  proposition.
\end{proof}


\section{Inégalité de \bsc{Liouville}}

On établit ici un analogue de l'inégalité bien connue de \bsc{Liouville}.
L'énoncé suivant est une application directe de la formule du produit :

\begin{prop} \label{PLiouvilleMal}
  Soit $V$ une sous-variété de $\Proj^n$ définie sur un corps de nombres
  $\cdn$ et $S$ un sous-ensemble fini de $M(\cdn)$. Pour tout point $\p z \in
  \Proj^n(\Qbar)$, on a soit $\p z \in V(\Qbar)$, soit
  \begin{equation}
    \prod_{v \in S} \Distv(\p z, V)^{e_v} \ge \frac1{H(V) H(\p z)^{ld}} \pmm,
  \end{equation}
  où $l = \dim V + 1$, $d = \deg V$ et $e_v = [\cdn_v : \Q_v] / [\cdn : \Q]$
  est le degré local relatif, de sorte que le membre de gauche est bien défini
  sur $\Qbar$.
\end{prop}

\begin{proof}
  On remarque en effet que
  \begin{equation}
    \prod_{v \in M(\cdn)} \Distv(\p z, V)^{e_v} = \prod_{v \in M(\cdn)}
    \frac{\mv{\md_z f}} {\mv f \Onv z^{ld}} = \frac{H(\md_z f)}{H(V) H(\p
      z)^{ld}} \pmm.
  \end{equation}
  Or, d'une part cette quantité est majorée par $\prod_{v \in S} \Distv(\p z,
  V)^{e_v}$ (la distance est bornée par $1$) et d'autre part $H(\md_z f) \ge
  1$ par la formule du produit dès que $\p z \not\in V(\Qbar)$.
\end{proof}

Cette version évidente de l'inégalité présente l'inconvénient de faire
apparaître la dimension de $V$ plus $1$ en exposant de $H(\p z)$. On peut s'en
affranchir, modulo une certaine constante ne dépendant que de la géométrie de
$V$ plongée. En effet, \bsc{Rémond} a donné \cite[prop.~6.1]{remdcl} une
version explicite de la proposition~2.10 de \cite{faldaav}, disant qu'on peut
définir $V$ par des équations de degré $d$ et de hauteur majorée de façon
explicite.

Nous allons montrer une version métrique de cette proposition. Plus
précisément, les résultats de \bsc{Rémond} et \bsc{Faltings} reposent sur le
fait que, génériquement, par une projection linéaire $\Proj^n \to \Proj^l$,
l'image de $V$ est une hypersurface de degré $d$. On tire ensuite en arrière
une équation de cette hypersurface pour obtenir une des équations de $V$
recherchées. Dans notre cas, il faudra comme chez \bsc{Rémond} contrôler la
hauteur des projections, mais également leur action sur la distance. Le
principal résultat technique de cette section s'énonce ainsi :

\begin{lem} \label{lProjection}
  Soient $V \in \Proj^n$ une variété de degré $d$, de dimension $l-1$, définie
  sur un corps de nombres $\cdn$ et $S$ un ensemble fini de places de $\cdn$.
  Il existe une constante $\cst{Cproj} = \cst{Cproj}(d, l, n, \abs S)$ telle
  que, pour tout point $\p z \in \Proj^n(\Qbar) \setminus V(\Qbar)$, il existe
  une projection linéaire $\pi : \Proj^n \setminus \zeros(\pi) \to \Proj^l$
  (où $\zeros(\pi)$ désigne le centre de la projection) telle que :
  \begin{enumerate}
    \item $\pi$ est surjective,
    \item $\zeros(\pi) \cap V = \emptyset$,
    \item $\p z \not\in \zeros(\pi)$,
    \item $\pi(\p z) \not\in \pi(V)$,
    \item $\forall v \in S,\ \Distv\big(\pi(\p z), \pi_* V)\big)
      \le \cst{Cproj} \cdot \Distv(\p z, V)$.
  \end{enumerate}
  Plus précisément, on peut choisir $\pi$ définie par des formes linéaires à
  coefficients entiers de valeur absolue inférieure ou égale à
  % $(2ld + d + l + 1 + 2\abs sld)/2$
  $ (\abs S /2 +2)(ld+1)$ et prendre
  \begin{equation}
    \newcst{Cproj} = \left( \frac {(l+1)^{9/2} (n+1)^3}4 \right)^{ld\dv} \big(
      (\abs S /2 +2) (ld+1) \big)^{2(ld+1)^{(l+1)(n+1)}} \pmm.
  \end{equation}
\end{lem}

Pour la démonstration, commençons par introduire quelques notations. Appelons
$Y = (Y_0,\ldots, Y_l)$ les coordonnées homogènes sur $\Proj^l$ et notons $v =
\big(v_p^{(k)}\big)_{0\le p \le l}^{1 \le k \le l}$ les coordonnées duales sur
la puissance $l$-ième de $(\Proj^l)^\vee$. On introduit enfin des groupes de
variables $t = \big(t_{pq}^{(k)}\big)_{0\le p, q \le l}^{1 \le k \le l}$ et
pour chaque anneau de polynômes contenant les variables $t$, l'idéal $I_t$
engendré par les $t_{pq}^{(k)} + t_{qp}^{(k)}$. La projection $\pi$ provient
d'un morphisme $\tilde\pi : \cdn[Y] \to \cdn [X]$ donné par $Y_p \mapsto
\sum_i m_i^{(p)} X_i$. On s'intéresse par la suite au cas où les $m_i^{(p)}$
sont entiers, de valeur absolue majorée par un certain $\Delta$. On note enfin
$m = \big(m_i^{(p)}\big)_{0\le i \le n}^{0 \le p \le l}$.

Par dualité, $\tilde\pi$ fournit un morphisme $\big(\tilde\pi^\vee
\big)^{\otimes l} : \cdn[u] \to \cdn[v]$, qu'on notera $\mu$, donné par
$u_i^{(k)} \mapsto \sum_p m_i^{(p)} v_p^{(k)}$. Il est immédiat à partir de la
définition que $\mu$ envoie la forme de \bsc{Chow} d'une variété sur celle de
son
image, c'est-à-dire que $\mu(f_V) = f_{\pi_* V}$. Par ailleurs, le
comportement de $\mu$ par rapport au morphisme $\md$ est se lit sur le
diagramme suivant :
\begin{equation}
  \xymatrix{
    \cdn[u] \ar@/^4ex/[rrr]^{\md_x} \ar[r]_-\md \ar[d]^\mu & \cdn[s,X]/I_s
    \ar[rr]_{ev_x} \ar@{.>}[dr]|-{\mu_2} & & \cdn[s]/I_s \ar[d]^{\mu_2}
    \\
    \cdn[v] \ar@/_4ex/[rrr]_{\md_{\pi(x)}}\ar[r]^-\md & \cdn[t,Y]/I_t
    \ar[r]^{\tilde\pi} & \cdn[t, X]/I_t \ar[r]^{ev_x} & \cdn[t]/I_t
  }
\end{equation}
où l'on a noté $ev_x$ le morphisme d'évaluation en $x$, et où $\mu_2$ est
donné par $s_{ij}^{(k)} \mapsto \sum_{p, q} m_i^{(p)} m_j^{(q)} t_{pq}^{(k)}$.
Un simple calcul montre que $\mu_2$ est bien défini et que le diagramme
commute. Ainsi, de même que $f_{\pi_* V}$ s'obtient en spécialisant $f_V$ en
certaines formes linéaires (c.-à-d. en prenant son image par $\mu$),
$\md_{\pi(x)} f_{\pi_* V}$ est aussi une spécialisation de $\md_x f_V$ (image
par $\mu_2$). On utilise ce denier fait pour établir le lemme suivant, mais on
retiendra également que les coefficients de $f_{\pi_* V}$ sont des polynômes
homogènes de degré $ld$ en $m$.

\begin{lem} \label{lNumerateur}
  Dans les notations précédentes, on a $\mv{\md_{\pi(x)} f_{\pi_* V}} \le
  \cst{Cmudxf} \cdot \mv{\md_x f_V}$, avec
  \begin{equation}
    \newcst{Cmudxf}
    = \cst{Cmudxf}(\Delta, n, l, d)
    = \left( \frac{l(l+1)n(n+1)}4 e^{\gamma_{l+1}} \Delta^2 \right)^{ld\dv}
    \pmm.
  \end{equation}
\end{lem}

\begin{proof}
  Aux places finies, le résultat est immédiat compte tenu du caractère entier
  de $m$. Aux places archimédiennes, on utilise la norme $L_1$ (longueur) des
  polynômes, qui se comporte bien par spécialisation :
  \begin{equation}
    \Onv[1]{\md_{\pi(x)} f_{\pi_* V}} \le \left( \frac{l(l+1)}2 \Delta^2
    \right)^{ld} \Onv[1]{\md_x f_V} \pmm,
  \end{equation}
  car on a substitué des formes de longueur $\le \Delta^2 l(l+1)/2$ dans la
  forme $\md_x f_V$ multihomogène de degré $d$ en $l$ groupes de variables. On
  compare alors la mesure à la longueur : $\Onv[1]{\md_x f_V} \le
  \big(n(n+1)/2\big)^{ld} \mv{\md_x f_V}$ et enfin
  \begin{equation}
    \mv{\md_{\pi(x)} f_{\pi_* V}}
    \le e^{ld\gamma_{l+1}} \Onv[1]{\md_{\pi(x)} f_{\pi_* V}} \pmm.
  \end{equation}
  Le lemme suit en concaténant ces trois inégalités.
\end{proof}

\begin{lem} \label{lCramer}
  Soit $R$ une matrice $r \times r$ inversible à coefficients entiers, de
  valeur absolue inférieure ou égale à $C$, et $v$ une place de $\Q$. Alors
  $\Onv{R^{-1}} \le r! \cdot C^{r-\dv}$, où $\Onv{\truc}$ désigne la norme du
  sup des coefficients aux places finies et la norme euclidienne de la famille
  des coefficients sinon.
\end{lem}

\begin{proof}
  On utilise les formules de \bsc{Cramer}. À la place archimédienne, on minore
  $\det R$ par $1$ et par $(r-1)! C^{r-1}$ la norme de chaque cofacteur. La
  norme de la comatrice et donc de l'inverse est ainsi majorée par
  $(r-1)(r-1)! C^{r-1} \le r! C^r$. Aux places finies, la comatrice est à
  coefficients entiers, et on majore $\av{\det R}^{-1}$ par $\av{\det
    R}[\infty] \le r! C^r$ par la formule du produit.
\end{proof}

Si les conditions~1 à~4 de la conclusion du lemme~\ref{lProjection} peuvent
être satisfaites en exigeant la non-annulation de certains polynômes en $m$,
comme chez \bsc{Rémond}, en revanche, la condition~5 s'obtient plutôt en
choisissant une valeur de $m$ en laquelle certains polynômes ne prendront pas
une valeur \og trop petite\fg{} aux places de $S$. L'objet du lemme suivant
est d'énoncer des conditions suffisantes pour qu'un tel choix soit possible.

\begin{lem} \label{lCube}
  Soient $N$, $D$ et $b$ des entiers naturels non nuls. On note
  $\mathcal{C}(D) = (\Z \cap [-D, D])^N \subset \Z^N$. Soient par ailleurs
  $\cdn$ un corps de nombres et $v$ une place de $\cdn$. Il existe une
  constante $\cst{cCube} = \cst{cCube}(D, N, b)$ telle que, pour tout
  polynôme $B$ de degré $b$ en $N$ variables, il existe une hypersurface $H$
  de degré $b$ qui contient tout les points $w \in \mathcal C(D)$ tels que
  \begin{equation}
    \av{B(w)} < \cst{cCube} \cdot \Onv{B} \pmm.
  \end{equation}
  De plus, posant $M = M(N, b) = \binom{N+b-1}{b-1}$, on peut prendre
  $\newcst{cCube} = M^{-\dv/2}\cdot M!^{-1} D^{-bM}$.
\end{lem}

\begin{proof}
  On procède par linéarisation du problème grâce à un plongement de
  \bsc{Veronese} de degré $b$. Pour chaque $w \in \mathcal C(D)$, on considère
  le vecteur colonne $w'$ formé des monômes de degré $b$ en $w$ et le vecteur
  ligne $B'$ des coefficients de $B$, ordonnés de telle sorte que $B(w) = B'
  \cdot w'$. Ces vecteurs $w'$ ont $M = \binom{N+b-1}{b-1}$ coordonnées,
  entières, bornées par $D^b$. On considère $W = (w_1, \dots, w_M)$ une
  famille d'éléments de $\mathcal{C}(D)$ et on note $W'$ la matrice carrée
  dont les colonnes sont $w'_1, \dots, w'_M$. Posant $B(W) = (B(w_1), \dots,
  B(w_n))$, on a ainsi $B(W) = B'W'$.

  Supposons maintenant que $W'$ est inversible. On a donc ainsi $B' =
  B(W)\cdot(W')^{-1}$, puis $\Onv{B(W)} \ge \Onv{(W')^{-1}}^{-1} \Onv{B'}$. Il
  existe ainsi par le lemme~\ref{ChoixI} un indice $i_v$ tel que
  \begin{equation}
    \Onv{B(w_{i_v})} \ge \sqrt{M}^{\,-\dv} \Onv{(W')^{-1}}^{-1} \Onv{B} \ge
    (\sqrt M^\dv \cdot M!)^{-1}D^{-bM} \Onv{B}
  \end{equation}
  par le lemme~\ref{lCramer}. Or le membre de droite est précisément
  $\cst{cCube} \Onv{B}$.

  Par contraposée, si $w_1, \dots, w_M$ sont dans $\cube{D}$ et que pour tout
  $i$ on a $\av{B(w_i)} < \cst{cCube} \Onv{B}$, alors $w_1', \dots, w_M'$ sont
  linéairement liés : il existe ainsi un hyperplan de $\cdn^M$ contenant les
  images de tous les points satisfaisant à la condition du lemme.
  L'hypersurface $H$ recherchée est alors obtenue en tirant en arrière cet
  hyperplan.
\end{proof}

\begin{rem}
  Comme on sait par ailleurs \cite[rem. précédant la prop.~4.1]{remivds} qu'on
  peut choisir $w$ dans $\cube{D}$ n'annullant pas un polynôme donné de degré
  $b$ dès que $D \le b/2$, on peut utiliser le lemme précédent pour trouver
  des points entiers de taille contrôlée qui ne sont pas \og trop près \fg{}
  d'une hypersurface donnée.
\end{rem}

\begin{proof}[Démonstration du lemme~\ref{lProjection}]
  On fixe $z$ et on cherche à interpréter les conditions à satisfaire comme un
  système d'inéquations polynomiales (au sens du lemme~précédent) en $m$.
  D'après \bsc{Rémond}, il existe un polynôme $P_z$ homogène de degré $d$ en
  chacun des $l+1$ groupes de variables $m^{(p)}$ (obtenu en spécialisant en
  $z$ le polynôme $P(M, \truc)$ de la preuve de \cite[prop.~6.2]{remdcl}), tel
  que $P_z(m) = 0$ si et seulement si $\pi(\p z) \in \pi(V)$ lorsque cela a un
  sens. On pose alors
  \begin{equation}
    A_z(m) = f_V(m^{(1)}, \dots, m^{(l)}) \cdot \det((m_i^{(p)})_{0 \le i,p
      \le l}) \cdot P_z(m) \pmm.
  \end{equation}
  C'est un polynôme homogène de degré global $2ld + l + d + 1$ ; les
  conditions~1, 2 et~4 de la conclusion du lemme sont satisfaites dès qu'il ne
  s'annule pas.

  Par ailleurs, pour chaque place $v$ de $S$, on fixe $g_v$ un coefficient (vu
  comme polynôme en $m$) de $f_{\pi_* V}$ tel que $\Onv{g_v} \ge \binom{n+d}d
  ^{-l\dv/2} \Onv{f_V}$, ce qui est possible par une variante du
  lemme~\ref{ChoixI} vu le nombre de tels coefficients. On regarde aussi $h_z
  = \sum m_i^{(0)}z_i$ comme une forme linéaire en $m$. On pose alors $\delta
  = \big(2ld + l + d + 1) + \abs{S}(ld + 1)\big)/2$ et par le
  lemme~\ref{lCube}, pour chaque $v \in S$, on associe à $g_v$ (resp. $h_z$)
  une forme $G_v$ de degré $ld$ (resp. une forme linéaire $H_{z, v}$) telle
  que pour tout $m \in \cube\delta$
  \begin{equation} \label{eAppCube}
    \begin{aligned}
      \av{g_v(m)} < \cst{cCube}(\Delta, (l+1)(n+1), ld) \Onv{g_v}
      & \Longrightarrow G_v(m) = 0 \\
      \av{h_{z, v}(m)} < \cst{cCube}(\Delta, (l+1), 1) \Onv{h_{z, v}}
      & \Longrightarrow H_{z, v}(m) = 0 \pmm.
    \end{aligned}
  \end{equation}
  Le polynôme $A_z\prod_v G_v H_{z, v}$ étant de degré $2\delta$, on peut
  choisir un point $m$ de $\cube{\delta}$ ne l'anullant pas ; on fixe
  désormais un tel $m$. En particulier, $h_z(m) \neq 0$ assure que $\pi(\p z)
  \not\in \pi(V)$, et comme $A_z(m) \neq 0$, les points~1 à~4 sont vérifiés.
  Pour le point~5, on utilise le lemme~\ref{lNumerateur} et les minorations
  déduites de~\eqref{eAppCube}. Pour le calcul de $\cst{cCube}$, on utilisera
  la majoration $\Delta \le (ld+1)(\abs{S}/2 + 2)$ et le fait que $\sqrt M \,
  M! \le M^M$ pour tout entier $M$.

  On a $\Onv{\pi(z)} \ge \av{h_z(m)}$ et $\Onv{h_z} = \Onv z$ par
  construction, et le choix de $m$ donne
  \begin{equation}
    \Onv z \le \cst{cCube} (\Delta, (l+1)(n+1), ld) \Onv{\pi(z)} \pmm,
  \end{equation}
  d'où
  \begin{equation} %\label{epiz}
    \Onv z
    \le \Onv{\pi(z)} \left( (l+1)(ld+1)(\abs{S}/2 + 2) \right)^{ld(l+1)}
    \pmm.
  \end{equation}
  On procède de même pour comparer $\mv{f_v}$ à $\mv{f_{\pi_* V}}$, en
  utilisant de plus le fait que $\mv{f_v} \le \Onv{f_v}
  \exp(ld\gamma_{n(n+1)/2}\dv)$ et $\Onv{f_{\pi_* V}} \le \mv{f_{\pi_* V}}
  \big( l(l+1)/2 \big)^{ld\dv}$. Il vient
  \begin{multline} % \label{epif}
    \mv{f_v} \le \mv{f_{\pi_* V}} \left( l(l+1)\sqrt{n(n+1)} \right)^{ld\dv}
    \\
    \cdot \left( (ld+1)^{(l+1)(n+1)+1} (\abs{S}/2 + 2)
    \right)^{(ld+1)^{(l+1)(n+1)}}
    \pmm.
  \end{multline}
  Or d'après le lemme~\ref{lNumerateur}, on a
  \begin{align} %\label{epimdf}
    \mv{\md_{\pi(x)}f_{\pi_* V}}
    & \le \cst{Cmudxf}(\Delta, n, l, d) \mv{\md_x f_V} \notag \\
    & \le \left( \frac{l(l+1)n(n+1)}4 e^{\gamma_{l+1}} (\abs S/2 +2)
    \right)^{ld\dv} \mv{\md_x f_V}  \pmm.
  \end{align}
  La conclusion suit en substituant ces trois dernières inégalités dans la
  définition de $\Distv(\pi(\p z), \pi_* V)$.
\end{proof}

\begin{coro}
  Si $\pi$ est choisie comme au lemme \ref{lProjection}, on a
  \begin{gather*}
    H(\pi(\p z))
    \le (n+1)^{3/2} (\abs S /2 +2) (ld+1) H(\p z) \\
    H(\pi_* V)
    \le \bigl(l(n+1)^2 \cdot e \cdot (\abs S /2 +2) (ld+1) H'V\bigr)
    \pmm.
  \end{gather*}
\end{coro}

\begin{proof}
  On utilise simplement le fait que $\pi$ est donnée par une matrice entière à
  coefficients majorés en valeur absolue par $(\abs S /2 +2) (ld+1)$ pour la
  première inégalité. Pour la deuxième, on utilise également le fait qu'une
  forme de \bsc{Chow} de $\pi_* V$ s'obtient en spécialisant une forme de
  \bsc{Chow} de $V$ en des formes linéaires à coefficient entiers également
  bornés par $(\abs S /2 +2) (ld+1)$, comme au lemme \ref{lNumerateur}.
\end{proof}

On peut alors démontrer notre deuxième version de l'inégalité de
\bsc{Liouville}.

\begin{prop} \label{pLiouvilleBien}
  Sous les mêmes hypothèses et notations qu'à la proposition
  \ref{PLiouvilleMal}, on a soit $\p z \in V(\Qbar)$, soit
  \begin{equation}
    \prod_{v \in S} \Distv(\p z, V)^{e_v} \ge \cst{cLiouville} \frac1{H(V)
      H(\p z)^d} \pmm,
  \end{equation}
  avec
  \begin{multline}
    \newcst[]{cLiouville}^{-1} = (n+1)^{3/2} \big( (l+1)^{11/2} (n+1)^5
    \big)^{ld} \\
    \cdot \big( (\abs S /2 +2)(ld +1) \big)^{\abs S(1 + 2(ld
      +1)^{(l+1)(n+1)})}
  \end{multline}
\end{prop}

\begin{proof}
  Si $\p z \not\in V(\Qbar)$, on choisit par le lemme~\ref{lProjection} une
  bonne projection $\pi$. On choisit une équation $E$ de $\pi_* V$, qui est
  donc de degré $d = \deg(V)$. En utilisant l'expression de la distance d'un
  point à une hypersurface, ainsi que de la hauteur de cette dernière, en
  fonction d'une de ses équations, il vient :
  \begin{align*}
    \prod_{v \in S} \Distv(\p z, V)^{e_v}
    & \ge \prod_{v \in S} \cst{Cproj}^{-e_v} \Distv(\pi(\p z), \pi_* V)^{e_v}
    \\ & \ge \prod_{v \in S} \cst{Cproj}^{-e_v}
    \cdot \prod_{v \in M(\cdn)}
    \left(\frac {\av{E(\p z)}} {\mv{E} \Onv{\pi(z)}^d} \right)^{e_v}
    \\ & \ge \prod_{v \in S} \cst{Cproj}^{-e_v}
    \frac1{H(\pi_* V) H(\pi(\p z))^d}
    \\ & \ge \prod_{v \in S} \cst{Cproj}^{-e_v}
    \frac1{H(V) H(\p z)^d}
    \\ & \qquad \cdot \frac1 {
      (n+1)^{3/2} \big( l(n+1)^2 \cdot e \big)^{ld} (\abs S /2 +2)(ld +1)
    }
    \\ &\ge \cst{cLiouville} \frac1{H(V) H(\p z)^d}
    \qedhere
  \end{align*}
\end{proof}


\section{Inégalité de \bsc{Mumford}}

La stratégie de la preuve est la suivante : si on a deux approximations
exceptionnelles, suffisamment proches dans l'espace de \bsc{Mordell-Weil} (au
sens des hypothèses (\ref{cone}) et (\ref{sect}) du théorème~\ref{Mumford}) et
de hauteur assez grande, on fabrique, (en prenant leur différence, sous
l'hypothèse que $\vai$ est un sous-groupe) une approximation de qualité telle
qu'elle appartiendra nécessairement à $\vai$, d'après une des inégalités de
\bsc{Liouville} énoncées à la section précédente.

\begin{lem} \label{Precis}
  Si $\p z = \p x - \p y$, où $\p x$ et $\p y$ satisfont l'hypothèse
  (\eqref{e:HA}) du théorème~\ref{Mumford}, on a $\prod_{v \in S} \Distv(\p z,
  \vai)^{e_v} \le \cst{CPrecis}H^{-\eps/d}$, avec
  \begin{equation}
    \newcst[]{CPrecis} = H(\va)^{3\cdot4^g} 10\sqrt2\,(n+1)^6
    2^{7g/2}\left(\frac{(2n^2+1)(n+1)^2}{2n} e^{\gamma_{\frac{(n+1)n}{2}} +
        \gamma_{n+1}} \right)^{ld}
  \end{equation}
  et $H = \min\big(H(\p x), H(\p y)\big)$.
\end{lem}

\begin{proof}
  On a par hypothèse $\max(\Distv(\p x, \vai), \Distv(\p y, \vai)) \le
  H^{-\lambda_v \eps}$.  Le fait~\ref{ClosestPoint} assure alors qu'il existe
  des points $\p x',\p y' \in \vai(\C_v)$ tels que
  \begin{equation}
    \max(\Distv(\p x, \p x'), \Distv(\p y, \p y'))
    \le H^{-\lambda_v \eps/d} e^{\dv \gamma_{n+1}}
    \pmm.
  \end{equation}
  On a alors en utilisant successivement le lemme~\ref{PonctuelAlg} et la
  proposition~\ref{MetricOp} :
  \begin{align*}
    \Distv(\p z, \vai)
    & \le \cst{CPonctuelAlg}\, \Distv(\p x - \p y, \p x' - \p y') \\
    & \le \cst{CPonctuelAlg}\, \cst{CMetricOp}\,
    \max(\Distv(\p x, \p x'), \Distv(\p y, \p y')) \\
    & \le \cst{CPonctuelAlg}\, \cst{CMetricOp}\,e^{\dv
      \gamma_{n+1}}H^{-\lambda_v \eps/d} \pmm.
  \end{align*}
  On prend alors le produit sur $v \in S$, en supposant (c'est le cas
  défavorable) que $S$ contient toutes les places divisant $2$ ou $\infty$.
  Comme $c_v(\va) \ge 1$, on a $\prod_{v \in S} c_v(\va) \le \prod_{v \in
    M(\cdn)} c_v(\va) \le H(\va)^{4^g}$. Vu la normalisation des $\lambda_v$,
  on a ainsi la conclusion du lemme.
\end{proof}

\begin{lem} \label{Petit}
  Soient $\p{x}$ et $\p{y}$ satisfaisant aux hypothèses (\ref{cone}) et
  (\ref{sect}) du théorème~\ref{Mumford}, notons $\p z$ leur différence. On a
  alors $\Hautn(\p z) \le (\rho^2/4 + 2\phi + \rho\phi) \Hautn(\p{x})$.
\end{lem}

\begin{proof}
  On note $\langle\truc; \truc \rangle$ le produit scalaire et $\Onv[N]{\truc}
  = \sqrt{\Hautn(\truc)}$ la norme dans l'espace de \bsc{Mordell-Weil}. On
  remarque de plus que l'hypothèse \ref{sect} implique $\Onv[N]{\p{y}} \le
  (1+\rho)^{1/2}\Onv[N]{\p{x}} \le (1+\rho/2) \Onv[N]{\p{x}}$. Il vient alors :
  \begin{align*}
    \Hautn(\p{z})
    & = \Onv[N]{\p{x}-\p{y}}^2 \\ & = \Onv[N]{\p{x}}^2 + \Onv[N]{\p{y}}^2 -
    2\langle \p{x}; \p{y} \rangle \\
    & = \left(\Onv[N]{\p{y}} - \Onv[N]{\p{x}}\right)^2 +
    2\Onv[N]{\p{x}}\Onv[N]{\p{y}}\left( 1- \cos(\p{x}, \p{y}) \right) \\
    & \le \left( \frac{\rho}{2}\Onv[N]{\p{x}} \right)^2 + 2\left(
      1+\frac{\rho}{2} \right) \Onv[N]{\p{x}}^2 \cdot \phi \\
    & \le ((\rho^2/4) + 2\phi + \rho\phi) \Hautn(\p{x})\pmm{.}\qedhere
  \end{align*}
\end{proof}

\begin{proof}[Démonstration (du théorème~\ref{Mumford}).]
  Elle consiste, en utilisant conjointement  les conclusions des lemmes
  précédents, à montrer que $\p z$ ne satisfait pas à la conclusion la
  proposition~\ref{PLiouvilleMal} dès que $\p x$ satisfait l'hypothèse
  (\ref{Loin}) du théorème~\ref{Mumford}.  On a besoin de pouvoir comparer les
  hauteurs projective et normalisée. On utilise à cet effet le lemme~3.9 de
  \cite{daphimhva2} et la remarque subséquente : $\lvert\Hautn(\truc) -
  h(\truc)\rvert \le B$, avec $B$ comme dans l'énoncé du théorème. En
  injectant ceci dans le lemme~\ref{Precis}, on a $\sum_{v \in S} e_v
  \log\Distv(\p z, \vai) \le -(\eps/d) \Hautn(\p x) + B\eps/d +
  \cst{CPrecis}$. De même, le lemme~\ref{Petit} donne $h(\p z) \le ((\rho^2/4)
  + 2\phi + \rho\phi) \Hautn(\p x) + B$. Ces deux estimations contredisent la
  conclusion de la proposition~\ref{PLiouvilleMal} dès que
  \begin{equation}
    \big(\eps/d - ld((\rho^2/4) + 2\phi + \rho\phi)\big) \Hautn(\p x)
    > h(\vai) + \cst{CPrecis} + B(ld + \eps/d)
    \pmm.
  \end{equation}
  Or, un calcul facile montre que la constante $C$ du théorème majore
  $\cst{CPrecis}$ et que la relation précédente est impliquée par
  l'hypothèse (\ref{Loin}) du théorème ; cette dernière oblige donc $\p z$ à
  être sur $\vai$.
\end{proof}

\emph{Remarque.} En utilisant la proposition~\ref{pLiouvilleBien} en lieu et
place de la proposition~\ref{PLiouvilleMal} ci-dessus, on obtient une variante
du théorème, avec la même conclusion mais en affaiblissant la condition
$ld^2((\rho^2/4) + 2\phi + \rho\phi)< \eps$ en $d^2((\rho^2/4) + 2\phi +
\rho\phi)< \eps$ et en remplaçant la condition 3 par
\begin{equation}
  \Hautn(\p{x}) > [h(\vai) + C + \log \cst{cLiouville} + B(d + \eps/d)] \cdot
  [\eps/d - d((\rho^2/4) + 2\phi + \rho\phi)]^{-1}
\end{equation}

\endinput

% vim: spell spelllang=fr
