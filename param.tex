% préambule général pour tous les trucs préparatoire à la thèse
% (en attendant un classe ou une extension un peu plus propre ?)
%

\usepackage{mathtools, amsmath, amsthm}
\usepackage[all]{xy}

\usepackage{ifmtarg, fixltx2e, xargs}

\usepackage{enumitem}

\usepackage{fancyhdr} \pagestyle{fancy}
\renewcommand\headrulewidth{0pt} \setlength\headheight{0pt}
\fancyhead{} \fancyfoot{}
\fancyfoot[C]{\thepage}
% \fancyfoot[RO, LE]{\today}

\usepackage{xspace}
\usepackage[british, frenchb]{babel}
\usepackage[babel=true, expansion=false]{microtype}
\frenchbsetup{AutoSpacePunctuation=false}

\newcommand*\notemarge[1]{\marginpar[\raggedleft #1]{\raggedright #1}}
\newcommand\todotext{\textsc{todo}}
\makeatletter
  \newcommand*\todo[1][]{%
    \leavevmode\notemarge{\todotext}%
    \@ifnotmtarg{#1}{[#1]}}
  \newcommand*\todom[1][]{\tag{\todotext%
    \@ifnotmtarg{#1}{ : #1}}}
\makeatother
%\newcommand\todo{\TextOrMath{\todot}{\todom}}

\newenvironment{enumthm}
  {\begin{enumerate}[label=(\textit{\roman*})]}
  {\end{enumerate}}

\newcommand*\lat[1]{\emph{#1}}
\newcommand*\eng[1]{%
  \foreignlanguage{english}{\emph{#1}}}
\newcommand*\defn[1]{\emph{#1}}
\newcommand*\pmm[1]{\text{ #1}}

\makeatletter
\newcommand*\ssub[1]{\@ifnotmtarg{#1}{_{#1}}}
\newcommand*\ssup[1]{\@ifnotmtarg{#1}{^{#1}}}
\newcommandx*\pexp[2][2]{%
  \@ifmtarg{#2}%
    {\cramped{^{(#1)}}}%
    {^{(#1)}_{#2}}}
\makeatother

\newcommand\suchthat{\ \middle\vert\ }

\newcommand*\std[1]{\mathbf{#1}} \newcommand\N{\std N} \newcommand\Z{\std Z}
\newcommand\Q{\std Q} \newcommand\R{\std R} \newcommand\C{\std C}
\newcommand\Proj{\std{P}} \newcommand\Aff{\std{A}} 
\newcommand\Qbar{\overline{\Q}}
\newcommand\cdn{\boldsymbol{k}} \newcommand\Cdn{\boldsymbol{K}}
\newcommand*\I[1]{\std{Id}_{#1}} \newcommand*\ind[1]{\std{1}_{#1}}
\renewcommand\ge{\geqslant} \renewcommand\le{\leqslant}
\newcommand\orbrack{\mathopen\rbrack} \newcommand\clbrack{\mathopen\lbrack}

\newcommand*\abs[1]{\left\lvert#1\right\rvert}
\newcommand*\norm[1]{\left\lVert#1\right\rVert}
\newcommand*\nnorm[1]{%
  \left\lvert\hspace{-1pt}\left\lvert\hspace{-1pt}%
  \left\lvert#1\right\rvert
  \hspace{-1pt}\right\rvert\hspace{-1pt}\right\rvert}

\newcommand\eps{\varepsilon}
\newcommand\truc{{\,\cdot\,}}
\DeclareMathOperator\disc{Disc}
\DeclareMathOperator\ord{ord}
\DeclareMathOperator\Div{div}

\newcommand\zeros{\mathcal Z}
\newcommand\ideal{\mathcal I}

\newcommand\mmax{{\mathrm{max}}}

\newcommand\diff{\mathrm d}

% \newcommand*\av[2][v]{\abs{#2}_{#1}}      % valeur absolue v-adique, ou indice en option
% \newcommand*\nv[2][v]{\norm{#2}_{#1}}       % norme v-adique, idem
% \newcommand*\nnv[2][v]{\nnorm{#2}_{#1}}     % norme v-adique d'une famille, idem
% \newcommand*\nvp[2][v]{\lVert#2\rVert_{#1}}     % norme idem mais en forçant une petite taille
\newcommand*\mv[2][v]{M_{#1}(#2)}     % mesure homogène en une place v ou autre
% \newcommand*\mahler[2][]{\mathcal M_{#1}(#2)}   % mesure de Mahler
% \newcommand\dv{{\delta\smash{_v}}}        % sélecteur v ultramétrique ou non
\newcommand\dpv{{\delta'_v}}        % idem plus places divisant 2
% \newcommand\Dv{\mathrm{dist}_v}       % distance v-adique projective
% \newcommand\A{\mathcal A}         % la variété abélienne
\newcommand\B{\mathcal B}       % une sous-v.a.
% \newcommand*\p[1]{\boldsymbol{#1}}      % un point ; on notera en « normal » ses coordonnées
% \newcommand\OA{\p 0}          % l'origine de la variété
% \newcommand\coa{\theta}         % des coordonnées de l'origine
% \newcommand\BA{\mathfrak b}         % vecteur "B" avec coordonnées inverses de l'origine
% \newcommand\hn{\smash{\hat h}}        % la hauteur normalisée
% \newcommand*\lgr[1]{{\lvert#1\rvert}}     % longueur d'un multi-indice
% \newcommand*\vlg[1]{\lgr#1}         % multi-longueur partielles d'un multi-multi-indice
\newcommand*\lgt[1]{\lVert#1\rVert}     % longueur totale
\newcommand\md{\mathfrak{d}}        % le morphisme « d »
\newcommand*\MW[2][\A]{MW_{\!#1}(#2)}     % l'espace de mordell-weil de #2 sur #1
\newcommand*\hatxs{\smash{\hat x}}      % x chapeau rapetissé
\newcommand*\hatys{\smash{\hat y}}      % y chapeau rapetissé
\newcommand*\cube[1]{\mathcal C(#1)}      % le « cube » \abs{w} \le #1 et w entier

\newcounter{cst}
\newcommand*\cst[2][, v]{%
  \ensuremath{c\ssub{\ref{#2}#1}}}
\newcommand*\newcst[2][, v]{%
  \refstepcounter{cst}\label{#2}%
  \cst[#1]{#2}}

\newcommand\iq{{\mathfrak q}}

\newcommand\vb{{\mathbf e}}
\newcommand\Mbase{{\mathcal M}}
\newcommand\monome{{\mathfrak m}}


\newtheorem{thm}{Théorème} \newtheorem{prop}[thm]{Proposition}
\newtheorem{lem}[thm]{Lemme} \newtheorem{coro}[thm]{Corollaire}
\newtheorem{fait}[thm]{Fait}

\theoremstyle{definition}
\newtheorem{rem}{Remarque} \newtheorem*{Rem}{Remarque}


\begin{document}

\section{Paramétrisation des variétés projectives}

Dans cette section, on considère une varété projective quelconque $Z$ de
dimension $u$, et on souhaite en exhiber une paramétrisation au voisinage d'un
point appartenant à un ouvert dense $U$. Par paramétrisation, on entend un
morphisme $\big(\Aff^1_{\hat{0}}\big)^u \times U \to Z$, où $\Aff^1_{\hat{0}}$
désigne le complété formel de la droite affine le long de l'origine. Un tel
morphisme correspond à un morphisme (continu) d'anneaux $A(Z) \to A(U)[[t_1,
\ldots, t_u]]$, où $A(\truc)$ désigne l'anneau des coordonnées. On va ci-desous
s'attacher à contrôler (degré, hauteur, dénominateur) l'image des formes
homogènes par ce morphisme en fonction des paramètres diophantiens (degré,
hauteur) de $Z$.

On utilise assez largement les notations de Rémond sans toujours les rappeler.
On utilise par contre la norme $L_1$ aux places archimédiennes pour les
polynômes. On note par ailleurs $\dv = 0$ si $v$ est finie et $1$ sinon.

\begin{lem}
  Soit $X = (X_1, \dots, X_u)$ une famille d'indéterminées, $L / \cdn(X)$ une
  extension finie. On choisit $y$ un élément de $L$, on note $\chi : \cdn[X, Y]
  \to L$ le $\cdn(X)$-morphisme donné par $Y \mapsto y$. On fixe $P$ de degré
  $D$ dans son noyau, on pose $R = \frac{\partial P}{\partial Y}$. On considère
  de plus les $\cdn$-dérivations de $\cdn(X)$ caractérisées par $\diff_i(X_j)$
  pour $1 \le i, j \le u$, et on notera de même leurs extensions à $L$. Pour
  chaque $\kappa \in \N^u$, on note $\diff^\kappa = \diff_1^{\kappa_1}\cdots
  \diff_u^{\kappa_u}$ et $\partial^\kappa = \frac1{\kappa!}\diff^\kappa$. Pour
  chaque $\kappa \neq 0$, il existe des polynômes $P_a^\kappa$, $a \in \{0,
  1\}$, tels que :
  \begin{enumthm}
    \item $\partial^\kappa y =
      \frac{\chi(P_a^\kappa)}{\chi(R)^{2\lgr\kappa-1}}$, \label{i-repres}
      pour tous $a$ et $\kappa$.
    \item $\deg P_a^\kappa \le (D -1)(2 \lgr\kappa -1)$,
      pour tous $a$ et $\kappa$.
    \item $\nv{P_\dv^{\kappa}} \le \nv P ^{2\lgr\kappa - 1}
  	  \cdot \left((8u)^{\lgr\kappa -1} D^{3\lgr\kappa -2} \right)^\dv$
      pour tout $\kappa$ et toute place $v$ de $\cdn$.
  \end{enumthm}
  En particulier, l'image de $\cdn[X, y]$ par les différents $\partial^\kappa$
  est contenue dans $\cdn[X, y, \chi(R)^{-1}]$.
\end{lem}

\begin{proof}
  Il s'agit en fait de compléter la preuve du lemme~6.1 de \bsc{Rémond}, en
  utilisant aux places archimédiennes une généralisation de la relation~(2.3.1)
  de \bsc{Fahri}. On reprend toutes les notations de \bsc{Rémond}. On va
  contruire $P_0^\kappa$ et $P_1^\kappa$ par récurrence sur la longueur de
  $\kappa$, en partant à chaque fois de $P_a^\kappa = - d_{k_0}P$ quand
  $\kappa_{k_0} = 1$ et $\kappa_k = 0$ sinon (cas $\lgr\kappa = 1$), car ce
  choix convient. Pour la suite, on fixe un $a$, un $\kappa$ de longueur au
  moins $2$, et on suppose qu'on a choisi un $P_a^{\kappa'}$ convenable pour
  chaque $\kappa'$ de longueur strictement inférieure à celle de $\kappa$.

  On commence par le cas ultramétrique et on note donc provisoirement $P^\kappa
  = P_0^\kappa$ pour alléger. Si de tels polynômes existent, ils doivent
  satisfaire à la relation
  \[
    P\left( X_1+t_1, \ldots, X_n+t_n, y + \sum_{\kappa \in \N^u\setminus\{0\}}
    \frac{\chi(P^\kappa)}{\chi(R)^{2\lgr\kappa -1}}\right) = 0 \pmm.
  \]
  On écrit alors le développement de \bsc{Taylor} de $P$, il vient :
  \begin{align*}
    0
    & = \sum_{\substack{\lambda\in\N^u \\ \mu \in \N}} \frac1{\lambda!\mu!}\,
      \diff\pexp{\lambda, \mu}P(X_1,\ldots, X_n,y) \left(\sum_{\kappa \in
      \N^u\setminus\{0\}} \frac{\chi(P^\kappa)}{\chi(R)^{2\lgr\kappa -1}}
      \right)^u t^\lambda \\
    & = \sum_{\substack{(\lambda,\mu)\in\N^{u+1}\setminus\{(0,0)\}\\
      l\pexp{i} \in \N^u\setminus\{0\}}} \chi\left( \frac1{\lambda!\mu!}\,
      \diff\pexp{\lambda,\mu}P \prod_{1\le i \le \mu}
      \frac{P^{l\pexp i}}{R^{2\lgr{l\pexp i} -1}} \right)
      t^{\sum l\pexp i + \lambda}
  \end{align*}
  où l'on a noté $(\lambda, \mu) = (\lambda_1, \ldots, \lambda_n, \mu)$. Chaque
  terme de cette série est donc nul, c'est-à-dire que pour tout $\kappa \neq 0$
  on a
  \[
    \chi \Biggl( \frac{P^\kappa}{R^{2\lgr\kappa - 2}} +
    \sum_{\substack{(\lambda,\mu)\in\N^{u+1}\setminus\{(0,0), (0,1)\}\\
    l\pexp{i} \in \N^u\setminus\{0\} \\ \sum l\pexp i + \lambda = \kappa}}
    \frac1{\lambda!\mu!}\,\diff\pexp{\lambda,\mu}P \prod_{1\le i \le \mu}
    \frac{P^{l\pexp i}}{R^{2\lgr{l\pexp i} -1}} \Biggr) = 0
  \]
  Réciproquement, il suffit donc de définir $P\kappa$ par la relation de
  récurrence
  \[ P^\kappa =
    \sum_{\substack{(\lambda,\mu)\in\N^{u+1}\setminus\{(0,0), (0,1)\}\\
    l\pexp{i} \in \N^u\setminus\{0\} \\ \sum l\pexp i + \lambda = \kappa}}
    \frac1{\lambda!\mu!}\,\diff\pexp{\lambda,\mu}P
    R^{2\lgr\kappa - 2}
    \prod_{1\le i \le \mu} \frac{P^{l\pexp i}}{R^{2\lgr{l\pexp i} -1}}
  \]
  pour que \ref{i-repres} soit satisfait pour $a=0$

  On majore alors le degré de $P^\kappa$ par récurrence :
  \begin{align*}
    \deg P^\kappa
    & \le D - \lgr\lambda - \mu + (D-1) ( 2\lgr\kappa - 2) \\
    & \le 1 - \lgr\lambda - \mu +(D-1)(2\lgr\kappa - 1) \\
    & \le (D - 1)(2\lgr\kappa - 1) \pmm,
  \end{align*}
  car $\lambda$ et $\mu$ ne sont pas simultanément nuls.

  La majoration de norme locale est par ailleurs immédiate vu les propriétés de
  la norme aux places ultramétriques.

  Traitons maintenant le cas archimédien (désormais $P^\kappa = P^\kappa_1$
  pour alléger). On part de la relation de récurrence suivante (Rémond, avec
  $Q_\kappa = P^\kappa/\kappa!$) où l'on rappelle que $\kappa'$ est tel que
  $\kappa_{k_0} = \kappa'_{k_0} + 1$ est $\kappa_k = \kappa'_k$ sinon.
  \[
    Q_\kappa = R^2 d_{k_0}Q_{\kappa'} - R d_{k_0}P d_Y Q_{\kappa'} + (2
    \lgr{\kappa'} -1)Q_{\kappa'}(d_{k_0} P d_Y R - R d_{k_0}R) \pmm.
  \]
  On en déduit immédiatement l'estimation de degré suivante :
  \[
    \deg P^\kappa \le 2(D -1) + \deg P^{\kappa'} \le (D
    -1)(2\lgr\kappa -1) \pmm.
  \]
  Pour la norme, on prouve que $\nv{Q_\kappa} \le \nv P ^{2 \lgr\kappa - 1}
  8^{\lgr\kappa -1} D^{3\lgr\kappa -2} (\lgr\kappa -1)!$ (ce qui implique le
  résultat annoncé vu que $\binom{\lgr\kappa}{\kappa} \le u^{\lgr\kappa}$) en
  exploitant l'estimation obtenue sur le degré sous la forme $D^\kappa \le
  2\lgr\kappa D$.
  \begin{align*}
    \nv{Q_\kappa}
    & \le 2 D^2 (\deg Q_{\kappa'})\nv P ^2 \nv{\smash{Q_{\kappa'}}} +
      2(2\lgr{\kappa'} -1) D^3 \nv {P}^2 \\
    & \le \nv P ^2 \nv{ \smash{Q_{\kappa'}}} 8D^3 \lgr{\kappa'} \\
    & \le \nv P ^{2 \lgr\kappa - 1} 8^{\lgr\kappa -1} D^{3\lgr\kappa -2}
      (\lgr\kappa -1)! \qedhere
  \end{align*}
\end{proof}

\begin{Rem}
  On peut obtenir des estimations similaires en fonction d'un majorant $D'$
  quelconque de $\deg_X P$ et $\deg_Y P$. Il vient par exemple $\deg_X
  P_a^\kappa \le D'\lgr\kappa$ et de même pour le degré en $Y$. On peut
  également remplacer $D$ par $D'$ et $8$ par $4$ dans la majoration de
  $\nv{P_1^\kappa}$ ci-dessus.
\end{Rem}

\begin{fait}
  Si le plongement $Z \hookrightarrow \Proj^n$ est \og adapté \fg{}, alors
  dans $K(Z)$ les variables $w_1, \ldots, w_u$ sont algébriquement
  indépendantes, et chaque $w_j$ est lié aux $w_1, \ldots, w_u$ par un
  polynôme $\widetilde P _j$ (obtenu en déshomogénéisant par $W_0 = 1$ le
  $P_j$ du lemme 4.2) de degré $D := \deg Z$, unitaire en la dernière
  variable, et tel qu'en toute place, $\nv{\smash{\widetilde P}_j} \le
  \nv{f_Z}$.
\end{fait}

\begin{proof}
  C'est la partie~4.1 de \bsc{Rémond} (p.\,114). Seule l'assertion sur la norme
  n'est pas énoncée sous cette forme par \bsc{Rémond}, mais elle vient de même
  que la proposition~4.2 en remarquant que $P_j$ est une spécialisation de $f_Z$
  qui annulle certaines variables et remplace les autres par des monômes
  unitaires.
\end{proof}

On note désormais $f(\kappa) = \max(0, 2\lgr\kappa - 1)$ et $g(\kappa) =
2(D-1)f(\kappa)$ si $\kappa \neq 0$, $g(\kappa) = 1$ sinon. On remarque de
suite que pour tous $l\pexp i$ dont la somme est $\kappa$ on a $\sum f(l\pexp i)
\le f(\kappa)$.

\begin{lem}
  Soit $Z \hookrightarrow \Proj^n$ une variété de degré $D$, plongée de façon
  adaptée, et $\widetilde Z$ sa partie affine $W_0 \neq 0$. On note $\chi \colon
  \cdn[X_1, \ldots X_n] \to K(Z)$ envoyant $X_i$ sur $w_i$.  Soit $\tilde\tau :
  A(\widetilde Z) \to A(U)[[t_1, \ldots, t_n]]$ le morphisme qui envoie $f$ sur
  $\sum\partial^\kappa f$, où $U$ est le complémentaire dans $\widetilde Z$ du
  diviseur découpé par les $\widetilde R_j = \frac{\partial \widetilde P _j}
  {\partial X_j}$. Il existe des polynômes $\widetilde P_{a, j}^\kappa \in
  \cdn[X_1, \ldots, X_u, X_j]$ pour $u < j \le n$, $a \in \{0, 1\}$ et $\kappa
  \in \N^u$, tels que :
  \begin{enumthm}
    \item $\tilde\tau(w_j) = \sum\limits_{\kappa\in\N^u}
      \frac{\chi(\widetilde P_{a, j}^\kappa)}{\chi(\widetilde
      R_j)^{f(\kappa)}} t^\kappa$ pour tous $a$ et $j$ ;
    \item $\deg \widetilde P_{a, j}^\kappa \le g(\kappa)$ pour tous $a$, $j$ et
      $\kappa$ ;
    \item $\nv{\widetilde P_{\dv, j}^\kappa} \le \bigl( \nv{f_Z} (\sqrt{8u}
      D^{3/2})^\dv \bigr)^{f(\kappa)}$ pour tous $j$ et $\kappa$.
  \end{enumthm}
\end{lem}

\begin{proof}
  Vu la définition de $\tilde\tau$, il suffit avec d'appliquer le lemme
  précédent, avec $L = K(Z)$, indépendament à chaque $w_j$ en tenant compte de
  l'information connue sur le degré et la norme des $P_j$.
\end{proof}

\begin{lem}
  Dans les hypothèses et notations précédentes, il existe des polynômes
  $\widetilde R$ et $\widetilde Q_{a, \gamma}^\kappa$ tels que, notant
  $w^\gamma$ l'image d'un monôme quelconque de $\cdn[X_1, \ldots, X_n]$ dans
  $K(Z)$, on ait
  \begin{enumthm}
    \item $\tilde\tau(w^\gamma) = \sum\limits_{\kappa\in\N^u}
      \frac
        {\chi(\widetilde Q_{a, \gamma}^\kappa)}
        {\chi(\widetilde R)^{f(\kappa)}}
      t^\kappa$ pour tous $a$ et $\gamma$ ;
    \item $\deg \widetilde Q_{a, \gamma}^\kappa \le (D-1)(n-u)f(\kappa) +
      \lgr\gamma$ pour tous $a$, $\gamma$ et $\kappa$ ;
    \item $\nv{\widetilde Q_{\dv, \gamma}^\kappa} \le 2^{\dv(\lgr\kappa +
        n(\lgr\gamma -1))} \big( \nv{f_Z} \cdot (\sqrt{8u} D^{3/2})^\dv
      \big)^{(n-u)f(\kappa)}$ pour tous $v$, $\gamma$ et $\kappa$ ;
    \item $\deg{\widetilde R} \le (D-1)(n-u)$ ;
    \item $\nv{\widetilde R} \le (D^\dv \nv{f_Z})^{n-u}$.
  \end{enumthm}
\end{lem}

\begin{proof}
  On applique le lemme précédent, en remarquant de plus que pour $1\le j \le u$
  on peut choisir $\widetilde Q_{a, j}^\kappa$ et $\widetilde R_j$ satisfaisant
  à sa condition \ref{i-repres} avec $\deg{\widetilde Q}_{a, j}^\kappa \le 0$,
  $\nv{\widetilde Q_{a, j}^\kappa} \le 1$ et $\widetilde R_j = 1$. Il vient
  alors :
  \begin{align*}
    \tilde\tau(w^\gamma) &= \prod_{j=1}^n \tilde\tau(w_j)^\gamma_j \\
    & = \prod_{j=1}^n \prod_{k_j=1}^{\gamma_j} \Bigg(
      \sum_{l\pexp{j, k_j}} \frac
        {\chi( \widetilde Q_{a, j}^{l^{(j, k_j)}})}
        {\chi(\widetilde R_j)^{f(l\pexp{j, k_j})}} t^{l\pexp{j, k_j}}
      \Bigg) \\
    & = \sum_\kappa \Bigg( \sum_{\sum l\pexp{j, k_j} = \kappa} \prod_{j=1}^n
      \prod_{k_j=1}^{\gamma_j} \frac
        {\chi(\widetilde Q_{a, j}^{l\pexp{j, k_j}})}
        {\chi(\widetilde R_j)^{f(l\pexp{j, k_j})}}
      \Bigg) t^\kappa \\
    & = \sum_\kappa \frac1{\chi(\widetilde R)^{f(\kappa)}}v
      \cdot\chi\Bigg(\underbrace{ \sum_{\sum l\pexp{j, k_j} = \kappa}
        \prod_{j=1}^n  \widetilde R_j^{f(\kappa) -
        \smash{\sum\limits_{k_j=1}^{\gamma_j}}f(l\pexp{j, k_j})}
        \prod_{k_j=1}^{\gamma_j} \widetilde Q_{a, j}^{l\pexp{j, k_j}} }
      _{=: \widetilde Q_{a, \gamma}^\kappa} \Bigg) t^\kappa \pmm,
  \end{align*}
  en posant $\widetilde R = \prod \widetilde R_j$. Les estimations de degré et
  de norme de $\widetilde R$ sont immédiates.  On remarque également que
  $f(\kappa) - \sum_{k_j=1}^{\gamma_j}f(l\pexp{j, k_j})$ est positif, de sorte
  que $\widetilde Q_{a, \gamma}^\kappa$ est en effet un polynôme. On évalue
  alors son degré grâce au résultat précédent :
  \begin{align*}
    \deg \widetilde Q_{a, \gamma}^\kappa
    & \le \sum_{j=u}^n \Big(\sum_{k_j = 1}^{\gamma_j} g(l\pexp{j, k_j}) +
      2(D-1)(f(\kappa) - \smash{\sum\limits_{k_j=1}^{\gamma_j}}f(l^{(j,
      k_j)})\Big) \\
    & \le 2(D-1)(n-u)f(\kappa) + \sum_{l\pexp{j, k_j} =0} \!\!1 \\
    & \le 2(D-1)(n-u)f(\kappa) + \lgr\gamma \pmm,
  \end{align*}
  où l'on a utilisé le fait que $g(l) - 2(D-1)f(l)$ vaut $1$ si $l$ est nul et
  $0$ sinon.

  On procède de même pour la norme. Aux places archimédiennes, on majore le
  nombre de termes en remarquant que l'ensemble de sommation est le produit des
  sous-ensembles de $\N^{\lgr\gamma}$ définis par $\sum_{j, k_j} l_i^ {(j, k_j)}
  = \kappa_i$ pour $i \in \{1,\ldots,n\}$, qui sont chacun de cardinal
  $\binom{\kappa_i + \lgr\gamma - 1}{\lgr\gamma - 1} \le 2^{\kappa_i +
    \lgr\gamma - 1}$. On majore ensuite la norme de chaque terme (pour $a =
  \dv$) par
  \begin{align*}
    & \prod_{j=u+1}^n \left( \nv{\smash{\widetilde R_j}}^{f(\kappa) - \sum
        f(l\pexp{j, k_j}}) \smash{\prod_{\substack{j, k_j \\ j > u}}}
      \nv{P_{\dv, j}}^{l\pexp{j, k_j}} \right) \\
    & \qquad\le \prod_{j=u+1}^n \big( (D^\dv \nv{f_Z})^{f(\kappa)} \big)
      \prod_{\substack{j, k_j \\ j > u}}  \big( (\sqrt{8u} D^{1/2})^\dv
      \big)^{f(l\pexp{j, k_j})} \\
    & \qquad\le \big( \nv{f_Z} \cdot (\sqrt{8u}D^{3/2})^\dv
      \big)^{(n-u)f(\kappa)} \pmm,
  \end{align*}
  d'où l'estimation annoncée et le lemme.
\end{proof}

\begin{lem} \label{l-param}
  Soit $Z$ une variété projective de degré $D$ et de dimension $u$, définie
  sur un corps de nombres $\cdn$, plongée de façon adaptée dans $\Proj^n$, dont
  on note $f_Z$ une forme de \bsc{Chow}. On considère le monomorphisme de
  paramétrisation $\tau \colon A(\widetilde Z) \to A(U)[[t]]$ (où $\widetilde Z
  = Z \subset (X_0)$ et $U$ est un certain ouvert de $Z$) qui applique chaque
  fonction $f$ sur son développement en série $\sum \partial^\kappa \tilde F \,
  t^\kappa$.  On note enfin $\chi \colon \cdn[X_0, \ldots, X_n] \to K(Z)$ le
  morphisme appliquant $X_0$ sur $1$ et $X_i$ sur $w_i$.

  Il existe une forme $R$ homogène de degré $(D-1)(n-u)$ et, pour tout entier
  $d$, des applications linéaires \[ \nu_a^\kappa \colon \cdn[X_0, \ldots,
  X_n]_d \to \cdn[X_0, \ldots, X_n]_{d + (D-1)(n-u)f(\kappa)} \] telles que,
  pour toute forme homogène $F$ et toute place $v$ de $\cdn$ :
  \begin{enumthm}
    \item $\tau(\chi(F)) = \chi\left( \frac1{X_0^d} \sum\limits_{\kappa \in
        \N^u} \frac {\nu_a^\kappa(F)} {R^{f(\kappa)}} \, t^\kappa \right)$  ;
    \item $\nv{\nu_\dv^\kappa(F)} \le \nv F \cdot \big( \nv{f_Z}
      (\sqrt{8u}D^{3/2})^\dv \big)^{(n-u)f(\kappa)} \cdot 2^{\dv (\lgr\kappa +
        u(d-1))}$ ; \label{i-norme}
    \item $\nv R \le (D^\dv \nv{F_Z})^{n-u}$.
  \end{enumthm}
\end{lem}

\begin{proof}
  On choisit pour $R$ l'homogénéisé de degré $(D-1)(n-u)$ de $\widetilde R$, et
  on définit $\nu_a^\kappa$ sur les monômes en imposant que l'image de $X^\beta$
  soit l'homogénéisé de degré adéquat de $Q_{a, \beta}^\kappa$. On applique
  alors le lemme précédent, en utilisant pour \ref{i-norme} la sous-linéarité de
  la norme.
\end{proof}


\section{Image par le plongement éclatant}
% ----------------------------------------

On considère $m \ge 2$ un entier fixé (à choisir ultérieurement), et $a = (a_1,
\ldots, a_m)$ un $m$-uplet d'entiers non nuls. À un tel $a$ on associe un
morphisme, dit plongement éclatant,
\begin{align*}
  \phi_a \colon\quad \A^m &\longrightarrow \A^m\times \A^{m-1} \\
  (x_1, \ldots, x_m)&\longmapsto (x_1, \ldots, x_m, a_1 x_1 - x_m, \ldots,
  a_{m-1} x_{m-1} - x_m) \pmm.
\end{align*}

On étudie dans cette partie l'image par un tel plongement éclatant d'une
sous-variété produit $Z$ de $\A$, qu'on supposera contenir l'origine $\OA$. Plus
particulièrement, on souhaite en contrôler une paramétrisation sur un ouvert
contenant l'origine, et calculer ses multidegrés quand on regarde l'image
plongée dans $(\Proj^n)^{2m-1}$ de la façon induite par le plongement $\A
\hookrightarrow \Proj^n$ initialement fixé. Si $Z = \prod_{l=1}^m Z_l$, on note
$u_l = \dim Z_l$ et $u=\sum_l u_l = \dim Z$.

\medskip
Commençons par la paramétrisation. On déduit tout d'abord du lemme~\ref{l-param}
une paramétrisation de $Z$. Pour cela, introduisons quelques notations : $X$
désigne désormais plusieurs groupes de variables $X\pexp1, \ldots, X\pexp m$
avec $X\pexp l = (X\pexp l[0], \ldots, X\pexp m[n])$. On utilise les notations
compactes naturelles pour les puissances. Par ailleurs, $u_\truc$ désigne le
vecteur $(u_1, \ldots, u_l$ et, quand on écrit un produit de vecteurs, il s'agit
du produit composante par composante. Enfin, on note $\vb_l$ le vecteur dont
toutes les coordonnées sont nulles, sauf la $l$-ième égale à $1$, et $\vb$ celui
dont toutes les coordonnées valent $1$.

\begin{lem} \label{l-par-prod}
  Sous les hypothèses du lemme~\ref{l-param}, il existe une famille de polynômes
  $U = (U_1, \ldots, U_m)$ homogènes de degrés respectifs $\deg U_l =
  (D-1)(n-u_l)\vb_l$, et, pour tout $d \in N^m$, deux applications linéraires
  \[
    \nu_{m, \dv}^\lambda \colon \cdn[X\pexp 0, \ldots, X\pexp m]_d \to
    \cdn[X\pexp 0, \ldots, X\pexp m]_{d + (D-1)(n-u_\truc)f(\lambda\pexp\truc)}
  \]
  telles que, pour tout forme $F\in \cdn[X]_d$ et toute place $v$ de $\cdn$ :
  \begin{enumthm}
    \item \label{i-repr-prod} $\tau(\chi(F)) =
      \chi\left( \frac1{X_0^d} \sum_{\lambda \in \cramped{\N^u}}
      \frac {\nu_{m, \dv}^\lambda(F)} {U^{f(\lambda)}} s^\lambda \right)$ ;
    \item \label{i-norm-prod} $\nv{\smash{\nu_{m, \dv}^\lambda(F)}} \le \nv F
      \nv{f_Z}^{2n\lgt\lambda} \bigl(
      (16uD^3)^{n\lgt{\lambda}} \cdot 2^{u(\lgr d -1)} \bigr)^\dv$ ;
      % \big(\nv{f_Z}    (\sqrt{8u}D^{3/2})^\dv \big)^{(n-u)f(\lambda)}
      % 2^{\dv(\lgt\lambda + u(\lgr d-1)}$ ; 
    \item $\nv{U_l} \le (D^\dv \nv{F_Z})^{n-u_l}$.
  \end{enumthm}
\end{lem}

\begin{proof}
  On pose $U_l = R(X\pexp l)$, où $R$ est le polynôme du lemme~\ref{l-param}.
  On définit ensuite $\nu_{m, \dv}^\lambda$ sur les monômes par $\nu_{m,
  \dv}^\lambda(\mathfrak m) = \prod_l \nu_\dv^{\lambda\pexp l}(\mathfrak m\pexp
  l)$, où $\mathfrak m$ est le produit des monômes $\mathfrak m\pexp l$ en les
  variables $X\pexp l$. On prolonge ensuite par linéarité, on applique alors le
  lemme~\ref{l-param} qui donne \ref{i-repr-prod} immédiatement. On établit
  d'abord~\ref{i-norm-prod} pour les monômes (en majorant au passage
  $f(\lambda\pexp{l}$ par $\lgr{\lambda\pexp l}$), et on conclut par
  sous-linéarité de la norme.  
\end{proof}

Pour continuer, on a besoin de représenter par des polynômes les morphismes de
multiplication par un entier et de différence sur $\A$. On utilise pour cela une
version de la proposition~5.2 de \bsc{Rémond}, adaptée pour tenir compte des
différences de normes utilisées, et du fait que sa constante $h_1$ est explicite
depuis les travaux de \bsc{David} et \bsc{Philippon}.

\begin{fait}[\bsc{David} et \bsc{Philippon}, prop.~3.7]
  Il existe une famille $S_{l, l'}$ de formes bihomogènes de degré $(2, 2)$
  représentant globalement le morphisme d'addition-soustraction dans $\A^2$ dans
  le plongement de \bsc{Mumford} modifié suivi d'un plongement de \bsc{Segre},
  et de la forme
  \[
    S_{l, l'}(X, Y) = \big(\coa_{k(l)}\coa_{k(l')}\big)^{-1}
    \sum_{(i, i'\!,j, j') \in E_x}
    \zeta_{i, i'\!, j, j'} X_i X_{i'} Y_j Y_{j'} \pmm,
  \]
  où $E_x$ est un sous-ensemble à $4^g$ éléments de $\{0, \ldots, n \}$, les
  $\zeta_{i, i'\!, j, j'}$ sont des racines quatrièmes de l'unité, et $k$ est
  une certaine fonction $\{0, \ldots, n \} \to \{0, \ldots, n \}$ telle que pour
  tout $i$, $\coa_{k(i)}^{-1}$ existe et est une coordonnée de $\BA$.
\end{fait}

En particulier, on a immédiatement les estimations de norme locale
$\nv{\smash{S_{l, l'}}} \le 4^{g\dv} \nv{\BA}^2$ et $\nnv{\smash{S_{l, l'}}} \le
\big(4^g (n+1)^2\big)^\dv \nv{\BA}^2$.

Par ailleurs, par projection sur un facteur, on en déduit l'existence de
familles de formes bihomogènes $S^+$ et $S^-$ de forme similaire (mais on peut
simplifier par un facteur $\coa_{k(l)}$ ou $\coa_{k(l')}$), représentant
respectivement l'addition et la soustraction au voisinage de n'importe quel
point fixé, et telles que $\nv{S^\bullet} \le 4^{g\dv} \nv\BA$.

\begin{fait}[\bsc{Rémond}, démonstration de la prop.~5.2]
  Pour tout entier $b$ non nul, il existe une famille de formes homogènes
  $Q^b_i$ représentant globalement la multiplication par $b$ sur $\A$ dans le
  plongement de \bsc{Mumford} modifié, de degré au plus $\frac98 m^2$, telles
  qu'en toute place $v$, $\nnv{Q^b} \le (2^g\sqrt{n+1})^{\dv(m^2-1)}$.
\end{fait}

On considère alors le morphisme $\tau \colon A(\widetilde Z) \to
A(O)[[t\pexp1, \ldots, t\pexp{m}]]$ de paramétrisation de $O$, qui est \og
représenté \fg{} au sens du lemme~\ref{l-par-prod} par des formes linéaires
$\nu_{m, \dv}^\kappa = \prod_{i=1}^m \nu_\dv^\kappa$, pour $O$ un ouvert
convenable. On souhaite contrôler de même le morphisme composé
\[
  \Omega_a \colon A(\phi_a(Z)) \to A(Z) \to A(O)[[t\pexp1,
  \ldots, t\pexp m]] \pmm,
\]
où le premier est le morphisme d'anneaux naturellement associé à $\phi_a$.

Pour cela, on commence par remarquer que celui-ci est localement
représenté par le morphisme
 \begin{align*}
   \Psi_a \colon \cdn[X\pexp1, \ldots, X\pexp m, Y\pexp1, \ldots, Y\pexp{m-1}]
   & \longrightarrow \cdn[X\pexp1, \ldots, X\pexp m] \\
   X\pexp i & \longmapsto X\pexp i \\
   Y\pexp j & \longmapsto S^-(Q_{a_j}(X\pexp j), X\pexp m) \\
 \end{align*}
Soit $F$ une forme multihomogène de degré $(\alpha_1, \ldots, \alpha_m,
\beta_1, \ldots, \beta_{m-1})$. Son image par $\Psi_a$ est
\[
  F(X\pexp1, \ldots, X\pexp m, S^-(Q^{a_1}(X\pexp 1), X\pexp m), \ldots,
  S^-(Q^{a_{m-1}}(X\pexp{m-1}), X\pexp m) \pmm,
\]
de degré $(\alpha_1 + 9/4 \beta_1 a_1^2, \ldots, \alpha_{m-1} + 9/4
\beta_{m-1} a_{m-1}^2, \alpha_m +2\lgr\beta)$, et dont on majore les normes
locales comme suit :
\begin{align*}
  \nv{\Psi_a(F)}
  & \le \nv F \prod_{j=1}^{m-1} \nnv{S^-(Q^{a_j}, \truc)}^{\beta_j} \\
  & \le \nv F \prod_{j=1}^{m-1} \left( \nnv{S^-} \nnv{Q^{a_j}}^2
    \right)^{\beta_j} \\
  & \le ((4^g(n+1))^\dv \nv\BA)^{\lgr\beta} \cdot (4^g(n+1))^{\dv\sum(a_j^2
      -1)\beta_j} \nv\BA^{2\sum(a_j^2-1)\beta_j} \\
  & \le \nv F \nv\BA^{\lgr\beta + 2\sum(a_j^2-1)\beta_j} \cdot
    (4^g(n+1))^{\dv(\lgr\beta + \sum(a_j^2-1)\beta_j)} \pmm.
\end{align*}

On a alors immédiatemment le résultat suivant.

\begin{lem}
  Dans les hypothèses et notations du lemme~\ref{l-par-prod}, considérons une
  forme $G \in \cdn[X, Y]$ multihomogène de degré $(\alpha, \beta)$. On pose 
  \[ d = (\alpha_1 + 9/4 \beta_1 a_1^2, \ldots, 
  \alpha_{m-1} + 9/4 \beta_{m-1} a_{m-1}^2, \alpha_m +2\lgr\beta) \pmm. \]
  Il existe alors des polynômes homogènes $V^\lambda_\dv(G)$ tels que :
  \begin{enumthm}
    \item $\deg_i(V_\dv^\lambda(G)) = d_i + (D-1)(n-u_i)f(\lambda\pexp i)$ ;
    \item $\nv{V_\dv^\lambda(G)} \le \nv G \nv\BA ^{\lgr\beta + 2\sum
      (a_j^2-1)\beta_j} \nv{f_Z}^{2n\lgt\lambda} \\ \null\qquad \cdot
      \Big( \big(4^g 2^{9u/4}(n+1)\big)^{\lgr\beta + \sum(a_j^2-1)\beta_j}
      \cdot (16uD^3)^{n\lgt\lambda} \cdot 2^{u(\lgr d -1)}
      \Big)^\dv$ ;
    \item $\Omega_a(\chi(G)) =\chi \left( \frac1{X_0^d} \sum_{\lambda \in \N^u}
      \frac {V_\dv^\lambda(G)} {U^{f(\lambda)}} \right)$,
  \end{enumthm}
  où $U$ est la famille de polynômes du lemme~\ref{l-par-prod}.
\end{lem}

\begin{proof}
  Il suffit d'appliquer le lemme~\ref{l-par-prod} à $\Psi_a(G)$.
\end{proof}

\end{document}
