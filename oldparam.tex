
\begin{proof}
  \dots

  On écrit alors le développement de \bsc{Taylor} de $P$, il vient :
  \begin{align*}
    0
    & = \sum_{\substack{\lambda\in\N^u \\ \mu \in \N}} \frac1{\lambda!\mu!}\,
      \diff\pexp{\lambda, \mu}P(X_1,\ldots, X_n,y) \left(\sum_{\kappa \in
      \N^u\setminus\{0\}} \frac{\chi(P^\kappa)}{\chi(R)^{2\lgr\kappa -1}}
      \right)^u t^\lambda \\
    & = \sum_{\substack{(\lambda,\mu)\in\N^{u+1}\setminus\{(0,0)\}\\
      l\pexp{i} \in \N^u\setminus\{0\}}} \chi\left( \frac1{\lambda!\mu!}\,
      \diff\pexp{\lambda,\mu}P \prod_{1\le i \le \mu}
      \frac{P^{l\pexp i}}{R^{2\lgr{l\pexp i} -1}} \right)
      t^{\sum l\pexp i + \lambda}
  \end{align*}
  où l'on a noté $(\lambda, \mu) = (\lambda_1, \ldots, \lambda_n, \mu)$. Chaque
  terme de cette série est donc nul, c'est-à-dire que pour tout $\kappa \neq 0$
  on a
  \begin{equation}
    \chi \Biggl( \frac{P^\kappa}{R^{2\lgr\kappa - 2}} +
    \sum_{\substack{(\lambda,\mu)\in\N^{u+1}\setminus\{(0,0), (0,1)\}\\
    l\pexp{i} \in \N^u\setminus\{0\} \\ \sum l\pexp i + \lambda = \kappa}}
    \frac1{\lambda!\mu!}\,\diff\pexp{\lambda,\mu}P \prod_{1\le i \le \mu}
    \frac{P^{l\pexp i}}{R^{2\lgr{l\pexp i} -1}} \Biggr) = 0
  \end{equation}
  Réciproquement, il suffit donc de définir $P\kappa$ par la relation de
  récurrence
  \begin{equation} - P^\kappa =
    \sum_{\substack{(\lambda,\mu)\in\N^{u+1}\setminus\{(0,0), (0,1)\}\\
    l\pexp{i} \in \N^u\setminus\{0\} \\ \sum l\pexp i + \lambda = \kappa}}
    \frac1{\lambda!\mu!}\,\diff\pexp{\lambda,\mu}P
    \cdot R^{2\lgr\kappa - 2}
    \cdot \prod_{1\le i \le \mu} \frac{P^{l\pexp i}}{R^{2\lgr{l\pexp i} -1}}
  \end{equation}
  pour que \ref{i:repres} soit satisfait pour $a=0$

  On majore alors le degré de $P^\kappa$ par récurrence :
  \begin{align*}
    \deg P^\kappa
    & \le D - \lgr\lambda - \mu + (D-1) ( 2\lgr\kappa - 2) \\
    & \le 1 - \lgr\lambda - \mu +(D-1)(2\lgr\kappa - 1) \\
    & \le (D - 1)(2\lgr\kappa - 1) \pmm,
  \end{align*}
  car $\lambda$ et $\mu$ ne sont pas simultanément nuls.

  La majoration de norme locale est par ailleurs immédiate vu les propriétés de
  la norme aux places ultramétriques.

  Traitons maintenant le cas archimédien (désormais $P^\kappa = P^\kappa_1$
  pour alléger). On part de la relation de récurrence suivante (\bsc{Rémond},
  avec $Q_\kappa = P^\kappa \cdot \kappa!$) où l'on rappelle que $\kappa'$ est
  tel que $\kappa_{k_0} = \kappa'_{k_0} + 1$ est $\kappa_k = \kappa'_k$ sinon.
  \begin{equation}
    Q_\kappa = R^2 d_{k_0}Q_{\kappa'} - R d_{k_0}P d_Y Q_{\kappa'} + (2
    \lgr{\kappa'} -1)Q_{\kappa'}(d_{k_0} P d_Y R - R d_{k_0}R) \pmm.
  \end{equation}
  On en déduit immédiatement l'estimation de degré suivante :
  \begin{equation}
    \deg P^\kappa \le 2(D -1) + \deg P^{\kappa'} \le (D
    -1)(2\lgr\kappa -1) \pmm.
  \end{equation}
  Pour la norme, on prouve que $\Onv{Q_\kappa} \le \Onv P ^{2 \lgr\kappa - 1}
  4^{\lgr\kappa -1} D^{3\lgr\kappa -2} (\lgr\kappa -1)!$ (ce qui implique le
  résultat annoncé vu que $\binom{\lgr\kappa}{\kappa} \le u^{\lgr\kappa - 1}$)
  en exploitant l'estimation obtenue sur le degré sous la forme $D^\kappa \le
  2\lgr\kappa D$.
  \begin{align*}
    \Onv{Q_\kappa}
    & \le 2 D^2 (\deg Q_{\kappa'})\Onv P ^2 \Onv{\smash{Q_{\kappa'}}} +
      2(2\lgr{\kappa'} -1) D^3 \Onv {P}^2 \\
    & \le \Onv P ^2 \Onv{ \smash{Q_{\kappa'}}} 4D^3 \lgr{\kappa'} \\
    & \le \Onv P ^{2 \lgr\kappa - 1} 4^{\lgr\kappa -1} D^{3\lgr\kappa -2}
      (\lgr\kappa -1)! \qedhere
  \end{align*}
\end{proof}

\begin{rem}
  On peut obtenir des estimations similaires en fonction d'un majorant $D'$
  quelconque de $\deg_X P$ et $\deg_Y P$. Il vient par exemple $\deg_X
  P_a^\kappa \le D'\lgr\kappa$ et de même pour le degré en $Y$. On peut
  également remplacer $D$ par $D'$ et $4$ par $2$ dans la majoration de
  $\nv1{P_1^\kappa}$ ci-dessus.
\end{rem}

\begin{fact} \label{f:dep-int}
  \worknote{À déplacer avec les autres trucs sur le plongement adapté ?}
  Si le plongement $Z \hookrightarrow \Proj^n$ est \og adapté \fg{}, alors
  dans $K(Z)$ les variables $w_1, \ldots, w_u$ sont algébriquement
  indépendantes, et chaque $w_j$ est lié aux $w_1, \ldots, w_u$ par un
  polynôme $\widetilde P _j$ (obtenu en déshomogénéisant par $W_0 = 1$ le
  $P_j$ du lemme 4.2 de \cite{remivds}) de degré $D := \deg Z$, unitaire en la
  dernière variable, et tel qu'en toute place, $\nv1{\smash{\widetilde P}_j}
  \le \nv1{f_Z}$.
\end{fact}

\begin{proof}
  C'est \cite[partie~4.1, p.~114]{remivds}. Seule l'assertion sur la
  norme n'est pas énoncée sous cette forme par \bsc{Rémond}, mais elle vient
  de même que la proposition~4.2 en remarquant que $P_j$ est une
  spécialisation de $f_Z$ qui annule certaines variables et remplace les
  autres par des monômes unitaires.
\end{proof}

On note désormais $f(\kappa) = \max(0, 2\lgr\kappa - 1)$ et $g(\kappa) =
2(D-1)f(\kappa)$ si $\kappa \neq 0$, $g(\kappa) = 1$ sinon. On remarque de
suite que pour tous $l\pexp i$ dont la somme est $\kappa$ on a $\sum f(l\pexp i)
\le f(\kappa)$.

\begin{lem}
  Soit $Z \hookrightarrow \Proj^n$ une variété de degré $D$, plongée de façon
  adaptée, et $\widetilde Z$ sa partie affine $W_0 \neq 0$. On note $\chi \colon
  \cdn[X_1, \ldots X_n] \to K(Z)$ envoyant $X_i$ sur $w_i$.  Soit $\tilde\tau :
  A(\widetilde Z) \to A(U)[[t_1, \ldots, t_n]]$ le morphisme qui envoie $f$ sur
  $\sum\partial^\kappa f$, où $U$ est le complémentaire dans $\widetilde Z$ du
  diviseur découpé par les $\widetilde R_j = \frac{\partial \widetilde P _j}
  {\partial X_j}$. Il existe des polynômes $\widetilde P_{a, j}^\kappa \in
  \cdn[X_1, \ldots, X_u, X_j]$ pour $u < j \le n$, $a \in \{0, 1\}$ et $\kappa
  \in \N^u$, tels que :
  \begin{enumthm}
    \item $\tilde\tau(w_j) = \sum\limits_{\kappa\in\N^u}
      \frac{\chi(\widetilde P_{a, j}^\kappa)}{\chi(\widetilde
      R_j)^{f(\kappa)}} t^\kappa$ pour tous $a$ et $j$ ;
    \item $\deg \widetilde P_{a, j}^\kappa \le g(\kappa)$ pour tous $a$, $j$ et
      $\kappa$ ;
    \item $\nv1{\widetilde P_{\dv, j}^\kappa} \le \bigl( \nv1{f_Z} (\sqrt{4u}
      D^{3/2})^\dv \bigr)^{f(\kappa)}$ pour tous $j$ et $\kappa$.
  \end{enumthm}
\end{lem}

\begin{proof}
  Vu la définition de $\tilde\tau$, il suffit avec d'appliquer le lemme
  précédent, avec $L = K(Z)$, indépendamment à chaque $w_j$ en tenant compte de
  l'information connue sur le degré et la norme des $P_j$.
\end{proof}

\begin{lem}
  Dans les hypothèses et notations précédentes, il existe des polynômes
  $\widetilde R$ et $\widetilde Q_{a, \gamma}^\kappa$ tels que, notant
  $w^\gamma$ l'image d'un monôme quelconque de $\cdn[X_1, \ldots, X_n]$ dans
  $K(Z)$, on ait
  \begin{enumthm}
    \item $\tilde\tau(w^\gamma) = \sum\limits_{\kappa\in\N^u}
      \frac
        {\chi(\widetilde Q_{a, \gamma}^\kappa)}
        {\chi(\widetilde R)^{f(\kappa)}}
      t^\kappa$ pour tous $a$ et $\gamma$ ;
    \item $\deg \widetilde Q_{a, \gamma}^\kappa \le (D-1)(n-u)f(\kappa) +
      \lgr\gamma$ pour tous $a$, $\gamma$ et $\kappa$ ;
    \item $\nv1{\widetilde Q_{\dv, \gamma}^\kappa} \le 2^{\dv(\lgr\kappa +
        n(\lgr\gamma -1))} \big( \nv1{f_Z} \cdot (\sqrt{4u} D^{3/2})^\dv
      \big)^{(n-u)f(\kappa)}$ pour tous $v$, $\gamma$ et $\kappa$ ;
    \item $\deg{\widetilde R} \le (D-1)(n-u)$ ;
    \item $\nv1{\widetilde R} \le (D^\dv \nv1{f_Z})^{n-u}$.
  \end{enumthm}
\end{lem}

\begin{proof}
  On applique le lemme précédent, en remarquant de plus que pour $1\le j \le u$
  on peut choisir $\widetilde Q_{a, j}^\kappa$ et $\widetilde R_j$ satisfaisant
  à la condition \ref{i:repres} avec $\deg{\widetilde Q}_{a, j}^\kappa \le 0$,
  $\nv1{\widetilde Q_{a, j}^\kappa} \le 1$ et $\widetilde R_j = 1$. Il vient
  alors :
  \begin{align*}
    \tilde\tau(w^\gamma) &= \prod_{j=1}^n \tilde\tau(w_j)^\gamma_j \\
    & = \prod_{j=1}^n \prod_{k_j=1}^{\gamma_j} \Bigg(
      \sum_{l\pexp{j, k_j}} \frac
        {\chi( \widetilde Q_{a, j}^{l^{(j, k_j)}})}
        {\chi(\widetilde R_j)^{f(l\pexp{j, k_j})}} t^{l\pexp{j, k_j}}
      \Bigg) \\
    & = \sum_\kappa \Bigg( \sum_{\sum l\pexp{j, k_j} = \kappa} \prod_{j=1}^n
      \prod_{k_j=1}^{\gamma_j} \frac
        {\chi(\widetilde Q_{a, j}^{l\pexp{j, k_j}})}
        {\chi(\widetilde R_j)^{f(l\pexp{j, k_j})}}
      \Bigg) t^\kappa \\
    & = \sum_\kappa \frac1{\chi(\widetilde R)^{f(\kappa)}}v
      \cdot\chi\Biggl(\underbrace{
        \sum_{\sum l\pexp{j, k_j} = \kappa} \prod_{j=1}^n
        \Bigl(
        \widetilde R_j^{f(\kappa) -
        \smash{\sum\limits_{k_j=1}^{\gamma_j}}f(l\pexp{j, k_j})}
        \prod_{k_j=1}^{\gamma_j} \widetilde Q_{a, j}^{l\pexp{j, k_j}} 
        \Bigr)
        }_{=: \widetilde Q_{a, \gamma}^\kappa}
      \Biggl) t^\kappa
      \pmm,
  \end{align*}
  en posant $\widetilde R = \prod \widetilde R_j$. Les estimations de degré et
  de norme de $\widetilde R$ sont immédiates.  On remarque également que
  $f(\kappa) - \sum_{k_j=1}^{\gamma_j}f(l\pexp{j, k_j})$ est positif, de sorte
  que $\widetilde Q_{a, \gamma}^\kappa$ est en effet un polynôme. On évalue
  alors son degré grâce au résultat précédent :
  \begin{align*}
    \deg \widetilde Q_{a, \gamma}^\kappa
    & \le \sum_{j=u}^n \Big(\sum_{k_j = 1}^{\gamma_j} g(l\pexp{j, k_j}) +
      2(D-1)(f(\kappa) - \smash{\sum\limits_{k_j=1}^{\gamma_j}}f(l^{(j,
      k_j)})\Big) \\
    & \le 2(D-1)(n-u)f(\kappa) + \sum_{l\pexp{j, k_j} =0} \!\!1 \\
    & \le 2(D-1)(n-u)f(\kappa) + \lgr\gamma \pmm,
  \end{align*}
  où l'on a utilisé le fait que $g(l) - 2(D-1)f(l)$ vaut $1$ si $l$ est nul et
  $0$ sinon.

  On procède de même pour la norme. Aux places archimédiennes, on majore le
  nombre de termes en remarquant que l'ensemble de sommation est le produit des
  sous-ensembles de $\N^{\lgr\gamma}$ définis par $\sum_{j, k_j} l_i^ {(j, k_j)}
  = \kappa_i$ pour $i \in \{1,\ldots,n\}$, qui sont chacun de cardinal
  $\binom{\kappa_i + \lgr\gamma - 1}{\lgr\gamma - 1} \le 2^{\kappa_i +
    \lgr\gamma - 1}$. On majore ensuite la norme de chaque terme (pour $a =
  \dv$) par
  \begin{align*}
    & \prod_{j=u+1}^n \left( \nv1{\smash{\widetilde R_j}}^{f(\kappa) - \sum
        f(l\pexp{j, k_j})} \smash{\prod_{\substack{j, k_j \\ j > u}}}
      \nv1{P_{\dv, j}}^{l\pexp{j, k_j}} \right) \\
    & \qquad\le \prod_{j=u+1}^n \big( (D^\dv \nv1{f_Z})^{f(\kappa)} \big)
      \prod_{\substack{j, k_j \\ j > u}}  \big( (\sqrt{4u} D^{1/2})^\dv
      \big)^{f(l\pexp{j, k_j})} \\
    & \qquad\le \big( \nv1{f_Z} \cdot (\sqrt{4u}D^{3/2})^\dv
      \big)^{(n-u)f(\kappa)} \pmm,
  \end{align*}
  d'où l'estimation annoncée et le lemme.
\end{proof}

\begin{lem} \label{l:param}
  Soit $Z$ une variété projective de degré $D$ et de dimension $u$, définie
  sur un corps de nombres $\cdn$, plongée de façon adaptée dans $\Proj^n$, dont
  on note $f_Z$ une forme de \bsc{Chow}. On considère le monomorphisme de
  paramétrisation $\tau \colon A(\widetilde Z) \to A(U)[[t]]$ (où $\widetilde Z
  = Z \setminus (X_0)$ et $U$ est un certain ouvert de $Z$) qui applique chaque
  fonction $f$ sur son développement en série $\sum \partial^\kappa \tilde F \,
  t^\kappa$.  On note enfin $\chi \colon \cdn[X_0, \ldots, X_n] \to K(Z)$ le
  morphisme appliquant $X_0$ sur $1$ et $X_i$ sur $w_i$.

  Il existe une forme $R$ homogène de degré $(D-1)(n-u)$ et, pour tout entier
  $d$, des applications linéaires \begin{equation} \nu_a^\kappa \colon \cdn[X_0, \ldots,
  X_n]_d \to \cdn[X_0, \ldots, X_n]_{d + (D-1)(n-u)f(\kappa)} \end{equation} telles que,
  pour toute forme homogène $F$ et toute place $v$ de $\cdn$ :
  \begin{enumthm}
    \item $\tau(\chi(F)) = \chi\left( \frac1{X_0^d} \sum\limits_{\kappa \in
        \N^u} \frac {\nu_a^\kappa(F)} {R^{f(\kappa)}} \, t^\kappa \right)$  ;
    \item $\nv1{\nu_\dv^\kappa(F)} \le \nv1 F \cdot \big( \nv1{f_Z}
      (\sqrt{4u}D^{3/2})^\dv \big)^{(n-u)f(\kappa)} \cdot 2^{\dv (\lgr\kappa +
        u(d-1))}$ ; \label{i:norme}
    \item $\nv1 R \le (D^\dv \nv1{F_Z})^{n-u}$.
  \end{enumthm}
\end{lem}

\begin{proof}
  On choisit pour $R$ l'homogénéisé de degré $(D-1)(n-u)$ de $\widetilde R$, et
  on définit $\nu_a^\kappa$ sur les monômes en imposant que l'image de $X^\beta$
  soit l'homogénéisé de degré adéquat de $Q_{a, \beta}^\kappa$. On applique
  alors le lemme précédent, en utilisant pour \ref{i:norme} la sous-linéarité de
  la norme.
\end{proof}

\subsection{Image par le plongement pondéré} \label{sub:param-wemb}

On souhaite contrôler (degré, hauteur, dénominateurs) une paramétrisation de
\( \wemb(\var) \) sur un ouvert contenant \( \wemb(\excep) \). On déduit tout
d'abord du lemme~\ref{l:param} une paramétrisation de $\var$. On reprend les
notations habituelles (section~\ref{sec:vojta-intro}) concernant \( \var \) et
les coordonnées multiprojectives.

De plus, on note $\vb_l$ le vecteur dont toutes les coordonnées sont nulles,
sauf la $l$-ième égale à $1$, et $\vb$ celui dont toutes les coordonnées
valent $1$. Enfin, quand on écrit un produit de vecteurs, il s'agit du produit
composante par composante.

\begin{lem} \label{l:par-prod}
  Sous les hypothèses du lemme~\ref{l:param}, il existe une famille de polynômes
  $U = (U_1, \ldots, U_m)$ homogènes de degrés respectifs $\deg U_l =
  (D-1)(n-u_l)\vb_l$, et, pour tout $d \in N^m$, deux applications linéaires
  \begin{equation}
    \nu_{m, \dv}^\lambda \colon \cdn[X\pexp 0, \ldots, X\pexp m]_d \to
    \cdn[X\pexp 0, \ldots, X\pexp m]_{d + (D-1)(n-u_\truc)f(\lambda\pexp\truc)}
  \end{equation}
  telles que, pour tout forme $F\in \cdn[X]_d$ et toute place $v$ de $\cdn$ :
  \begin{enumthm}
    \item \label{i:repr-prod} $\tau(\chi(F)) =
      \chi\left( \frac1{X_0^d} \sum_{\lambda \in \cramped{\N^u}}
      \frac {\nu_{m, \dv}^\lambda(F)} {U^{f(\lambda)}} s^\lambda \right)$ ;
    \item \label{i:norm-prod} $\nv1{\smash{\nu_{m, \dv}^\lambda(F)}} \le \nv1 F
      \nv1{f_Z}^{2n\lgt\lambda} \bigl(
      (16uD^3)^{n\lgt{\lambda}} \cdot 2^{u(\lgr d -1)} \bigr)^\dv$ ;
      % \big(\nv1{f_Z}    (\sqrt{4u}D^{3/2})^\dv \big)^{(n-u)f(\lambda)}
      % 2^{\dv(\lgt\lambda + u(\lgr d-1)}$ ;
    \item $\nv1{U_l} \le (D^\dv \nv1{F_Z})^{n-u_l}$.
  \end{enumthm}
\end{lem}

\begin{proof}
  On pose $U_l = R(X\pexp l)$, où $R$ est le polynôme du lemme~\ref{l:param}.
  On définit ensuite $\nu_{m, \dv}^\lambda$ sur les monômes par $\nu_{m,
  \dv}^\lambda(\mathfrak m) = \prod_l \nu_\dv^{\lambda\pexp l}(\mathfrak m\pexp
  l)$, où $\mathfrak m$ est le produit des monômes $\mathfrak m\pexp l$ en les
  variables $X\pexp l$. On prolonge ensuite par linéarité, on applique alors le
  lemme~\ref{l:param} qui donne \ref{i:repr-prod} immédiatement. On établit
  d'abord~\ref{i:norm-prod} pour les monômes (en majorant au passage
  $f(\lambda\pexp{l}$ par $\lgr{\lambda\pexp l}$), et on conclut par
  sous-linéarité de la norme.
\end{proof}

On considère alors le morphisme $\tau \colon A(\widetilde Z) \to
A(O)[[t\pexp1, \ldots, t\pexp{m}]]$ de paramétrisation de $O$, qui est \og
représenté \fg{} au sens du lemme~\ref{l:par-prod} par des formes linéaires
$\nu_{m, \dv}^\kappa = \prod_{i=1}^m \nu_\dv^\kappa$, pour $O$ un ouvert
convenable. On souhaite contrôler de même le morphisme composé
\begin{equation}
  \Omega_a \colon A(\phi_a(Z)) \to A(Z) \to A(O)[[t\pexp1,
  \ldots, t\pexp m]] \pmm,
\end{equation}
où le premier est le morphisme d'anneaux naturellement associé à $\phi_a$.

Soit $F$ une forme multihomogène de degré $(\alpha_1, \ldots, \alpha_m,
\beta_1, \ldots, \beta_{m-1})$. Le degré de son image par \( \wemba \) est
donné par le lemme~\ref{l:deg-wemba}. De la même façon, on majore ses normes
locales par :
\begin{align*}
  \nv1{\Psi_a(F)}
  & \le \nv1 F \prod_{j=1}^{m-1} \nnv1{S^-(Q^{a_j}, \truc)}^{\beta_j} \\
  & \le \nv1 F \prod_{j=1}^{m-1} \left( \nnv1{S^-} \nnv1{Q^{a_j}}^2
    \right)^{\beta_j} \\
  & \le ((4^g(n+1))^\dv \nv1\vai)^{\lgr\beta} \cdot (4^g(n+1))^{\dv\sum(a_j^2
      -1)\beta_j} \nv1\vai^{2\sum(a_j^2-1)\beta_j} \\
  & \le \nv1 F \nv1\vai^{\lgr\beta + 2\sum(a_j^2-1)\beta_j} \cdot
    (4^g(n+1))^{\dv(\lgr\beta + \sum(a_j^2-1)\beta_j)} \pmm.
\end{align*}

On a alors immédiatement le résultat suivant.

\begin{lem} \label{l:par-img}
  Dans les hypothèses et notations du lemme~\ref{l:par-prod}, considérons une
  forme $G \in \cdn[X, Y]$ multihomogène de degré $(\alpha, \beta)$. On pose
  \begin{equation} d = (\alpha_1 + 9/4 \beta_1 a_1^2, \ldots,
  \alpha_{m-1} + 9/4 \beta_{m-1} a_{m-1}^2, \alpha_m +2\lgr\beta) \pmm. \end{equation}
  Il existe alors des polynômes homogènes $V^\lambda_\dv(G)$ tels que :
  \begin{enumthm}
    \item $\deg_i(V_\dv^\lambda(G)) = d_i + (D-1)(n-u_i)f(\lambda\pexp i)$ ;
    \item $\nv1{V_\dv^\lambda(G)} \le \nv1 G \nv1\vai ^{\lgr\beta + 2\sum
      (a_j^2-1)\beta_j} \nv1{f_Z}^{2n\lgt\lambda} \\ \null\qquad \cdot
      \Big( \big(4^g 2^{9u/4}(n+1)\big)^{\lgr\beta + \sum(a_j^2-1)\beta_j}
      \cdot (16uD^3)^{n\lgt\lambda} \cdot 2^{u(\lgr d -1)}
      \Big)^\dv$ ;
    \item $\Omega_a(\chi(G)) =\chi \left( \frac1{X_0^d} \sum_{\lambda \in \N^u}
      \frac {V_\dv^\lambda(G)} {U^{f(\lambda)}} \right)$,
  \end{enumthm}
  où $U$ est la famille de polynômes du lemme~\ref{l:par-prod}.
\end{lem}

\begin{proof}
  Il suffit d'appliquer le lemme~\ref{l:par-prod} à $\Psi_a(G)$.
\end{proof}

\endinput

% vim: spell spelllang=fr
