\chapter{Déduction des résultats principaux}
\label{chap:union}

Nous déduisons ici des deux chapitres précédents des décomptes
d'approximations exceptionnelles. Plus précisément, on donne un décompte des
points de hauteur assez grande (au sens de~\eqref{e:Vbig} et~\eqref{e:Mbig}),
que nous appellerons grands points, en combinant les inégalités de \bsc{Vojta}
et de \bsc{Mumford} obtenues précédemment.

On en déduit ensuite un décompte des grands points avec une hypothèse
d'approximation sous la forme d'un produit sur les places, plus naturelle que
l'hypothèse technique avec un système d'inégalités. Cette forme constitue le
résultat principal de la thèse.

Enfin, pour les petits points, après avoir rappelé une méthode de décompte
grossière ne tenant pas compte de l'hypothèse d'approximation, on présente une
méthode pour les éliminer en renforçant ladite hypothèse, puis on discute de
choix possibles pour la valeur d'un paramètre laissé libre jusque là, afin
d'optimiser l'hypothèse et/ou le décompte obtenu.

\section{Obstruction au décompte absolu}
\label{sec:obstruction}



\section{Décomptes des grands points}
\label{sec:big-points}

Avant toute chose, remarquons qu'il n'y a pas de grandes approximations
exceptionnelles si \( \eps \) est trop grand, en vertu de l'inégalité de
\bsc{Liouville}.

\begin{sco} \label{s:nobig-liouville}
  Soit \( V \) une sous-variété de \( \projd \), de dimension \( u \) et de
  degré \( d \). Il n'existe aucun point \( x \in \projd(\Qbar) \)
  tel que
  \begin{align}
    \label{e:nobig-ha}
    0 < \prod\placerange \distv x V ^\degv
    & <
    \hautm[2] x ^{-\expapx}
    \\ \label{e:nobig-big}
    \hautm[2] x
    & >
    (n+1)^{3/2}
    (3d)^{d (u+1)}
    \, \hautm[1]{ \chow V }
  \end{align}
  pour \( \eps \ge d + 1 \).
\end{sco}

\begin{proof}
  Un tel point contredirait directement l'inégalité de \bsc{Liouville}
  (proposition~\vref{p:liouville}).
\end{proof}

On commence par majorer le nombre de points dans chaque cône tronqué.

\begin{lem} \label{l:big-by-cone}
  Soient \( V \) un translaté par un point algébrique d'une sous-variété
  abélienne \( \vai \) de \( \va \) de degré \( d \) et de dimension \( u \).
  On fixe un réel \( 0 < \expapx \), un entier \( \puiss \ge \genre + 1 \) et
  on considère une famille de points \( \ex[1], \dots, \ex[p] \in \va(\Qbar)
  \) deux à deux distincts modulo \( \vai \) et satisfaisant simultanément aux
  conditions suivantes :
  \begin{align}
    \label{e:big-ha}
    0 < \distv{\ex*} V
    & <
    \bigl( d \sbin{d+\dimp}{\dimp}^3 \bigr)^{-\dv/2}
    \hautm[2]{\ex*}^{-\wtapx \expapx}
    \quad \forall \place \in \placesapx
    \\ \label{e:big-big}
    \hautn{\ex[1]}
    & > \cst{ht-v-gen1} \Lambda_4^{(2\puiss\genre)^{\puiss\genre}}
    \\ \label{e:big-cos}
    \cos(\ex*, \ex[\fcti])
    & > 1 - \frac1{ \puiss \, \cst{v-gen} }
  \end{align}
  avec
  \begin{align}
    \cst{v-gen}
    & =
    \bigl(
        86 \nclmaps \cdot 5^\genre d \sbin{\dimp + d}{d}
        \, \expapx^{-1}
    \bigr)^{ \frac{\puiss}{\puiss-\genre} }
    \\
    \Lambda_4
    & =
    \cst{v-gen}
    \bigl( (\sqrt2 \puiss \genre d)^\genre \deg \va \bigr)^\puiss
    \\
    \cst{ht-v-gen1}
    & =
    d \max \bigl(
      d^\genre \hautl[1]{\va}, \hlclab, \htcmp
    \bigr)
    + (\genre + 1) \deg \va
    \Bigl(
      d^\dimp \ln(d) \dimp
    \\ & \qquad
      + d^\genre \bigl (
        \hautl[1] V
        + (u + 2) \ln (d + 1) ( d + \dimp + 1 )
        + \ln(\puiss/2)
      \bigr)
    \Bigr)
    \pmm.
  \end{align}
  On a alors nécessairement
  \begin{equation}
    p
    \le
    \sqrt{\frac d \expapx}
    (4\puiss\genre)^{\puiss\genre}
    \ln \Lambda_4
    \pmm.
  \end{equation}
\end{lem}

\begin{proof}
  Pour commencer, remarquons qu'on peut supposer \( \eps < d + 1 \). En effet,
  sinon le scolie~\vref{s:nobig-liouville} s'applique, car~\eqref{e:big-ha}
  implique immédiatement~\eqref{e:nobig-ha} et~\eqref{e:big-big} implique
  assez largement~\eqref{e:nobig-big} vu la définition de \( \cst{ht-v-gen1}
  \).

  Soit \( x_1, \dots, x_p \) une famille comme dans l'énoncé, qu'on suppose de
  plus ordonnée par hauteur normalisée croissante. On pose
  \( \phi =( \puiss \, \cst{v-gen})^{-1} \) et
  \( \rho = \sqrt{ \expapx / d } \).  On constate facilement que
  \( \phi \le \expapx / (2580 d) \) de sorte que ce couple \( (\rho, \phi) \)
  satisfait à~\eqref{e:rho-phi}. Par ailleurs, il est assez facile de voir que
  la condition~\eqref{e:big-big} est plus forte que~\eqref{e:Mbig} : en effet
  \( \cst{v-gen} \ge 2/\expapx \) et la parenthèse dans le membre de droite
  de~\eqref{e:Mbig} est largement majorée par \( 4 \cst{ht-v-gen1} \) en
  utilisant entre autres~\eqref{e:comp-h-hn-var} et le fait que \( \hautn \vai
    = 0 \) (prop.~9 de \cite{phiha1}).

  On peut donc appliquer le théorème~\vref{t:mumford} et conclure que
  \( \hautn{ x_{i+1} } > (1 + \sqrt{\expapx / d}) \hautn{ x_i } \) pour tout
  \( i \le p-1 \). On pose alors
  \begin{equation}
    \eta
    =
    \left\lceil
      (2\puiss\genre)^{\puiss\genre}
      \frac{ \ln \Lambda_4 }{ \ln (1 + \sqrt{\expapx / d}) }
    \right\rceil
  \end{equation}
  de sorte que, pour tout \( i \le p - \eta \), on a
  \begin{equation}
    \hautn{ x_{i+\eta} }
    >
    (1 + \sqrt{\expapx / d})^\eta
    \hautn{ x_i }
    \ge
    \Lambda_4^{(2\puiss\genre)^{\puiss\genre}}
    \hautn{ x_i }
    \pmm.
  \end{equation}
  Supposons alors que \( p \ge (\puiss - 1) \eta + 1 \) contrairement à la
  conclusion du lemme : on pourrait alors poser \( \ex* = x_{1 + (\fct-1)
      \eta} \) pour \( i \in {1, \puiss} \) et la famille \( (\ex*)_\fct \)
  contredirait alors le corollaire~\vref{c:vojta-gen2}, ce qui est absurde.

  Ainsi, on a
  \begin{equation}
    p
    \le
    (\puiss - 1)
    \left(
      (2\puiss\genre)^{\puiss\genre}
      \frac{ \ln \Lambda_4 }{ \ln (1 + \sqrt{\expapx / d}) }
      + 1
    \right)
    \le
    \puiss
    (2\puiss\genre)^{\puiss\genre}
    \frac{ \ln \Lambda_4 }{ \ln (1 + \sqrt{\expapx / d}) }
    \pmm.
  \end{equation}
  On remarque alors que la fonction \( x \mapsto \ln(1+x)/x \) est
  décroissante pour \( x > 0 \) pour minorer le dénominateur par \( \ln(2)
    \sqrt{\expapx / d} \) compte tenu de l'hypothèse sur \( \expapx \) ; on
  obtient alors, en remarquant de plus que \( \ln 2 \ge 1/2 \) :
  \begin{equation}
    p
    \le
    \puiss (2\puiss\genre)^{\puiss\genre}
    \cdot 2 \sqrt{\frac d \expapx}
    \cdot \ln \Lambda_4
  \end{equation}
  qui donne aisément le résultat annoncé.
\end{proof}


  % \begin{align}
  %   \ln \Lambda_4
  %   & =
  %   \frac\puiss{\puiss-\genre}
  %   \left(
  %     \ln 86
  %     + \ln \nclmaps
  %     + \genre \ln 5
  %     + \ln d
  %     + {\textstyle \ln \binom{n+d}d}
  %     - \ln \expapx
  %   \right)
  %   \\ & \qquad
  %   + \puiss \bigl(
  %     \deg\va + \genre (\ln2/2 + \ln(\puiss\genre) + \ln d)
  %   \bigr)
  %   \\ & \le
  %   2 \bigl(
  %     9/2
  %     + \ln \nclmaps
  %     + \genre \ln 5
  %     + (n+1) \ln(d+1)
  %     - \ln \expapx
  %   \bigr)
  %   \\ & \qquad
  %   + \puiss \deg\va
  %   + \puiss\genre (\ln(\puiss\genre) + \ln2/2)
  %   + \puiss\genre \ln d
  %   \\ & \le
  %   - \ln \expapx
  %   + 9
  %   + 2 \ln \nclmaps
  %   + \puiss \deg\va
  %   + \ln(d+1) (n+1 + \puiss\genre)
  %   + \puiss\genre (\ln(\puiss\genre) + 2)
  % \end{align}

On utilise ensuite le fait suivant.

\begin{fact}
  Soient \( r \) un entier et \( \gamma > 0 \) un réel. On peut recouvrir \(
    \R^r \) par \( \floor{(1 + \sqrt{8/\gamma})^r} \) ensembles dans chacun
  desquels deux points quelconques satisfont \( \cos(x, y) \ge 1 - \gamma \).
\end{fact}

\begin{proof}
  C'est le corollaire~6.1, p.~542 de~\cite{remdcl}.
\end{proof}

\begin{prop} \label{p:big-gen}
  Soient \( V \) un translaté par un point algébrique d'une sous-variété
  abélienne \( \vai \) de \( \va \) de degré \( d \) et de dimension \( u \).
  On considère de plus un sous-groupe \( \Gamma \subset \va(\Qbar) \) de rang
  fini \( r \).
  Pour tout réel \( \expapx > 0 \) et tout entier \( \puiss \ge \genre + 1 \),
  il existe au plus
  \begin{equation}
    \sqrt{\frac d \expapx}
    (4\puiss\genre)^{\puiss\genre}
    \ln \Lambda_4
    \left(
      3 \sqrt\puiss
      \bigl(
        86 \nclmaps \cdot 5^\genre d \sbin{\dimp + d}{d}
        \, \expapx^{-1}
        \bigr)^{ \frac{\puiss}{2(\puiss-\genre)} }
    \right)^r
  \end{equation}
  points de \( \Gamma \) distincts modulo \( \vai \) et satisfaisant
  aux deux conditions suivantes :
  \begin{align}
    0 < \distv{\ex*} V
    & <
    \bigl( d \sbin{d+\dimp}{\dimp}^3 \bigr)^{-\dv/2}
    \hautm[2]{\ex*}^{-\wtapx \expapx}
    \quad \forall \place \in \placesapx
    \\
    \hautn{\ex[1]}
    & > \cst{ht-v-gen1} \Lambda_4^{(2\puiss\genre)^{\puiss\genre}}
  \end{align}
  avec \( \Lambda_4 \) et \( \cst{ht-v-gen1} \) comme au
  lemme~\vref{l:big-by-cone}.
\end{prop}

\begin{proof}
  En utilisant le fait précédent, on recouvre l'espace euclidien \( \Gamma
    \otimes_\Z \R \) muni du produit scalaire donné par la hauteur de \NT par
  des ensembles dans lesquels deux points quelconques
  satisfont~\eqref{e:big-cos}. Le nombre d'ensembles nécessaires est au plus
  \begin{equation}
    \left(
      1 + \sqrt{8 \puiss}
      \bigl(
        86 \nclmaps \cdot 5^\genre d \sbin{\dimp + d}{d}
        \, \expapx^{-1}
        \bigr)^{ \frac{\puiss}{2(\puiss-\genre)} }
    \right)^r
    \le
    \left(
      3 \sqrt\puiss
      \bigl(
        86 \nclmaps \cdot 5^\genre d \sbin{\dimp + d}{d}
        \, \expapx^{-1}
        \bigr)^{ \frac{\puiss}{2(\puiss-\genre)} }
    \right)^r
  \end{equation}
  en remarquant que \( 1 + \sqrt{ 8 \cdot 86 \cdot 3 \cdot 5 } \le 3 \sqrt{ 86
      \cdot 3 \cdot 5 } \).

  Il ne reste plus qu'à appliquer lemme~\vref{l:big-by-cone} à chacun de ces
  ensembles pour conclure.
\end{proof}



\section{Hypothèse d'approximation produit}
\label{sec:ha-prod}

Comme annoncé dans l'introduction (page~\pageref{e:ha-prod}), on déduit ici de
l'énoncé précédent une version où l'hypothèse d'approximation fait apparaître
un produit sur l'ensemble des places plutôt qu'une inégalité par place.

Pour cela, on utilise le lemme suivant, qui n'est rien d'autre qu'une version
plus abstraite du lemme~2.12.1, p. 120 de~\cite{farhith}. Par ailleurs, par
rapport à cette référence, on a fixé \( \eps'(T) = \eps/2 \), d'où \( A(T) =
  \card T \), car c'est la valeur qui a le plus de sens pour les applications,
afin de simplifier légèrement l'énoncé et sa preuve. L'énoncé ci-dessous est
volontairement aussi abstrait que possible afin de mettre en lumière le
caractère purement combinatoire de la preuve et d'être utilisable dans le plus
grand nombre de situations ; immédiatement après l'énoncé on détaille la
correspondance avec la situation étudiée ici afin d'en éclaircir le sens.

\begin{lem}
  Soit \( A \) un ensemble et \( H \) une fonction de \( A \) dans \( [1,
    +\infty \mathclose[ \). Soit par ailleurs \( M \) un autre ensemble et,
  pour chaque élément \( v \) de \( M \) :
  \begin{itemize}
    \item une fonction \( d_v \colon A \to [0, 1] \) ;
    \item deux réels \( 0 < \degv < 1 \) et \( c_v > 0 \).
  \end{itemize}
  Pour tout réel \( \eps > 0 \) et tout partie finie \( S \) de \( M \), on
  définit
  \[
    E(\eps, S)
    =
    \set*{
      x \in A
      \text{ tel que }
      \prod_{v \in S} d_v(x)^\degv
      \le
      \bigl( \prod_{v \in S} c_v^\degv \bigr)
      H(x)^{-\eps}
    }
    \pmm.
  \]
  De plus, pour toute famille de réels \( (\lambda_v)_{v\in S} \), on définit
  \[
    E'(\eps, S, \lambda)
    =
    \set*{
      x \in A
      \text{ tel que }
      d_v(x)
      \le
      c_v H(x)^{-\lambda_v \eps}
      \quad \forall v \in S
    }
    \pmm.
  \]
  On note enfin \( \parts M \) l'ensemble des parties finies de \( M \) et on
  suppose qu'il existe une fonction \( g \colon \R_{>0} \times \parts M \to \N
  \) telle que pour tout \( \eps > 0 \), tout \( S \in \parts M \) non vide et
  toute famille \( (\lambda_v)_{v\in S} \) de réels positifs telle que \(
    \sum_{v\in S} \lambda_v \degv = 1 \), on ait
  \[
    \card E'(\eps, S, \lambda)
    \le
    g(\eps, S)
    \pmm.
  \]
  Alors, pour tout \( \eps > 0 \) et tout \( S \in \parts M \) on a
  \[
    \card E(\eps, S)
    \le
    \sum_{T \in \parts S \minusset\emptyset}
    \binom{2\card T - 1}{\card T - 1}
    g(\eps/2, T)
    \pmm.
  \]
\end{lem}

\begin{rem} \label{r:app-prod}
  Dans l'application prévue, les paramètres seront choisis de la façon
  suivante.  L'ensemble \( A \) sera l'intersection d'un système complet de
  représentants modulo \( \vai \) dans un sous-groupe de rang fini (voire de
  type fini) de \( \va(\Qbar) \) avec \( \va(\Qbar) \setminus V(\Qbar) \) et
  \( H \) la hauteur induite sur cet ensemble par le choix d'un plongement
  projectif de \( \va \). L'ensemble \( M \) sera l'ensemble des places d'un
  corps de nombres \( \cdn \) suffisamment gros et en chaque place
  \( \degv = [\cdn_v : \Q_v] / [\cdn : \Q] \) est le degré local divisé par le
  degré global. En revanche, le choix des constantes \( c_v \) dépendra de
  l'application envisagée. On reconnaît la condition sur la famille
  \( \lambda \), et la fonction \( g \) est alors donnée par l'un des énoncés
  de la section précédente.
\end{rem}

\begin{proof}
  On suppose \( S \) et \( \eps \) fixés. On remarque que pour toute famille
  \( \lambda \) telle que que \( \sum_{v \in S} \lambda_v \degv = 1 \) on a \(
    E'(\eps, S, \lambda) \subset E(\eps, S) \) ; à l'inverse on aimerait bien
  recouvrir \( E(\eps, S) \) par de tels ensembles, en nombre fini. Dans ce
  but, pour tout \( x \in A \) (tel que \( H(x) > 1 \) pour simplifier pour
  l'instant), on introduit une famille de réels \( \lambda(x) \) telle que
  \begin{equation} \label{e:x2wt}
    d_v(x)
    =
    c_v H(x)^{-\lambda_v(x) \eps}
    \quad \forall v \in S
  \end{equation}
  de sorte que \( x \in E'(\eps, S, \lambda(x)) \). De plus, si \( x \in E(\eps,
    S) \) on a \( \sum_{v \in S} \lambda_v(x) \degv \ge 1 \). Ainsi, on a
  recouvert \( E(\eps, S) \) par des ensembles de la forme \( E'(\eps, S,
    \lambda(x)) \) mais ceux-ci ne sont \lat{a priori} pas en nombre fini ;
  par ailleurs les familles \( \lambda(x) \) obtenues ne sont pas
  nécessairement composées que de nombres positifs. (Le fait que leur somme
  pondérée ne soit pas exactement \( 1 \) n'est en revanche pas gênant, il
  suffirait de choisir une famille \( \lambda'(x) \) plus petite (pour l'ordre
  produit) sommant à \( 1 \), car la fonction \( \lambda \mapsto E'(\eps, S,
    \lambda) \) est décroissante.)

  Pour forcer les \( \lambda_v(x) \) à être positifs\footnote{Ce qui n'est en
    soi pas indispensable (on pourrait en fait se dispenser de cette hypothèse
    dans les résultats de décompte précédents en considérant un ensemble de
    places plus petit) mais sera utile pour assurer la finitude et contrôler
    le cardinal de l'ensemble \( C \) ci-dessous.} on introduit la fonction \(
    \pi \colon E(\eps, S) \to \parts S \) définie par
  \begin{equation}
    \pi(x)
    =
    \set{
      v \in S
      \text{ tel que }
      d_v(x) \le c_v
    }
    \pmm.
  \end{equation}
  On remarque que \( \pi \) prend en fait ses valeurs dans \( \parts S
    \minusset\emptyset \).  Par ailleurs, pour toute partie non vide \( T \)
  de \( S \), on introduit l'ensemble
  \begin{equation}
    C
    =
    \set{
      (\lambda_v)_{v \in T}
      \text{ tel que }
      \lambda_v \degv \card T \in \N
      \text{ et }
      \sum_{v \in T} \lambda_v \degv = 1
    }
  \end{equation}
  qui est fini et de cardinal \( \binom{2\card T - 1}{\card T - 1} \) car il
  est en bijection avec l'ensemble des familles de \( \card T \) entiers
  naturels dont la somme est \( \card T \). Pour obtenir la conclusion du
  lemme, il suffit donc de montrer que
  \begin{equation}
    \pi^{-1}(T)
    =
    \bigcup_{\lambda \in C} E(\eps/2, T, \lambda)
  \end{equation}
  (où l'on a noté \( \pi^{-1}(T) = \pi^{-1}(\set T) \) pour alléger) car les
  \( \pi^{-1}(T) \) pour \( T \) parcourant l'ensemble des parties non vides
  de \( S \) forment une évidemment partition de \( E(\eps, S) \).

  Fixons donc \( \emptyset \neq T \in \parts S \) et \( x \in \pi^{-1}(T)
  \). Par définition de \( E(\eps, S) \) et de \( \pi \) on a
  \begin{align}
    \bigl( \prod_{v \in S} c_v^\degv \bigr)
    H(x)^{-\eps}
    \ge
    \prod_{v \in S} d_v(x)^\degv
    \ge
    \prod_{v \in T} d_v(x)^\degv
    \prod_{v \in S \setminus T} c_v^\degv
  \end{align}
  d'où \( x \in E(\eps, T) \). Par ailleurs, si \( H(x) = 1 \) il est clair
  que \( x \in E'(\eps, S, \lambda) \) pour n'importe quelle famille \(
    \lambda \in C \), on suppose donc \( H(x) > 1 \) et on associe à \( x \)
  une famille \( \lambda(x) \) comme en~\eqref{e:x2wt}. Cette fois-ci, les \(
    \lambda_v(x) \) sont tous positifs ou nuls. Par ailleurs, on a
  successivement
  \begin{align}
    2 \card T
    \sum_{v \in T} \lambda_v(x) \degv
    & \ge
    2 \card T
    \\
    \sum_{v \in T} \floor{ 2 \lambda_v(x) \degv \card T }
    & \ge
    \card T
  \end{align}
  donc il existe des entiers naturels \( a_v(x) \) inférieurs ou égaux à
  \( 2 \lambda_v(x) \degv \card T \) et dont la somme est exactement
  \( \card T \). On pose alors \( \lambda'(x) = a_v(x) / (\degv \card T) \) de
  sorte que \( \lambda'(x) \in C \) et \( \lambda'_v(x) \le 2 \lambda_v(x) \),
  ce qui implique que \( x \in E'(\eps/2, T, \lambda'(x)) \), achevant ainsi
  la preuve.
\end{proof}

En pratique, la fonction de comptage ne dépend pas de l'ensemble de places
considérées, on utilisera donc le scolie suivant.

\begin{sco}
  Dans la situation du lemme précédent, si l'on suppose de plus que la
  fonction \( g \) ne dépend pas de son deuxième argument, alors on a
  \( \card E(\eps, S) \le 5^{\card S} g(\eps/2) \).
\end{sco}

\begin{proof}
  C'est un calcul élémentaire, fait page~125 de \cite{farhith}.
\end{proof}

\begin{coro}
  Soient \( V \) un translaté par un point algébrique d'une sous-variété
  abélienne \( \vai \) de \( \va \) de degré \( d \) et de dimension \( u \).
  On considère de plus un sous-groupe \( \Gamma \subset \va(\Qbar) \) de rang
  fini \( r \).
  Pour tout réel \( \expapx > 0 \) et tout entier \( \puiss \ge \genre + 1 \),
  il existe au plus
  \begin{equation}
    2 \cdot 5^{\card S} \cdot
    \sqrt{ \frac{d}\expapx }
    (4\puiss\genre)^{\puiss\genre}
    \ln \Lambda_4
    \left(
      3 \sqrt\puiss
      \bigl(
        172 \nclmaps \cdot 5^\genre d \sbin{\dimp + d}{d}
        \, \expapx^{-1}
        \bigr)^{ \frac{\puiss}{2(\puiss-\genre)} }
    \right)^r
  \end{equation}
  points de \( \Gamma \) distincts modulo \( \vai \) et satisfaisant
  aux deux conditions suivantes :
  \begin{align}
    0 < \prod\placerange \distv{\ex*} V ^\degv
    & <
    \bigl( d \sbin{d+\dimp}{\dimp}^3 \bigr)^{-1/2}
    \hautm[2]{\ex*}^{-\expapx}
    \\
    \hautn{\ex[1]}
    & > \cst{ht-v-gen1} \Lambda_4^{(2\puiss\genre)^{\puiss\genre}}
  \end{align}
  avec \( \Lambda_4 \) et \( \cst{ht-v-gen1} \) comme au
  lemme~\vref{l:big-by-cone}.
\end{coro}

\begin{proof}
  On applique le scolie précédent à la proposition~\vref{p:big-gen} comme
  prévu à la remarque~\vref{r:app-prod}. On constate que vu la définition de
  \( \Lambda_4 \), en y remplaçant \( \expapx \) par \( \expapx / 2 \), on
  ajoute \( \ln(2) m / (m-g) \) à son logarithme et que
  \begin{equation}
    \ln \Lambda_4 + \ln(2) m / (m-g)
    \le
    \sqrt 2 \ln \Lambda_4
    \pmm;
  \end{equation}
 les autres changements dans le décompte sont évidents. Pour la condition
 d'approximation, on observe que \( \sum_\placerange \dv\degv \le 1 \), le cas
 le plus défavorable étant celui où \( S \) contient toutes les places
 archimédiennes.
\end{proof}

On peut alors simplifier légèrement la condition d'approximation en absorbant
la constante ; on aboutit au résultat suivant.

\begin{thm} \label{t:big-gen-prod}
  Soient \( V \) un translaté par un point algébrique d'une sous-variété
  abélienne \( \vai \) de \( \va \) de degré \( d \) et de dimension \( u \).
  On considère de plus un sous-groupe \( \Gamma \subset \va(\Qbar) \) de rang
  fini \( r \).
  Pour tout réel \( \expapx > 0 \) et tout entier \( \puiss \ge \genre + 1 \),
  il existe au plus
  \begin{equation}
    2 \cdot 5^{\card S} \cdot
    \sqrt{ \frac{d}\expapx }
    (4\puiss\genre)^{\puiss\genre}
    \ln \Lambda_5
    \left(
      3 \sqrt\puiss
      \bigl(
        174 \nclmaps \cdot 5^\genre d \sbin{\dimp + d}{d}
        \, \expapx^{-1}
        \bigr)^{ \frac{\puiss}{2(\puiss-\genre)} }
    \right)^r
  \end{equation}
  points de \( \Gamma \) distincts modulo \( \vai \) et satisfaisant
  aux deux conditions suivantes :
  \begin{align}
    0 < \prod\placerange \distv{\ex*} V ^\degv
    & <
    \hautm[2]{\ex*}^{-\expapx}
    \\
    \hautn{\ex[1]}
    & > \cst{ht-v-gen1} \Lambda_5^{(2\puiss\genre)^{\puiss\genre}}
  \end{align}
  avec
  \begin{align}
    \Lambda_5
    & =
    \bigl(
        87 \nclmaps \cdot 5^\genre d \sbin{\dimp + d}{d}
        \, \expapx^{-1}
    \bigr)^{ \frac{\puiss}{\puiss-\genre} }
    \bigl( (\sqrt2 \puiss \genre d)^\genre \deg \va \bigr)^\puiss
    \\
    \cst{ht-v-gen1}
    & =
    d \max \bigl(
      d^\genre \hautl[1]{\va}, \hlclab, \htcmp
    \bigr)
    + (\genre + 1) \deg \va
    \Bigl(
      d^\dimp \ln(d) \dimp
    \\ & \qquad
      + d^\genre \bigl (
        \hautl[1] V
        + (u + 2) \ln (d + 1) ( d + \dimp + 1 )
        + \ln(\puiss/2)
      \bigr)
    \Bigr)
    \pmm.
  \end{align}
\end{thm}

\begin{proof}
  On constate pour commencer que
  \begin{equation}
    \frac12 \ln \bigl( d \sbin{d+\dimp}{\dimp}^3 \bigr)
    \le
    \frac32 d \sbin{d+\dimp}{\dimp}^3
    \le
    \frac{ \Lambda_4^{(2mg)^{mg}} }{ 87 \eps }
  \end{equation}
  où la dernière inégalité est très large. On en déduit immédiatement que
  \begin{equation}
    \hautm[2]{ \ex* }^{\eps/87}
    \ge
    \bigl( d \sbin{d+\dimp}{\dimp}^3 \bigr)^{-1/2}
  \end{equation}
  puis que les hypothèses du présent énoncé impliquent celles de l'énoncé
  précédent en remplaçant \( \eps \) par \( 86\eps / 87 \).
\end{proof}

L'énoncé précédent laisse libre le choix du paramètre \( m \) qui peut être
n'importe quel entier supérieur ou égal à \( g + 1 \). Il est clair que la
condition sur la hauteur est d'autant plus forte que \( m \) est grande : en
effet, 

La dépendance en \( \expapx \) dans le décompte précédent est en
\begin{equation}
  \left( \frac1\expapx \right)^{\frac12 + \frac r2 \cdot \frac m{m-g}}
  \ln \left( \frac1\expapx \right)
  \pmm.
\end{equation}
L'exposant du premier facteur tend donc vers \( (r+1) / 2 \) lorsque \( m \)
tend vers l'infini. Cependant, quand \( m \) grandit, le facteur \(
  (4mg)^{mg} \) dans le décompte croît rapidement et par ailleurs la condition
de hauteur croît rapidement.



\section{Décompte grossier des petits points}
\label{sec:small-points}

Il ne semble pas évident d'exploiter l'hypothèse d'approximation dans le
décompte des petits points. Nous rappelons donc pour mémoire comment
compter les points de petite hauteur d'une variété abélienne appartenant à un
groupe de type fini donné : ce décompte repose sur une propriété élémentaire
en géométrie euclidienne, que l'on commence par rappeler.

\begin{fact}
  Soient \( E \) un espace euclidien de dimension \( r \) et deux réels \(
    \rho \) et \( \mu \). On peut recouvrir toute boule (fermée) de rayon \(
    \rho \) par des boules (ouvertes) de rayon \( \mu \) en nombre inférieur à
  \( ( 2 \, \frac\rho\mu + 1 )^r \).
\end{fact}

\begin{proof}
  C'est le lemme~6.1, p.~541 de \cite{remdcl}, où l'on a par ailleurs effectué
  le changement de notations \( \mu = \rho / \gamma \).  L'énonce donné dans
  la référence citée ne précise pas si les boules sont ouvertes ou fermées,
  mais on constate facilement que la preuve qui y est proposée fonctionne
  parfaitement pour la variante la plus forte, énoncée ci-dessus.
\end{proof}

On en déduit la majoration suivante du nombre de points de petite hauteur sans
aucune hypothèse d'approximation.

\begin{coro} \label{c:small-va}
  Soit \( \Gamma \) un sous-groupe de type fini de \( \va(\Qbar) \) ; on note
  \( r \) le rang de \( \Gamma \) et \( \hmin \) le minimum de \(
    \hautn x \) quand \( x \) parcourt l'ensemble des points d'ordre infini de
  \( \Gamma \).  Pour tout réel positif \( R \), on a
  \begin{equation}
    \card \set*{
      x \in \Gamma
      \text{ tel que }
      \hautn x \le R
    }
    \le
    \card \Gamma_\torsion
    \cdot
    \left( 1 + 2\sqrt{R / \hmin} \right)^r
  \end{equation}
  où \( \Gamma_\torsion \) désigne l'ensemble des points de torsion de \(
    \Gamma \).
\end{coro}

\begin{proof}
  Dans \( \Gamma \otimes_\Z \R \) muni de la structure euclidienne induite par
  la hauteur normalisée, on applique le fait précédent avec \( \rho = \sqrt R
  \) et \( \mu = \sqrt{\hmin} \) puis on remarque que la préimage dans \(
    \Gamma \) de chacune des boules ouvertes de rayon \( \mu \) est composée
  de points qui sont tous égaux modulo \( \Gamma_\torsion \).
\end{proof}

\begin{rem}
  Si dans l'énoncé précédent on prend \( \Gamma = \va(\cdn) \), on retrouve le
  lemme~2.11.1, p.~117 de~\cite{farhith}. On rappelle par ailleurs que dans ce
  cas (\todo??) donne une estimation de \( \card \Gamma_\torsion \) et
  (\todo??) une estimation de \( \hmin \).
\end{rem}

On en déduit pour finir un résultat de décompte complet (petits et grands
points) des approximations exceptionnelles.

\begin{thm} \label{t:inc-small}
  Soient \( V \) un translaté par un point algébrique d'une sous-variété
  abélienne \( \vai \) de \( \va \) de degré \( d \) et de dimension \( u \).
  On considère de plus un sous-groupe \( \Gamma \subset \va(\Qbar) \) de type
  fini ; on note \( r \) son rang et \( \hmin \) le minimum de \( \hautn x \)
  quand \( x \) parcourt l'ensemble des points d'ordre infini de \( \Gamma \).
  Le nombre de points de \( \Gamma \) distincts modulo \( \vai \) et
  satisfaisant à
  \begin{equation}
    0 < \distv x V
    <
    \bigl( d \sbin{d+\dimp}{\dimp}^3 \bigr)^{-\dv/2}
    \hautm[2] x ^{-\wtapx \expapx}
    \quad \forall \place \in \placesapx
  \end{equation}
  est majoré par
  \begin{equation}
    \card \Gamma_\torsion
    \left(
      \frac{
        5 \cst{ht-m=g+1}
        \Bigl(
          \expapx^{-1}
          \, d \sbin{\dimp + d}{d}
          \, \cst{lambda}
        \Bigr)^{(\sqrt2(\genre+1))^{2\genre(\genre+1)}}
      }{
        \min(1, \hmin)
      }
    \right)^{r/2}
  \end{equation}
  avec
  \begin{align}
    \newcst[]{ht-m=g+1}
    & =
    d \max \bigl(
      d^\genre \hautl[1]{\va}, \hlclab, \htcmp
    \bigr)
    + (\genre + 1) \deg \va
    \Bigl(
      d^\dimp \ln(d) \dimp
    \\ & \qquad
      + d^\genre \bigl (
        \hautl[1] V
        + (u + 2) \ln (d + 1) ( d + \dimp + 1 )
        + \ln((\genre+1)\/2)
      \bigr)
    \Bigr)
    \\
    \newcst[]{lambda}
    & =
    86 \nclmaps \deg \va
    \bigl( 8 \genre (\genre+1) \bigr)^\genre
    \pmm.
  \end{align}
\end{thm}

\begin{proof}
  Remarquons qu'on peut supposer \( r \ge 1 \) car sinon le résultat est
  banal.  En combinant la proposition~\vref{p:big-gen} et le
  corollaire~\vref{c:small-va} appliqué avec \( R = \cst{ht-v-gen1}
    \Lambda_4^{(2\puiss\genre)^{\puiss\genre}} \), on voit que le nombre de
  points en question est au plus
  \begin{equation}
    \sqrt{\frac d \expapx}
    2 \puiss
    (2\puiss\genre)^{\puiss\genre}
    \ln \Lambda_4
    \left( 3 \sqrt{ \puiss \cst{v-gen} } \right)^r
    +
    \card \Gamma_\torsion
    \cdot
    \left( 1 + \frac2{\sqrt{\hmin}}
      \left(
        \cst{ht-v-gen1} \Lambda_4^{(2\puiss\genre)^{\puiss\genre}}
      \right)^{1/2}
    \right)^r
    \pmm.
  \end{equation}
  On voit facilement que le premier terme est majoré par \( \Lambda_4^{3r} \)
  : en effet, on a
  \begin{align}
    6 \sqrt{\frac d \expapx}
    \cdot \sqrt{ \cst{v-gen} }
    & \le
    \cst{v-gen}
    \\
    \ln \Lambda_4
    & \le
    \Lambda_4
    \\
    \puiss \sqrt\puiss
    (2\puiss\genre)^{\puiss\genre}
    & \le
    (\sqrt2 \puiss \genre)^{2\puiss\genre}
  \end{align}
  qui donne bien le résultat voulu en prenant le produit vu la définition de
  \( \Lambda_4 \). En particulier, comme \( (2\puiss\genre)^{\puiss\genre} /
    2 \ge 8 \), le premier terme est largement majoré par \( \Lambda_4^{ \frac
      r2 (2\puiss\genre)^{\puiss\genre} } / \Lambda_4 \).

  Par ailleurs, le deuxième terme est majoré par
  \begin{equation}
    \card \Gamma_\torsion
    \left(
      \frac{
        (2 + \frac1{\Lambda_4})
      }{
        \min(1, \sqrt{\hmin})
      }
      \left(
        \cst{ht-v-gen1} \Lambda_4^{(2\puiss\genre)^{\puiss\genre}}
      \right)^{1/2}
    \right)^r
    \pmm.
  \end{equation}
  Il ne reste alors qu'à remarquer que \( \card \Gamma_\torsion \ge 1 \) et
  que \( (2 + \frac1{\Lambda_4})^r + \frac1{\Lambda_4} \le 5^{r/2} \) pour
  majorer la somme par
  \begin{equation}
    \card \Gamma_\torsion
    \left(
      \frac{
        5 \cst{ht-v-gen1} \Lambda_4^{(2\puiss\genre)^{\puiss\genre}}
      }{
        \min(1, \hmin)
      }
    \right)^{r/2}
    \pmm.
  \end{equation}
  Cette majoration est valable pour tout \( \puiss > \genre \) ;
  on remarque par ailleurs que cette quantité est croissante en \( \puiss \) :
  en effet, la seule partie qui ne l'est pas est l'exposant \( \puiss /
    (\puiss - \genre) \) dans la définition de \( \cst{v-gen} \) mais celui-ci
  est largement composé par le \( (2\puiss\genre)^{\puiss\genre} \) en
  exposant de \( \Lambda_4 \).  On pose donc \( \puiss = \genre + 1 \) pour
  obtenir la meilleure estimation possible, ce qui donne
  \begin{align}
    \Lambda_4^{(2\puiss\genre)^{\puiss\genre}}
    & =
    \left(
      \bigl(
        86 \nclmaps \cdot 5^\genre d \sbin{\dimp + d}{d}
        \, \expapx^{-1}
      \bigr)^{ \genre + 1 }
      \bigl( (\sqrt2 \genre(\genre+1)\Delta)^\genre \deg \va \bigr)^{\genre+1}
    \right)^{(2\genre(\genre+1))^{\genre(\genre+1)}}
    \\ & \le
    \Bigl(
      \expapx^{-1}
      \, d \sbin{\dimp + d}{d}
      \, \cst{lambda}
    \Bigr)^{(\genre+1) (2\genre(\genre+1))^{\genre(\genre+1)}}
  \end{align}
  puis on note que
  \(
    (\genre+1) \genre^{\genre(\genre+1)}
    \le
    (\genre+1)^{2\genre(\genre+1)}
  \) pour conclure.
\end{proof}

Ce résultat n'est pas particulièrement satisfaisant dans le sens où la
majoration est essentiellement donnée par celle provenant des petits points et
ne tenant donc pas compte de l'hypothèse d'approximation. La section suivante
présente une autre approche possible : éliminer les petits points.



\section{Élimination des petits points}
\label{sec:no-small-points}

On établit ici une variante du résultat principal où l'on se dispense de
l'hypothèse que les points sont grands en renforçant l'hypothèse
d'approximation pour éliminer les petits grâce à l'inégalité de
\bsc{Liouville}. Le résultat fondamental d'élimination des petits points est
le suivant.

\begin{lem}
  Soit \( V \) une sous-variété de \( \va \), de dimension \( u \) et de degré
  \( d \). Si \( R \) et \( F \) sont deux réels positifs,
  l'ensemble des points \( x \in \va(\Qbar) \) tels que
  \begin{equation}
    0
    <
    \prod\placerange
    \distv x V ^\degv
    \le
    \expb^{-F}
    \hautm[2]{x}^{-\expapx}
    \quad\text{et}\quad
    \hautn x \le R
  \end{equation}
  est vide dès que
  \( F
    >
    (d-\expapx) R
    + \hautl[1]{ \chow V }
    + d (u+1) \ln(3d)
    + \frac32 \ln(n+1)
    + (d-\expapx) \htcmp
  \).
\end{lem}

\begin{proof}
  Si l'ensemble en question n'est pas vide, on choisit un point \( x \) dedans
  et on lui applique l'inégalité de \bsc{Liouville}
  (proposition~\vref{p:liouville}) :
  \begin{equation}
    \frac1{
      (n+1)^{3/2}
      (3d)^{d (u+1)}
      \, \hautm[1]{ \chow V }
      \, \hautm[2] x ^d
    }
    \le
    \prod\placerange
    \distv x V ^\degv
    \le
    \expb^{-F}
    \hautm[2]{x}^{-\expapx}
  \end{equation}
  puis en prenant les opposés des logarithmes
  \begin{align}
    F
    & \le
    (d-\expapx) \hautl[2] x
    + \hautl[1]{ \chow V }
    + d (u+1) \ln(3d)
    + \frac32 \ln(n+1)
    \\ & \le
    (d-\expapx) R
    + \hautl[1]{ \chow V }
    + d (u+1) \ln(3d)
    + \frac32 \ln(n+1)
    + (d-\expapx) \htcmp
    \pmm.
  \end{align}
  Par contraposée, l'ensemble considéré est vide si cette inégalité est
  fausse.
\end{proof}

On en déduit immédiatement le corollaire pratique suivant.

\begin{coro} \label{c:kill-small}
  Soit \( V \) une sous-variété de \( \va \), de dimension \( u \) et de degré
  \( d \). Si \( R \) est un réel supérieur ou égal à
  \begin{equation}
    \frac1\expapx \left(
      \hautl[1]{ \chow V }
      + d (u+1) \ln(3d)
      + \frac32 \ln(n+1)
      + d \htcmp
    \right)
    \pmm,
  \end{equation}
  il n'existe aucun point \( x \in \va(\Qbar) \) tels que
  \begin{equation}
    0
    <
    \prod\placerange
    \distv x V ^\degv
    \le
    \expb^{- d R}
    \hautm[2]{x}^{-\expapx}
    \quad\text{et}\quad
    \hautn x \le R
    \pmm.
  \end{equation}
\end{coro}

\begin{proof}
  Application immédiate du lemme précédent avec \( F = d R \).
\end{proof}

On peut alors renforcer la condition d'approximation pour interdire
l'existence de petits points, ce qui donne la proposition suivante.

\begin{prop} \label{p:all-gen}
  Soient \( V \) un translaté par un point algébrique d'une sous-variété
  abélienne \( \vai \) de \( \va \) de degré \( d \) et de dimension \( u \).
  On considère de plus un sous-groupe \( \Gamma \subset \va(\Qbar) \) de rang
  fini \( r \).
  Pour tout réel \( \expapx > 0 \) et tout entier \( \puiss \ge \genre + 1 \),
  il existe au plus
  \begin{equation}
    \sqrt{\frac d \expapx}
    (4\puiss\genre)^{\puiss\genre}
    \ln \Lambda_4
    \left(
      3 \sqrt\puiss
      \bigl(
        86 \nclmaps \cdot 5^\genre d \sbin{\dimp + d}{d}
        \, \expapx^{-1}
        \bigr)^{ \frac{\puiss}{2(\puiss-\genre)} }
    \right)^r
  \end{equation}
  points de \( \Gamma \) distincts modulo \( \vai \) et satisfaisant
  à la condition suivante :
  \begin{align}
    0 < \distv{\ex*} V
    & <
    \bigl( d \sbin{d+\dimp}{\dimp}^3 \bigr)^{-\dv/2}
    \exp\left(
      - \wtapx d \cst{ht-v-gen1} \Lambda_4^{(2\puiss\genre)^{\puiss\genre}}
    \right)
    \hautm[2]{\ex*}^{-\wtapx \expapx}
    \quad \forall \place \in \placesapx
  \end{align}
  avec \( \Lambda_4 \) et \( \cst{ht-v-gen1} \) comme au
  lemme~\vref{l:big-by-cone}.
\end{prop}

\begin{proof}
  La condition d'approximation considérée implique en particulier
  \begin{equation}
    0
    <
    \prod\placerange \distv{\ex*} V ^\degv
    <
    \exp\left(
      - d \cst{ht-v-gen1} \Lambda_4^{(2\puiss\genre)^{\puiss\genre}}
    \right)
    \hautm[2]{\ex*}^{-\expapx}
  \end{equation}
  de sorte qu'on peut appliquer le corollaire~\vref{c:kill-small} avec
  \( R = \cst{ht-v-gen1} \Lambda_4^{(2\puiss\genre)^{\puiss\genre}} \)
  car cette quantité satisfait amplement l'hypothèse du corollaire. Ainsi, on
  voit que tous les points satisfaisant cette condition d'approximation
  satisfont aussi \( \hautn{\ex[1]} \ge R \) et on peut appliquer la
  proposition~\vref{p:big-gen} pour aboutir au résultat annoncé.
\end{proof}

On peut alors choisir \( \puiss \) de façon à optimiser ce résultat dans une
des deux directions suivantes : obtenir la condition la plus faible possible,
ou bien le décompte le plus précis possible. Dans le premier sens, on obtient
le résultat suivant.

\begin{coro}
  Soient \( V \) un translaté par un point algébrique d'une sous-variété
  abélienne \( \vai \) de \( \va \) de degré \( d \) et de dimension \( u \).
  On considère de plus un sous-groupe \( \Gamma \subset \va(\Qbar) \) de rang
  fini \( r \).
  Pour tout réel \( \expapx > 0 \), le nombre de points de \( \Gamma \)
  distincts modulo \( \vai \) et satisfaisant à la condition
  \begin{align}
    0 < \distv{\ex*} V
    & <
    \hautm[2]{\ex*}^{-\wtapx\expapx}
    \\ & \qquad \cdot
    \bigl( d \sbin{d+\dimp}{\dimp}^3 \bigr)^{-\dv/2}
    \exp\left(
      - \wtapx 2 d
      \cst{ht-m=g+1}
      \Bigl(
        \expapx^{-1}
        d \sbin{\dimp + d}{d}
        \cst{lambda}
      \Bigr)^{(\sqrt2(\genre+1))^{2\genre(\genre+1)}}
    \right)
  \end{align}
  pour tout \( \place \in \placesapx \)
  avec \( \cst{ht-m=g+1} \) et \( \cst{lambda} \) comme au
  théorème~\vref{t:inc-small} est au plus
  \begin{equation}
    \sqrt{\frac d \expapx}
    (4\genre(\genre+1))^{\genre(\genre+1)}
    \bigl(
      - \ln \expapx
      + \cst{ln-lambda}
    \bigr)
    \left(
      3 \sqrt{\genre+1}
      \bigl(
        86 \nclmaps \cdot 5^\genre d \sbin{\dimp + d}{d}
        \, \expapx^{-1}
        \bigr)^{ \frac{\genre+1}{2} }
    \right)^r
  \end{equation}
  avec
  \begin{equation}
    \newcst[]{ln-lambda}
    =
    9
    + 2 \ln \nclmaps
    + \ln(d+1) (n+1 + \genre(\genre+1))
    + (\genre+1) \left( \deg\va + 2\genre (\ln(\genre+1) + 1) \right)
    \pmm.
  \end{equation}
\end{coro}

\begin{proof}
  L'énoncé de la proposition~\vref{p:all-gen} est valable pour tout \( \puiss
    \ge \genre + 1 \).  Comme dans la preuve du
  théorème~\vref{t:inc-small}, on remarque la quantité \(
    \cst{ht-v-gen1} \Lambda_4^{(2\puiss\genre)^{\puiss\genre}}
  \) est minimale pour \( \puiss = \genre + 1 \) et on spécialise donc
  la proposition~\vref{p:all-gen} avec cette valeur.
\end{proof}

\textcolor{out}{\todo Dans l'autre sens, discussion et proposition de choix}

\cleardoublepage
\endinput

% vim: spell spelllang=fr
