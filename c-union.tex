\chapter{Déduction des résultats principaux}
\label{chap:union}

Nous déduisons ici des deux chapitres précédents des décomptes
d'approximations exceptionnelles. Plus précisément, on donne un décompte des
points de hauteur assez grande (au sens de~\eqref{e:Vbig} et~\eqref{e:Mbig}),
que nous appellerons grands points, en combinant l'inégalité de \bsc{Vojta} à
chacune des deux inégalités de \bsc{Mumford} obtenues précédemment.

Auparavant, on introduit des conditions sur les ensembles d'approximations
exceptionnelles considérés et on discute de leur caractère nécessaire. On
introduit également un résultat permettant de passer d'une condition
d'approximation avec un système d'inégalités à la forme plus naturelle d'un
produit sur les places. Enfin, on conclut ce chapitre par l'évocation de deux
méthodes de traitement des petits points : un décompte trivial, et une façon
de les éliminer.


\section{Obstruction au décompte absolu}
\label{sec:obstruction}

(À écrire.)



\section{Hypothèse d'approximation produit}
\label{sec:ha-prod}

Comme annoncé dans l'introduction (page~\pageref{e:ha-prod}), on montre ici
comment transformer un énoncé où la condition d'approximation fait apparaître
une inégalité par place en une version où l'hypothèse d'approximation fait
apparaître un produit sur l'ensemble des places.

Pour cela, on utilise le lemme suivant, qui n'est rien d'autre qu'une version
plus abstraite du lemme~2.12.1, p. 120 de~\cite{farhith}. Par ailleurs, par
rapport à cette référence, on a fixé \( \eps'(T) = \eps/2 \), d'où \( A(T) =
  \card T \), car c'est la valeur qui a le plus de sens pour les applications,
afin de simplifier légèrement l'énoncé et sa preuve. L'énoncé ci-dessous est
volontairement aussi abstrait que possible afin de mettre en lumière le
caractère purement combinatoire de la preuve et d'être utilisable dans le plus
grand nombre de situations ; immédiatement après l'énoncé on détaille la
correspondance avec la situation étudiée ici afin d'en éclaircir le sens.

\begin{lem}
  Soit \( A \) un ensemble et \( H \) une fonction de \( A \) dans \( [1,
    +\infty \mathclose[ \). Soit par ailleurs \( M \) un autre ensemble et,
  pour chaque élément \( v \) de \( M \) :
  \begin{itemize}
    \item une fonction \( d_v \colon A \to [0, 1] \) ;
    \item deux réels \( 0 < \degv < 1 \) et \( c_v > 0 \).
  \end{itemize}
  Pour tout réel \( \eps > 0 \) et tout partie finie \( S \) de \( M \), on
  définit
  \[
    E(\eps, S)
    =
    \set*{
      x \in A
      \text{ tel que }
      \prod_{v \in S} d_v(x)^\degv
      \le
      \bigl( \prod_{v \in S} c_v^\degv \bigr)
      H(x)^{-\eps}
    }
    \pmm.
  \]
  De plus, pour toute famille de réels \( (\lambda_v)_{v\in S} \), on définit
  \[
    E'(\eps, S, \lambda)
    =
    \set*{
      x \in A
      \text{ tel que }
      d_v(x)
      \le
      c_v H(x)^{-\lambda_v \eps}
      \quad \forall v \in S
    }
    \pmm.
  \]
  On note enfin \( \parts M \) l'ensemble des parties finies de \( M \) et on
  suppose qu'il existe une fonction \( g \colon \R_{>0} \times \parts M \to \N
  \) telle que pour tout \( \eps > 0 \), tout \( S \in \parts M \) non vide et
  toute famille \( (\lambda_v)_{v\in S} \) de réels positifs telle que \(
    \sum_{v\in S} \lambda_v \degv = 1 \), on ait
  \[
    \card E'(\eps, S, \lambda)
    \le
    g(\eps, S)
    \pmm.
  \]
  Alors, pour tout \( \eps > 0 \) et tout \( S \in \parts M \) on a
  \[
    \card E(\eps, S)
    \le
    \sum_{T \in \parts S \minusset\emptyset}
    \binom{2\card T - 1}{\card T - 1}
    g(\eps/2, T)
    \pmm.
  \]
\end{lem}

\begin{rem} \label{r:app-prod}
  Dans l'application prévue, les paramètres seront choisis de la façon
  suivante.  L'ensemble \( A \) sera l'intersection d'un système complet de
  représentants modulo \( \vai \) dans un sous-groupe de rang fini (voire de
  type fini) de \( \va(\Qbar) \) avec \( \va(\Qbar) \setminus \avar(\Qbar) \) et
  \( H \) la hauteur induite sur cet ensemble par le choix d'un plongement
  projectif de \( \va \). L'ensemble \( M \) sera l'ensemble des places d'un
  corps de nombres \( \cdn \) suffisamment gros et en chaque place
  \( \degv = [\cdn_v : \Q_v] / [\cdn : \Q] \) est le degré local divisé par le
  degré global. En revanche, le choix des constantes \( c_v \) dépendra de
  l'application envisagée. On reconnaît la condition sur la famille
  \( \lambda \), et la fonction \( g \) est alors donnée par l'un des énoncés
  de la section précédente.
\end{rem}

\begin{proof}
  On suppose \( S \) et \( \eps \) fixés. On remarque que pour toute famille
  \( \lambda \) telle que que \( \sum_{v \in S} \lambda_v \degv = 1 \) on a \(
    E'(\eps, S, \lambda) \subset E(\eps, S) \) ; à l'inverse on aimerait bien
  recouvrir \( E(\eps, S) \) par de tels ensembles, en nombre fini. Dans ce
  but, pour tout \( x \in A \) (tel que \( H(x) > 1 \) pour simplifier pour
  l'instant), on introduit une famille de réels \( \lambda(x) \) telle que
  \begin{equation} \label{e:x2wt}
    d_v(x)
    =
    c_v H(x)^{-\lambda_v(x) \eps}
    \quad \forall v \in S
  \end{equation}
  de sorte que \( x \in E'(\eps, S, \lambda(x)) \). De plus, si \( x \in E(\eps,
    S) \) on a \( \sum_{v \in S} \lambda_v(x) \degv \ge 1 \). Ainsi, on a
  recouvert \( E(\eps, S) \) par des ensembles de la forme \( E'(\eps, S,
    \lambda(x)) \) mais ceux-ci ne sont \lat{a priori} pas en nombre fini ;
  par ailleurs les familles \( \lambda(x) \) obtenues ne sont pas
  nécessairement composées que de nombres positifs. (Le fait que leur somme
  pondérée ne soit pas exactement \( 1 \) n'est en revanche pas gênant, il
  suffirait de choisir une famille \( \lambda'(x) \) plus petite (pour l'ordre
  produit) sommant à \( 1 \), car la fonction \( \lambda \mapsto E'(\eps, S,
    \lambda) \) est décroissante.)

  Pour forcer les \( \lambda_v(x) \) à être positifs\footnote{Ce qui n'est en
    soi pas indispensable (on pourrait en fait se dispenser de cette hypothèse
    dans les résultats de décompte précédents en considérant un ensemble de
    places plus petit) mais sera utile pour assurer la finitude et contrôler
    le cardinal de l'ensemble \( C \) ci-dessous.} on introduit la fonction \(
    \pi \colon E(\eps, S) \to \parts S \) définie par
  \begin{equation}
    \pi(x)
    =
    \set{
      v \in S
      \text{ tel que }
      d_v(x) \le c_v
    }
    \pmm.
  \end{equation}
  On remarque que \( \pi \) prend en fait ses valeurs dans \( \parts S
    \minusset\emptyset \).  Par ailleurs, pour toute partie non vide \( T \)
  de \( S \), on introduit l'ensemble
  \begin{equation}
    C
    =
    \set{
      (\lambda_v)_{v \in T}
      \text{ tel que }
      \lambda_v \degv \card T \in \N
      \text{ et }
      \sum_{v \in T} \lambda_v \degv = 1
    }
  \end{equation}
  qui est fini et de cardinal \( \binom{2\card T - 1}{\card T - 1} \) car il
  est en bijection avec l'ensemble des familles de \( \card T \) entiers
  naturels dont la somme est \( \card T \). Pour obtenir la conclusion du
  lemme, il suffit donc de montrer que
  \begin{equation}
    \pi^{-1}(T)
    =
    \bigcup_{\lambda \in C} E(\eps/2, T, \lambda)
  \end{equation}
  (où l'on a noté \( \pi^{-1}(T) = \pi^{-1}(\set T) \) pour alléger) car les
  \( \pi^{-1}(T) \) pour \( T \) parcourant l'ensemble des parties non vides
  de \( S \) forment une évidemment partition de \( E(\eps, S) \).

  Fixons donc \( \emptyset \neq T \in \parts S \) et \( x \in \pi^{-1}(T)
  \). Par définition de \( E(\eps, S) \) et de \( \pi \) on a
  \begin{align}
    \bigl( \prod_{v \in S} c_v^\degv \bigr)
    H(x)^{-\eps}
    \ge
    \prod_{v \in S} d_v(x)^\degv
    \ge
    \prod_{v \in T} d_v(x)^\degv
    \prod_{v \in S \setminus T} c_v^\degv
  \end{align}
  d'où \( x \in E(\eps, T) \). Par ailleurs, si \( H(x) = 1 \) il est clair
  que \( x \in E'(\eps, S, \lambda) \) pour n'importe quelle famille \(
    \lambda \in C \), on suppose donc \( H(x) > 1 \) et on associe à \( x \)
  une famille \( \lambda(x) \) comme en~\eqref{e:x2wt}. Cette fois-ci, les \(
    \lambda_v(x) \) sont tous positifs ou nuls. Par ailleurs, on a
  successivement
  \begin{align}
    2 \card T
    \sum_{v \in T} \lambda_v(x) \degv
    & \ge
    2 \card T
    \\
    \sum_{v \in T} \floor{ 2 \lambda_v(x) \degv \card T }
    & \ge
    \card T
  \end{align}
  donc il existe des entiers naturels \( a_v(x) \) inférieurs ou égaux à
  \( 2 \lambda_v(x) \degv \card T \) et dont la somme est exactement
  \( \card T \). On pose alors \( \lambda'(x) = a_v(x) / (\degv \card T) \) de
  sorte que \( \lambda'(x) \in C \) et \( \lambda'_v(x) \le 2 \lambda_v(x) \),
  ce qui implique que \( x \in E'(\eps/2, T, \lambda'(x)) \), achevant ainsi
  la preuve.
\end{proof}

En pratique, la fonction de comptage ne dépend pas de l'ensemble de places
considérées, on utilisera donc le scolie suivant.

\begin{sco} \label{s:ha-prod}
  Dans la situation du lemme précédent, si l'on suppose de plus que la
  fonction \( g \) ne dépend pas de son deuxième argument, alors on a
  \( \card E(\eps, S) \le 5^{\card S} g(\eps/2) \).
\end{sco}

\begin{proof}
  C'est un calcul élémentaire, fait page~125 de \cite{farhith}.
\end{proof}



\section{Grands points, cas des sous-groupes}
\label{sec:big-points}

On commence par majorer le nombre de points dans chaque cône tronqué.

\begin{lem} \label{l:big-by-cone}
  Soient \( \avar \) un translaté par un point algébrique d'une sous-variété
  abélienne \( \vai \) de \( \va \) de degré \( \adeg \) et de dimension \(
    \adim \).  On fixe un réel \( 0 < \eps \), un entier \( m \ge g + 1 \) et
  on considère une famille de points \( x_1, \dots, x_p \in \va(\Qbar)
  \) deux à deux distincts modulo \( \vai \) et satisfaisant simultanément aux
  conditions suivantes :
  \begin{align}
    \label{e:big-ha}
    0 < \distv{x_i} \avar
    & <
    \alpha_v^{-1}
    \hautm2{x_i}^{-\wtapx \eps}
    \quad \forall v \in \placesapx
    \\ \label{e:big-big}
    \hautn{x_1}
    & > \cst{ht-v-gen1} \Lambda_4^{(2mg)^{mg}}
    \\ \label{e:big-cos}
    \cos(x_i, x_j)
    & > 1 - \frac1{ m \, \cst{v-gen} }
  \end{align}
  avec
  \begin{align}
    \cst{v-gen}
    & =
    \bigl(
      86 \nclmaps \cdot 5^g \adeg s \, \eps^{-1}
    \bigr)^{ \frac{m}{m-g} }
    \\
    \Lambda_4
    & =
    \cst{v-gen}
    \bigl( (\sqrt2 m g \adeg)^g \deg \va \bigr)^m
    \\
    \cst{ht-v-gen1}
    & =
    \adeg \max \bigl( \adeg^g \hautl1{\va}, \hlclab, \htcmp \bigr)
    + (g + 1) \deg \va
    \Bigl(
      \adeg^n \ln(\adeg) n
    \\ & \qquad
      + \adeg^g \bigl (
        \hautl1 \avar + (\adim + 2) \ln(\adeg + 1) (\adeg + n+1) + \ln(m/2)
      \bigr)
    \Bigr)
  \end{align}
  et au choix :
  \begin{enumthm}
  \item \( s = \min(\sbin{\adeg+n}{n}, \card \placesapx) \) et \( \alpha_v =
      \bigl( \adeg \sbin{\adeg+n}{n}^3 \bigr)^{\dv/2} \) ; ou bien
  \item \( s = 1 \) et \( (\alpha_v)_v \) une certaine famille de réels tous
    supérieurs ou égaux à \( 1 \), satisfaisant \(
      \prod\placerange \alpha_v^\degv
      \le
      \hautm1 \avar ^2
      \cdot (\adeg+1)^{ 2 (\adim+2) (\adeg + n+1) } \cdot \frac{m^2}4
      \cdot \expb^{\adeg n} \adeg \sbin{\adeg+n}{n}^{1/2}
    \).
  \end{enumthm}
  On a alors nécessairement
  \begin{equation}
    p
    \le
    \sqrt{\frac \adeg \eps}
    (2mg)^{mg+1}
    \ln \Lambda_4
    \pmm.
  \end{equation}
\end{lem}

\begin{proof}
  Pour commencer, remarquons qu'on peut supposer \( \eps < \adeg + 1 \). En
  effet, sinon le corollaire~\vref{c:nobig-liouville} s'applique,
  car~\eqref{e:big-ha} implique immédiatement~\eqref{e:nobig-ha}
  et~\eqref{e:big-big} implique assez largement~\eqref{e:nobig-big} vu la
  définition de \( \cst{ht-v-gen1} \).

  Soit \( x_1, \dots, x_p \) une famille comme dans l'énoncé, qu'on suppose de
  plus ordonnée par hauteur normalisée croissante. On pose
  \( \phi =( m \, \cst{v-gen})^{-1} \) et
  \( \rho = \sqrt{ \eps / \adeg } \).  On constate facilement que
  \( \phi \le \eps / (2580 \adeg) \) de sorte que ce couple \( (\rho, \phi) \)
  satisfait à~\eqref{e:rho-phi-grp}. Par ailleurs, il est assez facile de voir
  que la condition~\eqref{e:big-big} est plus forte que~\eqref{e:Mbig} : en
  effet \( \cst{v-gen} \ge 2/\eps \cdot \adeg (\adim+1) \) et la parenthèse
  dans le membre de droite de~\eqref{e:Mbig} est largement majorée par \( 5
    \cst{ht-v-gen1} \).

  On peut donc appliquer le théorème~\vref{t:mumford-grp} avec \( x = x_i \)
  et \( y = x_{i+1} \) et conclure que
  \( \hautn{ x_{i+1} } > (1 + \sqrt{\eps / \adeg}) \hautn{ x_i } \) pour tout
  \( i \le p-1 \). On pose alors
  \begin{equation}
    \eta
    =
    \left\lceil
      (2mg)^{mg}
      \frac{ \ln \Lambda_4 }{ \ln (1 + \sqrt{\eps / \adeg}) }
    \right\rceil
    \pmm.
  \end{equation}
  On suppose que \( p \ge (m-1) \eta + 1 \) : on peut alors poser
  \( y_i = x_{1 + (i-1) \eta} \) pour tout \( i \in \set{1, \dots, m} \) de
  sorte que, pour tout \( i \le m-1 \), on a
  \begin{equation}
    \hautn{ y_{i+1} }
    >
    (1 + \sqrt{\eps / \adeg})^\eta
    \hautn{ y_i }
    \ge
    \Lambda_4^{(2mg)^{mg}}
    \hautn{ y_i }
  \end{equation}
  et la famille \( (y_i)_i \) contredit le corollaire~\vref{c:vojta-gen2}, ce
  qui est absurde.

  Ainsi, on a \( p \le (m-1) \eta \). On remarque que la fonction \( t
    \mapsto t / \ln(1+t) \) est croissante pour \( t > 0 \) et, comme on a
  supposé \( \eps \le d+1 \le 2d \), on a
  \begin{equation}
    \frac1{ \ln(1 + \sqrt{\eps/d}) }
    \le
    \frac{ \sqrt2 }{ \ln(1 + \sqrt2) }
    \sqrt{ \frac d \eps }
    \le
    2 \sqrt{ \frac d \eps }
  \end{equation}
  en vérifiant numériquement pour la dernière inégalité. On a alors
  \begin{equation}
    p
    \le
    (m - 1)
    \left(
      (2mg)^{mg} \ln \Lambda_4
      \cdot 2 \sqrt{ \frac d \eps }
      + 1
    \right)
    \le
    2 m
    (2mg)^{mg} \ln \Lambda_4
    \sqrt{ \frac d \eps }
  \end{equation}
  qui donne directement le résultat annoncé.
\end{proof}

On utilise ensuite le fait suivant.

\begin{fact} \label{f:nb-cones}
  Soient \( r \) un entier et \( \gamma > 0 \) un réel. On peut recouvrir \(
    \R^r \) par \( \floor{(1 + \sqrt{8/\gamma})^r} \) ensembles dans chacun
  desquels deux points quelconques satisfont \( \cos(x, y) \ge 1 - \gamma \).
\end{fact}

\begin{proof}
  C'est le corollaire~6.1, p.~542 de~\cite{remdcl}.
\end{proof}

\begin{coro} \label{c:big-gen}
  Soient \( \avar \) un translaté par un point algébrique d'une sous-variété
  abélienne \( \vai \) de \( \va \) de degré \( \adeg \) et de dimension \(
    \adim \).  On considère de plus un sous-groupe \( \grp \subset \va(\Qbar)
  \) de rang fini \( r \).  Pour tout réel \( \eps > 0 \) et tout entier \( m
    \ge g + 1 \), il existe au plus
  \begin{equation}
    \sqrt{\frac \adeg \eps}
    (2mg)^{mg+1}
    \ln \Lambda_4
    \left(
      3 \sqrt m
      \bigl(
        86 \nclmaps \cdot 5^g \adeg s \, \eps^{-1}
        \bigr)^{ \frac{m}{2(m-g)} }
    \right)^r
  \end{equation}
  points de \( \grp \) distincts modulo \( \vai \) et satisfaisant
  aux deux conditions suivantes :
  \begin{align}
    0 < \distv{x_i} \avar
    & <
    \alpha_v^{-1}
    \hautm2{x_i}^{-\wtapx \eps}
    \quad \forall v \in \placesapx
    \\
    \hautn{x_1}
    & > \cst{ht-v-gen1} \Lambda_4^{(2mg)^{mg}}
  \end{align}
  avec \( \Lambda_4 \), \( \cst{ht-v-gen1} \), \( s \)  et \( \alpha_v \)
  comme au lemme~\vref{l:big-by-cone}.
\end{coro}

\begin{proof}
  En utilisant le fait précédent avec \( \gamma = m \cst{v-gen} \) , on
  recouvre l'espace euclidien \( \grp \otimes_\Z \R \) muni du produit
  scalaire donné par la hauteur de \NT par des ensembles dans lesquels deux
  points quelconques satisfont~\eqref{e:big-cos}. Le nombre d'ensembles
  nécessaires est au plus
  \begin{equation}
    \left(
      1 + \sqrt{8 m}
      \bigl(
        86 \nclmaps \cdot 5^g \adeg s \, \eps^{-1}
        \bigr)^{ \frac{m}{2(m-g)} }
    \right)^r
    \le
    \left(
      3 \sqrt m
      \bigl(
        86 \nclmaps \cdot 5^g \adeg s \, \eps^{-1}
        \bigr)^{ \frac{m}{2(m-g)} }
    \right)^r
  \end{equation}
  en remarquant que \( 1 + \sqrt{ 8 \cdot 86 \cdot 3 \cdot 5 } \le 3 \sqrt{ 86
      \cdot 3 \cdot 5 } \).

  Il ne reste plus qu'à appliquer lemme~\vref{l:big-by-cone} à chacun de ces
  ensembles pour conclure.
\end{proof}

Il ne reste maintenant plus qu'à passer à une hypothèse d'approximation sous
sa forme produit.

\begin{coro}
  Soient \( \avar \) un translaté par un point algébrique d'une sous-variété
  abélienne \( \vai \) de \( \va \) de degré \( \adeg \) et de dimension \(
    \adim \).  On considère de plus un sous-groupe \( \grp \subset \va(\Qbar)
  \) de rang fini \( r \).  Pour tout réel \( \eps > 0 \) et tout entier \( m
    \ge g + 1 \), il existe au plus
  \begin{equation}
    2 \cdot 5^{\card S} \cdot
    \sqrt{ \frac{\adeg}\eps }
    (2mg)^{mg+1}
    \ln \Lambda_4
    \left(
      3 \sqrt m
      \bigl(
        172 \nclmaps \cdot 5^g \adeg s \, \eps^{-1}
      \bigr)^{ \frac{m}{2(m-g)} }
    \right)^r
  \end{equation}
  points de \( \grp \) distincts modulo \( \vai \) et satisfaisant
  aux deux conditions suivantes :
  \begin{align}
    0 < \prod\placerange \distv{x_i} \avar ^\degv
    & <
    \alpha^{-1}
    \hautm2{x_i}^{-\eps}
    \\
    \hautn{x_1}
    & > \cst{ht-v-gen1} \Lambda_4^{(2mg)^{mg}}
  \end{align}
  avec \( \Lambda_4 \) et \( \cst{ht-v-gen1} \) comme au
  lemme~\vref{l:big-by-cone} et au choix :
  \begin{enumthm}
  \item \( s = \min(\sbin{\adeg+n}{n}, \card \placesapx) \) et \( \alpha =
      \bigl( \adeg \sbin{\adeg+n}{n}^3 \bigr)^{1/2} \) ; ou bien
  \item \( s = 1 \) et \( \alpha =
      \hautm1 \avar ^2
      \cdot (\adeg+1)^{ 2 (\adim+2) (\adeg + n+1) } \cdot \frac{m^2}4
      \cdot \expb^{\adeg n} \adeg \sbin{\adeg+n}{n}^{1/2}
    \).
  \end{enumthm}
\end{coro}

\begin{proof}
  On applique le scolie~\vref{s:ha-prod} au corollaire~\vref{c:big-gen} comme
  prévu à la remarque~\vref{r:app-prod}. On constate que vu la définition de
  \( \Lambda_4 \), en y remplaçant \( \eps \) par \( \eps / 2 \), on
  ajoute \( \ln(2) m / (m-g) \) à son logarithme et que
  \begin{equation}
    \ln \Lambda_4 + \ln(2) m / (m-g)
    \le
    \sqrt 2 \ln \Lambda_4
    \pmm;
  \end{equation}
 les autres changements dans le décompte sont évidents.
\end{proof}

On peut alors simplifier légèrement la condition d'approximation en absorbant
la constante ; on aboutit au résultat suivant.

\begin{coro} \label{c:big-gen-prod}
  Soient \( \avar \) un translaté par un point algébrique d'une sous-variété
  abélienne \( \vai \) de \( \va \) de degré \( \adeg \) et de dimension \(
    \adim \).  On considère de plus un sous-groupe \( \grp \subset \va(\Qbar)
  \) de rang fini \( r \).  Pour tout réel \( \eps > 0 \) et tout entier \( m
    \ge g + 1 \), il existe au plus
  \begin{equation}
    2 \cdot 5^{\card \placesapx} \cdot
    \sqrt{ \frac{\adeg}\eps }
    (2mg)^{mg+1}
    \ln \Lambda_5
    \left(
      3 \sqrt m
      \bigl(
        174 \nclmaps \cdot 5^g \adeg
        \, \eps^{-1}
        \bigr)^{ \frac{m}{2(m-g)} }
    \right)^r
  \end{equation}
  points de \( \grp \) distincts modulo \( \vai \) et satisfaisant
  aux deux conditions suivantes :
  \begin{align}
    0 < \prod\placerange \distv{x_i} \avar ^\degv
    & <
    \hautm2{x_i}^{-\eps}
    \\
    \hautn{x_1}
    & > \cst{ht-v-gen1} \Lambda_5^{(2mg)^{mg}}
  \end{align}
  avec
  \begin{align}
    \Lambda_5
    & =
    \bigl(
        87 \nclmaps \cdot 5^g \adeg
        \, \eps^{-1}
    \bigr)^{ \frac{m}{m-g} }
    \bigl( (\sqrt2 m g \adeg)^g \deg \va \bigr)^m
    \\
    \cst{ht-v-gen1}
    & =
    \adeg \max \bigl(
      \adeg^g \hautl1{\va}, \hlclab, \htcmp
    \bigr)
    + (g + 1) \deg \va
    \Bigl(
      \adeg^n \ln(\adeg) n
    \\ & \qquad
      + \adeg^g \bigl (
        \hautl1 \avar
        + (\adim + 2) \ln (\adeg + 1) ( \adeg + n + 1 )
        + \ln(m/2)
      \bigr)
    \Bigr)
    \pmm.
  \end{align}
\end{coro}

\begin{proof}
  On commencer par majorer chacun des deux choix possibles pour \( \alpha \)
  dans le corollaire précédent. Pour le premier, on a immédiatement
  \begin{equation}
    \ln \bigl( \adeg \sbin{\adeg+n}{n}^3 \bigr)^{1/2}
    \le
    \frac12 \ln d + \frac32 n \ln(d+1)
    \le
    \cst{ht-v-gen1}
    \pmm.
  \end{equation}
  Pour le deuxième choix, on a
  \begin{align}
    \ln \alpha
    & =
    2 \hautl1 \avar
    + 2 (\adim+2) (\adeg + n+1) \ln (\adeg+1)
    + 2 \ln(\tfrac m 2)
    + \adeg n + \ln \adeg + \frac \adeg 2 \ln(n+1)
    \notag
    \\ & \le
    \adeg \bigl(
    \hautl1 \avar + (\adim+2) (\adeg + n+1) \ln (\adeg+1) + \ln(\tfrac m 2)
    \bigr)
    + 2 \adeg n
    \le
    \cst{ht-v-gen1}
    \label{e:maj-ht-alpha}
  \end{align}
  Ainsi, quel que soit le choix pour \( (s, \alpha) \) dans le corollaire
  précédent, on a assez largement
  \(
    \ln \alpha
    \le
    \cst{ht-v-gen1} \Lambda_5 \cdot \eps / 87
  \)
  dont on déduit immédiatement que
  \(
    \hautm2{ x_i }^{\eps/87}
    \ge
    \alpha
  \)
  puis que les hypothèses du présent énoncé impliquent celles du corollaire
  précédent en remplaçant \( \eps \) par \( 86\eps / 87 \). On applique donc
  le corollaire en question pour aboutir au résultat annoncé.

  Par ailleurs, comme les deux valeurs possibles de \( \alpha \) peuvent être
  absorbées à un coût équivalent, il est clair qu'on a intérêt à choisir la
  deuxième option pour le choix de \( (s, \alpha) \) afin d'avoir \( s = 1 \)
  qui minimise les différentes constantes.
\end{proof}

\begin{rem}
  Dans le cas où \( \card \placesapx = 1 \), on aura plutôt intérêt à utiliser
  le corollaire~\vref{c:big-gen} et prendre \( s = \card \placesapx \) puis
  supprimer le \( \alpha_v \) éventuel (si l'unique place de \( \placesapx \)
  est archimédienne) comme ci-dessus en remplaçant \( 86 \) par \( 87 \) dans
  le décompte et la définition de \( \Lambda_4 \).
\end{rem}

La dépendance en \( \eps \) dans le décompte précédent est en
\begin{equation}
  \left( \frac1\eps \right)^{\frac12 + \frac r2 \cdot \frac m{m-g}}
  \ln \left( \frac1\eps \right)
  \pmm.
\end{equation}
L'exposant du premier facteur tend donc vers \( (r+1) / 2 \) lorsque \( m \)
tend vers l'infini. Cependant, quand \( m \) grandit, le facteur \( (2mg)^{mg+1}
\) dans le décompte croît rapidement et la condition de hauteur aussi croît
rapidement. Il n'est pas évident de déterminer la valeur de \( m \) réellement
optimale, mais la valeur \( m = 2g \) semble être un bon compromis : elle
ramène le facteur devant \( r \) dans l'exposant de \( \eps \) à une constante
absolue, sans pour autant coûter beaucoup plus cher que \( g+1 \) sur les
autres facteurs.

\begin{coro} \label{c:big-m=2g}
  Soient \( \avar \) un translaté par un point algébrique d'une sous-variété
  abélienne \( \vai \) de \( \va \) de degré \( \adeg \) et de dimension \(
    \adim \).  On considère de plus un sous-groupe \( \grp \subset \va(\Qbar)
  \) de rang fini \( r \).  Pour tout réel \( \eps > 0 \) et tout entier \( m
    \ge g + 1 \), il existe au plus
  \begin{equation}
    2 \cdot 5^{\card \placesapx} \cdot
    \sqrt{ \frac{\adeg}\eps }
    (4g)^{4g^2+1}
    \ln \Lambda_5'
    \left(
        739 \nclmaps \cdot 7^g \adeg
        \, \eps^{-1}
    \right)^r
  \end{equation}
  points de \( \grp \) distincts modulo \( \vai \) et satisfaisant
  aux deux conditions suivantes :
  \begin{align}
    0 < \prod\placerange \distv{x_i} \avar ^\degv
    & <
    \hautm2{x_i}^{-\eps}
    \\
    \hautn{x_1}
    & > \cst{ht-v-gen1} (\Lambda_5')^{(4g)^{4g^2}}
  \end{align}
  avec
  \begin{align}
    \Lambda_5'
    & =
    \bigl( 87 \nclmaps \adeg \, \eps^{-1} \bigr)^2
    \bigl( 5 (\deg \va) (2\sqrt2 g^2 \adeg)^g \bigr)^{2g}
    \\
    \cst{ht-v-gen1}'
    & =
    \adeg \max \bigl(
      \adeg^g \hautl1{\va}, \hlclab, \htcmp
    \bigr)
    + (g + 1) \deg \va
    \Bigl(
      \adeg^n \ln(\adeg) n
    \\ & \qquad
      + \adeg^g \bigl (
        \hautl1 \avar
        + (\adim + 2) \ln (\adeg + 1) ( \adeg + n + 1 )
        + \ln(g)
      \bigr)
    \Bigr)
    \pmm.
  \end{align}
\end{coro}

\begin{proof}
  C'est le corollaire précédent appliqué avec \( m = 2g \) et en opérant
  quelques simplifications d'écriture et majorations.
\end{proof}



\section{Grands points, cas général}
\label{sec:any}

Dans toute cette section, on suppose (sans nécessairement le rappeler à chaque
énoncé) que \( \va \) est plongée dans \( \projd \)
par un plongement de \bsc{Mumford} modifié associé à une polarisation
principale, ce qui est nécessaire pour appliquer le
théorème~\vref{t:mumford-gen}. Cette hypothèse implique en particulier que
\( \deg \va = N = n+1 = 16^g \).

Par ailleurs, comme on l'a vu dans la section précédente, on aura intérêt dans
l'inégalité de \bsc{Vojta} (corollaire~\vref{c:vojta-gen2}) à toujours prendre
l'option \( s = 1 \) et à poser \( m = 2g \).  Commençons par énoncer
l'inégalité de \bsc{Vojta} sous ces conditions.

\begin{coro} \label{c:vojta-gen3}
  Soit \( \avar \) une variété de \( \projd \) de degré \( \adeg \) et de
  dimension \( \adim \), et \( 0 < \eps < 1 \) un nombre réel.  Il n'existe
  dans \( \va(\Qbar) \) aucune famille de points \( x_1, \dots, x_{2g} \)
  satisfaisant simultanément aux conditions suivantes :
  \begin{align}
    0 < \distv{x_i} \avar
    & <
    \alpha_v^{-1}
    \hautm2{x_i}^{-\wtapx \eps}
    \quad \forall v \in \placesapx
    \\
    \hautn{x_1}
    & > \cst{ht-v-gen2} \Lambda_6^{(4g)^{4g^2}}
    \\
    \hautn{x_i} & > \hautn{x_{i-1}}
    \cdot \Lambda_6^{(4g)^{4g^2}}
    \\
    \cos(x_i, x_j) & > 1 -
    \frac{ \eps^2 }{ 2^{15g + 14} \adeg^2 }
  \end{align}
  avec
  \begin{align}
    \Lambda_6
    & =
    2^{35g^3} \, \adeg^{2g^2 + 2} \, \eps^{-2}
    \\
    \newcst[\vaemb]{ht-v-gen2}
    & =
    \adeg \max \bigl(
      \adeg^g \hautl1{\va}, \hlclab, \htcmp
    \bigr)
    + 32^g \Bigl(
      \adeg^{16^g + 1} 16^g
      + \adeg^g \bigl ( \hautl1 \avar + 32^g \adeg^2 \bigr)
    \Bigr)
  \end{align}
  et \( (\alpha_v)_v \) une certaine famille de réels tous supérieurs ou égaux
  à \( 1 \), avec \( \sum\placerange \degv\alpha_v \le \cst{ht-v-gen2}
  \).
\end{coro}

\begin{proof}
  C'est le corollaire~\vref{c:vojta-gen2} avec \( m = 2g \) et l'option \( s
    = 1 \), où l'on a simplifié la majoration de la « hauteur » de \( \alpha
  \) en tenant compte de~\eqref{e:maj-ht-alpha}. On a par ailleurs utilisé les
  majorations suivantes :
  \begin{gather}
    \frac1{2g}
    \left( \frac{ \eps }{ 86 \cdot 16^g \cdot 5^g \cdot d } \right)^2
    \le
    \frac{ \eps^2 }{ 2^g \bigl( 2^7 \cdot 2^{7g} d \bigr)^2 }
    \\
    86^2 \cdot 40^{2g} \adeg^2 \, \eps^{-2}
    (32\sqrt2 g^2 \adeg)^{2g^2}
    \le
    86^2 \, 40^{2g} \, 2^{11g^2} g^{4g^2}
    \cdot \adeg^{2g^2 + 2} \, \eps^{-2}
    \le
    2^{35g^3} \, \adeg^{2g^2 + 2} \, \eps^{-2}
  \end{gather}
  où la dernière majoration provient d'une vérification numérique pour \( g =
    1 \). De plus, on a remarqué que
  \begin{align}
    (\adim + 2) \ln (\adeg + 1) ( \adeg + 16^g )
    + \ln g
    \le
    (g + 1) \adeg ( \adeg + 16^g + 1)
    \le
    2^g \adeg^2 \cdot 16^g
  \end{align}
  pour simplifier l'expression de \( \cst{ht-v-gen2} \).
\end{proof}

Posons maintenant et pour le reste de cette section
\begin{align}
  M
  & =
  \bigl(
  2^{34} \cdot [\cdn : \Q] \hlclab \cdot \adeg
  \bigr)^{ (r+1) g^{ 5(\adim + 1)^2 } }
  + 1
  \\
  \phi
  & =
  \frac{ \eps }{ 4 \adeg^M (2M)^{(M+1)\adim} }
  \\
  \rho
  & =
  \frac{ \sqrt\eps }{ \adeg^{M/2} (2M)^{(M+1)\adim/2} }
\end{align}
de sorte que \( (\rho, \phi) \) satisfait bien la
condition~\eqref{e:rho-phi-gen} (on peut de nouveau supposer \( \eps < 2\adeg
\) en vertu du corollaire~\vref{c:nobig-liouville} vu la condition de hauteur
qu'on imposera bientôt). Par ailleurs, à un facteur \( 2 \) près, chacune de
ces deux quantités est aussi grande au possible sous cette condition.

De plus, on observe les relations suivantes entre \( M \) et les différentes
quantités apparaissant dans l'inégalité de \bsc{Vojta} ci-dessus :
\begin{equation}
  \eps^2 \Lambda_6
  \le
  M
  \quad\text{et}\quad
  (4g)^{4g^2}
  \le
  \sqrt M
  \le
  M
\end{equation}
puis, en posant
\begin{equation}
  \newcst[\vaemb, \avar]{ht-comb}
  =
  \hautl1 \avar + \max \bigl( \hautl1{\va}, \hlclab, \htcmp \bigr)
\end{equation}
on a aisément
\begin{equation}
  \cst{ht-v-gen2}
  \le
  \cst{ht-comb} \, \adeg^{M-1} (2M)^{ (M+1)\adim}
  \pmm.
\end{equation}

Nous sommes maintenant prêts à suivre le plan de la
section~\vref{sec:big-points} en commençant par majorer le nombre de points
dans chaque cône tronqué.

\begin{lem}
  Soit \( \avar \) une sous-variété de \( \va \) ; on note \( \adeg = \deg
    \avar \) et \( \adim = \dim\avar \). On fixe un sous-groupe \( \grp
    \subset \va(\Qbar) \) de rang fini \( r \), puis un réel \( \eps > 0 \).
  On considère une famille \( x_1, \dots, x_p \) de points de \( \va \)
  satisfaisant à la condition \( (*) \) ainsi qu'aux conditions suivantes :
  \begin{align}
    \label{e:big-any-ha}
    0 < \distv{x_i} \avar
    & <
    \alpha_v^{-1}
    \hautm2{x_i}^{-\wtapx \eps}
    \quad \forall v \in \placesapx
    \\ \label{e:big-any-big}
    \hautn{x_1}
    & > \mu
    \\ \label{e:big-any-cos}
    \cos(x_i, x_j)
    & > 1 - \eps \phi
  \end{align}
  avec \( (\alpha)_v \) comme ci-dessus et
  \begin{equation}
    \mu =
    \cst{ht-comb}
    \, \eps^{-(4g)^{4g^2}}
    \adeg^{M-1} (2M)^{ (M+1)\adim + 4}
    \pmm.
  \end{equation}
  Alors on a
  \begin{equation}
    p
    \le
    2 g (M-1) \frac{ (4g)^{4g^2} \ln \Lambda_6 }{ \rho }
  \end{equation}
  avec \( \Lambda_6 \) comme ci-dessus.
\end{lem}

\begin{proof}
  D'après les remarques et majorations précédentes, il est assez clair que les
  conditions du présent énoncé impliquent les conditions correspondantes du
  théorème~\vref{t:mumford-gen} et du corollaire~\vref{c:vojta-gen3}.  On peut
  donc appliquer simultanément ces deux résultats et en conclure comme dans la
  preuve du lemme~\vref{l:big-by-cone} que
  \begin{equation}
    p
    \le
    g (M-1) \left(
      \frac{ (4g)^{4g^2} \ln \Lambda_6 }{ \ln(1+\rho) } + 1
    \right)
    \le
    2 g (M-1) \frac{ (4g)^{4g^2} \ln \Lambda_6 }{ \rho }
  \end{equation}
  compte tenu du fait que cette fois-ci il peut y avoir jusqu'à \( M-1 \)
  points par lunule de largeur \( 1 + \rho \).
\end{proof}

On en déduit un décompte complet des grands points.

\begin{coro}
  Soit \( \avar \) une sous-variété de \( \va \) ; on note \( \adeg = \deg
    \avar \) et \( \adim = \dim\avar \). On fixe un sous-groupe \( \grp
    \subset \va(\Qbar) \) de rang fini \( r \), puis un réel \( 0 < \eps <
    \expb^{-1} \).
  On considère une famille \( x_1, \dots, x_p \) de points de \( \va \)
  satisfaisant à la condition \( (*) \) ainsi qu'aux conditions suivantes :
  \begin{align}
    0 < \distv{x_i} \avar
    & <
    \alpha_v^{-1}
    \hautm2{x_i}^{-\wtapx \eps}
    \quad \forall v \in \placesapx
    \\
    \hautn{x_1}
    & > \mu
  \end{align}
  avec \( (\alpha_v)_v \) et \( \mu \) comme ci-dessus. Alors on a
  \begin{equation}
    p
    \le
    M^2 \Bigl( \adeg^{M} (2M)^{(M+1)\adim} \Bigr)^{(r+1)/2}
    \, \eps^{-r - 1/2} \ln(1/\eps)
    \pmm.
  \end{equation}
\end{coro}

\begin{proof}
  On applique le fait~\vref{f:nb-cones} avec \( \gamma = \eps\phi \) pour
  recouvrir l'espace euclidien \( \grp \otimes_\Z \R \) muni du produit
  scalaire donné par la hauteur de \NT par des ensembles dans lesquels deux
  points quelconques satisfont~\eqref{e:big-cos}. Le nombre d'ensembles
  nécessaires est au plus
  \begin{equation}
    \left( 1 + \sqrt{ \frac{ 8 }{ \eps\phi } } \right)^r
    \le
    \left( 1 + \frac{ 4\sqrt 2 }{ \rho \sqrt\eps } \right)^r
    \le
    \left( \frac{ 6 }{ \rho \sqrt\eps } \right)^r
  \end{equation}
  en utilisant le fait que \( \phi = \rho^2 / 4 \) d'après leurs définitions.
  Le lemme précédent donne le nombre de points dans chaque ensemble, on a donc
  \begin{equation}
    p
    \le
    2 g (M-1) \frac{ (4g)^{4g^2} \ln \Lambda_6 \, 6^r }{ \rho^{r+1} }
    \, \eps^{-r/2}
    \pmm.
  \end{equation}
  On remarque maintenant que
  \begin{equation}
    \ln \Lambda_6
    \le
    \ln M + 2 \ln(1/\eps)
    \le
    \sqrt M \ln(1/\eps)
  \end{equation}
  dès que \( \eps \le \expb^{-1} \) (on rappelle que \( a + b \le ab \) dès
  que les deux sont supérieurs ou égaux ) \( 2 \)).  Par ailleurs, on remarque
  que
  \begin{equation}
    2 g (4g)^{4g^2}
    \le
    (4g)^{4g^2+1}
    \le
    2^{4(g+1)g^2 + 1}
    \le
    (2^9)^{g^3}
  \end{equation}
  de sorte que \( 2 g (4g)^{4g^2} 6^r \le \sqrt M \) soit au final
  \begin{equation}
    p
    \le
    \frac{ M^2 }{ \rho^{r+1} }
    \, \eps^{-r/2} \ln(1/\eps)
    \le
    M^2 \Bigl( \adeg^{M} (2M)^{(M+1)\adim} \Bigr)^{(r+1)/2}
    \, \eps^{-r - 1/2} \ln(1/\eps)
  \end{equation}
  compte tenu de la définition de \( \rho \).
\end{proof}

\begin{coro}
  Soit \( \avar \) une sous-variété de \( \va \) ; on note \( \adeg = \deg
    \avar \) et \( \adim = \dim\avar \). On fixe un sous-groupe \( \grp
    \subset \va(\Qbar) \) de rang fini \( r \), puis un réel \( 0 < \eps <
    \expb^{-1} \).
  On considère une famille \( x_1, \dots, x_p \) de points de \( \va \)
  satisfaisant à la condition \( (*) \) ainsi qu'aux conditions suivantes :
  \begin{align}
    0
    & < \prod\placerange \distv{x_i} \avar^\degv
    < \hautm2{x_i}^{-\eps}
    \\
    \hautn{x_1}
    & >
    \cst{ht-comb}
    \, \eps^{-(4g)^{4g^2}}
    \adeg^{M-1} (3M)^{ (M+1)\adim + 4}
  \end{align}
  Alors on a
  \begin{equation}
    p
    \le
    5^{\card\placesapx}
    M^2 \Bigl( \adeg^{M} (3M)^{(M+1)\adim} \Bigr)^{(r+1)/2}
    \, \eps^{-r - 1/2} \ln(1/\eps)
  \end{equation}
  avec
  \begin{align}
    \cst{ht-comb}
    & =
    \hautl1 \avar + \max \bigl( \hautl1{\va}, \hlclab, \htcmp \bigr)
    \\
    M
    & =
    \bigl(
    2^{34} \cdot [\cdn : \Q] \hlclab \cdot \adeg
    \bigr)^{ (r+1) g^{ 5(\adim + 1)^2 } }
    + 1
  \end{align}
\end{coro}

\begin{proof}
  Ce résultat se déduit du corollaire précédent de la même façon que le
  corollaire~\vref{c:big-gen-prod} se déduit du corollaire~\vref{c:big-gen} :
  on commence par appliquer le scolie~\vref{s:ha-prod}, ce qui a pour coût,
  dans le décompte, de remplacer \( \eps \) par \( \eps/2 \) et d'ajouter un
  facteur \( 5^{\card\placesapx} \). Ensuite, pour absorber le \( \alpha \)
  restant dans la condition d'approximation, il suffit de diviser encore \(
    \eps \) par \( 2 \), cette fois non seulement dans le décompte mais aussi
  dans la condition de hauteur.

  On constate ensuite que les constantes apparues près des \( \eps \) sont
  facilement absorbées dans les autres facteurs en remplaçant \( 2M \) par \(
    3M \) comme ci-dessus.
\end{proof}



\section{Options pour les petits points}
\label{sec:small-points}

Il ne semble pas évident d'exploiter l'hypothèse d'approximation dans le
décompte des petits points. Nous rappelons donc pour mémoire comment
compter les points de petite hauteur d'une variété abélienne appartenant à un
groupe de type fini donné : ce décompte repose sur une propriété élémentaire
en géométrie euclidienne, que l'on commence par rappeler.

\begin{fact}
  Soient \( E \) un espace euclidien de dimension \( r \) et deux réels \(
    \rho \) et \( \mu \). On peut recouvrir toute boule (fermée) de rayon \(
    \rho \) par des boules (ouvertes) de rayon \( \mu \) en nombre inférieur à
  \( ( 2 \, \frac\rho\mu + 1 )^r \).
\end{fact}

\begin{proof}
  C'est le lemme~6.1, p.~541 de \cite{remdcl}, où l'on a par ailleurs effectué
  le changement de notations \( \mu = \rho / \gamma \).  L'énonce donné dans
  la référence citée ne précise pas si les boules sont ouvertes ou fermées,
  mais on constate facilement que la preuve qui y est proposée fonctionne
  parfaitement pour la variante la plus forte, énoncée ci-dessus.
\end{proof}

On en déduit la majoration suivante du nombre de points de petite hauteur sans
aucune hypothèse d'approximation.

\begin{coro} \label{c:small-va}
  Soit \( \grp \) un sous-groupe de type fini de \( \va(\Qbar) \) ; on note
  \( r \) le rang de \( \grp \) et \( \hmin(\grp) \) le minimum de \(
    \hautn x \) quand \( x \) parcourt l'ensemble des points d'ordre infini de
  \( \grp \).  Pour tout réel positif \( R \), on a
  \begin{equation}
    \card \set*{
      x \in \grp
      \text{ tel que }
      \hautn x \le R
    }
    \le
    \card \grp_\torsion
    \cdot
    \left( 1 + 2\sqrt{R / \hmin(\grp)} \right)^r
  \end{equation}
  où \( \grp_\torsion \) désigne l'ensemble des points de torsion de \( \grp
  \).
\end{coro}

\begin{proof}
  Dans \( \grp \otimes_\Z \R \) muni de la structure euclidienne induite par
  la hauteur normalisée, on applique le fait précédent avec \( \rho = \sqrt R
  \) et \( \mu = \sqrt{\hmin(\grp)} \) puis on remarque que la préimage dans \(
    \grp \) de chacune des boules ouvertes de rayon \( \mu \) est composée
  de points qui sont tous égaux modulo \( \grp_\torsion \).
\end{proof}

\begin{rem}
  Si dans l'énoncé précédent on prend \( \grp = \va(\cdn) \), on retrouve le
  lemme~2.11.1, p.~117 de~\cite{farhith}.
\end{rem}

Si l'on applique le résultat précédent en conjonction avec le
corollaire~\vref{c:big-gen-prod}, on constate assez facilement que le décompte
obtenu pour les petits points est largement supérieur à celui obtenu pour les
grands poins (à moins que \( \hmin(\grp) \) ne soit particulièrement grand),
ce qui n'est pas très satisfaisant. Nous établissons donc un résultat, basé
sur l'inégalité de \bsc{Liouville}, permettant de supprimer les petits points
quitte à renforcer l'hypothèse d'approximation.

\begin{lem}
  Soit \( \avar \) une sous-variété de \( \va \), de dimension \( \adim \) et
  de degré
  \( \adeg \). Si \( R \) et \( F \) sont deux réels positifs, l'ensemble des
  points \( x \in \va(\Qbar) \) tels que
  \begin{equation}
    0
    <
    \prod\placerange
    \distv x \avar ^\degv
    \le
    \expb^{-F}
    \hautm2{x}^{-\eps}
    \quad\text{et}\quad
    \hautn x \le R
  \end{equation}
  est vide dès que
  \( F
    >
    (\adeg-\eps) R
    + \hautl1{ \chow \avar }
    + \adeg (\adim+1) \ln(3\adeg)
    + \frac32 \ln(n+1)
    + (\adeg-\eps) \htcmp
  \).
\end{lem}

\begin{proof}
  Si l'ensemble en question n'est pas vide, on choisit un point \( x \) dedans
  et on lui applique l'inégalité de \bsc{Liouville}
  (proposition~\vref{p:liouville}) :
  \begin{equation}
    \frac1{
      (n+1)^{3/2}
      (3\adeg)^{\adeg (\adim+1)}
      \, \hautm1{ \chow \avar }
      \, \hautm2 x ^\adeg
    }
    \le
    \prod\placerange
    \distv x \avar ^\degv
    \le
    \expb^{-F}
    \hautm2{x}^{-\eps}
  \end{equation}
  puis en prenant les opposés des logarithmes
  \begin{align}
    F
    & \le
    (\adeg-\eps) \hautl2 x
    + \hautl1{ \chow \avar }
    + \adeg (\adim+1) \ln(3\adeg)
    + \frac32 \ln(n+1)
    \\ & \le
    (\adeg-\eps) R
    + \hautl1{ \chow \avar }
    + \adeg (\adim+1) \ln(3\adeg)
    + \frac32 \ln(n+1)
    + (\adeg-\eps) \htcmp
    \pmm.
  \end{align}
  Par contraposée, l'ensemble considéré est vide si cette inégalité est
  fausse.
\end{proof}

On en déduit immédiatement le corollaire pratique suivant.

\begin{coro} \label{c:kill-small}
  Soit \( \avar \) une sous-variété de \( \va \), de dimension \( \adim \) et
  de degré \( \adeg \). Si \( R \) est un réel supérieur ou égal à
  \begin{equation}
    \frac1\eps \left(
      \hautl1{ \chow \avar }
      + \adeg (\adim+1) \ln(3\adeg)
      + \frac32 \ln(n+1)
      + \adeg \htcmp
    \right)
    \pmm,
  \end{equation}
  il n'existe aucun point \( x \in \va(\Qbar) \) tel que
  \begin{equation}
    0
    <
    \prod\placerange
    \distv x \avar ^\degv
    \le
    \expb^{- \adeg R}
    \hautm2{x}^{-\eps}
    \quad\text{et}\quad
    \hautn x \le R
    \pmm.
  \end{equation}
\end{coro}

\begin{proof}
  Application immédiate du lemme précédent avec \( F = \adeg R \).
\end{proof}

On en déduit à titre d'exemple une variante du
corollaire~\vref{c:big-gen-prod} ; le lecteur en déduira sans peine des
variantes similaires des autres résultats de décompte.

\begin{coro} \label{c:all-gen}
  Soient \( \avar \) un translaté par un point algébrique d'une sous-variété
  abélienne \( \vai \) de \( \va \) de degré \( \adeg \) et de dimension \(
    \adim \).  On considère de plus un sous-groupe \( \grp \subset \va(\Qbar)
  \) de rang fini \( r \).  Pour tout réel \( \eps > 0 \) et tout entier \( m
    \ge g + 1 \), il existe au plus
  \begin{equation}
    2 \cdot 5^{\card \placesapx} \cdot
    \sqrt{ \frac{\adeg}\eps }
    (2mg)^{mg+1}
    \ln \Lambda_5
    \left(
      3 \sqrt m
      \bigl(
        174 \nclmaps \cdot 5^g \adeg s
        \, \eps^{-1}
        \bigr)^{ \frac{m}{2(m-g)} }
    \right)^r
  \end{equation}
  points de \( \grp \) distincts modulo \( \vai \) et satisfaisant
  à la condition suivante :
  \begin{align}
    0 < \prod\placerange \distv{x_i} \avar ^\degv
    & <
    \exp\left(
      - \adeg \cst{ht-v-gen1} \Lambda_5^{(2mg)^{mg}}
    \right)
    \hautm2{x_i}^{-\eps}
  \end{align}
  avec \( \Lambda_5 \) et \( \cst{ht-v-gen1} \) comme au
  corollaire~\vref{c:big-gen-prod}.
\end{coro}

\begin{proof}
  La condition d'approximation considérée permet d'appliquer le
  corollaire~\vref{c:kill-small} avec
  \( R = \cst{ht-v-gen1} \Lambda_5^{(2mg)^{mg}} \),
  quantité qui satisfait amplement l'hypothèse du corollaire. Ainsi, on voit
  que tous les points satisfaisant cette condition d'approximation satisfont
  aussi \( \hautn{x_i} \ge R \) et on peut appliquer le
  corollaire~\vref{c:big-gen-prod} pour conclure.
\end{proof}


\cleardoublepage
\endinput

% vim: spell spelllang=fr
