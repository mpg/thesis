% !TEX root = main.tex

\chapter{Déduction des résultats principaux}
\label{chap:union}

Nous déduisons ici des deux chapitres précédents des décomptes
d'approximations exceptionnelles, en procédant en deux temps : pour les points
de hauteur assez grande (au sens de~\eqref{e:Vbig} et/ou~\eqref{e:Mbig}), que
nous appellerons grands points, nous combinons les inégalités de \bsc{Vojta}
et de \bsc{Mumford} obtenues précédemment ; pour les petits points, on procède
à des décomptes triviaux soit en oubliant l'hypothèse d'approximation, soit en
renforçant celle-ci pour interdire leur existence.

\section{Décomptes des grands points}
\label{sec:big-points}

On commence par compter le nombre de points dans chaque cône tronqué.

\begin{lem} \label{l:big-by-cone}
  Soient \( V \) un translaté par un point algébrique d'une sous-variété
  abélienne \( \vai \) de \( \va \) de degré \( d \) et de dimension \( u \).
  Pour tout réel \( \expapx > \) et tout entier \( \puiss \ge \genre + 1 \),
  il existe au plus
  \begin{equation}
    \todots
  \end{equation}
  points distinct modulo \( \vai \) et satisfaisant simultanément aux
  conditions suivantes :
  \begin{align}
    0 < \distv{\ex*} V
    & <
    \bigl( \Delta \sbin{\Delta+\dimp}{\dimp}^3 \bigr)^{-\dv/2}
    \hautm[2]{\ex*}^{-\wtapx \expapx}
    \quad \forall \place \in \placesapx
    \\
    \hautn{\ex[1]}
    & > \cst{ht-v-gen1} \Lambda_4^{(2\puiss\genre)^{\puiss\genre}}
    \\
    \cos(\ex*, \ex[\fcti])
    & > 1 - \frac1{ \puiss \, \cst{v-gen} }
  \end{align}
  avec
  \begin{align}
    \cst{v-gen}
    & =
    \bigl(
        86 \nclmaps \cdot 5^\genre \Delta \sbin{\dimp + \Delta}{\Delta}
        \, \expapx^{-1}
    \bigr)^{ \frac{\puiss}{\puiss-\genre} }
    \\
    \Lambda_4
    & =
    \cst{v-gen}
    \bigl( (\sqrt2 \puiss \genre\Delta)^\genre \deg \va \bigr)^\puiss
    \\
    \cst{ht-v-gen1}
    & =
    \Delta \max \bigl(
      \Delta^\genre \hautl[1]{\va}, \hlclab, \htcmp
    \bigr)
    + (\genre + 1) \deg \va
    \Bigl(
      \Delta^\dimp \ln(\Delta) \dimp
    \\ & \qquad
      + \Delta^\genre \bigl (
        \hautl[1] V
        + (u + 2) \ln (\Delta + 1) ( \Delta + \dimp + 1 )
        + \ln(\puiss/2)
      \bigr)
    \Bigr)
    \pmm.
  \end{align}
\end{lem}

\begin{proof}
  ~\todo
\end{proof}

Le fait suivant permet de recouvrir l'espace de \MoW par de tels cônes.

\begin{fact}
  Soient \( r \) un entier et \( \gamma > 0 \) un réel. On peut recouvrir \(
    \R^r \) par \( \floor{(1 + \sqrt{8/\gamma})^r} \) ensembles dans chacun
  desquels deux points quelconques satisfont \( \cos(x, y) \ge 1 - \gamma \).
\end{fact}

\begin{proof}
  C'est le corollaire~6.1, p.~542 de~\cite{remdcl}.
\end{proof}

On déduit immédiatement de ce fait un décompte des grands points.

\begin{prop} \label{p:big-gen}
  Soient \( V \) un translaté par un point algébrique d'une sous-variété
  abélienne \( \vai \) de \( \va \) de degré \( d \) et de dimension \( u \).
  On considère de plus un sous-groupe \( \Gamma \subset \va(\Qbar) \) de rang
  fini \( r \).
  Pour tout réel \( \expapx > \) et tout entier \( \puiss \ge \genre + 1 \),
  il existe au plus
  \begin{equation}
    \todots
  \end{equation}
  points \( \Gamma \) distincts modulo \( \vai \) et satisfaisant
  simultanément aux deux conditions suivantes :
  \begin{align}
    0 < \distv{\ex*} V
    & <
    \bigl( \Delta \sbin{\Delta+\dimp}{\dimp}^3 \bigr)^{-\dv/2}
    \hautm[2]{\ex*}^{-\wtapx \expapx}
    \quad \forall \place \in \placesapx
    \\
    \hautn{\ex[1]}
    & > \cst{ht-v-gen1} \Lambda_4^{(2\puiss\genre)^{\puiss\genre}}
  \end{align}
  avec \( \cst{v-gen} \), \( \Lambda_4 \) et \( \cst{ht-v-gen1} \) comme au
  lemme~\ref{l:big-by-cone}.
\end{prop}



\section{Décomptes des petits points}
\label{sec:small-points}



\section{Décomptes complets}
\label{sec:all-points}



\section{Hypothèse d'approximation produit}
\label{sec:ha-prod}


\cleardoublepage
\endinput

% vim: spell spelllang=fr
