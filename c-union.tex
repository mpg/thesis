% !TEX root = main.tex

\chapter{Déduction des résultats principaux}
\label{chap:union}

Nous déduisons ici des deux chapitres précédents des décomptes
d'approximations exceptionnelles, en procédant en deux temps : pour les points
de hauteur assez grande (au sens de~\eqref{e:Vbig} et/ou~\eqref{e:Mbig}), que
nous appellerons grands points, nous combinons les inégalités de \bsc{Vojta}
et de \bsc{Mumford} obtenues précédemment ; pour les petits points, on procède
à des décomptes triviaux soit en oubliant l'hypothèse d'approximation, soit en
renforçant celle-ci pour interdire leur existence.

\section{Décomptes des grands points}
\label{sec:big-points}

On commence par compter le nombre de points dans chaque cône tronqué.

\begin{lem} \label{l:big-by-cone}
  Soient \( V \) un translaté par un point algébrique d'une sous-variété
  abélienne \( \vai \) de \( \va \) de degré \( d \) et de dimension \( u \).
  On fixe réel \( \expapx > 0 \), un entier \( \puiss \ge \genre + 1 \) et on
  considère une famille de points \( \ex[1], \dots, \ex[p] \in \va(\Qbar)
  \) distincts modulo \( \vai \) et satisfaisant simultanément aux conditions
  suivantes :
  \begin{align}
    \label{e:big-ha}
    0 < \distv{\ex*} V
    & <
    \bigl( \Delta \sbin{\Delta+\dimp}{\dimp}^3 \bigr)^{-\dv/2}
    \hautm[2]{\ex*}^{-\wtapx \expapx}
    \quad \forall \place \in \placesapx
    \\ \label{e:big-big}
    \hautn{\ex[1]}
    & > \cst{ht-v-gen1} \Lambda_4^{(2\puiss\genre)^{\puiss\genre}}
    \\ \label{e:big-cos}
    \cos(\ex*, \ex[\fcti])
    & > 1 - \frac1{ \puiss \, \cst{v-gen} }
  \end{align}
  avec
  \begin{align}
    \cst{v-gen}
    & =
    \bigl(
        86 \nclmaps \cdot 5^\genre d \sbin{\dimp + d}{d}
        \, \expapx^{-1}
    \bigr)^{ \frac{\puiss}{\puiss-\genre} }
    \\
    \Lambda_4
    & =
    \cst{v-gen}
    \bigl( (\sqrt2 \puiss \genre d)^\genre \deg \va \bigr)^\puiss
    \\
    \cst{ht-v-gen1}
    & =
    d \max \bigl(
      d^\genre \hautl[1]{\va}, \hlclab, \htcmp
    \bigr)
    + (\genre + 1) \deg \va
    \Bigl(
      d^\dimp \ln(d) \dimp
    \\ & \qquad
      + d^\genre \bigl (
        \hautl[1] V
        + (u + 2) \ln (d + 1) ( d + \dimp + 1 )
        + \ln(\puiss/2)
      \bigr)
    \Bigr)
    \pmm.
  \end{align}
  On a alors nécessairement
  \begin{equation}
    p
    \le
    (\puiss - 1) (2\puiss\genre)^{(\puiss\genre)}
    \frac{ \ln \Lambda_4 }{ \ln (1 + \sqrt{\expapx / d}) }
    \pmm.
  \end{equation}
\end{lem}

\begin{proof}
  Soit \( x_1, \dots, x_p \) une famille comme dans l'énoncé, qu'on suppose de
  plus ordonnée par hauteur normalisée croissante. On pose
  \( \phi =( \puiss \, \cst{v-gen})^{-1} \) et
  \( \rho = \sqrt{ \expapx / d } \).  On constate facilement que
  \( \phi \le \expapx / (2580 d) \) de sorte que ce couple \( (\rho, \phi) \)
  satisfait à~\eqref{e:rho-phi}. Par ailleurs, il est clair que la
  condition~\eqref{e:big-big} est plus forte que~\eqref{e:Mbig}, car \(
    \cst{v-gen} \ge 2/\expapx \) et la parenthèse dans le membre de droite
  de~\eqref{e:Mbig} est largement majorée par \( 3 \cst{ht-v-gen1}
  \).\worknote{sauf peut-être \( h(\vai) \) vs \( h(V) \)} On peut donc
  appliquer le théorème~\ref{t:mumford} et conclure que
  \( \hautn{ x_{i+1} } > (1 + \sqrt{\expapx / d}) \hautn{ x_i } \) pour tout
  \( i \le p-1 \).

  On pose alors
  \begin{equation}
    \eta
    =
    \left\lceil
      (2\puiss\genre)^{(\puiss\genre)}
      \frac{ \ln \Lambda_4 }{ \ln (1 + \sqrt{\expapx / d}) }
    \right\rceil
  \end{equation}
  de sorte que, pour tout \( i \le p - \eta \), on a
  \begin{equation}
    \hautn{ x_{i+\eta} }
    >
    (1 + \sqrt{\expapx / d})^\eta
    \hautn{ x_i }
    \ge
    \Lambda_4^{(2\puiss\genre)^{\puiss\genre}}
    \hautn{ x_i }
    \pmm.
  \end{equation}
  Supposons alors que \( p \ge (\puiss - 1) \eta + 1 \) contrairement à la
  conclusion du lemme : on pourrait alors poser \( \ex* = x_{1 + (\fct-1)
      \eta} \) pour \( i \in {1, \puiss} \) et la famille \( (\ex*)_\fct \)
  contredirait alors le corollaire~\ref{c:vojta-gen2}, ce qui est absurde.
\end{proof}

Le fait suivant permet de recouvrir l'espace de \MoW par de tels cônes.

\begin{fact}
  Soient \( r \) un entier et \( \gamma > 0 \) un réel. On peut recouvrir \(
    \R^r \) par \( \floor{(1 + \sqrt{8/\gamma})^r} \) ensembles dans chacun
  desquels deux points quelconques satisfont \( \cos(x, y) \ge 1 - \gamma \).
\end{fact}

\begin{proof}
  C'est le corollaire~6.1, p.~542 de~\cite{remdcl}.
\end{proof}

On déduit immédiatement de ce fait un décompte des grands points.

\begin{prop} \label{p:big-gen}
  Soient \( V \) un translaté par un point algébrique d'une sous-variété
  abélienne \( \vai \) de \( \va \) de degré \( d \) et de dimension \( u \).
  On considère de plus un sous-groupe \( \Gamma \subset \va(\Qbar) \) de rang
  fini \( r \).
  Pour tout réel \( \expapx > \) et tout entier \( \puiss \ge \genre + 1 \),
  il existe au plus
  \begin{equation}
    \todots
  \end{equation}
  points \( \Gamma \) distincts modulo \( \vai \) et satisfaisant
  simultanément aux deux conditions suivantes :
  \begin{align}
    0 < \distv{\ex*} V
    & <
    \bigl( \Delta \sbin{\Delta+\dimp}{\dimp}^3 \bigr)^{-\dv/2}
    \hautm[2]{\ex*}^{-\wtapx \expapx}
    \quad \forall \place \in \placesapx
    \\
    \hautn{\ex[1]}
    & > \cst{ht-v-gen1} \Lambda_4^{(2\puiss\genre)^{\puiss\genre}}
  \end{align}
  avec \( \cst{v-gen} \), \( \Lambda_4 \) et \( \cst{ht-v-gen1} \) comme au
  lemme~\ref{l:big-by-cone}.
\end{prop}



\section{Décomptes des petits points}
\label{sec:small-points}



\section{Décomptes complets}
\label{sec:all-points}



\section{Hypothèse d'approximation produit}
\label{sec:ha-prod}


\cleardoublepage
\endinput

% vim: spell spelllang=fr
