\chapter{Déduction des résultats principaux}
\label{chap:union}

Nous déduisons ici des deux chapitres précédents des décomptes
d'approximations exceptionnelles. Plus précisément, on donne un décompte des
points de hauteur assez grande (au sens de~\eqref{e:Vbig} et~\eqref{e:Mbig}),
que nous appellerons grands points, en combinant les inégalités de \bsc{Vojta}
et de \bsc{Mumford} obtenues précédemment.

On en déduit ensuite un décompte des grands points avec une hypothèse
d'approximation sous la forme d'un produit sur les places, plus naturelle que
l'hypothèse technique avec un système d'inégalités. Cette forme constitue le
résultat principal de la thèse.

On discute alors des choix possibles pour la valeur d'un paramètre laissé
libre jusque là, afin d'optimiser l'hypothèse et/ou le décompte obtenu.
Enfin, concernant les petits points, après avoir rappelé une méthode de
décompte grossière ne tenant pas compte de l'hypothèse d'approximation, on
présente une méthode pour les éliminer en renforçant ladite hypothèse.

\section{Obstruction au décompte absolu}
\label{sec:obstruction}

\todo (À écrire.)


\section{Décomptes des grands points}
\label{sec:big-points}

On commence par majorer le nombre de points dans chaque cône tronqué.

\begin{lem} \label{l:big-by-cone}
  Soient \( \avar \) un translaté par un point algébrique d'une sous-variété
  abélienne \( \vai \) de \( \va \) de degré \( \adeg \) et de dimension \(
    \adim \).  On fixe un réel \( 0 < \eps \), un entier \( m \ge g + 1 \) et
  on considère une famille de points \( x_1, \dots, x_p \in \va(\Qbar)
  \) deux à deux distincts modulo \( \vai \) et satisfaisant simultanément aux
  conditions suivantes :
  \begin{align}
    \label{e:big-ha}
    0 < \distv{x_i} \avar
    & <
    \alpha_v^{-1}
    \hautm2{x_i}^{-\wtapx \eps}
    \quad \forall v \in \placesapx
    \\ \label{e:big-big}
    \hautn{x_1}
    & > \cst{ht-v-gen1} \Lambda_4^{(2mg)^{mg}}
    \\ \label{e:big-cos}
    \cos(x_i, x_j)
    & > 1 - \frac1{ m \, \cst{v-gen} }
  \end{align}
  avec
  \begin{align}
    \cst{v-gen}
    & =
    \bigl(
      86 \nclmaps \cdot 5^g \adeg s \, \eps^{-1}
    \bigr)^{ \frac{m}{m-g} }
    \\
    \Lambda_4
    & =
    \cst{v-gen}
    \bigl( (\sqrt2 m g \adeg)^g \deg \va \bigr)^m
    \\
    \cst{ht-v-gen1}
    & =
    \adeg \max \bigl( \adeg^g \hautl1{\va}, \hlclab, \htcmp \bigr)
    + (g + 1) \deg \va
    \Bigl(
      \adeg^n \ln(\adeg) n
    \\ & \qquad
      + \adeg^g \bigl (
        \hautl1 \avar + (\adim + 2) \ln(\adeg + 1) (\adeg + n+1) + \ln(m/2)
      \bigr)
    \Bigr)
  \end{align}
  et au choix :
  \begin{enumthm}
  \item \( s = \min(\sbin{\adeg+n}{n}, \card \placesapx) \) et \( \alpha_v =
      \bigl( \adeg \sbin{\adeg+n}{n}^3 \bigr)^{\dv/2} \) ; ou bien
  \item \( s = 1 \) et \( (\alpha_v)_v \) une certaine famille de réels tous
    supérieurs ou égaux à \( 1 \), satisfaisant \(
      \prod\placerange \alpha_v
      \le
      \hautm1 \avar ^2
      \cdot (\adeg+1)^{ 2 (\adim+2) (\adeg + n+1) } \cdot \frac{m^2}4
      \cdot \expb^{\adeg n} \adeg \sbin{\adeg+n}{n}^{1/2}
    \).
  \end{enumthm}
  On a alors nécessairement
  \begin{equation}
    p
    \le
    \sqrt{\frac \adeg \eps}
    (2mg)^{mg+1}
    \ln \Lambda_4
    \pmm.
  \end{equation}
\end{lem}

\begin{proof}
  Pour commencer, remarquons qu'on peut supposer \( \eps < \adeg + 1 \). En
  effet, sinon le corollaire~\vref{c:nobig-liouville} s'applique,
  car~\eqref{e:big-ha} implique immédiatement~\eqref{e:nobig-ha}
  et~\eqref{e:big-big} implique assez largement~\eqref{e:nobig-big} vu la
  définition de \( \cst{ht-v-gen1} \).

  Soit \( x_1, \dots, x_p \) une famille comme dans l'énoncé, qu'on suppose de
  plus ordonnée par hauteur normalisée croissante. On pose
  \( \phi =( m \, \cst{v-gen})^{-1} \) et
  \( \rho = \sqrt{ \eps / \adeg } \).  On constate facilement que
  \( \phi \le \eps / (2580 \adeg) \) de sorte que ce couple \( (\rho, \phi) \)
  satisfait à~\eqref{e:rho-phi-grp}. Par ailleurs, il est assez facile de voir
  que la condition~\eqref{e:big-big} est plus forte que~\eqref{e:Mbig} : en
  effet \( \cst{v-gen} \ge 2/\eps \cdot \adeg (\adim+1) \) et la parenthèse
  dans le membre de droite de~\eqref{e:Mbig} est largement majorée par \( 5
    \cst{ht-v-gen1} \).

  On peut donc appliquer le théorème~\vref{t:mumford-grp} avec \( x = x_i \)
  et \( y = x_{i+1} \) et conclure que
  \( \hautn{ x_{i+1} } > (1 + \sqrt{\eps / \adeg}) \hautn{ x_i } \) pour tout
  \( i \le p-1 \). On pose alors
  \begin{equation}
    \eta
    =
    \left\lceil
      (2mg)^{mg}
      \frac{ \ln \Lambda_4 }{ \ln (1 + \sqrt{\eps / \adeg}) }
    \right\rceil
    \pmm.
  \end{equation}
  On suppose que \( p \ge (m-1) \eta + 1 \) : on peut alors poser
  \( y_i = x_{1 + (i-1) \eta} \) pour tout \( i \in \set{1, \dots, m} \) de
  sorte que, pour tout \( i \le m-1 \), on a
  \begin{equation}
    \hautn{ y_{i+1} }
    >
    (1 + \sqrt{\eps / \adeg})^\eta
    \hautn{ y_i }
    \ge
    \Lambda_4^{(2mg)^{mg}}
    \hautn{ y_i }
  \end{equation}
  et la famille \( (y_i)_i \) contredit le corollaire~\vref{c:vojta-gen2}, ce
  qui est absurde.

  Ainsi, on a \( p \le (m-1) \eta \). On remarque que la fonction \( t
    \mapsto t / \ln(1+t) \) est croissante pour \( t > 0 \) et, comme on a
  supposé \( \eps \le d+1 \le 2d \), on a
  \begin{equation}
    \frac1{ \ln(1 + \sqrt{\eps/d}) }
    \le
    \frac{ \sqrt2 }{ \ln(1 + \sqrt2) }
    \sqrt{ \frac d \eps }
    \le
    2 \sqrt{ \frac d \eps }
  \end{equation}
  en vérifiant numériquement pour la dernière inégalité. On a alors
  \begin{equation}
    p
    \le
    (m - 1)
    \left(
      (2mg)^{mg} \ln \Lambda_4
      \cdot 2 \sqrt{ \frac d \eps }
      + 1
    \right)
    \le
    2 m
    (2mg)^{mg} \ln \Lambda_4
    \sqrt{ \frac d \eps }
  \end{equation}
  qui donne directement le résultat annoncé.
\end{proof}

On utilise ensuite le fait suivant.

\begin{fact}
  Soient \( r \) un entier et \( \gamma > 0 \) un réel. On peut recouvrir \(
    \R^r \) par \( \floor{(1 + \sqrt{8/\gamma})^r} \) ensembles dans chacun
  desquels deux points quelconques satisfont \( \cos(x, y) \ge 1 - \gamma \).
\end{fact}

\begin{proof}
  C'est le corollaire~6.1, p.~542 de~\cite{remdcl}.
\end{proof}

\begin{prop} \label{p:big-gen}
  Soient \( \avar \) un translaté par un point algébrique d'une sous-variété
  abélienne \( \vai \) de \( \va \) de degré \( \adeg \) et de dimension \(
    \adim \).  On considère de plus un sous-groupe \( \grp \subset \va(\Qbar)
  \) de rang fini \( r \).  Pour tout réel \( \eps > 0 \) et tout entier \( m
    \ge g + 1 \), il existe au plus
  \begin{equation}
    \sqrt{\frac \adeg \eps}
    (2mg)^{mg+1}
    \ln \Lambda_4
    \left(
      3 \sqrt m
      \bigl(
        86 \nclmaps \cdot 5^g \adeg s \, \eps^{-1}
        \bigr)^{ \frac{m}{2(m-g)} }
    \right)^r
  \end{equation}
  points de \( \grp \) distincts modulo \( \vai \) et satisfaisant
  aux deux conditions suivantes :
  \begin{align}
    0 < \distv{x_i} \avar
    & <
    \alpha_v^{-1}
    \hautm2{x_i}^{-\wtapx \eps}
    \quad \forall v \in \placesapx
    \\
    \hautn{x_1}
    & > \cst{ht-v-gen1} \Lambda_4^{(2mg)^{mg}}
  \end{align}
  avec \( \Lambda_4 \), \( \cst{ht-v-gen1} \), \( s \)  et \( \alpha_v \)
  comme au lemme~\vref{l:big-by-cone}.
\end{prop}

\begin{proof}
  En utilisant le fait précédent avec \( \gamma = m \cst{v-gen} \) , on
  recouvre l'espace euclidien \( \grp \otimes_\Z \R \) muni du produit
  scalaire donné par la hauteur de \NT par des ensembles dans lesquels deux
  points quelconques satisfont~\eqref{e:big-cos}. Le nombre d'ensembles
  nécessaires est au plus
  \begin{equation}
    \left(
      1 + \sqrt{8 m}
      \bigl(
        86 \nclmaps \cdot 5^g \adeg s \, \eps^{-1}
        \bigr)^{ \frac{m}{2(m-g)} }
    \right)^r
    \le
    \left(
      3 \sqrt m
      \bigl(
        86 \nclmaps \cdot 5^g \adeg s \, \eps^{-1}
        \bigr)^{ \frac{m}{2(m-g)} }
    \right)^r
  \end{equation}
  en remarquant que \( 1 + \sqrt{ 8 \cdot 86 \cdot 3 \cdot 5 } \le 3 \sqrt{ 86
      \cdot 3 \cdot 5 } \).

  Il ne reste plus qu'à appliquer lemme~\vref{l:big-by-cone} à chacun de ces
  ensembles pour conclure.
\end{proof}



\section{Hypothèse d'approximation produit}
\label{sec:ha-prod}

Comme annoncé dans l'introduction (page~\pageref{e:ha-prod}), on déduit ici de
l'énoncé précédent une version où l'hypothèse d'approximation fait apparaître
un produit sur l'ensemble des places plutôt qu'une inégalité par place.

Pour cela, on utilise le lemme suivant, qui n'est rien d'autre qu'une version
plus abstraite du lemme~2.12.1, p. 120 de~\cite{farhith}. Par ailleurs, par
rapport à cette référence, on a fixé \( \eps'(T) = \eps/2 \), d'où \( A(T) =
  \card T \), car c'est la valeur qui a le plus de sens pour les applications,
afin de simplifier légèrement l'énoncé et sa preuve. L'énoncé ci-dessous est
volontairement aussi abstrait que possible afin de mettre en lumière le
caractère purement combinatoire de la preuve et d'être utilisable dans le plus
grand nombre de situations ; immédiatement après l'énoncé on détaille la
correspondance avec la situation étudiée ici afin d'en éclaircir le sens.

\begin{lem}
  Soit \( A \) un ensemble et \( H \) une fonction de \( A \) dans \( [1,
    +\infty \mathclose[ \). Soit par ailleurs \( M \) un autre ensemble et,
  pour chaque élément \( v \) de \( M \) :
  \begin{itemize}
    \item une fonction \( d_v \colon A \to [0, 1] \) ;
    \item deux réels \( 0 < \degv < 1 \) et \( c_v > 0 \).
  \end{itemize}
  Pour tout réel \( \eps > 0 \) et tout partie finie \( S \) de \( M \), on
  définit
  \[
    E(\eps, S)
    =
    \set*{
      x \in A
      \text{ tel que }
      \prod_{v \in S} d_v(x)^\degv
      \le
      \bigl( \prod_{v \in S} c_v^\degv \bigr)
      H(x)^{-\eps}
    }
    \pmm.
  \]
  De plus, pour toute famille de réels \( (\lambda_v)_{v\in S} \), on définit
  \[
    E'(\eps, S, \lambda)
    =
    \set*{
      x \in A
      \text{ tel que }
      d_v(x)
      \le
      c_v H(x)^{-\lambda_v \eps}
      \quad \forall v \in S
    }
    \pmm.
  \]
  On note enfin \( \parts M \) l'ensemble des parties finies de \( M \) et on
  suppose qu'il existe une fonction \( g \colon \R_{>0} \times \parts M \to \N
  \) telle que pour tout \( \eps > 0 \), tout \( S \in \parts M \) non vide et
  toute famille \( (\lambda_v)_{v\in S} \) de réels positifs telle que \(
    \sum_{v\in S} \lambda_v \degv = 1 \), on ait
  \[
    \card E'(\eps, S, \lambda)
    \le
    g(\eps, S)
    \pmm.
  \]
  Alors, pour tout \( \eps > 0 \) et tout \( S \in \parts M \) on a
  \[
    \card E(\eps, S)
    \le
    \sum_{T \in \parts S \minusset\emptyset}
    \binom{2\card T - 1}{\card T - 1}
    g(\eps/2, T)
    \pmm.
  \]
\end{lem}

\begin{rem} \label{r:app-prod}
  Dans l'application prévue, les paramètres seront choisis de la façon
  suivante.  L'ensemble \( A \) sera l'intersection d'un système complet de
  représentants modulo \( \vai \) dans un sous-groupe de rang fini (voire de
  type fini) de \( \va(\Qbar) \) avec \( \va(\Qbar) \setminus \avar(\Qbar) \) et
  \( H \) la hauteur induite sur cet ensemble par le choix d'un plongement
  projectif de \( \va \). L'ensemble \( M \) sera l'ensemble des places d'un
  corps de nombres \( \cdn \) suffisamment gros et en chaque place
  \( \degv = [\cdn_v : \Q_v] / [\cdn : \Q] \) est le degré local divisé par le
  degré global. En revanche, le choix des constantes \( c_v \) dépendra de
  l'application envisagée. On reconnaît la condition sur la famille
  \( \lambda \), et la fonction \( g \) est alors donnée par l'un des énoncés
  de la section précédente.
\end{rem}

\begin{proof}
  On suppose \( S \) et \( \eps \) fixés. On remarque que pour toute famille
  \( \lambda \) telle que que \( \sum_{v \in S} \lambda_v \degv = 1 \) on a \(
    E'(\eps, S, \lambda) \subset E(\eps, S) \) ; à l'inverse on aimerait bien
  recouvrir \( E(\eps, S) \) par de tels ensembles, en nombre fini. Dans ce
  but, pour tout \( x \in A \) (tel que \( H(x) > 1 \) pour simplifier pour
  l'instant), on introduit une famille de réels \( \lambda(x) \) telle que
  \begin{equation} \label{e:x2wt}
    d_v(x)
    =
    c_v H(x)^{-\lambda_v(x) \eps}
    \quad \forall v \in S
  \end{equation}
  de sorte que \( x \in E'(\eps, S, \lambda(x)) \). De plus, si \( x \in E(\eps,
    S) \) on a \( \sum_{v \in S} \lambda_v(x) \degv \ge 1 \). Ainsi, on a
  recouvert \( E(\eps, S) \) par des ensembles de la forme \( E'(\eps, S,
    \lambda(x)) \) mais ceux-ci ne sont \lat{a priori} pas en nombre fini ;
  par ailleurs les familles \( \lambda(x) \) obtenues ne sont pas
  nécessairement composées que de nombres positifs. (Le fait que leur somme
  pondérée ne soit pas exactement \( 1 \) n'est en revanche pas gênant, il
  suffirait de choisir une famille \( \lambda'(x) \) plus petite (pour l'ordre
  produit) sommant à \( 1 \), car la fonction \( \lambda \mapsto E'(\eps, S,
    \lambda) \) est décroissante.)

  Pour forcer les \( \lambda_v(x) \) à être positifs\footnote{Ce qui n'est en
    soi pas indispensable (on pourrait en fait se dispenser de cette hypothèse
    dans les résultats de décompte précédents en considérant un ensemble de
    places plus petit) mais sera utile pour assurer la finitude et contrôler
    le cardinal de l'ensemble \( C \) ci-dessous.} on introduit la fonction \(
    \pi \colon E(\eps, S) \to \parts S \) définie par
  \begin{equation}
    \pi(x)
    =
    \set{
      v \in S
      \text{ tel que }
      d_v(x) \le c_v
    }
    \pmm.
  \end{equation}
  On remarque que \( \pi \) prend en fait ses valeurs dans \( \parts S
    \minusset\emptyset \).  Par ailleurs, pour toute partie non vide \( T \)
  de \( S \), on introduit l'ensemble
  \begin{equation}
    C
    =
    \set{
      (\lambda_v)_{v \in T}
      \text{ tel que }
      \lambda_v \degv \card T \in \N
      \text{ et }
      \sum_{v \in T} \lambda_v \degv = 1
    }
  \end{equation}
  qui est fini et de cardinal \( \binom{2\card T - 1}{\card T - 1} \) car il
  est en bijection avec l'ensemble des familles de \( \card T \) entiers
  naturels dont la somme est \( \card T \). Pour obtenir la conclusion du
  lemme, il suffit donc de montrer que
  \begin{equation}
    \pi^{-1}(T)
    =
    \bigcup_{\lambda \in C} E(\eps/2, T, \lambda)
  \end{equation}
  (où l'on a noté \( \pi^{-1}(T) = \pi^{-1}(\set T) \) pour alléger) car les
  \( \pi^{-1}(T) \) pour \( T \) parcourant l'ensemble des parties non vides
  de \( S \) forment une évidemment partition de \( E(\eps, S) \).

  Fixons donc \( \emptyset \neq T \in \parts S \) et \( x \in \pi^{-1}(T)
  \). Par définition de \( E(\eps, S) \) et de \( \pi \) on a
  \begin{align}
    \bigl( \prod_{v \in S} c_v^\degv \bigr)
    H(x)^{-\eps}
    \ge
    \prod_{v \in S} d_v(x)^\degv
    \ge
    \prod_{v \in T} d_v(x)^\degv
    \prod_{v \in S \setminus T} c_v^\degv
  \end{align}
  d'où \( x \in E(\eps, T) \). Par ailleurs, si \( H(x) = 1 \) il est clair
  que \( x \in E'(\eps, S, \lambda) \) pour n'importe quelle famille \(
    \lambda \in C \), on suppose donc \( H(x) > 1 \) et on associe à \( x \)
  une famille \( \lambda(x) \) comme en~\eqref{e:x2wt}. Cette fois-ci, les \(
    \lambda_v(x) \) sont tous positifs ou nuls. Par ailleurs, on a
  successivement
  \begin{align}
    2 \card T
    \sum_{v \in T} \lambda_v(x) \degv
    & \ge
    2 \card T
    \\
    \sum_{v \in T} \floor{ 2 \lambda_v(x) \degv \card T }
    & \ge
    \card T
  \end{align}
  donc il existe des entiers naturels \( a_v(x) \) inférieurs ou égaux à
  \( 2 \lambda_v(x) \degv \card T \) et dont la somme est exactement
  \( \card T \). On pose alors \( \lambda'(x) = a_v(x) / (\degv \card T) \) de
  sorte que \( \lambda'(x) \in C \) et \( \lambda'_v(x) \le 2 \lambda_v(x) \),
  ce qui implique que \( x \in E'(\eps/2, T, \lambda'(x)) \), achevant ainsi
  la preuve.
\end{proof}

En pratique, la fonction de comptage ne dépend pas de l'ensemble de places
considérées, on utilisera donc le scolie suivant.

\begin{sco}
  Dans la situation du lemme précédent, si l'on suppose de plus que la
  fonction \( g \) ne dépend pas de son deuxième argument, alors on a
  \( \card E(\eps, S) \le 5^{\card S} g(\eps/2) \).
\end{sco}

\begin{proof}
  C'est un calcul élémentaire, fait page~125 de \cite{farhith}.
\end{proof}

\begin{coro}
  Soient \( \avar \) un translaté par un point algébrique d'une sous-variété
  abélienne \( \vai \) de \( \va \) de degré \( \adeg \) et de dimension \(
    \adim \).  On considère de plus un sous-groupe \( \grp \subset \va(\Qbar)
  \) de rang fini \( r \).  Pour tout réel \( \eps > 0 \) et tout entier \( m
    \ge g + 1 \), il existe au plus
  \begin{equation}
    2 \cdot 5^{\card S} \cdot
    \sqrt{ \frac{\adeg}\eps }
    (2mg)^{mg+1}
    \ln \Lambda_4
    \left(
      3 \sqrt m
      \bigl(
        172 \nclmaps \cdot 5^g \adeg s \, \eps^{-1}
      \bigr)^{ \frac{m}{2(m-g)} }
    \right)^r
  \end{equation}
  points de \( \grp \) distincts modulo \( \vai \) et satisfaisant
  aux deux conditions suivantes :
  \begin{align}
    0 < \prod\placerange \distv{x_i} \avar ^\degv
    & <
    \alpha^{-1}
    \hautm2{x_i}^{-\eps}
    \\
    \hautn{x_1}
    & > \cst{ht-v-gen1} \Lambda_4^{(2mg)^{mg}}
  \end{align}
  avec \( \Lambda_4 \) et \( \cst{ht-v-gen1} \) comme au
  lemme~\vref{l:big-by-cone} et au choix :
  \begin{enumthm}
  \item \( s = \min(\sbin{\adeg+n}{n}, \card \placesapx) \) et \( \alpha =
      \bigl( \adeg \sbin{\adeg+n}{n}^3 \bigr)^{1/2} \) ; ou bien
  \item \( s = 1 \) et \( \alpha =
      \hautm1 \avar ^2
      \cdot (\adeg+1)^{ 2 (\adim+2) (\adeg + n+1) } \cdot \frac{m^2}4
      \cdot \expb^{\adeg n} \adeg \sbin{\adeg+n}{n}^{1/2}
    \).
  \end{enumthm}
\end{coro}

\begin{proof}
  On applique le scolie précédent à la proposition~\vref{p:big-gen} comme
  prévu à la remarque~\vref{r:app-prod}. On constate que vu la définition de
  \( \Lambda_4 \), en y remplaçant \( \eps \) par \( \eps / 2 \), on
  ajoute \( \ln(2) m / (m-g) \) à son logarithme et que
  \begin{equation}
    \ln \Lambda_4 + \ln(2) m / (m-g)
    \le
    \sqrt 2 \ln \Lambda_4
    \pmm;
  \end{equation}
 les autres changements dans le décompte sont évidents.
\end{proof}

On peut alors simplifier légèrement la condition d'approximation en absorbant
la constante ; on aboutit au résultat suivant.

\begin{thm} \label{t:big-gen-prod}
  Soient \( \avar \) un translaté par un point algébrique d'une sous-variété
  abélienne \( \vai \) de \( \va \) de degré \( \adeg \) et de dimension \(
    \adim \).  On considère de plus un sous-groupe \( \grp \subset \va(\Qbar)
  \) de rang fini \( r \).  Pour tout réel \( \eps > 0 \) et tout entier \( m
    \ge g + 1 \), il existe au plus
  \begin{equation} \label{e:bgp-val}
    2 \cdot 5^{\card \placesapx} \cdot
    \sqrt{ \frac{\adeg}\eps }
    (2mg)^{mg+1}
    \ln \Lambda_5
    \left(
      3 \sqrt m
      \bigl(
        174 \nclmaps \cdot 5^g \adeg
        \, \eps^{-1}
        \bigr)^{ \frac{m}{2(m-g)} }
    \right)^r
  \end{equation}
  points de \( \grp \) distincts modulo \( \vai \) et satisfaisant
  aux deux conditions suivantes :
  \begin{align}
    0 < \prod\placerange \distv{x_i} \avar ^\degv
    & <
    \hautm2{x_i}^{-\eps}
    \\
    \hautn{x_1}
    & > \cst{ht-v-gen1} \Lambda_5^{(2mg)^{mg}}
  \end{align}
  avec
  \begin{align}
    \Lambda_5
    & =
    \bigl(
        87 \nclmaps \cdot 5^g \adeg
        \, \eps^{-1}
    \bigr)^{ \frac{m}{m-g} }
    \bigl( (\sqrt2 m g \adeg)^g \deg \va \bigr)^m
    \\
    \cst{ht-v-gen1}
    & =
    \adeg \max \bigl(
      \adeg^g \hautl1{\va}, \hlclab, \htcmp
    \bigr)
    + (g + 1) \deg \va
    \Bigl(
      \adeg^n \ln(\adeg) n
    \\ & \qquad
      + \adeg^g \bigl (
        \hautl1 \avar
        + (\adim + 2) \ln (\adeg + 1) ( \adeg + n + 1 )
        + \ln(m/2)
      \bigr)
    \Bigr)
    \pmm.
  \end{align}
\end{thm}

\begin{proof}
  On commencer par majorer chacun des deux choix possibles pour \( \alpha \)
  dans le corollaire précédent. Pour le premier, on a immédiatement
  \begin{equation}
    \ln \bigl( \adeg \sbin{\adeg+n}{n}^3 \bigr)^{1/2}
    \le
    \frac12 \ln d + \frac32 n \ln(d+1)
    \le
    \cst{ht-v-gen1}
    \pmm.
  \end{equation}
  Pour le deuxième choix, on a
  \begin{align}
    \ln \alpha
    & =
    2 \hautl1 \avar
    + 2 (\adim+2) (\adeg + n+1) \ln (\adeg+1)
    + 2 \ln(\tfrac m 2)
    + \adeg n + \ln \adeg + \frac \adeg 2 \ln(n+1)
    \notag
    \\ & \le
    \adeg \bigl(
    \hautl1 \avar + (\adim+2) (\adeg + n+1) \ln (\adeg+1) + \ln(\tfrac m 2)
    \bigr)
    + 2 \adeg n
    \le
    \cst{ht-v-gen1}
  \end{align}
  Ainsi, quel que soit le choix pour \( (s, \alpha) \) dans le corollaire
  précédent, on a assez largement
  \(
    \ln \alpha
    \le
    \cst{ht-v-gen1} \Lambda_5 \cdot \eps / 87
  \)
  dont on déduit immédiatement que
  \(
    \hautm2{ x_i }^{\eps/87}
    \ge
    \alpha
  \)
  puis que les hypothèses du présent énoncé impliquent celles du corollaire
  précédent en remplaçant \( \eps \) par \( 86\eps / 87 \). On applique donc
  le corollaire en question pour aboutir au résultat annoncé.

  Par ailleurs, comme les deux valeurs possibles de \( \alpha \) peuvent être
  absorbées à un coût équivalent, il est clair qu'on a intérêt à choisir la
  deuxième option pour le choix de \( (s, \alpha) \) afin d'avoir \( s = 1 \)
  qui minimise les différentes constantes.
\end{proof}

\begin{rem}
  Dans le cas où \( \card \placesapx = 1 \), on aura plutôt intérêt à utiliser
  la proposition~\ref{p:big-gen} et prendre \( s = \card \placesapx \) puis
  supprimer le \( \alpha_v \) éventuel (si l'unique place de \( \placesapx \)
  est archimédienne) comme ci-dessus en remplaçant \( 86 \) par \( 87 \) dans
  le décompte et la définition de \( \Lambda_4 \).
\end{rem}

La dépendance en \( \eps \) dans le décompte précédent est en
\begin{equation}
  \left( \frac1\eps \right)^{\frac12 + \frac r2 \cdot \frac m{m-g}}
  \ln \left( \frac1\eps \right)
  \pmm.
\end{equation}
L'exposant du premier facteur tend donc vers \( (r+1) / 2 \) lorsque \( m \)
tend vers l'infini. Cependant, quand \( m \) grandit, le facteur \( (2mg)^{mg+1}
\) dans le décompte croît rapidement et la condition de hauteur aussi croît
rapidement. La section suivante discute du choix de la valeur de \( m \).



\section{Choix du paramètre \( m \)}
\label{sec:opti-m}

Notons \( N(m) \) la quantité donnée par~\eqref{e:bgp-val} et \( C(m) =
  \cst{ht-v-gen1} \Lambda_5^{(2mg)^{mg}} \). Pour optimiser le
théorème~\ref{t:big-gen-prod}, on peut chercher à minimiser \( C(m) \) ou \(
  N(m) \), ou idéalement les deux, ce qui ne sera possible que dans certains
cas, nous allons le voir.

Le cas de \( C(m) \) est assez clair : cette quantité est croissante en \( m
\). En effet, la seule partie dans sa définition qui ne soit pas croissante en
\( m \) est le \( m / (m-g) \) en exposant du premier facteur de \( \Lambda_5
\), mais elle est largement compensée par l'exposant \( (2mg)^{mg} \). Pour
minimiser \( C(m) \), il suffit donc de poser \( m = g+1 \). On obtient le
résultat suivant.

\begin{coro} \label{c:big-dec}
  \todo (À écrire.)
\end{coro}

Attachons-nous maintenant à minimiser \( N(m) \) ou du moins à trouver une
valeur de \( m \) qui rend cette quantité relativement petite. On utilisera
le lemme suivant.

\begin{lem} \label{l:minifun}
  Soient \( a \ge 0 \) et \( b \ge 2 \) deux réels et \( c \ge 1 \) un entier
  ; on pose \( f(x) = x \ln x + bx + ac / (x-c) \) et \( d = 1 + b + \ln(1+c)
  \). Si \( a \le d/c \), la fonction \( f \) est croissante sur \( [c+1,
    +\infty \mathclose[ \).  Sinon, il existe un entier \( x_0 \ge c + 1 \)
  tel que \( f(x_0) \le 2 ac + (c+1) d \) ; on peut prendre
  \( x_0 = \sqrt{ac/d} \).
\end{lem}

\begin{proof}
  On a \( f'(x) = \ln x + 1 + b - a / (x-c)^2 \) ; cette fonction est
  croissante pour \( x > c \). De plus, \( f'(c+1) = \ln(c+1) + 1+b - a \) est
  positif si et seulement si \( ac \le 1 + b + \ln(1+c) \) ; dans ce cas \( f
  \) est bien croissante sur \( [c+1, +\infty \mathclose[ \).

  Posons \( d = 1 + b + \ln(1+c) \) et supposons donc \( ac \ge d \) désormais.
  Une vague analogie avec le lemme~2.14.7 page~150 de \cite{farhith} nous
  amène à poser \( t = \sqrt{ac/d} + c \) à considérer \( x_0 = \ceil t \).
  Par hypothèse, on a bien \( x_0 \ge c + 1 \) entier et
  \begin{align}
    f(x_0)
    & \le
    (t+1) \ln(t+1)
    + b (t+1)
    + ac / (t-c)
    \\ & \le
    (\sqrt{ac/d} + c+1) \ln(\sqrt{ac/d} + c+1)
    + b (\sqrt{ac/d} + c+1)
    + \sqrt{acd}
    \pmm.
  \end{align}
  On écrit alors
  \begin{equation}
    \ln(\sqrt{ac/d} + c+1)
    =
    \ln(\sqrt{a/dc} + 1 + 1/c) + \ln c
    \le
    \sqrt{a/dc} + 1/c + \ln c
  \end{equation}
  puis, en développant et en utilisant la définition de \( d \) ainsi que les
  hypothèses \( b \ge 2 \) et \( c \ge 1 \) entier :
  \begin{align}
    f(x_0)
    & \le
    (\sqrt{ac/d} + c+1) (\sqrt{a/dc} + 1/c + \ln c)
    + \sqrt{ac} (\sqrt d + b/\sqrt d)
    + b (c+1)
    \\ & \le
    a/d + \sqrt{a/dc} + \sqrt{ac/d} \ln c
    + \sqrt{ac/d}
    + \sqrt{a/dc}
    + \sqrt{ac} (\sqrt d + b/\sqrt d)
    \\ & \qquad
    + (c+1) (b + 1/c + \ln c)
    \\ & \le
    a/d
    + \sqrt{ac} \bigl( \sqrt d + (b + 2/c + 1 + \ln c) /\sqrt d \bigr)
    + (c+1) d
    \\ & \le
    a/d
    + \sqrt{ac} \bigl( \sqrt d + 5\sqrt d / 3 \bigr)
    + (c+1) d
    \pmm.
  \end{align}
  On utilise alors l'hypothèse \( ac \ge d \) pour majorer le deuxième terme
  par \( 5ac/ 3 \), puis on remarque que \( 1/d \le c/3 \) pour conclure.
\end{proof}

\begin{rem}
  Dans le cas \( c = 1 \) et \( b = \ln 2904 \), on peut comparer avec le
  lemme~2.14.7 page~150 de \cite{farhith}. Remarquons que les notations sont
  différentes : nous notons \( a \) la quantité qu'il note \( \ln a \) et
  notre fonction \( f \) diffère de sa fonction \( g \) d'un terme \( a \)
  (dans nos notations).

  Compte tenu de ces différentes et dans nos notations, \bsc{Farhi} trouve un
  entier \( m_0 \) tel que \( f(m_0) \le 183 a / \ln a \) alors que le lemme
  précédent donne un entier \( x_0 \) tel que \( f(x_0) \le 2 (a + 10) \) ;
  les deux sous la condition \( a \ge \ln 15788 \approx 10 \).  Notre résultat
  est évidemment moins fin pour les grandes valeurs de \( a \), mais il est
  meilleur tant que \( a \) reste petit (inférieur à \( \expb^{91} \) par
  exemple).

  Le \( m_0 \) de \bsc{Farhi} est essentiellement optimal, comme le montre son
  lemme ; c'est donc dans son estimation de \( f(m_0) \) qu'a lieu la perte
  pour les petites valeurs de \( a \). Le gain pour les grandes valeurs peut
  en revanche provenir du fait que \( m_0 \) est plus proche du réel
  minimiseur que \( x_0 \), de la méthode utilisée pour estimer son image, ou
  plus probablement des deux. Par ailleurs, les calculs de \bsc{Farhi}
  semblent difficiles à généraliser, d'une part car certaines de ses
  estimations reposent sur des vérifications numériques, d'autre par parce que
  si \( c \neq 1 \), la fonction ne semble plus satisfaire d'équation
  différentielle du même type que celle de sa page~153.
\end{rem}

Voyons comment ce lemme s'applique à notre situation. En notant \( \alpha =
  87 \nclmaps \cdot 5^g \adeg / \eps \), on a
\begin{align}
  \ln\Lambda_5
  & =
  \frac{m \ln\alpha}{m-g}
  + m \ln \bigl( (\sqrt2 m g \adeg)^g \deg\va \bigr)
  \le
  m^2
  (g+1) \ln(\alpha) \ln \bigl( (\sqrt2 g \adeg)^g \deg\va \bigr)
  \pmm.
\end{align}
On note alors \( \gamma_1 \) le facteur indépendant de \( m \) dans cette
majoration, de sorte que
\begin{align}
  N(m)
  & \le
  2 \cdot 5^{\card \placesapx} \cdot
  (\adeg / \eps)^{1/2}
  (2mg)^{mg+1}
  m^2 \gamma_1
  \left(
    3 \sqrt m (2\alpha)^{ \frac{m}{2(m-g)} }
  \right)^r
  \\ & \le
  \underbrace{
    4 g \cdot 5^{\card \placesapx}
    (\adeg / \eps)^{1/2}
    \gamma_1 3^r
  }_{\displaystyle \gamma_2}
  \cdot
  \underbrace{
    m^{mg} (2g)^{mg} m^{3 + r/2} (2\alpha)^{ \frac{rm}{2(m-g)} }
  }_{\displaystyle N_1(m)}
\end{align}
où \( \gamma_2 \) ne dépend pas de \( m \). On écrit alors
\begin{align}
  \ln N_1(m)
  & =
  gm \ln m +
  gm \ln(2g)
  + (3 + r/2) \ln m
  + \frac{m}{m-g} (r/2) \ln(2\alpha)
  \\ & \le
  gm \ln m +
  m \bigl( g \ln(2g) + \frac32 + \frac r4 \bigr)
  + \frac r2 \ln(2\alpha)
  + \frac{g}{m-g} \frac r2 \ln(2\alpha)
  \pmm.
\end{align}
En posant \( b = \ln(2g) + 3/(2g) + r/(4g) \) et \( a = r \ln(2\alpha) / (2g) \)
ainsi que \( c = g \), on a \( \ln N_1(m) \le g \, ( f(m) + a) \) où \( f \)
est la fonction du lemme~\ref{l:minifun}. On vérifie aisément que \( b \ge 2
\) et \( a \ge 0 \), de sorte qu'on peut appliquer le lemme. On pose donc
\begin{equation}
  d
  =
  1 + \ln(2g) + 3/(2g) + r/(4g) + \ln(1+g)
  \le
  2 \ln(2g) + 1 + \frac{ 6 + r }{ 4g }
\end{equation}
puis \( x_0 = \sqrt{ac/d} \). En supposant que \( ac \ge d \) (hypothèse sur
laquelle nous reviendrons),  le lemme cité dit alors que
\begin{equation}
  N_1(m)
  \le
  2ac + (c+1) d
  =
  \dots % XXX
\end{equation}






\section{Options pour les petits points}
\label{sec:small-points}

Il ne semble pas évident d'exploiter l'hypothèse d'approximation dans le
décompte des petits points. Nous rappelons donc pour mémoire comment
compter les points de petite hauteur d'une variété abélienne appartenant à un
groupe de type fini donné : ce décompte repose sur une propriété élémentaire
en géométrie euclidienne, que l'on commence par rappeler.

\begin{fact}
  Soient \( E \) un espace euclidien de dimension \( r \) et deux réels \(
    \rho \) et \( \mu \). On peut recouvrir toute boule (fermée) de rayon \(
    \rho \) par des boules (ouvertes) de rayon \( \mu \) en nombre inférieur à
  \( ( 2 \, \frac\rho\mu + 1 )^r \).
\end{fact}

\begin{proof}
  C'est le lemme~6.1, p.~541 de \cite{remdcl}, où l'on a par ailleurs effectué
  le changement de notations \( \mu = \rho / \gamma \).  L'énonce donné dans
  la référence citée ne précise pas si les boules sont ouvertes ou fermées,
  mais on constate facilement que la preuve qui y est proposée fonctionne
  parfaitement pour la variante la plus forte, énoncée ci-dessus.
\end{proof}

On en déduit la majoration suivante du nombre de points de petite hauteur sans
aucune hypothèse d'approximation.

\begin{coro} \label{c:small-va}
  Soit \( \grp \) un sous-groupe de type fini de \( \va(\Qbar) \) ; on note
  \( r \) le rang de \( \grp \) et \( \hmin(\grp) \) le minimum de \(
    \hautn x \) quand \( x \) parcourt l'ensemble des points d'ordre infini de
  \( \grp \).  Pour tout réel positif \( R \), on a
  \begin{equation}
    \card \set*{
      x \in \grp
      \text{ tel que }
      \hautn x \le R
    }
    \le
    \card \grp_\torsion
    \cdot
    \left( 1 + 2\sqrt{R / \hmin(\grp)} \right)^r
  \end{equation}
  où \( \grp_\torsion \) désigne l'ensemble des points de torsion de \( \grp
  \).
\end{coro}

\begin{proof}
  Dans \( \grp \otimes_\Z \R \) muni de la structure euclidienne induite par
  la hauteur normalisée, on applique le fait précédent avec \( \rho = \sqrt R
  \) et \( \mu = \sqrt{\hmin(\grp)} \) puis on remarque que la préimage dans \(
    \grp \) de chacune des boules ouvertes de rayon \( \mu \) est composée
  de points qui sont tous égaux modulo \( \grp_\torsion \).
\end{proof}

\begin{rem}
  Si dans l'énoncé précédent on prend \( \grp = \va(\cdn) \), on retrouve le
  lemme~2.11.1, p.~117 de~\cite{farhith}.
\end{rem}

Si l'on applique le résultat précédent en conjonction avec le
théorème~\vref{t:big-gen-prod}, on constate assez facilement que le décompte
obtenu pour les petits points est largement supérieur à celui obtenu pour les
grands poins (à moins que \( \hmin(\grp) \) ne soit particulièrement grand),
ce qui n'est pas très satisfaisant. Nous établissons donc un résultat, basé
sur l'inégalité de \bsc{Liouville}, permettant de supprimer les petits points
quitte à renforcer l'hypothèse d'approximation.

\begin{lem}
  Soit \( \avar \) une sous-variété de \( \va \), de dimension \( \adim \) et
  de degré
  \( \adeg \). Si \( R \) et \( F \) sont deux réels positifs, l'ensemble des
  points \( x \in \va(\Qbar) \) tels que
  \begin{equation}
    0
    <
    \prod\placerange
    \distv x \avar ^\degv
    \le
    \expb^{-F}
    \hautm2{x}^{-\eps}
    \quad\text{et}\quad
    \hautn x \le R
  \end{equation}
  est vide dès que
  \( F
    >
    (\adeg-\eps) R
    + \hautl1{ \chow \avar }
    + \adeg (\adim+1) \ln(3\adeg)
    + \frac32 \ln(n+1)
    + (\adeg-\eps) \htcmp
  \).
\end{lem}

\begin{proof}
  Si l'ensemble en question n'est pas vide, on choisit un point \( x \) dedans
  et on lui applique l'inégalité de \bsc{Liouville}
  (proposition~\vref{p:liouville}) :
  \begin{equation}
    \frac1{
      (n+1)^{3/2}
      (3\adeg)^{\adeg (\adim+1)}
      \, \hautm1{ \chow \avar }
      \, \hautm2 x ^\adeg
    }
    \le
    \prod\placerange
    \distv x \avar ^\degv
    \le
    \expb^{-F}
    \hautm2{x}^{-\eps}
  \end{equation}
  puis en prenant les opposés des logarithmes
  \begin{align}
    F
    & \le
    (\adeg-\eps) \hautl2 x
    + \hautl1{ \chow \avar }
    + \adeg (\adim+1) \ln(3\adeg)
    + \frac32 \ln(n+1)
    \\ & \le
    (\adeg-\eps) R
    + \hautl1{ \chow \avar }
    + \adeg (\adim+1) \ln(3\adeg)
    + \frac32 \ln(n+1)
    + (\adeg-\eps) \htcmp
    \pmm.
  \end{align}
  Par contraposée, l'ensemble considéré est vide si cette inégalité est
  fausse.
\end{proof}

On en déduit immédiatement le corollaire pratique suivant.

\begin{coro} \label{c:kill-small}
  Soit \( \avar \) une sous-variété de \( \va \), de dimension \( \adim \) et
  de degré \( \adeg \). Si \( R \) est un réel supérieur ou égal à
  \begin{equation}
    \frac1\eps \left(
      \hautl1{ \chow \avar }
      + \adeg (\adim+1) \ln(3\adeg)
      + \frac32 \ln(n+1)
      + \adeg \htcmp
    \right)
    \pmm,
  \end{equation}
  il n'existe aucun point \( x \in \va(\Qbar) \) tel que
  \begin{equation}
    0
    <
    \prod\placerange
    \distv x \avar ^\degv
    \le
    \expb^{- \adeg R}
    \hautm2{x}^{-\eps}
    \quad\text{et}\quad
    \hautn x \le R
    \pmm.
  \end{equation}
\end{coro}

\begin{proof}
  Application immédiate du lemme précédent avec \( F = \adeg R \).
\end{proof}

On en déduit à titre d'exemple une variante du théorème~\vref{t:big-gen-prod}
; le lecteur en déduira sans peine des variantes similaires des autres
résultats de décompte.

\begin{prop} \label{p:all-gen}
  Soient \( \avar \) un translaté par un point algébrique d'une sous-variété
  abélienne \( \vai \) de \( \va \) de degré \( \adeg \) et de dimension \(
    \adim \).  On considère de plus un sous-groupe \( \grp \subset \va(\Qbar)
  \) de rang fini \( r \).  Pour tout réel \( \eps > 0 \) et tout entier \( m
    \ge g + 1 \), il existe au plus
  \begin{equation}
    2 \cdot 5^{\card \placesapx} \cdot
    \sqrt{ \frac{\adeg}\eps }
    (2mg)^{mg+1}
    \ln \Lambda_5
    \left(
      3 \sqrt m
      \bigl(
        174 \nclmaps \cdot 5^g \adeg s
        \, \eps^{-1}
        \bigr)^{ \frac{m}{2(m-g)} }
    \right)^r
  \end{equation}
  points de \( \grp \) distincts modulo \( \vai \) et satisfaisant
  à la condition suivante :
  \begin{align}
    0 < \prod\placerange \distv{x_i} \avar ^\degv
    & <
    \exp\left(
      - \adeg \cst{ht-v-gen1} \Lambda_5^{(2mg)^{mg}}
    \right)
    \hautm2{x_i}^{-\eps}
  \end{align}
  avec \( \Lambda_5 \) et \( \cst{ht-v-gen1} \) comme au
  théorème~\vref{t:big-gen-prod}.
\end{prop}

\begin{proof}
  La condition d'approximation considérée permet d'appliquer le
  corollaire~\vref{c:kill-small} avec
  \( R = \cst{ht-v-gen1} \Lambda_5^{(2mg)^{mg}} \),
  quantité qui satisfait amplement l'hypothèse du corollaire. Ainsi, on voit
  que tous les points satisfaisant cette condition d'approximation satisfont
  aussi \( \hautn{x_i} \ge R \) et on peut appliquer le
  théorème~\vref{t:big-gen-prod} pour conclure.
\end{proof}


\cleardoublepage
\endinput

% vim: spell spelllang=fr
