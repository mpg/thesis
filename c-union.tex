% !TEX root = main.tex

\chapter{Déduction des résultats principaux}

\section{Réduction au cas d'un hyperplan}

\begin{lem} \todo
  Dans l'énoncé de \bsc{Vojta}, on peut supposer que \( \varapx \) est un
  diviseur de \( \va \), découpé par une forme sur \( \projd \).
\end{lem}

\begin{ideas}
  Il est clair que \( \varapx \) est une intersection de diviseurs découpés
  par des formes sur \( \projd \), mettons qu'on les note \(
    \divi_k \).  Alors l'ensemble des approximations exceptionnelles de \(
    \varapx \) est inclus dans l'union des ensembles d'approximations
  exceptionnelles des
  \( \divi_\alpha \). (Attention, je crois pas que ça marche avec
  l'intersection.)

  Donc on a qu'à appliquer \bsc{Vojta}-particulier à chacun de ses diviseurs, on
  perd sur le \( \puiss \) en passant au cas général. Pour que le
  résultat reste intéressant, il s'agit donc de bien contrôler la famille
  \( \divi_k \) : nombre d'éléments, leur degré et leur hauteur par
  rapport à celles de \( \varapx \). Il semble que \cite[prop.~6.1]{remdcl}
  soit un bon point de départ pour ça.
\end{ideas}

\begin{lem} \todo
  On peut supposer que le diviseur est découpé par \( \ch0 = 0 \).
\end{lem}

\begin{ideas}
  On fait un \bsc{Veronese} pour se ramener au cas où le diviseur est de degré~$1$,
  puis un changement de coordonnées pour avoir la bonne équation.
  Trucs à surveiller :
  \begin{enumthm}
    \item Comment se comporte la distance sous un \bsc{Veronese} ? Éléments de
      réponse : \cite{phidg} en haut de la page~89, ou probablement Jadot
      (voir la définition des distances avec indices, les normalisations,
      \dots).
  \item On veut travailler dans un plongement de \bsc{Mumford} modifié associé
      donc à un fibré totalement symétrique. Comment ça interagit avec la
      manip ? Faut-il exiger des trucs sur les \( \divi_\alpha \) au lemme
      précédent ?
  \end{enumthm}
\end{ideas}


\endinput

% vim: spell spelllang=fr

