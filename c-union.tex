% !TEX root = main.tex

\chapter{Déduction des résultats principaux}
\label{chap:union}

Nous déduisons ici des deux chapitres précédents des décomptes
d'approximations exceptionnelles, en procédant en deux temps : pour les points
de hauteur assez grande (au sens de~\eqref{e:Vbig} et/ou~\eqref{e:Mbig}), que
nous appellerons grands points, nous combinons les inégalités de \bsc{Vojta}
et de \bsc{Mumford} obtenues précédemment ; pour les petits points, on procède
à des décomptes triviaux soit en oubliant l'hypothèse d'approximation, soit en
renforçant celle-ci pour interdire leur existence.

\section{Décomptes des grands points}
\label{sec:big-points}

On commence par compter le nombre de points dans chaque cône tronqué.

\begin{lem} \label{l:big-by-cone}
  Soient \( V \) un translaté par un point algébrique d'une sous-variété
  abélienne \( \vai \) de \( \va \) de degré \( d \) et de dimension \( u \).
  On fixe réel \( 0 < \expapx < d \), un entier \( \puiss \ge \genre + 1 \) et
  on considère une famille de points \( \ex[1], \dots, \ex[p] \in \va(\Qbar)
  \) distincts modulo \( \vai \) et satisfaisant simultanément aux conditions
  suivantes :
  \begin{align}
    \label{e:big-ha}
    0 < \distv{\ex*} V
    & <
    \bigl( \Delta \sbin{\Delta+\dimp}{\dimp}^3 \bigr)^{-\dv/2}
    \hautm[2]{\ex*}^{-\wtapx \expapx}
    \quad \forall \place \in \placesapx
    \\ \label{e:big-big}
    \hautn{\ex[1]}
    & > \cst{ht-v-gen1} \Lambda_4^{(2\puiss\genre)^{\puiss\genre}}
    \\ \label{e:big-cos}
    \cos(\ex*, \ex[\fcti])
    & > 1 - \frac1{ \puiss \, \cst{v-gen} }
  \end{align}
  avec
  \begin{align}
    \cst{v-gen}
    & =
    \bigl(
        86 \nclmaps \cdot 5^\genre d \sbin{\dimp + d}{d}
        \, \expapx^{-1}
    \bigr)^{ \frac{\puiss}{\puiss-\genre} }
    \\
    \Lambda_4
    & =
    \cst{v-gen}
    \bigl( (\sqrt2 \puiss \genre d)^\genre \deg \va \bigr)^\puiss
    \\
    \cst{ht-v-gen1}
    & =
    d \max \bigl(
      d^\genre \hautl[1]{\va}, \hlclab, \htcmp
    \bigr)
    + (\genre + 1) \deg \va
    \Bigl(
      d^\dimp \ln(d) \dimp
    \\ & \qquad
      + d^\genre \bigl (
        \hautl[1] V
        + (u + 2) \ln (d + 1) ( d + \dimp + 1 )
        + \ln(\puiss/2)
      \bigr)
    \Bigr)
    \pmm.
  \end{align}
  On a alors nécessairement
  \begin{equation}
    p
    \le
    \sqrt{\frac d \expapx}
    (4\puiss\genre)^{\puiss\genre}
    \bigl(
      - \ln \expapx
      +
      \cst{big/cone}
    \bigr)
    \pmm,
  \end{equation}
  où
  \begin{equation}
    \newcst[]{big/cone}
    =
    9
    + 2 \ln \nclmaps
    + \puiss \deg\va
    + \ln(d+1) (n+1 + \puiss\genre)
    + \puiss\genre (\ln(\puiss\genre) + 2)
    \pmm.
  \end{equation}
\end{lem}

\begin{proof}
  Soit \( x_1, \dots, x_p \) une famille comme dans l'énoncé, qu'on suppose de
  plus ordonnée par hauteur normalisée croissante. On pose
  \( \phi =( \puiss \, \cst{v-gen})^{-1} \) et
  \( \rho = \sqrt{ \expapx / d } \).  On constate facilement que
  \( \phi \le \expapx / (2580 d) \) de sorte que ce couple \( (\rho, \phi) \)
  satisfait à~\eqref{e:rho-phi}. Par ailleurs, il est clair que la
  condition~\eqref{e:big-big} est plus forte que~\eqref{e:Mbig}, car \(
    \cst{v-gen} \ge 2/\expapx \) et la parenthèse dans le membre de droite
  de~\eqref{e:Mbig} est largement majorée par \( 3 \cst{ht-v-gen1}
  \).\worknote{sauf peut-être \( h(\vai) \) vs \( h(V) \)} On peut donc
  appliquer le théorème~\ref{t:mumford} et conclure que
  \( \hautn{ x_{i+1} } > (1 + \sqrt{\expapx / d}) \hautn{ x_i } \) pour tout
  \( i \le p-1 \).

  On pose alors
  \begin{equation}
    \eta
    =
    \left\lceil
      (2\puiss\genre)^{\puiss\genre}
      \frac{ \ln \Lambda_4 }{ \ln (1 + \sqrt{\expapx / d}) }
    \right\rceil
  \end{equation}
  de sorte que, pour tout \( i \le p - \eta \), on a
  \begin{equation}
    \hautn{ x_{i+\eta} }
    >
    (1 + \sqrt{\expapx / d})^\eta
    \hautn{ x_i }
    \ge
    \Lambda_4^{(2\puiss\genre)^{\puiss\genre}}
    \hautn{ x_i }
    \pmm.
  \end{equation}
  Supposons alors que \( p \ge (\puiss - 1) \eta + 1 \) contrairement à la
  conclusion du lemme : on pourrait alors poser \( \ex* = x_{1 + (\fct-1)
      \eta} \) pour \( i \in {1, \puiss} \) et la famille \( (\ex*)_\fct \)
  contredirait alors le corollaire~\ref{c:vojta-gen2}, ce qui est absurde.

  Ainsi, on a
  \begin{equation}
    p
    \le
    (\puiss - 1)
    \left(
      (2\puiss\genre)^{\puiss\genre}
      \frac{ \ln \Lambda_4 }{ \ln (1 + \sqrt{\expapx / d}) }
      + 1
    \right)
    \le
    \puiss
    (2\puiss\genre)^{\puiss\genre}
    \frac{ \ln \Lambda_4 }{ \ln (1 + \sqrt{\expapx / d}) }
    \pmm.
  \end{equation}
  On remarque alors que la fonction \( x \mapsto \ln(1+x)/x \) est
  décroissante pour \( x > 0 \) pour minorer le dénominateur par \( \ln(2)
    \sqrt{\expapx / d} \) compte tenu de l'hypothèse sur \( \expapx \) ; on
  obtient alors, en remarquant de plus que \( \ln 2 \ge 1/2 \) :
  \begin{equation}
    p
    \le
    \puiss (2\puiss\genre)^{\puiss\genre}
    \cdot 2 \sqrt{\frac d \expapx}
    \cdot \ln \Lambda_4
    \le
    (4\puiss\genre)^{\puiss\genre}
    \cdot \sqrt{\frac d \expapx}
    \cdot \ln \Lambda_4
    \pmm.
  \end{equation}
  On écrit alors
  \begin{align}
    \ln \Lambda_4
    & =
    \frac\puiss{\puiss-\genre}
    \left(
      \ln 86
      + \ln \nclmaps
      + \genre \ln 5
      + \ln d
      + {\textstyle \ln \binom{n+d}d}
      - \ln \expapx
    \right)
    \\ & \qquad
    + \puiss \bigl(
      \deg\va + \genre (\ln2/2 + \ln(\puiss\genre) + \ln d)
    \bigr)
    \\ & \le
    2 \bigl(
      9/2
      + \ln \nclmaps
      + \genre \ln 5
      + (n+1) \ln(d+1)
      - \ln \expapx
    \bigr)
    \\ & \qquad
    + \puiss \deg\va
    + \puiss\genre (\ln(\puiss\genre) + \ln2/2)
    + \puiss\genre \ln d
    \\ & \le
    - \ln \expapx
    + 9
    + 2 \ln \nclmaps
    + \puiss \deg\va
    + \ln(d+1) (n+1 + \puiss\genre)
    + \puiss\genre (\ln(\puiss\genre) + 2)
  \end{align}
\end{proof}

On utilise ensuite le fait suivant.

\begin{fact}
  Soient \( r \) un entier et \( \gamma > 0 \) un réel. On peut recouvrir \(
    \R^r \) par \( \floor{(1 + \sqrt{8/\gamma})^r} \) ensembles dans chacun
  desquels deux points quelconques satisfont \( \cos(x, y) \ge 1 - \gamma \).
\end{fact}

\begin{proof}
  C'est le corollaire~6.1, p.~542 de~\cite{remdcl}.
\end{proof}

\begin{prop} \label{p:big-gen}
  Soient \( V \) un translaté par un point algébrique d'une sous-variété
  abélienne \( \vai \) de \( \va \) de degré \( d \) et de dimension \( u \).
  On considère de plus un sous-groupe \( \Gamma \subset \va(\Qbar) \) de rang
  fini \( r \).
  Pour tout réel \( \expapx > \) et tout entier \( \puiss \ge \genre + 1 \),
  il existe au plus
  \begin{equation}
    \sqrt{\frac d \expapx}
    (4\puiss\genre)^{\puiss\genre}
    \bigl(
      - \ln \expapx
      +
      \cst{big/cone}
    \bigr)
    \left(
      3 \sqrt\puiss
      \bigl(
        86 \nclmaps \cdot 5^\genre d \sbin{\dimp + d}{d}
        \, \expapx^{-1}
        \bigr)^{ \frac{\puiss}{2(\puiss-\genre)} }
    \right)^r
  \end{equation}
  points de \( \Gamma \) distincts modulo \( \vai \) et satisfaisant
  aux deux conditions suivantes :
  \begin{align}
    0 < \distv{\ex*} V
    & <
    \bigl( \Delta \sbin{\Delta+\dimp}{\dimp}^3 \bigr)^{-\dv/2}
    \hautm[2]{\ex*}^{-\wtapx \expapx}
    \quad \forall \place \in \placesapx
    \\
    \hautn{\ex[1]}
    & > \cst{ht-v-gen1} \Lambda_4^{(2\puiss\genre)^{\puiss\genre}}
  \end{align}
  avec \( \cst{big/cone} \), \( \Lambda_4 \) et \( \cst{ht-v-gen1} \) comme au
  lemme~\ref{l:big-by-cone}.
\end{prop}

\begin{proof}
  En utilisant le fait précédent, on recouvre l'espace euclidien \( \Gamma
    \otimes_\Z \R \) muni du produit scalaire donné par la hauteur de \NT par
  des ensembles dans lesquels deux points quelconques
  satisfont~\eqref{e:big-cos}. Le nombre d'ensembles nécessaires est au plus
  \begin{equation}
    \left(
      1 + \sqrt{8 \puiss}
      \bigl(
        86 \nclmaps \cdot 5^\genre d \sbin{\dimp + d}{d}
        \, \expapx^{-1}
        \bigr)^{ \frac{\puiss}{2(\puiss-\genre)} }
    \right)^r
    \le
    \left(
      3 \sqrt\puiss
      \bigl(
        86 \nclmaps \cdot 5^\genre d \sbin{\dimp + d}{d}
        \, \expapx^{-1}
        \bigr)^{ \frac{\puiss}{2(\puiss-\genre)} }
    \right)^r
  \end{equation}
  en remarquant que \( 1 + \sqrt{ 8 \cdot 86 \cdot 3 \cdot 5 } \le 3 \sqrt{ 86
      \cdot 3 \cdot 5 } \).

  Il ne reste plus qu'à appliquer lemme~\ref{l:big-by-cone} à chacun de ces
  ensembles pour conclure.
\end{proof}




\section{Décomptes des petits points}
\label{sec:small-points}



\section{Décomptes complets}
\label{sec:all-points}



\section{Hypothèse d'approximation produit}
\label{sec:ha-prod}

On utilisera le lemme suivant, qui n'est rien d'autre qu'une version plus
abstraite du lemme~2.12.1, p. 120 de~\cite{farhith}. Par ailleurs, par rapport
à cette référence, on a fixé \( \eps'(T) = \eps/2 \), d'où \( A(T) = \card T
\), car c'est la valeur qui a le plus de sens pour les applications, afin de
simplifier légèrement l'énoncé et sa preuve. L'énoncé ci-dessous est
volontairement aussi abstrait que possible afin de mettre en lumière le
caractère purement combinatoire de la preuve et d'être utilisable dans
d'autres situations que celle étudiée ici ; juste après l'énoncé on détaille
la correspondance avec avec la situation étudiée ici afin d'en éclairer le
sens.

\begin{lem}
  Soit \( A \) un ensemble et \( H \) une fonction de \( A \) dans \( [1,
    +\infty \mathclose[ \). Soit par ailleurs \( M \) un autre ensemble et,
  pour chaque élément \( v \) de \( M \) :
  \begin{itemize}
    \item une fonction \( d_v \colon A \to [0, 1] \) ;
    \item deux réels \( 0 < \degv < 1 \) et \( 0 < c_v < 0 \).
  \end{itemize}
  Pour tout réel \( \eps > 0 \) et tout partie finie \( S \) de \( M \), on
  définit
  \[
    E(\eps, S)
    =
    \set*{
      x \in A
      \text{ tel que }
      \prod_{v \in S} d_v(x)^\degv
      \le
      \bigl( \prod_{v \in S} c_v^\degv \bigr)
      H(x)^{-\eps}
    }
    \pmm.
  \]
  De plus, pour toute famille de réels \( (\lambda_v)_{v\in S} \), on définit
  \[
    E'(\eps, S, \lambda)
    =
    \set*{
      x \in A
      \text{ tel que }
      d_v(x)
      \le
      c_v H(x)^{-\lambda_v \eps}
      \quad \forall v \in S
    }
    \pmm.
  \]
  On note enfin \( \parts M \) l'ensemble des parties finies de \( M \) et on
  suppose qu'il existe une fonction \( g \colon \R_{>0} \times \parts M \to \N
  \) telle que pour tout \( \eps > 0 \), tout \( S \in \parts M \) et toute
  famille \( (\lambda_v)_{v\in S} \) de réels positifs telle que \( \sum_{v\in
      S} \lambda_v \degv = 1 \), on ait
  \[
    \card E'(\eps, S, \lambda)
    \le
    g(\eps, S)
    \pmm.
  \]
  Alors, pour tout \( \eps > 0 \) et tout \( S \in \parts M \) on a
  \[
    \card E(\eps, S)
    \le
    \sum_{T \in \parts S}
    \binom{2\card T - 1}{\card T - 1}
    g(\eps/2, T)
    \pmm.
  \]
\end{lem}

Avant de passer à la preuve, expliquons comment nous choisirons les différents
paramètres. L'ensemble \( A \) sera un sous-groupe de rang fini (voire de type
fini) de \( \va(\Qbar) \) et \( H \) la hauteur induite sur cet ensemble par
le choix d'un plongement projectif de \( \va \). L'ensemble \( M \) sera
l'ensemble des places d'un corps de nombres \( \cdn \) suffisamment gros et en
chaque place \( \degv = [\cdn_v : \Q_v] / [\cdn : \Q] \) est le degré local
divisé par le degré global. En revanche, le choix des constantes \( c_v \)
dépendra de l'application envisagée. On reconnaît la condition sur la famille
\( \lambda \), et la fonction \( g \) est alors donnée par l'un des énoncés de
la section précédente.

\begin{proof}

\end{proof}


\cleardoublepage
\endinput

% vim: spell spelllang=fr
