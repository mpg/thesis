% !TEX root = main.tex

\chapter{Introduction} \label{chap:intro}

\section{Aperçu du problème}

Soit \( \va \) une variété abélienne définie sur un corps de nombres \( \cdn
\). Une fois \( \va \) plongée dans un espace projectif \( \projd \), on
dispose de deux notions héritées de l'espace projectif ambiant : une hauteur,
et (pour tout place \( v \) de \( \cdn \)) une distance \( v \)-adique
(définie en section~\vref{sec:distv}) entre un point et une variété.  Dans ces
conditions, \bsc{Faltings} a démontré le théorème d'approximation suivant
\cite[\eng{Theorem~II}]{faldaav}.

\begin{thm}[\bsc{Faltings}] \label{t:fal2}
  Soit \( \avar \) une sous-variété quelconque de \( \va \), \( v \)
  une place de \( \cdn \), et \( \eps > 0 \). Il n'existe qu'un nombre fini de
  points \( x \) dans \( \va(\cdn) \) tels que
  \begin{equation} \label{e:has}
    0
    <
    \distv x \avar
    \le
    \hautm{} x ^{-\eps}
    \pmm.
  \end{equation}
\end{thm}

Comme de nombreux énoncés de géométrie diophantiennes, ce résultat n'est
malheureusement pas effectif au sens suivant : on ne voit à l'heure actuelle
pas de moyen de borner la hauteur des points satisfaisant à l'hypothèse
d'approximation \eqref{e:has}. Ainsi que le fait remarquer \bsc{Faltings}
dans l'introduction de son article : « \eng{As far as I can see, everything
  here is ineffective beyond hope.} »

Il semble néanmoins raisonnable de vouloir majorer le nombre de points
rationnels satisfaisant à \eqref{e:has} (que nous appellerons à l'occasion
les approximations exceptionnelles).  Pour les courbes
elliptiques, ce travail a été accompli par \bsc{Farhi} \cite[chap.~2]{farhith}
et \cite{faraetr}.

L'objet du présent mémoire est de généraliser ces résultats en dimension
supérieure, c'est-à-dire d'obtenir, autant que possible, une version
quantitative explicite du théorème d'approximation de \bsc{Faltings}.

\medskip

Plus précisément, ce type d'énoncé quantitatif est généralement obtenu en
combinant une inégalité à la \bsc{Vojta} et une inégalité à la \bsc{Mumford},
dont nous rappelons brièvement des énoncés possibles, formulés de façon
générique avec une condition (C) qui peut être par exemple l'hypothèse
d'approximation \eqref{e:has} ci-dessus, ou une autre condition pour
l'ex-conjecture de \bsc{Mordell-Lang}.

Les deux inégalités s'énoncent dans l'espace de \bsc{Mordell-Weil} de \( \va \)
muni de la forme quadratique donnée par la hauteur normalisée de
\bsc{Néron-Tate}.

L'inégalité de \bsc{Vojta} affirme qu'il n'existe pas de suite \( x_1, \dots,
  x_m \) de points satisfaisant simultanément à la condition (C) et aux trois
conditions suivantes :
\begin{enumthm}
  \item \( \hautn{x_1} > \alpha \) ; \label{i:grand}
  \item \( \cos(x_i, x_j) > 1 - \beta \) pour tous \( i \) et \( j \) ;
    \label{i:proche}
  \item \( \hautn{x_i} > \gamma \hautn{x_{i-1}} \) pour \( i > 1 \) ;
\end{enumthm}
où l'angle est relatif à la structure euclidienne de l'espace. Nous
appellerons \emph{cône tronqué} une partie de l'espace délimitée par les
conditions \vref{i:grand} et \vref{i:proche}. Il est clair que l'espace privé
d'une boule de rayon \( \sqrt{\alpha} \) peut être recouvert par un nombre
fini de tels cônes tronqués dès qu'il est de dimension finie (ce qui est le
cas si on se place sur un corps de nombres). L'inégalité de \bsc{Vojta} assure
qu'il n'y a qu'un nombre fini de points sous la condition (C) dans chaque
cône, et permet donc de conclure à la finitude de l'ensemble des points
satisfaisant (C).

L'inégalité de \bsc{Mumford} peut s'énoncer de façon très similaire. Elle dit
qu'il n'existe pas de suite \( x_1, \dots, x_m \) de points satisfaisant
simultanément à la condition (C) et aux trois conditions suivantes :
\begin{enumthm}
  \item $\hautn{x_1} > \alpha'$ ;
  \item $\cos(x_i, x_j) > 1 - \beta'$ pour tous $i$ et $j$ ;
  \item $\abs{ \hautn y - \hautn x } < \delta$.
\end{enumthm}
Utilisée conjointement avec l'inégalité de \bsc{Vojta}, et à condition que les
constantes apparaissant dans ces deux inégalités soient effectives, elle
permet de majorer le nombre de points dans chaque cône tronqué, donc le nombre
total de point (modulo un résultat, assez indépendant, de décompte des «
petits » points).

La démonstration de \bsc{Faltings} consiste précisément à démontrer une
inégalité de \bsc{Vojta}, non effective, qui suffit à assurer la finitude.
L'objectif consiste donc à rendre effective cette inégalité de \bsc{Vojta} et
à lui adjoindre une inégalité de \bsc{Mumford}, elle aussi effective.


\section{Relation avec d'autres énoncés}

\subsection{Le théorème de \bsc{Roth}}

Le théorème de \bsc{Roth} \cite{daroraan}, dans sa version étendue aux places
quelconques par \bsc{Ridout} \cite{ripgtsrt}, peut s'énoncer de la façon
suivante.

\begin{thm}[\bsc{Roth}]
  Soit \( \xi \in \Qbar \) un algébrique. Soient par ailleurs \( \cdn \) un
  corps de nombres et \( v \) une place de \( \cdn \), étendue de façon
  arbitraire à \( \cdn(\xi) \). Pour tout \( \eps > 0 \), il n'existe qu'un
  nombre fini de points \( x \in \cdn \) tels que
  \begin{equation}
    \av{x - \xi}
    <
    \hautm2 x ^{-2-\eps}
    \pmm.
  \end{equation}
\end{thm}

Le lien avec le théorème qui nous occupe ici est clair : on passe de l'un à
l'autre en remplaçant \( \xi \) par \( \avar \) et \( \cdn = \aff1(\cdn) \)
par \( \va(\cdn) \), la distance locale étant bien sûr représentée par \(
  \av{x - \xi} \). Le théorème~2 de \cite{faldaav} est donc aux variétés
abéliennes ce que le théorème de \bsc{Roth} est à la droite.

Très rapidement après la démonstration initiale de \bsc{Roth}, on a su établir
des versions quantitatives du théorème. Plus précisément, la démonstration de
\bsc{Roth} consiste en un fait qu'on peut appeler, anachroniquement, une
inégalité à la \bsc{Vojta} : aucune suite \( x_1, \dots, x_m \)
d'approximations exceptionnelles n'est telle que \( \hautm{}{x_1} > c_1 \) et
\( \hautm{}{x_i} > c_2 \cdot \hautm{}{x_{i-1}} \) pour tout \( i > 1 \).

Pour établir une version quantitative du théorème de \bsc{Roth}, il a fallu
expliciter une valeur admissible des constantes \( c_1 \) et \( c_2 \), d'une
part, et d'autre part lui adjoindre une inégalité que j'appellerai encore
anachroniquement à la \bsc{Mumford}, disant qu'il existe une constante \( c_3
\) telle que deux approximations exceptionnelles \( x \) et \( y \), de
hauteur assez grande, satisfont toujours \( \hautm{} x > c_3 \hautm{} y \).
Dans le cas du théorème de \bsc{Roth}, ceci découle immédiatement de
l'inégalité de la taille. La conjonction de ces deux inégalités donne
clairement un décompte des approximations exceptionnelles de hauteur assez
grande.

Rappelons aussi que le théorème de \bsc{Roth} a été l'aboutissement d'une
longue série de théorèmes d'approximations moins précis, en ce sens que
l'exposant optimal \( 2+\eps \) n'était pas atteint. Ce série a débuté avec le
théorème de \bsc{Liouville}. Dans le cadre des espaces projectifs
l'équivalent de l'inégalité de \bsc{Liouville} peut s'énoncer ainsi :

\begin{prop}
  Soient \( \avar \) une sous variété de \( \projd \) de degré \( \adeg \) et
  \( x \in \projd(\Qbar) \) un point algébrique. Si \( x \) n'appartient pas à
  \( \avar \), on a
  \begin{equation}
    \distv x \avar
    \ge
    c(\avar) \cdot \hautm{}{x}^{-\adeg}
    \pmm.
  \end{equation}
\end{prop}

Nous établirons (proposition~\vref{p:liouville}) une version explicite de cette
inégalité qui, outre son intérêt propre, joue un rôle crucial dans la preuve de
nos inégalités à la \bsc{Mumford}.

On peut relever les différences suivantes entre cet énoncé et le théorème
d'approximation de \bsc{Faltings} :
\begin{itemize}
  \item l'inégalité de \bsc{Liouville} prend place dans un espace projectif,
    il n'y pas besoin d'une variété abélienne ambiante ;
  \item elle n'admet pas d'exceptions, contre un nombre fini d'approximations
    exceptionnelles pour \bsc{Faltings} ;
  \item en revanche, l'exposant de \( \hautm{} x \) est nettement plus mauvais.
\end{itemize}
Comme on le verra en \vref{s:siegel}, il est intéressant pour certaines
applications de disposer d'un exposant strictement inférieur à \( 1 \).


\subsection{L'ex-conjecture de \bsc{Mordell-Lang}}

Dès 1922, \bsc{Mordell} avait conjecturé l'énoncé suivant.

\begin{thm}[\bsc{Faltings}, ex-conjecture de \bsc{Mordell}]
  Soit \( C \) une courbe projective lisse de genre \( g \ge 2 \), définie sur
  un corps de nombre \( \cdn \). L'ensemble \( C(\cdn) \) des points
  rationnels de \( C \) est fini.
\end{thm}

Ce résultat a d'abord été prouvé par \bsc{Faltings} en 1983 comme conséquence
d'une conjecture de \bsc{Shafarevitch} \cite{faldaav}. La preuve fait
intervenir des espaces de modules de variétés abéliennes, et c'est à cette
occasion que \bsc{Faltings} a introduit la hauteur qui porte désormais son
nom, sur ces espaces. Néanmoins, cette preuve reste assez éloignée des
méthodes traditionnelles de l'approximation diophantienne.

Une preuve totalement indépendante a été publiée en 1991 par \bsc{Vojta}
\cite{vojstcc}. Elle se rapproche grandement des idées habituelles de
l'approximation diophantienne, en introduisant ce qu'on appelle maintenant
l'inégalité de \bsc{Vojta}. La preuve est ensuite simplifiée (« \eng{avoid[ing]
the difficult Arakelov theory in Vojta's paper} ») et étendue par
\bsc{Faltings} \cite{faldaav} pour prouver une conjecture de \bsc{Lang},
généralisant celle de \bsc{Mordell}, et qui s'énonce ainsi.

\begin{thm}[\bsc{Faltings}, ex-conjecture de \bsc{Mordell}-\bsc{Lang}]
  \label{t:fal1} ~\\
  Soit \( \avar \) une sous-variété d'une variété abélienne \( \va \), définie
  sur un corps de nombres \( \cdn \). Si \( \avar \) ne contient pas de
  translaté de sous-variété abélienne stricte, alors \( \avar(\cdn) \) est
  fini.
\end{thm}

Ceci généralise la conjecture de \bsc{Mordell}, qui correspond au cas où \(
  \avar \) est une courbe et \( \va \) sa jacobienne. Ce résultat est proche
de notre problème d'approximation dans le sens suivant : il consiste à montrer
la finitude des points rationnels \emph{sur} une sous-variété de variété
abélienne, alors que nous nous intéressons aux points \emph{proches} d'une
telle sous-variété. Il est d'ailleurs significatif que \bsc{Faltings} a prouvé
ces deux théorèmes (la conjecture de \bsc{Mordell-Lang} et celui dont une
version quantitative faite l'objet de ce mémoire cherche à rendre quantitatif)
dans le même article : une bonne partie des outils sont commun aux deux
preuves.

Une différence notable entre les deux situations est toutefois la suivante :
pour étudier les points qui sont proches d'une sous-variété, sans appartenir
à cette variété, on n'a pas besoin de supposer que celle-ci ne contient pas de
translaté de sous-groupe. En fait, le résultat reste valable même pour les
approximations d'une sous-variété abélienne.

Des versions quantitatives du théorème~\vref{t:fal1} ont été établies ensuite
en suivant la méthode de \bsc{Vojta}.  Signalons la relecture de la preuve par
\bsc{Bombieri} \cite{bommcr}, qui simplifie certains arguments en les
rapprochant de l'effectivité, et le travail de \bsc{De Diego} \cite{ddprf} sur
les familles de courbes. \bsc{Rémond} obtient, dans la lignée de travaux de
\bsc{Faltings} et \bsc{Bombieri}, une version totalement effective de
l'inégalité de \bsc{Vojta} \cite{remivds}, puis lui adjoint une inégalité à la
\bsc{Mumford}, établissant ainsi une version quantitative explicite
\cite{remdcl} de l'ex-conjecture de \bsc{Mordell-Lang}. Enfin, \bsc{Farhi}
\cite[chap.~3]{farhith} et \cite{faraptf} donne une version quantitative de
l'ex-conjecture de \bsc{Mordell}, démontrée dans un formalisme plus élémentaire
que celui de \bsc{Rémond}, et légèrement plus précise que son application
directe au cas des courbes.


\subsection{Le théorème de \bsc{Siegel} et une ex-conjecture de \bsc{Lang}}
\label{s:siegel}

% XXX citations
Le théorème de \bsc{Siegel}, démontré en 1929, affirme qu'une courbe de genre
supérieur ou égal à \( 1 \) ne possède qu'on nombre fini de points entiers. Sa
démonstration repose sur le théorème de \bsc{Roth} énoncé plus haut, et avait
été obtenu par \bsc{Siegel} avec la version faible de cet énoncé dont il
disposait en 1929.  Une généralisation du théorème a été conjecturée par
\bsc{Lang}, de façon analogue à sa généralisation de la conjecture de
\bsc{Mordell} : si \( \divi \) est un diviseur ample d'une variété abélienne
\( \va \), alors \( \va \setminus \divi \) ne possède qu'un nombre fini de
points entiers. Le théorème original s'en déduit là aussi en considérant la
courbe dans sa jacobienne (à cette différence qu'ici la courbe peut être sa
jacobienne, dans le cas \( g = 1 \)).

Cette conjecture de \bsc{Lang} est en fait un corollaire du théorème
d'approximation de \bsc{Faltings} : on remarque que la hauteur (relative à \(
  \divi \)) d'un point entier \( x \) est essentiellement donné par le produit
des inverse des distances \( v \)-adiques de \( x \) à \( \divi \) quand \( v
\) parcourt les places archimédiennes de \( \cdn \). Or ces distances sont
minorées par \( \hautm{}{x}^{-\eps} \) pour tout \( \eps>0 \), sauf pour un
nombre fini de points. Ceci est bien sûr contradictoire dès que \( \eps < 1
\), ce qui prouve que seules les approximations exceptionnelles de \( \divi \)
peuvent donner des points entiers. Dénombrer ces dernières donne donc
immédiatement une version quantitative de cette ex-conjecture de \bsc{Lang}.
C'est pour ce type d'applications qu'il devient essentiel dans l'énoncé
d'approximation de pouvoir prendre \( \eps \) petit, au moins inférieur à \( 1
\), alors que l'exposant de l'inégalité de \bsc{Liouville} ne suffit en aucun
cas.

Signalons qu'on connaît des versions quantitatives du théorème de \bsc{Siegel}
(\todo??). Par contre, à ma connaissance, la seule démonstration connue de sa
généralisation est celle de \bsc{Faltings} : en particulier on ne connaît pas
de version quantitative de cette ex-conjecture de \bsc{Lang}.

Enfin, en un sens, on peut avoir l'impression que le théorème de
\bsc{Faltings} (ex-conjecture de \bsc{Mordell-Lang}) rend obsolète le théorème
de \bsc{Siegel} et sa généralisation conjecturée par \bsc{Lang} : en effet,
n'avoir qu'un nombre fini de point rationnels implique de n'avoir qu'un nombre
fini de points entiers. En fait, les énoncés de type \bsc{Siegel} conservent
un intérêt essentiellement grâce à la restriction « ne pas contenir de
sous-variété abélienne » dans \bsc{Mordell-Lang} : si on prend le cas extrême
d'une variété abélienne, il est clair que (sur un corps de nombres pas trop
petit) elle possède une infinité de points rationnels, alors qu'elle n'a qu'un
nombre fini de points entiers.


\subsection{Formes linéaires de logarithmes abéliens}

\todo (Ou pas, finalement ?)


\section{Énoncés des résultats principaux}

\todo (À écrire.)


\section{Stratégie générale}

Il s'agit essentiellement d'employer la méthode de \bsc{Vojta}, en s'inspirant
des travaux de \bsc{Rémond} \cite{remivds,remivg,remdcl}, de \bsc{Farhi}
\cite[chap.~2]{farhith}, et de la preuve originale de \bsc{Faltings}
\cite{faldaav}.  Dans les grandes lignes, la preuve consistera donc à établir
une version explicite de l'inégalité à la \bsc{Vojta} obtenue par
\bsc{Faltings} et à lui adjoindre une inégalité à la \bsc{Mumford}, elle aussi
explicite.

Les arguments utiliseront un formalisme simple : plongements, coordonnées et
polynômes plutôt que fibrés (métrisés) et sections globales. Les outils
techniques essentiels sont ceux de la théorie de l'élimination tels que
rappelés par exemple dans \cite[chap.~5 à~7]{nesphilnm}.


\subsection {Inégalité de \bsc{Vojta}}

\todo (À écrire.)


\subsection{Inégalité de \bsc{Mumford}}

\todo (À écrire.)


\clearpage % XXX provisoire
\section{Définitions et notations}

\subsection{Formes homogènes et multihomogènes}

On notera généralement \( \vp = (\vp[0], \dots, \vp[n]) \) un groupe de
variables. Si \( \ip \in \N^{n+1} \) est un multiindice, on notera
\( \vp^\ip = \vp[0]^{\ip[0]} \cdots \vp[n]^{\ip[n]} \). On introduit également
les notations \( \ip! = \ip[0]! \cdots \ip[n]! \) et \( \lgr\ip = \ip[0] +
  \dots + \ip[n] \) ; on note alors \( \binom{\lgr\ip}{\ip} =
  \frac{\lgr\ip!}{\ip!} \) le coefficient multinomial. Pour les dérivées, on
utilisera les notations
\begin{equation}
  \frac{\partial^\ip}{\partial \vp^\ip}
  =
  \frac{ \partial^{\ip[0]} }{ \partial \vp^{\ip[0]} }
  \cdots
  \frac{ \partial^{\ip[n]} }{ \partial \vp^{\ip[n]} }
  \quad\text{et}\quad
  \der[\ip]
  =
  \frac1{\ip!} \frac{\partial^\ip}{\partial \vp^\ip}
  \pmm.
\end{equation}

Introduisons maintenant quelques notations pour les formes multihomogènes ;
nous travaillerons principalement avec des formes sur des puissances d'espaces
projectifs, nous nous limitons donc à ce cas pour alléger un peu. On notera
\( \vmp = (\vmp[1], \dots, \vmp[m]) \) un groupe de groupe de variables, où \(
  \vmp[i] = (\vmp[i][0], \dots, \vmp[i][n]) \). On étend naturellement les
notations projectives : pour \( \imp \in \N^{(n+1)m} \) on note \( \vmp^\imp =
  (\vmp[1])^{\imp[1]} \cdots (\vmp[m])^{\imp[m]} \) ; on procède de même pour
les factorielles et les dérivées (divisées). Par ailleurs, on note \( \vlg\imp
  = (\lgr{\imp[1]}, \dots, \lgr{\imp[m]}) \) le vecteur des longueurs de \(
  \ip \) et \( \lgt\imp = \sum_{i=1}^m \lgr{\imp[i]} \).


\subsection{Normes locales}

À chaque place \( v \) d'un corps de nombres \( \cdn \), on associe la
valeur absolue qui prolonge une des valeurs absolues standard de \( \Q \),
autrement dit \( \av p = p^{-1} \) si \( v \) divise un premier \( p \)
et \( \av 2 = 2 \) si \( v \) est archimédienne.

Si \( v \) est une place finie de \( \cdn \), on définit la norme \(
  v \)-adique d'une famille finie comme le maximum des valeurs absolues
\( v \)-adiques de ses éléments. Si \( v \) est archimédienne, on
utilisera les normes usuelles suivantes :
\begin{align}
  \nv\infty x & = \max_i{ \av{ x_i } } &
  \nv1 x & = \sum_i{ \av{ x_i } } &
  \nv2 x & = \Bigl( \sum_i{ \av{ x_i }^2 } \Bigr)^{1/2}
  \pmm.
\end{align}
On définit alors la norme d'un polynôme comme celle de la famille de ses
coefficients. Pour les polynômes (multi)homogènes, on sera également amené à
utiliser la variante suivante de la norme euclidienne :
\begin{equation}
  \nv\nh P
  =
  \bigl(
  \sum_\alpha \av{P_\alpha}^2
  \cdot {\textstyle \binom{\delta}{\alpha}^{-1} }
  \bigr)^{1/2}
  \quad\text{où }
  P = \sum_\alpha P_\alpha X^\alpha
  \text{ et }
  \delta = \deg P
\end{equation}
avec les notations de la sous-section précédente concernant les puissances et
coefficients multinomiaux. Cette norme est utilisée par exemple
dans~\cite{nesphilnm} p. 111, dont le lemme~3.3 (et sa démonstration) rappelle
ou établi quelques comparaison entre cette norme et d'autres quantités
usuelles.

Aux places archimédiennes, on dispose également de deux notions de mesure pour
les polynômes. La première est la classique mesure de Mahler, définie par
\begin{equation}
  \ln \mahler P
  =
  \int_0^1 \Diff t_1 \dots \int_0^1 \Diff t_n
  \ln \av{ P(\expb^{2\pi i t_1}, \dots, \expb^{2\pi i t_n}) }
\end{equation}
si \( P \neq 0 \) et \( \mahler(0) = 0 \). La seconde est une variante
introduite par \bsc{Philippon} pour les formes multihomogènes en intégrant sur
un produit de sphères et non de cercles ainsi qu'en ajoutant un facteur
correctif. Plus précisément :
\begin{equation}
  \log \mespph F
  =
  \bigl(
    \int_{S_{1}\times\cdots\times S_{l}}
    \log\av F\ \mu_1 \wedge \cdots \wedge \mu_l
  \bigr)
  + \sum_{k=1}^l \delta_k \cdot \gamma_{n_k}
  \pmm,
\end{equation}
où \( S_k = \{ u \in \C^{n_k},\ \normeuc u = 1 \} \) est la sphère de
dimension \( 2n_k-1 \) plongée dans \( \C^{n_k} \) muni de sa structure
hermitienne usuelle, \( \mu_k \) désigne la mesure sur \( S_k \) invariante
par l'action du groupe unitaire et normalisée par \( \mu_k(S_k) = 1 \), et
enfin
\begin{equation} \label{e:def-gamma-n}
  \gamma_n = \frac12 \sum_{q=1}^{n} \frac1q
  \pmm.
\end{equation}
On renvoie
à~\cite{nesphilnm} p.~86 pour un rappel des propriétés essentielles.

On rappelle au passage la définition du nombre de \bsc{Stoll}
\begin{equation} \label{e:def-stoll}
  \stoll n
  =
  \sum_{i=1}^{n} \gamma_i
  =
  \sum_{i=1}^{n} \sum_{j=1}^{i} \frac1{2j}
\end{equation}
et les majorations élémentaires
\begin{align}
  \gamma_n & \le \frac{1 + \ln n}2
  &
  \stoll n & \le \frac{(n+1) \ln(n+1)}2 \le n \ln(n+1)
\end{align}
obtenues en comparant avec des intégrales, par exemple.

On notera \( \nv\star\truc \) la norme locale en \( v \) correspondant à
la norme \( \star \) si \( v \) est archimédienne et à celle du maximum
sinon. De même, \( \mvp{} \) et \( \mvm{} \) désignent respectivement la
mesure de \bsc{Philippon} et de \bsc{Mahler} si \( v \) est archimédienne
et la norme du maximum (de la famille des coefficients) sinon.
Si \( \mathcal F = (F_1, \dots, F_p) \) est une famille de polynômes, on
définit de plus sa norme \( \nnv\star{ \mathcal F } \) comme la norme \(
  \nv\star\truc \) du vecteur des \( \nv\star{ F_i } \).

\subsection{Hauteurs}

On définit la hauteur additive d'une famille finie d'éléments de \( \cdn \),
d'un polynôme ou d'une forme (multi)homogène sur \( \cdn \) comme
\begin{equation}
  \hautl\star\truc
  =
  \sum_{v}
  \degv
  \ln \nv\star\truc
  \pmm,
\end{equation}
où \( \degv = [\cdn_v : \Q_v] / [\cdn : \Q] \),
la somme est prise sur l'ensemble des places de \( \cdn \) et \( \star \)
désigne l'une des normes ou mesures introduites ci-dessus.  On note également
\( \Hautm\star = \exp(\Hautl\star) \) la hauteur multiplicative.

La hauteur d'une famille, ainsi définie, est invariante par multiplication de
cette famille par un scalaire, et ne dépend pas non plus du corps de nombres.
Ceci permet de définir la hauteur d'un point de \( \projd(\Qbar) \) comme
celle de ses coordonnées homogène.

Si \( V \) est une sous-variété de \( \projd \) on peut lui associer des
formes de \bsc{Chow} associées à certains indices de la façon rappelée dans
\cite{nesphilnm} au début du chapitre~6. Nous ne considérerons généralement
que la forme d'indice \( (1, \dots, 1) \), que nous noterons \( \chow V \).
Cette forme est bien définie à multiplication par un scalaire près, la
définition \( \hautl\star V = \hautl\star{ \chow V } \) a donc un sens.

Par ailleurs, si l'on fixe un plongement \( \vaemb \colon \va \to \projd \),
on a des notions de hauteurs sur \( \va(\Qbar) \) associées à ce plongement.
Si de plus ce plongement est associé à un fibré \( \fibre \) symétrique, on
peut définir de façon classique une hauteur normalisée \( \Hautn \) (dite de
\NT), qui dépend encore du fibré \( \fibre \) mais plus du choix d'un
plongement particulier associé à ce fibré, ni de la norme ou mesure utilisée
aux places archimédiennes pour définir la hauteur projective.

Cette hauteur normalisée induit une forme quadratique définie positive sur
l'espace de \MoW de \( \va \), défini comme
\(
  \MW\Qbar = \va(\Qbar) \otimes_\Z \R
\).
Plus précisément, la partie \( \MW\cdn = \va(\cdn) \otimes_\Z \R \)
définie sur un corps de nombres \( \cdn \) est un espace euclidien de
dimension (finie) égale au rang de \( \va \) sur \( \cdn \). Par extension, si
\( \Gamma \) est un sous-groupe de \( \va(\Qbar) \), on note \( \MW\Gamma \)
le sous-espace de \( \MW\Qbar \) engendré par l'image de \( \Gamma \) ; cet
espace est de dimension finie si et seulement si \( \Gamma \) est de rang
fini.


\subsection{Plongement projectif de la variété abélienne} \label{sec:vaemb}

On supposera en général fixé un plongement \( \vaemb \colon \va \to \projd \)
projectivement normal et associé à un fibré symétrique.  On introduit alors
quelques constantes et notations associées à ce plongement.  L'existence de
constantes possédant les propriétés voulues est établie dans
l'annexe~\vref{chap:plong-mm} pour tout plongement satisfaisant les hypothèses
ci-dessus ; des valeurs explicites y sont éventuellement données pour certains
types de plongement.

Pour commencer, on note \( \htcmp \) une constante réelle telle que pour toute
sous-variété \( V \) de \( \va \) on ait
\begin{equation} \label{e:comp-h-hn-var}
  \abs{ \hautl2 V - \hautn V }
  \le
  \htcmp (\dim V + 1) \deg V
\end{equation}
où \( \Hautn \) est la hauteur normalisée telle que définie dans \cite{phiha1}
pour les variétés, dont les principales propriétés sont résumées par la
proposition~9 de cette même référence. En particulier on a
\begin{equation} \label{e:comp-h-hn}
  \abs{ \hautl2x - \hautn x } \le \htcmp
  \qquad \forall x \in \va(\Qbar)
  \pmm.
\end{equation}

On introduit une paire de constantes \( \hmclab \) et \( \hlclab = \ln \hmclab
\), dont la première se décompose en \( \hmclab = \prod_v \hmclab*^\degv
\) où les \( \hmclab* \) sont des réels tous supérieurs et presque tous égaux
à \( 1 \), possédant les propriétés suivantes.

\begin{enumerate}
  \item \label{i:clab}
    Il existe un entier \( \nclmaps \) et un ensemble \( \clmaps \) de \(
      \nclmaps \) ouverts recouvrant \( \va^2 \), possédant la propriété
    suivante. Pour tout couple d'entiers \( (a, b) \) n'admettant que \( 2
    \) et \( 3 \) comme diviseurs premiers, il existe une famille \(
      (\rmclab{a}{b}{\clmap})_{\clmap \in \clmaps} \) de \( (n+1)
    \)-uplets de formes telle que :
    \begin{enumerate}
      \item la famille \( (\rmclab{a}{b}{\clmap}[\ind])\indrange \) représente
        le morphisme \( (x, y) \mapsto ax - by \) dans le plongement \(
          \vaemb \) sur (l'image de) \( \clmap \) ;
      \item chaque forme \( \rmclab{a}{b}{\clmap}[\ind] \) est bihomogène de
        bidegré \( (2a^2, 2b^2) \) ;
      \item pour tout place \( v \) et toute carte \( \clmap \), on a
        \begin{equation} \label{e:clab-norm}
          \nnv1{ \rmclab{a}{b}{\clmap} } \le \hmclab*^{a^2 + b^2}
          \pmm;
        \end{equation}
      \item pour tout couple de points \( (x, y) \in \va^2 \) et tout place
        \( v \) il existe une carte \( \clmap \in \clmaps \) telle que
        \begin{equation} \label{e:clab-loc}
          \frac{
            \nv1{ \rmclab{a}{b}{ \clmap }(x, y) }
          }{
            \nv1 x ^{2a^2} \nv1 y ^{2b^2}
          }
          \ge
          \hmclab*^{-(a^2 + b^2)}
          \pmm.
        \end{equation}
    \end{enumerate}
  \item \label{i:addsub}
    Il existe une famille \( F \) de formes homogènes de degré \( 2 \)
    représentant globalement le morphisme \( (x, y) \mapsto (x+y, x-y) \) dans
    le plongement \(
      \va^2
      \stackrel{\vaemb^2}{\longrightarrow}
      (\projd)^2
      \stackrel{s}{\longrightarrow}
      \proj{n^2 + 2n}
    \) (où la seconde flèche est le plongement de \bsc{Segre}) telle que
    \begin{enumerate}
      \item \( \nnv1 F \le \hmclab* \) ;
      \item Pour tout \( z \in s \circ \vaemb^2 (\va) \) on a
        \begin{equation} \label{e:addsub-loc}
          \frac{
            \nv2{ F(z) }
          }{
            \nnv2 F \nv2 z ^2
          }
          \ge
          \hmclab*^{-1}
        \end{equation}
        en toute place \( v \).
    \end{enumerate}
\end{enumerate}

\begin{rem} \label{r:vaemb}
  \worknote{On pourrait en fait dissocier \( \hmclab \) en deux (les deux
    points ci-dessus). À voir une fois les relations réellement établies.}
  L'annexe~\vref{chap:plong-mm} établit en fait des relations entre \( \htcmp
  \) et \( \hlclab \) (ainsi que \( \hautl2{ \chow\va } \)) ; on formulera
  néanmoins les énoncés suivants avec une dépendance explicite en chacune de
  ces constantes pour des raisons de clarté et d'optimalité pour les cas où
  l'une serait mieux contrôlée que les autres.

  Par ailleurs, dans tous les cas considérés, on aura \( n \ge 2 \) et \(
    N \ge 3 \), et le degré de (l'image de) \( \va \) dans le plongement sera
  au moins \( 3 \). On exploitera ces hypothèses pour simplifier quelques
  estimations.
\end{rem}


\subsection{Distances et conditions d'approximations}
\label{sec:distv}

Pour chaque place \( v \) de \( \cdn \), on pose
\begin{equation}
  \distv x y
  =
  \frac{ \nv2{ x \wedge y } }{ \nv2 x \nv2 y }
  \pmm.
\end{equation}
Cette définition correspond, pour les points, à la distance d'indice \( 1 \)
utilisée dans \cite{phidg} et \cite{jadotth}. Cette dernière référence
(notamment lemme~3.2 \textit{(ii)}, page 51) montre qu'il s'agit bien d'une
distance au sens usuel du terme (ultramétrique si \( v \) l'est). Par
ailleurs, il est évident à partir de la définition que la distance est
toujours majorée par \( 1 \).

Pour la distance entre un point et une variété, on s'écarte par contre
légèrement des références citée, en posant
\begin{equation}
  \distv x V
  =
  \min_{y \in V(\Cv)} \distv x y
  \pmm.
\end{equation}
Les deux notions sont en fait très comparables, voir la section suivante. La
notion qu'on vient de définir bénéficie par contre de la propriété suivante :
si \( V \subset V' \) alors \( \distv x {V'} \le \distv x V \), qu'on utilisera
couramment pour se ramener au cas où \( V \) est une hypersurface. Évidemment,
on a \( \distv x V = 0 \Leftrightarrow x \in V(\Cv) \).

\medskip

Soient \( \cdn \) un corps de nombres, \( \placesapx \) un ensemble fini de
places de \( \cdn \) et \( (\wtapx)\placerange \) une famille de réels telle
que \( \sum\placerange \wtapx \degv = 1 \). On considère de plus une famille
de constantes \( (c_v) \), un réel \( \expapx > 0 \) et on associe à toutes
ces données la condition d'approximation suivante.
\begin{equation} \label{e:ha-sys}
  \distv x \avar
  \le
  c_v
  \hautm2{x}^{-\wtapx\expapx}
  \quad \forall v \in \placesapx
  \pmm.
\end{equation}
Si \( \cdn' \) est une extension finie de \( \cdn \), on note \( \placesapx'
\) l'ensemble des places \( w \) de \( \cdn' \) qui sont au-dessus d'un
élément \( v \) de \( \placesapx \) et on pose \( \wtapx[w] = \wtapx \)
ainsi que \( c_w = c_v \). On a toujours
\begin{equation}
  \sum_{w \in \placesapx'}
  \wtapx[w]
  \frac{ [\cdn'_w : \Q_v] }{ [\cdn' : \Q] }
  =
  1
\end{equation}
et il est clair que la condition~\eqref{e:ha-sys} est équivalente à
\begin{equation} \label{e:ha-sys-ext}
  \distv[w] x \avar
  \le
  c_w
  \hautm2{x}^{-\wtapx[w]\expapx}
  \quad \forall w \in \placesapx'
  \pmm.
\end{equation}
Autrement dit, une condition comme~\eqref{e:ha-sys} a un sens naturel pour
tout \( x \in \projd(\Qbar) \).

Enfin, on considérera aussi des conditions d'approximation de la forme
\begin{equation} \label{e:ha-prod}
  \prod\placerange
  \distv x \avar ^\degv
  \le
  c
  \hautm2{x}^{-\expapx}
  \pmm.
\end{equation}
Il est clair qu'une telle condition a également un sens sur \( \Qbar \). Par
ailleurs, si \( c = \prod\placerange c_v \), la condition~\eqref{e:ha-sys}
implique immédiatement~\eqref{e:ha-prod}. Autrement dit, un énoncé formulé
avec une condition de type~\eqref{e:ha-prod} est plus fort que le même énoncé
avec une condition de type~\eqref{e:ha-sys}. Cependant, on verra
(section~\vref{sec:ha-prod}) que l'énoncé plus faible implique en fait l'énoncé
fort, quitte à introduire le cardinal de \( \placesapx \) dans le décompte.

\medskip

Vu que toutes les conditions considérées ont un sens sur \( \Qbar \), dans les
chapitres~\vref{chap:vojta} et~\vref{chap:mumford} on se préoccupera assez peu
du corps de nombres sur lequel on travaille. La notation \( \cdn \) désignera
n'importe quel corps de nombres « assez grand », c'est-à-dire contenant un
corps de définition de tous les objets en jeu, mais aussi, par exemple, des
racines carrées de coefficients multinomiaux quand c'est utile pour travailler
avec un plongement de \bsc{Veronese} remodelé.  En particulier, tous les
objets considérés (points, variétés, etc.) sont supposés définis sur \( \Qbar
\) sauf mention explicite du contraire.


\section{Compléments sur les distances} \label{sec:distv-cmp}

On commence par établir une expression simple de la distance entre un point et
un hyperplan, cas qui sera fondamental pour l'inégalité de \bsc{Vojta} par
exemple, puis on étudie les liens entre la distance définie ci-dessus et celle
utilisée notamment dans~\cite{phidg}. Certains des résultats établis ici ne
sont pas utiles pour le reste de la thèse\footnote{Plus précisément, les seuls
  résultats utilisés sont la proposition~\vref{p:dv-hp}, le
  lemme~\vref{l:dv-common-i} et la proposition~\vref{p:dv-p2alg-hs}.} mais sont
inclus afin de donner une vision assez complète des relations entre ces deux
notions.

Par ailleurs, dans toute cette section, contrairement au reste du texte, les
objets (points, variétés) considérés ne sont pas supposés définis sur \( \Qbar
\), les résultats sont valables pour des objets définis sur \( \Cv \). En
revanche, les résultats d'existence (fait~\vref{f:closest-point} en
particulier) ne permettent en aucun cas de garantir le caractère algébrique
des objets obtenus.

\begin{prop} \label{p:dv-hp}
  Si \( E \) est un hyperplan de \( \projd \), d'équation \( L \), on a
  \begin{equation}
    \distv x E
    =
    \frac{ \av{L(x)} }{ \nv2 L \nv2 x }
    \quad \forall x \in \projd(\Cv)
    \pmm.
  \end{equation}
\end{prop}

\begin{proof}
  On commence par se ramener au cas \( L = X_0 \). Pour cela, si \( v \)
  est ultramétrique, on commence par choisir un coefficient de \( L \) de
  valeur absolue maximale : quitte à permuter les coordonnées, on peut
  supposer que c'est celui d'indice \( 0 \). En divisant par ce coefficient,
  on se ramène au cas \( L = X_0 + l_1 X_1 + \dots + l_n X_n \) avec \(
    \av{ l_i } \le 1 \) entiers.  Il est alors facile de voir que le
  changement de coordonnées donné par \( x'_0 = L(x) \) et \( x'_i = x_i \)
  pour \( i \ge 1 \) est une isométrie (son inverse est également à
  coefficients de module inférieur ou égal à \( 1 \)), de sorte qu'on s'est
  bien ramené au cas \( L = X_0 \) sans modifier les quantités apparaissant
  dans chacun des deux membres, ni perdre en généralité.

  Aux places archimédiennes, on plonge \( \cdn_v \) dans \( \C \) et on
  note \( \overline E \) le relevé de \( E \) dans \( \C^{n+1} \). On munit ce
  dernier de la forme hermitienne standard et on considère un vecteur unitaire
  normal à \( \overline E \), qu'on note \( a_0 \). On complète ce vecteur en
  un base orthonormée et on considère le changement de coordonnées qui envoie
  cette base sur la base canonique : étant une transformation unitaire, il ne
  modifie aucun des deux membres de l'égalité à prouver, et nous ramène bien
  au cas \( L = X_0 \).

  Par ailleurs, l'égalité qu'on cherche à montrer est banale si \( x \in
    E(\Cv) \). On peut donc supposer que ce n'est pas le cas, et même
  choisir \( (1, x_1, \dots, x_n) \) comme représentant que \( x \). Le membre
  de droite est alors égal à \( \nv2 x ^{-1} \).

  Soit maintenant \( y = (0, y_1, \dots, y_n) \in E(\Cv) \). On a
  \( x \wedge y = (-y_1, \dots, -y_n, x_1 y_2 - x_2 y_1, \dots) \), donc ce
  \( \nv2{ x \wedge y } \ge \nv2 y \), avec égalité si et seulement si \( y \)
  est colinéaire à \( (0, x_1, \dots, x_n) \).  Ainsi, pour tout \( y \in
    E(\Cv) \) on a \( \distv x y \ge \nv2 x ^{-1} \), où l'égalité est
  possible, ce qui achève la preuve.
\end{proof}

Remarquons que le distance entre points et variété définie ici ne dépend en
fait que de l'ensemble des points de la variété mais pas de la structure
géométrique (multiplicités) de celle-ci. Une autre notion de distance,
développée dans \cite{jadotth} et \cite{phidg} pour le cas projectif, et
\cite{remgdmp} pour le cas multiprojectif (qui englobe bien sûr le précédent)
prend au contraire en compte cette structure. Cette distance est définie par
\begin{equation}
  \distalgv x V
  =
  \frac{
    \mvp{\md_x f_V}
  }{
    \mvp{f_V} \nv2 x ^{\Delta(u+1)}
  }
\end{equation}
où \( f_V \) est une forme de \bsc{Chow} de \( V \), son degré est noté \(
  \Delta \) et sa dimension \( u \). Le morphisme \( \md_x \) est défini dans
les références citées (p.~88 de \cite{phidg} par exemple) ; par ailleurs, par
rapport à ces références, on considère uniquement la distance d'indice \( (1,
  \dots, 1) \). Comme pour la notion de distance précédemment définie, on a \(
  0 \le \distv x V \le 1 \) avec égalité à gauche si et seulement si \( x \in
  V \) (voir la référence précédente).

Nous appellerons cette distance \emph{algébrique}, par opposition à la
distance ensembliste définie ci-dessus. Examinons un peu les relations entre
ces deux notions. Tout d'abord, dans le cas où \( V = \set y \) est réduite à
un point, on a \( \distv x y = \distalgv x V \) d'après \cite[p. 50]{jadotth}.
Par ailleurs, dans le cas où \( V \) est une hypersurface de degré \( \Delta
\) et d'équation \( F \), la proposition~3.6 (p.~64) de cette même référence
montre que
\begin{equation}
  \distalgv x V
  =
  \frac{ \av{F(x)} }{ \mvp F \nv2 x ^\Delta }
  \pmm.
\end{equation}
Si \( V \) est en fait un hyperplan, on a \( \mvp F = \nv2 F \) d'après la
proposition~4, p.~266 de~\cite{phiha1} (reprise dans la proposition~2.3,
point~(ii), p.~23 de \cite{jadotth} sous une forme plus proche de nos
notations), de sorte que la proposition~\vref{p:dv-hp} dit en fait simplement
que \( \distalgv \truc V = \distv \truc V \) dans ce cas particulier.

Dans le cas général, on a la comparaison suivante.
\begin{fact} \label{f:closest-point}
  Pour toute variété \( V \subset \Proj^n \) et tout point \( x \in
    \Proj^n(\C_v) \), il existe un point \( y \in V(\C_v) \) tel que \(
    \Distv(x, y) \le \Distv(x, V)^{1/\deg V} e^{\dv\gamma_{n+1}} \).
\end{fact}

\begin{proof}
  Pour les places infinies, c'est la \eng{closest point property} p.~89
  de~\cite{phidg}. Pour les places finies, on constate que la preuve s'adapte,
  car le résultat crucial (lemme~5.2 de la référence citée) est en fait vrai
  aux places finies sans le dernier facteur (et en remplaçant évidemment la
  mesure par la norme du sup).
\end{proof}

Dans l'autre sens, on a le résultat suivant.
\begin{prop} \label{p:dv-p2alg}
  Pour toute variété \( V \subset \projd \) et tous points \( x \in
    \projd(\Cv) \) et \( y \in V(\Cv) \), on a
  \begin{equation}
    \distalgv x V
    \le
    \distv x y ^m
    \bigl( 2 (n+1)^{3/2} \bigr)^{(m + 3ld) \dv}
  \end{equation}
  où \( l = \dim V + 1 \), \( d = \deg V \) et \( m \) est la multiplicité de
  \( V \) en \( y \).
\end{prop}

La démonstration repose sur un développement de \( \md f \) autour de \( y
\) relativement à \( X \), qui permet de quantifier le fait que, \( \md_y f \)
étant nul, \( \md_x f \) doit être petit pour \( x \) voisin de \( y \).
De façon générale, pour écrire un développement de \bsc{Taylor} , il faudra
s'assurer de l'existence d'une carte affine \( X_i \neq 0 \) contenant à la
fois \( x \) et \( y \). On voit facilement que la non-existence d'un telle
carte affine impliquerait \( \distv x y  = 1 \), rendant le résultat banal. En
pratique, on pourra supposer sans problème que \( x \) et \( y \) sont
suffisamment proches.  Par ailleurs, il sera utile de pouvoir de plus choisir
un indice \( i \) de sorte que \( \av{x_i} \) (resp.  \( \av{y_i} \)) soit
comparable  à \( \nv2 x \) (resp. \( \nv2 y \)). Le lemme élémentaire suivant
montre que c'est possible.

\begin{lem} \label{l:dv-common-i}
  Pour chaque \( x \in \projd \), il existe \( i \in \set{0, \dots, n} \) tel
  que \( \frac{ \av{x_i} }{ \nv2 x } \ge (\frac1{\sqrt{n+1}})^\dv \). De
  plus, pour le même indice \( i \), pour tout point \( y \) satisfaisant
  \( \distv x y  \le \eps_0 \) avec \( \eps_0 < (\frac{1}{\sqrt{n+1}})^\dv \),
  on a
  \begin{equation}
    \frac{\av{y_i}}{\nv2 y}
    \ge
    \left( \frac{\sqrt{n+1}}{n+2} - \eps_0 \right)^\dv
    \pmm.
  \end{equation}
  En particulier, si \( \eps_0 < (\frac1{4\sqrt{n+1}})^\dv \) on a \(
    \frac{\av{y_i}}{\nv2 y} \ge (\frac1{2\sqrt{n+1}})^\dv \) et \( y_i \neq 0
  \).
\end{lem}

\begin{proof}
  Pour le premier point, il suffit de choisir \( i \in \set{ 0, \dots, n } \)
  maximisant \( \av{x_i} \). Afin d'alléger les notations, on supposera par
  la suite \( i = 0 \). Pour chaque \( z = (z_0, \dots, z_n) \) on notera \(
    \hat{z} = (z_1, \dots, z_n) \) le vecteur obtenu en omettant la première
  coordonnée. On a alors, vu l'hypothèse sur \( x \) et la définition de la
  distance :
  \begin{equation} \label{e:dvci-hat}
    \eps_0 \nv2 y \nv2 x
    \ge
    \nv2{y \wedge x}
    \ge
    \nv2{x_0 \hatys - y_0 \hatxs}
    \pmm,
  \end{equation}
  où la deuxième inégalité vient en remarquant que toutes les coordonnées de
  \( x_0 \hatys - y_0 \hatxs \) apparaissent également comme coordonnées de \(
    y \wedge x \).

  On traite d'abord le cas ultramétrique, par l'absurde. En effet, si on avait
  \( \av{y_0} < \nv2{\hatys} = \nv2{y} \), il viendrait \( \nv2{x_0 \hatys} >
    \nv2{y_0 \hatxs} \) et la propriété ultramétrique appliquée au dernier
  membre de \eqref{e:dvci-hat} donnerait \( \eps_0 \nv2{y}\nv2{x}  \ge
    \av{x_0}\nv2{\hatys} \), puis \( \eps_0 \ge 1 \) contrairement aux
  hypothèses.

  Pour le cas archimédien, l'inégalité triangulaire dans le membre de droite
  de \eqref{e:dvci-hat} donne
  \( \eps_0 \nv2{y}\nv2{x}  \ge \av{x_0}\nv2{\hatys} - \av{y_0}\nv2{\hatxs}
  \).  En divisant par \( \nv2{y}\nv2{x} \) puis en remarquant que \(
    \nv2{\hatxs}/\nv2{x} \le 1 \) il vient :
  \begin{align}
    \frac{\av{y_0}}{\nv2{y}}
    \ge
    \frac{\nv2{\hatxs}}{\nv2{x}}
    \cdot \frac{\av{y_0}}{\nv2{y}}
    \ge
    \frac{\nv2{\hatys}}{\nv2{y}}
    \cdot \frac{\av{x_0}}{\nv2{x}}
    - \eps_0
    \ge
    \frac{\nv2{\hatys}}{\nv2{y}}
    \cdot
    \frac{1}{\sqrt{n+1}}
    - \eps_0
    \pmm.
  \end{align}
  Notons \( t = \av{y_0}/\nv2{y} \in [0; 1] \) ; comme \( \nv2{y}^2 =
    \av{y_0}^2 + \nv2{\hatys}^2 \), on réécrit l'inégalité précédente sous
  la forme \( t \ge \sqrt{\frac{1-t^2}{n+1}}-\eps_0 \), ou encore :
  \begin{equation}
    \left( \frac{n+2}{n+1} \right) t^2
    + 2t\eps_0
    + \eps_0^2
    - \frac{1}{n+1}
    \ge
    0
    \pmm,
  \end{equation}
  qui, compte tenu de l'hypothèse sur \( \eps_0 \) et de la positivité de \( t
  \), implique :
  \begin{align*}
    t
    & \ge
    \frac{n+1}{n+2}
    \sqrt{ \frac{-\eps_0^2}{n+1}+\frac{n+2}{(n+1)^2} }
    - \left( \frac{n+1}{n+2} \right) \eps_0
    \\ & \ge
    \frac{\sqrt{n+1}}{n+2} - \eps_0
    \pmm,
  \end{align*}
  comme annoncé. Le cas particulier s'en déduit immédiatement en substituant.
\end{proof}

\begin{proof}[\proofname{} de la proposition~\vref{p:dv-p2alg}.]
  Notons pour commencer qu'on peut supposer \( \dim V \ge 1 \) (donc \( n \ge
    2 \)) car on sait que les deux distances sont égales si \( V \) est un
  point. On peut de plus supposer que \( \distv x y  < (n+1)^{-9\dv/2} \) car
  sinon la conclusion du lemme est vide, la distance étant de toutes façons
  bornée par \( 1 \). (On peut en fait supposer une majoration plus forte,
  mais celle-ci est largement suffisante.) Les hypothèses du lemme précédent
  sont satisfaites ; en l'utilisant (et quitte à renuméroter) on peut supposer
  que \( y_0 = x_0 = 1 \), \( \nv2 y \le (\sqrt{n+1})^\dv \) et \( \nv2 x \le
    (2 \sqrt{n+1})^\dv \).

  Par ailleurs, \( \nv2 y \le \nv2 x + \nv2 {y-x} \le \nv2 x + \nv2{x \wedge
      y} \), car chaque coefficient de \( x - y \) est aussi un coefficient de
  \( x \wedge y \).  En substituant \( \nv2{x \wedge y} = \distv x y  \nv2 x
    \nv2 y \) et en divisant l'inégalité obtenue par \( \nv2 x \), compte tenu
  de l'hypothèse sur la distance, il vient \( \nv2 y / \nv2 x \le 1 +
    (n+1)^{-4} \le 82/81 \).

  On reprend les notations de la section~4 (page 117) de~\cite{remgdmp}
  concernant le morphisme \( \md \).  On regarde \( \md f \) comme un polynôme
  en \( s \) à coefficients dans \( k[X] \) et on appelle \( \mathcal G \) la
  famille de ses coefficients ; chacun d'entre eux est homogène de degré \( ld
  \) et s'annule en \( y \).  Plus précisément, d'après
  \cite[prop.~3]{phitzee}, la multiplicité de \( V \) en \( y \) est le plus
  grand entier \( k \) tel que les dérivées d'ordre total \( k-1 \) de toutes
  les formes de la famille \( \mathcal G \) soient nulles en \( y \). Le
  développement d'une forme \( G \in \mathcal G \) autour du point \( y \)
  s'écrit donc :
  \begin{equation}
    G(x)
    =
    \sum_{k=m}^{ld} \ \overbrace{
      \sum_{\substack{\alpha \in \N^n \\ \lgr\alpha = k}}
      \underbrace{
        \frac 1{\alpha!} \frac{\partial^k G}{\partial X^\alpha} (y)
        \prod_{1 \le i \le n} (x_i - y_i)^{\alpha_i}
      }_{\textstyle R_{k, \alpha}}
    }^{\textstyle R_k}
    \pmm.
  \end{equation}

  On majore maintenant \( \av{G(x)} / \nv2 x ^{ld} \) en procédant terme à
  terme.
  \begin{align*}
    \frac {\nv2{R_{k, \alpha}}} {\nv2 x ^{ld}}
    & \le
    \nv*2{ \frac1{\alpha!} \frac{\partial^k G}{\partial X^\alpha} }
    \nv2 y^{ld-k}
    \nv2{x \wedge y}^k
    / \nv2 x^{ld}
    \\ & \le
    2^{ld\dv} \nv2 G \nv2 y^{ld-k} \nv2 x^k \nv2 y^k
    \distv x y ^k
    / \nv2 x^{ld}
    \\ & \le
    \left( \frac{164}{81} \right)^{ld\dv}
    \nv2 G
    \distv x y ^k
    (\sqrt{n+1})^{k\dv}
    \pmm.
  \end{align*}
  (Pour la dernière estimation, on a majoré \( \nv2 y \) par \( \sqrt{n+1} \) et
  \( (\nv2 y / \nv2 x)^{ld -k} \) par \( (82/81)^{ld} \).)
  On remarque alors qu'il y a au plus \( (n+1)^k \) termes dans \( R_k \) :
  \begin{equation}
    \frac{ \nv2{R_k} }{ \nv2 x^{ld} }
    \le
    \left( \frac{164}{81} \right)^{ld\dv}
    \nv2 G
    \distv x y ^k
    (n+1)^{3k\dv/2}
    \pmm,
  \end{equation}
  puis, si \( v \) est finie, \( \nv2{ G(x) } / \nv2 x ^{ld} \le \nv2 G
    \distv x y ^m \) par l'inégalité ultramétrique et, si \( v \) est
  infinie :
  \begin{align*}
    \frac{\nv2{G(x)}}{\nv2 x^{ld}}
    & \le
    \left( \frac{164}{81} \right)^{ld}
    \nv2 G
    \sum_{k=m}^{ld}
    \distv x y ^k (n+1)^{3k/2}
    \\ & \le
    \left( \frac{164}{81} \right)^{ld}
    \nv2 G
    \distv x y ^m (n+1)^{3m/2}
    % \\ & \phantom{\le} \qquad
    \cdot \sum_{k=0}^{ld-m} \bigl(\distv x y  (n+1)^{3/2}\bigr)^k
    \pmm,
  \end{align*}
  où la dernière somme est majorée par \( 27/26 \), d'où finalement :
  \begin{equation} \label{e:dvpa-comp}
    \frac{ \nv2{\mathcal G(x)} }{\nv2 x^{ld}}
    \le
    \nnv2{ \mathcal G }
    \distv x y ^m
    \left(
      \frac{27}{26} (n+1)^{3m/2}
      \left( \frac{164}{81} \right)^{ld}
    \right)^\dv
    \pmm.
  \end{equation}
  La fin de la démonstration consiste alors en des comparaisons de normes et
  mesures, qu'on peut résumer ainsi :
  \begin{equation}
    \frac{ \mvp{\md_x f} }{ \nv2 x ^{ld} }
    \ll
    \frac{ \nv2{\md_x f} }{ \nv2 x ^{ld} }
    =
    \frac{ \nv2{\mathcal G(x)} }{ \nv2 x ^{ld} }
    \ll
    \nnv2{ \mathcal G }
    =
    \nv2{ \md f }
    \ll
    \mvp{\md f}
    \ll
    \mvp f
    \pmm.
  \end{equation}
  Plus précisément, vu la définition de la mesure et une majoration facile de
  l'intégrale dans cette définition, on a
  \begin{equation}
    \mvp{\md_x f}
    \le
    \nv2{\md_x f}
    \cdot \exp(\dv \cdot ld \gamma_{(n+1)n/2})
  \end{equation}
  où l'on a noté \( \gamma_j = \sum_{i=1}^j \frac1{2i} \). La deuxième
  majoration est donnée par~\eqref{e:dvpa-comp} ci-dessus ; pour la troisième
  on a
  \begin{equation}
    \nv2{ \md f }
    \le
    \mvp{\md f}
    \cdot (n(n+1)^2/2)^{ld\dv}
  \end{equation}
  d'après \cite[dém. du lemme~3.3]{remgdmp}. Pour la dernière, on utilise le
  fait que la distance est bornée par \( 1 \), autrement dit que
  \( \mvp{\md_x f} \le \mvp f \) dès que \( \nv2 x = 1 \) : en reportant ceci
  dans la définition de \( \mvp\truc \) et en intégrant, il vient facilement
  \begin{equation}
    \mvp{\md f}
    \le
    \mvp f
    \cdot \exp(\dv\cdot ld \gamma_{n+1})
    \pmm.
  \end{equation}

  On met alors ces estimations bout à bout et on utilise la majoration
  classique \( \gamma_n \le (1 + \ln n)/2 \) pour obtenir
  \begin{align}
    \distalgv x V
    & \le
    \exp(\dv \cdot ld \gamma_{(n+1)n/2})
    \cdot
    \distv x y ^m
    \left(
      \frac{27}{26} (n+1)^{3m/2}
      \left( \frac{164}{81} \right)^{ld}
    \right)^\dv
    \\ & \hphantom{\le} \quad
    \cdot (n(n+1)^2/2)^{ld\dv}
    \cdot \exp(\dv\cdot ld \gamma_{n+1})
    \\ & \le
    \distv x y ^m
    \left(
      \frac{27}{26} (n+1)^{3m/2}
    \right)^\dv
    \\ & \hphantom{\le} \quad
    \cdot \left(
      \frac{164}{81}
      \sqrt{\frac{ \expb n(n+1) }2}
      \, \frac{ n(n+1)^2 }2
      \, \sqrt{ \expb (n+1) }
    \right)^{ld\dv}
    \\ & \le
    \distv x y ^m
    \left(
      \bigl( 2(n+1)^{3/2} \bigr)^m
      \bigl( 2(n+1)^{9/2} \bigr)^{ld}
    \right)^\dv
  \end{align}
  en remarquant que \( 82 \expb / (81 \sqrt2) \le 2 \), ce qui achève la
  preuve.
\end{proof}

Les deux dernières propositions permettent donc de comparer la distance
algébrique et la distance ponctuelle en toute généralité. On voit bien
intervenir la structure géométrique de la variété \lat{via} son degré ou sa
multiplicité en un point (évidemment majorée par le degré) respectivement. En
conséquence, le corollaire suivant est assez naturel : il montre que les deux
distances coïncident dans le cas d'un sous-variété linéaire, ce qu'on savait
déjà dans les cas particulier des points et des hyperplans.

\begin{coro}
  Si \( V \) est une sous-variété linéaire de \( \projd \), pour tout point \(
    x \in \projd(\Cv) \) on a \( \distv x V = \distalgv x V \).
\end{coro}

\begin{proof}
  Si \( v \) est finie, c'est une conséquence immédiate des deux propositions
  précédentes. Sinon, on utilise la proposition~3.10, p.~76 de \cite{jadotth}
  et sa preuve, qui montre qu'on peut supposer que \( V \) est définie par \(
    \vp[0] = \dots = \vp* = 0 \) et que \( x = (0, \dots, x_\ind,
    x_{\ind+1}, 0, \dots, 0) \). Si l'on pose \( y = (0, \dots, 0,
    1, 0, \dots, 0) \) où le \( 1 \) est en \( (\ind+1) \)-ième position (en
  particulier, \( y \in V \)), on a
  \begin{equation}
    \distv x y
    =
    \frac{ \nv2{ x \wedge y } }{ \nv2 x \, \nv2 y }
    =
    \frac{ \av{ x_\ind } }{ \nv2 x }
    =
    \distalgv x V
  \end{equation}
  où la dernière égalité est le résultat de la proposition citée. Par
  ailleurs, il est clair que pour tout \( y \in V \), c'est-à-dire
  \( y = (0, \dots, 0, y_{\ind+1}, \dots, y_n) \),  on a
  \begin{equation}
    \distv x y ^2
    =
    \frac{
      \av{ x_\ind }^2 \nv2 y ^2
      + \av{ x_{\ind+1} } \sum_{\indi=\ind+2}^n \av{ y_\indi }^2
    }{
      \nv2 x ^2 \nv2 y ^2
    }
    \ge
    \frac{ \av{ x_\ind }^2 }{ \nv2 x ^2 }
    =
    \distalgv x V ^2
  \end{equation}
  ce qui achève la preuve.
\end{proof}

Par ailleurs, dans le cas particulier où \( V \) est une hypersurface, on peut
obtenir une version un peu plus précise de~\vref{p:dv-p2alg} en utilisant
l'expression particulière de la distance algébrique dans ce cas. Cette version
nous sera utile par exemple pour démontrer une inégalité de
\bsc{Liouville} (section~\vref{sec:liouville}).

\begin{prop} \label{p:dv-p2alg-hs}
  Pour toute hypersurface \( V \subset \projd \) et tous points \( x \in
    \projd(\Cv) \) et \( y \in V(\Cv) \), on a
  \begin{equation}
    \distalgv x V
    \le
    \distv x y ^m
    \, \bigl( (17/8)^d (n+1)^{d + 3m/2} \bigr)^\dv
  \end{equation}
  où \( d = \deg V \) et \( m \) est la multiplicité de \( V \) en \( y \).
\end{prop}

\begin{proof}
  On procède comme pour la propriété précédente, sauf que cette fois-ci on
  peut seulement supposer que \( \distv x y \le (4 (n+1)^{7/2})^{-\dv} \),
  dont on déduit également que \( \nv2 y / \nv2 x \le 82/81 \). On développe
  alors une équation \( G \) de \( V \), et non plus un coefficient de \( \md
  f \) ; on obtient ainsi, au lieu de~\eqref{e:dvpa-comp}, la majoration
  suivante :
  \begin{equation} \label{e:dv-p2alg-hs}
    \frac{ \av{G(x)} }{\nv2 x^{d}}
    \le
    \nv2{ G }
    \distv x y ^m
    \left(
      \frac{27}{26} (n+1)^{3m/2}
      \left( \frac{164}{81} \right)^{d}
    \right)^\dv
    \pmm.
  \end{equation}
  On utilise à nouveau~\cite[démonstration. du lemme~3.3]{remgdmp}, qui donne
  \( \nv2 G \le \mvp G (n+1)^d \), pour conclure.
\end{proof}


\cleardoublepage
\endinput

% vim: spell spelllang=fr
