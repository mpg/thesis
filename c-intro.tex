% !TEX root = main.tex

\chapter{Introduction} \label{chap:intro}

\section{Aperçu du problème}

Soit $\va$ une variété abélienne définie sur un corps de nombres~$\cdn$. Une
fois $\va$ munie d'un fibré ample et symétrique $\fibre$ et plongée dans un
espace projectif $\projd$ par un tel fibré, on dispose de deux notions héritées
de l'espace projectif ambiant : une hauteur, et (pour tout place $\place$ de
$\cdn$) une distance $\place$-adique (définie précisément plus bas) entre un
point et une variété.

Dans ces conditions, \bsc{Faltings} a démontré le théorème d'approximation
suivant \cite[\eng{Theorem~II}]{faldaav}.

\begin{thm}[\bsc{Faltings}] \label{t:fal2}
  Soit $\varapx$ une sous-variété quelconque de $\va$, $\place$ une place de $\cdn$,
  et $\eps > 0$. Il n'existe qu'un nombre fini de points $x$ dans
  $\va(\cdn)$ tels que
  \begin{equation}
    0 < \distv{x}{\varapx} \le \hautm{x}^{-\eps}
    \pmm,
    \label{e:has}
  \end{equation}
  où $\Hautm$ désigne la hauteur de \bsc{Weil} multiplicative.
\end{thm}

Comme de nombreux énoncés de géométrie diophantiennes, ce résultat n'est
malheureusement pas effectif au sens suivant : on ne voit à l'heure actuelle
pas de moyen de borner la hauteur des points satisfaisant à l'hypothèse
d'approximation \eqref{e:has}. Ainsi que le fait remarquer \bsc{Faltings}
dans l'introduction de son article : \og \eng{As far as I can see, everything
  here is ineffective beyond hope.} \fg

Il semble néanmoins raisonnable de vouloir majorer le nombre de points
rationnels satisfaisant à \eqref{e:has} (que nous appellerons à l'occasion
les approximations exceptionnelles), ou au moins de donner quelques
informations quantitatives explicites à leur sujet. C'est à cette question que
nous nous intéressons ici.

\medskip

Plus précisément, ce type d'énoncé quantitatif est généralement obtenu en
combinant une inégalité à la \bsc{Vojta} et une inégalité à la \bsc{Mumford},
dont nous rappelons brièvement des énoncés possibles, formulés de façon
générique avec une condition (C) qui peut être par exemple l'hypothèse
d'approximation \eqref{e:has} ci-dessus, ou une autre condition pour
l'ex-conjecture de \bsc{Mordell-Lang}.

Les deux inégalités s'énoncent dans l'espace de \bsc{Mordell-Weil} de $\va$
muni de la forme quadratique donnée par la hauteur normalisée de
\bsc{Néron-Tate}.

L'inégalité de \bsc{Vojta} affirme qu'il n'existe pas de suite $x_1, \dots,
x_m$ de points satisfaisant simultanément à la condition (C) et aux trois
conditions suivantes :
\begin{enumthm}
  \item $\hautn{x_1} > \alpha$ ; \label{i:grand}
  \item $\cos(x_i, x_j) > 1 - \beta$ pour tous $i$ et $j$ ; \label{i:proche}
  \item $\hautn{x_i} > \gamma \hautn{x_{i-1}}$ pour $i > 1$ ;
\end{enumthm}
où l'angle est relatif à la structure euclidienne de l'espace. Nous
appellerons \emph{cône tronqué} une partie de l'espace délimitée par les
conditions \ref{i:grand} et \ref{i:proche}. Il est clair que l'espace privé
d'une boule de rayon $\sqrt{\alpha}$ peut être recouvert par un nombre fini de
tels cônes tronqués dès qu'il est de dimension finie (ce qui est le cas si on
se place sur un corps de nombres). L'inégalité de \bsc{Vojta} assure qu'il
n'y a qu'un nombre fini de points sous la condition (C) dans chaque cône,
et permet donc de conclure à la finitude de l'ensemble des points satisfaisant
(C).

L'inégalité de \bsc{Mumford} peut s'énoncer de façon très similaire. Elle dit
qu'il n'existe pas de suite $x_1, \dots, x_m$ de points satisfaisant
simultanément à la condition (C) et aux trois conditions suivantes :
\begin{enumthm}
  \item $\hautn{x_1} > \alpha'$ ;
  \item $\cos(x_i, x_j) > 1 - \beta'$ pour tous $i$ et $j$ ;
  \item $\abs{ \hautn y - \hautn x } < \delta$.
\end{enumthm}
Utilisée conjointement avec l'inégalité de \bsc{Vojta}, et à condition que les
constantes apparaissant dans ces deux inégalités soient effectives, elle
permet de majorer le nombre de points dans chaque cône tronqué, donc le nombre
total de point (modulo un résultat, assez indépendant, de décompte des «
petits » points).

La démonstration de \bsc{Faltings} consiste précisément à démontrer une
inégalité de \bsc{Vojta}, non effective, qui suffit à assurer la finitude.
L'objectif consiste donc à rendre effective cette inégalité de \bsc{Vojta} et
à lui adjoindre une inégalité de \bsc{Mumford}, elle aussi effective.


\section{Relation avec d'autres énoncés}

\subsection{Le théorème de \bsc{Roth}}

Le théorème de \bsc{Roth} \cite{daroraan}, dans sa version étendue aux places
quelconques par \bsc{Ridout} \cite{ripgtsrt}, peut s'énoncer de la façon
suivante.

\begin{thm}[\bsc{Roth}]
  Soit $\xi \in \Qbar$ un algébrique. Soient par ailleurs $\cdn$ un corps de
  nombres et $\place$ une place de $\cdn$, étendue de façon arbitraire à
  $\cdn(\xi)$. Pour tout $\eps > 0$, il n'existe qu'un nombre fini de
  points $x \in \cdn$ tels que
  \begin{equation}
    \av{x - \xi} < H(x)^{-2-\eps} \pmm.
  \end{equation}
\end{thm}

Le lien avec le théorème précédent est clair : on passe de l'un à l'autre en
remplaçant $\xi$ par $\varapx$ et $\cdn = \aff{1}(\cdn)$ par $\va(\cdn)$, la
distance locale étant bien sûr représentée par $\av{x - \xi}$. Le
théorème~2 de \cite{faldaav} est donc aux variétés abéliennes ce que le
théorème de \bsc{Roth} est à la droite.

Très rapidement après la démonstration initiale de \bsc{Roth}, on a su établir
des versions quantitatives du théorème. Plus précisément, la démonstration de
\bsc{Roth} consiste en un fait qu'on peut, anachroniquement, appeler une
inégalité à la \bsc{Vojta} : aucune suite $x_1, \dots, x_m$ d'approximations
exceptionnelles n'est telle que $\hautm{x_1} > c_1$ et
$\hautm{x_i} > c_2 \cdot \hautm{x_{i-1}}$ pour tout $i > 1$.

Pour établir une version quantitative du théorème de \bsc{Roth}, il a fallu
expliciter une valeur admissible des constantes $c_1$ et $c_2$, d'une part, et
d'autre part lui adjoindre une inégalité que j'appellerai encore
anachroniquement à la \bsc{Mumford}, disant qu'il existe une constante $c_3$
telle que deux approximations exceptionnelles $x$ et $y$, de hauteur assez
grande, satisfont toujours $H(x) > c_3 H(y)$. Dans le cas du théorème de
\bsc{Roth}, ceci découle immédiatement de l'inégalité de la taille. La
conjonction de ces deux inégalités donne clairement un décompte des
approximations exceptionnelles de hauteur assez grande.

Rappelons aussi que le théorème de \bsc{Roth} a été l'aboutissement d'une
longue série de théorèmes d'approximations moins précis, en ce sens que
l'exposant optimal $2+\eps$ n'était pas atteint. Ce série a débuté avec le
théorème de \bsc{Liouville}. Dans le cadre des variétés abéliennes,
l'équivalent de l'inégalité de \bsc{Liouville} peut s'énoncer ainsi :

\begin{prop}
  Soient $\varapx$ une variété projective de degré $d$, et $x \in
  \proj{n}(\Qbar)$ un point algébrique. Si $x$ n'appartient pas à $\varapx$,
  on a
  \begin{equation}
    \distv{x}{\varapx}
    \ge
    c(n, d) \hautm{\varapx}^{-1} \cdot \hautm{x}^{-d} \pmm.
  \end{equation}
\end{prop}

Dans le cas où $\varapx$ est une hypersurface, c'est une application directe du
théorème du produit (et $c(n, d) = 1$), et on peut se ramener à ce cas en
général, quitte à prendre une valeur beaucoup plus petite pour $c(n, d)$.
Comme on le verra en \ref{s:siegel}, il est intéressant dans les applications
de disposer d'un exposant meilleur que $d$.


\subsection{L'ex-conjecture de \bsc{Mordell-Lang}}

Dès 1922, \bsc{Mordell} avait conjecturé l'énoncé suivant, aujourd'hui
un théorème \cite{faldaav}.

\begin{thm}[\bsc{Faltings}, ex-conjecture de \bsc{Mordell}]
  Soit $C$ une courbe projective lisse de genre $g \ge 2$, définie sur un
  corps de nombre $\cdn$. L'ensemble $C(\cdn)$ des points rationnels de $C$
  est fini.
\end{thm}

Ce résultat a d'abord été prouvé par \bsc{Faltings} en 1983 comme conséquence
d'une conjecture de \bsc{Shafarevitch}. La preuve fait intervenir des espaces
de modules de variétés abéliennes, et c'est à cette occasion que
\bsc{Faltings} a introduit la hauteur qui porte désormais son nom, sur cet
espace. Néanmoins, cette preuve reste assez éloignée des méthodes
traditionnelles de l'approximation diophantienne.

Une preuve totalement indépendante a été publiée en 1991 par \bsc{Vojta}. Elle
se rapproche grandement des idées habituelles de l'approximation
diophantienne, en introduisant ce qu'on appelle maintenant l'inégalité de
\bsc{Vojta}. La preuve est ensuite simplifiée (« \eng{avoid[ing] the difficult
  Arakelov theory in Vojta's paper} ») et étendue par \bsc{Faltings} pour
prouver une conjecture de \bsc{Lang}, généralisant celle de \bsc{Mordell}, et
qui s'énonce ainsi.

\begin{thm}[\bsc{Faltings}, ex-conjecture de \bsc{Lang}] \label{t:fal1}
  Soit $\varapx$ une sous-variété d'une variété abélienne $\va$, définie sur un
  corps de nombres $\cdn$. Si $\varapx$ ne contient pas de translaté de sous-variété
  abélienne stricte, alors $\varapx(\cdn)$ est fini.
\end{thm}

Ceci généralise la conjecture de \bsc{Mordell}, qui correspond au cas où $\varapx$
est une courbe et $\va$ sa jacobienne. Ce résultat est proche de notre problème
d'approximation dans le sens suivant : il consiste à montrer la finitude des
points rationnels \emph{sur} une sous-variété de variété abélienne, alors que
nous nous intéressons aux points \emph{proches} d'une telle sous-variété. Il
est d'ailleurs significatif que \bsc{Faltings} a prouvé ces deux théorèmes (la
conjecture de \bsc{Mordell-Lang} et celui que je cherche à rendre quantitatif)
dans le même article : une bonne partie des outils sont commun aux deux
preuves.

Une différence notable entre les deux situations est toutefois la suivante :
pour étudier les points qui sont proches d'une sous-variété, sans appartenir
à cette variété, on n'a pas besoin de supposer que celle-ci ne contient pas de
translaté de sous-groupe. En fait, le résultat reste valable même pour les
approximations d'une sous-variété abélienne.

Des versions quantitatives du théorème~\ref{t:fal1} ont été établies ensuite.
Signalons la relecture de la preuve par \bsc{Bombieri}, qui simplifie certains
arguments en les rapprochant de l'effectivité, et le travail de \bsc{De Diego}
sur les familles de courbes. \bsc{Rémond} obtient, dans la lignée de travaux de
\bsc{Faltings} et \bsc{Bombieri}, une version totalement effective de
l'inégalité de \bsc{Vojta} \cite{remivds}, puis lui adjoint une inégalité à la
\bsc{Mumford}, établissant ainsi une version quantitative \cite{remdcl}
explicite de l'ex-conjecture de \bsc{Mordell-Lang}. Enfin, \bsc{Farhi}
\cite[chapitre~3]{farhith} donne dans sa thèse une version quantitative de
l'ex-conjecture de \bsc{Mordell}, démontrée dans un formalisme plus élémentaire
que celui de \bsc{Rémond}, et légèrement plus précise que son application
directe au cas des courbes.


\subsection{Le théorème de \bsc{Siegel} et une ex-conjecture de \bsc{Lang}}
\label{s:siegel}

Le théorème de \bsc{Siegel}, démontré en 1929, affirme qu'une courbe de genre
supérieur ou égal à 1 ne possède qu'on nombre fini de points entiers. Sa
démonstration repose sur le théorème de \bsc{Roth} énoncé plus haut, et avait
été obtenu par \bsc{Siegel} avec la version faible de cet énoncé dont il
disposait en 1929.  Une généralisation du théorème a été conjecturée par
\bsc{Lang}, de façon analogue à sa généralisation de la conjecture de
\bsc{Mordell} : si $\divi$ est un diviseur ample d'une variété abélienne $\va$,
alors $\va \setminus \divi$ ne possède qu'un nombre fini de points entiers. Le
théorème original s'en déduit là aussi en considérant la courbe dans sa
jacobienne (à cette différence qu'ici la courbe peut être sa jacobienne, dans
le cas $g=1$).

Cette conjecture de \bsc{Lang} est en fait un corollaire du
théorème~\ref{t:fal2} : on remarque que la hauteur (relative à $\divi$) d'un point
entier $x$ est essentiellement donné par le produit des inverse des distances
$v$-adiques de $x$ à $\divi$ quand $v$ parcourt les places archimédiennes de
$\cdn$. Or ces distances sont minorées par $\hautm{x}^{-\eps}$ pour tout
$\eps>0$, sauf pour un nombre fini de points. Ceci est bien sûr contradictoire
dès que $\eps < 1$, ce qui prouve que seules les approximations
exceptionnelles de $\divi$ peuvent donner des points entiers. Dénombrer ces
dernières donne donc immédiatement une version quantitative de cette
ex-conjecture de \bsc{Lang}. C'est pour ce type d'applications qu'il devient
essentiel dans l'énoncé d'approximation de pouvoir prendre $\eps$ petit, au
moins inférieur à $1$, alors que l'exposant $d$ de l'inégalité de
\bsc{Liouville} ne suffit en aucun cas.

Signalons qu'on connaît des versions quantitatives du théorème de \bsc{Siegel}
(??). Par contre, à ma connaissance, la seule démonstration connue de sa
généralisation est celle de \bsc{Faltings} : en particulier on ne connaît pas
de version quantitative de cette ex-conjecture de \bsc{Lang}.

Enfin, en un sens, on peut avoir l'impression que le théorème de
\bsc{Faltings} (ex-conjecture de \bsc{Mordell-Lang}) rend obsolète le théorème
de \bsc{Siegel} et sa généralisation conjecturée par \bsc{Lang} : en effet,
n'avoir qu'un nombre fini de point rationnels implique de n'avoir qu'un
nombre fini de points entiers. En fait, les énoncés de type \bsc{Siegel}
conservent un intérêt essentiellement grâce à la restriction \og ne pas
contenir de sous-variété abélienne \fg dans \bsc{Mordell-Lang} : si on prend
le cas extrême d'une variété abélienne, il est clair que (sur un corps de
nombres pas trop petit) elle possède une infinité de points rationnels, alors
qu'elle n'a qu'un nombre fini de points entiers.


\subsection{Formes linéaires de logarithmes abéliens}

\todo


\section{Stratégie générale}

Il s'agit essentiellement d'employer la méthode de \bsc{Vojta}, en s'inspirant
des travaux de \bsc{Rémond} \cite{remivds,remivg,remdcl}, de \bsc{Farhi}
\cite{farhith}, et de la preuve originale de \bsc{Faltings} \cite{faldaav}.
Dans les grandes lignes, la preuve consistera donc à établir une inégalité à
la \bsc{Mumford}, explicite, et à rendre explicites les constantes dans
l'inégalité à la \bsc{Vojta} obtenue par \bsc{Faltings}.


\subsection {Inégalité de \bsc{Vojta}}

Le schéma général de cette partie de la preuve est commun à celui adopté dans
\cite{remivds} et \cite{faldaav} : on suppose pour commencer qu'il existe une
suite d'approximations $x_1, \ldots, x_m$ satisfaisant simultanément à
\eqref{e:has} et aux trois conditions suivantes :
\begin{enumthm}
  \item $\hautn{x_1} > \alpha$ ; \label{i:vs-grand}
  \item $\cos(x_i, x_j) > 1 - \beta$ pour tous $i$ et $j$ ; \label{i:vs-angle}
  \item $\hautn{x_i} > \gamma \hautn{x_{i-1}}$ pour $i > 1$. \label{i:vs-ecart}
\end{enumthm}
On regarde alors cette suite comme un point de $\va^m$, et on cherche à
construire par récurrence, en partant de $\va^m$,  une suite
décroissante de sous-variétés (irréductibles) produit $\var\pexp{u} =
\prod_i \var\pexp{u}[i]$ pour $u$ variant de $gm$ à $m-1$ telles que :
\begin{enumthm}
  \item $x \in \var\pexp u$ ;
  \item $\dim \var\pexp u = u$ ;
  \item $\deg \var\pexp u \le f(u)$ et $\deg \var\pexp u[i] \le f_i(u)$ ;
  \item $\sum_{i=1}^m a_i \hautm{\var\pexp u[i]} \le g(u, a)$ ; \label{i:ht}
\end{enumthm}
où les $a_i$ sont des entiers choisis de telle sorte que la quantité $a_i^2
\hautn{x_i}$ soit à peu près constante, et $f$, $f_i$ et $g$ sont des
fonctions de $u$ à expliciter (mais $g$ est linéaire en $\abs a^2$). Plus
précisément, on s'attend à ce que $f$ et $f_i$ ne dépendent que du degré de
$\varapx$ est des dimensions ambiantes, et que $g$ ne dépende que de la hauteur de
$\va$, et des degrés et dimensions ambiants.

Supposons un moment cette suite construite, et voyons comment elle permet de
conclure. On regarde $\var\pexp{m-1}$ : c'est un produit dont un des facteurs est
nécessairement réduit à un des $x_i$. Mais l'inégalité \ref{i:ht}, le fait que
$h(x_i)a_i^2$ soit à peu près constant, et la dépendance de $g$ en $\abs a$
permettent alors de majorer la hauteur de $x_1$, de façon à contredire le
point \ref{i:vs-grand} des hypothèses faites sur $x$.

Il suffit donc de savoir construire une telle suite, c'est-à-dire étant donné
une sous-variété produit $\var$ de $\va^m$, de dimension $u$, satisfaisant aux
hypothèses ci-dessus, de savoir construire une variété produit $\var'$ de
dimension $u-1$, satisfaisant aux même hypothèses.

Plus précisément, on cherchera à construire $\var'$ comme une composante de
l'intersection de $\var$ avec une forme $T$ homogène en le groupe de variable
$X_i$ (et indépendante des autres), ne s'annulant pas identiquement sur $\var$,
dont il faut contrôler le degré et la hauteur : les théorèmes de \bsc{Bézout}
géométrique et arithmétique assurent alors que $\var'$ satisfait toujours aux
hypothèses sur la dimension, le degré et la hauteur. Par ailleurs, le choix de
la composante assurera que $x \in \var'$.

Reste donc à savoir construire une telle forme $T$. Pour cela, on va utiliser
la méthode de \bsc{Thue} (dite aussi « méthode de transcendance ») :
construire une forme auxiliaire $F$ de degré et hauteur contrôlés, grâce à un
lemme de \bsc{Siegel}, puis extrapoler pour constater de cette forme s'annule
beaucoup, et utiliser cette information dans un lemme de zéros.

Ici, le lemme de zéros utilisé sera une variante explicite du lemme de produit
de \bsc{Faltings}, dont la conclusion est justement l'existence d'une forme
$T$ comme celle recherchée, pour peu qu'on contrôle bien l'indice en $x$ de
$F$ relativement à des poids suffisamment étagés : ici, les poids sont
contrôlés par les $a_i$, donc en définitive par les rapport des hauteurs des
$x_i$ successifs : c'est l'hypothèse~\ref{i:vs-ecart} qui garanti ainsi leur
étagement.

Dans toute la suite, on supposera que la variété qu'on cherche à approcher est
en fait un diviseur de $\va$. Cette hypothèse n'est pas restrictive dans le
sens où une variété $\varapx$ quelconque est toujours incluse dans un diviseur de
degré comparable et, à une certaine constante près qu'on pourrait expliciter
(dépendant au plus du degré de $\varapx$ et des dimensions ambiantes), la distance
d'un point à la variété est majorée par celle du point au diviseur.

Pour construire cette forme auxiliaire, on ne travaillera en fait pas sur $\var$
directement, mais sur son image par un certain plongement. Ce dernier fait
intervenir les $a_i$ définis ci-dessus, et joue dans la construction un rôle
semblable au fibré $\mathcal L(\eps, s)$ de \cite{faldaav} ou $\mathcal
Q_{\eps, a, d}$ de \cite{remivds}.
\begin{align*}
  \phi \colon \var & \to \va^{2m-1} \\
  (y_1, \dots, y_m)  & \mapsto
  (y_1, \dots, y_m, a_1 y_1 - y_m, \dots a_{m-1} y_{m-1} - y_m)
\end{align*}
On cherche alors à construire une forme sur $\var$, s'annulant avec un indice
élevé le long de $\var \cap \divi^m$, et provenant (\lat{via} $\phi^*$) d'une forme
sur $\phi(\var)$ de multidegré $\delta(\eps_0 a_1^2, \dots, \eps_0 a_m^2, 1,
\dots, 1)$, pour un certain $\eps_0$.

On utilise pour ça un lemme de \bsc{Siegel}. Pour peu que $\delta$ soit assez
grand (et c'est sa raison d'être), le théorème de \bsc{Hilbert} multiprojectif
estime la dimension de l'espace des formes sur $\var$ de degré adéquat. Les
formules d'addition et de multiplication de \cite{daphimhva2} nous
permettent d'estimer la hauteur du système ; cependant une estimation
brutale ne permet pas de contrôler correctement son rang. Il s'agit alors de
réduire le système modulo les équations de $\var$, en présentant ce dernier comme
un revêtement fini de l'espace multiprojectif $\proj{u}$, comme par exemple
dans le lemme~2.5 de \cite{remivg}. Un dénominateur, lié au lieu de
ramification du revêtement considéré, apparaît cependant, qu'on espère
néanmoins contrôler suffisamment pour qu'il ne perturbe pas trop la
construction.

On cherche ensuite à extrapoler, c'est-à-dire à montrer que notre forme
auxiliaire s'annule aussi en $x$, et même avec un indice encore assez élevé
pour pouvoir ensuite appliquer le lemme du produit. Pour cela, l'idée est bien
sûr de majorer la hauteur de $F(x)$, puis d'estimer sa composante locale en
$\place$ exploitant le fait que $x$ est $\place$-adiquement proche de $\divi$,  et
de conclure à une contradiction par la formule du produit à moins de $F(x)$ ne
soit nul. Dans la phrase précédente, il faut en fait remplacer $F$ par ses
dérivées jusqu'à un ordre convenable. C'est en tous cas l'hypothèse
d'approximation \eqref{e:has} qui établit la contrainte nécessaire en liant
$\distv{x}{\divi}$ à $H(x)$, donc indirectement $\av{F(x)}$ à $H(F(x))$.

Pour estimer la norme locale de $F(x)$ en $\place$, la difficulté est que $F$
s'annule le long de $\var \cap \divi$, alors qu'on sait que $x$ est proche de $\divi$,
mais pas nécessairement de $\var \cap \divi$. En fait, $F$ est congrue modulo l'idéal
de $\var$ à une forme s'annulant le long de $\divi$. On exploite alors le fait que
$x$ est sur $\var$ pour extrapoler plutôt sur cette nouvelle forme, de façon à
mieux exploiter l'hypothèse sur $x$. Enfin, pour l'extrapolation, on utilise
une paramétrisation locale de $\var$ (ou $\va$ ?) au voisinage de $x$, afin
d'estimer confortablement les dérivées.

\medskip

Si tout s'est bien passé, c'est fini : il ne reste plus (!) qu'à remonter les
explications que j'ai données essentiellement dans l'ordre inverse de leur
dépendance logique, et à vérifier que les constantes s'ajustent bien entre
elles en raccordant les différentes étapes, et notamment la récurrence\dots


\subsection{Inégalité de \bsc{Mumford}}

On montre une version de cette inégalité dans le cas particulier où la
sous-variété à approcher est une variété abélienne.  L'énoncé obtenu est le
suivant.

\begin{thm} \label{t:mumford}
  Soient $\vai$ une sous-variété abélienne de $\va$, $x$ et $y$
  deux points de $\va(\Qbar)$ satisfaisant à la condition d'approximation
  \eqref{e:has} ci-dessus. Il existe des constantes $\phi$, $\rho$ et $B$
  telles que si
  \begin{enumthm}
    \item $\cos(x, y) \ge 1 - \phi$ ;
    \item $\hautn x \le \hautn y \le (1+\rho) \hautn y$ ;
    \item $\hautn x > B$ ;
  \end{enumthm}
  alors $x - y \in \vai(\Qbar)$.
\end{thm}

Des valeurs admissibles pour $B$, $\rho$ et $\phi$ sont totalement explicitées
en fonction du degré de $\vai$, de sa dimension, du $\eps$ apparaissant dans
\eqref{e:has}, de la dimension et de a hauteur de $\va$, ainsi que la dimension
du plongement ambiant.

L'approche adoptée la suivante : étant donné deux points satisfaisant aux
trois conditions du théorème, on considère leur différence $z$. La
géométrie euclidienne élémentaire montre qu'il est de petite hauteur
normalisée, donc de petite hauteur projective. Une constante de comparaison
explicite entre ces deux hauteurs est fournie par \cite{daphimhva2}.

Il s'agit ensuite de montrer que cette différence reste une bonne
approximation de $\vai$ : c'est ici qu'on utilise que $\vai$ est un groupe.
Pour cela, on utilise la comparaison entre distance ponctuelle et distance
algébrique établie dans \cite[p. 103]{phidg}, qu'on complète par une inégalité
dans l'autre sens.  On se ramène ainsi à montrer que si des points sont
proches, alors leur somme l'est aussi. Ce résultat est en lui-même trivial (la
loi de groupe, algébrique, est lipschitzienne en toute place), mais expliciter
une constante de \bsc{Lipschitz} nécessite d'utiliser assez soigneusement les
formules d'addition de \cite{daphimhva2}.

Une fois ceci établi, on constate que $z$ est une très bonne
approximation de $\vai$, de petite hauteur, suffisamment pour contredire
l'inégalité de \bsc{Liouville}, à moins que $z$ ne soit justement sur
$\vai$ ; ce qui établit le théorème~\ref{t:mumford}.


\section{Définitions et notations}

On fixe une variété abélienne \( \va \) de dimension \( \genre \), définie sur
\( \Qbar \), munie d'un fibré \( \fibre \) ample et symétrique.  On choisit
sur \( \fibre^{\otimes 16} \) une structure thêta et on plonge \( \va \) dans
un espace projectif \( \projd \) avec \( \dimp = 16^\genre \deg\fibre - 1 \)
par un plongement de \bsc{Mumford} modifié (voir~\ref{sec:plong-mm-def}), noté
\( \mmemb \). On identifiera la plupart du temps \( \va \) et son image par \(
  \mmemb \).

On note par ailleurs \( \oa \) l'origine de \( \va \), et on fixe un système
de coordonnées homogènes \( \coa \) de son image par \( \mmemb \). On fixe
également un corps de nombres \( \cdn \) sur lequel \( \va \), \( \mmemb \) et
\( \coa \) sont définis, et qui contient les racines quatrièmes de l'unité.

Si \( \place \) est une place finie de \( \cdn \), on définit la norme \(
  \place \)-adique d'une famille finie comme le maximum des valeurs absolues
\( \place \)-adiques de ses éléments. On défini alors la norme d'un polynôme
comme celle de la famille de ses coefficients. Aux places infinies, on dispose
des différentes normes et mesures définies en~\ref{sec:nma-def}.

On notera \( \nv\star\truc \) la norme locale en \( \place \) correspondant à
la norme \( \star \) si \( \place \) est archimédienne et à celle du maximum
sinon. De même, \( \mvp{} \) et \( \mvm{} \) désignent respectivement la
mesure de \bsc{Philippon} et de \bsc{Mahler} si \( \place \) est archimédienne
et la norme du maximum sinon.

On définit alors la hauteur additive d'une famille finie d'éléments de \( \cdn
\), ou d'un polynôme ou d'une forme (multi)homogène sur \( \cdn \) comme
\begin{equation}
  \hautl[\star]\truc
  =
  \sum_{\place}
  \frac {[\cdn_\place : \Q_\place]} {[\cdn : \Q]}
  \ln \nv\star\truc
  \pmm,
\end{equation}
où la somme est prise sur l'ensemble des places de \( \cdn \) et \( \star \)
désigne l'une des normes ou mesures archimédiennes de l'annexe~\ref{chap:nma}.
On note également \( \Hautm[\star] = \exp(\Hautl[\star]) \) la hauteur
multiplicative.

Pour un point de \( \va(\Qbar) \), ces deux grandeurs dépendent du plongement
\( \mmemb \) fixé, mais pas du système de coordonnées homogènes choisies, ni
même du corps \( \cdn \) utilisé. On utilisera par ailleurs la hauteur
normalisée de \NT \( \Hautn \) associé à l'une des hauteurs \(
  \Hautl[\star] \), qui dépend toujours du fibré \( \fibre \), mais plus du
choix d'un plongement associé à ce fibré, ni de la norme ou mesure utilisée
aux places archimédiennes.

La hauteur normalisée possède par ailleurs des propriétés agréables vis-à-vis
des opérations de \( \va \). En particulier, elle induit une forme quadratique
définie positive sur l'espace de \MoW de \( \va \), défini comme
\begin{equation}
  \MW\cdn = \va(\cdn) \otimes_\Z \R
  \pmm.
\end{equation}
Ce dernier est ainsi un espace vectoriel euclidien, de dimension finie (égale
au rang de \( \va \) sur \( \cdn \)) dès que \( \cdn \) est un corps de
nombres.

Pour chaque place \( \place \) de \( \cdn \), on pose
\begin{equation}
  \distv x y
  =
  \frac{ \nv2{ x \wedge y } }{ \nv2 x \nv2 y }
  \pmm.
\end{equation}
Cette définition correspond, pour les points, à la distance d'indice \( 1 \)
utilisée dans \cite{phidg} et \cite{jadotth}. Cette dernière référence
(notamment lemme~3.2 \textit{(ii)}, page 51) montre qu'il s'agit bien d'une
distance au sens usuel du terme. Pour la distance entre un point et une
variété, on s'écarte par contre légèrement des références citée, en posant
\begin{equation}
  \distv x V
  =
  \min_{y \in V(\Cv)} \distv x y
  \pmm.
\end{equation}
Les deux notions sont en fait très comparables, voir la section suivante. La
notion qu'on vient de définir bénéficie par contre de la propriété suivante :
si \( V \subset V' \) alors \( \distv x V' \le \distv x V \), qu'on utilisera
couramment pour se ramener au cas où \( V \) est une hypersurface.

\medskip

On choisit un ensemble fini \( \placesapx \) de \( \cdn \), une sous-variété
\( \varapx \) de \( \va \), et un réel \( \expapx > 0 \).  On considère alors
la condition d'approximation suivante.
\begin{equation} \tag{\textsc{HA}} \label{e:HA}
  \distv x \varapx
  \le
  \hautm[2]{x}^{-\wtapx\expapx}
  \quad \forall \place \in \placesapx
  \pmm,
\end{equation}
où les \( \wtapx \) sont des réels positifs tels que
\( \sum_{\place \in \placesapx} \wtapx [\cdn_v : \Q_v]/[\cdn : \Q] = 1 \).

On constate alors que ce système d'inégalités a en fait un sens pour $x
\in \projd(\Qbar)$ : pour chaque place $w$ prolongeant une place $\place$ de
$\placesapx$, on impose la condition sur la distance $w$-adique avec
$\wtapx[w] = \wtapx$. La normalisation choisie fait que le système ainsi
obtenu est indépendant de l'extension $\Cdn/\cdn$ dans laquelle on l'écrit.
Ainsi, une large partie des raisonnements effectués par la suite se fera sur
\( \Qbar \).


\section{Compléments sur les distances} \label{sec:distv-cmp}

On commence par établir une expression simple de la distance entre un point et
un hyperplan, cas qui sera fondamental dans plusieurs applications, puis on
étudie les liens entre la distance définie ci-dessus et celle utilisée
notamment dans~\cite{phidg}.

\begin{prop} \label{p:dv-hp}
  Si \( E \) est un hyperplan de \( \projd \), d'équation \( L \), on a
  \begin{equation}
    \distv x E
    =
    \frac{ \av{L(x)} }{ \nv2 L \nv2 x }
    \quad \forall x \in \projd(\Cv)
    \pmm.
  \end{equation}
\end{prop}

\begin{proof}
  On commence par se ramener au cas \( L = X_0 \). Pour cela, si \( \place \)
  est ultramétrique, on commence par choisir un coefficient de \( L \) de
  valeur absolue maximale : quitte à permuter les coordonnées, on peut
  supposer que c'est celui d'indice \( 0 \). En divisant par ce coefficient,
  on se ramène au cas \( L = X_0 + l_1 X_1 + \dots + l_n X_n \) avec les \(
    l_i \) entiers.  Il est alors facile de voir que le changement de
  coordonnée donné par \( x'_0 = L(x) \) et \( x'_i = x_i \) pour \( i \ge 1
  \) est une isométrie (son inverse est également à coefficients entiers), de
  sorte qu'on s'est bien ramené au cas \( L = X_0 \) sans modifier les
  quantités apparaissant dans chacun des deux membres, ni perdre en
  généralité.

  Aux places archimédiennes, on plonge \( \cdn_\place \) dans \( \C \) et on
  note \( \overline E \) le relevé de \( E \) dans \( \C^{n+1} \). On munit ce
  dernier de la forme hermitienne standard et on considère un vecteur unitaire
  normal à \( \overline E \), qu'on note \( a_0 \). On complète ce vecteur en
  un base orthonormée et on considère le changement de coordonnée qui envoie
  cette base sur la base canonique : étant donné par une transformation
  unitaire, il ne modifie aucun des deux membres de l'égalité à prouver, et
  nous ramène bien au cas \( L = X_0 \).

  Par ailleurs, l'égalité qu'on cherche à montrer est banale si \( x \in
    E(\Cv) \). On peut donc supposer que ce n'est pas le cas, et même
  choisir \( (1, x_1, \dots, x_n) \) comme représentant que \( x \). Le membre
  de droite est alors égal à \( \nv2 x ^{-1} \).

  Soit maintenant \( y = (0, y_1, \dots, y_n) \in E(\Cv) \). On a
  \( x \wedge y = (-y_1, \dots, -y_n, x_1 y_2 - x_2 y_1, \dots) \), donc ce
  \( \nv2{ x \wedge y } \ge \nv2 y \), avec égalité si et seulement si \( y_i
    = x_i \) pour tout \( i \in \set{1, \dots, n} \). Ainsi, pour tout \( y
    \in E(\Cv) \) on a \( \distv x y \ge \nv2 x ^{-1} \), où l'égalité est
  possible, ce qui achève la preuve.
\end{proof}

Remarquons que le distance entre points et variété définie ici ne dépend en
fait que de l'ensemble des points de la variété mais pas de la structure
géométrique (multiplicités) de celle-ci. Une autre notion de distance,
développée dans \cite{jadotth} et \cite{phidg} pour le cas projectif, et
\cite{remgdmp} pour le cas multiprojectif (qui englobe bien sûr le précédent)
prend au contraire en compte cette structure. Cette distance est définie par
\begin{equation}
  \distalgv x V
  =
  \frac{
    \mvp{\md_x f_V}
  }{
    \mvp{f_V} \nv2 x ^{\Delta(u+1)}
  }
\end{equation}
où \( f_V \) est une forme de \bsc{Chow} de \( V \), son degré est noté \(
  \Delta \) et sa dimension \( u \). Le morphisme \( \md_x \) est défini dans
les références citées (p. 88 de \cite{phidg} par exemple) ; par ailleurs, par
rapport à ces références, on considère uniquement la distance d'indice \( (1,
  \dots, 1) \).

Nous appellerons cette distance \emph{algébrique}, par opposition à la
distance ensembliste définie ci-dessus. Examinons un peu les relations entre
ces deux notions. Tout d'abord, dans le cas où \( V = \set y \) est réduite à
un point, on a \( \distv x y = \distalgv x V \) d'après \cite[p. 50]{jadotth}.
Par ailleurs, dans le cas où \( V \) est une hypersurface de degré \( \Delta
\) et d'équation \( F \), la proposition~3.6 (p.~64) de cette même référence
montre que
\begin{equation}
  \distalgv x V
  =
  \frac{ \av{F(x)} }{ \mvp F \nv2 x ^\Delta }
  \pmm.
\end{equation}
Si \( V \) est en fait un hyperplan, on a \( \mvp F = \nv2 F \) d'après la
proposition~4 (p.~266) de~\cite{phiha1}, de sorte que la
proposition~\ref{p:dv-hp} dit en fait simplement que \( \Distalgv = \Distv \)
dans ce cas particulier.
% Plutôt que phiha1, on peu citer Jadot, prop 2.3 (ii) p. 23, qui le cite,
% mais énonce le truc sous une forme plus directe.
% Sinon y'a aussi la page 86 de phidg, au moment où il cite Lau1...

Dans le cas général, on a la comparaison suivante.
\begin{fact} \label{f:closest-point}
  Pour toute variété \( V \subset \Proj^n \) et tout point \( x \in
    \Proj^n(\C_v) \), il existe un point \( y \in V(\C_v) \) tel que \(
    \Distv(x, y) \le \Distv(x, V)^{1/\deg V} e^{\dv\gamma_{n+1}} \).
\end{fact}

\begin{proof}
  Pour les places infinies, c'est la \eng{closest point property} p.~89
  de~\cite{phidg}. Pour les places finies, on constate que la preuve s'adapte,
  car le résultat crucial (lemme~5.2 de la référence citée) est en fait vrai
  aux places finies sans le dernier facteur (et en remplaçant évidemment la
  mesure par la norme du sup).
\end{proof}

Dans l'autre sens, on a le résultat suivant.
\begin{prop} \label{p:dv-p2alg}
  Pour toute variété \( V \subset \projd \) et tous points \( x \in
    \projd(\Cv) \) et \( y \in V(\Cv) \), on a
  \begin{equation}
    \distalgv x V
    \le
    \distv x y ^m
    \bigl( 2 (n+1)^{3/2} \bigr)^{(m + 3ld) \dv}
  \end{equation}
  où \( l = \dim V + 1 \), \( d = \deg V \) et \( m \) est la multiplicité de
  \( V \) en \( y \).
\end{prop}

La démonstration repose sur un développement de \( \md f \) autour de \( y
\) relativement à \( X \), qui permet de quantifier le fait que, \( \md_y f \)
étant nul, \( \md_x f \) doit être petit pour \( x \) voisin de \( y \).
De façon générale, pour écrire un développement de \bsc{Taylor} , il faudra
s'assurer de l'existence une carte affine \( X_i \neq 0 \) contenant à la fois
\( x \) et \( y \). On voit facilement que la non-existence d'un telle
carte affine impliquerait \( \distv x y  = 1 \), rendant le résultat banal. En
pratique, on pourra supposer sans problème que \( x \) et \( y \) sont
suffisamment proches.  Par ailleurs, il sera utile de pouvoir de plus choisir
un indice \( i \) de sorte que \( \av{x_i} \) (resp.  \( \av{y_i} \)) soit
comparable  à \( \nv2 x \) (resp. \( \nv2 y \)). Le lemme élémentaire suivant
montre que c'est possible.

\begin{lem} \label{l:dv-common-i}
  Pour chaque \( x \in \projd \), il existe \( i \in \set{0, \dots, n} \) tel
  que \( \frac{ \av{x_i} }{ \nv2 x } \ge (\frac1{\sqrt{n+1}})^\dv \). De
  plus, pour le même indice \( i \), pour tout point \( y \) satisfaisant
  \( \distv x y  \le \eps_0 \) avec \( \eps_0 < (\frac{1}{\sqrt{n+1}})^\dv \),
  on a
  \begin{equation}
    \frac{\av{y_i}}{\nv2 y}
    \ge
    \left( \frac{\sqrt{n+1}}{n+2} - \eps_0 \right)^\dv
    \pmm.
  \end{equation}
  En particulier, si \( \eps_0 < (\frac1{4\sqrt{n+1}})^\dv \) on a \(
    \frac{\av{y_i}}{\nv2 y} \ge (\frac1{2\sqrt{n+1}})^\dv \) et \( y_i \neq 0
  \).
\end{lem}

\begin{proof}
  Pour le premier point, il suffit de choisir \( i \in \set{ 0, \dots, n } \)
  maximisant \( \av{x_i} \). Afin d'alléger les notations, on supposera par
  la suite \( i = 0 \). Pour chaque \( z = (z_0, \dots, z_n) \) on notera \(
    \hat{z} = (z_1, \dots, z_n) \) le vecteur obtenu en omettant la première
  coordonnée. On a alors, vu l'hypothèse sur \( x \) et la définition de la
  distance :
  \begin{equation} \label{e:dvci-hat}
    \eps_0 \nv2 y \nv2 x
    \ge
    \nv2{y \wedge x}
    \ge
    \nv2{x_0 \hatys - y_0 \hatxs}
    \pmm,
  \end{equation}
  où la deuxième inégalité vient en remarquant que toutes les coordonnées de
  \( x_0 \hatys - y_0 \hatxs \) apparaissent également comme coordonnées de \(
    y \wedge x \).

  On traite d'abord le cas ultramétrique, par l'absurde. En effet, si on avait
  \( \av{y_0} < \nv2{\hatys} = \nv2{y} \), il viendrait \( \nv2{x_0 \hatys} >
    \nv2{y_0 \hatxs} \) et la propriété ultramétrique appliquée au dernier
  membre de \eqref{e:dvci-hat} donnerait \( \eps_0 \nv2{y}\nv2{x}  \ge
    \av{x_0}\nv2{\hatys} \), puis \( \eps_0 \ge 1 \) contrairement aux
  hypothèses.

  Pour le cas archimédien, l'inégalité triangulaire dans le membre de droite
  de \eqref{e:dvci-hat} donne
  \( \eps_0 \nv2{y}\nv2{x}  \ge \av{x_0}\nv2{\hatys} - \av{y_0}\nv2{\hatxs}
  \).  En divisant par \( \nv2{y}\nv2{x} \) puis en remarquant que \(
    \nv2{\hatxs}/\nv2{x} \le 1 \) il vient successivement :
  \begin{align}
    \frac{\nv2{\hatxs}}{\nv2{x}}
    \cdot \frac{\av{y_0}}{\nv2{y}}
    & \ge
    \frac{\nv2{\hatys}}{\nv2{y}}
    \cdot \frac{\av{x_0}}{\nv2{x}}
    - \eps_0
    \pmm, \notag
    \\
    \frac{\av{y_0}}{\nv2{y}}
    & \ge
    \frac{\nv2{\hatys}}{\nv2{y}}
    \cdot
    \frac{1}{\sqrt{n+1}}
    - \eps_0
    \pmm.
  \end{align}
  Notons \( t = \av{y_0}/\nv2{y} \in [0; 1] \) ; comme \( \nv2{y}^2 =
    \av{y_0}^2 + \nv2{\hatys}^2 \), on réécrit l'inégalité précédente sous
  la forme \( t \ge \sqrt{\frac{1-t^2}{n+1}}-\eps_0 \), ou encore :
  \begin{equation}
    \left( \frac{n+2}{n+1} \right) t^2
    + 2t\eps_0
    + \eps_0^2
    - \frac{1}{n+1}
    \ge
    0
    \pmm,
  \end{equation}
  qui, compte tenu de l'hypothèse sur \( \eps_0 \) et de la positivité de \( t
  \), implique :
  \begin{align*}
    t
    & \ge
    \frac{n+1}{n+2}
    \sqrt{ \frac{-\eps_0^2}{n+1}+\frac{n+2}{(n+1)^2} }
    - \left( \frac{n+1}{n+2} \right) \eps_0
    \\ & \ge
    \frac{\sqrt{n+1}}{n+2} - \eps_0
    \pmm,
  \end{align*}
  comme annoncé. Le cas particulier s'en déduit immédiatement en substituant.
\end{proof}

\begin{proof}[\proofname{} de la proposition~\ref{p:dv-p2alg}.]
  Notons pour commencer qu'on peut supposer \( \dim V \ge 1 \) (donc \( n \ge
    2 \)) car on sait que les deux distances sont égales si \( V \) est un
  point.  peut supposer \( n \ge 2 \).  On peut de plus supposer que \( \distv
    x y  < (n+1)^{-9\dv/2} \) car sinon la conclusion du lemme est vide, la
  distance étant de toutes façons bornée par \( 1 \). (On peut en fait
  supposer une majoration plus forte, mais celle-ci est largement suffisante.)
  Les hypothèses du lemme précédent sont satisfaites ; en l'utilisant (et
  quitte à renuméroter) on peut supposer que \( y_0 = x_0 = 1 \), \( \nv2 y
    \le (\sqrt{n+1})^\dv \) et \( \nv2 x \le (2 \sqrt{n+1})^\dv \).

  Par ailleurs, \( \nv2 y \le \nv2 x + \nv2 {y-x} \le \nv2 x + \nv2{x \wedge
      y} \), car chaque coefficient de \( x - y \) est aussi un coefficient de
  \( x \wedge y \).  En substituant \( \nv2{x \wedge y} = \distv x y  \nv2 x
    \nv2 y \) et en divisant l'inégalité obtenue par \( \nv2 x \), compte tenu
  de l'hypothèse sur la distance, il vient \( \nv2 y / \nv2 x \le 1 +
    (n+1)^{-4} \le 82/81 \).

  On reprend les notations de la section~4 (page 117) de~\cite{remgdmp}
  concernant le morphisme \( \md \).  On regarde \( \md f \) comme un polynôme
  en \( s \) à coefficients dans \( k[X] \) et on appelle \( \mathcal G \) la
  famille de ses coefficients ; chacun d'entre eux est homogène de degré \( ld
  \) et s'annule en \( y \).  Plus précisément, d'après
  \cite[prop.~3]{phitzee}, la multiplicité de \( V \) en \( y \) est le plus
  grand entier \( k \) tel que les dérivées d'ordre total \( k-1 \) de toutes
  les formes de la famille \( \mathcal G \) soient nulles en \( y \). Le
  développement d'une forme \( G \in \mathcal G \) autour du point \( y \)
  s'écrit donc :
  \begin{equation}
    G(x)
    =
    \sum_{k=m}^{ld} \ \overbrace{
      \sum_{\substack{\alpha \in \N^n \\ \lgr\alpha = k}}
      \underbrace{
        \frac 1{\alpha!} \frac{\partial^k g}{\partial X^\alpha} (y)
        \prod_{1 \le i \le n} (x_i - y_i)^{\alpha_i}
      }_{\textstyle R_{k, \alpha}}
    }^{\textstyle R_k}
    \pmm.
  \end{equation}

  On majore maintenant \( \av{G(x)} / \nv2 x ^{ld} \) en procédant terme à
  terme.
  \worknote{On ne pourrait pas être plus précis sur les dérivées (leur somme
  en fait) ?}
  \begin{align*}
    \frac {\nv2{R_{k, \alpha}}} {\nv2 x ^{ld}}
    & \le
    \nv*2{ \frac1{\alpha!} \frac{\partial^k G}{\partial X^\alpha} }
    \nv2 y^{ld-k}
    \nv2{x \wedge y}^k
    / \nv2 x^{ld}
    \\ & \le
    2^{ld\dv} \nv2 G \nv2 y^{ld-k} \nv2 x^k \nv2 y^k
    \distv x y ^k
    / \nv2 x^{ld}
    \\ & \le
    \left( \frac{164}{81} \right)^{ld\dv}
    \nv2 G
    \distv x y ^k
    (\sqrt{n+1})^{k\dv}
    \pmm.
  \end{align*}
  On remarque alors qu'il y a au plus \( (n+1)^k \) termes :
  \begin{equation}
    \frac{ \nv2{R_k} }{ \nv2 x^{ld} }
    \le
    \left( \frac{164}{81} \right)^{ld\dv}
    \nv2 G
    \distv x y ^k
    (n+1)^{3k\dv/2}
    \pmm,
  \end{equation}
  puis, si \( v \) est finie, \( \nv2{ G(x) } / \nv2 x ^{ld} \le \nv2 G
    \distv x y ^m \) par l'inégalité ultramétrique et, si \( v \) est
  infinie :
  \begin{align*}
    \frac{\nv2{g(x)}}{\nv2 x^{ld}}
    & \le
    \left( \frac{164}{81} \right)^{ld}
    \nv2 G
    \sum_{k=m}^{ld}
    \distv x y ^k (n+1)^{3k/2}
    \\ & \le
    \left( \frac{164}{81} \right)^{ld}
    \nv2 G
    \distv x y ^m (n+1)^{3m/2}
    % \\ & \phantom{\le} \qquad
    \cdot \sum_{k=0}^{ld-m} \bigl(\distv x y  (n+1)^{3/2}\bigr)^k
    \pmm,
  \end{align*}
  où la dernière somme est majorée par \( 27/26 \), d'où finalement :
  \begin{equation} \label{e:dvpa-comp}
    \frac{ \nv2{\mathcal G(x)} }{\nv2 x^{ld}}
    \le
    \nnv2{ \mathcal G }
    \distv x y ^m
    \left(
      \frac{27}{26} (n+1)^{3m/2}
      \left( \frac{164}{81} \right)^{ld}
    \right)^\dv
    \pmm.
  \end{equation}
  La fin de la démonstration consiste alors en des comparaisons de normes :
  \begin{equation}
    \frac{ \mvp{\md_x f} }{ \nv2 x ^{ld} }
    \stackrel{(a)}{\ll}
    \frac{ \nv2{\md_x f} }{ \nv2 x ^{ld} }
    =
    \frac{ \nv2{\mathcal G(x)} }{ \nv2 x ^{ld} }
    \stackrel{(b)}{\ll}
    \nnv2{ \mathcal G }
    =
    \nv2{ \md f }
    \stackrel{(c)}{\ll}
    \mvp{\md f}
    \stackrel{(d)}{\ll}
    \mvp f
    \pmm,
  \end{equation}
  où l'on peut choisir \( \exp(\dv \cdot ld \gamma_{(n+1)n/2}) \) pour \( (a)
  \) vu la définition et une majoration facile de l'intégrale, tandis que pour
  \( (c) \) l'on peut prendre \( (n(n+1)^2/2)^{ld\dv} \) d'après
  \cite[dém. du lemme~3.3]{remgdmp}. La constante \( (b) \) est donnée par
  l'inégalité \eqref{e:dvpa-comp} ci-dessus ; la dernière découle du fait que \(
    \mvp{\md_x f} \le \mvp f \) dès que \( \nv2 x = 1 \) (c.-à-d. du fait que
  la distance est majorée par \( 1 \)) : en effet, en reportant ceci dans la
  définition de \( \mvp\truc \) et en intégrant, il vient \( \mvp{\md f} \le
    \exp(\dv\cdot ld \gamma_{n+1}) \mvp f \).

  En mettant tout ceci bout à bout, et en utilisant la majoration classique \(
    \gamma_n \le (1 + \ln n)/2 \), il vient :
  \begin{align}
    \distalgv x V
    & \le
    \exp(\dv \cdot ld \gamma_{(n+1)n/2})
    \cdot
    \distv x y ^m
    \left(
      \frac{27}{26} (n+1)^{3m/2}
      \left( \frac{164}{81} \right)^{ld}
    \right)^\dv
    \\ & \hphantom{\le} \quad
    \cdot (n(n+1)^2/2)^{ld\dv}
    \cdot \exp(\dv\cdot ld \gamma_{n+1})
    \\ & \le
    \distv x y ^m
    \left(
      \frac{27}{26} (n+1)^{3m/2}
    \right)^\dv
    \\ & \hphantom{\le} \quad
    \cdot \left(
      \frac{164}{81}
      \sqrt{\frac{ \expb n(n+1) }2}
      \, \frac{ n(n+1)^2 }2
      \, \sqrt{ \expb (n+1) }
    \right)^{ld\dv}
    \\ & \le
    \distv x y ^m
    \left(
      \bigl( 2(n+1)^{3/2} \bigr)^m
      \bigl( 2(n+1)^{9/2} \bigr)^{ld}
    \right)^\dv
  \end{align}
  en remarquant que \( 82 \expb / (81 \sqrt2) \le 2 \), ce qui achève la
  preuve.
\end{proof}

Ensemble, les deux résultats précédents montrent que \( \distv x V = \distalgv
  x V \) dès que \( V \) est linéaire et \( \place \) finie. C'est
probablement aussi le cas aux places archimédiennes (on sait que c'est vrai
pour les points et les hyperplans), mais on ne le montre pas pour l'instant.

Par ailleurs, dans le cas particulier où \( V \) est une hypersurface, on peut
obtenir une version un peu plus précise de~\ref{p:dv-p2alg} en utilisant
l'expression particulière de la distance algébrique dans ce cas. Cette version
nous sera utile par exemple pour démontrer une inégalité de
\bsc{Liouville} (section~\ref{sec:liouville}).

\begin{prop} \label{p:dv-p2alg-hs}
  Pour toute hypersurface \( V \subset \projd \) et tous points \( x \in
    \projd(\Cv) \) et \( y \in V(\Cv) \), on a
  \begin{equation}
    \distalgv x V
    \le
    \distv x y ^m
    \, \bigl( (17/8)^d (n+1)^{d + 3m/2} \bigr)^\dv
  \end{equation}
  où \( d = \deg V \) et \( m \) est la multiplicité de \( V \) en \( y \).
\end{prop}

\begin{proof}
  On procède comme pour la propriété précédente, sauf que cette fois-ci on
  peut seulement supposer que \( \distv(x, y) \le (4 (n+1)^{7/2})^{-\dv} \),
  dont on déduit également que \( \nv2 y / \nv2 x \le 82/81 \). On développe
  alors une équation \( G \) de \( V \), et non plus un coefficient de \( \md
  f \) ; on obtient ainsi, au lieu de~\eqref{e:dvpa-comp}, la majoration
  suivante :
  \begin{equation} \label{e:dv-p2alg-hs}
    \frac{ \av{G(x)} }{\nv2 x^{d}}
    \le
    \nv2{ G }
    \distv x y ^m
    \left(
      \frac{27}{26} (n+1)^{3m/2}
      \left( \frac{164}{81} \right)^{d}
    \right)^\dv
    \pmm.
  \end{equation}
  On utilise à nouveau~\cite[dém. du lemme~3.3]{remgdmp}, qui donne \( \nv2 G
  \le \mvp G (n+1)^d \), pour conclure.
\end{proof}


\cleardoublepage
\endinput

% vim: spell spelllang=fr
