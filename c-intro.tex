% !TEX root = main.tex

\chapter{Introduction} \label{chap:intro}

\section{Aperçu du problème}

Soit \( \va \) une variété abélienne définie sur un corps de nombres \( \cdn
\). Une fois \( \va \) plongée dans un espace projectif \( \projd \), on
dispose de deux notions héritées de l'espace projectif ambiant : une hauteur,
et (pour tout place \( v \) de \( \cdn \)) une distance \( v \)-adique
(définie en section~\vref{sec:distv}) entre un point et une variété.  Dans ces
conditions, \bsc{Faltings} a démontré le théorème d'approximation suivant
\cite[\eng{Theorem~II}]{faldaav}.

\begin{thm}[\bsc{Faltings}] \label{t:fal2}
  Soit \( \avar \) une sous-variété quelconque de \( \va \), \( v \)
  une place de \( \cdn \), et \( \eps > 0 \). Il n'existe qu'un nombre fini de
  points \( x \) dans \( \va(\cdn) \) tels que
  \begin{equation} \label{e:has}
    0
    <
    \distv x \avar
    \le
    \hautm{} x ^{-\eps}
    \pmm.
  \end{equation}
\end{thm}

Comme de nombreux énoncés de géométrie diophantiennes, ce résultat n'est
malheureusement pas effectif au sens suivant : on ne voit à l'heure actuelle
pas de moyen de borner la hauteur des points satisfaisant à l'hypothèse
d'approximation \eqref{e:has}. Ainsi que le fait remarquer \bsc{Faltings}
dans l'introduction de son article : « \eng{As far as I can see, everything
  here is ineffective beyond hope.} »

Néanmoins, il semble \lat{a priori} raisonnable de vouloir majorer
explicitement le nombre de points rationnels satisfaisant à \eqref{e:has} (que
nous appellerons à l'occasion les approximations exceptionnelles).  Pour les
courbes elliptiques, ce travail a été accompli dans \cite{grosiipec} pour les
courbes de hauteur assez grandes, puis de façon indépendante par \bsc{Farhi}
(\cite[chap.~2]{farhith} ou \cite{faraetr}) pour toutes les courbes, avec des
constantes d'apparence assez différente.

L'objet du présent mémoire, qui s'inscrit dans la lignée des travaux de
\bsc{Farhi}, est de généraliser ce résultat en dimension supérieure,
c'est-à-dire d'obtenir, autant que possible, une version quantitative
explicite du théorème d'approximation de \bsc{Faltings}.

\medskip

Plus précisément, ce type d'énoncé quantitatif est généralement obtenu en
combinant une inégalité à la \bsc{Vojta} et une inégalité à la \bsc{Mumford},
dont nous rappelons brièvement des énoncés possibles, formulés de façon
générique avec une condition (C) qui peut être par exemple l'hypothèse
d'approximation \eqref{e:has} ci-dessus, ou une autre condition pour
l'ex-conjecture de \bsc{Mordell-Lang}.

Les deux inégalités s'énoncent dans l'espace de \bsc{Mordell-Weil} de \( \va \)
muni de la forme quadratique donnée par la hauteur normalisée de
\bsc{Néron-Tate}.

L'inégalité de \bsc{Vojta} affirme qu'il n'existe pas de suite \( x_1, \dots,
  x_m \) de points satisfaisant simultanément à la condition (C) et aux trois
conditions suivantes :
\begin{enumthm}
  \item \( \hautn{x_1} > \alpha \) ; \label{i:grand}
  \item \( \cos(x_i, x_j) > 1 - \beta \) pour tous \( i \) et \( j \) ;
    \label{i:proche}
  \item \( \hautn{x_i} > \gamma \hautn{x_{i-1}} \) pour \( i > 1 \) ;
\end{enumthm}
où l'angle est relatif à la structure euclidienne de l'espace. Nous
appellerons \emph{cône tronqué} une partie de l'espace délimitée par les
conditions \oref{i:grand} et \oref{i:proche} ci-dessus. Il est clair que
l'espace privé d'une boule de rayon \( \sqrt{\alpha} \) peut être recouvert
par un nombre fini de tels cônes tronqués dès qu'il est de dimension finie (ce
qui est le cas si on se place sur un corps de nombres). L'inégalité de
\bsc{Vojta} assure que, sous la condition (C), il n'y a qu'un nombre fini de
points dans chaque cône, et permet donc de conclure à la finitude de
l'ensemble des points satisfaisant (C).

L'inégalité de \bsc{Mumford} peut s'énoncer de façon très similaire. Elle dit
qu'il n'existe pas de suite \( x_1, \dots, x_m \) de points satisfaisant
simultanément à la condition (C) et aux trois conditions suivantes :
\begin{enumthm}
  \item \( \hautn{x_i} > \alpha' \) pour tout \( i \) ;
  \item \( \cos(x_i, x_j) > 1 - \beta' \) pour tous \( i \) et \( j \) ;
  \item \( \abs{ \hautn{x_i} - \hautn{x_j} } < \delta \) pour tous \( i \) et
    \( j \).
\end{enumthm}
Utilisée conjointement avec l'inégalité de \bsc{Vojta}, et à condition que les
constantes apparaissant dans ces deux inégalités soient effectives, elle
permet de majorer le nombre de points dans chaque cône tronqué, donc le nombre
total de point (modulo un résultat, assez indépendant, de décompte des «
petits » points).

La démonstration de \bsc{Faltings} consiste précisément à démontrer une
inégalité de \bsc{Vojta}, non effective, qui suffit à assurer la finitude.
L'objectif consiste donc à rendre effective cette inégalité de \bsc{Vojta} et
à lui adjoindre une inégalité de \bsc{Mumford}, elle aussi effective.


\section{Relation avec d'autres énoncés}

\subsection{Le théorème de \bsc{Roth}}

Le théorème de \bsc{Roth} \cite{daroraan}, concernant l'approximation des
algébriques par les rationnels, a d'abord été démontré pour la place
archimédienne de \( \Q \), puis étendu aux places quelconques par \bsc{Ridout}
\cite{ripgtsrt} ; en voici un énoncé moderne (cas particulier du
théorème~D.2.1 page~304 de \cite{hinsidgi}, en tenant compte des différences
de normalisation).

\begin{thm}[\bsc{Roth}]
  Soit \( \xi \in \Qbar \) un nombre algébrique. Soient par ailleurs \( \cdn
  \) un corps de nombres et \( v \) une place de \( \cdn \), étendue de façon
  arbitraire à \( \cdn(\xi) \). Pour tout \( \eps > 0 \), il n'existe qu'un
  nombre fini de points \( x \in \cdn \) tels que
  \begin{equation}
    \av{x - \xi}^\degv
    <
    \hautm2 x ^{-2-\eps}
  \end{equation}
  où \( \degv = [\cdn(\xi)_v : \Q_v] / [\cdn(\xi) : \Q] \) et la valeur
  absolue est normalisée de façon à prolonger une des valeurs absolues
  usuelles de \( \Q \).
\end{thm}

Le lien avec le théorème qui nous occupe ici est clair : on passe de l'un à
l'autre en remplaçant \( \xi \) par \( \avar \) et \( \cdn = \aff1(\cdn) \)
par \( \va(\cdn) \), la distance locale étant bien sûr représentée par \(
  \av{x - \xi} \). Le théorème~2 de \cite{faldaav} est donc aux variétés
abéliennes ce que le théorème de \bsc{Roth} est à la droite, avec l'exposant
\( 2 + \eps \) remplacé par \( \eps \).

Très rapidement après la démonstration initiale de \bsc{Roth}, on a su établir
des versions quantitatives du théorème. Plus précisément, la démonstration de
\bsc{Roth} consiste en un fait qu'on peut appeler, anachroniquement, une
inégalité à la \bsc{Vojta} : aucune suite \( x_1, \dots, x_m \)
d'approximations exceptionnelles n'est telle que \( \hautm{}{x_1} > c_1 \) et
\( \hautm{}{x_i} > c_2 \cdot \hautm{}{x_{i-1}} \) pour tout \( i > 1 \).

Pour établir une version quantitative du théorème de \bsc{Roth}, il a fallu
expliciter une valeur admissible des constantes \( c_1 \) et \( c_2 \), d'une
part, et d'autre part lui adjoindre une inégalité que j'appellerai encore
anachroniquement à la \bsc{Mumford}, disant qu'il existe une constante \( c_3
\) telle que deux approximations exceptionnelles \( x \) et \( y \), de
hauteur assez grande, satisfont toujours \( \hautm{} x > c_3 \hautm{} y \).
Dans le cas du théorème de \bsc{Roth}, ceci découle immédiatement de
l'inégalité de la taille. La conjonction de ces deux inégalités donne
clairement un décompte des approximations exceptionnelles de hauteur assez
grande.

Rappelons aussi que le théorème de \bsc{Roth} a été l'aboutissement d'une
longue série de théorèmes d'approximations moins précis, en ce sens que
l'exposant optimal \( 2+\eps \) n'était pas atteint. Cette série a débuté avec
le théorème de \bsc{Liouville}. Dans le cadre des espaces projectifs
l'équivalent de l'inégalité de \bsc{Liouville} peut s'énoncer ainsi :

\begin{prop}
  Soient \( \avar \) une sous variété de \( \projd \) de degré \( \adeg \) et
  \( x \in \projd(\Qbar) \) un point algébrique. Si \( x \) n'appartient pas à
  \( \avar \), on a
  \begin{equation}
    \distv x \avar ^\degv
    \ge
    c(\avar) \cdot \hautm{}{x}^{-\adeg}
  \end{equation}
  où \( \degv \) est le degré local, divisé par le degré global, d'un corps de
  définition de \( x \).
\end{prop}

Nous établirons (proposition~\vref{p:liouville}) une version explicite de cette
inégalité qui, outre son intérêt propre, joue un rôle crucial dans la preuve de
nos inégalités à la \bsc{Mumford}. Un corollaire immédiat de l'inégalité de
\bsc{Liouville} est le suivant.

\begin{coro} \label{c:liou-intro}
  Soient \( \avar \) une sous variété de \( \projd \) de degré \( \adeg \) et
  \( \eps > 0 \) un réel. Alors l'ensemble des points \( x \in \projd(\Qbar)
  \) tels que
  \begin{equation}
    \distv x \avar ^\degv
    \ge
    \hautm{}{x}^{-\adeg-\eps}
  \end{equation}
  est de hauteur bornée explicitement ; en particulier cet ensemble est fini
  si l'on exige que \( x \) appartienne à un corps de nombres de degré borné.
\end{coro}

On peut relever les différences suivantes entre cet énoncé et le théorème
d'approximation de \bsc{Faltings} : l'inégalité de \bsc{Liouville} prend place
dans un espace projectif, il n'y pas besoin d'une variété abélienne ambiante ;
on sait expliciter la borne sur la hauteur des approximation exceptionnelles ;
en revanche, l'exposant de \( \hautm{} x \) n'est pas aussi fin.  Comme on le
verra en \vref{s:siegel}, il est intéressant pour certaines applications de
disposer d'un exposant inférieur ou égal à \( 1 \), ce qui n'est jamais le cas
dans le corollaire précédent, même si \( \avar \) est linéaire.


\subsection{L'ex-conjecture de \bsc{Mordell-Lang}}

Dès 1922, \bsc{Mordell} avait conjecturé l'énoncé suivant.

\begin{thm}[\bsc{Faltings}, ex-conjecture de \bsc{Mordell}]
  Soit \( C \) une courbe projective lisse de genre \( g \ge 2 \), définie sur
  un corps de nombre \( \cdn \). L'ensemble \( C(\cdn) \) des points
  rationnels de \( C \) est fini.
\end{thm}

Ce résultat a d'abord été prouvé par \bsc{Faltings} en 1983 comme conséquence
d'une conjecture de \bsc{Shafarevitch} \cite{falmor}. La preuve fait
intervenir des espaces de modules de variétés abéliennes, et c'est à cette
occasion que \bsc{Faltings} a introduit la hauteur qui porte désormais son
nom, sur ces espaces. Néanmoins, cette preuve reste assez éloignée des
méthodes traditionnelles de l'approximation diophantienne.

Une preuve totalement indépendante a été publiée en 1991 par \bsc{Vojta}
\cite{vojstcc}. Elle se rapproche grandement des idées habituelles de
l'approximation diophantienne, en introduisant ce qu'on appelle maintenant
l'inégalité de \bsc{Vojta}. La preuve est ensuite simplifiée (« \eng{avoid[ing]
the difficult Arakelov theory in Vojta's paper} ») et étendue par
\bsc{Faltings} \cite{faldaav} pour prouver une conjecture de \bsc{Lang},
généralisant celle de \bsc{Mordell}, et qui s'énonce ainsi.

\begin{thm}[\bsc{Faltings}, ex-conjecture de \bsc{Mordell}-\bsc{Lang}]
  \label{t:fal1} ~\\
  Soit \( \avar \) une sous-variété d'une variété abélienne \( \va \), définie
  sur un corps de nombres \( \cdn \). Si \( \avar \) ne contient pas de
  translaté de sous-variété abélienne stricte, alors \( \avar(\cdn) \) est
  fini.
\end{thm}

Ceci généralise la conjecture de \bsc{Mordell}, qui correspond au cas où \(
  \avar \) est une courbe et \( \va \) sa jacobienne. Ce résultat est proche
de notre problème d'approximation dans le sens suivant : il consiste à montrer
la finitude des points rationnels \emph{sur} une sous-variété de variété
abélienne, alors que nous nous intéressons aux points \emph{proches} d'une
telle sous-variété. Il est d'ailleurs significatif que \bsc{Faltings} a prouvé
ces deux théorèmes (la conjecture de \bsc{Mordell-Lang} et celui que ce
mémoire cherche à rendre quantitatif) dans le même article : une bonne partie
des outils est commune aux deux preuves.

Une différence notable entre les deux situations est toutefois la suivante :
pour étudier les points qui sont proches d'une sous-variété, sans appartenir
à cette variété, on n'a pas besoin de supposer que celle-ci ne contient pas de
translaté de sous-groupe. En fait, le résultat reste valable même pour les
approximations d'une sous-variété abélienne.

Des versions quantitatives du théorème~\vref{t:fal1} ont été établies ensuite
en suivant la méthode de \bsc{Vojta}.  Signalons la relecture de la preuve par
\bsc{Bombieri} \cite{bommcr}, qui simplifie certains arguments en les
rapprochant de l'effectivité, et le travail de \bsc{De Diego} \cite{ddprf} sur
les familles de courbes. \bsc{Rémond} obtient, dans la lignée de travaux de
\bsc{Faltings} et \bsc{Bombieri}, une version totalement effective de
l'inégalité de \bsc{Vojta} \cite{remivds}, puis lui adjoint une inégalité à la
\bsc{Mumford}, établissant ainsi une version quantitative explicite
\cite{remdcl} de l'ex-conjecture de \bsc{Mordell-Lang}. Enfin, \bsc{Farhi}
\cite[chap.~3]{farhith} et \cite{faraptf} donne une version quantitative de
l'ex-conjecture de \bsc{Mordell}, démontrée dans un formalisme plus élémentaire
que celui de \bsc{Rémond}, et légèrement plus précise que son application
directe au cas des courbes.


\subsection{Le théorème de \bsc{Siegel} et une ex-conjecture de \bsc{Lang}}
\label{s:siegel}

Le théorème de \bsc{Siegel} \cite{siegel} affirme qu'une courbe de genre
supérieur ou égal à \( 1 \) ne possède qu'on nombre fini de points entiers. Sa
démonstration repose sur le théorème de \bsc{Roth} énoncé plus haut, et avait
été obtenu par \bsc{Siegel} avec la version faible de cet énoncé dont il
disposait en 1929. Pour les courbes de genre supérieur ou égal à \( 2 \), ce
théorème est en quelque sorte surpassé par l'ex-conjecture de \bsc{Mordell},
mais il conserve un intérêt pour les courbes elliptiques. La généralisation
suivante du théorème avait été conjecturée par \bsc{Lang} \cite{laipc}.

\begin{thm}[\bsc{Faltings}, ex-conjecture de \bsc{Lang}]
  Soit \( \va \) une variété abélienne plongée dans \( \projd \) et \( \divi
  \) un hyperplan de \( \projd \). Alors \( \va \setminus \divi \) ne possède
  qu'un nombre fini de points entiers.
\end{thm}

C'est en fait un corollaire \cite[cor.~6.2, p.~576]{faldaav} du théorème
d'approximation de \bsc{Faltings} : on remarque que la hauteur (relative à \(
  \divi \)) d'un point entier \( x \) est égale au produit des inverses des
distances \( v \)-adiques de \( x \) à \( \divi \) quand \( v \) parcourt les
places archimédiennes de \( \cdn \). Ainsi, les points entiers de \( \va
  \setminus \divi \) sont des approximations exceptionnelles de \( \divi \) au
sens du théorème~\vref{t:fal2} pour \( \eps = 1 \), elles sont donc en nombre
fini.

C'est pour ce type d'applications qu'il devient essentiel dans le théorème
d'approximation de pouvoir prendre \( \eps \le 1 \) et que l'inégalité de
\bsc{Liouville} (corollaire~\vref{c:liou-intro}) ne suffit pas.

Les résultats obtenus ici permettent de donner un décompte explicite des
points entiers, malheureusement pas dans le cas général, mais au moins dans
le \( \va \) est simple (et principalement polarisée). À notre connaissance,
une telle version quantitative explicite de cette ex-conjecture de \bsc{Lang},
même dans un cas particulier, est nouvelle pour une variété abélienne de
dimension au moins \( 2 \) ; néanmoins, les constantes que nous obtenons ne
sont pas satisfaisantes (corollaire~\vref{c:lang}).



\section{Énoncés des résultats principaux}

Comme on l'a mentionné précédemment, il semble \lat{a priori} difficile de
borner la hauteur des approximations exceptionnelles, mais plus réaliste de
borner leur nombre. En fait, il apparaît que dans notre situation, ces deux
questions sont en général intimement liées : en effet, comme le montrera en
détails le corollaire~\vref{c:factory}, certains approximations
exceptionnelles n'arrivent pas seule mais engendrent en fait une « grappe »
d'approximations de qualité semblable, dont le cardinal est minoré en fonction
de la hauteur de l'approximation qui engendre cette grappe. Ce phénomène est
directement lié à l'existence possible de translatés de sous-variétés
abéliennes contenus dans la variété approchée.

Ainsi, majorer le cardinal de l'ensemble des approximations exceptionnelles,
donc de chaque grappe, reviendrait à majorer la hauteur de certains
approximations exceptionnelles, voire de toutes les approximations
exceptionnelles, résultat qui ne semble pas accessible à l'heure actuelle.

Par contre, on peut compter les grappes d'approximations exceptionnelles ou,
autrement dit majorer le cardinal d'ensembles d'approximations exceptionnelles
ne contenant qu'un point dans chaque grappe. On introduit dans ce but la
définition suivante.

\begin{tdef} \label{d:cond}
  Soient \( F \) un sous-ensemble de \( \va(\Qbar) \) et \( \tau \) un réel
  positif. On dit que \( F \) satisfait \( \cond*\tau \) s'il existe une
  sous-variété abélienne \( \vai \) de \( \va \) dont un translaté est contenu
  dans \( \avar \) et deux points distincts \( x \) et \( y \) dans \( F \),
  tels que \( x - y \in \vai \) et \( \hautn{ x-y } \le \tau \hautn x \).

  On dit que \( F \) satisfait la condition \( \cond\tau \) si \( F \) ne
  satisfait pas \( \cond*\tau \).
\end{tdef}

Nous allons donc nous attacher à contrôler le cardinal d'ensemble
d'approximations exceptionnelles satisfaisant \( \cond\tau \) pour un certain
\( \tau \) explicite. La section~\vref{sec:obstruction} discute plus en
détails ce phénomène de grappe et le choix de la condition utilisée pour
le mettre de côté. Signalons tout de même avant de continuer que, si \( \va
\) est simple, il n'y a pas de grappes, la condition ci-dessus est vide et
les résultats ci-dessous donnent donc des décomptes complets de toutes les
approximations exceptionnelles.

\medskip

Le premier résultat de décompte que nous donnons concerne le cas particulier
où la variété approchée est un translaté d'une sous-variété abélienne de \(
  \va \).

Pour simplifier, on note \( \vacst \) une constant ne dépendant que de la
hauteur de \( \va \), qui peut être explicitée totalement, et \( N \) le
cardinal d'un système complet de formes représentant l'addition de \( \va \)
(voir la section~\vref{sec:vaemb}).

\begin{thm}[corollaire~\vref{c:all-gen}]
  Soit \( \avar \) un translaté par un point algébrique d'une sous-variété
  abélienne \( \vai \) de \( \va \), de degré \( \adeg \) et de dimension \(
    \adim \).  On considère de plus un sous-groupe \( \grp \subset \va(\Qbar)
  \) de rang fini \( r \) et un réel \( \eps > 0 \).  Soit enfin une famille
  \( x_1, \dots, x_p \) de points de \( \grp \) et \( \placesapx \) un ensemble
  fini de places de \( \Q(x_1, \dots, x_p) \) ; on suppose que cette famille
  satisfait à \( \cond{\eps/2d} \) et que, pour tout \( i \) :
  \begin{align}
    0 < \prod\placerange \distv{x_i} \avar ^\degv
    & <
    \hautm2{x_i}^{-\eps}
    \exp(- \adeg \alpha \Lambda^{(4g)^{4g^2}})
  \end{align}
  avec
  \begin{align}
    \Lambda
    & =
    \bigl( 87 \nclmaps \adeg \, \eps^{-1} \bigr)^2
    \bigl( 5 (\deg \va) (3 g^2 \adeg)^g \bigr)^{2g}
    \\
    \alpha
    & =
    \adeg^{g+1} \vacst
    + (g + 1) (\deg \va) \adeg^g \hautl1 \avar
    \\ & \qquad
    + (g + 1) (\deg \va)
    \bigl(
      (4\adeg)^{n+1}
      + \adeg^{g+1} (g + 2) ( \adeg + n + 1 )
    \bigr)
    \pmm.
  \end{align}
  Alors on a
  \begin{equation}
    p
    \le
    2 \cdot 5^{\card \placesapx} \cdot
    \sqrt{ \frac{\adeg}\eps }
    (4g)^{4g^2+1}
    \ln \Lambda
    \left(
        739 \nclmaps \cdot 7^g \adeg
        \, \eps^{-1}
    \right)^r
    \pmm.
  \end{equation}
\end{thm}

Comme on le constate, la condition d'approximation a été renforcée (facteur \(
  \exp(- \dots) \)). Ceci a pour but d'éliminer les points de petite hauteur
afin de fournir un décompte complet. Sous l'hypothèse d'approximation plus
naturelle (sans ce dernier facteur), on obtient un décompte seulement pour les
points de grande hauteur (corollaire~\vref{c:big-m=2g}).

Signalons une interprétation possible de ce type d'énoncé : on a
\begin{equation}
  \sum\placerange \degv \ln \distv{x} \avar
  \ge
  - \eps \hautl2{x}
  - \adeg \alpha \Lambda^{(4g)^{4g^2}}
\end{equation}
sauf pour un nombre fini de \( x \) (les approximations exceptionnelles).
Dans l'inégalité ci-dessus, dans le cas où \( \avar \) est une sous-variété
abélienne de \( \va \), le membre de gauche peut être interprété comme une
forme linéaire de logarithmes abéliens. La minoration obtenue présente alors
la particularité de séparer \( \hautl2 x \) de \( d \), qui représente la
hauteur des formes linéaires. Cependant, la dépendant en \( d \) est nettement
moins bonne que celle obtenue dans les minorations usuelles de formes
linéaires de logarithmes et surtout, l'inégalité admet des exceptions, bien
qu'elles soient en nombre fini et que ce nombre soit en partie contrôlé.

\medskip

Le deuxième décompte que nous donnons concerne le cas général. On suppose que
\( \va \) est principalement polarisée et plongée dans \( \projd \) par un
plongement de \bsc{Mumford} modifié (section~\vref{sec:vaemb}).

\begin{thm}[corollaire~\vref{c:big-any}]
  Soit \( \avar \) une sous-variété de \( \va \), de degré \( \adeg \) et de
  dimension \( \adim \). On considère de plus un sous-groupe \( \grp \subset
    \va(\Qbar) \) de rang fini \( r \) et un réel \( 0 < \eps \le 1 \).
  Soit enfin une famille \( x_1, \dots, x_p \) de points de \( \grp \)
  satisfaisant à \( \cond{ \eps / \adeg^M (2M)^{ (M+1)\adim} } \) et aux
  conditions suivantes, pour tout \( i \) :
  \begin{align}
    0
    & < \prod\placerange \distv{x_i} \avar^\degv
    < \hautm2{x_i}^{-\eps}
    \\
    \hautn{x_i}
    & >
    ( \hautl1 \avar + \vacst )
    \, \eps^{-2(4g)^{4g^2}}
    \adeg^M (3M)^{ (M+1)\adim + 3}
  \end{align}
  où \( \placesapx \) est un ensemble fini de places de \( \Q(x_1, \dots, x_p)
  \) et
  \(
    M
    =
    \bigl(
    2^{34} \, \vacst \, \adeg
    \bigr)^{ (r+1) g^{ 5(\adim + 1)^2 } }
    + 1
    \pmm.
  \)
  Alors on a
  \begin{equation}
    p
    \le
    5^{\card\placesapx}
    M^2 \Bigl( \adeg^{M} (3M)^{(M+1)\adim} \Bigr)^{(r+1)/2}
    \, \eps^{-r - 1/2} \ln(\expb/\eps)
    \pmm.
  \end{equation}
\end{thm}

On constate que la dépendance en les différents paramètres est
significativement moins bonne que celle qu'on pourrait attendre (et en
particulier que celle obtenue dans le cas précédent). Ceci semble être une
faiblesse de la méthode utilisée pour prouver l'inégalité de \bsc{Mumford}
dans ce cas.

Signalons enfin qu'on peut déduire du résultat précédent un décompte
explicite, quoi que peu satisfaisant, des points entiers d'un ouvert affine
d'une variété abélienne simple : c'est le corollaire~\vref{c:lang}.



\section{Stratégie générale}

Il s'agit essentiellement d'employer la méthode de \bsc{Vojta}, en s'inspirant
des travaux de \bsc{Rémond} \cite{remivds,remivg,remdcl}, de \bsc{Farhi}
\cite[chap.~2]{farhith}, et de la preuve originale de \bsc{Faltings}
\cite{faldaav}.  Dans les grandes lignes, la preuve consistera donc à établir
une version explicite de l'inégalité à la \bsc{Vojta} obtenue par
\bsc{Faltings} et à lui adjoindre une inégalité à la \bsc{Mumford}, elle aussi
explicite.

Les arguments utiliseront un formalisme simple : plongements, coordonnées et
polynômes plutôt que fibrés (métrisés) et sections globales. Les outils
techniques essentiels sont ceux de la théorie de l'élimination tels que
rappelés par exemple dans \cite[chap.~5 à~7]{nesphilnm}.

Le plan de la thèse est le suivant :
\begin{itemize}
  \item les sections suivantes du présent chapitre introduisent les
    principales notations et le cadre général dans lequel on se place, en
    rappelant les propriétés essentielles des notions utilisée ainsi qu'en
    prouvant au besoin quelque propriétés nouvelles mais assez simples ;
  \item le chapitre deux établit une inégalité de \bsc{Vojta}, en commençant
    par un cas particulier avant d'en déduire le cas général ;
  \item le chapitre trois établit une inégalité de \bsc{Liouville} et deux
    inégalités de \bsc{Mumford} : la première concerne un cas particulier mais
    présente de bien meilleures constantes que la seconde, traitant le cas
    général ;
  \item enfin, le chapitre quatre déduit de ces inégalités des décomptes de
    grands points avec certaines restrictions, après avoir détaillé les
    obstructions qui nous obligent à imposer lesdites restrictions.
\end{itemize}
Présentons maintenant un peu plus en détail les stratégies mises en œuvre pour
établir les résultats techniques principaux que sont les inégalités de
\bsc{Vojta} et \bsc{Mumford}, en commençant par cette dernière, qui est la
plus directe.

\medskip

Le remarque essentielle pour l'inégalité de \bsc{Mumford} est que si deux
points satisfont à ses hypothèses, leur différence sera de hauteur très petite
relativement aux points initiaux. Une proposition clé contrôle l'action de la
soustraction sur la distance ; on voit alors que cette différence est ainsi
très proche d'une nouvelle variété : dans le cas où la variété approchée est
un translaté de sous-variété abélienne, il s'agit de la sous-variété abélienne
en question, dans le cas général on contrôle le degré et la hauteur de la
variété projective obtenue.

Ainsi, la différence est une approximation exceptionnelle d'une nouvelle
variété et comme sa hauteur a chuté, elle est en fait tellement exceptionnelle
que l'inégalité de \bsc{Liouville} la contraint à être sur cette variété. Dans
le cas d'une sous-variété abélienne, c'est la conclusion voulue ; dans le cas
général, on doit invoquer la version quantitative de l'ex-conjecture de
\bsc{Mordell}-\bsc{Lang} donnée par \bsc{Rémond} pour conclure.

\medskip

Pour l'inégalité de \bsc{Vojta}, on se ramène au cas particulier où la variété
approchée est un hyperplan standard de la façon suivante. Si \( \avar \) est
une variété quelconque, on choisit une hypersurface qui la contient, puis par
un plongement de \bsc{Veronese} on transforme cette hypersurface en hyperplan,
et un changement de coordonnées linéaire transforme ce dernier en l'hyperplan
standard \( \vp[0] = 0 \). À chaque étape on contrôle explicitement l'action
sur la distance, le degré et la hauteur des objets en jeu, et les différentes
constantes associées au plongement.

L'idée principale de cette partie est alors la suivante : en supposant qu'il
existe un \( m \)-uplet d'approximations exceptionnelles satisfaisant aux
hypothèses du théorème, on le regarde comme un point dans \( \va^m \) et on
l'enferme dans des sous-variétés produit de \( \va^m \) de dimension
décroissante et de degrés et hauteurs contrôlés. Au final, l'un des facteurs
de cette variété sera réduit à un point, et le contrôle obtenu sur sa hauteur
contredira les hypothèses, achevant la preuve.

Pour cela, il s'agit, étant donné une telle variété, de produire une forme sur
un des facteurs, s'annulant au point étudié, par laquelle couper pour faire
décroître la dimension de la variété dans laquelle on l'a enfermé ; nous
l'appellerons \emph{forme motrice}. Si cette forme est suffisamment bien
contrôlée, le théorème de \bsc{Bézout}, dans ses versions géométrique et
arithmétique, permet alors de contrôler le degré et la hauteur de
l'intersection.

Le cœur technique de la preuve est donc la construction de cette forme
motrice, pour laquelle on utilise la méthode de \bsc{Thue}. De façon
classique, on commence par utiliser un lemme de \bsc{Siegel} pour produire une
forme auxiliaire de multidegré prescrit (et étagé sur les différents facteurs)
et de hauteur contrôlé, qui s'annule avec un indice (lui aussi étagé) élevé le
long de l'hyperplan approché. L'étape d'extrapolation montre alors que la
forme auxiliaire s'annule au point étudié, puis un lemme de zéros (en
l'occurrence, une variante du théorème du produit) permet de conclure à
l'existence de la forme motrice, ce qui clôt la preuve.



\section{Définitions et notations}

\subsection{Formes homogènes et multihomogènes}

On notera généralement \( \vp = (\vp[0], \dots, \vp[n]) \) un groupe de
variables. Si \( \ip \in \N^{n+1} \) est un multiindice, on notera
\( \vp^\ip = \vp[0]^{\ip[0]} \cdots \vp[n]^{\ip[n]} \). On introduit également
les notations \( \ip! = \ip[0]! \cdots \ip[n]! \) et \( \lgr\ip = \ip[0] +
  \dots + \ip[n] \) ; on note alors \( \binom{\lgr\ip}{\ip} =
  \frac{\lgr\ip!}{\ip!} \) le coefficient multinomial. Pour les dérivées, on
utilisera les notations
\begin{equation}
  \frac{\partial^\ip}{\partial \vp^\ip}
  =
  \frac{ \partial^{\ip[0]} }{ \partial \vp^{\ip[0]} }
  \cdots
  \frac{ \partial^{\ip[n]} }{ \partial \vp^{\ip[n]} }
  \quad\text{et}\quad
  \der[\ip]
  =
  \frac1{\ip!} \frac{\partial^\ip}{\partial \vp^\ip}
  \pmm.
\end{equation}

Introduisons maintenant quelques notations pour les formes multihomogènes ;
nous travaillerons principalement avec des formes sur des puissances d'espaces
projectifs, nous nous limitons donc à ce cas pour alléger un peu. On notera
\( \vmp = (\vmp[1], \dots, \vmp[m]) \) un groupe de groupes de variables, où \(
  \vmp[i] = (\vmp[i][0], \dots, \vmp[i][n]) \). On étend naturellement les
notations projectives : pour \( \imp \in \N^{(n+1)m} \) on note \( \vmp^\imp =
  (\vmp[1])^{\imp[1]} \cdots (\vmp[m])^{\imp[m]} \) ; on procède de même pour
les factorielles et les dérivées (divisées). Par ailleurs, on note \( \vlg\imp
  = (\lgr{\imp[1]}, \dots, \lgr{\imp[m]}) \) le vecteur des longueurs de \(
  \ip \) et \( \lgt\imp = \sum_{i=1}^m \lgr{\imp[i]} \).


\subsection{Normes locales}

À chaque place \( v \) d'un corps de nombres \( \cdn \), on associe la
valeur absolue qui prolonge une des valeurs absolues standard de \( \Q \),
autrement dit \( \av p = p^{-1} \) si \( v \) divise un premier \( p \)
et \( \av 2 = 2 \) si \( v \) est archimédienne.

Si \( v \) est une place finie de \( \cdn \), on définit la norme \(
  v \)-adique d'une famille finie d'éléments de \( \cdn \) comme le maximum
des valeurs absolues \( v \)-adiques de ses éléments. Si \( v \) est
archimédienne, on utilisera les normes usuelles suivantes :
\begin{align}
  \nv\infty x & = \max_i{ \av{ x_i } } &
  \nv1 x & = \sum_i{ \av{ x_i } } &
  \nv2 x & = \Bigl( \sum_i{ \av{ x_i }^2 } \Bigr)^{1/2}
  \pmm.
\end{align}
On définit alors la norme d'un polynôme comme celle de la famille de ses
coefficients.  Si \( \mathcal F = (F_1, \dots, F_p) \) est une famille de
polynômes, on définit de plus sa norme \( \nnv\star{ \mathcal F } \) comme la
norme \( \nv\star\truc \) du vecteur des \( \nv\star{ F_i } \).

Aux places archimédiennes, on dispose également de deux notions de mesure pour
les polynômes. La première est la classique mesure de \bsc{Mahler}, définie
pour \( v \) archimédienne par :
\begin{equation}
  \ln \mvm P
  =
  \int_0^1 \Diff t_1 \dots \int_0^1 \Diff t_n
  \ln \av{ P(\expb^{2\pi i t_1}, \dots, \expb^{2\pi i t_n}) }
\end{equation}
si \( P \neq 0 \) et \( \mvm0 = 0 \) ; aux places ultramétriques on notera
\( \mvm P = \nv\infty P \).

La seconde, que nous appellerons
mesure projective, est une variante introduite par \bsc{Philippon} pour les
formes multihomogènes en intégrant sur un produit de sphères et non de cercles
ainsi qu'en ajoutant un facteur correctif. On conservera \( \mvp P = \nv\infty
  P \) aux places finies, et aux places infinies, si \( P \) est multihomogène
de multidegré \( \delta \) en \( l \) groupes de \( n_k + 1 \) variables :
\begin{equation}
  \log \mvp P
  =
  \bigl(
    \int_{S_{1}\times\cdots\times S_{l}}
    \log\av P\ \mu_1 \wedge \cdots \wedge \mu_l
  \bigr)
  + \sum_{k=1}^l \delta_k \cdot \gamma_{n_k}
  \pmm,
\end{equation}
où \( S_k = \{ u \in \C^{n_k},\ \normeuc u = 1 \} \) est la sphère de
dimension \( 2n_k-1 \) plongée dans \( \C^{n_k} \) muni de sa structure
hermitienne usuelle, \( \mu_k \) désigne la mesure sur \( S_k \) invariante
par l'action du groupe unitaire et normalisée par \( \mu_k(S_k) = 1 \), et
enfin
\begin{equation} \label{e:def-gamma-n}
  \gamma_n = \frac12 \sum_{q=1}^{n} \frac1q
  \pmm.
\end{equation}
On renvoie à~\cite{nesphilnm} p.~86 pour un rappel des propriétés
essentielles.

On rappelle au passage la définition du nombre de \bsc{Stoll}
\begin{equation} \label{e:def-stoll}
  \stoll n
  =
  \sum_{i=1}^{n} \gamma_i
  =
  \sum_{i=1}^{n} \sum_{j=1}^{i} \frac1{2j}
\end{equation}
et les majorations élémentaires
\begin{align} \label{e:maj-stoll}
  \gamma_n & \le \frac{1 + \ln n}2 \le \ln(n+1)
  &
  \stoll n & \le \frac{(n+1) \ln(n+1)}2 \le n \ln(n+1)
\end{align}
obtenues en comparant avec des intégrales, par exemple.

Parmi les comparaisons de normes et mesures qui nous seront utiles, rappelons
celles entre mesure projective et norme euclidienne ou longueur.  Dans la
démonstration du lemme~3.3, p.~111 de \cite[chap.~7]{nesphilnm}, il est
établit que si \( P \) est une forme multihomogène de multidegré \( \delta \)
en plusieurs groupes de \( n + 1 \) variables, dont on note \( p_\alpha \) les
coefficients, on a
\begin{equation}
  \av{ p_\alpha }
  \le
  \binom \delta \alpha ^\dv
  \mvm P
  \le
  \binom \delta \alpha ^\dv
  \mvp P
\end{equation}
ce qui en sommant sur les multimultiindices \( \alpha \) tels que \(
  \vlg\alpha = \delta \), donne immédiatement, par la formule multinomiale :
\begin{equation} \label{e:l1<mespph}
  \nv1 P
  \le
  (n+1)^{\lgr\delta \dv}
  \mvp P
  \pmm.
\end{equation}
Dans l'autre sens, la définition de la mesure projective donne immédiatement
\begin{equation} \label{e:mespph<l2}
  \mvp P
  \le
  \nv2 P
  \exp( \dv \sum_k \delta_k \cdot \gamma_{n} )
  \le
  \nv2 P
  (n+1)^{\lgr\delta \dv}
  \pmm.
\end{equation}

\subsection{Hauteurs}

On définit la hauteur additive d'une famille finie d'éléments de \( \cdn \)
(en particulier, d'un vecteur ou d'un polynôme \lat{via} la famille de ses
coefficients) comme
\begin{equation}
  \hautl\star\truc
  =
  \sum_{v}
  \degv
  \ln \nv\star\truc
  \pmm,
\end{equation}
où \( \degv = [\cdn_v : \Q_v] / [\cdn : \Q] \),
la somme est prise sur l'ensemble des places de \( \cdn \) et \( \star \)
désigne l'une des normes ci-dessus ; pour une forme (multi)homogène, on notera
\( \Hautl\htpph \) la hauteur obtenue en utilisant la mesure \( \mvp{} \) au
lieu de la norme dans la formule ci-dessus.  On note également \( \Hautm\star
  = \exp \Hautl\star \) la hauteur multiplicative.

La hauteur d'une famille, ainsi définie, est invariante par multiplication de
cette famille par un scalaire, et ne dépend pas non plus du corps de nombres.
Ceci permet de définir la hauteur d'un point de \( \projd(\Qbar) \) comme
celle de ses coordonnées homogènes.

Si \( \avar \) est une sous-variété de \( \projd \) on peut lui associer des
formes de \bsc{Chow} relatives à certains indices de la façon rappelée dans
\cite{nesphilnm} au début du chapitre~6. Nous ne considérerons généralement
que la forme d'indice \( (1, \dots, 1) \), que nous noterons \( \chow \avar \).
Cette forme est bien définie à multiplication par un scalaire près, la
définition \( \hautl\star \avar = \hautl\star{ \chow \avar } \) a donc un sens.

Parmi les différentes hauteurs définies pour les points, \( \Hautl\infty \)
est la hauteur de \bsc{Weil} et \( \Hautl2 \) est la hauteur projective ;
pour les variétés, la hauteur projective est \( \Hautl\htpph \).
Dans le cas où \( \avar = \set x \) est réduite à un point, on a \(
  \hautl\htpph V = \hautl2 x \), ce qui justifie la terminologie.

Nous utiliserons principalement la hauteur projective, pour ses propriétés
géométriques, et la hauteur \( \Hautl1 \), pour les propriétés agréables de la
norme \( L_1 \) (longueur) aux places archimédiennes. Rappelons donc comment
ces hauteurs se comparent dans le cas d'une variété : cette comparaison
découle directement de~\eqref{e:l1<mespph} et~\eqref{e:mespph<l2} appliqués au
cas particulier d'une forme de \bsc{Chow} :
\begin{equation} \label{e:ht-pph-1}
  \abs{ \hautl\htpph{ \avar } - \hautl1{ \avar } }
  \le
  \ln(n + 1) (\dim \avar + 1) \deg \avar
  \pmm.
\end{equation}

Par ailleurs, si l'on fixe un plongement \( \vaemb \colon \va \to \projd \),
on a des notions de hauteurs sur \( \va(\Qbar) \) associées à ce plongement.
Si de plus ce plongement est associé à un fibré \( \fibre \) symétrique, on
peut définir de façon classique une hauteur normalisée \( \Hautn \) (dite de
\NT), qui dépend encore du fibré \( \fibre \) mais plus du choix d'un
plongement particulier associé à ce fibré, ni de la norme ou mesure utilisée
aux places archimédiennes pour définir la hauteur projective.

Cette hauteur normalisée induit une forme quadratique définie positive sur
l'espace de \MoW de \( \va \), défini comme
\(
  \MW\Qbar = \va(\Qbar) \otimes_\Z \R
\).
Plus précisément, la partie \( \MW\cdn = \va(\cdn) \otimes_\Z \R \)
définie sur un corps de nombres \( \cdn \) est un espace euclidien de
dimension (finie) égale au rang de \( \va \) sur \( \cdn \). Par extension, si
\( \grp \) est un sous-groupe de \( \va(\Qbar) \), on note \( \MW\grp \)
le sous-espace de \( \MW\Qbar \) engendré par l'image de \( \grp \) ; cet
espace est de dimension finie si et seulement si \( \grp \) est de rang
fini.


\subsection{Plongement projectif de la variété abélienne} \label{sec:vaemb}

On supposera en général fixé un plongement \( \vaemb \colon \va \to \projd \)
associé à un fibré symétrique \( \fibre \). On supposera de plus le plongement
projectivement normal (c'est-à-dire tel que l'application
\( \Gamma(\projd, \mathcal O(i)) \to \Gamma(\va, \fibre^{\otimes i}) \) soit
surjective pour tout \( i \ge 1 \)) ; cette condition n'est pas restrictive car
si \( \fibrei \) est un fibré ample, alors \( \fibrei^{\otimes l} \) est très
ample et satisfait la condition dès que \( l \ge 3 \).

On introduit alors quelques constantes et notations associées à ce plongement,
dont on justifie brièvement l'existence sous ces hypothèses. On discutera
ensuite de la possibilité d'expliciter certaines d'entre elles dans certains
cas.

\begin{fact} \label{f:htcmp}
  Il existe une constante réelle \( \htcmp \) telle que pour toute
  sous-variété \( \avar \) de \( \va \) on ait
  \begin{equation} \label{e:comp-h-hn-var}
    \abs{ \hautl\htpph \avar - \hautn \avar }
    \le
    \htcmp (\dim \avar + 1) \deg \avar
  \end{equation}
  où \( \Hautn \) est la hauteur normalisée telle que définie dans \cite{phiha1}
  pour les variétés.  En particulier on a
  \begin{equation} \label{e:comp-h-hn}
    \abs{ \hautl2x - \hautn x } \le \htcmp
  \end{equation}
  pour tout \( x \in \va(\Qbar) \).
\end{fact}

\begin{proof}
  Inclus dans la proposition~9 page~281 de~\cite{phiha1}.
\end{proof}

\begin{lem} \label{l:hmclab}
  Il existe un entier \( \nclmaps \), un ensemble \( \clmaps \) de \( \nclmaps
  \) ouverts recouvrant \( \va^2 \), et une constante \( \hmclab \) qui
  se décompose en \( \hmclab = \prod_v \hmclab*^\degv \) où les \( \hmclab* \)
  sont des réels tous supérieurs et presque tous égaux à \( 1 \), possédant la
  propriété suivante.

  Pour tout couple d'entiers relatifs \( (a, b) \) n'admettant que \( 2 \) et
  \( 3 \) comme diviseurs premiers, il existe une famille \(
    (\rmclab{a}{b}{\clmap})_{\clmap \in \clmaps} \) de \( (n+1) \)-uplets de
  formes telle que :
  \begin{enumerate}
    \item la famille
      \( (\rmclab{a}{b}{\clmap}[\ind])_{\ind\in\set{0, \dots n}} \) représente
      le morphisme \( (x, y) \mapsto ax - by \) dans le plongement \( \vaemb
      \) sur (l'image de) \( \clmap \) ;
    \item chaque forme \( \rmclab{a}{b}{\clmap}[\ind] \) est bihomogène de
      bidegré \( (2a^2, 2b^2) \) ;
    \item pour tout place \( v \) et toute carte \( \clmap \), on a
      \begin{equation} \label{e:clab-norm}
        \nnv1{ \rmclab{a}{b}{\clmap} } \le \hmclab*^{a^2 + b^2}
        \pmm;
      \end{equation}
    \item pour tout \( (x, y) \in \va^2 \) et toute place
      \( v \) il existe une carte \( \clmap \in \clmaps \) telle que
      \begin{equation} \label{e:clab-loc}
        \frac{
          \nv1{ \rmclab{a}{b}{ \clmap }(x, y) }
        }{
          \nv1 x ^{2a^2} \nv1 y ^{2b^2}
        }
        \ge
        \hmclab*^{-(a^2 + b^2)}
        \pmm.
      \end{equation}
      (Par ailleurs, cette quantité est majorée par \( \hmclab*^{a^2 + b^2} \)
      d'après le point précédent et les propriétés de la norme utilisée.)
  \end{enumerate}
\end{lem}

\begin{proof}
  Le fait~3 page~276 de~\cite{phiha1}, appliqué à \( S = \set{2, 3} \),
  donne\footnote{À quelques différences de normes et notations près, qui sont
    sans incidence vu qu'on n'annonce pas de valeur explicite.} pour tout
  entier relatif\footnote{La référence citée énonce son résultat avec des
    entiers naturels, mais le résultat de \bsc{Serre} sur lequel elle se base
    est en fait valable pour des entiers relatifs.} \( a \) n'ayant que \( 2
  \) et \( 3 \) comme facteurs premiers, une famille de \( n+1 \) formes \(
    F\pexp a \) de degré \( a^2 \) représentant globalement la multiplication
  par \( a \) dans le plongement fixé, et une famille de constantes \(
    (C_{2,3,v})_v \) presque toutes égales à \( 1 \), telles que, pour toute
  place \( v \),
  \begin{align}
    \nnv1{ F\pexp a }
    \le
    C_{2, 3, v}^{a^2}
    \qquad\text{et, pour tout \( x \in \va(\Qbar) \),}\quad
    \frac{ \nv1{ F\pexp a (x) } }{ \nv1{ x }^{a^2} }
    \ge
    C_{2, 3, v}^{-a^2}
    \pmm.
  \end{align}
  On conclut directement en composant avec le système complet de formes
  représentant la soustraction donné par le lemme suivant.
\end{proof}

\begin{lem} \label{l:hmpm}
  Il existe un entier \( \nclmaps \), un ensemble \( \clmaps \) de \( \nclmaps
  \) ouverts recouvrant \( \va^2 \), une constante \( \hmpm \) qui se
  décompose en \( \hmpm = \prod_v (\hmpm*)^\degv \) où les \( \hmpm* \) sont
  des réels tous supérieurs et presque tous égaux à \( 1 \), et enfin une
  famille \( (D\pexp\clmap)_{\clmap \in \clmaps} \) de formes bihomogènes de
  bidegré \( (2, 2) \) telles que
  \begin{enumerate}
    \item la famille \( (D\pexp\clmap[\ind])\indrange \) représente le
      morphisme de soustraction dans le plongement \( \vaemb \) sur (l'image
      de) \( \clmap \) ;
    \item pour toute place \( v \) et toute carte \( \clmap \), on a
      \( \nnv1{ D\pexp\clmap } \le \hmpm* \) ;
    \item pour tout \( (x, y) \in \va^2 \) et toute place
      \( v \) il existe une carte \( \clmap \in \clmaps \) telle que
      \begin{equation} \label{e:pm-loc}
        \frac{
          \nv1{ D\pexp\clmap (x, y) }
        }{
          \nv1 x ^2 \nv1 y ^2
        }
        \ge
        (\hmpm*)^{-1}
        \pmm.
      \end{equation}
      (Par ailleurs, cette quantité est majorée par \( \hmpm* \) d'après le
      point précédent et les propriétés de la norme utilisée.)
  \end{enumerate}
\end{lem}

\begin{proof}
  Vu les hypothèses sur le plongement, \cite{larucsal} assure qu'il existe un
  système complet de formules de bidegré \( (2, 2) \) représentant l'addition.
  On en déduit aisément un système complet de formules pour la soustraction de
  même degré en composant sur le deuxième facteur avec une famille de formules
  linéaires représentant globalement la multiplication par \( -1 \). Cette
  famille répond à toutes les exigences du lemme (à condition de choisir \(
    \hmpm* \) en conséquence) sauf peut-être le dernier point, pour lequel on
  utilise le même argument que pour la fin de la preuve du fait~3 page~276
  de~\cite{phiha1} cité précédemment, mais appliqué cette fois à la famille \(
    (D\pexp\clmap[\ind])^{\clmap\in\clmaps}_{\ind\in\set{0, \dots, n}} \).
\end{proof}

\begin{rem} \label{r:hm-clab-pm}
  On peut évidemment supposer que \( \hmclab* \ge \hmpm* \) pour tout \( v \)
  car le lemme~\vref{l:hmpm} n'est qu'un cas particulier du
  lemme~\vref{l:hmclab}.
\end{rem}

\begin{nota} \label{n:vaemb}
  On fixe désormais des constantes et formes telles que données par les deux
  lemmes précédents et le fait~\vref{f:htcmp}. On note de plus \( \hlclab = \ln
    \hmclab \) et \( \hlpm = \ln \hmpm \).
\end{nota}

\begin{rem} \label{r:vaemb}
  On supposera que \( n \ge 2 \) et \( N \ge 3 \), et le degré de (l'image de)
  \( \va \) dans le plongement est au moins \( 3 \), ce qui est le cas dans
  les plongements usuels : plongement de \bsc{Weierstrass} pour une courbe
  elliptique ou en général, plongement de \bsc{Mumford} modifié (voir
  ci-dessous).
\end{rem}

Discutons maintenant des valeurs explicites qu'on peut donner à certaines de
ces constantes pour des plongements bien choisis. On considère principalement
les plongements de \bsc{Mumford} modifiés tels que définis dans
\cite{daphimhva2}. Nous noterons \( \vaht \) la quantité notée \(
  \hautl{}\va \) dans cette référence, car dans nos notations, \( \hautl{}\va
\) désigne plutôt la hauteur de la forme de \bsc{Chow} de \( \va \) dans ce
plongement, et
\begin{equation} \label{e:vahtr}
  \vahtr = [\vacdn : \Q] \max( \vaht, 1 )
\end{equation}
où \( \vacdn \) est un corps de définition de \( \va \) plongée ; cette
quantité est celle notée \( h_0(\va) \) dans la référence en question
(théorème~1.4 page~641).

La proposition~3.9 page~665 de la référence citée dit alors qu'on peut choisir
\begin{equation}
  \htcmp = 4^{g+1} \vaht + 3g \ln 2
  \pmm.
\end{equation}
Par ailleurs, leur proposition~3.7 page~662 donne une famille totalement
explicite de formes représentant globalement le morphisme
d'addition-soustraction \( (x, y) \mapsto (x+y, x-y) \). On en déduit par
projection sur un des facteurs un système complet totalement explicite
représentant au choix l'addition ou la soustraction, avec un atlas à \( n+1 \)
cartes (qui sont plus précisément les préimages des hyperplans standard
par le morphisme de soustraction ou d'addition).

De plus, leur lemme~3.11 page~666 donne une famille représentant globalement
la multiplication par \( 2 \) et possédant des propriétés analogues à celles
exigées par le lemme~\vref{l:hmclab}. En rapprochant ces formules de celles
représentant le morphisme d'addition-soustraction (dont le carré est la
multiplication par deux sur les deux facteurs), il est possible d'expliciter
\( \hmpm* \) en fonction des constantes (explicites) données par le lemme~3.11
de la référence citée.

Le point d'achoppement pour finir d'expliciter \( \hmclab \) est donc
l'existence de formules globales pour la multiplication par \( 3 \) qui soient
suffisamment bien contrôlées. À notre connaissance, on ne dispose pas à ce
jour de formules explicites comme pour la multiplication par \( 2 \) ; il
serait néanmoins possible de majorer la hauteur d'un famille représentant
globalement la multiplication par \( 3 \), puis, grâce à un
\emph{nullstellensatz} effectif, d'en déduire une constante telle que la
minoration voulue soit satisfaite.

Dans le cas particulier des courbes elliptiques, si l'on choisit de les
plonger dans \( \proj2 \) par un plongement de \bsc{Weierstrass}, on dispose
alors de formules globales totalement explicites pour la multiplication par
n'importe quel entier, ainsi que d'un système complet de \( 3 \) familles de
formes de bidegré \( (2,2) \) représentant l'addition, données par la
section~2.13 (pages 126 à 142) de \cite{farhith}. On pourrait en déduire
une valeur totalement explicite de \( \hmclab \) pour les plongement de
\bsc{Weierstrass}. Par ailleurs, \bsc{Farhi} donne une valeur explicite de
\( \htcmp \) à partir de ces formules (théorème~2.13.3 page~141).

\medskip

Pour conclure cette section, remarquons que, si \( \htcmp \) est utilisée
assez abondamment dans tous les contextes, en revanche seule la démonstration
de l'inégalité de \bsc{Vojta} utilise \( \hlclab \) et \( N \) : l'inégalité
de \bsc{Mumford} repose uniquement sur \( \hlpm \) et n'utilise pas la valeur
de \( N \).


\subsection{Distances et conditions d'approximations}
\label{sec:distv}

Pour chaque place \( v \) de \( \cdn \), on pose
\begin{equation}
  \distv x y
  =
  \frac{ \nv2{ x \wedge y } }{ \nv2 x \nv2 y }
  \pmm.
\end{equation}
Cette définition correspond, pour les points, à la distance d'indice \( 1 \)
utilisée dans \cite[chap.~6]{nesphilnm} et \cite{jadotth}. Cette dernière
référence (notamment lemme~3.2 \textit{(ii)}, page 51) montre qu'il s'agit
bien d'une distance au sens usuel du terme (ultramétrique si \( v \) l'est).
Par ailleurs, il est évident à partir de la définition que la distance est
toujours majorée par \( 1 \).

Pour la distance entre un point et une variété, on s'écarte par contre
légèrement des références citée, en posant
\begin{equation}
  \distv x \avar
  =
  \min_{y \in \avar(\Cv)} \distv x y
  \pmm.
\end{equation}
Les deux notions sont en fait très comparables, voir la section suivante. La
notion qu'on vient de définir bénéficie par contre de la propriété suivante :
si \( \avar \subset \avar' \) alors \( \distv x {\avar'} \le \distv x \avar
\), qu'on utilisera couramment pour se ramener au cas où \( \avar \) est une
hypersurface. Évidemment, on a \( \distv x \avar = 0 \Leftrightarrow x \in
  \avar(\Cv) \).

\medskip

Soient \( \cdn \) un corps de nombres, \( \placesapx \) un ensemble fini de
places de \( \cdn \) et \( \eps > 0 \) un réel. Le but est d'étudier les
points satisfaisant à la condition condition d'approximation
\begin{equation} \label{e:ha-prod}
  \prod\placerange
  \distv x \avar ^\degv
  \le
  c
  \hautm2{x}^{-\eps}
\end{equation}
où \( c \) est une constante réelle et \( \degv = [\cdn_v : \Q_v] / [\cdn :
  \Q] \), de sorte que le membre de droite ne dépend pas du corps de nombre
(contenant \( x \)) utilisé pour le calculer. Autrement dit, une telle
condition a un sens sur \( \Qbar \).

Malheureusement, cette condition n'est pas facile à exploiter techniquement.
On introduit donc, comme il est d'usage, un autre type de condition faisant
apparaître une inégalité par place. Plus précisément, soient
\( \cdn_0 \) un corps de nombres,  \( \placesapx_0 \) un ensemble fini de
places de \( \cdn \) et \( (\wtapx)_{v \in \placesapx_0} \) une famille de
réels telle que \( \sum_{v\in\placesapx_0} \wtapx \degv(\cdn_0) = 1 \). On
utilisera des conditions d'approximations du type
\begin{equation} \label{e:ha-sys}
  \distv x \avar
  \le
  c_v
  \hautm2{x}^{-\wtapx\eps}
  \quad \forall v \in \placesapx_0
\end{equation}
où les \( c_v \) sont des constantes réelles. Il est clair
que~\eqref{e:ha-sys} est une hypothèse plus forte que~\eqref{e:ha-prod}, mais
on verra à la section~\vref{sec:ha-prod} qu'il suffit qu'un décompte sous
l'hypothèse forte implique en fait un décompte sous l'hypothèse plus faible.

Par ailleurs, il est facile de donner à cette condition un sens sur \( \Qbar
\), de la façon suivante. Si \( \cdn \) est une extension finie de \( \cdn_0
\), on note \( \placesapx
\) l'ensemble des places \( v \) de \( \cdn \) qui sont au-dessus d'un
élément \( v_0 \) de \( \placesapx_0 \) et on pose \( \wtapx[v] = \wtapx[v_0] \)
ainsi que \( c_v = c_{v_0} \). On a toujours
\begin{equation} \label{e:wtapx-1}
  \sum\placerange \wtapx \degv = 1
\end{equation}
et il est clair que la condition~\eqref{e:ha-sys} est équivalente à
\begin{equation} \label{e:ha-sys-ext}
  \distv x \avar
  \le
  c_v
  \hautm2{x}^{-\wtapx\eps}
  \quad \forall v \in \placesapx
  \pmm.
\end{equation}

\begin{rem} \label{r:ha-cdn}
  Vu que toutes les conditions considérées ont un sens sur \( \Qbar \), dans
  la suite on se préoccupera en général assez peu du corps de nombres utilisé.
  Sauf indication contraire, \( \cdn \) désignera un corps de nombres « assez
  grand », contenant en particulier un corps de définition des approximations
  exceptionnelles considérées, et les notations \( \placesapx \) et \( \wtapx
  \) auront le sens ci-dessus ; en particulier on supposera toujours
  que~\eqref{e:wtapx-1} est satisfait sans forcément le rappeler.

  En particulier, dans l'ensemble de ce mémoire, tous les objets considérés
  (points, variétés, etc.) sont supposés définis sur \( \Qbar \) sauf mention
  explicite du contraire.
\end{rem}


\section{Compléments sur les distances} \label{sec:distv-cmp}

On commence par établir une expression simple de la distance entre un point et
un hyperplan, cas qui sera fondamental pour l'inégalité de \bsc{Vojta} par
exemple, puis on étudie les liens entre la distance définie ci-dessus et celle
utilisée notamment dans~\cite[chap.~6 et~7]{nesphilnm}. Certains des résultats
établis ici ne sont pas utiles pour le reste de la thèse\footnote{Plus
  précisément, les seuls résultats utilisés sont la
  proposition~\vref{p:dv-hp}, le lemme~\vref{l:dv-common-i} et la
  proposition~\vref{p:dv-p2alg-hs}.} mais sont inclus afin de donner une
vision assez complète des relations entre ces deux notions.

Par ailleurs, dans toute cette section, contrairement au reste du texte, les
objets (points, variétés) considérés ne sont pas supposés définis sur \( \Qbar
\), les résultats sont valables pour des objets définis sur \( \Cv \). En
revanche, les résultats d'existence (fait~\vref{f:closest-point} en
particulier) ne permettent en aucun cas de garantir le caractère algébrique
des objets obtenus.

\begin{prop} \label{p:dv-hp}
  Si \( E \) est un hyperplan de \( \projd \), d'équation \( L \), on a
  \begin{equation}
    \distv x E
    =
    \frac{ \av{L(x)} }{ \nv2 L \nv2 x }
    \quad \forall x \in \projd(\Cv)
    \pmm.
  \end{equation}
\end{prop}

\begin{proof}
  On commence par se ramener au cas \( L = X_0 \). Pour cela, si \( v \)
  est ultramétrique, on commence par choisir un coefficient de \( L \) de
  valeur absolue maximale : quitte à permuter les coordonnées, on peut
  supposer que c'est celui d'indice \( 0 \). En divisant par ce coefficient,
  on se ramène au cas \( L = X_0 + l_1 X_1 + \dots + l_n X_n \) avec \(
    \av{ l_i } \le 1 \) entiers.  Il est alors facile de voir que le
  changement de coordonnées donné par \( x'_0 = L(x) \) et \( x'_i = x_i \)
  pour \( i \ge 1 \) est une isométrie (son inverse est également à
  coefficients de module inférieur ou égal à \( 1 \)), de sorte qu'on s'est
  bien ramené au cas \( L = X_0 \) sans modifier les quantités apparaissant
  dans chacun des deux membres, ni perdre en généralité.

  Aux places archimédiennes, on plonge \( \cdn_v \) dans \( \C \) et on
  note \( \overline E \) le relevé de \( E \) dans \( \C^{n+1} \). On munit ce
  dernier de la forme hermitienne standard et on considère un vecteur unitaire
  normal à \( \overline E \), qu'on note \( a_0 \). On complète ce vecteur en
  un base orthonormée et on considère le changement de coordonnées qui envoie
  cette base sur la base canonique : étant une transformation unitaire, il ne
  modifie aucun des deux membres de l'égalité à prouver, et nous ramène bien
  au cas \( L = X_0 \).

  Par ailleurs, l'égalité qu'on cherche à montrer est banale si \( x \in
    E(\Cv) \). On peut donc supposer que ce n'est pas le cas, et même
  choisir \( (1, x_1, \dots, x_n) \) comme représentant de \( x \). Le membre
  de droite est alors égal à \( \nv2 x ^{-1} \).

  Soit maintenant \( y = (0, y_1, \dots, y_n) \in E(\Cv) \). On a
  \( x \wedge y = (y_1, \dots, y_n, x_1 y_2 - x_2 y_1, \dots) \) et donc
  \( \nv2{ x \wedge y } \ge \nv2 y \), avec égalité si et seulement si \(
    (y_1, \dots, y_n) \) est colinéaire à \( (x_1, \dots, x_n) \).  Ainsi,
  pour tout \( y \in E(\Cv) \) on a \( \distv x y \ge \nv2 x ^{-1} \) et
  l'égalité est atteinte, ce qui achève la preuve.
\end{proof}

Remarquons que la distance entre points et variétés définie ici ne dépend en
fait que de l'ensemble des points de la variété mais pas de la structure
géométrique (multiplicités) de celle-ci. Une autre notion de distance,
développée dans \cite{jadotth} et \cite[chap.~6]{nesphilnm} pour le cas
projectif, et \cite[chap.~7]{nesphilnm} pour le cas multiprojectif (qui
englobe bien sûr le précédent) prend au contraire en compte cette structure.
Cette distance est définie par
\begin{equation}
  \distalgv x \avar
  =
  \frac{
    \mvp{\md_x \chow\avar}
  }{
    \mvp{\chow\avar} \nv2 x ^{\adeg(\adim+1)}
  }
\end{equation}
où \( \chow\avar \) est une forme de \bsc{Chow} de \( \avar \), son degré est
noté \( \adeg \) et sa dimension \( \adim \). Le morphisme \( \md_x \) est
défini dans les références citées (p.~88 de \cite{nesphilnm} par exemple) ;
par ailleurs, par rapport à ces références, on considère uniquement la
distance d'indice \( (1, \dots, 1) \). Comme pour la notion de distance
précédemment définie, on a \( 0 \le \distv x \avar \le 1 \) avec égalité à
gauche si et seulement si \( x \in \avar \) (voir la référence précédente).

Nous appellerons cette distance \emph{algébrique}, par opposition à la
distance ensembliste définie ci-dessus. Examinons un peu les relations entre
ces deux notions. Tout d'abord, dans le cas où \( \avar = \set y \) est
réduite à un point, on a \( \distv x y = \distalgv x \avar \) d'après \cite[p.
50]{jadotth}.  Par ailleurs, dans le cas où \( \avar \) est une hypersurface
de degré \( \Delta \) et d'équation \( F \), la proposition~3.6 (p.~64) de
cette même référence montre que
\begin{equation}
  \distalgv x \avar
  =
  \frac{ \av{F(x)} }{ \mvp F \nv2 x ^\Delta }
  \pmm.
\end{equation}
Si \( \avar \) est en fait un hyperplan, on a \( \mvp F = \nv2 F \) d'après la
proposition~4, p.~266 de~\cite{phiha1} (reprise dans la proposition~2.3,
point~(ii), p.~23 de \cite{jadotth} sous une forme plus proche de nos
notations), de sorte que la proposition~\vref{p:dv-hp} dit en fait simplement
que \( \distalgv \truc \avar = \distv \truc \avar \) dans ce cas particulier.

Dans le cas général, on a la comparaison suivante.
\begin{fact} \label{f:closest-point}
  Pour toute variété \( \avar \subset \Proj^n \) et tout point \( x \in
    \Proj^n(\C_v) \), il existe un point \( y \in \avar(\C_v) \) tel que \(
    \Distv(x, y) \le \Distv(x, \avar)^{1/\deg \avar} e^{\dv\gamma_{n+1}} \).
\end{fact}

\begin{proof}
  Pour les places infinies, c'est la \eng{closest point property} p.~89
  de~\cite{nesphilnm}. Pour les places finies, on constate que la preuve
  s'adapte, car le résultat crucial (lemme~5.2 de la référence citée) est en
  fait vrai aux places finies sans le dernier facteur (et en remplaçant
  évidemment la mesure par la norme du sup).
\end{proof}

Dans l'autre sens, on a le résultat suivant.
\begin{prop} \label{p:dv-p2alg}
  Pour toute variété \( \avar \subset \projd \) et tous points \( x \in
    \projd(\Cv) \) et \( y \in \avar(\Cv) \), on a
  \begin{equation}
    \distalgv x \avar
    \le
    \distv x y ^m
    \bigl( 2 (n+1)^{3/2} \bigr)^{(m + 3ld) \dv}
  \end{equation}
  où \( l = \dim \avar + 1 \), \( \adeg = \deg \avar \) et \( m \) est la
  multiplicité de
  \( \avar \) en \( y \).
\end{prop}

La démonstration repose sur un développement de \( \md f \) autour de \( y
\) relativement à \( X \), qui permet de quantifier le fait que, \( \md_y f \)
étant nul, \( \md_x f \) doit être petit pour \( x \) voisin de \( y \).
De façon générale, pour écrire un développement de \bsc{Taylor} , il faudra
s'assurer de l'existence d'une carte affine \( X_i \neq 0 \) contenant à la
fois \( x \) et \( y \). On voit facilement que la non-existence d'un telle
carte affine impliquerait \( \distv x y  = 1 \), rendant le résultat banal. En
pratique, on pourra supposer sans problème que \( x \) et \( y \) sont
suffisamment proches.  Par ailleurs, il sera utile de pouvoir de plus choisir
un indice \( i \) de sorte que \( \av{x_i} \) (resp.  \( \av{y_i} \)) soit
comparable  à \( \nv2 x \) (resp. \( \nv2 y \)). Le lemme élémentaire suivant
montre que c'est possible.

\begin{lem} \label{l:dv-common-i}
  Pour chaque \( x \in \projd \), il existe \( i \in \set{0, \dots, n} \) tel
  que \( \frac{ \av{x_i} }{ \nv2 x } \ge (\frac1{\sqrt{n+1}})^\dv \). De
  plus, pour le même indice \( i \), pour tout point \( y \) satisfaisant
  \( \distv x y  \le \eps_0 \) avec \( \eps_0 < (\frac{1}{\sqrt{n+1}})^\dv \),
  on a
  \begin{equation}
    \frac{\av{y_i}}{\nv2 y}
    \ge
    \left( \frac{\sqrt{n+1}}{n+2} - \eps_0 \right)^\dv
    \pmm.
  \end{equation}
  En particulier, si \( \eps_0 < (\frac1{4\sqrt{n+1}})^\dv \) on a \(
    \frac{\av{y_i}}{\nv2 y} \ge (\frac1{2\sqrt{n+1}})^\dv \) et \( y_i \neq 0
  \).
\end{lem}

\begin{proof}
  \newcommand*\hatxs {\smash{\hat x}}
  \newcommand*\hatys {\smash{\hat y}}
  Pour le premier point, il suffit de choisir \( i \in \set{ 0, \dots, n } \)
  maximisant \( \av{x_i} \). Afin d'alléger les notations, on supposera par
  la suite \( i = 0 \). Pour chaque \( z = (z_0, \dots, z_n) \) on notera \(
    \hat{z} = (z_1, \dots, z_n) \) le vecteur obtenu en omettant la première
  coordonnée. On a alors, vu l'hypothèse sur \( x \) et la définition de la
  distance :
  \begin{equation} \label{e:dvci-hat}
    \eps_0 \nv2 y \nv2 x
    \ge
    \nv2{y \wedge x}
    \ge
    \nv2{x_0 \hatys - y_0 \hatxs}
    \pmm,
  \end{equation}
  où la deuxième inégalité vient en remarquant que toutes les coordonnées de
  \( x_0 \hatys - y_0 \hatxs \) apparaissent également comme coordonnées de \(
    y \wedge x \).

  On traite d'abord le cas ultramétrique, par l'absurde. En effet, si on avait
  \( \av{y_0} < \nv2{\hatys} = \nv2{y} \), il viendrait \( \nv2{x_0 \hatys} >
    \nv2{y_0 \hatxs} \) et la propriété ultramétrique appliquée au dernier
  membre de \eqref{e:dvci-hat} donnerait \( \eps_0 \nv2{y}\nv2{x}  \ge
    \av{x_0}\nv2{\hatys} \), puis \( \eps_0 \ge 1 \) contrairement aux
  hypothèses.

  Pour le cas archimédien, l'inégalité triangulaire dans le membre de droite
  de \eqref{e:dvci-hat} donne
  \( \eps_0 \nv2{y}\nv2{x}  \ge \av{x_0}\nv2{\hatys} - \av{y_0}\nv2{\hatxs}
  \).  En divisant par \( \nv2{y}\nv2{x} \) puis en remarquant que \(
    \nv2{\hatxs}/\nv2{x} \le 1 \) il vient :
  \begin{align}
    \frac{\av{y_0}}{\nv2{y}}
    \ge
    \frac{\nv2{\hatxs}}{\nv2{x}}
    \cdot \frac{\av{y_0}}{\nv2{y}}
    \ge
    \frac{\nv2{\hatys}}{\nv2{y}}
    \cdot \frac{\av{x_0}}{\nv2{x}}
    - \eps_0
    \ge
    \frac{\nv2{\hatys}}{\nv2{y}}
    \cdot
    \frac{1}{\sqrt{n+1}}
    - \eps_0
    \pmm.
  \end{align}
  Notons \( t = \av{y_0}/\nv2{y} \in [0; 1] \) ; comme \( \nv2{y}^2 =
    \av{y_0}^2 + \nv2{\hatys}^2 \), on réécrit l'inégalité précédente sous
  la forme \( t \ge \sqrt{\frac{1-t^2}{n+1}}-\eps_0 \), ou encore :
  \begin{equation}
    \left( \frac{n+2}{n+1} \right) t^2
    + 2t\eps_0
    + \eps_0^2
    - \frac{1}{n+1}
    \ge
    0
    \pmm,
  \end{equation}
  qui, compte tenu de l'hypothèse sur \( \eps_0 \) et de la positivité de \( t
  \), implique :
  \begin{align*}
    t
    & \ge
    \frac{n+1}{n+2}
    \sqrt{ \frac{-\eps_0^2}{n+1}+\frac{n+2}{(n+1)^2} }
    - \left( \frac{n+1}{n+2} \right) \eps_0
    \\ & \ge
    \frac{\sqrt{n+1}}{n+2} - \eps_0
    \pmm,
  \end{align*}
  comme annoncé. Le cas particulier s'en déduit immédiatement en substituant.
\end{proof}

\begin{proof}[\proofname{} de la proposition~\vref{p:dv-p2alg}.]
  Notons pour commencer qu'on peut supposer \( \dim \avar \ge 1 \) (donc \( n
    \ge 2 \)) car on sait que les deux distances sont égales si \( \avar \)
  est un point. On peut de plus supposer que \( \distv x y  < (n+1)^{-9\dv/2}
  \) car sinon la conclusion du lemme est vide, la distance étant de toutes
  façons bornée par \( 1 \). (On peut en fait supposer une majoration plus
  forte, mais celle-ci est largement suffisante.) Les hypothèses du lemme
  précédent sont satisfaites ; en l'utilisant (et quitte à renuméroter) on
  peut supposer que \( y_0 = x_0 = 1 \), \( \nv2 y \le (\sqrt{n+1})^\dv \) et
  \( \nv2 x \le (2 \sqrt{n+1})^\dv \).

  Par ailleurs, \( \nv2 y \le \nv2 x + \nv2 {y-x} \le \nv2 x + \nv2{x \wedge
      y} \), car chaque coefficient de \( x - y \) est aussi un coefficient de
  \( x \wedge y \).  En substituant \( \nv2{x \wedge y} = \distv x y  \nv2 x
    \nv2 y \) et en divisant l'inégalité obtenue par \( \nv2 x \), compte tenu
  de l'hypothèse sur la distance, il vient \( \nv2 y / \nv2 x \le 1 +
    (n+1)^{-4} \le 82/81 \).

  On reprend les notations de la section~4 (page 117)
  de~\cite[chap.~7]{nesphilnm} concernant le morphisme \( \md \).  On regarde
  \( \md f \) comme un polynôme en \( s \) à coefficients dans \( k[X] \) et
  on appelle \( \mathcal G \) la famille de ses coefficients ; chacun d'entre
  eux est homogène de degré \( ld \) et s'annule en \( y \).  Plus
  précisément, d'après \cite[prop.~3]{phitzee}, la multiplicité de \( \avar \)
  en \( y \) est le plus grand entier \( k \) tel que les dérivées d'ordre
  total \( k-1 \) de toutes les formes de la famille \( \mathcal G \) soient
  nulles en \( y \).  Le développement d'une forme \( G \in \mathcal G \)
  autour du point \( y \) s'écrit donc :
  \begin{equation}
    G(x)
    =
    \sum_{k=m}^{ld} \ \overbrace{
      \sum_{\substack{\alpha \in \N^n \\ \lgr\alpha = k}}
      \underbrace{
        \frac 1{\alpha!} \frac{\partial^k G}{\partial X^\alpha} (y)
        \prod_{1 \le i \le n} (x_i - y_i)^{\alpha_i}
      }_{\textstyle R_{k, \alpha}}
    }^{\textstyle R_k}
    \pmm.
  \end{equation}

  On majore maintenant \( \av{G(x)} / \nv2 x ^{ld} \) en procédant terme à
  terme.
  \begin{align*}
    \frac {\nv2{R_{k, \alpha}}} {\nv2 x ^{ld}}
    & \le
    \nv*2{ \frac1{\alpha!} \frac{\partial^k G}{\partial X^\alpha} }
    \nv2 y^{ld-k}
    \nv2{x \wedge y}^k
    / \nv2 x^{ld}
    \\ & \le
    2^{ld\dv} \nv2 G \nv2 y^{ld-k} \nv2 x^k \nv2 y^k
    \distv x y ^k
    / \nv2 x^{ld}
    \\ & \le
    \left( \frac{164}{81} \right)^{ld\dv}
    \nv2 G
    \distv x y ^k
    (\sqrt{n+1})^{k\dv}
    \pmm.
  \end{align*}
  (Pour la dernière estimation, on a majoré \( \nv2 y \) par \( \sqrt{n+1} \) et
  \( (\nv2 y / \nv2 x)^{ld -k} \) par \( (82/81)^{ld} \).)
  On remarque alors qu'il y a au plus \( (n+1)^k \) termes dans \( R_k \) :
  \begin{equation}
    \frac{ \nv2{R_k} }{ \nv2 x^{ld} }
    \le
    \left( \frac{164}{81} \right)^{ld\dv}
    \nv2 G
    \distv x y ^k
    (n+1)^{3k\dv/2}
    \pmm,
  \end{equation}
  puis, si \( v \) est finie, \( \nv2{ G(x) } / \nv2 x ^{ld} \le \nv2 G
    \distv x y ^m \) par l'inégalité ultramétrique et, si \( v \) est
  infinie :
  \begin{align*}
    \frac{\nv2{G(x)}}{\nv2 x^{ld}}
    & \le
    \left( \frac{164}{81} \right)^{ld}
    \nv2 G
    \sum_{k=m}^{ld}
    \distv x y ^k (n+1)^{3k/2}
    \\ & \le
    \left( \frac{164}{81} \right)^{ld}
    \nv2 G
    \distv x y ^m (n+1)^{3m/2}
    % \\ & \phantom{\le} \qquad
    \cdot \sum_{k=0}^{ld-m} \bigl(\distv x y  (n+1)^{3/2}\bigr)^k
    \pmm,
  \end{align*}
  où la dernière somme est majorée par \( 27/26 \), d'où finalement :
  \begin{equation} \label{e:dvpa-comp}
    \frac{ \nv2{\mathcal G(x)} }{\nv2 x^{ld}}
    \le
    \nnv2{ \mathcal G }
    \distv x y ^m
    \left(
      \frac{27}{26} (n+1)^{3m/2}
      \left( \frac{164}{81} \right)^{ld}
    \right)^\dv
    \pmm.
  \end{equation}
  La fin de la démonstration consiste alors en des comparaisons de normes et
  mesures, qu'on peut résumer ainsi :
  \begin{equation}
    \frac{ \mvp{\md_x f} }{ \nv2 x ^{ld} }
    \ll
    \frac{ \nv2{\md_x f} }{ \nv2 x ^{ld} }
    =
    \frac{ \nv2{\mathcal G(x)} }{ \nv2 x ^{ld} }
    \ll
    \nnv2{ \mathcal G }
    =
    \nv2{ \md f }
    \ll
    \mvp{\md f}
    \ll
    \mvp f
    \pmm.
  \end{equation}
  Plus précisément, vu la définition de la mesure et une majoration facile de
  l'intégrale dans cette définition, on a
  \begin{equation}
    \mvp{\md_x f}
    \le
    \nv2{\md_x f}
    \cdot \exp(\dv \cdot ld \gamma_{(n+1)n/2})
  \end{equation}
  où l'on a noté \( \gamma_j = \sum_{i=1}^j \frac1{2i} \). La deuxième
  majoration est donnée par~\eqref{e:dvpa-comp} ci-dessus ; pour la troisième
  on a
  \begin{equation}
    \nv2{ \md f }
    \le
    \mvp{\md f}
    \cdot (n(n+1)^2/2)^{ld\dv}
  \end{equation}
  d'après~\eqref{e:l1<mespph}. Pour la dernière, on utilise le
  fait que la distance est bornée par \( 1 \), autrement dit que
  \( \mvp{\md_x f} \le \mvp f \) dès que \( \nv2 x = 1 \) : en reportant ceci
  dans la définition de \( \mvp\truc \) et en intégrant, il vient facilement
  \begin{equation}
    \mvp{\md f}
    \le
    \mvp f
    \cdot \exp(\dv\cdot ld \gamma_{n+1})
    \pmm.
  \end{equation}

  On met alors ces estimations bout à bout et on utilise la majoration
  classique \( \gamma_n \le (1 + \ln n)/2 \) pour obtenir
  \begin{align}
    \distalgv x \avar
    & \le
    \exp(\dv \cdot ld \gamma_{(n+1)n/2})
    \cdot
    \distv x y ^m
    \left(
      \frac{27}{26} (n+1)^{3m/2}
      \left( \frac{164}{81} \right)^{ld}
    \right)^\dv
    \\ & \hphantom{\le} \quad
    \cdot (n(n+1)^2/2)^{ld\dv}
    \cdot \exp(\dv\cdot ld \gamma_{n+1})
    \\ & \le
    \distv x y ^m
    \left(
      \frac{27}{26} (n+1)^{3m/2}
    \right)^\dv
    \\ & \hphantom{\le} \quad
    \cdot \left(
      \frac{164}{81}
      \sqrt{\frac{ \expb n(n+1) }2}
      \, \frac{ n(n+1)^2 }2
      \, \sqrt{ \expb (n+1) }
    \right)^{ld\dv}
    \\ & \le
    \distv x y ^m
    \left(
      \bigl( 2(n+1)^{3/2} \bigr)^m
      \bigl( 2(n+1)^{9/2} \bigr)^{ld}
    \right)^\dv
  \end{align}
  en remarquant que \( 82 \expb / (81 \sqrt2) \le 2 \), ce qui achève la
  preuve.
\end{proof}

Les deux dernières propositions permettent donc de comparer la distance
algébrique et la distance ponctuelle en toute généralité. On voit bien
intervenir la structure géométrique de la variété \lat{via} son degré ou sa
multiplicité en un point (évidemment majorée par le degré) respectivement. En
conséquence, le corollaire suivant est assez naturel : il montre que les deux
distances coïncident dans le cas d'un sous-variété linéaire, ce qu'on savait
déjà dans les cas particulier des points et des hyperplans.

\begin{coro}
  Si \( \avar \) est une sous-variété linéaire de \( \projd \), pour tout
  point \( x \in \projd(\Cv) \) on a \( \distv x \avar = \distalgv x \avar \).
\end{coro}

\begin{proof}
  Si \( v \) est finie, c'est une conséquence immédiate des deux propositions
  précédentes. Sinon, on utilise la proposition~3.10, p.~76 de \cite{jadotth}
  et sa preuve, qui montre qu'on peut supposer que \( \avar \) est définie par
  \( \vp[0] = \dots = \vp* = 0 \) et que \( x = (0, \dots, x_\ind, x_{\ind+1},
    0, \dots, 0) \). Si l'on pose \( y = (0, \dots, 0, 1, 0, \dots, 0) \) où
  le \( 1 \) est en \( (\ind+1) \)-ième position (en particulier, \( y \in
    \avar \)), on a
  \begin{equation}
    \distv x y
    =
    \frac{ \nv2{ x \wedge y } }{ \nv2 x \, \nv2 y }
    =
    \frac{ \av{ x_\ind } }{ \nv2 x }
    =
    \distalgv x \avar
  \end{equation}
  où la dernière égalité est le résultat de la proposition citée. Par
  ailleurs, il est clair que pour tout \( y \in \avar \), c'est-à-dire
  \( y = (0, \dots, 0, y_{\ind+1}, \dots, y_n) \),  on a
  \begin{equation}
    \distv x y ^2
    =
    \frac{
      \av{ x_\ind }^2 \nv2 y ^2
      + \av{ x_{\ind+1} } \sum_{\indi=\ind+2}^n \av{ y_\indi }^2
    }{
      \nv2 x ^2 \nv2 y ^2
    }
    \ge
    \frac{ \av{ x_\ind }^2 }{ \nv2 x ^2 }
    =
    \distalgv x \avar ^2
  \end{equation}
  ce qui achève la preuve.
\end{proof}

Par ailleurs, dans le cas particulier où \( \avar \) est une hypersurface, on
peut obtenir une version un peu plus précise de~\vref{p:dv-p2alg} en utilisant
l'expression particulière de la distance algébrique dans ce cas. Cette version
nous sera utile par exemple pour démontrer une inégalité de \bsc{Liouville}
(section~\vref{sec:liouville}).

\begin{prop} \label{p:dv-p2alg-hs}
  Pour toute hypersurface \( \avar \subset \projd \) et tous points \( x \in
    \projd(\Cv) \) et \( y \in \avar(\Cv) \), on a
  \begin{equation}
    \distalgv x \avar
    \le
    \distv x y ^m
    \, \bigl( (17/8)^\adeg (n+1)^{\adeg + 3m/2} \bigr)^\dv
  \end{equation}
  où \( \adeg = \deg \avar \) et \( m \) est la multiplicité de \( \avar \) en \(
    y \).
\end{prop}

\begin{proof}
  On procède comme pour la propriété précédente, sauf que cette fois-ci on
  peut seulement supposer que \( \distv x y \le (4 (n+1)^{7/2})^{-\dv} \),
  dont on déduit également que \( \nv2 y / \nv2 x \le 82/81 \). On développe
  alors une équation \( G \) de \( \avar \), et non plus un coefficient de \(
    \md f \) ; on obtient ainsi, au lieu de~\eqref{e:dvpa-comp}, la majoration
  suivante :
  \begin{equation} \label{e:dv-p2alg-hs}
    \frac{ \av{G(x)} }{\nv2 x^{\adeg}}
    \le
    \nv2{ G }
    \distv x y ^m
    \left(
      \frac{27}{26} (n+1)^{3m/2}
      \left( \frac{164}{81} \right)^{\adeg}
    \right)^\dv
    \pmm.
  \end{equation}
  On utilise à nouveau~\eqref{e:l1<mespph}, qui donne
  \( \nv2 G \le \mvp G (n+1)^\adeg \), pour conclure.
\end{proof}


\cleardoublepage
\endinput

% vim: spell spelllang=fr
