% préambule général pour tous les trucs préparatoires à la thèse
% (en attendant un classe ou une extension un peu plus propre ?)
%
\documentclass[a4paper, 11pt, twoside, BCOR=1cm]{scrartcl}

\setlength{\overfullrule}{5pt}

\usepackage[utf8]{inputenc}
\usepackage[T1]{fontenc}

\usepackage[sc, osf]{mathpazo} \linespread{1.05}
\usepackage{amssymb}
\renewcommand\sfdefault{iwona}

\usepackage{mathtools, amsthm}
\usepackage[all]{xy}

\usepackage{ifmtarg, fixltx2e, mparhack, xargs}
\usepackage{enumitem}
\usepackage{git-info, eso-pic, xcolor}

\usepackage{xspace}
\usepackage[expansion=false, babel=true]{microtype}
\usepackage[british, frenchb]{babel}
\frenchbsetup{og=«,fg=»,
  AutoSpacePunctuation=false,
  SuppressWarning=true}

\usepackage[colorlinks=true, urlcolor=black, linkcolor=black, citecolor=black,
  bookmarks=true, bookmarksnumbered=true]{hyperref}

% use git information
\AddToShipoutPicture{%
  \AtPageLowerLeft{%
    \parbox[b]{\paperwidth}{%
      \centering \itshape \color{black!30}%
      \jobname.tex $\le$ \gitinfo{@head}{dt}%
      \par \kern2cm
      }}}

% provide generic title information
\author{Manuel \bsc{P\'egouri\'e-Gonnard}}
\subtitle{Notes de travail}
\date{}
\hypersetup{pdfauthor={Manuel Pégourié-Gonnard}}

\newcommand*\notemarge[1]{\marginpar[\raggedleft #1]{\raggedright #1}}
\newcommand\todotext{\textsc{todo}}
\makeatletter
  \newcommand*\todo[1][]{%
    \leavevmode\notemarge{\todotext}%
  \@ifnotmtarg{#1}{\emph{#1}}}
  \newcommand*\todom[1][]{\tag{\todotext%
    \@ifnotmtarg{#1}{ : #1}}}
\makeatother
%\newcommand\todo{\TextOrMath{\todot}{\todom}}

\newenvironment{enumthm}
  {\begin{enumerate}[label=(\textit{\roman*})]}
  {\end{enumerate}}

\newcommand*\lat[1]{\emph{#1}}
\newcommand*\eng[1]{%
  \foreignlanguage{english}{\emph{#1}}}
\newcommand*\defn[1]{\emph{#1}}
\newcommand*\pmm[1]{\text{ #1}}

\makeatletter
\newcommand*\ssub[1]{%
  \@ifnotmtarg{#1}{_{#1}}}
\newcommand*\ssup[1]{%
  \@ifnotmtarg{#1}{^{#1}}}
\newcommandx*\pexp[2][2]{%
  \@ifmtarg{#2}{%
    \@ifnotmtarg{#1}{\cramped{^{(#1)}}}%
    }{%
    \@ifnotmtarg{#1}{^{(#1)}}_{#2}}}
\makeatother

\newcommand\suchthat{\ \middle\vert\ }

\newcommand*\std[1]{\mathbf{#1}} \newcommand\N{\std N} \newcommand\Z{\std Z}
\newcommand\Q{\std Q} \newcommand\R{\std R} \newcommand\C{\std C}
\newcommand\Proj{\std{P}} \newcommand\Aff{\std{A}}
\newcommand\Qbar{\overline{\Q}}
\newcommand\cdn{\boldsymbol{k}} \newcommand\Cdn{\boldsymbol{K}}
\newcommand*\I[1]{\std{Id}_{#1}} \newcommand*\ind[1]{\std{1}_{#1}}
\renewcommand\ge{\geqslant} \renewcommand\le{\leqslant}
\newcommand\orbrack{\mathopen\rbrack} \newcommand\clbrack{\mathopen\lbrack}

\newcommand*\abs[1]{\left\lvert#1\right\rvert}
\newcommand*\norm[1]{\left\lVert#1\right\rVert}
\newcommand*\nnorm[1]{%
  \left\lvert\hspace{-1pt}\left\lvert\hspace{-1pt}%
  \left\lvert#1\right\rvert
  \hspace{-1pt}\right\rvert\hspace{-1pt}\right\rvert}

\newcommand\eps{\varepsilon}
\newcommand\truc{{\,\cdot\,}}
\DeclareMathOperator\disc{Disc}
\DeclareMathOperator\ord{ord}
\DeclareMathOperator\Div{div}

\newcommand\zeros{\mathcal Z}
\newcommand\ideal{\mathcal I}

\newcommand\mmax{{\mathrm{max}}}

\newcommand\diff{\mathrm d}

\newtheorem {thm}        {Théorème}
\newtheorem {prop} [thm] {Proposition}
\newtheorem {lem}  [thm]  {Lemme}
\newtheorem {coro} [thm] {Corollaire}
\newtheorem {fact} [thm] {Fait}
\newtheorem {scho} [thm] {Scholie}
\newtheorem*{flemme}     {À démontrer plus tard}
  \theoremstyle{definition}
\newtheorem {tdef} [thm] {Définition}
  \theoremstyle{remark}
\newtheorem*{rem}        {Remarque}
\newtheorem*{Rem}        {Remarque}
