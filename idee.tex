\documentclass[a4paper, 11pt, twoside]{article}
\usepackage[utf8]{inputenc} % àéïôù = aeiou %

\usepackage{svn-multi}
\svnid{$Id$}
\newcommand\svnnatdate{%
  \begingroup \year\svnyear \month\svnmonth \day\svnday \relax 
  \today \endgroup}

\usepackage[T1]{fontenc}
\usepackage{textcomp}
\usepackage[sc, osf]{mathpazo}
\usepackage[scaled]{helvet}
\usepackage{amssymb}

% préambule général pour tous les trucs préparatoire à la thèse
% (en attendant un classe ou une extension un peu plus propre ?)
%

\usepackage{mathtools, amsmath, amsthm}
\usepackage[all]{xy}

\usepackage{ifmtarg, fixltx2e, xargs}

\usepackage{enumitem}

\usepackage{fancyhdr} \pagestyle{fancy}
\renewcommand\headrulewidth{0pt} \setlength\headheight{0pt}
\fancyhead{} \fancyfoot{}
\fancyfoot[C]{\thepage}
% \fancyfoot[RO, LE]{\today}

\usepackage{xspace}
\usepackage[british, frenchb]{babel}
\usepackage[babel=true, expansion=false]{microtype}
\frenchbsetup{AutoSpacePunctuation=false}

\newcommand*\notemarge[1]{\marginpar[\raggedleft #1]{\raggedright #1}}
\newcommand\todotext{\textsc{todo}}
\makeatletter
  \newcommand*\todo[1][]{%
    \leavevmode\notemarge{\todotext}%
    \@ifnotmtarg{#1}{[#1]}}
  \newcommand*\todom[1][]{\tag{\todotext%
    \@ifnotmtarg{#1}{ : #1}}}
\makeatother
%\newcommand\todo{\TextOrMath{\todot}{\todom}}

\newenvironment{enumthm}
  {\begin{enumerate}[label=(\textit{\roman*})]}
  {\end{enumerate}}

\newcommand*\lat[1]{\emph{#1}}
\newcommand*\eng[1]{%
  \foreignlanguage{english}{\emph{#1}}}
\newcommand*\defn[1]{\emph{#1}}
\newcommand*\pmm[1]{\text{ #1}}

\makeatletter
\newcommand*\ssub[1]{\@ifnotmtarg{#1}{_{#1}}}
\newcommand*\ssup[1]{\@ifnotmtarg{#1}{^{#1}}}
\newcommandx*\pexp[2][2]{%
  \@ifmtarg{#2}%
    {\cramped{^{(#1)}}}%
    {^{(#1)}_{#2}}}
\makeatother

\newcommand\suchthat{\ \middle\vert\ }

\newcommand*\std[1]{\mathbf{#1}} \newcommand\N{\std N} \newcommand\Z{\std Z}
\newcommand\Q{\std Q} \newcommand\R{\std R} \newcommand\C{\std C}
\newcommand\Proj{\std{P}} \newcommand\Aff{\std{A}} 
\newcommand\Qbar{\overline{\Q}}
\newcommand\cdn{\boldsymbol{k}} \newcommand\Cdn{\boldsymbol{K}}
\newcommand*\I[1]{\std{Id}_{#1}} \newcommand*\ind[1]{\std{1}_{#1}}
\renewcommand\ge{\geqslant} \renewcommand\le{\leqslant}
\newcommand\orbrack{\mathopen\rbrack} \newcommand\clbrack{\mathopen\lbrack}

\newcommand*\abs[1]{\left\lvert#1\right\rvert}
\newcommand*\norm[1]{\left\lVert#1\right\rVert}
\newcommand*\nnorm[1]{%
  \left\lvert\hspace{-1pt}\left\lvert\hspace{-1pt}%
  \left\lvert#1\right\rvert
  \hspace{-1pt}\right\rvert\hspace{-1pt}\right\rvert}

\newcommand\eps{\varepsilon}
\newcommand\truc{{\,\cdot\,}}
\DeclareMathOperator\disc{Disc}
\DeclareMathOperator\ord{ord}
\DeclareMathOperator\Div{div}

\newcommand\zeros{\mathcal Z}
\newcommand\ideal{\mathcal I}

\newcommand\mmax{{\mathrm{max}}}

\newcommand\diff{\mathrm d}

% objets mathématiques utiles
% 
\newcommand\va{\mathcal A}
\newcommand\point{x}
\newcommand\pointi{y}
\newcommand\coord{{\mathrm{x}}}
\newcommand\coordi{{\mathrm{y}}}
\newcommand\genre{g}
\newcommand\fibre{\mathcal L}
\newcommand\proj[2][]{%
  \Proj\ssup{#2}\ssub{#1}}
\newcommand\place{v}

\DeclareMathOperator\Haut{H}
\newcommand\haut[1]{%
  \Haut(#1)}

\newcommand\absv[2][\place]{%
  \abs{#2}\ssub{#1}}
\newcommand\normv[2][\place]{%
  \norm{#2}\ssub{#1}}
\newcommand\distv[3][\place]{%
  \mathop{\mathrm{dist}\ssub{#1}}(#2,#3)}

\newcommand\noref[1]{%
  [?#1?]}

\newtheorem{thm}{Théorème}

\renewcommand*\bsc{}
\frenchbsetup{og=«,fg=»}

\title{Approximation dans les variétés abéliennes}
\date{\svnnatdate}
\author{Manuel \bsc{Pégourié-Gonnard}}

\begin{document}

\maketitle

\section{L'énoncé du problème}

Soit $\va$ une variété abélienne, de dimension $\genre$, définie sur un corps
de nombres $\cdn$. On suppose selon ses goûts que $\va$ est
munie d'un fibré ample et symétrique $\fibre$, ou que $\va$ est plongée une
fois pour toute dans un espace projectif $\proj{n}$ par un plongement associé
à un tel fibré.

On dispose alors sur $\va$ de deux notions intéressantes, héritées de
l'espace projectif ambiant : une hauteur, et (pour tout place $\place$ de
$\cdn$) une distance $\place$-adique. Je ne dirai pas grand-chose de la
hauteur, avec laquelle je suppose le lecteur familier. Parlons un peu de la
notion de distance, et de quelques-une de ses propriétés.

La distance que j'utilise est celle définie dans \cite{pphdg}, qui se
mesure entre un point (fermé) et une sous-variété (pas forcément réduite ni
irréductible, c'est-à-dire un sous-schéma fermé) de $\proj{n}$. Dans le cas
où la sous-variété en question est une hypersurface $H$, définie par une
équation $F$, la distance entre un point $\point$ de coordonnées $\coord$ et
$V$ est définie comme
\[
  \distv{\point}{H} = 
  \frac {\absv{F(\coord)}} {\normv{F}\normv{\coord}}
  \pmm.
\]
Intuitivement, cette quantité traduit bien la notion de distance, et
notamment s'annule si et seulement si $\point$ est sur $H$. La théorie de
l'élimination permet d'étendre cette idée au cas général.

On peut aussi définir une notion \og naïve \fg de distance entre les points de
l'espace projectif, par la formule
\[
  \distv{\point}{\pointi} = 
  \frac {\normv{\coord \wedge \coordi}} {\normv\coord \normv\coordi}
  \pmm.
\]
Pour peu qu'on ai choisi une bonne norme aux places archimédiennes, ceci
définit une distance au sens usuel \cite{jadotth}. On peut alors en déduire une
notion de distance entre un point $\point$ et une variété $V$, en prenant le
minimum de la distance entre $\point$ et les points de $V(\C_\place)$. Ceci
définit une notion différente de la précédent, mais les deux sont comparables.

J'attire l'attention du lecteur sur le fait que cette notion locale,
projective, de distance n'a \lat{a priori} rien à voir avec la notion de
distance dans l'espace de \bsc{Mordell}-\bsc{Weil}, qui est globale et repose
sur l'existence d'une variété abélienne ambiante. Maintenant que nous sommes
familiers avec les objets en jeu, énonçons le théorème~2 de
\cite{falda}.

\begin{thm}
  Soit $V$ une sous-variété quelconque d'une variété abélienne $\va$ plongée
  comme précédemment, $\place$ une place de $\cdn$, et $\eps > 0$. Il n'existe
  qu'un nombre fini de points $\point$ dans $\va(\cdn)$ tels que
  \[
    0 < \distv{\point}{V} \le \haut{\point}^{-\eps} \pmm, 
    \tag{\textsc{ha}} \label{e-approx}
  \]
  où $\Haut$ désigne la hauteur de \bsc{Weil} multiplicative. 
\end{thm}

Comme de nombreux énoncés de géométrie diophantiennes, ce résultat n'est
malheureusement pas effectif au sens suivant : on ne voit à l'heure actuelle
pas de moyen de borner la hauteur des points satisfaisant à l'hypothèse
d'approximation \eqref{e-approx}. Ainsi que le fait remarquer \bsc{Faltings}
dans l'introduction de son article : \og \eng{As far as I can see, everything
  here is ineffective beyond hope.} \fg

Il semble néanmoins raisonnable de vouloir majorer le nombre de points
rationnels satisfaisant à \eqref{e-approx} (que nous appellerons à l'occasion
les approximations exceptionnelles), ou au moins de donner quelques
informations quantitatives explicites à leur sujet. C'est à cette question que
ma thèse vise à répondre.

Plus précisément, ce type d'énoncé quantitatif est généralement obtenu en
combinant une inégalité à la \bsc{Vojta} et une inégalité à la \bsc{Mumford}.
Il est peut-être utile de rappeler ici brièvement en quoi consistent ces deux
inégalités. Toutes deux s'énoncent dans l'espace de \bsc{Mordell-Weil} de la
variété, muni de la forme quadratique donnée par la hauteur normalisée de
\bsc{Néron-Tate}. Je les énonce ici sous une forme générique avec une
condition (C) qui peut être par exemple l'hypothèse d'approximation
\eqref{e-approx} ci-dessus, ou une autre condition pour l'ex-conjecture de
\bsc{Mordell-Lang}.

L'inégalité de \bsc{Vojta} affirme qu'il n'existe pas de suite $x_1, \dots,
x_m$ de points satisfaisant simultanément à la condition (C) et aux trois
conditions suivantes :
\begin{enumthm}
  \item $\hautn{x_1} > \alpha$ ; \label{i-grand}
  \item $\angleabs{x_i}{x_j} < \beta$ pour tous $i$ et $j$ ; \label{i-proche}
  \item $\hautn{x_i} > \gamma \hautn{x_{i-1}}$ pour $i > 1$ ; 
\end{enumthm}
où l'angle est relatif à la structure euclidienne de l'espace. Nous
appellerons \emph{cône tronqué} une partie de l'espace délimitée par les
conditions \ref{i-grand} et \ref{i-proche}. Il est clair que l'espace privé
d'une boule de rayon $\sqrt{\alpha}$ peut être recouvert par un nombre fini de
tels cônes tronqués dès qu'il est de dimension finie (ce qui est la cas si on
se place sur un corps de nombre. L'inégalité de \bsc{Vojta} assurant qu'il n'y
a qu'un nombre fini de points sous la condition (C) dans chaque cône, permet e
conclure à la finitude.

L'inégalité de \bsc{Mumford} peut s'énoncer de façon très similaire. Elle dit
qu'il n'existe pas de paire de points $x$ et $y$ satisfaisant à (C) et aux
conditions suivantes :
\begin{enumthm}
  \item $\hautn{y} > \hautn{x_1} > \alpha$ ; 
  \item $\angleabs x y < \beta$ ;
  \item $\hautn{y} < \delta \hautn{x}$.
\end{enumthm}
Utilisée conjointement avec l'inégalité de \bsc{Vojta}, et à condition que les
constantes apparaissant dans ces deux inégalités soient effectives, elle
permet de majorer le nombre de points dans chaque cône tronqué, donc le nombre
total de point (modulo un résultat, assez indépendant, de décompte des «
petits » points).

La démonstration de \bsc{Faltings} consiste précisément à démontrer une
inégalité de \bsc{Vojta}, non effective, qui suffit à assurer la finitude. Mon
travail consiste donc d'une part à rendre effective cette inégalité de
\bsc{Vojta}, et à lui adjoindre une inégalité de \bsc{Mumford}, elle aussi
effective.

\section{Relation avec d'autres énoncés}

\subsection{Le théorème de Roth}

Le théorème de \bsc{Roth}, dans sa version étendue aux places quelconques par
\bsc{Ridout}, peut s'énoncer de la façon suivante.

\begin{thm}
  Soit $\alpha \in \Qbar$ un algébrique. Soient par ailleurs $\cdn$ un corps de
  nombres et $\place$ une place de $\cdn$, étendue de façon arbitraire à
  $\cdn(\alpha)$. Pour tout $\eps > 0$, il n'existe qu'un nombre fini de
  points $x \in \cdn$ tels que 
  \[ 
    \absv{x - \alpha} < H(x)^{-2-\eps} \pmm.
  \]
\end{thm}

Le lien avec le théorème précédent est clair : on passe de l'un à l'autre en
remplaçant $\alpha$ par $V$ et $\cdn = \aff{1}(\cdn)$ par $\va(\cdn)$, la
distance étant bien sûr représentée par $\absv{x - \alpha}$. Le théorème~2 de
\cite{falda} est donc aux variétés abéliennes ce que le théorème de
\bsc{Roth} est à la droite.

Très rapidement après la démonstration initiale de \bsc{Roth}, on a su établir
des versions quantitatives du théorème. Plus précisément, la démonstration de
\bsc{Roth} consiste en un fait qu'on peut, anachroniquement, appeler une
inégalité à la \bsc{Vojta} : il n'existe pas de suite $x_1, \dots, x_m$
d'approximations exceptionnelles telle que $\haut{x_1} > c_1$ et pour tout $i >
1$, $\haut{x_i} > c_2 \cdot \haut{x_{i-1}}$. 

Pour établir une version quantitative du théorème de \bsc{Roth}, il a fallu
expliciter une valeur admissible des constantes $c_1$ et $c_2$, d'une part, et
d'autre part lui adjoindre un inégalité que j'appellerai encore
anachroniquement à la \bsc{Mumford}, disant qu'il existe une constante $c_3$
telle que deux approximations exceptionnelles $x$ et $y$, de hauteur assez
grande, satisfont toujours $H(x) > c_3 H(y)$. Dans le cas du théorème de
\bsc{Roth}, ceci découle immédiatement de l'inégalité de la taille. La
conjonction de ces deux inégalités donne clairement un décompte des
approximations exceptionnelles de hauteur assez grande.

Rappelons aussi que le théorème de \bsc{Roth} a été l'aboutissement d'une
longue série de théorèmes d'approximations moins précis, en ce sens que
l'exposant optimal $2+\eps$ n'était pas atteint. Ce série a débuté avec le
théorème de \bsc{Liouville}. Dans le contexte de ma thèse, l'équivalent de
l'inégalité de \bsc{Liouville} peut s'énoncer ainsi :

\begin{thm}
  Soient $V$ une variété projective de degré $d$, et $x \in \proj{n}(\Qbar)$
  un point algébrique. Si $x$ n'appartient pas à $V$, on a 
  \[
    \distv{x}{V} \ge c(n, d) \haut{V}^{-1} \cdot \haut{x}^{-d} \pmm.
  \]
\end{thm}

Dans le cas où $V$ est une hypersurface, c'est une application directe du
théorème du produit (et $c(n, d) = 1$), et on peut se ramener à ce cas en
général, quitte à prendre une valeur beaucoup plus petite pour $c(n, d)$.
Comme on le verra en \ref{s-siegel}, il est intéressant dans les applications
de disposer d'un exposant meilleur que $d$.

\subsection{L'ex-conjecture de Mordell-Lang}

Dès 1922, \bsc{Mordell} avait conjecturé l'énoncé suivant, aujourd'hui
théorème de \bsc{Faltings}.

\begin{thm}
  Soit $C$ une courbe projective lisse de genre $g \ge 2$, définie sur un
  corps de nombre $\cdn$. L'ensemble $C(\cdn)$ des points rationnels de $C$
  est fini.
\end{thm}

Ce résultat a d'abord été prouvé par \bsc{Faltings} en 1983 comme conséquence
d'une conjecture de \bsc{Shafarevitch}. La preuve fait intervenir des espaces
de modules de variétés abéliennes, et c'est à cette occasion que
\bsc{Faltings} a introduit la hauteur qui porte désormais son nom, sur cet
espace. Néanmoins, cette preuve reste assez éloignée des méthodes
traditionnelles de l'approximation diophantienne.

Une preuve totalement indépendante a été publiée en 1991 par \bsc{Vojta}. Elle
se rapproche grandement des idées habituelles de l'approximation
diophantienne, en introduisant ce qu'on appelle maintenant l'inégalité de
\bsc{Vojta}. La preuve est ensuite simplifiée (« \eng{avoid[ing] the difficult
  Arakelov theory in Vojta's paper} ») et étendue par \bsc{Faltings} pour
prouver une conjecture de \bsc{Lang}, généralisant celle de \bsc{Mordell}, et
qui s'énonce ainsi.

\begin{thm}
  Soit $V$ une sous-variété d'une variété abélienne $\va$, définie sur un
  corps de nombres $\cdn$. Si $V$ ne contient pas de translaté de sous-variété
  abélienne stricte, alors $V(\cdn)$ est fini.
\end{thm}

Ceci généralise la conjecture de \bsc{Mordell}, qui correspond au cas où $V$
est une courbe et $\va$ sa jacobienne. Ce résultat est proche du sujet de ma
thèse dans le sens suivant : il consiste à montrer la finitude des points
rationnels \emph{sur} une sous-variété de variété abélienne, alors que je
cherche à contrôler les points \emph{proches} d'une telle sous-variété. Il est
d'ailleurs significatif que \bsc{Faltings} a prouvé ces deux théorèmes (la
conjecture de \bsc{Mordell-Lang} et celui que je cherche à rendre quantitatif)
dans le même article : une bonne partie des outils sont commun aux deux
preuves.

Une différence notable entre les deux situations est toutefois la suivante :
pour étudier les points qui sont proches d'une sous-variété, sans appartenir
à cette variété, on n'a pas besoin de supposer que celle-ci ne contient pas de
translaté de sous-groupe. En fait, le résultat reste valable même pour les
approxitaions d'une sous-variété abélienne.

Des versions quantitatives du théorème 1 de \cite{falda} ont été établies
ensuite. Signalons la relecture de la preuve par \bsc{Bombieri}, qui simplifie
certains arguments en les rapprochant de l'effectivité, et le travail de
\bsc{De Diego} sur les familles de courbes. En 1999, \bsc{Rémond} prouve une
version totalement effective de l'inégalité de \bsc{Vojta}, puis lui adjoint
une inégalité à la \bsc{Mumford}, établissant ainsi une version quantitative
explicite de l'ex-conjecture de \bsc{Mordel-Lang}. Enfin, le chapitre~3 de la
thèse de \bsc{Farhi} \cite{farhith} donne une version quantitative de
\bsc{Mordell}, démontrée dans un formalisme plus élémentaire que celui de
\bsc{Rémond}.

\subsection{Le théorème de Siegel et une ex-conjecture de Lang}
\label{s-siegel}

Le théorème de \bsc{Siegel}, démontré en 1929, affirme qu'une courbe de genre
supérieur ou égal à 1 ne possède qu'on nombre fini de points entiers. Sa
démonstration repose sur le théorème de \bsc{Roth} énoncé plus haut, et avait
été obtenu par \bsc{Siegel} avec la version faible de cet énoncé dont il
disposait en 1929.  Une généralisation du théorème a été conjecturée par
\bsc{Lang}, de façon analogue à sa généralisation de la conjecture de
\bsc{Mordell} : si $E$ est un diviseur ample d'une variété abélienne $\va$,
alors $A \setminus E$ ne possède qu'un nombre fini de points entiers. Le
théorème original s'en déduit là aussi en considérant la courbe dans sa
jacobienne (à cette différence qu'ici la courbe peut être sa jacobienne, dans
le cas $g=1$).

Cette conjecture de \bsc{Lang} est en fait un corollaire du théorème~2 de
\cite{falda} : on remarque que la hauteur (relative à $E$) d'un point
entier $x$ est essentiellement donné par le produit des inverse des distances
$v$-adiques de $x$ à $E$ quand $v$ parcourt les places archimédiennes de
$\cdn$. Or ces distances sont minorées par $\haut(x)^{-\eps}$ pour tout
$\eps>0$, sauf pour un nombre fini de points. Ceci est bien sûr contradictoire
dès que $\eps < 1$, ce qui prouve que seules les approximations
exceptionnelles de $E$ peuvent donner des points entiers. Dénombrer ces
dernières donne donc immédiatement une version quantitative de cette
ex-conjecture de \bsc{Lang}. C'est pour ce type d'applications qu'il devient
essentiel dans l'énoncé d'approximation de pouvoir prendre $\eps$ petit, au
moins inférieur à $1$, alors que l'exposant $d$ de l'inégalité de
\bsc{Liouville} ne suffit en aucun cas.

Signalons qu'on connaît des versions quantitatives du théorème de \bsc{Siegel}
(??). Par contre, à ma connaissance, la seule démonstration connue de sa
généralisation est celle de \bsc{Faltings} : en particulier on ne connaît pas
de version quantitative de cette ex-conjecture de \bsc{Lang}.

Enfin, en un sens, on peut avoir l'impression que le théorème de
\bsc{Faltings} (ex-conjecture de \bsc{Mordell-Lang}) rend obsolète le théorème
de \bsc{Siegel} et sa généralisation conjecturée par \bsc{Lang} : en effet,
n'avoir qu'un nombre fini de point rationnels implique de n'avoir qu'un
nombre fini de points entiers. En fait, les énoncés de type \bsc{Siegel}
conservent un intérêt essentiellement grâce à la restriction \og ne pas
contenir de sous-variété abélienne \fg dans \bsc{Mordell-Lang} : si on prend
le cas extrême d'un variété abélienne, il est clair que (sur un corps de
nombres pas trop petit) elle possède une infinité de points rationnels, alors
qu'elle n'a qu'un nombre fini de points entiers. 

\section{Formes linéaires de logarithmes abéliens}

Il faudra que je lise des trucs à ce sujet\dots

\section{Approche envisagée}\label{s-approche}

Il s'agit essentiellement d'employer la méthode de \bsc{Vojta}, en s'inspirant
des travaux de \bsc{Rémond} \cite{remivds,remivg,remdcl}, de \bsc{Farhi}
\cite{farhith}, et de la preuve originale de \bsc{Faltings} \cite{falda}. Dans
les grandes lignes, la preuve consistera donc à établir une inégalité à la
\bsc{Mumford}, explicite, et à rendre explicites les constantes dans
l'inégalité à la \bsc{Vojta} obtenue par \bsc{Faltings}.

\subsection{Inégalité de \bsc{Mumford}}

J'ai pour le moment obtenu une version de cette inégalité dans le cas
particulier où la sous-variété à approcher est une variété abélienne.
L'énoncé obtenu est la suivant.

\begin{thm} \label{thm-mumford}
  Soient $\vai$ une sous-variété abélienne de $\va$, $\point$ et $\pointi$
  deux points de $\va(\Qbar)$ satisfaisant à la condition d'approximation
  \eqref{e-approx} ci-dessus. Il existe des constantes $\phi$, $\rho$ et $B$
  telles que si
  \begin{enumthm}
    \item $\cos(\point, \pointi) \ge 1 - \phi$ ;
    \item $\hautn{\point} \le \hautn{\pointi} \le (1+\rho) \hautn{\pointi}$ ;
    \item $\hautn{\point} > B$ ;
  \end{enumthm}
  alors $\point - \pointi \in \vai(\Qbar)$.
\end{thm}

Des valeurs admissibles pour $B$, $\rho$ et $\phi$ sont totalement explicitées
en fonction du degré de $\vai$, de sa dimension, du $\eps$ apparaissant dans
\eqref{e-approx}, de la dimension et de a hauteur de $\va$, ainsi que la dimension
du plongement ambiant.

L'approche adoptée la suivante : étant donné deux points satisfaisant aux
trois conditions du théorème, on considère leur différence $\pointii$. La
géométrie euclidienne élémentaire montre qu'il est de petite hauteur
normalisée, donc de petite hauteur projective. Une constante de comparaison
explicite entre ces deux hauteurs est fournie par \cite{daphiminvaii}.

Il s'agit ensuite de montrer que cette différence reste une bonne
approximation de $\vai$ : c'est ici qu'on utilise que $\vai$ est un groupe.
Pour cela, on utilise la comparaison entre distance ponctuelle et distance
algébrique établie dans \cite[p. 103]{pphdg}, qu'on complète par une inégalité
dans l'autre sens.  On se ramène ainsi à montrer que si des points sont
proches, alors leur somme l'est aussi. Ce résultat est en lui-même trivial (la
loi de groupe, algébrique, est lipschitzienne en toute place), mais expliciter
une constante de \bsc{Lipschitz} nécessite d'utiliser assez soigneusement les
formules d'addition de \cite{daphiminvaii}.

Une fois ceci établi, on constate que $\pointii$ est une très bonne
approximation de $\vai$, de petite hauteur, suffisamment pour contredire
l'inégalité de \bsc{Liouville}, à moins que $\pointii$ ne soit justement sur
$\vai$ ; ce qui établit le théorème~\ref{thm-mumford}.

\subsection{Inégalité de Vojta}

Le schéma général de cette partie de la preuve est commun à celui adopté dans
\cite{remivds} et \cite{falda} : on suppose pour commencer qu'il existe une
suite d'approximations $x_1, \ldots, x_m$ satisfaisant simultanément à
\eqref{e-approx} et aux trois conditions suivantes :
\begin{enumthm}
  \item $\hautn{x_1} > \alpha$ ; \label{iv-grand}
  \item $\angleabs{x_i}{x_j} < \beta$ pour tous $i$ et $j$ ; \label{iv-angle}
  \item $\hautn{x_i} > \gamma \hautn{x_{i-1}}$ pour $i > 1$. \label{iv-ecart}
\end{enumthm}
On regarde alors cette suite comme un point de $\va^m$, et on cherche à
construire par récurrence, en partant de $\va^m$,  une suite
décroissante de sous-variétés (irréductibles) produit $Y\pexp{u} =
\prod_i Y\pexp{u}[i]$ pour $u$ variant de $gm$ à $m-1$ telles que :
\begin{enumthm}
  \item $x \in Y\pexp u$ ;
  \item $\dim Y\pexp u = u$ ;
  \item $\deg Y\pexp u \le f(u)$ et $\deg Y\pexp u[i] \le f_i(u)$ ;
  \item $\sum_{i=1}^m a_i \haut{Y\pexp u[i]} \le g(u, a)$ ; \label{i-ht}
\end{enumthm}
où les $a_i$ sont des entiers choisis de telle sorte que la quantité $a_i^2
\hautn{x_i}$ soit à peu près constante, et $f$, $f_i$ et $g$ sont des
fonctions de $u$ à expliciter (mais $g$ est linéaire en $\abs a^2$). Plus
précisément, on s'attend à ce que $f$ et $f_i$ ne dépendent que du degré de
$V$ est des dimensions ambiantes, et que $g$ ne dépende que de la hauteur de
$\va$, et des degrés et dimensions ambiants.

Supposons un moment cette suite construite, et voyons comment elle permet de
conclure. On regarde $Y\pexp{m-1}$ : c'est un produit dont un des facteurs est
nécessairement réduit à un des $x_i$. Mais l'inégalité \ref{i-ht}, le fait que
$h(x_i)a_i^2$ soit à peu près constant, et la dépendance de $g$ en $\abs a$
permettent alors de majorer la hauteur de $x_1$, de façon à contredire le
point \ref{iv-grand} des hypothèses faites sur $x$.

Il suffit donc de savoir construire une telle suite, c'est-à-dire étant donné
une sous-variété produit $Y$ de $\va^m$, de dimension $u$, satisfaisant aux
hypothèses ci-dessus, de savoir construire une variété produit $Y'$ de
dimension $u-1$, satisfaisant aux même hypothèses. 

Plus précisément, on cherchera à construire $Y'$ comme une composante de
l'intersection de $Y$ avec une forme $T$ homogène en le groupe de variable
$X_i$ (et indépendante des autres), ne s'annulant pas identiquement sur $Y$,
dont il faut contrôler le degré et la hauteur : les théorèmes de \bsc{Bézout}
géométrique et arithmétique assurent alors que $Y'$ satisfait toujours aux
hypothèses sur la dimension, le degré et la hauteur. Par ailleurs, le choix de
la composante assurera que $x \in Y'$. 

Reste donc à savoir construire une telle forme $T$. Pour cela, on va utiliser
la méthode de \bsc{Thue} (dite aussi « méthode de transcendance ») :
construire une forme auxiliaire $F$ de degré et hauteur contrôlés, grâce à un
lemme de \bsc{Siegel}, puis extrapoler pour constater de cette forme s'annule
beaucoup, et utiliser cette information dans un lemme de zéros. 

Ici, le lemme de zéros utilisé sera une variante explicite du lemme de produit
de \bsc{Faltings}, dont la conclusion est justement l'existence d'une forme
$T$ comme celle recherchée, pour peu qu'on contrôle bien l'indice en $x$ de
$F$ relativement à des poids suffisamment étagés : ici, les poids sont
contrôlés par les $a_i$, donc en définitive par les rapport des hauteurs des
$x_i$ successifs : c'est l'hypothèse~\ref{iv-ecart} qui garanti ainsi leur
étagement.

Dans toute la suite, on supposera que la variété qu'on cherche à approcher est
en fait un diviseur $E$ de $A$. Cette hypothèse n'est pas restrictive dans le
sens où une variété $V$ quelconque est toujours incluse dans un diviseur de
degré comparable et, à une certaine constante près qu'on pourrait expliciter
(dépendant au plus du degré de $V$ et des dimensions ambiantes), la distance
d'un point à $V$ est majorée par sa distance à $E$.

Pour construire cette forme auxiliaire, on ne travaillera en fait pas sur $Y$
directement, mais sur son image par un certain plongement. Ce dernier fait
intervenir les $a_i$ définis ci-dessus, et joue dans la construction un rôle
semblable au fibré $\mathcal L(\eps, s)$ de \cite{falda} ou $\mathcal Q_{eps,
  a, d}$ de \cite{remivds}.
\begin{align*}
  \phi \colon Y & \to \va^{2m-1} \\
  (y_1, \dots, y_m)  & \mapsto 
  (y_1, \dots, y_m, a_1 y_1 - y_m, \dots a_{m-1} y_{m-1} - y_m)
\end{align*}
On cherche alors à construire une forme sur $Y$, s'annulant avec un indice
élevé le long de $Y \cap E^m$, et provenant (\lat{via} $\phi^*$) d'une forme
sur $\phi(Y)$ de multidegré $\delta(\eps_0 a_1^2, \dots, \eps_0 a_m^2, 1,
\dots, 1)$, pour un certain $\eps_0$.

On utilise pour ça un lemme de \bsc{Siegel}. Pour peu que $\delta$ soit assez
grand (et c'est sa raison d'être), le théorème de \bsc{Hilbert} multiprojectif
estime la dimension de l'espace des formes sur $Y$ de degré adéquat. Les
formules d'addition et de multiplication de \cite{daphiminvaii} nous
permettent d'estimer la hauteur du système ; cependant une estimation
brutale ne permet pas de contrôler correctement son rang. Il s'agit alors de
réduire le système modulo les équations de $Y$, en présentant ce dernier comme
un revêtement fini de l'espace multiprojectif $\proj{u}$, comme par exemple
dans le lemme~2.5 de \cite{remivg}. Un dénominateur, lié au lieu de
ramification du revêtement considéré, apparaît cependant, qu'on espère
néanmoins contrôler suffisamment pour qu'il ne perturbe pas trop la
construction.

On cherche ensuite à extrapoler, c'est-à-dire à montrer que notre forme
auxiliaire s'annule aussi en $x$, et même avec un indice encore assez élevé
pour pouvoir ensuite appliquer le lemme du produit. Pour cela, l'idée est bien
sûr de majorer la hauteur de $F(x)$, puis d'estimer sa composante locale en
$\place$ exploitant le fait que $x$ est $\place$-adiquement proche de $E$,  et
de conclure à une contradiction par la formule du produit à moins de $F(x)$ ne
soit nul. Dans la phrase précédente, il faut en fait remplacer $F$ par ses
dérivées jusqu'à un ordre convenable. C'est en tous cas l'hypothèse
d'approximation \eqref{e-approx} qui établit la contrainte nécessaire en liant
$\distv{x}{E}$ à $H(x)$, donc indirectement $\absv{F(x)}$ à $H(F(x))$.

Pour estimer la norme locale de $F(x)$ en $\place$, la difficulté est que $F$
s'annule le long de $Y \cap E$, alors qu'on sait que $x$ est proche de $E$,
mais pas nécessairement de $Y \cap E$. En fait, $F$ est congrue modulo l'idéal
de $Y$ à une forme s'annulant le long de $E$. On exploite alors le fait que
$x$ est sur $Y$ pour extrapoler plutôt sur cette nouvelle forme, de façon à
mieux exploiter l'hypothèse sur $x$. Enfin, pour l'extrapolation, on utilise
un paramétrisation locale de $Y$ (ou $\va$ ?) au voisinage de $x$, afin
d'estimer confortablement les dérivées.

\medskip

Si tout s'est bien passé, c'est fini : il ne reste plus (!) qu'à remonter les
explications que j'ai données essentiellement dans l'ordre inverse de leur
dépendance logique, et à vérifier que les constantes s'ajustent bien entre
elles en raccordant les différentes étapes, et notamment la récurrence\dots

\printbibliography

\end{document}
