\documentclass[a4paper, 11pt]{article}
\usepackage[utf8]{inputenc} % àéïôù = aeiou %

\usepackage{svn-multi}
\svnid{$Id$}
\newcommand\svnnatdate{%
  \begingroup \year\svnyear \month\svnmonth \day\svnday \relax 
  \today \endgroup}

\usepackage[T1]{fontenc}
\usepackage{textcomp}
\usepackage[sc, osf]{mathpazo}
\usepackage[scaled]{helvet}
\usepackage{amssymb}

\usepackage{lipsum}

% préambule général pour tous les trucs préparatoire à la thèse
% (en attendant un classe ou une extension un peu plus propre ?)
%

\usepackage{mathtools, amsmath, amsthm}
\usepackage[all]{xy}

\usepackage{ifmtarg, fixltx2e, xargs}

\usepackage{enumitem}

\usepackage{fancyhdr} \pagestyle{fancy}
\renewcommand\headrulewidth{0pt} \setlength\headheight{0pt}
\fancyhead{} \fancyfoot{}
\fancyfoot[C]{\thepage}
% \fancyfoot[RO, LE]{\today}

\usepackage{xspace}
\usepackage[british, frenchb]{babel}
\usepackage[babel=true, expansion=false]{microtype}
\frenchbsetup{AutoSpacePunctuation=false}

\newcommand*\notemarge[1]{\marginpar[\raggedleft #1]{\raggedright #1}}
\newcommand\todotext{\textsc{todo}}
\makeatletter
  \newcommand*\todo[1][]{%
    \leavevmode\notemarge{\todotext}%
    \@ifnotmtarg{#1}{[#1]}}
  \newcommand*\todom[1][]{\tag{\todotext%
    \@ifnotmtarg{#1}{ : #1}}}
\makeatother
%\newcommand\todo{\TextOrMath{\todot}{\todom}}

\newenvironment{enumthm}
  {\begin{enumerate}[label=(\textit{\roman*})]}
  {\end{enumerate}}

\newcommand*\lat[1]{\emph{#1}}
\newcommand*\eng[1]{%
  \foreignlanguage{english}{\emph{#1}}}
\newcommand*\defn[1]{\emph{#1}}
\newcommand*\pmm[1]{\text{ #1}}

\makeatletter
\newcommand*\ssub[1]{\@ifnotmtarg{#1}{_{#1}}}
\newcommand*\ssup[1]{\@ifnotmtarg{#1}{^{#1}}}
\newcommandx*\pexp[2][2]{%
  \@ifmtarg{#2}%
    {\cramped{^{(#1)}}}%
    {^{(#1)}_{#2}}}
\makeatother

\newcommand\suchthat{\ \middle\vert\ }

\newcommand*\std[1]{\mathbf{#1}} \newcommand\N{\std N} \newcommand\Z{\std Z}
\newcommand\Q{\std Q} \newcommand\R{\std R} \newcommand\C{\std C}
\newcommand\Proj{\std{P}} \newcommand\Aff{\std{A}} 
\newcommand\Qbar{\overline{\Q}}
\newcommand\cdn{\boldsymbol{k}} \newcommand\Cdn{\boldsymbol{K}}
\newcommand*\I[1]{\std{Id}_{#1}} \newcommand*\ind[1]{\std{1}_{#1}}
\renewcommand\ge{\geqslant} \renewcommand\le{\leqslant}
\newcommand\orbrack{\mathopen\rbrack} \newcommand\clbrack{\mathopen\lbrack}

\newcommand*\abs[1]{\left\lvert#1\right\rvert}
\newcommand*\norm[1]{\left\lVert#1\right\rVert}
\newcommand*\nnorm[1]{%
  \left\lvert\hspace{-1pt}\left\lvert\hspace{-1pt}%
  \left\lvert#1\right\rvert
  \hspace{-1pt}\right\rvert\hspace{-1pt}\right\rvert}

\newcommand\eps{\varepsilon}
\newcommand\truc{{\,\cdot\,}}
\DeclareMathOperator\disc{Disc}
\DeclareMathOperator\ord{ord}
\DeclareMathOperator\Div{div}

\newcommand\zeros{\mathcal Z}
\newcommand\ideal{\mathcal I}

\newcommand\mmax{{\mathrm{max}}}

\newcommand\diff{\mathrm d}

% objets mathématiques utiles
% 
\newcommand\va{\mathcal A}
\newcommand\point{x}
\newcommand\pointi{y}
\newcommand\coord{{\mathrm{x}}}
\newcommand\coordi{{\mathrm{y}}}
\newcommand\genre{g}
\newcommand\fibre{\mathcal L}
\newcommand\proj[2][]{%
  \Proj\ssup{#2}\ssub{#1}}
\newcommand\place{v}

\DeclareMathOperator\Haut{H}
\newcommand\haut[1]{%
  \Haut(#1)}

\newcommand\absv[2][\place]{%
  \abs{#2}\ssub{#1}}
\newcommand\normv[2][\place]{%
  \norm{#2}\ssub{#1}}
\newcommand\distv[3][\place]{%
  \mathop{\mathrm{dist}\ssub{#1}}(#2,#3)}

\newcommand\noref[1]{%
  [?#1?]}

\newtheorem{thm}{Théorème}


\title{Approximation dans les variétés abéliennes}
\date{\svnnatdate\thanks{%
    Date de dernière modification --- 
    révision \no \svnrev\ du dépôt \emph{Subversion}.}}
\author{Manuel \bsc{Pégourié-Gonnard}}

\begin{document}

\maketitle

\begin{abstract}
  Ce document se veut un aperçu un peu technique mais lisible (par un
  théoricien des nombres, mettons) de mon sujet de thèse : un énoncé du
  problème, ses relations avec d'autres questions en géométrie diophantienne,
  et une description (forcément un peu vague pour l'instant) des techniques
  que je souhaite utiliser pour y répondre. 
\end{abstract}

\section{L'énoncé du problème}

Soit $\va$ une variété abélienne, de dimension $\genre$, définie sur un corps
de nombres $\cdn$. On suppose selon ses goûts que $\va$ est
munie d'un fibré ample et symétrique $\fibre$, ou que $\va$ est plongée une
fois pour toute dans un espace projectif $\proj{n}$ par un plongement associé
à un tel fibré.

On dispose alors sur $\va$ de deux notions intéressantes, héritées de
l'espace projectif ambiant : une hauteur, et (pour tout place $\place$ de
$\cdn$) une distance $\place$-adique. Je ne dirai pas grand-chose de la
hauteur, avec laquelle je suppose le lecteur familier. Parlons un peu de la
notion de distance, et de quelques-une de ses propriétés.

La distance que j'utilise est celle définie dans \noref{Pph-LNM}, qui se
mesure entre un point (fermé) et une sous-variété (pas forcément réduite ni
irréductible, c'est-à-dire un sous-schéma fermé) de $\proj{n}$. Dans le cas
où la sous-variété en question est une hypersurface $H$, définie par une
équation $F$, la distance entre un point $\point$ de coordonnées $\coord$ et
$V$ est définie comme
\[
  \distv{\point}{H} = 
  \frac {\absv{F(\coord)}} {\normv{F}\normv{\coord}}
  \pmm.
\]
Intuitivement, cette quantité traduit bien la notion de distance, et
notament s'annule si et seulement si $\point$ est sur $H$. La théorie de
l'élimination permet d'étendre cette idée au cas général.

On peut aussi définir une notion \og naïve \fg de distance entre les points de
l'espace projectif, par la formule
\[
  \distv{\point}{\pointi} = 
  \frac {\normv{\coord \wedge \coordi}} {\normv\coord \normv\coordi}
  \pmm.
\]
Pour peu qu'on ai choisi une bonne norme aux places archimédiennes, ceci
définit une distance au sens usuel \noref{Jadot}. On peut alors en déduire une
notion de distance entre un point $\point$ et une variété $V$, en prenant le
minimum de la distance entre $\point$ et les points de $V(\C_\place)$. Ceci
définit une notion différente de la précédent, mais les deux sont comparables.

J'attire l'attention du lecteur sur le fait que cette notion locale,
projactive, de distance n'a \lat{a priori} rien à voir avec la notion de
distance dans l'espace de \bsc{Mordell}-\bsc{Weil}, qui est globale et repose
sur l'existence d'une variété abélienne ambiante. Maintenant que nous sommes
familiers avec les objets en jeu, énonçons le théorème~2 de
\noref{Faltings91}.

\begin{thm}
  Soit $V$ une sous-variété quelconque d'une variété abélienne $\va$ plongée
  comme précédement, $\place$ une place de $\cdn$, et $\eps > 0$. Il n'existe
  qu'un nombre fini de points $\point$ dans $\va(\cdn)$ tels que
  \[
    0 < \distv{\point}{V} \le \haut{\point}^{-\eps} \pmm, 
    \tag{\textsc{ha}} \label{e-approx}
  \]
  où $\Haut$ désigne la hauteur de \bsc{Weil} multiplicative. 
\end{thm}

Comme de nombreux énoncés de géométrie diophantiennes, ce résultat n'est
malheureusement pas effectif au sens suivant : on ne voit à l'heure actuelle
pas de moyen de borner la hauteur des points satisfaisant à l'hypothèse
d'approximation \eqref{e-approx}. Ainsi que le fait remarquer \bsc{Faltings}
dans l'introduction de son article : \og \eng{As far as I can see, everything
  here is ineffective beyong hope.} \fg

Il semble néamoins raisonnable d'espérer majorer le nombre de points
rationnels satisfaisant à \eqref{e-approx} (que nous appellerons à l'occasion
les approximations exceptionnelles), ou au moins de donner quelques
informations quantitatives explicites à leur sujet. C'est à cette question que
me thèse vise à répondre.

\section{Ses relations avec d'autres énoncés}

\subsection{Le théorème de Roth}

\subsection{L'ex-conjecture de Mordell-Lang}

\subsection{Le théorème de Siegel et une ex-conjecture de Lang}

\section{Approche envisagée}

\end{document}


