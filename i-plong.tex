% !TEX root = main

\chapter{Plongement projectif et formulaire} \label{chap:plong-mm}

\section{Rappel du plongement et notations} \label{sec:plong-mm-def}

\later Pour l'instant, on utilise sans les rappeler les définitions de
\cite{daphimhva2}, ainsi que les notations, sauf sur les points suivants : on
omettra systématiquement les indices se rapportant à la puissance de
\( \fibre \) utilisée et on notera plus volontiers en indice les coordonnées.
On écrira ainsi $\coa_{(a, l)} = \Delta_{a, l}(\coa)  = \Delta_{(a,
  l)}^{(2)}(\coa) = \coa_{\fibre^{\otimes 4}}(a, l)$. Par ailleurs, bien
que les coordonnées soient naturellement indexées (\lat{ibid.} p. 651) par
$\mathcal{Z}_2 \times \widehat{K_2(4)}$, on les indexera souvent par $\{0,
  \dots, \dimp\}$ pour alléger l'écriture  ; ainsi on écrira $\coa = (\coa_0,
\dots \coa_n)$ des coordonnées de l'origine.

\endinput

% vim: spell spelllang=fr
