\documentclass[final, cover]{mpg-preth}
\usepackage{lipsum}

\begin{document}

\thispagestyle{empty}

\subsection*{Résumé}

Le but de la thèse est d'établir une version quantitative du théorème suivant
: toute sous-variété d'une variété abélienne n'admet qu'un nombre fini
d'approximations d'exposant strictement positif. Cet énoncé a été obtenu par
\bsc{Faltings} en 1991 ; la majeure partie des outils qu'il utilise sont
communs avec sa preuve de l'ex-conjecture de \bsc{Mordell-Lang}. Il implique
en particulier une extension du théorème de \bsc{Siegel} conjecturée par
\bsc{Lang} : toute variété abélienne n'a qu'un nombre fini de points entiers.

On utilise la méthode de \bsc{Vojta} en suivant les travaux de \bsc{Rémond}
(version quantitative de \bsc{Mordell-Lang}) : le cœur de la thèse consiste à
établir une inégalité à la \bsc{Vojta} explicite ; on établit ensuite une
inégalité à la \bsc{Mumford} avant d'en déduire un décompte des approximations
exceptionnelles.

Toutefois, le cas où la variété approchée contient des translatés de
sous-variétés abéliennes non nulles nécessite d'imposer une condition
supplémentaire pour parvenir à un décompte explicite : sans ces conditions,
un tel décompte impliquerait dans certains cas un résultat effectif, qui
semble hors de portée à l'heure actuelle.

\medskip\noindent
\textbf{Mots-clés} : approximation diophantienne, variété abélienne, méthode
de Vojta, inégalité de Mumford, décompte explicite.

\subsection*{Abstract --- Diophantine approximation on abelian varieties}

This thesis aims at providing a quantitative version of the following theorem :
there are only finitely many approximations with positive exponent of any
subvariety of an abelian variety. This theorem was proved by Faltings in 1991
using mostly the same tools as his proof of the Mordell-Lang conjecture. A
corollary is the following extension of Siegel's theorem, conjectured by Lang:
any abelian variety has only finitely many integral points.

We proceed with Vojta's method, following Rémond's work on a quantitative
version of Mordell-Lang: the technical heart of the thesis is the proof of an
explicit version of a suitable variant of Vojta's inequality ; we then
establish an inequality \emph{à la} Mumford and an explicit bound for the
number of exceptional approximations follows.

However, we need an additional hypothesis to get an explicit bound when the
variety considered contains translates of a positive-dimensional abelian
subvariety. Indeed, in some cases, an explicit bound for the number of points
would give an explicit bound for their height, which seems to be out of reach
at the present time.

\medskip\noindent
\textbf{Keywords}: diophantine approximation, abelian variety, Vojta's
inequality, \mbox{Mumford}'s inequality, explicit bound.

\end{document}
