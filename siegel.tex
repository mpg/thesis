\documentclass{mpg-preth}

\title{Vers une fonction auxiliaire}
\subtitle{Ébauche de notes de travail}

\begin{document}

\maketitle

\section{Situation générale}

Soit $\va$ une variété abélienne sur un corps de nombres $\cdn$, plongée dans
un espace projectif $\proj\dimp$ muni de coordonnées homogènes $\ch0, \ldots,
\ch\dimp$. On suppose que $\va$ n'est pas contenue dans l'hyperplan
d'équation $\ch0 = 0$, et on considère le diviseur $\divi$ découpé par
$\ch0$ sur $\va$. On note enfin $\ideal\va$ l'idéal homogène saturé
définissant $\va$ dans $\cdn [\ch0, \ldots, \ch\dimp]$.

On considère $\puiss$ un entier positif, et $(\proj\dimp)^\puiss$ l'espace
multi-projectif associé à l'anneau multi-gradué
\begin{math}
  \cdn [
    \cmh1[0], \dots, \cmh1[\dimp];
    \dots;
    \cmh\puiss[0], \dots \cmh\puiss[\dimp]
  ]
  =
  \cdn [ \cmh1, \dots, \cmh\puiss ]
  =
  \cdn [ \cmmh ]
\end{math}.
On s'intéresse à une sous-variété produit de $\va^\puiss$ plongée dans
$(\proj\dimp)^\puiss$, notée
\begin{math}
  \var = \var[1] \times \dots \times \var[\puiss]
\end{math}
et dont aucun facteur n'est contenu dans un hyperplan d'aquation $\cmh*0 = 0$.
On notera en outre $\vardim* = \dim \var*$ et $\vardim = (\vardim*)_\fct$ puis
$\vardeg* = \deg \var*$, et $\varfc*$ une forme de \bsc{Chow} de $\var*$. On
note enfin $\varid$ l'idéal multi-homogène saturé de $\var$ et $\varid*$ ceux
de ses facteurs.

De façon générale, si $A$ est une algèbre graduée et $\Ideal$ un idéal
homogène, on notera $A_d$ et $\Ideal_d$ leur partie homogène de degré $d$ ; on
utilisera la même notation pour les algèbres et idéaux multi-gradués, où $d$
désignera une famille d'entiers.

\begin{tdef}
  Suivant \cite{remivg}, on dit qu'un plongement
  \begin{math}
    \iota\colon V \embedin \proj\dimp
  \end{math}
  d'une véritété de dimension $t$ dans un espace projectif muni de coordonnées
  homogènes $X_0, \dots, X_\dimp$ est \emph{adapté} si les deux conditions
  suivantes sont satisfaites :
  \begin{enumthm}
    \item $V \cap \zeros{X_0, \dots, X_t} = \emptyset$ ;
    \item $\korper{V}$ est engendré par
      $\frac{X_1}{X_0}, \dots, \frac{X_{t+1}}{X_0}$.
  \end{enumthm}
\end{tdef}

\begin{lem}
  Soit $\iota \colon V \embedin \proj\dimp$ un sous-schéma fermé intègre de
  degré $\Delta$, non contenu dans l'hyperplan d'équation $X_0 = 0$. Il existe
  une matric $M \in \GL_{n+1}(\Q)$, à coefficients entiers de valeur absolue
  (archimédienne) majorée par $\max(\frac\Delta2, 1)$, telle que, si $e_M$ est
  l'automorphisme linéaire de $\proj\dimp$ associé à $M$, alors $e_M \circ
  \iota$ est un plongement adapté et que $X_0$ soit invariant par ce
  changement de coordonnées.
\end{lem}

\begin{proof} \later
  On reprend la preuve de la proposition 2.2 de \cite{remivg} (p.~469). Au
  moment de choisir des formes linéaires $W_0, \dots, W_n$ telles que
  \begin{equation*}
    \chow V (W_0, \dots, W_\dimp) \neq 0
    \pmm,
  \end{equation*}
  on commence en fait par fixer $W_0 = \ch0$. Le polynôme $\varfc(W_0, \truc,
  \dots, \truc)$ est multihomogène de degré $\Delta$ en chaque variable ; vu
  l'hypothèse sur $V$, il est non nul grâce au théorème fondamental de
  l'élimination. On peut donc choisir $W_1, \dots W_n$ comme dans
  \cite{remivg} et continuer la preuve sans autre modification.
\end{proof}

\begin{lem}
  Soit $V \colon \embedin \proj\dimp$ un sous-schéma fermé intègre, et $M \in
  \GL_{\dimp+1}(\Qbar)$ à coefficients entiers dans l'intervalle $[-B, B]$, où
  $B$ est un réel positif fixé, et $e_M$ l'automorphisme linéaire de
  $\proj\dimp$ associé à $M$. Alors
  \[
    \haut[1]{e_M(V)}
    \le
    \haut[1]{V}
    \cdot \bigl( (n+1)B \bigr)^{\Delta(n+1)}
    \pmm.
  \]
\end{lem}

\begin{proof} \later
  Inégalité classique sur la norme $L_1$ et la spécialisation des polynômes.
\end{proof}

\begin{scho}
  Vu les hypothèses sur $V$, on peut sans perte de généralité supposer de plus
  que $V \embedin \proj\dimp$ est adapté, quitte à multiplier sa hauteur
  ($L_1$) par
  $\bigl ((\dimp + 1) \max (\frac\Delta2, 1) \bigr)^{\Delta(\dimp+1)}$.

  On supposera donc désormais que chaque $\var*$ est plongé de façon adaptée
  dans son facteur $\proj\dimp$. Ceci implique que
  $\varfc*(\cmh*0, \dots,\cmh*{\vardim*})$
  est non nul pour tout $\fct$ ; on normalisera donc notre choix de $\varfc*$
  (qui est unique à une constante multiplicative près) en imposant que cette
  quantité vaille $1$.
\end{scho}

\section{Idéal d'annulation}

Introduisons un rationnel $\epsz > 0$, un entier $\delta$ tel que
$\epsz\delta \in \N$, et un réel $\epsi > 0$. Pour fixer les idées, dans
l'énoncé de la proposition~\ref{build-aux}, on demandera à $\delta$ d'être
assez grand, et à $\epsi$ d'être majoré par une certaine expression dépendant
de $\epsz$ (voir fin de la section~\ref{sec-comp-dim}). Considérons par
ailleurs une famille d'entiers $a_1 \ge \dots \ge a_m = 1$ et un réel
$\epsiii$ satisfaisant à
\[
  \sum_{\fct=1}^\puiss (a_\fct^2 + 1) < a_1^2 (1 + \epsiii)
  \pmm.
\]

On fera plus tard une hypothèse sur la famille $a$ de sorte qu'il soit
possible de choisir $\epsiii$ plutôt petit, mais cette information n'est pas
utile pour le moment.

On introduit alors l'ensemble suivant (cylindre sur un dessous d'escalier) :
\[
  \Stairs_a^\delta = \left\{
    (\lambda\pexp1, \dots, \lambda\pexp\puiss) \in \N^{m(n+1)}
    \text{ tels que }
    \frac {\lambda\pexp1[0]} {a_1^2}
    + \dots +
    \frac {\lambda\pexp{\puiss-1}[0]} {a_{\puiss-1}^2}
    \frac {\lambda\pexp\puiss[0]} {\puiss-1}
    \le \epsi \delta
    \right\}
  \pmm,
\]
auquel on associe l'idéal d'annulation
\[
  \Vanish*_a^\delta = \left\{
    \sum_{\lambda \in \N^{\puiss(\dimp+1)}} f_\lambda
    \in \cdn [\cmmh]
    \text{ tels que }
    \lambda \in \Stairs_a^\delta
    \implies
    f_\lambda = 0
    \right\}
  \pmm.
\]
On considère enfin $\Vanish_a^\delta = \Vanish_a^\delta + \varid$. Par abus,
on notera de même son image dans $\ring\var = \cdn [\cmmh] / \varid$, ce qui
est légitime puisqu'on a fait en sorte que $\varid \subset \Vanish_a^\delta$.

\section{Plongement éclatant}

Par la suite, on se donnera un point
$(\excep = \excep[1], \dots, \excep[\puiss])$
de $\var(\Qbar)$, où les $\excep*$ sont de hauteur grandes et étagées entre
elles, contenus dans un secteur de cône de l'espace de \bsc{Mordell-Weil}. On
définira alors les poids $\weight*$ par
\begin{equation}
  \weight* =
  \left\lfloor
  \frac {\norm{\excep*}} {\norm{\excep1}}
  + \frac12
  \right\rfloor
  \pmm,
\end{equation}
où $\norm\truc$ représente la norme associée à la hauteur normalisée dans
l'espace de \bsc{Mordell-Weil} de $\va$ (on commet l'abus consistant à noter
encore $\excep*$ l'image de $\excep*$ dans cet espace).

La géométrie euclidienne montre\footnote{
  Les détails sont présents dans la preuve de l'inégalité à la \bsc{Mumford}.
  Cette information est en fait inutile ici ; on souhaite seulement justifier
  vaguement le rôle du plongement éclatant.}
alors que $\weight* \excep* - \excep[\puiss]$ est de hauteur petite devant les
hauteurs des différents $\excep*$. Pour nous aider à exploiter ultérieurement
cette information, on introduit un plongement, dit \emph{éclatant}, associé à
$\weight$, défini par
\begin{align}
  \wemb \colon \var
  & \longto \va^\puiss \times \va^{\puiss-1}
  = \va^{2\puiss-1} \subset (\projd)^{2\puiss-1}
  \\
  (x_1, \dots, x_\puiss) 
  & \longmapsto
  (x_1, \dots, x_\puiss;
  \weight[1] x_1 - x_\puiss, \weight[\puiss-1] x_{\puiss-1} - x_\puiss)
  \pmm.
\end{align}

On munit l'espace d'arrivée $(\projd)^{2\puiss-1}$ des coordonnées
multi-homogènes $\cmmh, \cmmhi = \cmh1, \dots, \cmh\puiss, \cmhi1,
\cmhi{\puiss-1}$ ; dans ce contexte quand $i$ et $j$ sont deux indices non
précisés, on supposera implicitement $1 \le i \le \puiss$ et $1 \le j \le
\puiss-1$.

On va maintenant s'attacher à représenter localement ce morphisme par une
famille de formes multi-homogènes. Pour cela, on a besoin de savoir
représenter les opérations de $\va$ (addition, soustraction, multiplication
par un entier) par des polynômes, globalement ou localement. Le résultat
fondamental est le fait suivant, qui est une simple reformulation de la
proposition~3.6 de \cite{daphimhva2}.

\begin{fact} \label{fact-addsub}
  Il existe une famille $S_{l, l'}$ de formes bi-homogènes de bi-degré $(2,
  2)$ représentant globalement de morphisme d'addition-soustraction
  \begin{align}
    \va \times \va
    & \to
    \va \times \va
    \\
    (x, y)
    & \mapsto
    (x + y, x - y)
  \end{align}
  dans le plongement de \bsc{Mumford} modifié suivi d'un plongement de
  \bsc{Segre}. On peut les prendre de la forme
  \begin{equation}
    S_{l, l'} (V, W)
    =
    \coi{k(l)} \coi{k(l')}
    \sum_{p, p', q, q'} \in E 
    \zeta_{p, p', q, q'} V_p V_{p'} W_q W_{q'}
    \pmm,
  \end{equation}
  où $E$ est un sous-ensemble à $4^\genre$ éléments de $\{0, \dots, \dimp\}$,
  les $\zeta_{p, p', q, q'}$ sont des racines quatrièmes de l'unité, et $k$
  est une certaine fonction $\{0, \dots, \dimp\} \to \{0, \dots, \dimp\}$
  telle que pour tout $p$, $\coi^{-1}_{k(p)}$ est un coordonnée (non nulle) de
  $\OA$.

  En particulier, on peut estimer les normes locales :
  $\norm{S_{l, l'}}_{v, 1} \le 4^{\genre\dv} \norm{\coi}^2_{v, 1}$
  pour tout $(l, l')$ et
  $\nnorm{S}_{v, 1} \le
  \bigl( 4^\genre (\dimp+1)^2 \bigr)^\dv \norm{\coi}^2_{v, 1}$
  pour la famille.
\end{fact}

Cette représentation suppose $\va^2$ plongée dans un espace projectif
$\proj{\dimp^2 + 2\dimp}$. Si l'on veut en déduire simplement, par projection
sur un facteur, des représentations de l'addition et de la soustraction, on
doit d'abord « redescendre » $\va^2$ dans $(\projd)^2$ : ceci se fait
simplement par des projections linéaires. Or celles-ci ont un centre en lequel
elles ne sont pas définies, et qui rencontre $\va$. On n'obtient donc pas une
représentation globale de l'addition et de la soustraction, mais une famille
de représentations locales. En remarquant de plus qu'on peut dans chacune de
ces formes simplifier par $\coi_{k(l')}$ puisqu'on projette à $l'$ fixé, on a
donc le résultat suivant.

\begin{coro}\label{c-addsub-form}
  Tout couple de points de $\va$ est contenu dans un ouvert $\opdef$ de
  $\va^2$ sur lequel il existe deux familles de formes bi-quadratiques
  $(S_l^{+})$ et $(S_l^{-})$ pour $0 \le l \le \dimp$ représentant sur
  $\opdef$ l'addition (resp. la soustraction) de $\va$, telles que
  \begin{equation}
    \norm{S^{\bullet}_l}_{v,1}
    \le
    4^{\genre\dv} \norm{\coi}_{v, 1}
    \quad \forall l \in \{0, \dots, \dimp\}
    \pmm,
  \end{equation}
  où $\bullet \in \{ +, - \}$.
\end{coro}

Par ailleurs, on sait que la multiplication par un entier $b$ peut être
représentée globalement par une famille de formes homogènes de degré $b^2$. On
connaît même \cite[prop. 3.8]{daphimhva2} une telle famille de façon
totalement explicite pour $b = 2$, donc à chaque fois que $b$ est une
puissance de $2$, en itérant.

On ne sait \lat{a priori} pas expliciter une telle famille dans le cas
général\footnote{
  Pour les courbes elliptiques, c'est fait dans \cite[th. 2.13.2]{farhith}
  pour tout $b$.},
mais on peut néanmoins obtenir des formules locales de multiplication, en
combinant les formules locales d'addition et de multiplication par deux.

\begin{fact}\label{f-mult-form}
  Pour tout $j \in \{ 0, \dots, \dimp \}$ et pour tout entier positif, il
  existe une famille de formes homogènes $Q_{b, i, j}$ pour $i \in \{ 0,
    \dots, \dimp \}$ représentant la multiplication par $b$ sur l'ouvert $V_j
  \neq 0$, telle que, pour tout $i$ :
  \begin{gather}
    \deg Q_{b, i, j} = \left\lceil \frac98 b^2 \right\rceil
    \\
    \norm {Q_{b, i, j}}_{v, 1}
    \le
    \norm{\coi}_{v, 1}^{b^2 -1}
    \cdot \bigl( 2^\genre \sqrt{\dimp+1} \bigr)^{\dv(b^2-1)}
    \pmm.
  \end{gather}
\end{fact}

\begin{proof} \later
  Tout est contenu dans la démonstration de la proposition~5.2 de \cite[pp.
  126-128]{remivds} une fois remarqué qu'on peut choisir
  \[
    P_{1, 1, 0, 0, l, l'} = S_{l, l'}
    \pmm,
  \]
  où le membre de gauche reprend les notations de la preuve de \bsc{Rémond},
  alors que le membre de droite est la famille donnée par le
  fait~\ref{fact-addsub}, qui n'était pas disponible au moment de la
  rédaction de \cite{remivds}.

  De plus, \bsc{Rémond} donne seulement la majoration du degré. Il est en
  fait plus commode de fixer une valeur exacte du degré, ce que l'on peut
  faire en multipliant par $V_j$ sans changer l'ouvert sur lequel les
  formules sont valables, ni leur hauteur.

  On a par ailleurs utilisé le norme $L_1$ aux places archimédiennes, ce qui
  réduit légèrement la partie archimédienne de la constante par rapport à
  celle de \bsc{Rémond}.
\end{proof}

En appliquant ce formulaire à notre plongement éclatant, on voit que tout
point $x \in \var(\Qbar)$ est contenu dans un ouvert $\opdef \subset \var$ sur
lequel il existe des polynômes représentant
$\wemb\vert_\opdef \colon \opdef \to (\projd)^{2\puiss-1}$.
Plus précisément, il existe un morphisme d'algèbres, dépendant de $\opdef$,
\begin{align}
  \wemba \colon \cdn [\cmmh, \cmmhi] & \to \cdn[\cmmh] \\
  \cmh{i} & \mapsto \cmh{i} \\
  \cmhi{j} & \mapsto S^{-}(Q_{a_j}(\cmh{j}, \cmh{\puiss}))
\end{align}
tel que le diagramme suivant, dont les flèches verticales sont les projections
canoniques, commute :
\[
  \dots % TODO
\]
On a choisi $S^{-}$ et $Q_{a_j}$ parmi les familles de formes données
respectivement part le corollaire~\ref{c-addsub-form} et le
fait~\ref{f-mult-form} de sorte que les formules obtenues soient valables au
voisinages de $x$.

Les informations connues sur les degrés de $S^{-}$ et des $Q_{a_j}$ donnent
immédiatement le résultat suivant.

\begin{lem}
  Soient $\wemba$ un morphisme d'algèbre comme ci-dessus et $F \in
  \Qbar[\cmmh, \cmmhi]$ une forme multi-homogène de multi-degré $(\alpha,
  \beta)$ où $\alpha \in \N^\puiss$ et $\beta \in \N^{\puiss-1}$. On a alors
  \[
    \deg \wemba(F)
    \le
    \bigr(
    \alpha_1 + \frac94 \beta_1 a_1^2,
    \dots,
    \alpha_{\puiss-1} + \frac94 \beta_{\puiss-1} a_{\puiss-1}^2,
    \alpha_\puiss + 2 \lgr\beta
    \bigl)
    \pmm.
  \]
\end{lem}

\clearpage

\section{Stratégie de construction de la forme auxiliaire}

\clearpage

\section{Deux calculs de dimension} \label{sec-comp-dim}

\begin{prop} \label{build-aux}
  \dots
\end{prop}
\printbibliography

\end{document}
