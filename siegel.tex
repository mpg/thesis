\documentclass{mpg-preth}
\newcommand\va{{\mathcal A}}
\newcommand\coi{{\mathfrak b}} % corrdonnées de l'origine inversées
\newcommand\point{x}
\newcommand\pointi{y}
\newcommand\coord{{\mathrm{x}}}
\newcommand\coordi{{\mathrm{y}}}
\newcommand\genre{g}
\newcommand\fibre{{\mathcal L}}
\newcommand\place{v}
\newcommand\divi{E}
\newcommand\dimp{n}

\newcommand\ph{\ssub} % partie homogène
\newcommand\mat{M} % matrice

% multiprojectif
\newcommand\puiss{m}
\newcommand\fct{i}
\newcommand\ch[1]{X\ssub{#1}}
\newcommand\cmh[1]{X\pexp{#1}}

% la sous-variété produit et ses attribus
\newcommand\varname{Z}
\newcommand\var[1][]{\varname\pexp{#1}}
\newcommand\vardim[1][]{{u\ssub{#1}}}
\newcommand\vardeg[1][]{{D\ssub{#1}}}
\newcommand\varht[1][]{\Haut\pexp{#1}[\varname]}
\newcommand\varfc[1][]{\Chow\pexp{#1}[\varname]}
\newcommand\varid[1][]{\Ideal\pexp{#1}[\varname]}

\newcommand*\proj[2][]{%
  \Proj\ssup{#2}\ssub{#1}}
\newcommand*\aff[2][]{%
  \Aff\ssup{#2}\ssub{#1}}

\newcommand\Chow{f}

\newcommand\Haut{h}
\newcommand\haut[1]{%
  \Haut(#1)}
\newcommand\hautn[1]{%
  \mathop{\smash{\hat \Haut}}(#1)}

\newcommand\Ideal{\mathcal I}
\let\ideal\relax % en attendant de rendre mes trucs cohérents
\newcommand*\ideal[1]{%
  \Ideal\ssub{#1}}

\newcommand*\absv[2][\place]{%
  \abs{#2}\ssub{#1}}
\newcommand*\normv[2][\place]{%
  \norm{#2}\ssub{#1}}
\newcommand*\distv[3][\place]{%
  \mathop{\mathrm{dist}\ssub{#1}}(#2,#3)}

\newcommand*\scalaire[2]{%
  \langle#1,#2\rangle}
\newcommand*\angleabs[2]{%
  \widehat{(#1,#2)}}

\newcommand*\noref[1]{%
  [#1]}

% théorèmes
\newtheorem{thm}      {Théorème}
\newtheorem{prop}[thm]{Proposition}
\newtheorem{lem} [thm]{Lemme}
\newtheorem{coro}[thm]{Corollaire}
\newtheorem{fact}[thm]{Fait}
\newtheorem{scho}[thm]{Scholie}
\newtheorem*{flemme}{À démontrer plus tard}
\theoremstyle{definition}
\newtheorem{tdef}[thm]{Définition}


\title{Vers une fonction auxiliaire}
\subtitle{Ébauche de notes de travail}

\begin{document}

\maketitle

\section{Situation générale}

Soit $\va$ une variété abélienne sur un corps de nombres $\cdn$, plongée dans
un espace projectif $\proj\dimp$ muni de coordonnées homogènes $\ch0, \ldots,
\ch\dimp$. On suppose que $\va$ n'est pas contenue dans l'hyperplan
d'équation $\ch0 = 0$, et on considère le diviseur $\divi$ découpé par
$\ch0$ sur $\va$. On note enfin $\ideal\va$ l'idéal homogène saturé
définissant $\va$ dans $\cdn [\ch0, \ldots, \ch\dimp]$.

On considère $\puiss$ un entier positif, et $(\proj\dimp)^\puiss$ l'espace
multiprojectif associé à l'anneau multi-gradué
\begin{math}
  \cdn [
    \cmh1[0], \dots, \cmh1[\dimp];
    \dots;
    \cmh\puiss[0], \dots \cmh\puiss[\dimp]
  ]
  =
  \cdn [ \cmh1, \dots, \cmh\puiss ]
\end{math}.
On s'intéresse à une sous-variété produit de $\va^\puiss$ plongée dans
$(\proj\dimp)^\puiss$, notée
\begin{math}
  \var = \var[1] \times \dots \times \var[\puiss]
\end{math}.
On notera en outre
\begin{align*}
  \vardim[\fct] &= \dim {\var[\fct]}
  & \vardim &= \dim \var = \sum_\fct \vardim[\fct]
  \\
  \vardeg[\fct] &= \deg {\var[\fct]}
  & \vardeg &= \deg \var = \sum_\fct \vardeg[\fct]
  \\
  \varht[\fct] &= \haut {\var[\fct]}
  & \varht &= \haut \var = \sum_\fct \varht[\fct]
\end{align*}
et $\varfc[\fct]$ une forme de \bsc{Chow} de $\var[\fct]$, de sorte que
$\varht[\fct] = \haut{\varfc[\fct]}$. On note enfin $\varid$ l'idéal homogène
saturé de $\var$ et $\varid[\fct]$ ceux de ses facteurs.

De façon générale, si $A$ est une algèbre graduée et $\Ideal$ un idéal
homogène, on notera $A\ph{d}$ et $\Ideal\ph{d}$ leur partie homogène de degré
$d$.

\begin{tdef}
  Plongement adapté : voir \bsc{Rémond}.
\end{tdef}

\begin{lem}\label{l-adapt-exists}
  Soit $\iota\colon \varname \hookrightarrow \proj{\dimp}$ un sous-schéma
  fermé intègre de degré $\vardeg$, non contenu dans l'hyperplan d'équation
  $\ch0 = 0$. Il existe une matrice $\mat \in GL_{\dimp+1}(\bar\Q)$, à
  coefficients entiers de valeur absolue (archimédienne) majorée au sens large
  par $\max(\frac\vardeg2, 1)$, telle que, si $e_\mat$ est l'automorphisme
  linéaire de $\proj{\dimp}$ associé à $\mat$, alors $e_\mat \circ \iota$ est
  un plongement adapté.
\end{lem}

\begin{proof}
  On reprend la preuve de la proposition~2.2 de \noref{IVG} (p.\,469). Au
  moment de choisir des formes linéaires $V_0, \dots, V_n$ telles que
  \[
  f_Z(V_0, \dots, V_n) \neq 0 \pmm,
  \]
  on commence en fait par fixer $V_0 = X_0$. Le polynôme $f(V_0, \truc, \dots,
  \truc)$ est multi-homogène de degré $D$ en chaque variable ; vu l'hypothèse
  sur $Z$, il est non nul grâce au théorème fondamental de l'élimination. On
  peut donc choisir $V_1, \dots, V_n$ comme dans \noref[IVG] et continuer la
  preuve sans autre modification.
\end{proof}

\begin{lem}\label{l-adapt-cost}
  Soit $Z \hookrightarrow \proj{n}$ un sous-schéma fermé intègre, et
  $
  \mat \in GL_{\dimp+1}(\bar\Q) \cap \Z^{\dimp+1} \cap [-B, B]^{\dimp+1}
  $,
  où $B$ est un réel positif fixé, et $e_\mat$ l'automorphisme linéaire de
  $\proj{n}$ associé à $\mat$. Alors
  \[
  \Haut_1(e_\mat(Z)) \leq B^{\vardeg(\dimp+1)} \pmm.
  \]
\end{lem}

\begin{proof}
  Inégalité classique sur la norme $L_1$ et la spécialisation des polynômes.
\end{proof}

\begin{scho}
  Vu les hypothèses sur $\var$, on peut sans perte de généralité supposer de
  plus que $\var \hookrightarrow \proj{\dimp}$ est adapté, quitte à multiplier
  sa hauteur ($L_1$) par $\max(\frac\vardeg2, 1)^{\vardeg(\dimp+1})$.

  On fera donc désormais cette hypothèse.
\end{scho}

\section{Idéal d'annulation}

Introduisons un rationnel $\eps_0 > 0$ et un entier $\delta$ tel que
$\eps_0\delta \in \N$. Fixons un réel $\eps_1 > 0$, liée à $\eps_0$ par une
inégalité \todo[à expliciter ultérieurement], et notons $\tilde\delta =
\eps_1\delta$.
Considérons par ailleurs une famille d'entiers $a_1 \ge \dots \ge a_\puiss =
1$. Cette famille sera fixée à la fin de la section en cours, il suffit pour
l'instant de savoir que ses éléments seront suffisamment étagés pour qu'on
puisse choisir $\eps_3$ pas trop gros \todo[à expliciter] tel que
\[
  \sum_{i=1}^{m} (a_i^2 +1) < a_1^2 (1 + \eps_3) \pmm.
\]

On introduit alors le dessous de multi-escalier
\[
  L_a^\delta = \left\{
  (\lambda\pexp1, \ldots, \lambda\pexp\puiss) \in \N^{\puiss(\dimp+1)}
  \text{ tels que }
  \frac {\lambda\pexp1[0]} {a_1^2}
  + \dots +
  \frac {\lambda\pexp\puiss[0]} {m-1}
  \le \eps_1 \delta
  \right\}
  \pmm,
\]
auquel on associe l'idéal d'annulation
\[
  \mathfrak{\tilde q}_a^\delta
  = \left\{
  \sum_{\lambda \in \N^{\puiss(\dimp+1)}} f_\lambda X^\lambda \in \cdn [X]
  \text{ tels que }
  \lambda \in L_a^\delta \Rightarrow f_\lambda = 0
  \right\}
  \pmm.
\]
On considère enfin $\mathfrak{q}_a^\delta = \mathfrak{\tilde q}_a^\delta +
\varid$. Par abus, on notera de même son image dans $\mathcal{A}_{\var} = \cdn
[X] / \varid$, ce qui est légitime puisqu'on a fait en sorte que $\varid
\subset q_a^\delta$.

On dira que $q_a^\delta \subset \mathcal{A}_{\var}$ est l'idéal des formes sur
$\var$ s'annulant le long de $\divi$ avec indice au moins $\eps_1 \delta$
relativement à $a$.

On fixe désormais une famille $(e_1, \dots, e_\puiss)$ de points de
$\var(\bar\Q)$. La famille d'entiers $a$ introduite ci-dessus est alors définie
par
\[
  a_i = \left\lfloor
  \frac {\norm{x_i}} {\norm{x_m}} + \frac12
  \right\rfloor
  \pmm,
\]
où $\norm\truc$ représente la norme associée à la hauteur normalisée dans
l'espace de \bsc{Mordell}-\bsc{Weil} de $\va$ (on commet l'abus consistant à
toujours noter $e_i$ l'image de $e_i$ dans cet espace).

Les points $e_\fct$ sont supposés de hauteurs suffisamment grandes et étagés
pour qu'il soit possible de choisir $\eps_3$ comme plus haut.

\section{Plongement éclatant}

La troisième et dernière hypothèse qui sera faite sur les $e_i$ est qu'il sont
angulairement proches dans l'espace de \bsc{Mordell}-\bsc{Weil}. La géométrie
euclidienne montre alors que $a_i e_i - e_m$ est de hauteur petite. Pour nous
aider à exploiter ultérieurement cette information, on introduit un
plongement, dit \emph{éclatant}, associé à $a$, définit par
\begin{alignat*}{2}
  \phi_a \colon
  && \var
  & \longrightarrow \va^{2\puiss-1}
  \\
  && (x_1, \dots, x_\puiss)
  & \longmapsto (x_1, \dots, x_m, a_1 x_1 - x_m, a_{m-1} x_{m-1} - x_m)
\end{alignat*}

On munit l'espace d'arrivée $\proj{2m-1}$ des coordonnées multi-homogènes
$
X, Y = X\pexp1, \dots X\pexp\puiss, Y\pexp1, \dots, Y\pexp{\puiss-1}
$
; dans toute la suite quand $i$ et $j$ sont deux indices non précisés, on
supposera implicitement $1 \le i \le \puiss$ et $1 \le j \le \puiss-1$.

Tout point $x \in \var(\bar\Q)$, est contenu dans un ouvert $U \subset \var$
sur lequel il existe des polynômes représentant $\phi_a\vert_U \colon U \to
\proj{2\puiss-1}$. Plus précisément, il existe un morphisme d'algèbres,
dépendant de $U$,
\begin{alignat*}{2}
  \psi_a \colon
  && \cdn [X, Y] & \to \cdn [X]
  \\
  && X\pexp{i} & \mapsto X\pexp{i}
  \\
  && Y\pexp{j} & \mapsto S( Q_{a_j}(X\pexp{j}), X\pexp\puiss )
\end{alignat*}
tel que le diagramme suivant, dont les flèches verticales sont les projections
canoniques, commute.
\[
  \xymatrix{
  \cdn [X, Y] \ar[r]^{\psi_a} \ar[d]^{\chi}
  & \cdn [X] \ar[d]^{\chi}
  \\
  \cdn [X, Y] / \ideal{\phi_a(\var)} \ar[r]_{\phi_a^*}
  & \cdn [X] / \ideal{\var}
  }
\]
Les formes $S$ et $Q_{a_j}$ dépendent de l'ouvert $U$ et sont données
explicitement par \noref{mhsvva2} et \noref{ivds}.

\begin{prop}\label{p-control-embed}
  Soient $\psi_a$ un morphisme d'algèbre comme ci-dessus, et $F \in \bar\Q [
  X, Y]$ une forme multi-homogène de multi-degré $(\alpha, \beta)$, où $\alpha
  \in \N^\puiss$ et $\beta \in \N^{\puiss-1}$. On a alors
  \begin{align*}
  \deg \psi_a(F)
  & =
  \psi_a^{\deg} (\alpha, \beta)
  &
  \normv{\psi_a(F)}
  & \le
  \psi_a^{\mathrm{norm}} (\alpha, \beta, \normv{F})
  \end{align*}
  où l'on peut prendre
  \begin{align*}
  \psi_a^{\deg} (\alpha, \beta)
  & =
  \alpha_1 + \frac94 \beta_1 a_1^2, \dots,
  \alpha_{m-1} + \frac94 \beta{m-1} a_{m-1}^2,
  \alpha_m + 2 \abs\beta
  \\
  \psi_a^{\mathrm{norm}} (\alpha, \beta, N)
  & =
  N
  \cdot \normv\coi ^{\abs\beta + 2\sum(a_j^2 -1)\beta_j)}
  \cdot \big( 4^\genre(\dimp+1) \big)
  ^ {\delta_v (\abs\beta + \sum (a_j^2-1)\beta_j)}
  \end{align*}
\end{prop}

\begin{proof}
  Voir les notes de la précédente tentative construction d'une fonction
  auxiliaire.
\end{proof}

\section{La problématique}

Il s'agit de trouver une forme multi-homogène non nulle sur $\var$, de petite
hauteur, provenant d'une forme sur $\phi(\var)$, qui s'annule le long de
$\divi$ avec un indice (pondéré par $a$) élevé relativement à son multi-degré.

Plus précisément, on pose $d = \delta \cdot (\eps_0 a_1^2, \dots, \eps_0
a_\puiss^2, 1, \dots, 1$ et on note $d' = \psi_a^{\deg}(d)$.

\end{document}
