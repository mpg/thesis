\documentclass[11pt, twoside, a4paper, draft]{article}
%%%%%%%%%%%%%%%%%%%%%%%%%%%%%%%%%%%%%%%%%%%%%%%%
%%  Attention, UTF-8 inside : àéïôù =? aeiou  %%
%%%%%%%%%%%%%%%%%%%%%%%%%%%%%%%%%%%%%%%%%%%%%%%%
\usepackage[utf8]{inputenc}

% préambule général pour tous les trucs préparatoire à la thèse
% (en attendant un classe ou une extension un peu plus propre ?)
%

\usepackage{mathtools, amsmath, amsthm}
\usepackage[all]{xy}

\usepackage{ifmtarg, fixltx2e, xargs}

\usepackage{enumitem}

\usepackage{fancyhdr} \pagestyle{fancy}
\renewcommand\headrulewidth{0pt} \setlength\headheight{0pt}
\fancyhead{} \fancyfoot{}
\fancyfoot[C]{\thepage}
% \fancyfoot[RO, LE]{\today}

\usepackage{xspace}
\usepackage[british, frenchb]{babel}
\usepackage[babel=true, expansion=false]{microtype}
\frenchbsetup{AutoSpacePunctuation=false}

\newcommand*\notemarge[1]{\marginpar[\raggedleft #1]{\raggedright #1}}
\newcommand\todotext{\textsc{todo}}
\makeatletter
  \newcommand*\todo[1][]{%
    \leavevmode\notemarge{\todotext}%
    \@ifnotmtarg{#1}{[#1]}}
  \newcommand*\todom[1][]{\tag{\todotext%
    \@ifnotmtarg{#1}{ : #1}}}
\makeatother
%\newcommand\todo{\TextOrMath{\todot}{\todom}}

\newenvironment{enumthm}
  {\begin{enumerate}[label=(\textit{\roman*})]}
  {\end{enumerate}}

\newcommand*\lat[1]{\emph{#1}}
\newcommand*\eng[1]{%
  \foreignlanguage{english}{\emph{#1}}}
\newcommand*\defn[1]{\emph{#1}}
\newcommand*\pmm[1]{\text{ #1}}

\makeatletter
\newcommand*\ssub[1]{\@ifnotmtarg{#1}{_{#1}}}
\newcommand*\ssup[1]{\@ifnotmtarg{#1}{^{#1}}}
\newcommandx*\pexp[2][2]{%
  \@ifmtarg{#2}%
    {\cramped{^{(#1)}}}%
    {^{(#1)}_{#2}}}
\makeatother

\newcommand\suchthat{\ \middle\vert\ }

\newcommand*\std[1]{\mathbf{#1}} \newcommand\N{\std N} \newcommand\Z{\std Z}
\newcommand\Q{\std Q} \newcommand\R{\std R} \newcommand\C{\std C}
\newcommand\Proj{\std{P}} \newcommand\Aff{\std{A}} 
\newcommand\Qbar{\overline{\Q}}
\newcommand\cdn{\boldsymbol{k}} \newcommand\Cdn{\boldsymbol{K}}
\newcommand*\I[1]{\std{Id}_{#1}} \newcommand*\ind[1]{\std{1}_{#1}}
\renewcommand\ge{\geqslant} \renewcommand\le{\leqslant}
\newcommand\orbrack{\mathopen\rbrack} \newcommand\clbrack{\mathopen\lbrack}

\newcommand*\abs[1]{\left\lvert#1\right\rvert}
\newcommand*\norm[1]{\left\lVert#1\right\rVert}
\newcommand*\nnorm[1]{%
  \left\lvert\hspace{-1pt}\left\lvert\hspace{-1pt}%
  \left\lvert#1\right\rvert
  \hspace{-1pt}\right\rvert\hspace{-1pt}\right\rvert}

\newcommand\eps{\varepsilon}
\newcommand\truc{{\,\cdot\,}}
\DeclareMathOperator\disc{Disc}
\DeclareMathOperator\ord{ord}
\DeclareMathOperator\Div{div}

\newcommand\zeros{\mathcal Z}
\newcommand\ideal{\mathcal I}

\newcommand\mmax{{\mathrm{max}}}

\newcommand\diff{\mathrm d}

% \newcommand*\av[2][v]{\abs{#2}_{#1}}      % valeur absolue v-adique, ou indice en option
% \newcommand*\nv[2][v]{\norm{#2}_{#1}}       % norme v-adique, idem
% \newcommand*\nnv[2][v]{\nnorm{#2}_{#1}}     % norme v-adique d'une famille, idem
% \newcommand*\nvp[2][v]{\lVert#2\rVert_{#1}}     % norme idem mais en forçant une petite taille
\newcommand*\mv[2][v]{M_{#1}(#2)}     % mesure homogène en une place v ou autre
% \newcommand*\mahler[2][]{\mathcal M_{#1}(#2)}   % mesure de Mahler
% \newcommand\dv{{\delta\smash{_v}}}        % sélecteur v ultramétrique ou non
\newcommand\dpv{{\delta'_v}}        % idem plus places divisant 2
% \newcommand\Dv{\mathrm{dist}_v}       % distance v-adique projective
% \newcommand\A{\mathcal A}         % la variété abélienne
\newcommand\B{\mathcal B}       % une sous-v.a.
% \newcommand*\p[1]{\boldsymbol{#1}}      % un point ; on notera en « normal » ses coordonnées
% \newcommand\OA{\p 0}          % l'origine de la variété
% \newcommand\coa{\theta}         % des coordonnées de l'origine
% \newcommand\BA{\mathfrak b}         % vecteur "B" avec coordonnées inverses de l'origine
% \newcommand\hn{\smash{\hat h}}        % la hauteur normalisée
% \newcommand*\lgr[1]{{\lvert#1\rvert}}     % longueur d'un multi-indice
% \newcommand*\vlg[1]{\lgr#1}         % multi-longueur partielles d'un multi-multi-indice
\newcommand*\lgt[1]{\lVert#1\rVert}     % longueur totale
\newcommand\md{\mathfrak{d}}        % le morphisme « d »
\newcommand*\MW[2][\A]{MW_{\!#1}(#2)}     % l'espace de mordell-weil de #2 sur #1
\newcommand*\hatxs{\smash{\hat x}}      % x chapeau rapetissé
\newcommand*\hatys{\smash{\hat y}}      % y chapeau rapetissé
\newcommand*\cube[1]{\mathcal C(#1)}      % le « cube » \abs{w} \le #1 et w entier

\newcounter{cst}
\newcommand*\cst[2][, v]{%
  \ensuremath{c\ssub{\ref{#2}#1}}}
\newcommand*\newcst[2][, v]{%
  \refstepcounter{cst}\label{#2}%
  \cst[#1]{#2}}

\newcommand\iq{{\mathfrak q}}

\newcommand\vb{{\mathbf e}}
\newcommand\Mbase{{\mathcal M}}
\newcommand\monome{{\mathfrak m}}


\newtheorem{thm}{Théorème} \newtheorem{prop}[thm]{Proposition}
\newtheorem{lem}[thm]{Lemme} \newtheorem{coro}[thm]{Corollaire}
\newtheorem{fait}[thm]{Fait} 
\newtheorem*{flemme}{À démontrer plus tard}

\theoremstyle{definition}
\newtheorem{rem}{Remarque} \newtheorem*{Rem}{Remarque}

\begin{document}

\section*{Prologue}

Je reprends les notations précédentes.  On a donc une variété $Z$ de dimenson
$u$, produit, plongée dans $(\Proj^n)^m$ mais aussi dans $(\Proj^n)^{2m-1}$ par
le plongement \og éclatant \fg{} $\phi_a$. 

\begin{flemme}
  Soit $\alpha \in \N^m$ et $\beta \in \N^{m-1}$ tels que $\lgr\alpha +
  \lgr\beta = u$. Alors
  \[
    \deg_{\alpha, \beta}(\phi_a(Z))= \prod_{j=1}^{m-1} a_j^{2(u_j - \alpha_j)}
    \deg_{\alpha, \beta}(\phi_1(Z))
  \]
  De plus, $\deg_{\alpha, \beta}(\phi_1(Z)) \ge D = \deg(Z)$ (au moins) lorsque
  l'une des deux conditions suivantes est satisfaite :
  \begin{enumthm}
    \item $\alpha = (0, \ldots, 0, u_m)$ et $\beta = (u_1, \ldots, u_{m-1})$ ;
    \item $\alpha_m = 0$ et il existe un $j_0 \neq m$ tel que $\alpha_{j_0} +
      \beta_ {j_0} = u_{j_0} + u_m$ et, dès que $j \neq j_0$, $\beta_j = u_j$ et
      $\alpha_j = 0$.
  \end{enumthm}
\end{flemme}

\section{Construction de la forme auxiliaire}

Soient $\eps_0$ et $\eps_1$ deux réels strictement positifs fixés, et $\delta$
un (grand) entier naturel tel que $\eps_0 \delta$ soit entier. On fixe $\eps_3 >
0$ tel que $\sum a_l^2 + m -1 \le (1 + \eps_3) a_1^2$. On suppose par ailleurs
que $\eps_0 \le 1/2$ et on fait enfin une hypothèse à préciser ultérieurement
liant $\eps_0$ et $\eps_1$. On introduit l'ensemble 
\[
  L_\delta = \left\{ (\lambda\pexp 1, \ldots, \lambda\pexp m) \in \prod \N^{u_i}
  \,\suchthat\, 
  \frac{\lgr{\lambda\pexp1}}{a_1^2} + \ldots +
  \frac{\lgr{\lambda\pexp m}}{a_{m-1}^2} + \frac{\lgr{\lambda\pexp{m}}}{m-1} 
  \le \eps_1 \delta \right\}
\]
et on considère l'idéal $\iq_\delta$ de $\cdn[[t\pexp1, \ldots, t\pexp m]]$
défini par l'annulation de tous les coefficients d'indice contenu dans
$L_\delta$. 

On pose alors 
\[
  d = (\eps_0\delta a_1^2, \ldots, \eps_0\delta a_m^2, \delta, \ldots, \delta)
  \in \N^{2m-1}
\]
et on cherche jusqu'à la fin de cette section à construire une forme homogène de
degré $d$ dont l'évalutaion en l'origine soit dans $\Omega_a^{-1}(\iq_\delta)$,
qui ne s'annule pas identiquement sur $Z$, et dont on contrôlera la hauteur.

Pour cela, on utilisera la version suivante du lemme de \bsc{Siegel}
(conséquence directe du lemme de \bsc{Bombieri} et \bsc{Vaaler} en tenant compte
des différences de hauteurs utilisées).

\begin{fait}
  Soient $l_1, \ldots, l_M$ des formes linéaires sur $\cdn^N$. On note $H_\mmax$
  la hauteur de l'ensemble de leur coefficients (en utilisant la norme du sup
  aux places archimédiennes). Si $M < N$, il existe un zéro commun de cette
  famille de formes dans $\cdn^N$ de hauteur ($L_1$) majorée par
  \[
    N \sqrt{\abs{\disc\cdn}} ( \sqrt N H_\mmax )^\frac M{N-M}
  \]
\end{fait}

Voyons comment s'écrit le système exprimant les conditions d'annulation en
l'origine de la forme $P$ recherchée. Les images des monômes de degré $d$ dans
$\cdn[X, Y]_d / \ideal(\phi_a(Z))$ forment une famille génératrice, dont on
extrait une base.  On note $N$ son cardinal, $\Mbase$ sa préimage dans $\cdn[X,
Y]$ et on cherche $P$ sous la forme $\sum_{\monome\in\Mbase} p_\monome \monome$,
où  $(p_\monome)$ est un vecteur non nul de $\cdn^N$.

La condition d'annulation s'écrit alors (en reprenant les notations du lemme~9
précédent) 
\[
  \sum_{\monome\in\Mbase}
  p_\monome V_\dv^\lambda(\monome)(\coa, \ldots, \coa)
  = 0 \quad 
  \forall \lambda\in L_\delta \pmm.
\]
On note ensuite $M$ le cardinal de $L_\delta$, $w_{\lambda, \monome} =
V_\dv^\lambda(\monome)(\coa, \ldots, \coa)$ (ce nombre ne dépendant en fait pas
de $\dv$), et $H_\mmax$ la hauteur ($L_\infty$) de la famille des $w_{\lambda,
  \monome}$ pour $\lambda \in L_\delta$ et $\monome \in \Mbase$. On introduit
enfin $e = M/(N-M)$. On va maintenant s'attacher à estimer $N$ et à majorer
successivement $M$, $e$ et $H_\mmax$.

Pour estimer $N$, on commence par supposer que $\delta$ est assez grand, de
sorte que $N$ est donné par la valeur en $d$ du polynôme de \bsc{Hilbert} de
$\phi_a(Z)$. 
\begin{align*}
  N 
  & = \sum_{\lgr\alpha + \lgr\beta = u}
    \deg_{\alpha, \beta}(\phi_a(Z))
    \cdot \frac{ 
      (\eps_0 \delta a_\truc^2)^\alpha \delta^\beta}{%
      \alpha! \beta!}
    + o(\delta^u) \\
  & = \sum_{\lgr\alpha + \lgr\beta = u}
    \deg_{\alpha, \beta}(\phi_1(Z)) \prod_l a_l^{2u_l - \alpha_l}
    \cdot \frac{ 
      \eps_0^{\lgr\alpha} \prod_l a_l^{2\alpha_l} \delta^u}{%
      \alpha! \beta!}
    + o(\delta^u) \\
  & = \delta^u \prod_l a_l^{2u_l} \cdot 
    \sum_{\lgr\alpha + \lgr\beta = u}
    \frac{
      \deg_{\alpha, \beta} (\phi_1(Z)) \eps_0^{\lgr\alpha}}{
      \alpha! \beta!}
    + o(\delta^u)
\end{align*}
On admet pour l'instant que la somme apparaissant à la dernière ligne est
comprise entre $D \eps_0^u / u!$ et $D (8m)^u$.

On écrit ensuite que $M$ est, à $o(\delta^u)$ près, donné par le volume du
simplexe correspondant dans $\R^u$ :
\[
  M = \frac{
    (\prod_l a_l^{2u_l}) (m-1)^{u_m} (\eps_1 \delta)^u}{
    u!}
  + o(\delta^u) \pmm.
\]

On est alors en mesure d'estimer $e$ :
\begin{align*}
  e 
  & \le \frac{
      (\prod_l a_l^{2u_l}) (m-1)^{u_m} (\eps_1 \delta)^u / u! + o(\delta^u)}{
      (\prod_l a_l^{2u_l}) \delta^u D \eps_0^u / u! 
      - (\prod_l a_l^{2u_l}) (m-1)^{u_m} (\eps_1 \delta)^u / u! +o(\delta^u)} \\
  & \le \frac{
      (m-1)^{u_m} \eps_1^u + o(1)}{
      mD\eps_0^u - (m-1)^{u_m} \eps_1^u + o(1)} \\
  & \le \frac {\eps_2} {1 - \eps_2} + o(1) \pmm,
\end{align*}
en posant $\eps_2 = \frac {(m-1)^{u_m} \eps_1^u} {m D\eps_0^u}$. On
suppose alors que $\eps_2 \le 1/2$ (c'est l'hypothèse sur $\eps_0$ et $\eps_1$
évoquée plus haut), ce qui permet de conclure que $e \le 1 + o(1)$.

Passons maintenant à l'estimation de hauteur. Elle est basée sur le lemme~9
précédent, et le fait que $\av{w_{\lambda, \monome}} \le
\nv{V_\dv^\lambda(\monome)} \nv{\coa}^{\lgr{\deg V_ \dv^{\lambda}(\monome)}}$ en
toute place $v$. On note 
\[
  d_\Omega = \delta \bigl(a_1^2 (\eps_0 + \tfrac94), \ldots, 
  a_{m-1}^2 (\eps_0 + \tfrac94), \eps_0 + 2(m-1) \bigl) \pmm.
\]
On sait alors que $\deg_l(V_\dv^\lambda(\monome)) = (d_\Omega)_l + 
(D-1) (n-u_l) f(\lambda\pexp l)$. Il vient ainsi :
\begin{align*}
  \lgr{\deg V_ \dv^{\lambda}(\monome)} 
  & \le \delta \bigl( (\eps_0 + \tfrac94) \sum_l a_l^2 + \eps_0 + 2(m-1) \bigr)
    + 2(D-1) n\lgt\lambda \\
  & \le a_1^2 \delta \bigl( 3(1+\eps_0) + 2n (D-1) \eps_1 \bigr) \pmm.
\end{align*}
Par ailleurs,
\begin{align*}
  \nv{V_\dv^\lambda(\monome)} 
  & \le \nv\B ^{\delta(\sum_j 2(a_j^2 -1) + m-1)} \cdot \nv{f_Z}^{2n\lgt\lambda}
    \\ & \qquad \cdot \left( 
    \bigl( 4^g 2^{9u/4} (n+1) \bigr)^{\delta(\sum_j 2(a_j^2-1) + m-1)}
    \cdot (16uD^3)^{n\lgt\lambda}
    \right)^\dv \\
  & \le \nv\B ^{2(1+\eps_3) a_1^2\delta} \cdot \nv{f_Z}^{2\eps_1 a_1^2\delta}
    \\ & \qquad \cdot \left(
    \bigl( 4^{2g+3u} (n+1)^2 \bigr)^{(1+\eps_3) a_1^2\delta}
    \cdot (16uD^3)^{n\eps_1 a_1^2\delta}
    \right)^\dv \pmm.
\end{align*}

On a ainsi l'estimation suivante \todo[à revoir : corrections ci-dessus et
$\sqrt N$ oublié] pour la hauteur logarithmique du système : 
\begin{multline*}
  h_\mmax \le \delta a_1^2 \bigl( 2\eps_1 h(Z) + 2(1+\eps_3) h(\B) 
  + 3(1+\eps_0) h(\coa)  + 3n\eps_1D  \\ + (1+\eps_3)((4g+6u)\ln(2)
  + 2\ln(n+1)) + n\eps_1\ln(16u)) \bigr) \pmm.
\end{multline*}

\end{document}
