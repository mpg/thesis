\documentclass{mpg-preth}

\title{Vers une fonction auxiliaire}
\subtitle{Ébauche de notes de travail}

\begin{document}

\maketitle

\section{Situation générale}

Soit $\va$ une variété abélienne sur un corps de nombres $\cdn$, plongée dans
un espace projectif $\proj\dimp$ muni de coordonnées homogènes $\ch0, \ldots,
\ch\dimp$. On suppose que $\va$ n'est pas contenue dans l'hyperplan
d'équation $\ch0 = 0$, et on considère le diviseur $\divi$ découpé par
$\ch0$ sur $\va$. On note enfin $\ideal\va$ l'idéal homogène saturé
définissant $\va$ dans $\cdn [\ch0, \ldots, \ch\dimp]$.

On considère $\puiss$ un entier positif, et $(\proj\dimp)^\puiss$ l'espace
multi-projectif associé à l'anneau multi-gradué
\begin{math}
  \cdn [
    \cmh1[0], \dots, \cmh1[\dimp];
    \dots;
    \cmh\puiss[0], \dots \cmh\puiss[\dimp]
  ]
  =
  \cdn [ \cmh1, \dots, \cmh\puiss ]
  =
  \cdn [ \cmmh ]
\end{math}.
On s'intéresse à une sous-variété produit de $\va^\puiss$ plongée dans
$(\proj\dimp)^\puiss$, notée
\begin{math}
  \var = \var[1] \times \dots \times \var[\puiss]
\end{math}
et dont aucun facteur n'est contenu dans un hyperplan d'aquation $\cmh*0 = 0$.
On notera en outre $\vardim* = \dim \var*$ et $\vardim = (\vardim*)_\fct$ puis
$\vardeg* = \deg \var*$, et $\varfc*$ une forme de \bsc{Chow} de $\var*$. On
note enfin $\varid$ l'idéal multi-homogène saturé de $\var$ et $\varid*$ ceux
de ses facteurs.

De façon générale, si $A$ est une algèbre graduée et $\Ideal$ un idéal
homogène, on notera $A_d$ et $\Ideal_d$ leur partie homogène de degré $d$ ; on
utilisera la même notation pour les algèbres et idéaux multi-gradués, où $d$
désignera une famille d'entiers.

\begin{tdef}
  Suivant \cite{remivg}, on dit qu'un plongement
  \begin{math}
    \iota\colon V \embedin \proj\dimp
  \end{math}
  d'une véritété de dimension $t$ dans un espace projectif muni de coordonnées
  homogènes $X_0, \dots, X_\dimp$ est \emph{adapté} si les deux conditions
  suivantes sont satisfaites :
  \begin{enumthm}
    \item $V \cap \zeros{X_0, \dots, X_t} = \emptyset$ ;
    \item $\korper{V}$ est engendré par
      $\frac{X_1}{X_0}, \dots, \frac{X_{t+1}}{X_0}$.
  \end{enumthm}
\end{tdef}

\begin{lem}
  Soit $\iota \colon V \embedin \proj\dimp$ un sous-schéma fermé intègre de
  degré $\Delta$, non contenu dans l'hyperplan d'équation $X_0 = 0$. Il existe
  une matric $M \in \GL_{n+1}(\Q)$, à coefficients entiers de valeur absolue
  (archimédienne) majorée par $\max(\frac\Delta2, 1)$, telle que, si $e_M$ est
  l'automorphisme linéaire de $\proj\dimp$ associé à $M$, alors $e_M \circ
  \iota$ est un plongement adapté et que $X_0$ soit invariant par ce
  changement de coordonnées.
\end{lem}

\begin{proof} \later
  On reprend la preuve de la proposition 2.2 de \cite{remivg} (p.~469). Au
  moment de choisir des formes linéaires $W_0, \dots, W_n$ telles que
  \begin{equation*}
    \chow V (W_0, \dots, W_\dimp) \neq 0
    \pmm,
  \end{equation*}
  on commence en fait par fixer $W_0 = \ch0$. Le polynôme $\varfc(W_0, \truc,
  \dots, \truc)$ est multihomogène de degré $\Delta$ en chaque variable ; vu
  l'hypothèse sur $V$, il est non nul grâce au théorème fondamental de
  l'élimination. On peut donc choisir $W_1, \dots W_n$ comme dans
  \cite{remivg} et continuer la preuve sans autre modification.
\end{proof}

\begin{lem}
  Soit $V \colon \embedin \proj\dimp$ un sous-schéma fermé intègre, et $M \in
  \GL_{\dimp+1}(\Qbar)$ à coefficients entiers dans l'intervalle $[-B, B]$, où
  $B$ est un réel positif fixé, et $e_M$ l'automorphisme linéaire de
  $\proj\dimp$ associé à $M$. Alors
  \[
    \haut[1]{e_M(V)}
    \le
    \haut[1]{V}
    \cdot \bigl( (n+1)B \bigr)^{\Delta(n+1)}
    \pmm.
  \]
\end{lem}

\begin{proof} \later
  Inégalité classique sur la norme $L_1$ et la spécialisation des polynômes.
\end{proof}

\begin{scho}
  Vu les hypothèses sur $V$, on peut sans perte de généralité supposer de plus
  que $V \embedin \proj\dimp$ est adapté, quitte à multiplier sa hauteur
  ($L_1$) par
  $\bigl ((\dimp + 1) \max (\frac\Delta2, 1) \bigr)^{\Delta(\dimp+1)}$.

  On supposera donc désormais que chaque $\var*$ est plongé de façon adaptée
  dans son facteur $\proj\dimp$. Ceci implique que
  $\varfc*(\cmh*0, \dots,\cmh*{\vardim*})$
  est non nul pour tout $\fct$ ; on normalisera donc notre choix de $\varfc*$
  (qui est unique à une constante multiplicative près) en imposant que cette
  quantité vaille $1$.
\end{scho}

\section{Idéal d'annulation}

\clearpage
\printbibliography

\end{document}
