\documentclass{mpg-preth}

\title{Vers une fonction auxiliaire}
\subtitle{Ébauche de notes de travail}

\begin{document}

\maketitle

\section{Situation générale}

Soit $\va$ une variété abélienne sur un corps de nombres $\cdn$, plongée dans
un espace projectif $\proj\dimp$ muni de coordonnées homogènes $\ch0, \ldots,
\ch\dimp$. On suppose que $\va$ n'est pas contenue dans l'hyperplan
d'équation $\ch0 = 0$, et on considère le diviseur $\divi$ découpé par
$\ch0$ sur $\va$. On note enfin $\ideal\va$ l'idéal homogène saturé
définissant $\va$ dans $\cdn [\ch0, \ldots, \ch\dimp]$.

On considère $\puiss$ un entier positif, et $(\proj\dimp)^\puiss$ l'espace
multi-projectif associé à l'anneau multi-gradué
\begin{math}
  \cdn [
    \cmh1[0], \dots, \cmh1[\dimp];
    \dots;
    \cmh\puiss[0], \dots \cmh\puiss[\dimp]
  ]
  =
  \cdn [ \cmh1, \dots, \cmh\puiss ]
  =
  \cdn [ \cmmh ]
\end{math}.
On s'intéresse à une sous-variété produit de $\va^\puiss$ plongée dans
$(\proj\dimp)^\puiss$, notée
\begin{math}
  \var = \var[1] \times \dots \times \var[\puiss]
\end{math}
et dont aucun facteur n'est contenu dans un hyperplan d'aquation $\cmh*0 = 0$.
On notera en outre $\vardim* = \dim \var*$ et $\vardim = (\vardim*)_\fct$ puis
$\vardeg* = \deg \var*$, et $\varfc*$ une forme de \bsc{Chow} de $\var*$.

\clearpage
\printbibliography

\end{document}
