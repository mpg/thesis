% préambule spécifique provisoire pour mumford-points

\newcommand{\av}[2][v]{\left\lvert#2\right\rvert_{#1}}
\newcommand{\nv}[2][v]{\left\lVert#2\right\rVert_{#1}}
\newcommand{\nnv}[2][v]{%
  \left\lvert\hspace{-1pt}%
  \left\lvert\hspace{-1pt}%
  \left\lvert#2\right\rvert%
  \hspace{-1pt}\right\rvert%
  \hspace{-1pt}\right\rvert_{#1}}
\newcommand{\nvp}[2][v]{\|#2\|_{#1}}
\newcommand{\dv}{{\delta_v}}
\newcommand{\Dv}{\mathrm{dist}_v}
\newcommand{\A}{\mathcal{A}}
\newcommand{\p}[1]{{\boldsymbol{#1}}}
\newcommand{\OA}{\p{0}}
\newcommand{\coa}{\theta}
\newcommand{\BA}{\mathfrak{B}}
\newcommand{\hn}{\hat{h}}
\newcommand{\lgr}[1]{{|#1|}}
\newcommand{\vlg}[1]{\lgr #1}

