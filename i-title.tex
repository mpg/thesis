% !TEX root = front.tex

\begin{titlepage}
  \centering

  {
    \Large
    \MakeUppercase{Thèse de doctorat de}\\
    \MakeUppercase{l'université Pierre et Marie Curie}
    \par
  }
  \vspace{1em}
  Spécialité mathématiques

  \vspace{3em}

  Présentée par \\ [1em]
  Manuel \bsc{Pégourié-Gonnard} \\ [3em]
  pour obtenir le grade de \\ [1em]
  \textsc{Docteur de l'université Pierre et Marie Curie}

  \vspace{\stretch{1}}

  Sujet de la thèse \\ [1em]
  {
    \LARGE\bfseries
    Approximation diophantienne \\
    sur les variétés abéliennes
    \par
  }

  \vspace{\stretch{1}}

  Soutenue le 22 octobre 2012 devant le jury composé de \\ [1em]
  \begin{tabular}{ll}
    Patrice \bsc{Philippon}, \emph{directeur}
    & Directeur de recherche, CNRS \\
    Philipp \bsc{Habegger}, \emph{rapporteur}
    & Professeur, université de Francfort \\
    Gaël \bsc{Rémond}, \emph{rapporteur}
    & Directeur de recherche, CNRS \\
    Francesco \bsc{Amoroso}
    & Professeur, université de Caën \\
    Daniel \bsc{Bertrand}
    & Professeur, université Pierre et Marie Curie \\
    Sinnou \bsc{David}
    & Professeur, université Pierre et Marie Curie \\
    Guillaume \bsc{Maurin}
    & Maître de conférences, université Pierre et Marie Curie \\
    Fabien \bsc{Pazuki}
    & Maître de conférences, université de Bordeaux \\
  \end{tabular}

\end{titlepage}

\endinput

% vim: spell spelllang=fr
