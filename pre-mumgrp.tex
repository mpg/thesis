\newcommand*\av[2][v]{\abs{#2}_{#1}}      % valeur absolue v-adique, ou indice en option
\newcommand*\nv[2][v]{\norm{#2}_{#1}}       % norme v-adique, idem
\newcommand*\nnv[2][v]{\nnorm{#2}_{#1}}     % norme v-adique d'une famille, idem
\newcommand*\nvp[2][v]{\lVert#2\rVert_{#1}}     % norme idem mais en forçant une petite taille
\newcommand*\mv[2][v]{M_{#1}(#2)}     % mesure homogène en une place v ou autre
% \newcommand*\mahler[2][]{\mathcal M_{#1}(#2)}   % mesure de Mahler
\newcommand\dv{{\delta\smash{_v}}}        % sélecteur v ultramétrique ou non
\newcommand\dpv{{\delta'_v}}        % idem plus places divisant 2
\newcommand\Dv{\mathrm{dist}_v}       % distance v-adique projective
\newcommand\A{\mathcal A}         % la variété abélienne
\newcommand\B{\mathcal B}       % une sous-v.a.
\newcommand*\p[1]{\boldsymbol{#1}}      % un point ; on notera en « normal » ses coordonnées
\newcommand\OA{\p 0}          % l'origine de la variété
\newcommand\coa{\theta}         % des coordonnées de l'origine
\newcommand\BA{\mathfrak b}         % vecteur "B" avec coordonnées inverses de l'origine
\newcommand\hn{\smash{\hat h}}        % la hauteur normalisée
% \newcommand*\lgr[1]{{\lvert#1\rvert}}     % longueur d'un multi-indice
% \newcommand*\vlg[1]{\lgr#1}         % multi-longueur partielles d'un multi-multi-indice
\newcommand*\lgt[1]{\lVert#1\rVert}     % longueur totale
\newcommand\md{\mathfrak{d}}        % le morphisme « d »
\newcommand*\MW[2][\A]{MW_{\!#1}(#2)}     % l'espace de mordell-weil de #2 sur #1
\newcommand*\hatxs{\smash{\hat x}}      % x chapeau rapetissé
\newcommand*\hatys{\smash{\hat y}}      % y chapeau rapetissé
\newcommand*\cube[1]{\mathcal C(#1)}      % le « cube » \abs{w} \le #1 et w entier

\newcounter{cst}
\newcommand*\cst[2][, v]{%
  \ensuremath{c\ssub{\ref{#2}#1}}}
\newcommand*\newcst[2][, v]{%
  \refstepcounter{cst}\label{#2}%
  \cst[#1]{#2}}

\newcommand\iq{{\mathfrak q}}

\newcommand\vb{{\mathbf e}}
\newcommand\Mbase{{\mathcal M}}
\newcommand\monome{{\mathfrak m}}
